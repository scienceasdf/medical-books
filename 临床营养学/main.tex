\PassOptionsToPackage{unicode=true}{hyperref} % options for packages loaded elsewhere
\PassOptionsToPackage{hyphens}{url}
%
\documentclass[
  12pt,
  twoside,
  banjiao]{ctexbook}
\usepackage{lmodern}
\usepackage{amssymb,amsmath}
\usepackage{ifxetex,ifluatex}

\usepackage{unicode-math}
\defaultfontfeatures{Scale=MatchLowercase}
\defaultfontfeatures[\rmfamily]{Ligatures=TeX,Scale=1}

% use upquote if available, for straight quotes in verbatim environments
\IfFileExists{upquote.sty}{\usepackage{upquote}}{}
\IfFileExists{microtype.sty}{% use microtype if available
  \usepackage[]{microtype}
  \UseMicrotypeSet[protrusion]{basicmath} % disable protrusion for tt fonts
}{}

\usepackage{xcolor}
\usepackage{subfig}
\usepackage{xurl} % add URL line breaks if available
\usepackage{bookmark}
\usepackage[version=4]{mhchem}
\usepackage{hyperref}
\usepackage{float}
%\usepackage{placeins}
\usepackage[section]{placeins}
\makeatletter
\AtBeginDocument{%
  \expandafter\renewcommand\expandafter\subsection\expandafter{%
    \expandafter\@fb@secFB\subsection
  }%
}
\makeatother
\hypersetup{
  pdftitle={临床营养学},
  pdfauthor={github},
  pdfborder={0 0 0},
  breaklinks=true,
  bookmarksdepth=5}
\urlstyle{same}  % don't use monospace font for urls
\usepackage{longtable,booktabs}
% Allow footnotes in longtable head/foot
\IfFileExists{footnotehyper.sty}{\usepackage{footnotehyper}}{\usepackage{footnote}}
\makesavenoteenv{longtable}
\usepackage{graphicx,grffile}
\makeatletter
\def\maxwidth{\ifdim\Gin@nat@width>\linewidth\linewidth\else\Gin@nat@width\fi}
\def\maxheight{\ifdim\Gin@nat@height>\textheight\textheight\else\Gin@nat@height\fi}
\makeatother
% Scale images if necessary, so that they will not overflow the page
% margins by default, and it is still possible to overwrite the defaults
% using explicit options in \includegraphics[width, height, ...]{}
\setkeys{Gin}{width=\maxwidth,height=\maxheight,keepaspectratio}

\setlength{\emergencystretch}{3em}  % prevent overfull lines
% Redefines (sub)paragraphs to behave more like sections
\ifx\paragraph\undefined\else
  \let\oldparagraph\paragraph
  \renewcommand{\paragraph}[1]{\oldparagraph{#1}\mbox{}}
\fi
\ifx\subparagraph\undefined\else
  \let\oldsubparagraph\subparagraph
  \renewcommand{\subparagraph}[1]{\oldsubparagraph{#1}\mbox{}}
\fi

% set default figure placement to htbp
\makeatletter
\def\fps@figure{htbp}
\makeatother


\setcounter{secnumdepth}{5}
\usepackage{framed}
\usepackage{multirow}
\usepackage{ctex}
\usepackage{rotating}
\usepackage{tablefootnote}
\usepackage{caption}

\setCJKmainfont{思源宋体}
\setCJKfallbackfamilyfont{\CJKrmdefault}{宋体}
\setmainfont{思源宋体}
\usepackage[a4paper,top=1in, bottom=1in, left=0.8in, right=0.8in]{geometry}
\setlength{\parindent}{2em}
\setlength{\parskip}{0em}

%\newfontfamily\apostrophefont[Ligatures=TeX,Color=FF0000]{Liberation Serif}
\newfontfamily\apostrophefont[Ligatures=TeX]{Liberation Serif}
\XeTeXinterchartokenstate=1
\newXeTeXintercharclass \apostrophe

% Assign the new class to all Latin capital letters
\makeatletter
\@tempcnta=`'
\loop\unless\ifnum\@tempcnta>`'
  \XeTeXcharclass \@tempcnta \apostrophe
  \advance \@tempcnta by 1
\repeat
\makeatother

% Setup font change
\XeTeXinterchartoks 0 \apostrophe   = {\begingroup\apostrophefont}
\XeTeXinterchartoks \apostrophe 0   = {\endgroup}
\XeTeXinterchartoks 4095 \apostrophe = {\begingroup\apostrophefont}
\XeTeXinterchartoks \apostrophe 4095 = {\endgroup}

\renewcommand {\thetable} {\arabic{chapter}-\arabic{section}-\arabic{table}}
\renewcommand {\thefigure} {\arabic{chapter}-\arabic{section}-\arabic{figure}}
\newcommand\subsectiontitleformat[1]{\noindent 【#1】}
\usepackage{chngcntr}
\counterwithout{subsection}{chapter}
\counterwithout{subsection}{section}
\counterwithin{figure}{section}

\ctexset {
  section = {
    name
    = {第,节},
    number = \chinese{section},
  },
  subsection = {
    name 
    = {,、},
    number = \chinese{subsection}
  }
}


\title{临床营养学}
\author{github \\ 项目主页:\url{https://github.com/scienceasdf/medical-books}\\ 新书下载:\url{https://github.com/scienceasdf/medical-books/releases/latest}}


\begin{document}
\maketitle
\frontmatter
{
\setcounter{tocdepth}{1}
\tableofcontents
\addcontentsline{toc}{chapter}{目录}
}
\newpage

\mainmatter
\chapter{总论}

\section{CT机的基本构造和原理}

1895年11月8日,德国著名物理学家威·康·伦琴(W.K.Roentgen)在一次阴极真空射线管放电实验中偶然发现了X线,它不仅是对物理学的巨大贡献,也为放射诊断学的创立和发展奠定了基础。100多年来放射诊断学获得了迅猛的发展。1969年英国物理学家G.N.Hounsfield利用人体对X线的吸收原理,结合计算机的图像重建和处理功能设计了计算机断层扫描机(computed
tomography,简称CT),这一成果于1972年向全世界宣告。这种图像质量好、诊断价值高的成像方法,使放射诊断学发生了重大突破,是对现代医学的卓越贡献。为此,Hounsfield获得了1979年诺贝尔医学生物奖。

\subsection{CT的发展简史}

1967年CT的基本组成部分即重建数学、计算机技术和X线探测器都已具备。那时,Hounsfield在EMI实验研究中心,从事图像识别和利用计算机存储手写字技术的研究。他证实了有可能采用一种与直接电视光栅方式不同的另一种存储方法,这种方法使信息检索更为有效。

首先,有人提议从三维物体的各个方向取读数。但后来的断层方法似乎更适用于图像重建和诊断。它意味着仅需从单一平面里获取透射的读数。因此,每个光束通路都可以看作联立方程中的许多方程之一,必须解这组联立方程才能获得该平面的图像。此原理用数学模拟法加以研究。然后经过反复实验,并用X线进行人脑组织标本扫描研究,于1969年Hounsfield设计成功CT机。第一个原型设备于1971年9月安装在Atkinson
Morley医院,1971年10月4日检查了第一个病人。在1972年4月英国放射学家研究年会上宣告EMI扫描机诞生了,接着在同年11月芝加哥的北美放射学会(RSNA)年会上向全世界宣布。1973年在英国放射学杂志上报道。1974年美国George
Town医学中心工程师Ledley设计了全身CT机,从此CT告别了只限于头颅检查的时代。1985年开发了滑环技术,1989年单方向连续螺旋型CT即螺旋CT的问世,是滑环技术的体现,是CT发展的重大突破。1991年开发了亚毫米扫描和双螺旋CT。1998年多层螺旋CT机问世,SIEMENS、GE、PICKER、TOSHIBA公司都相继生产。1983年超高速CT(ultrafast
CT,UFCT)又称电子束CT(electron beam technology
CT,EBCT)由美国Imatron公司率先研制成功,并于1993年推向市场。EBCT进一步开拓了CT的应用范围,例如心脏功能和形态学研究,心、脑、肾、冠状动脉的血流量测定等。2004年GE公司率先推出64层(64排探测器)螺旋CT,并在北美放射学会年会发布这一信息。此后SIEMENS公司亦研制出64层螺旋CT(探测器为40排,机架每旋转1周利用中间的32排探测器即可获得64层图像)。在2005年北美放射学会上,SIEMENS公司又推出了双源CT(SOMATOM
Definition),SOMATOM
Definition基于西门子64层螺旋CT的成熟技术,配备了两个同步旋转的X射线源、探测器,每组X射线源、探测器组合只需转动90\textsuperscript{o}
就可以获得质量很好的心脏图像;基于0.33s的机架旋转时间,它可获得83ms的时间分辨率,使心脏成像不受心率影响。

\subsection{CT机的基本结构和成像原理}

\subsubsection{基本结构}

1.X线发生系统:高压发生器、X线球管、冷却系统、前准直器(去除散射线,使X线呈束状排列)。

2.X线探测部分:探测器、模数转换器(将探测器形成的电信号转换成数字信号,输入计算机)、后准直器(去除被照物体后的散射线)。探测器分为气体和固体两大类。固体探测器由闪烁晶体构成,有碘化钠、碘化铯、钨酸镉、锗酸铋等;气体探测器采用气体电离室的原理,一般多用氙气。固体探测器灵敏度较高,但其几何利用率较低,而气体探测器则与之相反。

3.支架部分:扫描架和检查床。

4.计算机系统:第3、第4代CT机包括阵列处理机(图像处理计算机)和主计算机(中央处理系统)。

5.图像贮存、显示和记录部分:主要指磁盘(硬盘)、光盘、磁带或软盘、显示器和照相机等。

6.操作控制部分:主要指操作台的键盘。

\subsubsection{成像原理}

CT的成像原理与普通X线相仿,只是图像的载体用探测器代替了胶片或荧光屏。CT扫描时用高度准直的X线束扫描人体的某部位,并围绕该部位做360°匀速转动,穿过人体的X线再经过准直后,由探测器接受。探测器接受的大量信息经模数(A/D)转换器将模拟量转换成数字输入计算机,计算机计算出该断面上各单位体积的X线吸收值(CT值)并排列成数字矩阵。数字矩阵再经数模转换(D/A),用灰白不同的灰度等级在监视器荧屏上显示,就获得了该部位的横断解剖结构图像。不同密度的组织对X线的吸收量不同。探测器分辨X线量的敏感程度较X线透视和X线摄影高的多,其对组织的密度分辨力较常规X线检查约高10~30倍。

\subsection{CT机的分代}

CT机的发展通常以“代”来划分,主要依据X线球管和探测器的关系、探测器的数目、排列方式和两者的运动方式来划分。其实“代”并不完全反映CT机的性能优劣,即并非代数越高CT机性能越好,如第3、第4代CT各有其优点。

1.第1代CT机:X线为单射束,单个或数个探测器,运动方式为平移加旋转,扫描一层需数分钟,只能限于头颅扫描。

2.第2代CT机:它与第1代无质的区别。X线为小角度(3°~30°)扇形X线束,探测器从数个至几十个。运动方式仍为平移加旋转,扫描时间缩短至18秒左右。虽已扩大至全身应用,但运动伪影很明显,故实际仍限于头颅扫描。

3.第3代CT机:X线为角度较大(30°~45°)的扇形X线束,探测器也相应呈扇形排列,数目多达数百个。运动方式为旋转式,扫描时间一般为2~5秒,最快可达1秒,使CT检查应用于全身。滑环技术及随之应用的螺旋CT是第3代机型的重大突破。

4.第4代CT机:与第3代基本相同。探测器排列呈圆周状,固定在扫描架四周,仅为X线球管旋转。实际为第3代的变型,并无明显优越性,仅有少数厂家生产。

5.第5代CT机:即超高速CT机(UFCT),又称电子束CT(EBCT),与以前的CT机已有根本区别。采用电子枪结构,使每次扫描时间可缩短至30ms,大大有利于心脏CT扫描。

EBCT即UFCT,主要由电子枪、聚集线圈、偏转线圈、8排探测器群、台面高速移动的检查床和控制系统组成。采用电子枪发射电子束,经聚焦后由偏转线圈控制,使电子线旋转,并轰击四个平行的钨靶环,从而获得旋转的X线源,再采用8排探测器群来收集扫描数据。目前,扫描速度可达50ms。由于有4个靶环,一次可进行4次扫描,最快每秒可扫描24次,故对心脏、冠状动脉等心血管的研究有特殊的作用。它的优点是:①扩大了影像诊断范围;②提高了图像质量(减轻了运动伪影);③减少了造影剂用量,并提高了高峰显影质量;④增加了单位时间的检查人数;⑤可做血流量、血流速度和弥散等功能检查。

\section{CT的应用和进展}

\subsection{CT图像重建的常用数学演算方式}

通常使用的演算方式有:标准演算法(standard
resolution)、边缘演算法(edge resolution)和平滑演算法(smooth
resolution)等。可根据受检部位的组织成分及密度差异,选择合适的数学演算方式。标准算法适于分辨率要求一般的普通CT图像重建如头颅等。软组织演算法对密度差异很近的组织分辨率较好,常用于腹部脏器的检查。骨密度演算法的图像分辨率最佳,可以分辨密度差异很大的组织,适用于观察骨质及内耳、乳突等,也可用于肺部小病灶的高分辨率CT检查(HRCT)。

\subsection{影响CT成像的因素}

1.窗宽和窗位:形成CT图像的数字矩阵都是CT值,即组织密度的代表。空气的CT值约为-1000Hu,骨皮质的CT值约为1000Hu。而人眼大约能分辨16个灰阶。如果某幅图像内既含有空气,亦有骨皮质,则上下CT值范围约2000Hu差值,每个灰阶所包含的CT值范围为125Hu。那么,CT值相差在125Hu内的组织结构显示为同一灰阶(即同一灰度)而不能在CT图像上各自显示。所以就要在观察某幅图像或观察某部位的组织结构时,选择合适的CT值范围和该范围的中点,来观察或显示某幅图像,这一CT值范围即窗宽,其中点即为窗位。如观察脑组织窗宽为100Hu,窗位为35Hu;肺部窗宽为1000~1500Hu,窗位为-700Hu左右;纵隔窗宽为200~300Hu,窗位30~50Hu。

2.伪影与噪声:①伪影:在扫描中由于某种因素的影响而产生的被检物体不存在的假象。有机器因素造成的环状伪影、条状伪影、点状伪影等;亦有因人体内密度差异(例如骨骼、手术金属物、胃肠内钡剂)造成的伪影;还有运动性(如胃肠蠕动、呼吸、病人身体移动)伪影。扫描条件不当亦造成伪影。②噪声:分为光子噪声和组织噪声。前者亦称扫描噪声,即X线穿透人体后到达检测器的光子量有限,其在矩阵内各图像点(像素)上的分布不是绝对均匀所造成,以致均质组织或水在各图像点上的CT值不是相等的,为减少噪声需增加X线量。组织噪声为各种组织(如脂肪和脑组织)的平均CT值变异所致,即同一组织CT值常在一定范围内变化,以致不同组织可具有同一CT值。

3.部分容积效应:像素代表一个体积,此体积内可能含有各种不同组织。其CT值实际代表的是单位体积内各种组织CT值的平均数。例如骨骼和气体加在一起可以类似肌肉密度(CT值)。因此在较高密度区域中间的较小低密度灶的CT值常偏高,反之亦然。

4.空间分辨率与密度分辨率:前者是指影像中能显示的最小细节,后者是指能显示的最小密度差别。

\subsection{CT的检查方法}

\subsubsection{常用检查方法}

常用的CT检查方法有平扫描、增强扫描、动态CT扫描、靶扫描(亦称目标CT扫描、放大CT扫描)、高分辨率CT技术(HRCT)。

用于肝脏的特殊增强扫描即所谓的肝脏CT血管造影有两种方式,包括肝动脉插管的动脉造影CT(CTA)和经脾动脉或肠系膜上动脉注入造影剂的门静脉期扫描,又称经动脉门脉血管造影CT(CTAP)。

\subsubsection{特殊检查方法}

1.CT透视与实时螺旋CT扫描:其原理相同,即在接近0.6秒的延迟时间后CT图像以6帧/秒速度显示,能达到实时观察的目的。CT透视主要用于CT介入穿刺;实时螺旋扫描能在扫描期间评价增强程度、选择扫描时机等。

2.CT血管造影术:或称CT血管成像术(CT
angiography,CTA)是螺旋CT三维(3D)重建技术的应用结果,主要用于颈动脉、颅内动脉、胸主动脉、腹主动脉、髂动脉、肺动脉、肾动脉、肠系膜动脉及内脏静脉(如门静脉)成像。CT仿真内镜如CT胆管成像、泌尿系尿路成像等亦是螺旋CT三维重建的应用体现。

3.仿真内镜术(Virtual
endoscopy,VE):是将CT或MR获得的原始容积数据与计算机三维图像技术相结合,借助导航技术(navigation)或漫游技术(flythrough)以及伪彩技术来逼真的模拟腔道内镜检查的一种方法。于1994年Vining等首次报道应用于结肠CTVE。

目前CTVE主要用于:①气管和支气管;②鼻咽腔、鼻窦、喉和中耳;③胃和结肠;④大血管;⑤胆道;⑥肾盂、输尿管和膀胱;⑦脑室和椎管;⑧关节腔等。但它存在着不能显示病变的颜色、不能显示腔内扁平病变、定性能力差等缺点。目前虽处于初步认识阶段,但值得进一步深入研究。

4.CT灌注成像(CT perfusion
imaging):是指静脉注射造影剂的同时对选定的层面进行连续多次扫描,以获得该层面内每一像素的时间-密度曲线。根据该曲线利用不同的数学模型计算出血流量、血容量、对比剂的平均通过时间、造影剂峰值时间等参数,以此评价组织器官的灌注状态。它反映的是生理功能的改变,因此是一种功能成像,可用于了解脑、肝、肾、胰腺、心脏的灌注状态。灌注参数还能较准确的反映头颈部、肝、肾、胰腺和肺等部位的肿瘤内血管变化和血液动力学改变,对肿瘤的诊断及恶性肿瘤的分级有重要意义,且为治疗方案的选择提供有价值的信息。

5.CT定量骨密度测定:定量骨密度测定为骨矿物质含量测定的重要方法。其方法很多,如单光子吸收法和双光子吸收法等,其中以CT双效能定量测量法(定量CT法)比较可靠。

\subsection{常用的螺旋CT三维重建技术}

常用的三维重建技术有:多平面重建法(multi-planar
reformation,MPR)、最大强度投影法(maximum intensity
projection,MIP)、最小强度投影法(minimum intensity
projection,MinIP)、遮蔽表面显示法(shaded surface
display,SSD)、容积再现法(volume rendering,VR)以及曲面重建法(curved
planar reformation,CPR)、透明重建(Ray sum)等。

\subsection{应用和副反应}

\subsubsection{药理}

CT增强所用的造影剂主要为经肾脏排泄的含碘造影剂,但也有用硫酸钡制剂和胆道造影剂者。钡剂用于胃肠道检查;经肝脏排泄的胆道造影剂(包括口服和静脉用药)只用于胆道增强。

目前,CT检查使用的经肾脏排泄的造影剂多为水溶性造影剂,且均为三碘苯环的衍生物(图\ref{fig1-1})。根据其结构分为4型:①离子型单体:常用的有复方泛影葡胺、安其格纳芬(Angiografin);②离子型双聚体:常用的有碘卡明;③非离子型单体:常用的有优维显(碘普罗胺)、欧乃派克(碘苯六醇)、碘必乐(碘异酞醇)等;④非离子型双聚体:常用的有伊索显(碘曲仑)。

\begin{figure}[!htbp]
 \centering
 \includegraphics[width=.7\textwidth,height=\textheight,keepaspectratio]{./images/Image00002.jpg}
 \captionsetup{justification=centering}
 \caption{三碘苯环的基本分子式}
 \label{fig1-1}
  \end{figure} 

离子型者苯环上1位侧链为羧基盐(---COOR),具此结构的造影剂水溶性高,在水溶液中可离解成阴离子(含有三碘的苯环)及阳离子(葡甲胺、钠、钙、镁)。非离子型者苯环上1位侧链为酰胺衍生物(---CONH),其水溶性很高,但不离解于水中。单体造影剂是指一分子造影剂含有一个三碘化苯环,双聚体则有两个三碘化苯环。

经肾脏排泄造影剂的临床应用主要受下列因素影响:①碘浓度:与造影剂的增强效果有关。根据其浓度可分为4类:特高浓度(400mg/ml)、高浓度(350~400mg/ml)、中浓度(280~320mg/ml)、低浓度(80~240mg/ml)造影剂。特高浓度偶用于心脏、大血管造影和经静脉注射的动脉造影。中、高浓度造影剂尤其高浓度造影剂适用于快速静脉注射后的CT动态扫描。CT脑室或CT脊髓造影适用低浓度造影剂。②渗透压:高渗造影剂的副作用高于低渗造影剂;离子型渗透压高于非离子型;单体造影剂的渗透压高于双聚体者。③黏度:分子大、浓度高、温度低者黏度高。黏度越高给大剂量快速注射带来困难,且易形成微小血管的阻塞而引起局部的缺血缺氧。

\subsubsection{给药方式}

理想造影剂应具备以下条件:①显影清楚;②无毒、副作用;③易于吸收和排出;④使用方便;⑤性质稳定,易储存;⑥价格低廉。

除离子型造影剂碘卡明和非离子型造影剂优维显(碘普罗胺)、欧乃派克(碘苯六醇)、碘必乐(碘异酞醇)、伊索显(碘曲仑)可用于脑室及脊髓造影外,其他肾脏排泄造影剂禁用于脑室和椎管造影。因为这类造影剂容易进入蛛网膜下腔,可损害血脑屏障,引起抽搐及至死亡。

给药方式和用药量如下:

1.静脉团注法:亦称快速注射法。将某一剂量的高浓度造影剂加压快速注入静脉,在造影剂经血循环大量进入靶器官的供血动脉时开始CT扫描。这种方法可提供CT所需的高质量增强情况,现已成为常规增强方式。一般情况下造影剂用量为1.5~2ml/kg体重(成人一般注入80~100ml),注射流率为1~8ml/s。

2.静脉滴注法:以20~30ml/min的速度注入含碘量约为300mg/ml的造影剂100ml后,再行CT扫描的方法。

3.动脉注射给药法:主要用于肝脏肿瘤的诊断,即选择性注入肝动脉、脾动脉及肠系膜上动脉的CTA和CTAP扫描术,以0.7~1ml/s的流率注入70~100ml。

4.胆系造影增强:30ml胆影葡胺缓慢注射(大于5分钟)或100ml胆影葡胺静脉滴注给药(快速团注易引起严重副反应)。给药后30~60分钟达最佳强化。

5.蛛网膜下腔给药:由腰穿注入水溶性碘造影剂后做脊髓或脑室扫描。椎管造影浓度一般为200~300mg/ml,剂量10~15ml。脑室造影浓度为150
mg/ml,剂量5ml。

6.胃肠充盈造影:常用2%的碘水造影剂,用量无统一标准。①胃十二指肠检查前20~30分钟服500~800ml,上床前再服200~300ml。②小肠检查前2~3小时服800~1000ml以充盈结肠,检查前1~2小时服300~500ml以充盈小肠远端,检查前15~30分钟再服300~500ml充盈胃及小肠近端。③结肠检查一般灌肠注入1500~1800ml。

\subsubsection{含碘造影剂的副反应}

一般根据反应的轻重和需治疗的程度进行分类(见表\ref{tab1-1})。离子型和非离子型造影剂副反应发生率有明显差异,前者约为5%,后者约为1.3%,但后者重度反应明显少,约为0.01%
。所以对有肝、肾、心疾病、糖尿病、虚弱、恶病质和过敏体质者等高危人群尽可能选用非离子型造影剂。离子型和非离子型造影剂对肝肾功能的影响区别不大。

\begin{table}[htbp]
\centering
\caption{造影剂副反应的分类}
\label{tab1-1}
\includegraphics[width=\textwidth,height=\textheight,keepaspectratio]{./images/Image00003.jpg}
\end{table}

\subsection{CT的发展方向}

目前推出基于CT的肿瘤放疗系统即CT模拟定位系统,其软件、硬件专门为CT模拟设计制造,还配有立体定位介入引导系统,可帮助医师模拟和介入(活检或引流等),并配有组织间近距离放疗的CT立体定位机械手臂系统。

介入性CT可用于脑、肺、纵隔、肝、胰、肾、肾上腺、腹膜后淋巴结、盆腔肿块的穿刺活检及肿瘤治疗,亦可用于骨骼肌肉的穿刺活检。对胸腹部脓肿、腹部和盆腔囊肿(如肝、肾囊肿)进行穿刺引流、硬化治疗,其定位准确性更高。尤其对颅内血肿和脓肿的穿刺抽吸引流更具独到之处。在颈臂神经丛和腹腔神经丛神经阻断术中是其他影像学手段所不及。CT亦可用于植物神经阻断术。

\section{常用的CT技术术语}

\subsection{平扫描和增强扫描}

扫描(scan):CT机扫描架内的X线球管围绕人体旋转进行X线照射,探测器接收到衰减程度不同的X线,转换成电信号,输入计算机重建成图像,每旋转一次的动作称为扫描。

平扫描(simple
scan):不向血管内注射造影剂的一般扫描程序称为平扫。检查腹部虽然多口服造影剂以充盈胃肠,但仍叫平扫。

增强扫描(contrast enhancement
scan):即造影增强,以CE或+C表示。应用碘水造影剂注射入静脉或动脉内,使心血管、组织器官或病灶密度增加,有利于对病变或正常组织器官的显示。

\subsection{快速连续扫描和延迟扫描}

动态扫描(dynamic
scan):按设定的部位,自扫描起始位到终止位,自动地进行逐层扫描,扫描后自动处理并显示图像,可分为动床式和同层动态扫描。此种方法主要用于增强扫描。

快速连续扫描(fast continuous
scan):对感兴趣的某区,自动地进行多次快速扫描,了解器官功能活动情况、造影剂充盈与排泄情况,显示同一层次在不同时间的变化。实际属于同层动态扫描。

延迟扫描(delay
scan):部分病例需要在团注增强扫描结束后一段时间内再做病灶区域或整个脏器扫描。如疑肝血管瘤,可在团注造影剂后5~15分钟再做局部扫描,以观察病灶是否被造影剂充填。

\subsection{间隔扫描和重叠扫描}

薄层扫描(thin slice
scan):一般指层厚≤5mm的扫描。主要为了减少部分容积效应而进行此扫描。

间隔扫描(interval
scan):亦称间断扫描,即层距大于层厚的扫描。其扫描不是连续扫描,可以按一定间隔进行隔层扫描,减少了扫描层数。如层厚为10mm,层距为12mm、15mm、20mm,则为间隔扫描。

重叠扫描(overlap
scan):层距小于层厚的扫描。如层厚为5mm,而层距(间隔)为3mm,则为重叠扫描。

\subsection{定位扫描}

定位扫描(scan ogram)又称为topgraph或scout
view。即在X线球管固定时扫描出一幅人体正位或侧位像,用以做出扫描层次、方向、层距、间隔及扫描次数等计划。定位扫描像有时可代替普通X线片,供诊断参考。

\subsection{靶CT扫描}

靶CT扫描(target scan)亦称目标扫描(object scan)、放大扫描(magnify
scan),是针对某一感兴趣区做局部的CT扫描,即应用小显示野(DFOV)扫描。由于被显示的范围小而矩阵不变,在一定单位体积的区域内像素相对增多,故可明显提高空间分辨率。也可以在普通CT扫描结束后,利用收集的原始扫描数据做局部的靶图像重建。后一种方法因有扫描数据保存,故可做各种部位、大小和成像方式的图像靶重建。靶扫描或靶重建与CT图像的单纯放大不同。后者仅是把图像的某部分放大,并无从根本上改变像素的大小和成像方式,所以其分辨率未提高,其清晰度反而下降。

\subsection{螺旋扫描}

螺旋扫描(spiral scan or helical
scan)是建立在滑环技术上,是在一次数据扫描过程中X线管和探测器不停地向一个方向旋转(第4代CT机只是X线管旋转),检查床亦同时向前推进,整个扫描的轨迹呈螺旋形。故螺旋扫描采集的数据是某一器官的容积数据,因此重建时可以采用任意的重建距离来进行重建而获得相应的图像幅数。

在扫描过程中X线管每旋转一周,检查床推进的距离不一定要和层厚相等,检查床推进距离可以等于、大于或小于层厚。床推进距离和层厚之比称为螺距指数(pitch
index)简称螺距。螺距指数=床推进距离/层厚。床推进距离和层厚一致时螺距为1,床推进距离大于层厚时螺距>1,反之则<1。

\subsection{图像的重建和重组}

重建(reconstraction):是利用图像的原始数据来进行处理所形成的图像。

重组(refomatting):是利用已经形成的图像进行重新组合,如用来形成冠状面、矢状面、多平面(MPR)、三维(3D)图像。故重组与重建两者含义是有区别的,但多习惯于将重组亦称为重建。

\subsection{高分辨率CT技术}

高分辨率CT(hight resolntion
CT,HRCT)技术,即利用CT机具有的特殊软件,专为显示肺部弥漫性间质性病变以及结节病变等细微结构的重建方法,可使空间分辨率显著提高。一般采用1~2mm的薄层扫描,故亦可称为薄层高分辨率CT。实际上多采用骨密度演算法重建,所以也适用于观察骨质情况及内耳、中耳、鼻窦、眼眶等结构。

\subsection{窗功能和双窗}

窗宽(window
width):以W.W或W表示,即在观察某幅图像时所选择的CT值范围。观察不同的组织器官应选择合适的窗宽。

窗位(window level或window
center):以W.L、L或C表示,即所选择窗宽的CT值范围的中心值。

窗功能(window
function):即在观察某幅图像时,通过窗宽及窗位的调节,使所需观察的组织、器官及病变清楚显示称为窗功能。

双窗(dual
window):即为显示不同组织器官使用双窗位及双窗宽,具体数字在监视器或CT片上分别显示。例如胸部双窗显示,可在显示纵隔图像的同时,也显示肺的细节,有利于观察肺部病变与纵隔的关系。

\subsection{CT值及其换算公式}

CT值是指X线穿过人体后,探测器检测并换算出的组织、器官的衰减值,它所反应的是组织、器官的密度。其换算公式如下:

CT值=(μ组织-μ水)/μ水×α

μ组织为人体组织的吸收系数,μ水为水的吸收系数,α为分度因数。在换算时将水的吸收系数调节为1,空气的吸收系数为0,μ组织是相对水和空气而言的。α目前均采用Hunsfield的单位,其分度因数为1000,故水的CT值为0Hu,空气约为-1000Hu,骨的吸收系数最高可达水的两倍,即μ骨为2,故其CT值可高达1000Hu。最早的CT机采用EMI单位,其分度因数为500,故其CT值是Hu单位的一半,如空气为-500EMI单位。

\subsection{感兴趣区}

感兴趣区(region of
interest,ROI)即我们对图像某部分进行CT值测量分析的区域,其中有3项指示在画面上。

1.平均值(mean,m):即ROI内的CT值。

2.标准偏差(standard deviation,SD):即ROI内CT值的标准偏差。

3.面积(area,a):即ROI内的面积,以mm\textsuperscript{2} 表示。

\subsection{层厚和层距}

层厚(thickness or
slice)是指CT断层每个层面的厚度。层距(interval)是指每个扫描层面间的距离。根据层厚和层距的关系可分为连续扫描、间隔扫描和重叠扫描。

\subsection{像素和像体素}

矩阵(matrix):是一个数学概念,它表示一个横成行、纵成列的数字阵列。如320×320,512×512,1024×1024等。CT机将计算的人体断面各点的CT值以矩阵排列,构成分布图。矩阵由极小的方格所组成,其格数越多即矩阵越大,则显示的图像越清晰细致。

像素和像体素(pixel
voxel):一幅CT图像是由许多矩阵排列的小单元(小方格)组成,这些组成图像的基本单元称为像素。像素所表示的每一个小单元内具有一定宽度和一定厚度(层厚)的立方体称为像体素。体素是一个三维概念,而像素是一个二维概念,像素实际是体素成像时的表现,矩阵越大像素越小。

\subsection{周围间隙现象和伪影}

部分容积效应(partial volume
effect):亦称为平均值效应。每个像素的CT值为此像素或体素内各种物质CT值的平均值,故如果层厚过大,则一个像素内常含有两种或两种以上密度互不相同的物质,从而不能确切地反映其组织密度,这一现象称为部分容积效应。可通过减小层厚减轻部分容积效应的影响。

周围间隙现象(peripheral space
phenomenon):在同一层面内,与层面垂直的两个相邻且密度不同的物体,其物体边缘部的CT值不能准确测得,结果在CT图像上也不能清晰地分辨出两者的交界,这种现象亦称为边缘效应。

伪影(artifacts):由于某些因素的影响,图像中产生实际并不存在的各种形状的假象,称为伪影。

\subsection{空间分辨率和密度分辨率}

空间分辨率(spatial
resolution):又称高对比分辨率,是指CT对物体空间大小(几何尺寸)的分辨能力。通常用每厘米内的线对数或用可辨别最小物体的直径来表示。空间分辨率与被检物体的密度差别也有关,密度差别小则空间分辨率亦也相应减小。影响空间分辨率的主要因素为探测器的几何尺寸、探测器间的间隙和总的原始数据。重建算法也是影响空间分辨率的重要因素。

密度分辨率(density
resolution):又称低对比分辨率,是指CT对密度差别的分辨能力。以百分比表示,如密度分辨率为0.35%,即表示两个物质的密度差>0.35%时,CT即可将它们分辨出来。噪声和信噪比是影响密度分辨率的重要因素。

以上二者是相互制约的,空间分辨率与像素大小密切相关,一般为像素宽度的1.5倍。像素越小,数目越多,空间分辨率提高,图像越清晰,但在X线源总能量不变的条件下,每个单位容积所获得的光子数却按比例减少,使密度分辨率下降。

\subsection{“多层”和“多排”}

“多层”(multi-slice)和“多排”(multi-row)是两个完全不同的概念。1998年全球各相关公司相继推出了4层螺旋CT,然而不同的厂家采用了不同的探测器设计理念。如探测器的排列有对称和不对称之分,有8、16、34排不同的排列,但均为同步获得4层图像的扫描能力。2001年西门子公司率先推出了16层螺旋CT,而探测器的排列是24排。GE公司64层螺旋CT,为64排探测器;SIEMENS公司64层螺旋CT,探测器为40排,机架每旋转1周利用中间的32排探测器即可获得64层图像。故
“排”是指探测器的物理排列数目;而“层”是指数据采集系统同步获得图像的能力,即机架每旋转一周能够同步采集几层图像。所以,“多层螺旋CT”更加符合人们通常的理解且更趋合理。

\protect\hypertarget{text00009.html}{}{}


\chapter{损伤的修复}

\chapterabstract{本章主要介绍机体组织损伤后的修复过程,要求掌握修复、再生、纤维性修复、肉芽组织的概念,不同类型细胞的再生能力,肉芽组织的构成及其在修复过程中的作用,熟悉常见组织的再生过程,瘢痕组织的形态及对机体的影响,创伤愈合的基本过程和皮肤的创伤愈合,了解细胞再生的影响因素,骨折愈合过程和影响创伤愈合的因素。}
\begin{framed}
	{案例2-1}

	{【病例摘要】}

	患者,男,65岁,因意识不清,突发倒地入院。CT检查示右侧基底节出血灶,外科行血肿清除术后,生命体征平稳,但患者仍无自主意识,长期卧床。术后20天,患者左侧肩胛部见一压疮灶,直径约4
	cm,深部组织坏死明显,清创术后数日,见压疮灶内有红色颗粒状组织覆盖。

	{【问题】}

	(1)该压疮灶内红色颗粒状组织是什么?由哪些成分构成?

	(2)该红色颗粒状组织有何功能?
\end{framed}
机体对损伤所造成的缺损进行修补恢复的过程,称为修复(repair)。修复过程可包括两种不同的形式:由损伤周围邻近的同种细胞来修复,称为再生(regeneration);由纤维结缔组织来修复,最后局部纤维化,形成瘢痕,称为纤维性修复。

\section{再生}

\subsection{再生的类型}

\paragraph{生理性再生}
生理过程中,许多组织细胞不断衰老、死亡,同时又由同种细胞通过分裂增生补充,这种再生称为生理性再生。例如皮肤表层角化细胞经常脱落,表皮基底层细胞不断增生分化,予以补充,胃黏膜上皮三天左右更新一次,血细胞也在不断更新等,皆属生理性再生。

\paragraph{病理性再生}
在病理状态下,组织细胞坏死或缺损后,通过周围同种细胞增生来恢复原有的结构和功能,称为病理性再生。如皮肤表皮损伤后,基底层以上各层细胞坏死,由基底层细胞增生、分化,恢复表皮的结构和功能。

\subsection{不同类型细胞的再生能力}

按再生能力不同,将人体组织细胞分为三类。

\paragraph{不稳定细胞(Labile cells)}
这类细胞再生能力强,在生理状态下经常进行周期活动,不断分裂增生,以补充衰老死亡的细胞,在病理状态下也具有强大的再生能力。例如全身的上皮细胞、淋巴造血细胞。上皮细胞包括皮肤表皮、胃肠道和呼吸道的黏膜上皮、泌尿道的移行上皮以及腺体的导管上皮等。

\paragraph{稳定细胞(stable cells)}
这类细胞在生理状态下增生现象不明显,处于细胞增殖周期的静止期(G{0}
期),但具有潜在的再生能力,在损伤的刺激下,则进入DNA合成前期(G{1}
期),表现出较强的再生能力。属于这类细胞的有各种腺体及腺样器官的实质细胞,如肝、胰、内分泌腺、汗腺、皮脂腺及肾小管的上皮细胞等;还包括间叶细胞及其衍生的各种细胞,例如成纤维细胞、骨、软骨、脂肪、平滑肌细胞等。

\paragraph{永久性细胞(permanent cells)}
这类细胞在生理状态下较为恒定,基本上无再生能力,故不能分裂增生,一旦遭受损伤则成为永久性缺失。属于这类的细胞有神经细胞、心肌细胞及骨骼肌细胞。心肌细胞和骨骼肌细胞虽有微弱的再生能力,但因速度极慢,以至损伤处被快速增生的纤维结缔组织替代,通过瘢痕修复。

\subsection{常见组织的再生过程}

\paragraph{上皮组织的再生}
(1)被覆上皮再生:皮肤的复层鳞状上皮受损伤时,创缘或基底部残存的基底细胞则分裂、增生,向缺损中心移动。初起为单层,完全覆盖缺损后,细胞开始分化,形成多层,以后角化。黏膜上皮也以同样的方式再生,新生的黏膜上皮细胞初起为立方形,以后增高变为柱状。

(2)腺上皮再生:腺体受损伤后,若基底膜未被破坏,残存的腺上皮分裂增生,可恢复原有的结构和功能。若腺体(包括基底膜)完全破坏,则难以再生。肝细胞有活跃的再生能力,但如肝内网状支架塌陷,再生的肝细胞则形成结构紊乱的肝细胞结节。

\paragraph{血管的再生}
毛细血管多以出芽方式再生。原有毛细血管内皮细胞肥大、分裂增生,形成向血管外突起的幼芽。开始幼芽为实心的细胞条索,在血流冲击下形成管腔,并有血液通过,进而互相吻合构成毛细血管网(图\ref{fig2-1})。为适应功能需要,毛细血管不断改建,部分管腔关闭消失,部分管壁增厚,成为小动脉、小静脉,其平滑肌等成分可由血管外未分化的间叶细胞分化而来。

大血管离断后需手术吻合,吻合处两侧的内皮细胞分裂增生,互相连接,恢复原来的内膜结构。离断处的肌层难以再生,由结缔组织连接,通过瘢痕修复。

\paragraph{纤维组织再生}
纤维组织普遍分布于机体各部位,再生能力很强,是病理性再生中最常见的现象。在损伤的刺激下,局部静止状态的纤维细胞,或未分化的间叶细胞分化形成幼稚的纤维母细胞。幼稚的纤维母细胞胞体大、胞浆丰富略嗜碱性,两端常有突起。电镜下胞浆内有丰富的粗面内质网和高尔基器,提示其合成蛋白的功能活跃。当纤维母细胞停止分裂后,开始合成并分泌原胶原蛋白,在细胞周围形成胶原纤维。随着细胞的成熟,周围胶原纤维逐渐增多,于是胞体大、有突起的纤维母细胞则变成长梭形的半静止状态的纤维细胞(图\ref{fig2-2})。
\begin{figure}[!h]
	\begin{center}
		\includegraphics{./images/Image00024.jpg}
	\end{center}
	\captionsetup{justification=centering}
	\caption{毛细血管再生模式图 \\ {\small 毛细血管内皮细胞增生;增生的内皮细胞形成条索,并出现管腔;新生的毛细血管相互连接、沟通}}
	\label{fig2-1}
\end{figure}
%\FloatBarrier


\begin{figure}[!h]
	\begin{center}
		\includegraphics{./images/Image00025.jpg}
	\end{center}
	\captionsetup{justification=centering}
	\caption{纤维母细胞产生胶原纤维并转变为纤维细胞的模式图}
	\label{fig2-2}
\end{figure}
%\FloatBarrier

\paragraph{神经组织的再生}
脑和脊髓内的神经细胞破坏后不能再生,由再生能力较强的胶质细胞形成胶质纤维填补,形成胶质瘢痕。但神经纤维断离后,如果与其相连的神经细胞仍然存活,则可再生。首先断处远侧端的神经髓鞘及轴突崩解吸收,断处近侧一小段神经纤维亦发生同样变化。然后两端的神经膜细胞增生,将断端连接,并产生髓磷脂将轴突包绕,形成髓鞘。近端新生的轴突伸向远端髓鞘内,最终达到该神经末稍,可以完全恢复其功能。由于神经轴突生长缓慢(每天延长1~2
mm),再生过程常需数月以上才能完成。如果近端再生的神经轴突未能向远端髓鞘内伸展,只在断裂处长出很多细支,与周围增生的纤维组织缠绕在一起,可形成瘤状物,即创伤性神经瘤(traumatic
neuroma),可引起顽固性疼痛。为防止上述情况发生,临床上常施行神经吻合术或对截肢神经断端作适当处理。

\section{纤维性修复}

纤维性修复开始于肉芽组织增生,填补组织缺损,以后肉芽组织经过纤维化的过程,转化为胶原纤维为主的瘢痕组织,这种修复便告完成。

\subsection{肉芽组织}

肉芽组织(granulation
tissue)由新生的毛细血管、增生的纤维母细胞及多少不等的炎细胞组成,在创伤表面常呈鲜红色,颗粒状,柔软湿润,似新鲜肉芽(图\ref{fig2-3}),故此得名。组织损伤后24小时内,血管内皮细胞及纤维母细胞开始增生,新生的毛细血管管壁的基底膜和胶原纤维尚不完整,故血管通透性大,富有蛋白的液体甚至红细胞漏出到血管外间隙,使肉芽组织呈水肿样外观。新生的毛细血管常呈平行排列,与创面垂直生长,近伤口表面处互相吻合,形成弓状突起。与此同时,局部组织的纤维母细胞受刺激,分裂增生,并产生胶原纤维(图\ref{fig2-4})。毛细血管与血管之间增生的纤维母细胞一起构成小团块,均匀分布,突起于创面,呈颗粒状。肉芽组织中有些细胞外形似纤维母细胞,除能产生胶原纤维外,胞浆中还含有丰富的肌动蛋白和肌凝蛋白,电镜下胞浆内具有丰富的肌微丝,具有类似平滑肌的收缩能力,这种变异的细胞被称为肌纤维母细胞,在创伤收缩中起重要作用。肌纤维母细胞的起源不明,可能来自未分化的间叶细胞,也可能是一种特殊分化的纤维母细胞。炎细胞中以巨噬细胞为主,也可有中性粒细胞及淋巴细胞。巨噬细胞和中性粒细胞具有吞噬细菌和组织碎片的作用,这些细胞坏死后释放的蛋白水解酶能分解坏死组织及纤维蛋白。

\begin{figure}[!h]
	\begin{center}
		\includegraphics{./images/Image00026.jpg}
	\end{center}
	\captionsetup{justification=centering}
	\caption{创口表面颗粒状肉芽组织}
	\label{fig2-3}
\end{figure}


\begin{figure}[!h]
	\begin{center}
		\includegraphics{./images/Image00027.jpg}
	\end{center}
	\captionsetup{justification=centering}
	\caption{肉芽组织镜下观}
	\label{fig2-4}
\end{figure}

肉芽组织在修复过程中有抗感染及保护创面,机化血凝块、坏死组织及其他异物,填补伤口或其他组织缺损等作用。

\subsection{瘢痕组织}

肉芽组织形成的初期呈鲜红色、颗粒状,如嫩芽,以后细胞间水分逐渐减少,纤维母细胞合成胶原纤维,并逐渐转变为纤维细胞。随着细胞外胶原纤维增多,多数毛细血管逐渐关闭、退化、消失,少数改建为小动脉、小静脉。肉芽组织中的炎细胞也先后消失。经过上述纤维化过程,肉芽组织转变为血管稀少,主要由胶原纤维组成的瘢痕组织(图\ref{fig2-5})。肉眼观:呈灰白色,质硬,缺乏弹性。瘢痕组织中胶原纤维经过不断的溶解、形成和改建,最终排列方向与创面平行,以适应伤口修复后的强度需要。

\begin{figure}[!htbp]
	\centering
	\includegraphics{./images/Image00028.jpg}
	\caption{肉芽组织转变为瘢痕组织镜下观(HE染色,低倍) \\ {\small 毛细血管明显减少,胶原纤维沉积增多}}
	\label{fig2-5}
\end{figure}

瘢痕组织中血管少,细胞少,胶原纤维较多较粗,常有玻璃样变性。由于瘢痕组织内肌纤维母细胞的收缩及后期瘢痕内水分明显减少,引起病灶体积缩小,此即瘢痕收缩。瘢痕收缩可引起组织、器官表面凹陷或器官变形,还可造成腔道狭窄。关节附近的瘢痕可致关节运动障碍。瘢痕愈大,影响愈甚。发生在重要器官的瘢痕收缩后将造成严重后果。例如,心瓣膜上的瘢痕可引起瓣膜闭锁不全或瓣膜口狭窄,造成血流动力学的改变,严重者可导致心力衰竭。一般情况下,瘢痕中的胶原纤维在胶原酶的作用下逐渐降解吸收,瘢痕缓慢变小、变软,偶尔瘢痕中胶原纤维形成过多,可成为大而不规则的硬结。少数“瘢痕体质”者,轻微创伤后就可形成明显的瘢痕,过度的瘢痕形成称为瘢痕疙瘩。

\section{创伤愈合}

创伤愈合(wound
healing)是机体组织遭受创伤后进行再生修复的过程,它包括创伤周围特异性组织细胞再生,以及肉芽组织形成、纤维化,最后形成瘢痕组织的复杂过程。

\subsection{创伤愈合的基本过程}

\paragraph{伤口的早期变化}
伤口局部有不同程度的组织损伤、出血及炎症反应。血液和炎性渗出物中的纤维蛋白凝固成血凝块充满缺口,血凝块表面脱水、干燥形成痂皮。血凝块和痂皮对伤口起填充和保护作用,血凝块中的血小板及单核细胞等具有促进局部细胞再生的作用。

\paragraph{伤口收缩}
2~3天后,伤口边缘的皮肤和皮下组织向中心移动,创面缩小。动物实验证明,有些部位的创面在15天内可缩小80%,对愈合十分有利。创面缩小与肉芽组织中肌纤维母细胞收缩有关。

\paragraph{肉芽组织增生及瘢痕形成}
大约从第3天开始,自创缘长出肉芽组织,并向伤口中的血凝块内延伸,机化血凝块。第5~6天起,纤维母细胞产生胶原纤维,其后一周胶原纤维形成极为活跃,以后逐渐缓慢下来。随胶原纤维增多,形成瘢痕组织,大约在伤后一个月瘢痕完全形成。瘢痕组织抗拉力的强度只有正常皮肤的70%~80%,因此腹壁和心脏等部位的较大瘢痕,在内压的作用下可膨出形成腹壁疝或室壁瘤。

\paragraph{表皮及其他组织再生}
表皮再生经过细胞移动、细胞增生及细胞分化三个连续过程。受伤后24小时内,创缘上皮基底层细胞,开始在血凝块下面向伤口中心移动、增生,伤后48小时连接成片,形成菲薄的单层上皮,然后分化。伤后5天内就可恢复原有上皮层厚度并具有角化层的正常表皮结构。

如伤口过大(直径>20
cm)再生表皮难以将创口完全覆盖,往往需要植皮。毛囊、汗腺、皮脂腺等组织若完全破坏,则不能再生,由瘢痕修复。

\subsection{皮肤的创伤愈合}

根据创伤程度及有无感染可分为三种类型。

\paragraph{一期愈合(healing by first intention)}
见于组织缺损少、创缘整齐、创面对合好、无感染、炎症反应轻微的伤口。例如手术切口,切口内只有少量血凝块,创缘炎症反应轻微,第二天表皮再生,在48小时内形成连续的上皮细胞层,覆盖创面,将之与炎性渗出物及血凝块分开。第三天肉芽组织从创缘长出并很快填满伤口,5~6天胶原形成(此时可拆线),2~3周完全愈合,留下一条线状瘢痕(图\ref{fig2-6})。

\begin{figure}[!htbp]
	\centering
	\includegraphics{./images/Image00029.jpg}
	\caption{皮肤一期愈合}
	\label{fig2-6}
\end{figure}

\paragraph{二期愈合(healing by second intention)}
见于创伤组织缺损大,创缘不整齐,伴有感染,炎症反应明显的伤口。愈合由创伤底部向上进行,由于创伤大,需要较多的肉芽组织才能填补缺损,这类创伤坏死组织出血多,并有感染,影响上皮细胞增生移行及肉芽组织的生长,需要清除坏死组织,控制感染,创伤才能愈合。二期愈合和一期愈合的基本过程相同,但需时较长。由于二期愈合肉芽组织增生明显,愈合后形成的瘢痕较大(图\ref{fig2-7}),常影响脏器的外形和功能。若条件允许,可行清创术以达到一期愈合的目的。

\paragraph{痂下愈合(healing under scar)}
创伤表面的血液、渗出液及坏死组织凝固干燥,形成黑褐色硬痂,在痂下进行上述的愈合过程(图\ref{fig2-8}),待上皮再生完成后,硬痂脱落。其愈合时间通常较无痂者长。如痂下有较多的渗出液,易继发感染,不利于愈合。
\begin{figure}[!htbp]
	\centering
	\includegraphics{./images/Image00030.jpg}
	\caption{创伤愈合}
	\label{fig2-7}
\end{figure}

\begin{figure}[!htbp]
	\centering
	\includegraphics{./images/Image00031.jpg}
	\caption{痂下愈合(HE染色,低倍){\small 皮肤创面有血痂形成,上皮已经再生完成,肉芽组织内仍有较多的炎细胞浸润}}
	\label{fig2-8}
\end{figure}


\subsection{骨折愈合}

骨折通常可分为外伤性骨折和病理性骨折两大类。骨的再生能力很强,骨折后大都能完全恢复,其愈合基础是骨膜细胞再生。因其结构和功能的特殊性,愈合过程较复杂,可分为以下几个阶段(图\ref{fig2-9})。

\begin{figure}[!htbp]
	\centering
	\includegraphics{./images/Image00032.jpg}
	\caption{骨折愈合过程}
	\label{fig2-9}
\end{figure}

\paragraph{血肿形成}
骨折时,局部骨和软组织受损伤,血管破裂出血,填充在骨折两端及其周围组织间,形成血肿。骨折局部还可见轻度的炎症反应。

\paragraph{纤维性骨痂形成}
骨折2~3天后,血肿开始由肉芽组织取代而机化,增生的肉芽组织填充和桥接骨折断端,使局部呈梭形膨大,继而纤维化,称为纤维性骨痂,起到初步固定作用。

\paragraph{骨性骨痂形成}
骨折愈合过程进一步发展,纤维性骨痂逐渐分化出骨母细胞及软骨母细胞。骨母细胞分泌基质,逐渐成熟为骨细胞,形成类骨组织,类骨组织经钙盐沉着后变为骨组织,即骨性骨痂。此过程约需几周。骨性骨痂中骨小梁排列紊乱,结构不够致密,仍达不到正常功能需要。软骨母细胞也可经过软骨内化骨形成骨性骨痂,但所需时间较长。软骨的形成与骨折后断端固定不良有关。

\paragraph{骨痂改建或再塑}
上述骨痂形成后,骨折断端被幼稚的、排列不规则的编织骨连接起来,属临床愈合。为了适应生理要求,还需要进一步改建为成熟的板状骨,并重新恢复皮质骨和骨髓腔的正常关系。改建是在破骨细胞的骨质吸收及骨母细胞新骨形成协调作用下进行的。改建后新骨的排列将适应该骨活动时承受压力的方向。骨痂的改建过程在儿童需1~2年,成人需要更长时间。

\subsection{影响创伤愈合的因素}

影响创伤愈合的因素多种多样,了解的目的是为了避免不利因素,创造有利条件,加速组织再生修复。

\subsubsection{全身因素}

\paragraph{年龄}
儿童和青少年较老年人组织再生能力强,愈合快。这可能与老年人常有动脉粥样硬化、血液供应减少、代谢减慢、免疫力降低等有关。

\paragraph{营养}
营养物质缺乏,特别是蛋白质和维生素C,对愈合有很大影响。长期蛋白质缺乏,其中含硫氨基酸蛋氨酸、胱氨酸缺乏时影响前胶原分子形成,不仅使创面愈合速度减慢,而且抗张力强度减低。锌缺乏时将影响DNA和RNA的合成,细胞增生缓慢,延缓创伤愈合。

\paragraph{疾病}
某些疾病,如糖尿病、尿毒症、肿瘤恶病质及一些免疫缺陷病等均可影响再生修复。糖尿病患者白细胞功能降低,对细菌微生物的易感性增加。此外,凡引起小血管闭塞及神经的病变都将影响愈合。

\paragraph{激素}
特别是皮质醇类激素能抑制炎症的渗出反应。临床上用皮质醇处理的病人,创伤处巨噬细胞稀少,影响肉芽组织的形成和创伤收缩。因此,在炎症修复过程中皮质醇类激素的使用要慎重。

\subsubsection{局部因素}

\paragraph{感染和异物}
感染使渗出物增多,从而增加局部创口的张力,甚至引起伤口裂开。许多化脓菌产生的毒素和酶能引起组织坏死,基质和胶原纤维溶解,加重局部损伤,因此只有当创伤局部感染被控制后,修复才能顺利进行。异物(如丝线等)可对局部组织有刺激作用,引起异物反应,妨碍修复。

\paragraph{局部血循环障碍}
血液供应对创伤愈合很重要,凡是引起动脉血供应不足,或静脉血流不畅的疾病都将影响局部创伤的愈合。如下肢静脉曲张患者,小腿发生溃疡后,常迁延不愈,变为慢性溃疡。X线长期照射的部位,小动脉壁增厚,管腔变窄,局部组织供血不良,损伤后修复缓慢。

\paragraph{神经支配}
正常的神经支配对维持组织结构及功能极为重要,失去神经支配的组织就失去了对损伤的反应。正常的神经功能与再生修复亦有一定关系,例如麻风病引起的溃疡不易愈合,这与麻风病患者肢体神经受累有关。

\section{再生修复的机制}

组织损伤修复的机制极为复杂,涉及损伤局部的炎症反应、各种化学因子的释放、干细胞和纤维母细胞的激活和增殖、细胞外基质的产生以及与细胞之间的相互作用、增生程度的控制、修复后重塑等(图\ref{fig2-10})。

\begin{figure}[!htbp]
	\centering
	\includegraphics{./images/Image00033.jpg}
	\caption{损伤修复机制}
	\label{fig2-10}
\end{figure}


\subsubsection{干细胞}

干细胞(stem
cell)是一类未充分分化且具有自我复制能力(self-renewing)的多潜能细胞。在一定条件下,它可以分化成多种功能细胞。根据干细胞所处的发育阶段分为胚胎干细胞(embryonic
stem cell,ES细胞)和成体干细胞(somatic stem
cell),近年科学家还在实验室用基因工程方法构建了诱导型多能干细胞。根据干细胞的发育潜能分为三类:全能干细胞(totipotent
stem cell,TSC)、多能干细胞(pluripotent stem
cell)和单能干细胞(unipotent stem
cell)(专能干细胞)。干细胞具有再生各种组织器官和人体的潜在功能。

组织损伤后,干细胞激活,可向特定方向分化、增殖,修复组织缺损。

\subsubsection{生长因子}

细胞受到损伤因素刺激后,可通过释放多种生长因子(growth
factor),刺激同类细胞或同一胚层发育来的细胞增生,促进修复过程。生长因子在细胞移动、收缩和分化中也发挥重要作用。常见的有以下几种:

\paragraph{血管内皮生长因子(vascular endothelial growth factor,VEGF)}
是至今发现的最强的血管通透促进剂,可促进内皮细胞增殖,在胚胎发育、创伤愈合等生理及病理过程中具有明显的促血管增生作用。

\paragraph{纤维母细胞生长因子(fibroblast growth factor,FGF)}
具有广泛的生物学活性,能影响多种细胞(血管内皮细胞、平滑肌细胞、纤维母细胞等)的生长、分化及功能。FGF可使血管内皮细胞分裂并诱导其产生蛋白溶解酶,后者溶解基膜,便于内皮细胞穿越生芽。

\paragraph{血小板源性生长因子(platelet derived growth factor,PDGF)}
主要由黏附于血管损伤处血小板的α颗粒释放,能刺激血管平滑肌细胞、纤维母细胞和胶质细胞等的分裂、增殖,通过刺激胶原合成和胶原酶的活化作用,调节细胞外基质的更新。

\paragraph{表皮生长因子(epidermal growth factor,EGF)}
通过作用于靶细胞膜上的特异性受体而发挥多种生物学效应,是一种强有力的促细胞分裂、分化和增殖的因子,对上皮细胞、纤维母细胞、平滑肌细胞都有促进增殖的作用。

\paragraph{转化生长因子(transforming growth factor,TGF)}
TGF-α可与EGF受体结合,与EGF具有类似作用。TGF-β具有复杂的生物学功能,对纤维母细胞和平滑肌细胞增生的作用依其浓度而异,高浓度可抑制
PDGF受体表达,使其生长受到抑制,低浓度诱导PDGF合成、分泌。

\paragraph{肿瘤坏死因子(tumor necrosis factor,TNF)}
是多功能的多肽,可促进内皮细胞分化,诱导基质产生,也可间接刺激其他细胞产生血管生长因子。在体内可促进内皮细胞形成血管,在体外可刺激培养的内皮细胞形成管样结构。

\subsubsection{细胞外基质及其受体}

人体各种组织均由细胞外基质(extracellular
matrix,ECM)构成支架,它的主要作用是把细胞连接在一起,借以支撑和维持组织的生理结构和功能。ECM能影响细胞的形态、分化、迁移、增殖和生物学功能,在调控胚胎发育、创伤修复及肿瘤浸润转移等方面都起着重要作用。研究表明,尽管不稳定细胞和稳定细胞都具有完全再生能力,但能否重新构建为正常结构尚依赖ECM。

ECM的主要成分如下:

\paragraph{胶原蛋白和弹力蛋白}
胶原蛋白(collagen)是ECM的主要组成成分,几乎分布于所有组织中,为多细胞生物提供细胞外支架。目前发现的胶原类型达18种之多,其中Ⅰ~Ⅳ型含量较多。Ⅰ、Ⅱ、Ⅲ型胶原为纤维性胶原,Ⅰ和Ⅲ型主要分布于间质结缔组织中,Ⅱ型胶原则主要分布于软骨;Ⅳ型胶原为基底膜胶原,在基底膜主要基质蛋白成分中占60%。弹力蛋白(elastin)分子结构与胶原蛋白相似,但分子间交联较少。主要存在于血管、皮肤、韧带、肺等组织中,分子量约70kD,对维持组织的弹性与张力起重要作用。

\paragraph{蛋白多糖}
蛋白多糖(proteoglycans)是ECM的另一重要成分,其结构包括核心蛋白及与其相连接的多糖或多个多糖聚合形成的氨基多糖(glycosaminoglycans)。常见的蛋白多糖有硫酸肝素、硫酸软骨素、硫酸皮肤素、硫酸角质素和透明质酸等,其功能主要是通过介导一系列生物大分子之间的信息传递参与组织的发育和维持正常的生理功能。透明质酸是大分子蛋白多糖复合物的骨架,与调节细胞增殖和迁移有关。

\paragraph{黏附性糖蛋白}
黏附性糖蛋白(adhesive
glycoproteins)既能与其他细胞外基质结合,又能与特异性的细胞表面蛋白结合,将不同的细胞外基质与细胞之间联系起来。纤维连接蛋白(fibronectin)作为一种多功能的黏附性糖蛋白,能使细胞与各种基质成分发生粘连,与细胞黏附、细胞迁移等功能直接相关。层黏连蛋白(laminin)可与细胞表面的特异性受体结合,也可与基质成分如IV型胶原和硫酸肝素结合,还可介导细胞与结缔组织基质黏附。

\paragraph{整合素}
整合素(integrins)是位于细胞膜上的细胞外基质受体,对细胞和细胞外基质的黏附起介导作用,可将来自细胞外基质之信号传入细胞。其特殊类型在白细胞黏附过程中还可诱导细胞与细胞间相互作用。

\subsubsection{抑素与接触抑制}

抑素(chalon)具有组织特异性,似乎任何组织都可以产生一种抑素抑制本身的增殖。如已分化的表皮细胞能分泌表皮抑素,抑制基底细胞增殖。当已分化的表皮细胞丧失时,抑素分泌终止,基底细胞分裂增生,直到增生分化的细胞达到足够数量或抑制达到足够浓度为止。TGF-β虽然对某些间叶细胞增殖起促进作用,但对上皮细胞则是一种抑素。此外干扰素-α、前列腺素E2和肝素在组织培养中对成纤维细胞及平滑肌细胞的增生都有抑素样作用。

皮肤创伤,缺损部周围上皮细胞移动,分裂增生,将创伤面覆盖而相互接触时,或部分切除后的肝脏,当肝细胞增生达到原有大小时,细胞停止生长,不至堆积起来。这种现象称为接触抑制(contact
inhibition)。细胞缝隙连接(可能还有桥粒)也许参与接触抑制的调控。

\begin{center}
	\textbf{知识链接}
\end{center}
\chapterabstract{生物敷料可以与伤口密切贴合,保持愈合环境湿润,减轻疼痛,辅助局部使用药物和内源性分子促进伤口愈合。胶原、透明质酸等材料制备的生物敷料不仅具有止血促凝作用,还可影响生长因子(VEGF、FGF、TGF)分泌,诱导多种细胞增殖分化,有利于伤口愈合。}

{【附】与创伤愈合有关的生长因子}

对单核细胞具有趋化作用:PDGF、FGF、TGF-β

纤维母细胞迁移:PDGF、EGF、FGF、TGF-β、TNF

纤维母细胞增殖:PDGF、CTGF、EGF、FGF、TNF

血管生成:VEGF、FGF

胶原合成:TGF-β、PDGF、TNF

分泌胶原酶:PDGF、FGF、EGF、TNF、TGF-β抑制物

\section*{复习与思考}

{一、名词解释}

修复 再生 纤维性修复 稳定性细胞 永久性细胞 肉芽组织 一期愈合

{二、问答题}

1. 试述肉芽组织的结构及其在修复过程中的作用。

2. 影响细胞再生的因素有哪些?

3. 影响创伤愈合的因素有哪些?

4. 试述骨折愈合的基本过程。

\chapter{超声在休克和循环功能监测及支持中的应用}

\section{前沿学术综述}

超声心动图是目前能够在床旁提供实时有关心脏结构和功能信息的唯一影像工具。多普勒心脏超声技术可以更加详细地评估患者的血流动力学改变,因而更有助于快速明确导致急性循环衰竭的机制与原因。由于可以在很短的时间内准确评估血流动力学状态,心脏超声对于休克或存在循环衰竭的重症患者,无论是早期识别与评估,还是整个诊疗过程中都有理由成为适合的理想的监测工具。另外,随着科学技术和电子技术的快速进步、经食管的多平面探头的出现,使心脏超声的图像质量大幅提高,使一些过去经胸心脏超声很难获得满意图像的患者也可以获得可靠的相关信息。目前许多研究表明,心脏超声在重症患者的应用,可以促使患者的治疗产生有益的改变
\protect\hyperlink{text00009.htmlux5cux23ch1-8}{\textsuperscript{{[}1{]}}}
。同时,值得关注的是,肺部超声、肾脏超声在重症监测的快速发展进一步丰富了超声在休克和循环功能监测及支持中的应用,因此超声作为有前途的重症监测与支持工具在重症医学科的应用中逐渐走向成熟与普及。

\subsubsection{心脏超声在重症医学科中应用的发展与特点}

早期的综合重症医学科,心脏超声检查大多由通过资质认证的心脏专科医生来进行,主要目的是快速准确获得图像,帮助诊断心血管疾病,如心包填塞、急性心肌梗死的并发症、自发的主动脉夹层和创伤性主动脉损伤等。而对于血流动力学的无创评估仅仅是应用二维技术联合多普勒模式来测量每搏输出量和每分心脏输出量。事实上,当时的重症医学科医生对心脏超声的潜力和作用缺乏全面的认识。直到20世纪80年代中期,一些重症医学科医生中的先行者开始拓展应用心脏超声对血流动力学的全面而详尽的评估。首先推荐用于感染性休克和急性呼吸窘迫综合征患者,应用心脏超声替代右心漂浮导管进行血流动力学评估,并且率先开始自己进行心脏超声检查,尤其是可以24小时随时床旁进行重复检查和评估,并且指导治疗。随后由于在循环衰竭诊断与评估应用的扩展、随着监测和测量经验的积累,尤其经食管超声(TEE)准确度的增加,重症患者床旁超声的应用价值逐步得到认识和肯定,有研究表明其对治疗支持的影响和预测病死率有重要作用。

但直到上世纪90年代,重症医学科医生对心脏超声的兴趣才真正开始明显增加,主要原因有:心脏漂浮导管研究出现大量阴性甚至负面结果;与传统有创血流动力学评估手段相比心脏超声无创、实用;大量相关研究文献发表和大量相关重症医学科医生心脏超声培训课程出现使得重症医学科医生的超声应用技能得以明显提高。在这一时期,一些官方组织开始推荐经食管超声作为急性循环衰竭的一线评估手段。

近年来,功能血流动力学评估概念的提出,再次间接推动了心脏超声在重症医学科循环衰竭患者中的应用。越来越多证据显示,超声检查参数可准确评估重症医学科机械通气的感染性休克患者的心功能和液体反应性,而这些参数丰富了重症医学科时刻存在的心功能和液体反应性评估指标,同时大大激发了重症医学科医生对心脏超声的兴趣
\protect\hyperlink{text00009.htmlux5cux23ch2-8}{\textsuperscript{{[}2{]}}}
。

\subsubsection{心脏超声在评估心脏前负荷及容量反应性方面的作用}

众所周知,在重症医学科管理血流动力学不稳定的患者时,最常见的临床行为就是实现以提高心输出量和组织灌注为目的的血管内容量和心脏前负荷的最佳化调节。而在此调节过程中,无论是让患者处于容量不足还是容量过负荷状态均会产生严重的后果,评估患者的容量状态极为重要。所以在有指征给患者输液时,进行容量反应性的评估尤为重要,而心脏超声给我们提供了更多更准确更便捷的选择。

心脏超声能够评估患者的容量状态,是传统有创血流动力学监测评估的有益补充,更有可能更加可信可靠。一般情况下,经胸心脏超声已经可以提供足够可用的信息。当经胸超声图像欠理想时,经食管超声检查可以提供理想图像,用于比经胸心脏超声更准确的评估心内流量、心肺相互作用、上腔静脉的扩张变异度等。

心脏超声对容量状态的评估可采用静态或动态指标,静态指标即单一的测量心脏内径大小和流量快慢;动态指标用来判断液体反应性,包括自主或机械通气时呼吸负荷的变化、被动抬腿试验和容量负荷试验等。其中,动态指标临床使用更多。心肺相互作用的指标如上腔静脉塌陷率、下腔静脉扩张指数、左室射血的呼吸变化率等,用于预测窦性心律、无自主呼吸机械通气患者的容量反应性;被动抬腿试验相当于内源性容量负荷试验,通过超声观察抬腿前后左室射血流速增加情况来预测容量反应性,无论患者自主呼吸或机械通气、任何心律情况下,均可应用。临床治疗中,可动态和静态指标联合应用进行评估。如严重低血容量时评估的超声征象:功能增强但容积很小的左室;自主呼吸时下腔静脉吸气塌陷非常小;机械通气患者呼气末下腔静脉呼吸变化非常小。

评估容量反应性时,必须考虑以下因素:①容量反应性的评估需要测量多个参数,综合分析;②左室或右室内径大小的变化对容量反应性的预测不可靠;③评估容量反应性时,必须考虑自主呼吸与正压通气对采用指标的不同影响,当患者存在心律失常或自主呼吸时,应用心肺相互作用的指标评估容量反应性并不准确,可选择被动抬腿试验;④非心脏超声获得的心肺相互作用评估容量反应性(如脉压呼吸变化率)的假阳性原因(尤其严重右心衰)易于通过心脏超声检查明确。

总之,心脏超声在评估心脏前负荷及容量反应性方面可用、有效且极具前景。

\subsubsection{心脏超声在评估心功能中的作用}

重症患者心功能的改变非常常见,如心功能衰竭或心肌抑制,此时心室收缩、舒张功能的定量分析对于病情监测、指导治疗和判断预后具有十分重要的临床意义。心脏超声通过二维心脏超声、M型心脏超声、利用几何模型的容量测定、辛普森法、组织多普勒技术、Tei指数和三维心脏超声等方法对心脏功能进行评估,无创且便捷。心功能测定包括左(右)心室收缩和舒张功能测定,其中,左心室功能检测在临床病情评估和治疗中最为重要
\protect\hyperlink{text00009.htmlux5cux23ch3-8}{\textsuperscript{{[}3{]}}}
。

射血分数是目前研究最多,且最为临床所接受的心脏功能指标,具有容易获得(甚至有经验的操作者目测的结果与实测结果相差很小,相关系数达0.91)、可重复性好以及能够较早评价全心收缩功能等优点(不同于环周纤维缩短率,在有节段异常时,也经常发生改变)。目前研究表明,射血分数是与预后最相关的心功能指标。射血分数的测量方法很多,其中Simpson最准确,被美国超声学会所推荐。但最大的缺陷在于对心内膜边缘的确认水平要求足够高,两腔像与四腔像要求垂直,而且操作略显繁杂费时。射血分数值作为一个最重要的评价心脏收缩功能指标,也具有明显的局限性,受前后负荷的影响非常明显。前负荷增加通过Frank-Starling机制增加射血分数值,而后负荷增加抑制射血分数值,如在没有血管活性药物支持、仅扩容治疗的感染性休克患者,前负荷稳定或增加,同时血压/外周阻力明显下降都会导致射血分数测量值不能代表心肌的真实收缩功能。另外一个重要的心功能指标是平均环周纤维缩短率,最大优点在于不依赖于前负荷改变,同时,经过心率纠正后的指标心率纠正的平均环周纤维缩短率,由于去除了心率的影响,似乎比射血分数能更好地反映心肌收缩功能。

有研究显示组织多普勒技术测定的心肌收缩速度可以代表全心室功能,尤其可反映二尖瓣环心肌收缩速度。另外,有研究表明,尽管存在对前后负荷的依赖,在肥厚性心肌病和舒张功能不全的患者,运用组织多普勒技术测定的心肌收缩速度指标可以在显性心肌肥厚和心脏收缩功能不全之前即发现渐进的心肌收缩功能受损,同时,这些指标对受心脏前后负荷的影响不大
\protect\hyperlink{text00009.htmlux5cux23ch4-8}{\textsuperscript{{[}4{]}}}
。

综上所述,近年来,在心脏超声多普勒技术领域,评估左心室收缩功能的进展主要集中在两个方向。首先是探索对负荷依赖程度低的指标,即接近心肌内在性能的指标,如左心室等容收缩压力增加速率,不依赖后负荷而对前负荷轻度依赖,同时,已有许多研究表明这些指标有助于预后判断;其次是研究心肌本身的指标,以往的许多指标大多依赖于血容量(腔室的大小)和血流(多普勒流速和压力的变化)进行测量,而随着超声多普勒技术的进步,尤其是组织多普勒的发展,最近的研究则侧重于应用无创技术测量心肌本身或其内在的机能。目前可测得的主要指标包括心肌收缩速度、左室质量、应变和应变率以及与应力的关系等。这些指标对患者预后影响的研究尚少,尤其缺乏大规模研究,仅发现充血性心衰患者心肌收缩速度<5厘米/秒可预测心脏不良事件的发生。

组织多普勒技术测定的Tei指数又称为心肌做功指数,心肌做功指数=(心室等容收缩时间+心室等容舒张时间)/心室射血时间。该指数于1995年由日本学者Tei提出,无创、敏感,能综合反映心室收缩及舒张功能,是可行的评价左室功能的指标,是对常规测定的血流多普勒参数的重要补充。目前尚无公认的正常值。

实时三维心脏超声全面、快速准确地测定左室功能,一直是心脏超声工作者的梦想。有人研究应用这一新的技术,测定正常人和心脏病患者的左室射血分数,并与常规双平面二维改良Simpson's法测定左室射血分数进行对照,证明实时一次心动周期三维超声即能准确、快速测定左室射血分数。实时三维心脏超声可以产生实时三维的心脏图像及左室容积时间曲线,克服了二维超声的限制,在测量心室容积时不需要几何形状的假定,不受心脏几何形态的影响,因而测量的结果更为准确,能全面实时地观察和测量动态心室的整体及局部容积大小、运动及功能状态,从而提高心功能评估的可靠性,是一种无创的新方法。

\subsubsection{心脏超声对外周血管阻力的评估}

心脏超声多普勒技术可以直接测量外周血管阻力,但不易方便和简单使用,因此在临床工作当中,经常根据临床和心脏超声的检查结果进行排除诊断,如在心脏负荷足够同时左右心脏收缩功能均满意的情况下,仍然存在低血压则提示外周血管阻力低。

\subsubsection{心脏超声在特殊情况下的应用}

严重感染和感染性休克是常见病、多发病,与急性心肌梗死发病率相当,甚至高于许多肿瘤的发病率,是住院患者最常见的死亡原因之一,且病死率随着年龄增加而增加,甚至大于急性心肌梗死,达到30%~60%。其中,早期出现心功能异常的患者若表现为低心排,死亡率>80%。另有研究提示,合并出现心血管损害的全身性感染患者,死亡率由20%升至70%~90%。

临床上常见严重感染和感染性休克时,心输出量并不降低或反而增加,但合并心肌功能不全。这种心功能不全多出现于感染性休克早期,往往难以早期发现及处理,造成的危害极大。随着心脏超声在评估左室心脏功能应用的进展,目前已被应用于感染性休克相关的心肌抑制的早期发现与指导支持治疗
\protect\hyperlink{text00009.htmlux5cux23ch5-8}{\textsuperscript{{[}5{]}}}
。目前常用指标有射血分数、环周纤维缩短率、心肌收缩速度等,而应用应变和应变率以及与应力的关系等对于早期发现与感染相关的心肌抑制及指导正性肌力药物应用具有更好的前景。

无论是围手术期还是严重创伤患者,缺血性心脏病非常常见。局部心肌缺血导致局部心肌运动异常。临床实际中,局部心肌缺血的评估最常用到的方法是对二维超声显像室壁运动和室壁增厚率进行目测。与心肌节段的室壁增厚率相比较,二维超声应变成像对心肌缺血的变化更加敏感。急性心肌梗死后可出现多种舒张期充盈异常即左心室舒张功能异常,表现为二尖瓣血流频谱E峰峰值速度减低,A峰峰值速度增高,E/A比值<1,E峰减速时间延长,等容舒张时间延长,肺静脉血流频谱S/D峰值比值增加等。另外,随着彩色多普勒心脏超声在临床的广泛运用,急性心肌梗死后左室舒张功能得到更全面深刻的认识,对临床治疗方案的制定和调整也起到重要作用。心肌应变测量的是心肌各节段的变形,在定量评价心肌各节段的收缩和舒张功能时,心肌应变与心肌的收缩和舒张功能密切相关,因此能准确评估心肌收缩和舒张功能。

急性肺血栓栓塞是临床上一种危重心肺疾病,心脏超声对其病变程度、治疗效果及预后评估有重要作用,已经普遍应用于临床。超声检查急性肺血栓栓塞一般包括心脏超声检查及下肢深静脉检查。尤其对于确诊的急性肺血栓栓塞患者,超声探测到中度、重度右室功能障碍者,其近期及长期病死率均明显升高,而不伴有右室负荷过重的患者,近期预后良好。因此超声能够根据右室功能状态进行危险度分层及预后判断。心脏超声可以动态、无创、重复估测肺动脉压力,因此可以判断治疗效果,可以作为随访追踪的一种快速、简便的检查手段。

\subsubsection{肺部超声在循环监测与支持中的作用}

最近几年来,随着肺部超声的进步与推广,成为能够发现与评估不同肺部与胸腔病变的有力技术。肺部超声常见征象与特点包括:①正常通气,胸膜线下平行排列的A线;②肺间质肺泡综合征,彗星尾征,根据B线的间隔不同分为B7线(B线间隔大约7mm,主要是肺小叶间隔增厚)和B3线(B线间隔3mm);③肺实变征,包括组织样征、碎片征和支气管气象;④胸腔积液,静态征象为四边形征,动态征象为水母征和正弦波征;⑤气胸,肺点消失。

以上是常见肺部病变的超声表现。对于肺水肿患者,肺水含量的评估非常重要,肺部超声获得B线可以早期发现在血气分析改变之前的肺水肿,而且超声具有简单、无创、无放射性和实时性等优点。超声监测导向诊断的难点在于急性心源性肺水肿与ARDS肺水肿的鉴别,最新有研究表明,循环支持过程中,肺部超声的A-优势型表现提示肺动脉嵌顿压<13mmHg的可能性大;而在B-优势型时,提示肺动脉嵌顿压>18mmHg可能性大
\protect\hyperlink{text00009.htmlux5cux23ch6-8}{\textsuperscript{{[}6{]}}}
。

\subsubsection{重症肾脏超声在循环监测及休克支持中的作用}

肾脏是休克时最容易受损或最早受损的器官之一,重症患者病变过程中易并发急性肾损伤。术后患者发生率1%,重症患者达到35%,尤其感染性休克患者发生率在50%以上。因此预测、发现和评估急性肾损伤非常重要。重症肾脏超声能够床旁及时无创监测肾脏改变,能够同时关注和监测肾脏大循环与微循环情况,为休克循环监测和支持提供了新的重要思路。

总之,重症超声包括超声心动图、肺部超声和重症肾脏超声在血流动力学评估,尤其对于心脏功能、容量反应性等血流动力学评估的作用越来越重大;在重症医学科常见的重症疾病如休克的监测与支持等诸多方面都开始发挥举足轻重的作用,已经被众多重症医学科医生所接受和掌握。因此,全世界范围内的重症医学科医师的重症超声培训和认证正在如火如荼地进行。

\section{临床问题}

\subsection{超声评价血流动力学的作用}

\subsubsection{为什么超声是评价重症患者血流动力学的重要方法?}

在重症患者中,血流动力学不稳定(急性或慢性)是很常见的问题。长期低血压可能导致器官缺血、功能紊乱等不良后果。相反,快速的诊断和早期干预可以避免血流动力学的进一步恶化。然而,仅仅依靠临床常规检查尚不足以做出正确的诊疗决策。对于不常见的临床问题,临床疑诊是建立鉴别诊断和灵活应用诊疗技术来做出诊疗决策的关键。超声心动图就是能够在不同疾病的快速诊断中发挥重要作用的技术之一。因此,对于患者血流动力学不稳定的原因和监测,超声心动图能够发挥强大作用,可以用于评估前负荷、后负荷和心肌收缩力。各类研究表明,超声心动图的应用使至少1/4的重症患者的治疗有所改变。

应该强调的是,应用超声心动图来评估重症患者,能快速而可靠地排查像肺栓塞和心包填塞等能引起患者血流动力学不稳定的主要病因,而这些操作可由经过简易超声心动图检查训练的重症医学科医生或者急诊医生完成,并且是血流动力学不稳定重症患者评估的关键一步。

在排除了一些主要病因之后,需评价患者的容量状态和心功能。最重要也是最常使用的评价左心室整体或者局部室壁运动的方法,是多切面的定性评估。这种方法快速而有效,并且与核素扫描结果具有很好的一致性。超声心动图的检查结果不仅能评估局部室壁运动,还能通过估计射血分数来评估左心室整体功能。心室功能的定量评估能提供可测量性更好的、误差更少的评价方法。但需要警惕的是,所有有效的评估方法都既有长处,又有各自局限性。

\subsubsection{如何看待经胸壁超声心动图、经食管超声心动图和手持设备在重症医学科的作用?}

在重症医学科中,经食管超声心动图经常被认为比经胸壁超声心动图更有优势,因为后者常常由于下列原因得到的图像质量欠佳:比如术后患者由于机械通气(呼气末正压>15cm
H\textsubscript{2}
O)无法调整体位、缺乏合作耐心、胸壁水肿以及由于伤口敷料、胸腔引流管、胸腹壁开放而使视野阻断。经胸壁超声心动图在被检患者中的成功率为50%~80%,而经食管超声心动图的成功率高达90%。但近年来,更多研究表明经胸壁超声心动图有助于诊疗的超声切面获得率在86%以上。另外,经胸壁超声心动图的常规实施过程面临很多问题。与经胸壁超声心动图相比,经食管超声心动图耗时更长,对专业知识要求更高,而且经食管置入探针有误入气道而阻塞气道的风险。另外,虽然经食管超声心动图会产生像食管穿孔这样的严重并发症,但其可能性较小,大约只有0.01%。

手持式可移动设备轻巧、简单而且方便,能提供定性评估。手持式设备在经超声引导下胸穿以及中心静脉置管等操作中作用明显。新一代的电池供电的检查设备也已出现,这些设备在血流动力学不稳定的重症医学科患者中的地位和应用在进一步加强。

不管检查形式怎样,检查过程本身必须是完整的,并且跟从业人员在训练中要求的一样全面。如果初期检查因为不同原因有所限制,或者结果存在疑问,要求更加有经验的从业人员及早进行更全面的检查。全面检查就是尽量避免罕见疾病的漏诊。经过反复练习之后,完整的检查过程应该在数分钟内完成。合理的检查程序应该是在体格检查的基础上定位于可疑病变部位或结构。一旦解决了直接问题,接下来应该做更加全面的检查,对于可疑病变部位能够有更加充分的检查时间。目前的指南上有经食管超声心动图和经胸壁超声心动图检查的标准图像,以确保所有结构都是从多角度去查看的,而单个结构能被完整而准确的评估并且根据需要被记录下来。标准切面能保证任何结构不被遗漏,还能为从业人员的相互交流提供有效的媒介。

\subsection{超声在容量及容量反应性监测中的作用}

\subsubsection{什么是容量状态与容量反应性?超声检查在其中有什么作用?}

血管内容量和心脏前负荷的最佳化调节是提高心输出量和改善组织灌注的重要环节,通常是血流动力学支持最早期的临床行为。在此调节过程中,评估患者的容量状态极为重要。因为无论是让患者处于容量不足还是容量过负荷状态均会导致严重的后果。所以在有指征给患者输液时,进行容量反应性的评估尤为重要。

目前对容量治疗有反应定义为给予液体治疗后,心输出量指数或每搏输出量指数较前增加≥15%。心脏对容量治疗有反应的生理机制是基于Frank-Starling机制:当心功能处于心功能曲线上升支时,增加前负荷,则可以显著增加心输出量,改善血流动力学,提高氧输送,从而改善组织灌注;而心功能处于平台期时,提高前负荷的潜能有限,扩容则难以进一步增加心输出量,反而可能带来肺水肿等容量过多的危害。

提出容量反应性近20年来,大量研究力图寻找简单可靠并且敏感快捷的指标或方法来预测,进而指导液体治疗,如何选择和应用这些指标也一直是研究的热点。目前预测容量治疗反应的指标或方法,主要包括传统的静态前负荷参数(前负荷压力指标及前负荷容积指标)的监测、容量负荷试验,以及近来研究较多的经心肺相互作用的动态前负荷参数(收缩压变异度、脉搏压变异度、每搏输出量变异度等)和被动腿抬高试验等。

心脏超声能够评估患者的容量状态和容量反应性,是传统有创血流动力学监测评估的有益补充,更有可能比之更加可信可靠。当经胸超声图像欠理想时,经食管超声可以提供理想图像,用于比经胸心脏超声更准确地评估心内流量、心肺相互作用、上腔静脉的变异度等。当然,一般情况下,经胸心脏超声已经可以提供足够可用的信息。心脏超声对容量状态和容量反应性的评估一般包括静态指标和动态指标,静态指标即单一的测量心脏内径、面积及容积大小和流量的快慢;动态指标,广义包括流量和内径大小对于动态手段的变化(自主或机械通气时呼吸负荷的变化、被动腿抬高试验、容量负荷试验等),狭义即指心肺相互关系引导的动态指标。

\subsubsection{根据临床经常面临的容量和容量反应性问题,超声临床判断评估的流程与思路及评估的指标与方法是什么?}

(1)严重容量不足或输液有明显限制时液体反应性的评估 当患者没有进行容量状态和容量反应性评估的指征时,首先可以快速判断是否存在严重容量不足或输液有明显限制及容量过负荷,此时应用的大多为静态指标。

严重低血容量时,预测容量反应性阳性结果的可能非常大。超声评估指标包括:功能增强但容积很小的左室,左心室舒张末期面积<5.5cm\textsuperscript{2}
/m\textsuperscript{2}
体表面积;在自主呼吸时下腔静脉内径小且吸气塌陷非常明显;在机械通气患者呼气末下腔静脉内径非常小,常见<9mm,并且容易随呼吸变化。

容量过负荷或输液限制明显,预测容量反应性阴性可能很大时的超声评估指标包括:在无心包填塞时上下腔静脉有明显充盈表现(扩张或固定);严重右室功能不全及过负荷(右室大于左室的超声证据);心脏超声估测有很高的左室充盈压,如很高的E/E'值。

类似的这些静态指标在评估容量反应性时,有多种影响因素。所以单纯根据一个静态指标评估容量反应性可靠性很差,但对于评估容量明显缺乏和明显过负荷时,却较为可靠,即尽管不敏感,但特异性很强。

(2)既不是严重容量不足、也不是容量过负荷时容量反应性的评估 当患者既不是严重容量不足、也不是容量过负荷,即容量反应性判断比较困难时,此时包括完全机械通气和自主呼吸两种不同的情况,选择的指标和方法如下。

1)完全机械通气容量反应性的评估:在完全机械通气的无心律失常患者,选择心肺相互作用相关的动态指标可以预测容量反应性,如主动脉流速和左室每搏射血的呼吸变化率以及上腔静脉塌陷率、下腔静脉扩张指数等,并且研究证明同非超声获得的动态指标一样,上述指标均明确优于静态指标。

近年来,随着对心肺相互作用认识的进步,在机械通气的患者,左室每搏输出量的呼吸变化率可以作为容量反应性的指标,但由于床旁左室每搏输出量的测量依然复杂而相对困难,所以一些左室每搏输出量呼吸变化率的替代指标被应用和研究,包括动脉监测的脉压呼吸变化率和脉搏轮廓推导的每搏输出量变化率。当然随着心脏超声在重症医学科的更广泛应用,尤其对于血流动力学不稳定患者评估的应用,一些超声检查可以获得的左室每搏输出量呼吸变化率的替代指标被认识和研究应用。2000年前后,Feissel等应用经食管超声测量主动脉瓣环的主动脉血流速的呼吸变化率判断容量反应性,2005年Monnet和Teboul等应用食管多普勒探头直接测量降主动脉峰流速的呼吸变化率来预测容量反应性,均取得理想结果;在儿童相关的研究中,进一步证明经胸超声获得的主动脉峰流速呼吸变化率在预测液体反应性、评估心脏前负荷储备时优于脉搏压变异度和收缩压变异度。另外,在动物研究(阶梯失血兔子模型)中,无论应用经食管超声测量主动脉流速还是经胸超声测量的主动脉血流速度积分呼吸变化率,均可高度准确预测容量反应性。

须说明的是,主动脉流速的测量无论经食管还是经胸,都存在一定的技术问题。而外周的动脉血管,包括桡动脉、肱动脉和股动脉等,其超声血流图像易于获得,因此,近年来研究显示肱动脉峰值血流速的呼吸变化率可预测患者的容量反应性,其敏感度和特异度都达到了90%以上,不亚于脉搏压变异度等动态指标,尤其优于一些静态指标。当然优点还在于完全无创,同时简单易学,甚至于需要培训的时间很短且不需要经验的积累。

对于非外周动脉流速的测量有限性在于需要减低操作者依赖性和进行可重复性可靠性研究,而对于外周动脉,仅仅需要关注局部肌肉收缩对测量的影响。另外尤其要注意这些指标只适用于没有自主呼吸及心律失常的机械通气患者。

使用具有心内膜自动描记功能的超声诊断仪时,可以用左室每搏射血面积呼吸变化率来预测液体反应性。

尽管大规模的荟萃综述分析认为脉搏压变异度是最理想的判断容量反应性的动态指标,但研究对比的对象是收缩压变异度和每搏输出量变异度。在应用超声进行评估时,由于主动脉流速甚至外周动脉的流速变化早于每搏输出量,因此,未来的研究需进一步明确其优越性。

以往的研究多以机械通气的休克患者为研究对象,最近一个关于自主呼吸志愿者的研究证实,在一些较单纯的情况下,如仅仅低血容量时,在自主呼吸状态下主动脉流速的呼吸变化率也可以预测液体反应性,不过此研究需要进一步验证。

另外,还可以通过判断腔静脉的变异度判断容量反应性,如下腔静脉呼吸扩张率和上腔静脉呼吸塌陷率。有研究表明,感染性休克患者下腔静脉扩张率为18%时,预测液体反应性的敏感性和特异度均在90%以上,而上腔静脉呼吸塌陷率的预测值为36%,预测容量反应性的敏感性和特异度也均在90%以上。但需要关注的是,影响腔静脉变异度的因素除了容量状态外还有右心功能和静脉顺应性。下腔静脉呼吸扩张率提出较早,但直到近年,随着对正压通气对下腔静脉影响认识的进步才被广泛接受和应用;而上腔静脉呼吸塌陷率的认识得益于经食管超声在重症患者中的广泛应用,尤其用于对血流动力学不稳定患者的评估
\protect\hyperlink{text00009.htmlux5cux23ch7-8}{\textsuperscript{{[}7{]}}}
。最近,针对失血性休克、全身性感染、蛛网膜下腔出血的患者,尤其慢性肾衰接受肾脏替代治疗患者的研究,进一步显示出腔静脉变异度的临床意义,但依然没有统一的预测值,仍需扩大研究规模。

2)自主呼吸或存在心律失常时容量反应性评估:对于存在自主呼吸或心律失常患者容量反应性的评估,可选择应用被动腿抬高试验相关的超声指标,相当于内源性的容量负荷试验,被动腿抬高试验产生300~450ml血浆快速输入。有研究表明,可应用超声观察每搏输出量的替代指标如被动腿抬高试验前后左室射血流速和流速积分变化来预测容量反应性,并且已经证明其敏感性和特异度均优于收缩压力和心率等;而在具有心内膜自动描记功能的超声诊断仪时,可以用左室每搏射血面积在被动腿抬高试验前后变化情况来预测液体反应性。

除应用左室射血流速和流速积分变化来预测容量反应性,最新有研究发现对于全身性感染和重症胰腺炎患者,在被动腿抬高试验前后应用外周动脉如股动脉峰值流速的变化与每搏输出量、脉压变化都可以用来预测液体反应性,前后变化分别为8%、10%和9%,同时研究还发现用心率来代表被动腿抬高试验前后自主神经功能时,前后没有变化,使得临床可操作性明显增强,当然除了选择股动脉还可以考虑其他外周动脉,如桡动脉和肱动脉等。

最近的一项包括9个相关研究的被动腿抬高试验荟萃分析认为,被动腿抬高试验相关的心指数和每搏输出量变化优于脉搏压的变化来预测液体反应性,可喜的是,其中6个研究应用了超声技术,入选患者数居多,所以随着未来有关主动脉流速和外周动脉流速的研究的增加,或许会有不同结论产生。

当然,在完全机械通气时和任何心律情况下,无论此时能不能合理应用动态指标,也可选择应用被动腿抬高试验相关的超声指标。

3)选择容量负荷试验进行容量反应性评估:当以上的方法依然不能合理预测容量反应性时,最终在谨慎考虑输液限制情况下,还可以选择容量负荷试验。此时,可选择超声测量每搏输出量、心输出量和左心室舒张末期面积变化以及多普勒测量左室充盈压变化判断容量负荷试验。最近的研究表明,容量负荷试验前后应用外周动脉流速变化如股动脉流速变化同样可以预测容量反应性,应该说除需要承担液体过负荷风险外,在评估容量反应性上完全与被动腿抬高试验接近,甚至于更可靠些
\protect\hyperlink{text00009.htmlux5cux23ch8-8}{\textsuperscript{{[}8{]}}}
。

\subsubsection{超声容量反应性评估时的注意事项是什么?}

在评估容量反应性时,一定要认真考虑以下因素:①液体反应性的评估需要测量多个参数,因为没有任何一个指标是绝对和排他的,临床上应该结合具体临床情况联合应用,最终有助于准确评估容量反应性;②心脏超声获得的心肺相互作用评估容量反应性的动态指标不但有助于评估容量反应性,同时心脏超声易于发现非超声获得的动态指标的假阳性(尤其严重右心衰),但依然需要更多的研究来证明临床价值。

总之,心脏超声在评估前负荷及容量反应性方面可用、有效,且极具前景。在应用心脏超声时,无论评估的流程还是指标的选择均有一定科学内涵,应该在应用时进一步设计合理的临床研究来证实临床有效性,期待能够对死亡率和致残率以及并发症发生率产生深远的影响。

\subsection{左心室功能的超声心动图评估}

\subsubsection{左心室功能评估的要点是什么?}

心室收缩与舒张功能及其随时间变化的评价在重症患者中作用很大。由于超声心动图以二维图像来展示三维结构,所以在诊断或者治疗之前,每个结构至少要得到相互垂直的两个切面的图像。新出现的或者进一步恶化的室壁运动异常可能提示急性心肌缺血或者缺血所致损伤,而像重症感染等多种重症疾病所导致室壁运动异常并非心室局部的功能障碍,而是心室的整体功能异常,因此全心室收缩功能评估十分重要。

心室收缩功能同时依赖于心脏的前负荷和后负荷,所以必须在不同负荷状态下评估收缩功能才能确保得到真实结果。另外还要注意连续评估的重要性,不能仅仅依赖某一次评估的结果得出结论。压力容积关系是不依赖于容量状态的左心室心肌收缩力的评估方法。超声心动图中用来评估整个左心室收缩功能的定性和半定量测量指标有射血分数、缩短分数、面积变化分数、左心室功能评估的Simpson法、二尖瓣环运动、用二尖瓣反流束计算等容收缩压力增加速率、使用标准17-节段模型和应变率来评估局部室壁运动异常。最常用的方法是射血分数。

\subsubsection{左心室收缩功能定性评估的首要问题是什么?}

评价左心室的收缩功能时,首先要明确以下问题:心室充盈如何?心肌有足够的收缩力吗?在冠脉分布的范围内心肌收缩一致吗?

\subsubsection{如何运用左心室标准的17节段分法进行视觉评估左室功能?}

左心室功能评估的形式多种多样,如心脏MRI、超声心动图、核素扫描、血管造影等。为了能统一术语,美国心脏学会达成共识,将左心室分成17个不同的节段。沿心脏长轴左心室分为基底段、中段和心尖段,基底段和中段又各自进一步分为6个节段,尖段分为4个节段,再加上第17节段的心尖帽部。相应的冠脉分布为:左前降支提供心脏的前壁和前间壁前2/3的血供,左回旋支提供左心室侧壁的血供,右冠状动脉提供室间隔后1/3和左心室下壁的血供。室壁运动评分和指数可以用来进行半定量评估。左心室收缩力依赖心脏从基底部到心尖部的运动、室壁的厚度和左心室螺旋挤压和旋转运动。心室壁的切面厚度以及左心室局部心内膜运动幅度对心室壁运动的评估十分重要。室壁运动评分描述如下:

正常(>30%心内膜运动幅度,>50%室壁厚度);

轻度运动功能减退(10%~30%心内膜运动幅度,30%~50%室壁厚度);

严重运动功能减退(<20%心内膜运动幅度,<30%室壁厚度);

运动不能(心内膜运动幅度为零,<10%室壁厚度);

运动障碍(收缩期反常运动)。

室壁运动评分指数是指局部的室壁运动分数,是一种主观评估方法,分数之间没有真正意义的线性关系。缺乏血流灌注的心肌将表现为异常的室壁运动。只有多个切面的图像才能真正反映左心室受损情况和相应冠脉分布情况。仅仅是心内膜运动幅度的改变可能是心肌栓塞造成的,而室壁厚度改变是缺血的确切指征。经过多次室壁厚度的测量可以得出以下结论:沿长轴平面很难获得连续的室壁厚度数据;多角度多平面测量可以减小误差;确定边界、方位和角度值。

\subsubsection{什么是射血分数及测量方法?}

每搏输出量等于舒张末容积与收缩末容积之差。射血分数等于每搏输出量除以舒张末容积。可以在经胸壁超声心动图的左室长轴和短轴不同平面测量,但美国超声心动图学会建议使用修改后的Simpson法,计算两个平面的射血分数然后取平均值。该方法可通过经食管超声心动图的经中段食管切面、四腔切面、二腔切面进行计算。局限性在于测量时要求心内膜边界能清晰显示,而二尖瓣环的钙化通常会干扰心内膜边界的探查;在四腔切面中,因为超声束与心室侧壁平行,所以会出现侧壁信号丢失的情况;左心室内小梁形成也会干扰心内膜边界的探查。在这种情况下,使用造影剂能提高边界成像的清晰度。左心室尖部常因为透视原理而缩小。

\subsubsection{怎样进行左心室收缩功能的超声心动图定量评估?}

(1)心输出量的计算

心输出量=心率×每搏输出量

在重症医学科中,肺动脉导管可以用来测量心输出量。但目前的证据显示,肺动脉导管的使用并没有明显优势,所以超声心动图对心输出量的测定具有重要作用。左右心室的心输出量都可以通过超声心动图来测量。左心室心输出量测量的可重复性和准确性更高:

左心室流出道面积=左心室流出道半径\textsuperscript{2} ×3.14。

心率可以通过心电图测量,或者从一个速度-时间积分到另一个速度-时间积分进行推算。每搏输出量等于左心室流出道面积乘以左心室主动脉瓣收缩期射血速度-时间积分。当血液从左心室射进圆柱体形的主动脉,每搏输出量就可以通过圆柱体血液的高度来计算,而这个高度就是速度-时间积分。圆柱体形的底是左心室流出道,而流出道面积能够很容易进行计算。圆柱体的高,也就是速度-时间积分,是通过经胸壁超声心动图时的心尖五腔切面、经食管超声心动图(TEE)时经胃主动脉瓣切面或者经胃主动脉瓣长轴切面运用脉冲多普勒测量左心室流出道的血流得出。该参数的准确测定基于左心室流出道面积在收缩期恒定不变的基础之上。左心室流出道半径的测量误差将使面积计算的误差放大。为了使误差最小化,图像的灰度要减小,而左心室流出道要尽量大;另一个假设是通过左心室流出道的血流是层流。这个假设通过脉冲多普勒上的窄流速带和平滑的光谱信号来证实。将样本体积的液体流通过两个互相垂直的切面来解释液体流的中心流速和边缘流速相等,以此证实平均流速分布图的存在。需要强调的是,多普勒射束应该与血流平行或者<20°。多普勒信号记录的是与血流平行的拦截角,所以能准确测量血流速度。左心室流出道直径和脉冲多普勒应该在同一解剖位置进行测量以保持脉冲多普勒的空间与即时关系。选择某一个靠近动脉瓣的位置当作常规测量点可以减小误差。因为在不同心率下血流动力学有所不同,因此这些测量应该在同一时间点进行,当在不同时间点评估心输出量时,所有的测量都要重复进行。

(2)不同部位每搏输出量的测量 使用经食管超声心动图时,一般选择左心室流出道作为最主要的测量点,然后就是肺动脉和右心室流出道。经食管超声心动图测量每搏输出量时,可以选择在主动脉瓣瓣叶尖端或者升主动脉。升主动脉直径是从胸骨旁长轴切面测量的,从胸骨上切迹或者心尖部的经胸壁超声心动图切面测出。二尖瓣口每搏输出量也可以通过脉冲多普勒在二尖瓣瓣叶尖端测得。因为二尖瓣的复杂几何特征和大量的假设,一般不选择该处作为心输出量的常规测量点。在心脏的右侧,可以选择三尖瓣或者肺动脉来测量每搏输出量。右心室心输出量也可以测量得出。然而,大的肺动脉直径不是固定的,而是依赖于切面的不同而不同;另外,并非时时都能取到与右心室射出血流平行的多普勒图。

\subsubsection{左心室收缩功能的超声心动图半定量测量方法如何采用?}

(1)测量缩短分数 缩短分数是一种评价左心室整体收缩功能的一维测量方法。经左心室乳头肌短轴的M型超声能测量出该参数的值。M型超声的定格分析用来计算缩短分数。缩短分数=(左心室舒张期内径-左心室收缩期内径)/左心室舒张期内径×100(正常值>25%)。正常值在25%~45%之间。

缩短分数的测量是一种基本的粗糙的左心室整体收缩功能的评估方法,优点是快捷而且可重复性高,M型超声检查可以节约很多时间,而且心内膜边界显示非常清晰。在测量过程中需注意,如果局部心室壁存在异常运动,容易产生误差;一维平面的斜切可能导致长度测量的误差。因此,在这个半定量测量中加入另外维度的测量可以增加整体功能评估的准确性。

(2)测量面积变化分数 面积变化分数是测量左心室收缩功能的二维参数。测量的准确性依赖于获得足够清晰的心内膜边界,边界显示不清晰时进行描记是十分困难和耗时的。面积变化分数可以定量评估射血分数。面积变化分数=(左心室舒张末面积-左心室收缩末面积)/左心室舒张末面积×100%。正常值>50%~75%。

面积变化分数高度依赖后负荷,也一定程度依赖前负荷。其中,经胃乳头肌短轴切面计算的面积变化分数与放射性核素血管造影术测量有很好的相关性。

(3)等容收缩压力增加速率的测量 评价左心室功能指标在射血期很容易得到,但这些指标的负荷依赖性明显影响心室功能的客观和准确评估。等容收缩压力增加速率对心肌收缩能力的变化较为敏感,受前后负荷变化影响较小,对左室心肌收缩力的评估较为准确,可用来反映心肌收缩力的变化。测量方法如下:连续波超声多普勒测定二尖瓣反流的速度,测量从1m/秒增加到3m/秒所需时间。根据简化的伯努利方程(压力=4×速度\textsuperscript{2}
),等容收缩压力增加速率(dP/dt)可以表示为:dP/dt=32/Δt;即运用简化的伯努利方程,速度为3m/秒时,压力为4×3\textsuperscript{2}
=36mmHg;速度为1m/秒时,压力为4×1\textsuperscript{2}
=4mmHg,压力差为32mmHg。用压力差除以速度从1m/秒增加到3m/秒所需的时间Δt,等容收缩压力增加速率即可计算出来。正常值>1200mmHg/秒,小于1000mmHg/秒则为异常。左心室功能良好的状态下,该时间可以大大缩短。值得注意的是,测量该指数时患者必须存在二尖瓣反流。

(4)运用组织多普勒成像评估心室功能 组织多普勒成像是一种量化测量左心室整体和局部功能的手段。组织多普勒显示的二尖瓣环下行速度可以评估左心室的收缩功能。心肌组织速率一般在二尖瓣环的室间隔、侧壁、下壁、前壁、后壁和前间隔部位测量。从上述部位得到的二尖瓣环下行平均峰速度可以衍生出以下计算方程:左心室射血分数=8.2×二尖瓣环平均峰速+3%。

该方程可以评估心内膜边界显示欠佳患者的整体左心室功能,缺点在于不能鉴别真正的心肌运动与心肌被动牵拉运动或者心室的整体位移运动。这些参数能从节段性应变成像模式中获得。

(5)比较少用的左心室收缩功能半定量测量工具

1)压力-容积环 压力0容积环的Y轴代表压力,X轴代表容量,压力-容积环的斜率反映心肌收缩能力,不受心脏前后负荷的影响。左室收缩功能增强压力-容积环向左上移动,反之,收缩功能下降时移向右下。将心室不同前负荷所对应的不同环的收缩末压点相连,即可得到反映收缩末期压力-容积环的变化关系,也被称作弹量。该方法测定需要足够的时间,而且前负荷的改变易于影响患者病情的稳定,因此,不具有实用性,尤其不适用于重症患者。

2)室壁应力和左心室质量 室壁应力是指施加在单位心肌面积上的力,取决于心腔容积、压力和室壁厚度。室壁应力包括圆周、子午或径向三个方面。通常计算收缩末期的圆周及子午室壁应力。将心肌体积乘以特异的心肌密度即可计算出左室心肌质量。超声心动图可以通过评估左室流出道的收缩速度加速度以及心肌收缩的应变率得到收缩末弹量。心肌做功指数(Tei)是另一种心肌收缩功能的评估方法,通过等容收缩期与等容舒张期之和除以射血时间得到,然而心肌做功指数的临床实用性仍有争议。

(6)左心室收缩功能半定量测量新技术

1)运用组织多普勒、应变和应变率评估心功能 多普勒组织成像和斑点追踪成像是新近发明的测量局部心肌功能的重要方法。组织速度信号是一种低速信号,它通过除去室壁过滤,并使用低增益放大,使得心肌组织速率测定成为可能。放置在心肌特定部位获得的脉冲多普勒或定向的M超声都可以用来展示心肌组织速率。当室壁运动异常与标准评估相混淆时,可用组织多普勒来鉴别。多普勒组织成像的常见缺陷包括:只能测量与超声束平行的运动成分;不能鉴别心室平行的位移运动;不能鉴别被邻近组织牵拉的运动与正常收缩运动。应变和应变率可用来测量在超声扫描线上出现的变形。传感器定位十分敏感,比多普勒的角度依赖性更敏感。心肌峰速度、应变率以及应变能识别静息状态以及应激状态下的局部心肌功能异常。斑点追踪成像可避免角度依赖性,能得到更准确的组织速度、应变率和应变力,用于测量两个维度的变形。在静息、应激(应力)、局部缺血等状态下的局部功能是运用多普勒组织成像或者斑点追踪成像进行应变率和应变评估的指征,将其与三维斑点追踪成像技术相结合是评估左心室功能的有力工具。

2)有利于分辨心内膜边界评估心室功能的新技术 心内膜边界的清晰度在左心室功能评估中十分重要。处于不同状态时,如肥胖或者肺气肿的患者,心内膜边界不太清晰。超声心动图造影技术在这些患者中有重要作用。彩色室壁运动技术通过声学定量原理能够将组织和血液区分开来,自动勾勒出心内膜边界,能够动态定量分析左室功能。在有室壁瘤或者其他心室不对称等异常情况下,该方法的有效性需要进行校正。这种情况下,三维超声能够真实反映左心室功能。

\subsection{左心舒张功能评估}

\subsubsection{如何应用跨二尖瓣左心室充盈评估左心舒张功能?}

左心室的舒张功能与收缩功能同等重要,舒张功能正常可防止肺静脉淤血和心源性肺水肿。超声心动图检查可通过测定跨二尖瓣左心室充盈、肺静脉血流模式和二尖瓣环侧壁心肌速度来评估左心室舒张功能。

将脉冲多普勒取样窗放置在二尖瓣瓣叶尖端可以获得舒张早期最大流速E和心房收缩期最大流速A。正常左心室E峰一般大于A峰。左心室肥厚或老年患者,E/A比值<1,反映舒张功能受损。E峰加速度与左心房压力除以τ的比值成正比,其中τ是等容期左心室压力下降的指数时间常数。为了保证每搏输出量,在有进行性舒张功能障碍的患者中存在进行性左心房压力增高的代偿,以将受损的舒张形态逆转到假性正常化。当左心室功能严重受损,在很短的充盈时间内出现左房压的极度上升,表现为经典的减速时间减少和高E/A比值。这些参数都是随着前负荷的变化而改变,单凭这种评估方法不能鉴别舒张功能不全的所有形式,还可能造成一些病例的漏诊。一些特定方法像Valsalva试验等可以帮助鉴别假性正常化的形态和进行性左心室舒张功能障碍。

\subsubsection{如何应用肺静脉血流脉冲多普勒评估左心舒张功能?}

肺静脉血流脉冲多普勒是一种通过评估跨二尖瓣充盈来诊断心室舒张功能障碍的辅助手段。将脉冲多普勒放置在肺静脉入左心房开口的远心端,能得到收缩波S、舒张波D和心房波A。在心房收缩产生的心房逆转波大小和形态最有临床应用价值。跨二尖瓣时间与肺静脉A波时间的差值有助于预测左心室舒张末压。

\subsubsection{如何应用M型彩色多普勒测量血流加速度?}

舒张期通过二尖瓣血流的时空图与左心室舒张有关,而这个时空图就是血流加速度。将彩色多普勒取样窗放置在左心室流入道,再将M型取样线穿过此窗口即可获得血流加速度。将色彩基线调整至最大二尖瓣口流速的30%~40%,然后计算红蓝渐变斜率即可计算血流加速度。与跨二尖瓣口血流充盈评估相比,血流加速度一般不会出现假性正常化,当其<45cm/秒提示左室舒张功能障碍。该方法的主要局限性是可重复性不高。当跨二尖瓣血流充盈和肺静脉脉冲多普勒相结合在左室舒张功能不全的诊断中不明确时,多普勒组织成像在外侧二尖瓣环获得的E峰、A峰以及血流加速度等附加标准有助于鉴别舒张功能障碍的程度。

\subsubsection{如何进行左心室充盈压评估?}

肺动脉导管可以用来测量左心室充盈压。在没有任何远端梗阻情况下的肺小动脉嵌顿压近似于舒张末期左心室压力,在左心室顺应性正常的情况下,该压力可以间接反映左心室舒张末期容积,也就是左心室前负荷。而在高龄或者高血压患者中,左心室肥厚以及左心室顺应性降低较常见,导致舒张末期左心室压力与左心室舒张末期容积关系发生改变。此时,超声心动图检查有助于评估左心室舒张末期压力和舒张功能。常用的指标为左心室的被动跨二尖瓣充盈(E峰)和与之相对应的侧面二尖瓣环移位(E'峰)关系及比值,比值>15,提示左心室舒张末压>15mmHg;比值<8,提示左心室舒张末压<15mmHg。E'速度<5cm/秒则提示心室顺应性减低。

\subsubsection{如何对左心室容积进行半定量评估?}

通过压力测量来评估左心室容量状态是临床常用的方法。然而,对于部分特定的患者,特别是机械通气患者,压力与充盈容积的对应关系并不准确,因此,压力指标不能准确反映患者容量状态。而超声心动图中有很多方法评估左心室容积和压力,既可以单次使用,也可以重复应用以监测患者对补液的反应。因此,在临床的应用逐步得到推广。左心室具有对称性,有两个相对相等的短轴,而长轴从心底指向心尖。长轴方向心尖较圆钝,近心尖侧左心室为半椭圆形,而心底侧为圆柱形,所以在短轴切面呈圆形。因此,在测量和计算左心室容积时,可假设为M型超声或者二维切面时的形状。但使用这些参数来评估正常或者异常形状的左心室时仍需要谨慎分析。

左心室舒张末容积、左心室舒张末表面积、上腔静脉塌陷率、下腔静脉宽度、容量反应性等都可用来评估左心前负荷。低血容量的诊断指标包括舒张末直径<25mm、左心室腔收缩闭塞和左心室舒张末表面积<55cm\textsuperscript{2}
。在经食管超声心动图的经胃乳头肌短轴平面可以比较容易得出这些参数。存在基础心脏疾病或者左心室低顺应性的患者,左心室的压力容积关系都将改变,最适左心室舒张末表面积将比正常人的更大。这就突出了对于既定的左心室舒张末表面积与每搏输出量测量的匹配关系。相对于单次测量结果,连续测量左心室舒张末容积更加可靠,但非常耗时,同时在实践中很难实现。追踪容量状态变化能证实与左心室舒张末表面积测量的相关性,左心室舒张末表面积是通过追踪上述切面的左心室舒张末静态轮廓来计算的。此过程可以通过使用自动声学定量边界监测系统来简化。收缩末与舒张末的容积都应进行检测,随着时间的推移,还可以追踪容量状态的变化。收缩末心室腔闭塞或者叫“乳头肌亲吻征”是低血容量的征象,预测心室收缩末表面积减少的敏感性达100%,但特异性只有30%。

二尖瓣环(E')的组织多普勒成像与二尖瓣口E波血流模式相结合可以预测左心室顺应性和平均舒张压。E/E'比值<8表示心室顺应性良好,>15表示左心室平均充盈压高,顺应性低。中间值的评估还需要结合其他参数,比如肺静脉血液回流和二尖瓣流入减速时间。

\subsection{左心室功能评估的新技术}

\subsubsection{如何利用三维技术进行左心室功能评估?}

实时图像重建能获得左心室图像。当进行三维图像重建时,通过一个固定的传感器在3°或5°标准下可获得一系列的二维图像。平面的数量和二维图像的质量共同决定三维图像的质量。矩阵阵列传感器的发明使得多线图像同时用于重建一组超声数据。但对于左心室,需要将连续心动周期获得的数据组整合起来进行评估。

左心室容量和功能也能通过三维方法来计算。与MRI相比,该方法观察者之间的主观误差少,图像重建的假设成分少,因而能够更加准确评估左心室的前负荷和射血分数。随着图像分析时间的进一步减少以及更多先进科技的出现,三维未来将成为评估重症患者左室容积和功能的最好方法,但该方法也有一定的局限性:三维容积中的线条密度比二维图像低,所以经常需要填描;当图像是从垂直于很多器官的固定传感器得到的,那么结果会是质量欠佳的图像。另外,随着呼吸运动和心律失常会出现结果的伪像。

\subsection{右心功能的评估}

\subsubsection{如何评估右室收缩功能?}

因为右室缺乏特殊的形态,心脏超声很难定量评估右室功能。因此,在正常和疾病状态下,通常仅能对右室形态大小与功能进行定性评估。判断右室扩张程度、室间隔左向偏移及运动情况是定性评估右室功能常用的基本方法。近年来,有研究逐步探索定量评估右心大小及功能的指标和方法,这些指标包括面积变化分数、三尖瓣环位移、组织多普勒三尖瓣环心肌收缩速度和心肌做功指数。最近三维超声技术的发展将进一步有助于临床准确评估右室大小及功能。其他的复杂技术如应变与应变率等目前仅在有经验的实验室作为特殊临床或试验研究应用,尚未应用于临床。

\subsubsection{如何评估右室舒张功能?}

对于右室功能障碍的患者,应测定右室舒张功能。三尖瓣E/A比、E/E'比及右房大小,已被证明均是有效的指标。右室舒张功能的分级如下:三尖瓣E/A比<0.8,提示松弛不良;三尖瓣E/A比0.8~2.1、同时E/E'比>6或肝静脉舒张期流量显著,提示假性充盈;三尖瓣E/A比>2.1、结合减速时间<120毫秒提示限制性充盈。进一步的研究需要针对上述指标的敏感性及特异性进行探讨,并研究分级与患者预后间的关系。

\subsection{超声在感染性休克循环支持中的作用}

\subsubsection{感染性休克的血流动力学特点是什么?}

感染性休克是重症患者转入重症医学科的常见原因之一。感染性休克的分子病理生理学机制复杂,以外周血管阻力降低、有效循环血量减少和组织灌注不足为特征的血流动力学改变是其显著的临床特点,因此超声心动图在感染性休克患者的病情监测和床旁管理中逐步得到应用。感染性休克的病理生理学特点包括低血容量、左室收缩和舒张功能障碍、右室收缩功能障碍及外周血管麻痹。超声心动图使重症医学科医师能识别这些过程,监控其发展,并采取相应的治疗性干预。

\subsubsection{感染性休克的容量特点是什么?}

感染性休克患者的容量特点是有效循环血量不足。表现为绝对或相对低血容量。绝对低血容量是指总循环血量减少,常为感染性休克早期的表现,需要立即纠正,常见的原因包括:非显性丢失,如由于发热、出汗和过度通气经皮肤和呼吸道丢失所致;经胃肠道丢失,如腹泻和呕吐;经第三间隙丢失,如胰腺炎、烧伤、软组织损伤、血管渗漏、低胶体渗透压、腹水、胸水;液体摄入过少,如精神状态改变、身体虚弱、医院内液体复苏不足。

相对低血容量由血液在外周和中心腔室内异常分布所致。相对血容量不足在感染性休克中常见,并可在初步液体复苏后持续存在。这类患者总血容量可能正常,但血容量分布在中心腔室以外。血管扩张是由于外周血管收缩机制障碍和血管扩张机制的异常激活所致。

无论低血容量是绝对、相对还是混合性,导致的后果一致,均表现为组织氧供减少和组织缺氧。液体复苏通过增加静脉回流、前负荷、心输出量和动脉压(收缩压、平均压和脉压)来改善感染性休克的容量状态。识别并纠正低血容量状态是感染性休克治疗的一个重要目标。

\subsubsection{感染性休克时左室收缩功能障碍的特点是什么?}

感染性休克患者常出现心肌收缩障碍。实验和临床研究表明多种因素共同作用导致感染性休克产生心肌功能抑制,如心肌水肿、心肌细胞凋亡、细胞因子作用(尤其是白介素-1、白介素-6和肿瘤坏死因子-α)以及一氧化氮激活。虽然无冠脉灌注和心肌能量代谢异常,但肌钙蛋白水平升高却很常见。

由于传统的左室收缩功能的超声心动图参数受左室前后负荷的影响,因此,超声心动图识别左室收缩功能障碍很难。如心室前负荷降低而血管扩张导致的低血压患者射血分数可以正常。在容量复苏和使用血管加压药物调整合适的前后负荷前提下,再进行超声心动图检查才能真正显示心室收缩功能的改变。而前后负荷的进一步变化又可以改变超声心动图的结果。因此,射血分数正常并不能排除左室功能障碍。临床和实验研究显示,感染性休克发生早期出现可逆的左室功能抑制,表现为左室压力容积曲线右移,射血分数下降,警示临床医生可能需要控制后负荷和给予强心治疗。感染性休克中左室收缩功能障碍的改善与生存率变化的关系仍存在争议。Parker等的研究首先显示两者具有相关性;但Vieillard-Baron等进行的研究没有得出类似的结果。有假说认为感染性休克左室扩张与收缩功能受到抑制有关,是心脏为维持心输出量而做的适应性改变,该假说已被部分超声心动图检查所证实。

\subsubsection{感染性休克时左室舒张功能障碍特点是什么?}

感染性休克常伴有左室舒张功能障碍,并与死亡率增加相关。这主要与肌钙蛋白水平升高、细胞因子活性(肿瘤坏死因子-α、白介素-8、白介素-10)增加有关。舒张功能障碍常与收缩功能障碍同时发生,但约20%的病例单独出现。

\subsubsection{超声心动图在感染性休克管理中的应用特点是什么?}

有效循环血量降低在感染性休克患者中很常见,而早期足够的容量复苏与患者的预后显著相关。因此,临床治疗中不能因等待超声心动图检查而延迟液体复苏。入院前和急诊的临床评估有助于获得初步的信息来决定容量复苏的补液量。入住重症医学科后需要关注的问题是患者是否还需要进一步进行容量复苏、是否需要继续调整血管活性药物的使用。在这种情况下,超声心动图是评估容量状态和心功能的理想工具,有助于识别低血容量、评价左室收缩期和舒张期功能障碍和右室功能障碍。最初的评估结果有助于制定治疗计划,而后续治疗过程中的监测有助于评估治疗效果、疾病变迁并识别新问题的出现
\protect\hyperlink{text00009.htmlux5cux23ch9-8}{\textsuperscript{{[}9{]}}}
。

\subsubsection{超声心动图如何评估感染性休克患者的容量反应性?}

对感染性休克患者进行容量复苏是初始复苏的重要部分,但容量复苏过度则导致相反的后果。利用床旁超声心动图检查可评估容量状态和容量反应性,常选用动态容量指标来进行评价。

下腔静脉直径的呼吸变异是判断容量反应性的有效方法,但要求患者必须有机械通气支持并完全没有自主呼吸。此外,超声心动图显示感染性休克患者小的高动力左室(收缩末左室腔消失)或小的下腔静脉直径(<10mm)提示患者存在容量反应性。

具有高级重症超声心动图检测能力的重症医学医师能通过多种多普勒方法来了解感染性休克患者是否需要进一步容量复苏。对于无自主呼吸、窦性心律的机械通气患者,可用经食管超声心动图测得的上腔静脉直径的呼吸变异测定容量反应性,也可通过多普勒测得的每搏输出量的呼吸变异进行判断。对于有自主呼吸和心律不齐的患者,可采用被动抬腿前后用多普勒测量每搏输出量和心输出量判断容量反应性。

\subsubsection{超声心动图如何评估感染性休克患者的左室收缩功能?}

感染性休克早期,常出现左室收缩功能受损,且通常在感染性休克恢复后7~10天完全恢复。感染性休克患者血流动力学改变呈“高动力”状态的高排低阻表现。对心脏功能非容量依赖性指数的研究显示,即使心输出量和射血分数正常或升高,但患者仍表现为收缩功能损害。超声心动图检查结果易于将高动力的左室收缩误读为左室充盈不足和后负荷过低。进行容量复苏和血管活性药物治疗调整后负荷后,超声心动图检查可以确切显示左室收缩功能受损
\protect\hyperlink{text00009.htmlux5cux23ch10-8}{\textsuperscript{{[}10{]}}}
。

左室收缩功能的评价依赖于射血分数的测定。超声心动图检查可以通过多种方法测定射血分数值。M型超声依赖于左室直径的测量。Teichholz方法测量的技术要求较高,要求在心室中央水平和胸骨旁长轴测量左室的直径,M型探头与左室壁垂直,重症患者往往心脏难以朝向适合测量的方向,加上由机械运动周期导致的平移运动伪影和用直径测量来定义复杂的三维结构所导致的内在的几何假设,M型射血分数测量方法可能不是测量重症患者射血分数的可靠方法。另外,该方法不能用于有室间隔异常的患者,机械通气的重症患者是否有效尚未得到证实。另一种方法是面积测量法。在胸骨旁短轴的乳头肌水平(使用经食管超声心动图)测量舒张末和收缩末左室腔的面积。尽管在理论上该方法优于基于直径的测量方法,面积测量法仍然易受室间隔异常和平移运动伪影的影响。准确测定射血分数可以采用Simpson方法,通过2个直角平面的顶面观来测量左室舒张末和收缩末面积(顶面四腔和顶面二腔视图)。该方法测定费时、需要明确心内肌边界、较高的测量技术(如理想的轴线和避免平移运动伪影)以及高质量的设备。

射血分数测定有助于评价左室收缩功能,但不能反映每搏输出量和心输出量。低灌注高动力的左室可以表现为射血分数正常,而每搏输出量和心输出量可能不足。同样,扩张而收缩功能下降的左室射血分数虽低,每搏输出量和心输出量可能并不降低。因此,临床治疗中往往需要测量每搏输出量和心输出量,这需要使用多普勒进行测定。在经胸壁超声心动图心尖五腔切面或经食管超声心动图胃深部视图测量,多普勒探头的脉冲波置于左室流出道,超声波束与血流方向平行。主动脉收缩期血流速度曲线下面积与每搏输出量成正比。主动脉收缩期血流速度时间积分乘以左室流出道面积即得到每搏输出量和心输出量。射血分数反映左室收缩功能,而每搏输出量和心输出量反映氧输送。感染休克早期的检查可能显示射血分数显著下降。恢复期检查可显示左室功能完全正常,这为患者的临床管理提供了重要信息。如果没有再次检查,患者可能被视为有慢性左心衰,从而进行不恰当的长期治疗。

\subsubsection{超声心动图如何评估感染性休克的左室舒张功能?}

感染性休克患者常出现左室舒张功能异常。舒张功能的测定非常重要,有助于评估左室舒张压和左房压,评价左心室对容量的耐受性,以尽早采取有效的治疗手段防止左室舒张压升高导致肺动脉压升高和肺水肿。一旦发现左室舒张末期压力升高可以及时采取治疗性干预,如限制液体输注和利尿,以保证在改善组织灌注的同时降低肺水肿发生风险。

传统测量方法依赖于多普勒分析负荷依赖性的二尖瓣流入量,也可以通过非负荷依赖性方法测量二尖瓣环组织的纵向运动多普勒速度(E')。另外,多普勒超声心动图检查可通过多种方法评估肺动脉嵌顿压。采用多普勒脉冲在顶面四腔视图上测量跨二尖瓣舒张期流速,E/A>2与肺动脉嵌顿压力>18mmHg显著相关,其阳性预测值为100%;收缩期前向运动速度/收缩期和舒张期速度<45%提示肺动脉嵌顿压力>12mmHg,其阳性预测值为100%;肺静脉反向A波时间大于二尖瓣流入A波时间提示肺动脉嵌顿压力>15mmHg,阳性预测值为83%;二尖瓣环组织多普勒测量二尖瓣E波速度比E'(E/E')>9提示肺动脉嵌顿压力>15mmHg。

\subsubsection{超声心动图如何评估感染性休克的右室功能?}

感染病原菌、毒素、炎症介质、感染性休克并发症等同样可损害右室功能。急性肺损伤、缺氧性肺血管收缩和正压通气都可能增加右室后负荷而导致急性肺心病。超声心动图有助于识别急性肺心病,从而有利于采取措施降低右室后负荷,缓解右室扩张。

\subsubsection{如何利用超声心动图对血管外周阻力进行评估?}

心脏超声多普勒技术可以直接测量外周血管阻力,但不易方便和简单使用,因此在临床工作当中,较少应用超声心动图检查评价外周血管阻力,而常根据临床和心脏超声的检查结果进行除外诊断,如在心脏前负荷充足的同时左右心脏收缩功能均满意的情况下仍然存在低血压,提示外周血管阻力降低。

\subsubsection{超声心动图在感染性休克管理中的临床应用流程是什么?}

低血容量、有效循环血量降低导致组织灌注不足是感染性休克的最主要特点。除立即使用有效抗生素抗感染治疗,早期的治疗应给予足量的容量复苏合并使用血管活性药物以改善组织灌注。该治疗常在重症医学科外已开始执行。患者转入重症医学科后,医师需要进行评估并进一步制定治疗计划。初步超声心动图检查首先有助于排除其他或并存的导致休克的原因,如早期未发现的心包填塞、严重瓣膜疾病、室间隔异常、缺血性心肌病或肺栓塞;其次有助于进行血流动力学评估,以指导进一步容量管理和血管活性药物的调整。

超声心动图检查显示以下特征性的改变时,提示需要继续进行容量复苏:①显示下腔静脉直径小或高动力的左室、收缩末室腔消失;②没有自主呼吸的机械通气患者,下腔静脉直径或每搏输出量随呼吸发生显著的变异;③有自主呼吸的机械通气患者,测量的被动腿抬高试验变异度>12%。

超声心动图检查有助于评价左心功能,指导血管活性药物的使用和调整。感染性休克患者常合并左室收缩功能下降,但并不说明患者一定需要使用血管活性药物。通过超声心动图检查有助于判断患者是否需要应用正性肌力药物。最常用的方法为直接测量每搏输出量和心输出量。超声检查即使显示左室收缩功能降低,但如果每搏输出量和心输出量在正常范围,没有必要使用强心治疗;如果每搏输出量和心输出量降低以至氧供减少,则有使用正性肌力药物的指征。如果无法进行量化的每搏输出量和心输出量测量,需要综合临床表现来决定是否使用正性肌力药物。

超声心电图检查有助于识别患者有无急性肺心病。多种因素可导致感染性休克患者出现急性肺心病。如细菌毒素、炎症介质、不恰当的机械通气治疗等。右室扩张和室间隔运动障碍,对急性肺心病有重要诊断意义。急性肺心病的识别有助于临床医师及时采取有效措施降低右室后负荷。

\subsection{超声心动图与重症相关心肌梗死}

\subsubsection{超声如何早期发现重症相关心肌梗死?}

无论是围手术期还是严重创伤的重症患者,缺血性心脏病常见,心肌局部缺血导致局部心肌运动异常。临床实际中,超声检查评估局部心肌缺血最常用的方法是进行二维超声显像检查,目测室壁运动和室壁增厚率。与心肌节段的室壁增厚率相比,二维超声应变成像对心肌缺血的变化更加敏感。

心肌应变是指心肌各节段的变形,与心肌的收缩和舒张功能密切相关,因此超声检查心肌应变可用于评估心肌收缩和舒张功能。

随着彩色多普勒心脏超声在临床的广泛运用,使急性心肌梗死后心脏功能、包括左室舒张功能异常得到全面深入的认识,对临床治疗方案的制定也起到重要作用。急性心肌梗死后可出现左心室舒张功能异常,表现为二尖瓣血流频谱E峰峰值速度减低,A峰峰值速度增高,E/A比值<1,E峰减速时间延长,等容舒张时间延长,肺静脉血流频谱S/D峰值比值增加等。

\subsection{超声心动图与急性肺动脉栓塞}

\subsubsection{超声心动图如何早期发现急性肺动脉栓塞?}

急性肺血栓栓塞是临床上一种危重的心肺疾病,超声心动图检查对其病变程度、治疗效果及预后评估有重要作用,已经普遍应用于临床。超声检查急性肺血栓栓塞应心脏超声检查及下肢深静脉检查。心脏超声可以从直接征象及间接征象为诊断急性肺血栓栓塞提供重要辅助诊断依据,其中,直接征象包括肺动脉和左右肺动脉主干内血栓;右心内血栓伴右心扩大、肺动脉高压;血栓到达肺动脉以前,可被腔静脉入右房处的Eustachil瓣、三尖瓣或右心耳阻截,如果同时伴有右心室扩大或肺动脉高压,则可以直接诊断急性肺血栓栓塞。

心脏超声检测急性肺血栓栓塞的间接征象包括肺动脉高压及肺源性心脏病征象。具体表现在以下几方面:栓子栓塞肺动脉,受机械、神经反射和体液因素的综合影响,肺血管阻力升高,右心后负荷增大,导致右心系统扩大;右室壁运动幅度减低;室间隔与左室后壁运动不协调,在左室短轴切面,室间隔向左心室膨出,左心室呈“D”字形改变;由于右心扩大,导致三尖瓣瓣环扩大,可引起不同程度三尖瓣反流及肺动脉压力增高,频谱多普勒可以测得三尖瓣反流压差,并可据此估测肺动脉压力;此外,还可见多普勒改变、肺动脉血流流速曲线发生特征性改变,主要表现为加速、减速时间缩短及频谱形态发生改变,如果伴有肺动脉高压,则血流频谱表现为收缩早期突然加速,加速支陡直,峰值流速前移至收缩早期,而后提前减速,呈直角三角形改变,有时可于收缩晚期血流再次加速,出现第二个较低的峰。

心脏超声可通过上述直接征象来直接诊断急性肺血栓栓塞,但临床检查发现直接征象的概率往往较低,主要原因为:当肺栓塞栓子位于肺动脉外周血管时,往往难以检出;新鲜的血栓回声多较低,超声不易识别;而机化的血栓与血管壁融合,也不易区分。间接征象可以提示诊断,更重要的是对具有胸痛、呼吸困难、心悸、气短等症状的患者进行鉴别诊断,主要与冠心病、急性心肌梗死、主动脉夹层、心包积液等疾病鉴别。对于确诊的急性肺血栓栓塞患者,如超声探测到中度、重度右室功能障碍,则其近期及长期病死率明显升高,而不伴有右室负荷过重的患者,近期预后良好。可见,除辅助诊断外,心脏超声检查还能够根据右室功能状态进行疾病危险度分层及预后判断。由于心脏超声可以动态、无创、重复估测肺动脉压力,因此也是疗效判断、随访追踪的一种快速、简便的检查手段。

\subsection{肺部超声在循环监测与支持中的作用}

\subsubsection{常见的肺部超声征象包括哪些?}

最近几年来,随着肺部超声的进步与推广,超声检查成为肺部和胸腔疾病诊疗的重要手段。正常和疾病状态下肺部超声常见的特征性的表现有:①正常通气征象------胸膜线下平行排列的A线;②肺间质肺泡综合征------彗星尾征,根据B线的不同间隔分为B7线(B线间隔大约7mm,主要是肺小叶间隔增厚)和B3线(B线间隔3mm);③肺实变征------组织样征和碎片征,可见支气管气象;④胸腔积液征象------静态征象为四边形征,动态征象包括水母征和正弦波征;⑤气胸征象------平流征,超声诊断气胸的优势是快速、直接。

\subsubsection{如何认识肺部超声对血流动力学性肺水肿的评估作用?}

血流动力学性肺水肿患者通常需要进行肺水含量的评估。肺部超声检查获得的B线提示患者出现肺水肿,该表现往往出现在血气分析改变之前。另外,超声具有简单、无创、无放射性和实时性等特点,可以实时监测肺水肿的改变。例如,随着肺水肿的增加,由肺间质水肿发展为肺泡水肿,肺部超声检查的B线也相应发生变化
\protect\hyperlink{text00009.htmlux5cux23ch11-8}{\textsuperscript{{[}11{]}}}
\textsuperscript{,}
\protect\hyperlink{text00009.htmlux5cux23ch12-8}{\textsuperscript{{[}12{]}}}
。

\subsubsection{如何利用超声监测鉴别急性心源性(血流动力学性)肺水肿与急性呼吸窘迫综合征肺水肿?}

肺部超声监测导向诊断的难点在于鉴别急性心源性(血流动力学性)肺水肿和急性呼吸窘迫综合征肺水肿。最新有研究对比急性呼吸窘迫综合征与急性心源性(血流动力学性)肺水肿超声征象的不同。研究纳入7个征象:肺泡间质综合征、胸膜线异常征象、胸膜滑动征消失、存在未受损伤的区域、肺部实变、胸腔积液和肺搏动征。研究结果表明:由于两种疾病发病的病理生理机制不同,肺部超声表现也不同。心源性肺水肿时,超声肺彗星尾征的绝对数量与血管外肺水含量明显相关,甚至随着肺部含水量的增加从黑肺到黑白肺直至白肺发展;急性呼吸窘迫综合征时,早期CT能发现的所有特点包括肺部及胸腔改变均可由肺部超声检查发现,包括不均匀的含有未受损伤区域的肺部间质综合征、胸膜线异常改变及肺实变和胸腔积液等。可见肺部超声有助于床旁即时鉴别诊断急性呼吸窘迫综合征肺水肿与急性心源性(血流动力学性)肺水肿
\protect\hyperlink{text00009.htmlux5cux23ch13-8}{\textsuperscript{{[}13{]}}}
\textsuperscript{~}
\protect\hyperlink{text00009.htmlux5cux23ch15-8}{\textsuperscript{{[}15{]}}}
。

\subsubsection{肺部超声如何估测肺动脉嵌压?}

在循环支持的过程中,有研究表明,肺超的A-优势型表现提示肺动脉嵌压<13mmHg的可能性大,而在B-优势型时,提示肺动脉嵌压>18mmHg的可能性较大。

\subsection{重症肾脏超声在循环监测及休克支持中的作用}

\subsubsection{肾脏超声在休克循环监测中也具有重要作用吗?}

肾脏是休克时最容易受损或最早受损的器官之一,术后患者发生率达到1%,而在重症患者则达到35%,尤其感染性休克患者发生率在50%以上。因此肾功能的评估和急性肾损伤的早期诊断非常重要
\protect\hyperlink{text00009.htmlux5cux23ch16-8}{\textsuperscript{{[}16{]}}}
。重症肾脏超声能够床旁及时、无创监测肾脏大循环与微循环的改变,为休克循环监测提供新的诊断依据。

\subsubsection{在循环监测及休克支持中如何应用肾脏超声?}

近年来,应用超声多普勒技术监测肾脏阻力指数成为评估肾脏灌注的重要工具。过去的研究表明,肾脏阻力指数与疾病的进展明确相关,建议肾脏阻力指数用于监测肾脏移植后功能不全、尿路梗阻等。近年,由于超声监测肾脏阻力指数无创、简单、可重复性强,成为重症患者首选监测急性肾损伤发生发展的重要工具,尤其有益于调整休克的血流动力学策略。另外,由于超声造影技术的进展,使床旁定量实时监测大血管与微血管血流成为可能,尤其对于休克时肾脏灌注的变化,包括对于治疗干预的变化均有重要的监测价值。

重症超声是重症医学科中指导血流动力学监测和治疗的有效方法,它为重症医学提供了连续动态管理重症患者的重要床旁工具。

\begin{center}\rule{0.5\linewidth}{\linethickness}\end{center}

参考文献

\protect\hyperlink{text00009.htmlux5cux23ch1-8-back}{{[}1{]}} .Morris
C,Bennett S,Burn S,et al.Echocardiography in the intensive care
unit:current position,future directions.JICS,2010,11:90-97.

\protect\hyperlink{text00009.htmlux5cux23ch2-8-back}{{[}2{]}} .Danilo
T,Marcelo L,Tatiana M,et al.Echocardiography for hemodynamic
evaluation in the intensive care unit.Shock.2010,34S(1):59-62.

\protect\hyperlink{text00009.htmlux5cux23ch3-8-back}{{[}3{]}} .Price
S,Nicol E,Gibson DG,et al.Echocardiography in the critically
ill:current and potential roles.Intensive Care Med,2006,32:48-59.

\protect\hyperlink{text00009.htmlux5cux23ch4-8-back}{{[}4{]}} .Gerstle
J,Shahul S,Mahmood F.Echocardiographically derived parameters of
fluid responsiveness.Int Anesthesiol Clin.2010,48(1):37-44.

\protect\hyperlink{text00009.htmlux5cux23ch5-8-back}{{[}5{]}}
.Vieillard-Baron A,Caille V,Charron C,et al.The actual incidence of
global left ventricular hypokinesia in adult septic shock.Crit Care
Med,2008,36:1701-1706.

\protect\hyperlink{text00009.htmlux5cux23ch6-8-back}{{[}6{]}} .Price
S,Via G,Sloth E,et al.World Interactive Network Focused On Critical
UltraSound ECHO - ICU Group:Echocardiography practice training and
accreditation in the intensive care:document for the World Interactive
Network Focusedon Critical Ultrasound(WINFOCUS).Cardiovasc
Ultrasound,2008,6:49.

\protect\hyperlink{text00009.htmlux5cux23ch7-8-back}{{[}7{]}} .Vincent
Caille1,Jean-Bernard Amiel,Cyril Charron,et al.Echocardiography:a
help in the weaning process.Critical Care,2010,14:R120.

\protect\hyperlink{text00009.htmlux5cux23ch8-8-back}{{[}8{]}} .Salem
R,Vallee F,Rusca M,et al.Hemodynamic monitoring by echocardiography
in the ICU:the role of the new echo techniques.Current Opinionin
Critical Care,2008,14(5):561-568.

\protect\hyperlink{text00009.htmlux5cux23ch9-8-back}{{[}9{]}}
.王小亭,刘大为,张宏民,等.扩展的目标导向超声心动图方案对感染性休克患者的影响.中华医学杂志,2011,91(27):1879-1883.

\protect\hyperlink{text00009.htmlux5cux23ch10-8-back}{{[}10{]}}
.王小亭,刘大为.重视心脏多普勒超声在重症医学领域中的应用.中华内科杂志,2011,50(07).

\protect\hyperlink{text00009.htmlux5cux23ch11-8-back}{{[}11{]}}
.Bellani G,Mauri T,Pesenti A.Imaging in acute lung in jury and acute
respiratory distress syndrome.Curr Opin Crit
Care,2012,18(1):29-34.

\protect\hyperlink{text00009.htmlux5cux23ch12-8-back}{{[}12{]}} .Rajan
GR.Ultrasound lung comets:a clinically useful sign in acute
respiratory distress syndrome/acute lunginjury.Crit Care
Med,2007,35(12):2869-2870.

\protect\hyperlink{text00009.htmlux5cux23ch13-8-back}{{[}13{]}}
.Jambrik Z,Gargani L,Adamicza A,et al.B-lines quantify the lung
water content:a lung ultrasound versus lung gravimetry study in acute
lung injury.Ultrasound Med Biol,2010,36(12):2004-2010.

{[}14{]}.Copetti R,Soldati G,Copetti P.Chest sonography:a useful
tool to differentiate acute cardiogenic pulmonary edema from acute
respiratory distress syndrome.Cardiovasc
Ultrasound,2008,29(6):16.

\protect\hyperlink{text00009.htmlux5cux23ch15-8-back}{{[}15{]}}
.王小亭,刘大为.超声监测导向的ARDS诊断与治疗.重症医学年鉴,2012.

\protect\hyperlink{text00009.htmlux5cux23ch16-8-back}{{[}16{]}} .Le
Dorze M,Bouglé A,Deruddre S,et al.Renal Doppler Ultrasound:A New
Tool to Assess Renal Perfusion in Critical
Illness.Shock,2012,37(4):360-365.

\protect\hypertarget{text00010.html}{}{}


\chapter{药物相互作用}

\section{概述}

药物相互作用(Drug-Drug
Interaction,DDI)是指同时或相继使用两种或两种以上药物时,由于药物之间的相互影响而导致其中一种或几种药物作用的强弱、持续时间甚至性质发生不同程度改变的现象。

药物相互作用有广义和狭义之分。广义药物相互作用是指联合用药时所发生的疗效变化。疗效变化虽然有多种多样表现,但结果只有两种可能,即作用加强或作用减弱。从临床角度考虑,作用加强可表现为疗效提高,也可表现为毒性加大;作用减弱可表现为疗效降低,也可表现为毒性减轻。虽然多药联用的情况非常普遍,但药物相互作用常常只在对患者造成有害影响时才引起充分注意。狭义的药物相互作用通常是指两种或两种以上药物同时或相继使用时产生的不良影响,可以是药效降低甚至治疗失败,也可以是毒性增加,这种不良影响是单一药物应用时所没有的。

一个典型的药物相互作用对(interaction
pair)由两个药物组成:药效发生变化的药物称为目标药(object drug或index
drug),引起这种变化的药物称为相互作用药或促发药(interacting
drug或precipitating
drug)。一个药物可以在某一相互作用对中是目标药(如苯妥英钠-西咪替丁),而在另一相互作用对中是相互作用药(如多西环素-苯妥英钠)。

\subsection{按发生机制分类}

\subsubsection{体外药物相互作用}

体外药物相互作用是指在患者用药之前(即药物尚未进入机体以前),药物相互间发生化学或物理性相互作用,使药性发生变化。即一般所称化学配伍禁忌或物理配伍禁忌,故又称之为物理化学性相互作用。

\subsubsection{药动学相互作用}

药物在其吸收、分布、代谢和排泄过程的任一环节发生相互作用,均可影响药物在血浆或其作用靶位的浓度,最终使其药效或不良反应发生相应改变。

\subsubsection{药效学相互作用}

两种或两种以上的药物作用于同一受体或不同受体,产生疗效的协同、相加或拮抗作用,而对药物的血浆或作用靶位的浓度可无明显影响。

应当注意的是,有时药物相互作用的产生可以是几种机制并存。

\subsection{按严重程度分类}

\subsubsection{轻度药物相互作用}

造成的影响临床意义不大,无须改变治疗方案。如对乙酰氨基酚能减弱呋塞米的利尿作用,但并不会显著影响临床疗效,也无须改变剂量。

\subsubsection{中度药物相互作用}

药物联用虽会造成确切的不良后果,但临床上仍会在密切观察下使用。如异烟肼与利福平合用,利福平是肝药酶诱导剂,会促进异烟肼转化为具有肝毒性的代谢物乙酰异烟肼,而利福平本身也有肝功能损害作用,两者合用会增强肝毒性作用,但两药联用对结核杆菌有协同抗菌作用,所以这一联合用药对肝功能正常的结核病患者仍是首选用药方案之一,但在治疗过程中应定期检查肝功能。

\subsubsection{重度药物相互作用}

药物联用会造成严重的毒性反应,需要重新选择药物,或须改变用药剂量及给药方案。如特非那定与许多药物(大环内酯类、咪唑类、H{2}
受体阻断药、口服避孕药等)合用时代谢过程受阻,其原形对心脏毒性较大,可致患者室性心动过速而死亡。骨骼肌松弛药与氨基糖苷类抗生素庆大霉素等合用,可能增强及延长骨骼肌松弛作用,甚至引起呼吸肌麻痹。

此外,按药物相互作用发生的概率大小可分为:肯定、很可能、可能、可疑、不可能等几个等级。这主要是根据已发表的临床研究或体外研究、病例报告、临床前研究等文献结果进行判断。按发生的时间过程,有的药物相互作用可立即发生,如四环素类抗生素与含钙、铝、镁的抗酸药发生络合反应,可使四环素的吸收立即下降。另一些药物相互作用的影响可能需要数小时或几天后才表现出来,如华法林的抗凝作用可被合用的维生素K逐渐减弱。

\section{体外药物相互作用}

体外药物相互作用是指在患者用药之前(即药物尚未进入机体以前),药物相互间发生化学或物理性相互作用,使药性发生变化。即一般所称化学配伍禁忌或物理配伍禁忌。

\subsection{分类}

\subsubsection{可见配伍变化}

包括溶液混浊、产气、沉淀、结晶及变色。可见配伍变化,应在混合后仔细观察,大多数是可以避免的。有些可见配伍变化不是立即发生的,而是在使用过程中逐渐出现的,更应该引起足够重视。如20%磺胺嘧啶钠注射液(pH值为9.5~11)加入10%的葡萄糖注射液(pH值为3.2~5.5)中,由于pH值的改变,可使磺胺嘧啶微结晶析出,这种结晶输入血管可造成栓塞。

\subsubsection{不可见配伍变化}

包括水解反应、效价下降、聚合变化及肉眼不能直接观察到的直径50μm以下的微粒等,潜在的影响药物对人体的安全性和有效性。如在氨基酸注射液中不能加入对酸不稳定的药物,因为该类药物在氨基酸营养液中容易降解;维生素C(pH值为5.8~6.9)与偏碱性的氨茶碱(pH值为9.0~9.5)溶液混合时,外观无变化,但效价降低。

\subsection{常见注射剂配伍变化产生的原因}

\subsubsection{沉淀}
\paragraph{注射液溶媒组成改变}

因改变溶媒的性质而析出沉淀。某些注射剂内含非水溶剂,目的是使药物溶解或制剂稳定,若把这类药物加入水溶液中,由于溶媒性质的改变而析出药物产生沉淀。如氯霉素注射液(含乙醇、甘油等)加入5%葡萄糖注射液或0.9%氯化钠注射液中,可析出氯霉素沉淀。
\paragraph{电解质的盐析作用}

主要是对亲水胶体或蛋白质药物自液体中被脱水或因电解质的影响而凝集析出。如氟罗沙星注射剂与0.9%氯化钠注射液合用可发生盐析作用而出现沉淀。
\paragraph{pH值改变}

pH值发生改变时,药物的溶解性也会发生改变,会导致药物的析出。5%硫喷妥钠10mL加入5%葡萄糖注射液500mL中,由于溶液pH值下降导致产生沉淀。
\paragraph{形成配合物}

如米诺环素与\ce{Ca^2+} 、\ce{Mg^2+} 等金属离子形成难溶性配合物而析出沉淀。

\subsubsection{变色}

出现新的颜色,或原有颜色消失。酚类化合物、水杨酸及其衍生物以及含酚羟基的药物如肾上腺素与铁盐发生配合反应,或受空气氧化,都能产生有色物质。

\subsubsection{产气}

碳酸盐、碳酸氢盐与酸类药物配伍,铵盐与碱类药物配伍,均可产生气体。

\subsubsection{效价下降}

某些药物在水溶液中不稳定,易分解失效,与其他药物合用,可加速分解,致药物活性下降。如氨苄西林在含乳酸根的复方氯化钠注射液中,由于乳酸根可加速氨苄西林的水解,4h效价损失20%。

\subsubsection{聚合反应}

氨苄西林1%({w/v}
)的储备液在放置期间,会发生变色、溶液变黏稠、形成沉淀,这是由于形成聚合物所致。

\subsection{注射剂配伍变化的预测}

根据注射药物的理化性质,将预测符号分为7类。

AI类为水不溶性的酸性物质制成的盐,与pH值较低的注射液配伍时易产生沉淀。如青霉素类、头孢菌素类、苯妥英钠等。

BI类为水不溶性的碱性物质制成的盐,与pH值较高的注射液配伍时易产生沉淀。如红霉素乳糖酸盐、盐酸氯丙嗪、盐酸普鲁卡因等。

AS类为水溶性的酸性物质制成的盐,其本身不因pH值变化而析出沉淀。如维生素C、氨茶碱、葡萄糖酸钙、甲氨蝶呤(MTX)等。

BS类为水溶性的碱性物质制成的盐,其本身不因pH值变化而析出沉淀。如硫酸阿托品、硫酸多巴胺、硫酸庆大霉素、盐酸林可霉素等。

N类为水溶性无机盐或水溶性不成盐的有机物,其本身不因pH值变化而析出沉淀,但可导致AS、BI类药物产生沉淀。如氯化钾、葡萄糖、碳酸氢钠、氯化钠等。

C类为有机溶媒或增溶剂制成不溶性注射液(如氢化可的松),与水溶性注射剂配伍时,常由于溶解度改变而析出沉淀。如氯霉素、维生素K{1}
、地西泮等。

P类为水溶性的具有生理活性的蛋白质(如胰岛素),pH值变化、重金属盐、乙醇等均可影响其活性或使其产生沉淀。如抗利尿激素、透明质酸酶、催产素、肝素等。

\section{药动学方面的相互作用}

药物代谢动力学(pharmacokinetics,PK)简称药动学,是研究药物在体内变化规律的一门学科。药动学的研究内容主要包括:一是药物的体内过程,包括吸收、分布、代谢和排泄;二是药物在体内随时间变化的速率过程。前者主要描述药物在体内变化过程的一般特点;后者主要以数学公式定量地描述药物随时间改变的变化过程。

机体对药物的处理是药物与机体相互作用的一个重要组成部分,药动学过程包括药物在其吸收、分布、代谢和排泄过程的任一环节发生相互作用,均可影响药物在血浆或其作用靶位的浓度,最终使其药效或不良反应发生相应改变。

\subsection{影响药物吸收的相互作用}

药物由给药部位进入血液循环的过程称为吸收。除静脉注射和静脉滴注给药外,其他血管外给药途径都存在吸收过程。临床常用的血管外给药途径可分为消化道给药、注射给药、呼吸道给药及皮肤黏膜给药,口服是最常用的给药途径。药物在胃肠道吸收时相互影响的因素有如下几个方面。

\subsubsection{pH值的影响}

药物在胃肠道的吸收主要通过被动转运。药物的脂溶性愈大、非解离型比值越大,越易吸收。胃肠道的pH值可通过影响药物的溶解度和解离度,进而影响药物的吸收。如酸性药物在酸性环境以及碱性药物在碱性环境下解离度低,非解离型药物占大多数,因而药物脂溶性较高,较易透过生物膜被吸收;反之,酸性药物在碱性环境或碱性药物在酸性环境下解离度高,因而药物脂溶性低,扩散透过生物膜的能力差,吸收减少。药物与能改变胃肠道pH值的其他药物合用,其吸收将会受到影响。如水杨酸类药物在酸性环境下吸收较好,若同时服用抗酸药碳酸氢钠,将减少水杨酸类药物的吸收。

\subsubsection{配合作用与吸附作用的影响}

含有2、3价的阳离子(\ce{Ca^2+} 、\ce{Al^3+} 、\ce{Mg^2+}
等)能与四环素类抗生素、异烟肼、喹诺酮类抗菌药物等形成不溶性或难以吸收的配合物,从而影响药物吸收。如口服的四环素与金属离子(\ce{Ca^2+}
、\ce{Al^3+} 、\ce{Mg^2+} 等)配合,使其吸收减少。

阴离子交换树脂如考来烯胺、考来替泊,对酸性分子如阿司匹林、地高辛、华法林、环孢素、甲状腺素等有很强的亲和力,妨碍了这些药物的吸收。药用炭、白陶土等吸附剂也可使一些与其一同服用的药物吸收减少,如林可霉素与白陶土同服,其血药浓度只有单独服用时的10%。

这些药物相互作用可采用增加给药时间间隔的方法来避免。

\subsubsection{胃肠运动的影响}

大多数口服药物主要在小肠上部吸收,因此改变胃排空和肠蠕动速度的药物能影响目标药物到达小肠吸收部位的时间和在小肠滞留的时间,从而影响目标药物吸收程度和起效时间。

一般而言,胃肠蠕动加快,药物起效快,但在小肠滞留时间短,可能吸收不完全;胃肠蠕动减慢,药物起效慢,吸收可能完全。这在溶解度低和难吸收的药物中表现得比较明显。如地高辛片剂在肠道内溶解度较低,与促进胃肠蠕动的甲氧氯普胺等合用,地高辛的血药浓度可降低约30%,有可能导致治疗失败;而与抑制胃肠蠕动的溴丙胺太林合用,地高辛的血药浓度可提高30%左右,如不调整地高辛剂量,就可能中毒;而口服快速溶解的地高辛溶液或胶囊,则溴丙胺太林对其吸收的影响相对较小。但是,对那些在胃的酸性环境中会被灭活的药物如左旋多巴,抑制胃肠蠕动的药物可增加其在胃黏膜脱羧酶的作用下转化为多巴胺(DA),从而降低其口服生物利用度。

\subsubsection{肠吸收功能的影响}

抗肿瘤药物如环磷酰胺、长春碱以及对氨基水杨酸、新霉素等能破坏肠壁黏膜,引起吸收不良。如环磷酰胺可使合用的地高辛吸收减少,血药浓度降低,疗效下降。

\subsubsection{食物的影响}

一般情况下食物可减少药物的吸收。如利福平、异烟肼等可因进食而吸收缓慢,但对药物吸收总量未有影响。但某些脂溶性药物,如灰黄霉素与高脂肪的食物同服,可明显增加吸收量。

\subsubsection{肠道菌群的影响}

消化道的菌群主要位于大肠内,胃和小肠内数量极少。因此,主要在小肠内吸收的药物较少受到肠道菌群的影响。口服地高辛后,在部分患者的肠道中,地高辛能被肠道菌群大量代谢灭活,如同时服用红霉素等能抑制这些肠道菌群的抗生素,可使地高辛血浆浓度增加一倍。

部分药物结合物经胆汁分泌,在肠道细菌的作用下可水解为有活性的原药而重吸收,形成肠肝循环。抗菌药物通过抑制细菌可抑制这些药物的肠肝循环。如抗生素可抑制口服避孕药中炔雌醇的肠肝循环,导致循环血中雌激素水平下降。

\subsubsection{其他因素的影响}

消化液是某些药物重要的吸收条件。硝酸甘油片舌下含服,需要充分的唾液帮助其崩解和吸收,如同服抗胆碱药,则由于唾液分泌减少而使之降效。

某些药物合并用药可影响胃肠道黏膜内外酶和酶系统,从而影响药物的吸收。如秋水仙碱能抑制肠黏膜中多种酶系统(如蔗糖酶、麦芽糖酶、乳酸酶等),导致维生素B{12}
的吸收不良。

另外,口服以外的给药途径也有可能因相互作用而影响吸收。如应用局麻药时,常加入微量肾上腺素以收缩血管,延缓局麻药的吸收,达到延长局麻药作用时间、减少不良反应的效果。

\subsection{影响药物分布的相互作用}

药物吸收后,通过各种生理屏障经血液转运到组织器官的过程称为分布(distribution)。分布过程中的药物相互作用方式,可表现为相互竞争血浆蛋白结合部位,改变游离型药物的比例,或改变药物在某些组织的分布量,从而影响它们在靶部位的浓度。

\subsubsection{竞争血浆蛋白结合部位}

药物经吸收进入血液循环后,大部分药物或其代谢产物均不同程度地与血浆蛋白发生可逆性结合,称结合型药物;另一部分为游离型药物。

当药物合用时,它们可在蛋白结合部位发生竞争,结果是与蛋白亲和力较强的药物可将另一种亲和力较弱的药物从血浆蛋白结合部位上置换出来,使后一种药物的游离型增多。由于游离型的药物分子才能跨膜转运,产生生物活性,并能被分布、代谢与排泄,因此这种蛋白结合的置换可对被置换药物的药动学和药效学产生一定的影响。

通过体外试验很容易证明,许多药物间均存在这种蛋白结合的置换现象。因此,过去一度认为它是临床上许多药物相互作用的一个重要机制。但近年来,更严谨的研究得出结论:大多数置换性相互作用并不产生严重的临床后果,因为置换使游离型药物增多的同时,相应分布、消除的比例也增加,仅引起血药浓度的短暂波动。

保泰松与华法林的相互作用研究是对蛋白结合置换现象的临床意义进行重新认识的典型例子。保泰松可以增强华法林的抗凝作用而致出血不止。过去一直认为,保泰松将华法林从其血浆蛋白结合部位置换出来,游离型华法林浓度升高导致出血。并据此认为任何非甾体抗炎药(NSAID)均以这种方式增强华法林的抗凝作用。现在的研究认识到,华法林是R和S两种异构体的混合物,S异构体的活性较R强5倍;保泰松除了竞争置换出华法林外,还可抑制S-华法林的代谢(由CYP2C9/18催化)而促进R-华法林代谢(由CYP1A2、CYP3A4催化),这样表面上药物总的半衰期不变,但血浆中活性高的S-华法林的比例增大,因而抗凝作用增强。

药物在蛋白结合部位的置换反应能否产生明显的临床后果,取决于目标药的药理学特性,那些蛋白结合率高、分布容积小、半衰期长和安全范围小的药物被置换下来后,往往发生药物作用的显著增强而导致不良的临床后果。表\ref{tab4-1}列出了一些常见的通过血浆蛋白置换而发生药物相互作用的实例。

\begin{longtable}[]{@{}lll@{}}
    \caption{血浆蛋白置换引起的药物相互作用}
    \label{tab4-1}\\
    \toprule
目标药(被置换药物) & 相互作用药 & 临床后果\tabularnewline
\midrule
甲苯磺丁脲 & 水杨酸、保泰松、磺胺药 & 低血糖\tabularnewline
华法林 & 水杨酸、水合氯醛 & 出血倾向\tabularnewline
MTX & 水杨酸、呋塞米、磺胺药 & 粒细胞缺乏症\tabularnewline
硫喷妥钠 & 磺胺药 & 麻醉时间延长\tabularnewline
卡马西平、苯妥英钠 & 维拉帕米 & 两药毒性增强\tabularnewline
\bottomrule
\end{longtable}

\subsubsection{改变组织分布}
\paragraph{改变组织血流量}

某些作用于心血管系统的药物可通过改变组织血流而影响与其合用药物的组织分布。如去甲肾上腺素减少肝脏血流量,使得利多卡因在肝脏的分布量减少,导致代谢减慢、血药浓度增高;而异丙肾上腺素增加肝脏血流量,增加利多卡因在肝脏中的分布及代谢,使其血药浓度降低。
\paragraph{组织结合位点上的竞争置换}

与药物在血浆蛋白上的置换一样,类似的反应也可发生于组织结合位点上,而且置换下来的游离型药物可返回到血液中,使血药浓度升高。由于组织结合位点的容量一般都很大,通常对血药浓度影响不大,但有时也能产生有临床意义的药效变化。例如奎尼丁能将地高辛从骨骼肌的结合位点上置换下来,可使90%患者的地高辛血药浓度升高约1倍,两药合用时,地高辛用量应减少30%~50%。

\subsection{影响药物代谢的相互作用}

药物在体内发生化学结构的改变称为代谢,或称为生物转化。药物代谢的主要场所是肝脏,肝脏进行药物代谢主要依赖于微粒体中的多种酶系。药物经代谢后可转化为无活性物质;或使原来无药理活性的药物转变为有活性的代谢产物;或将活性药物转化为其他活性物质;或产生有毒物质。影响药物代谢的相互作用占药动学相互作用的40%,是一种具有重要临床意义的药动学相互作用。

\subsubsection{酶诱导}

某些药物能增加肝药酶的合成或提高肝药酶的活性,称之为酶诱导。酶诱导使目标药的代谢加快,一般是导致作用减弱或作用时间缩短。具有酶诱导作用的常见药物如表\ref{tab4-2}所示。如口服抗凝血药双香豆素期间加服苯巴比妥,后者使血中双香豆素的浓度下降,抗凝作用减弱,表现为凝血酶原时间缩短。因此,如果这两类药物合用,必须应用较大剂量才能维持其治疗效应。

\begin{longtable}{ccc}
    \caption{常见的酶诱导及相互作用}
    \label{tab4-2}\\
    \toprule
    药物种类 & 受影响药物 & 相互作用结果\tabularnewline
\midrule
巴比妥类 & 巴比妥类、洋地黄毒苷、类固醇激素& \multirow{4}{3cm}{血药浓度下降、药效减弱或不良反应减轻}\tabularnewline             
保泰松、苯妥英钠 & 口服降血糖药、氢化可的松、茶碱 & ~\tabularnewline
利福霉素 & 口服抗凝药、地高辛、普萘洛尔、美托洛尔等 & ~\tabularnewline
灰黄霉素 & 口服抗凝药 & ~\tabularnewline
\bottomrule
\end{longtable}





需要指出的是,酶诱导促使药物代谢增加,但不一定均导致药物疗效下降,因为有些药物的药效是由其活性代谢物引起的。如环磷酰胺在体外无活性,只有经肝药酶代谢活化生成磷酰胺氮芥,才能与DNA烷化发挥药理作用,抑制肿瘤细胞的生长增殖。另外,如果药物经代谢生成毒性代谢产物,与酶诱导剂合用就可能会导致不良反应增加。如异烟肼产生肝毒性代谢物乙酰异烟肼,若与利福平合用,后者的酶诱导作用将加重异烟肼的肝毒性。

\subsubsection{酶抑制}

一些药物能减少肝药酶的合成或者降低肝药酶的活性,称之为酶抑制。临床上因肝药酶的抑制而引起的药物相互作用较肝药酶诱导所引起的药物相互作用常见。肝药酶被抑制,将使另一药物的代谢减少,因而加强或延长其作用。具有酶抑制作用的常见药物如表\ref{tab4-3}所示。如氯霉素与双香豆素合用,明显加强双香豆素的抗凝血作用,这是由于氯霉素抑制肝药酶,使双香豆素的半衰期延长2~4倍。

\begin{longtable}{ccc}
    \caption{常见的酶抑制及相互作用}
    \label{tab4-3}\\
    \toprule
    药物种类 & 受影响药物 & 相互作用结果\tabularnewline
\midrule
西咪替丁、阿司匹林 & 苯二氮䓬类药物& \multirow{4}*{血药浓度上升、药效增强或出现毒性反应}\tabularnewline
氯霉素、异烟肼 & 苯妥英钠、口服降血糖药 & ~\tabularnewline
别嘌醇 & 口服抗凝药、AZA\footnote{AZA表示硫唑嘌呤(azathioprine)} & ~\tabularnewline
肾上腺皮质激素 & 三环类抗抑郁药、环磷酰胺 & ~\tabularnewline
\bottomrule
\end{longtable}

有些药物在体内通过各自的灭活酶而被代谢,若这些酶被抑制,将加强相应药物的作用。食物中的酪胺在吸收过程中被肠壁和肝脏的单胺氧化酶所灭活,因而不呈现作用。但在服用单胺氧化酶抑制剂期间,若食用酪胺含量高的食物如奶酪、红葡萄酒等,由于肠壁及肝脏的单胺氧化酶已被抑制,被吸收的酪胺不经破坏,大量到达去甲肾上腺素能神经末梢,引起末梢中的去甲肾上腺素大量释放出来,使动脉血压急剧升高,产生高血压危象,危及患者生命。

虽然酶抑制可导致相应目标药自机体的清除减慢,体内药物浓度升高,但酶抑制能否引起有临床意义的药物相互作用取决于多种因素。
\paragraph{目标药的毒性及治疗窗的大小}

药物相互作用能产生临床意义的药物通常其治疗窗很窄,即治疗剂量和中毒剂量之间的范围很小;或其剂量-反应曲线陡峭,药物浓度虽然只有轻微改变,但是其效果差异变化显著。如抗过敏药阿司咪唑具有心脏毒性,与酮康唑、红霉素等酶抑制剂合用时,由于代谢受阻血药浓度显著上升,可出现致死性的心脏毒性。而酮康唑抑制舍曲林的代谢则不会引起严重的心血管不良反应。
\paragraph{是否存在其他代谢途径}

如果目标药可由多种肝药酶催化代谢,当其中一种酶受到抑制时,药物可代偿性经由其他途径消除,药物代谢速率所受影响可不大。但对主要由某一种肝药酶代谢的药物,如果代谢酶受到抑制,则容易产生明显的药物浓度和效应的变化。
\paragraph{与能抑制多种肝药酶的药物合用}

有些药物能抑制多种肝药酶,在临床上容易发生与其他药物的相互作用。如H{2}
受体阻断剂西咪替丁,其结构中的咪唑环可与肝药酶中的血红素部分紧密结合,故能抑制多种肝药酶而影响许多药物在体内的代谢。目前已报道有70多种药物的肝清除率在与西咪替丁合用后,出现不同程度的下降。临床上当药物与西咪替丁合用时,应注意调整剂量,必要时可用雷尼替丁代替西咪替丁。

酶抑制引起的药物相互作用常常导致药物作用的增强及不良反应的发生,但也有例外。如奎尼丁是酶抑制剂,而可待因须经肝药酶代谢生成吗啡产生镇痛作用,两者合用可使可待因的镇痛作用明显减弱,药效降低。

\subsection{影响药物排泄的相互作用}

药物及其代谢产物经机体的排泄器官或分泌器官排出体外的过程称为排泄。大多数影响药物排泄的相互作用发生在肾脏。当一种药物改变肾小管液的pH值、干扰肾小管的主动转运过程或重吸收过程或影响到肾脏的血流量时,就能影响一些其他药物的排泄,尤其对以原形排出的药物影响较大。

\subsubsection{改变尿液pH值}

尿液的pH值通过影响解离型/非解离型药物的比例,改变进入肾小管内药物的重吸收。这主要是因为大多数药物为有机弱电解质,在酸性尿液中,弱酸性药物(pKa为3.0~7.5)大部分以非解离型存在,脂溶性高,易通过肾小管上皮细胞重吸收;而弱碱性药物(pKa为7.5~10)的情况相反,大部分以解离型存在,随尿液排出多。临床上可通过碱化尿液增加弱酸性药物的肾清除率,如苯巴比妥多以原形自肾脏排泄,当过量中毒时,可用碳酸氢钠碱化尿液,减少重吸收,促进苯巴比妥的排泄而解毒。同理,酸化尿液可促进碱性药物的排泄。

但在药物的相互作用中,尿液pH值改变的临床意义甚小,因为除小部分药物直接以原形排出,大多数药物经代谢失活后,最终从肾脏消除;同时能大幅度改变尿液pH值的药物在临床上也很少使用。

\subsubsection{干扰肾小管分泌}

肾小管分泌是一种主动转运过程,要通过肾小管的特殊转运载体,包括酸性药物载体和碱性药物载体。当两种酸性药物合用时(或两种碱性药物合用),可相互竞争酸性(或碱性)载体,竞争力弱的药物,经由肾小管分泌的量减少,肾脏排泄减慢,有可能增强其疗效或毒性。如痛风患者合用丙磺舒和吲哚美辛,两者竞争酸性载体,可使吲哚美辛的分泌减少,排泄减慢,不良反应发生率明显增加。

但是有些药物间的这种竞争可被用于产生有益的治疗目的。如丙磺舒和青霉素竞争肾小管上的酸性转运系统,可延缓青霉素的经肾排泄过程,使其发挥持久的治疗作用。

\subsubsection{改变肾脏血流量}

减少肾脏血流量的药物可妨碍药物的经肾排泄,但这种情况在临床上并不多见。肾脏的血流量部分受到肾组织中扩血管的前列腺素生成量的调控。有报道指出,如果这些前列腺素的合成被吲哚美辛等药物抑制,则锂的肾排泄量会降低,并伴有血清锂水平的升高。这提示合用锂盐和NSAIDs的患者,应密切监测血清锂水平。

\section{药效学方面的相互作用}

药效学方面的药物相互作用是指不同药物通过与疾病相关药物靶点的影响,使一种药物增强或减弱另一种药物的效应或不良反应的现象。相互作用结果可分为药物效应的相加、协同和拮抗。

\subsection{相加或协同作用}

相加作用(addition effect)或协同作用(synergistic
effect)是指作用于疾病相关靶点的两种药物合用的效果等于(相加)或大于(协同)单用效果之和。相加或协同作用是临床用药的主要目的。

\subsubsection{表现为药理作用的增强}

如磺胺甲噁唑(SMZ)和甲氧苄啶(TMP)通过双重阻断机制(SMZ抑制二氢叶酸合成酶,TMP抑制二氢叶酸还原酶),协同阻断敏感菌的四氢叶酸合成,抗菌活性是两药单独等量应用时的数倍至数十倍,甚至呈现杀菌作用,且抗菌谱扩大,并减少细菌耐药性的产生。常将SMZ与TMP按5∶1的比例制成复方磺胺甲噁唑(SMZco)用于临床。另外,临床上常用青霉素和庆大霉素联用抗感染、异烟肼和利福平联用抗结核,这些联用都表现为治疗效应的增强。

\subsubsection{表现为药理作用的相加}

如应用一般治疗剂量的巴比妥类药物或其他具有中枢神经系统抑制作用的药物时,饮用少量酒即可引起昏睡,因为乙醇具有非特异性中枢神经系统的抑制作用,致使药理作用的相加。

\subsubsection{表现为增加药物不良反应的风险}

如治疗帕金森病的抗胆碱药物,与具有抗胆碱作用的其他药物(如氯丙嗪、H{1}
受体阻断药、三环类抗抑郁药)合用时可产生性质协同的相互作用,常可出现过度的抗胆碱能效应,在老年患者甚至可能出现抗胆碱危象。口服广谱抗生素抑制肠道菌群后,可使维生素K合成减少,从而增加香豆素类抗凝药的活性,应适当减少抗凝药的剂量。临床常见的药物相加或协同作用如表\ref{tab4-4}所示。

\begin{longtable}{cc}
    \caption{临床常见的药物相加或协同作用}
    \label{tab4-4}\\
\toprule
\endhead
相互作用药物 & 药理效应\tabularnewline
\midrule
NSAIDs和华法林 & 增加出血的风险\tabularnewline
血管紧张素转换酶抑制剂和氨苯蝶啶 & 增加高血钾的风险\tabularnewline
维拉帕米和β受体拮抗剂 & 心动过缓和停搏\tabularnewline
氨基糖苷类和呋塞米 & 增加耳、肾毒性\tabularnewline
骨骼肌松弛药和氨基糖苷类 & 增加骨骼肌松弛作用\tabularnewline
乙醇与苯二氮䓬类 &
增强镇静作用\tabularnewline
MTX与复方磺胺甲噁唑 & 骨髓巨幼红细胞症\tabularnewline
\bottomrule
\end{longtable}

\subsection{拮抗作用}

拮抗作用是指两种或两种以上药物合用所产生的效应小于其中一种药物单用的效应。在临床上,通常要尽量避免药物治疗作用的相互拮抗。根据作用机制,可将药物的拮抗作用分为两类。

\subsubsection{竞争性拮抗}

两种药物在共同的作用部位或受体上产生了拮抗作用。本类相互拮抗作用可发挥治疗作用,如在治疗虹膜炎时,交替使用毛果芸香碱和阿托品,可防止虹膜粘连;也可产生药理性拮抗作用,在药物中毒时抢救患者的生命。
如用苯二氮䓬
类受体拮抗剂氟马西尼抢救苯二氮䓬
类过量中毒;用α-肾上腺素受体激动剂去甲肾上腺素对抗氯丙嗪过量引起的低血压。

\subsubsection{非竞争性拮抗}

作用物与拮抗物不是作用于同一受体或同一部位,也可出现拮抗作用。如较大剂量的氯丙嗪用于治疗精神分裂症时,因阻断黑质-纹状体通路的多巴胺受体,使中枢乙酰胆碱作用相对增强,可引起锥体外系反应,而苯海索具有中枢抗胆碱作用,可减轻锥体外系反应;氨茶碱可因兴奋中枢而引起失眠,常合用催眠药加以对抗;维生素B{6}
能增加外周多巴脱羧酶活性,加速左旋多巴在外周部位脱羧,减少左旋多巴进入中枢的量,降低左旋多巴的疗效,产生对抗左旋多巴的作用。


\chapter{消化系统}


\section{正常X线解剖}

一、正常X线表现

胃肠道疾病的检查主要应用透视、腹部X线平片以及钡剂造影,显示胃肠道的位置、轮廓、腔的大小、内腔及粘膜皱襞的情况,但对于胃肠道肿瘤的内部结构、胃肠壁的浸润程度和转移等尚有一定困难,还需与其他检查相结合。目前,对于胃肠道疾病的检查,首选当是钡剂造影的检查方法。

1.咽部 是胃肠道的开始部分,它是含气空腔。吞钡造影正位观察,上方正中为会厌,两旁充钡小囊状结构为会厌谷。会厌谷外下方较大的充钡空腔是梨状窝,近似菱形且两侧对称,梨状窝中间的透亮区为喉头,勿误为病变。正常情况下,一次吞咽动作即可将钡剂送入食管,吞钡时梨状窝暂时充满钡剂,但片刻即排入食管。

2.食管 是一个连接下咽部与胃的肌肉管道,起于第6颈椎水平与下咽部相连。食管入口与咽部连接处及膈的食管裂孔处各有一生理狭窄区,为上、下食管括约肌。

食管充盈像:食管吞钡充盈,轮廓光滑整齐,宽度可达2~3cm。正位观察位于中线偏左,胸上段更偏左,管壁柔软,伸缩自如。右前斜位是观察食管的常规位置,在其前缘可见三个压迹,从上至下为主动脉弓压迹、左主支气管压迹、左心房压迹。于主动脉弓压迹与左主支气管压迹之间,食管显示略膨出,注意不要误认为憩室。

食管粘膜像:少量充钡,粘膜皱襞表现为数条纵行、相互平行的纤细条纹状阴影。这些粘膜皱襞通过裂孔时聚拢,经贲门与胃小弯的粘膜皱襞相连续。

透视下观察,正常食管有两种蠕动。第一蠕动为原发性蠕动,系由下咽动作激发,使钡剂迅速下行,数秒钟达胃内。第二蠕动又称继发蠕动波,由食物团对食管壁的压力所引起,始于主动脉弓水平,向下推进。所谓第三蠕动波是食管环状肌的局限性不规则收缩运动,形成波浪状或锯齿状边缘,出现突然,消失迅速,多发于食管下段,常见于老年人和食管贲门失迟缓症者。

另外,当深吸气时膈肌下降,食管裂孔收缩,致使钡剂暂时停顿于膈上方,形成食管下端膈上一小段长4~5cm的一过性扩张,称之膈壶腹,呼气时消失,属于正常现象。

此外,贲门上方3~4cm长的一段食管,是从食管过渡到胃的区域,称之食管前庭段,具有特殊的神经支配和功能。此段是一高压区,有防止胃内容物反流的重要作用。现将原来所定的下食管括约肌与胃食管前庭段统称为下食管括约肌。它的左侧壁与胃底形成一个锐角切迹,称为食管胃角或贲门切迹。

3.胃 一般分为胃底、胃体、胃窦三部分及胃小弯和胃大弯。胃底为贲门水平线以上部分,立位时含气,称胃泡。贲门至胃角(胃体与胃窦小弯拐角处,也称胃角切迹)的一段称胃体。胃角至幽门管斜向右上方走行的一部分,称胃窦。幽门为长约5mm的短管,宽度随括约肌收缩而异,将胃与十二指肠相连。胃轮廓的右缘为胃小弯,左缘是胃大弯。胃的形状与体形、张力及神经系统的功能状态有关,一般可分为4种类型:牛角型(位置、张力均高,呈横位,上宽下窄,胃角不明显,形如牛角。多见于肥胖体形的人);钩型(位置、张力中等,胃角明显,胃的下极大致位于髂嵴水平,形如鱼钩)。瀑布型(胃底大呈囊袋状向后倾,胃泡大,胃体小,张力高。充钡时,钡剂先进入后倾的胃底,充满后再溢入胃体,犹如瀑布)。长钩型(又称为无力型胃,位置、张力均低,胃腔上窄下宽如水袋状,胃下极位于髂嵴水平以下。见于瘦长体形的人)。

胃的轮廓在胃小弯侧及胃窦大弯侧光滑整齐,胃体大弯侧呈锯齿状,系横、斜走行的粘膜皱襞所致。

胃的粘膜皱襞像,可见皱襞间沟内充以钡剂,呈致密的条纹状影。皱襞则显示为条状透亮影。胃小弯侧的皱襞平行整齐,一般可见3~5条。角切迹以后,一部分沿胃小弯走向胃窦,一部分呈扇形分布,斜向大弯。胃体大弯侧的粘膜皱襞为楔形、横行而呈不规则的锯齿状。胃底部粘膜皱襞排列不规则,相互交错呈网状。胃窦部的粘膜皱襞可为纵行、斜行及横行,收缩时为纵行,舒张时以横行为主,排列不规则。

胃的双对比造影显示粘膜皱襞的细微结构即胃小区、胃小沟。正常胃小区为1~3mm大小,呈圆形、椭圆形或多角形大小相似的小隆起,其由于钡剂残留在周围浅细的胃小沟而显示出,呈细网眼状。正常的胃小沟粗细一致,轮廓整齐,密度淡而均匀,宽约1mm以下。

胃的蠕动来源于肌层的波浪状收缩,由胃体上部开始,有节律地向幽门方向推进,波形逐渐加深,一般同时可见2~3个蠕动波。胃窦没有蠕动波,是整体向心性收缩,使胃窦呈一细管状,将钡剂排入十二指肠;之后,胃窦又整体舒张,恢复原来状态。但不是每次胃窦收缩都有钡剂排入十二指肠。胃的排空受胃的张力、蠕动、幽门功能和精神状态等影响,一般于服钡后2~4小时排空。

4.十二指肠 十二指肠全程呈C形,在描述时,可将十二指肠全程称为十二指肠曲。上与幽门连接,下与空肠连接,一般分为球部、降部、水平部和升部。球部呈锥形,两缘对称,尖部指向右后方,底部平整,球底两侧称为隐窝或穹隆,幽门开口于底部中央。球部轮廓光滑整齐,粘膜皱襞为纵行、彼此平行的条纹。降部以下粘膜皱襞的形态与空肠相似,呈羽毛状。球部的运动为整体性收缩,可一次将钡剂排入降部。降、升部的蠕动多呈波浪状向前推进。十二指肠正常时可有逆蠕动。

低张力造影时,十二指肠管径可增加一倍,粘膜皱襞呈横行排列的环状或呈龟背状花纹。降部的外侧缘形成光滑的曲线。内缘中部可见一肩状突起,称为岬部,为乳头所在处,其下的一段较平直。平直段内可见纵行的粘膜皱襞。十二指肠乳头易于显示,位于降部中段的内缘附近,呈圆形或椭圆形透明区,一般直径不超过1.5cm。

5.空肠和回肠 空肠和回肠之间没有明确的分界,但上段空肠与下段回肠的表现大不相同。空肠大部分位于左上中腹,多见于环状皱襞,蠕动活跃,常显示为羽毛状影像,如肠内钡剂少则表现为雪花状影像,回肠肠腔略小,皱襞少而浅,蠕动不活跃,常显示为充盈像,轮廓光滑。肠管内钡剂较少、收缩或加压时可显示粘膜皱襞影像,呈纵行或斜行。末端回肠自盆腔向右上行与盲肠相连。回盲瓣的上下缘呈唇状突起,在充钡的盲肠中形成透明影。小肠的蠕动是推进性运动,空肠蠕动迅速有力,回肠慢而弱。有时可见小肠的分节运动。服钡后2~6小时钡的先端可达盲肠,7~9小时小肠排空。

6.大肠 大肠分盲肠、升结肠、横结肠、降结肠、乙状结肠和直肠,绕行于腹腔四周。升、横结肠转弯处为肝曲,横、降结肠转弯处为脾曲。横结肠和乙状结肠的位置及长度变化较大,其余各段较固定。直肠居于骶骨前缘并与之紧密相连。大肠中直肠壶腹最宽,其次为盲肠,盲肠以下各肠管逐渐变小。但其长度和宽度随肠管充盈状态及张力有所不同。

大肠充钡后,X线主要特征为结肠袋,表现为对称的袋状突出。它们之间由半月襞形成不完全的间隔。结肠袋的数目、大小、深浅因人因时而异,横结肠以上较明显,降结肠以下逐渐变浅,至乙状结肠接近消失,直肠则没有结肠袋。

大肠粘膜皱襞为纵、横、斜三种方向交错结合状表现。盲肠、升结肠、横结肠皱襞密集,以斜行和横行为主,降结肠以下皱襞渐稀且以纵行为主。

大肠的蠕动主要是总体蠕动,右半结肠出现强烈的收缩,呈细条状,将钡剂迅速推向远侧。结肠的充盈和排空时间差异较大,一般服钡后6小时可达肝曲,12小时可达脾曲,24~48小时排空。

阑尾在服钡或钡灌肠时均可能显影,呈长条状,位于盲肠内下方。一般粗细均匀,边缘光滑,易推动。阑尾不显影、充盈不均匀或其中有粪石造成的充盈缺损,不一定是病理性的改变,阑尾排空时间与盲肠相同,但有时可延迟达72小时。

双对比造影时膨胀而充气肠腔的边缘为约1mm宽的光滑而连续线条状影,勾画出结肠的轮廓,结肠袋变浅,粘膜面可显示出与肠管横径平行的无数微细浅沟,称之为无名沟或无名线。它们既可平行又可交叉形成微细的网状结构,从而构成细长的纺锤形小区,与胃小区相似。小区大小为1mm×(3~4)mm。小沟与小区为结肠双对比造影能显示粘膜面的最小单位,为结肠病变早期诊断的基础。

另外,在结肠X线检查时,某些固定部位较经常见到有收缩狭窄区,称为生理性收缩环。狭窄段自数毫米至数厘米长,形态多有改变,粘膜皱襞无异常,一般易与器质性病变相鉴别。但在个别情况下,当形态较固定时,注意与器质性病变鉴别。

二、检查方法及其目的

1.透视和腹部X线平片 主要用于急腹症,如胃肠道穿孔、肠梗阻等。急腹症的X线检查应简单、迅速、准确,以尽量减轻患者痛苦。

(1)腹部仰卧前后位:照片应包括横膈至耻骨联合,为观察腹部解剖构造及病理变化最好的位置。

(2)腹部立位前后位:照片应包括横膈至耻骨联合,可观察:①是否存在液平面。②是否存在气腹。③腹腔内阴影是否随体位变化。④能更细致地观察肠管。⑤了解肠间隔是否增厚。

(3)侧卧位水平投照:方法:①患者采取左侧卧位,X线水平方向投照。照片应包括全腹部,要特别注意右胁腹部、右下胸部应摄于片中。②患者采取右侧卧位,X线水平方向投照。照片也应包括全腹部,但以左胁腹部及左下胸部为重点。此二位置可进一步验证其他位置之所见,对不能站立的患者也可采用此位置投照,以观察是否存在气腹及液平面等。

(4)腹部侧位:患者仰卧,床面为半立位(角度35°~40°),以剑突为中心,X线水平方向投照。此位置检查的主要目的是观察剑突下是否有游离气体存在,及肠腔内是否存在液平面。

(5)后前立位胸片:要求曝光时间短(1/20~1/50秒)。照片目的:①了解是否存在引起急腹症的胸部病变(如下叶肺炎、食管下端穿孔及膈疝等)。②某些腹部疾病可并发异常的胸部X线表现。例如老年人,由于肠系膜血管病变引起的急腹症,其胸部X线检查可发现心脏疾患的证据(如心脏扩大、不正常的房室外形、心力衰竭等),有助于诊断和治疗。③还可查出与急腹症无关的其他疾病,而对手术及术后处理有重要意义。④膈下是否存在游离气体。总之,常规胸部X线检查是诊断急腹症不可缺少的重要步骤。

2.造影检查 消化道造影仍为胃肠道疾病的主要检查方法,造影检查有粘膜法、充盈法、加压法和气钡双重造影法等四种基本方法。粘膜法是用少量钡剂涂布于粘膜表面显示粘膜皱襞的方法,对于病变的早期诊断有重要价值,所摄片称粘膜像。充盈法,胃肠道某一器官或某器官的一部分有较多钡剂充盈,主要显示该部的轮廓,摄片称充盈像,病变的切线位时可见其轮廓异常,较大肿块可显示充盈缺损,较小的肿块可因钡剂掩盖而漏诊。加压法,加压使该部的钡剂减少变薄,有利于较小的隆起性病变的显示,摄片多为某器官的局部点片,称加压像。双重造影法是先后引入一定量的阳性造影剂硫酸钡悬混液和阴性造影剂气体,以显示胃肠道的细微结构,其照片称双重造影像。气体为最常用的阴性造影剂,故又称气钡双重造影,已广泛地用于胃肠道各部位。双重造影分为低张和非低张双重造影,以低张双重造影显示最佳。双重造影技术与纤维内镜的配合已使胃肠道疾病的早期诊断有了突破性进展。

消化道造影检查根据检查部位的不同分成食管造影、上消化道造影、小肠系造影和钡剂灌肠造影。需要指出的是当怀疑消化道穿孔和肠梗阻时,禁用钡餐造影而改用口服有机碘溶液。

\textbf{【X线表现】}
 上方充钡的小囊为会厌谷,下方圆形透亮区为喉头,勿误为占位引起的充盈缺损。喉头两侧为对称的梨状窝。两侧梨状窝汇入中央即为食管开口,即食管第一生理狭窄处(图5-1-1A)。右前斜位食管充盈像,显示食管吞钡充盈,轮廓光滑整齐,其前缘可见三个压迹,从上至下为主动脉弓压迹(为半月弧形,压迹深度随年龄递增)、左主支气管压迹(其与主动脉弓之间食管往往相对膨出为正常表现,不要误认为食管憩室)、左心房压迹(较长而浅,左心房增大,压迹可增宽,甚至食管局部后移)(图5-1-1B)。右前斜位食管粘膜像,管腔内显示2~5条纵行、相互平行的纤细条纹状阴影,即食管粘膜皱襞,其宽度不超过2mm(图5-1-1C)。左前斜位片如图5-1-1D。

\begin{figure}[!htbp]
 \centering
 \includegraphics{./images/Image00233.jpg}
 \captionsetup{justification=centering}
 \caption{食管钡餐造影片}
 \label{fig5-1-1}
  \end{figure} 

\textbf{【X线诊断】}  正常食管钡餐片。

\begin{figure}[!htbp]
 \centering
 \includegraphics{./images/Image00234.jpg}
 \captionsetup{justification=centering}
 \caption{食管第3蠕动波}
 \label{fig5-1-2}
  \end{figure} 

\textbf{【X线表现】}
 所谓第3蠕动波是食管环状肌的局限性不规则收缩运动,形成波浪状或锯齿状边缘,出现突然,消失迅速,多发于食管下段,常见老年人和食管贲门失迟缓症者。

\textbf{【X线诊断】}  食管贲门失迟缓症;食管第3蠕动波。

\begin{figure}[!htbp]
 \centering
 \includegraphics{./images/Image00235.jpg}
 \captionsetup{justification=centering}
 \caption{胃的X线解剖部位划分及命名}
 \label{fig5-1-3}
  \end{figure} 

(1)贲门:食管进入胃的开口处。

(2)胃底:贲门横线以上区域。

(3)贲门区:以贲门为中心,半径约为2.5cm的圆形区域。

(4)胃小弯:胃的右上侧边缘。

(5)胃大弯:胃的左外下侧边缘。

(6)胃角(角切迹):胃小弯转折处。

(7)胃窦:角切迹与胃大弯最低点连线与幽门之间的区域。

(8)胃体:胃窦与胃底之间的区域。

(9)幽门管:胃部通向十二指肠球部的细短管状结构。

\begin{figure}[!htbp]
 \centering
 \includegraphics{./images/Image00236.jpg}
 \captionsetup{justification=centering}
 \caption{胃钡餐造影片}
 \label{fig5-1-4}
  \end{figure} 

\textbf{【X线表现】}
 胃的轮廓在胃小弯侧及胃窦大弯侧光滑整齐,胃体大弯侧呈锯齿状,系横、斜走行的粘膜皱襞所致。

胃的粘膜皱襞像,可见皱襞间沟内充以钡剂,呈致密的条纹状影。皱襞则显示为条状透亮影。胃小弯侧的皱襞平行整齐,一般可见3~5条,平均宽约0.5cm。角切迹以后,一部分沿胃小弯走向胃窦,一部分呈扇形分布,斜向大弯。胃体大弯侧的粘膜皱襞为楔形、横行而呈不规则的锯齿状,宽0.2~0.4cm,大于0.5cm为异常表现。胃底部粘膜皱襞排列不规则,相互交错呈网状。胃窦部的粘膜皱襞可为纵行、斜行及横行,收缩时为纵行,舒张时以横行为主,排列不规则。

\textbf{【X线诊断】}  正常胃的粘膜皱襞。

\begin{figure}[!htbp]
 \centering
 \includegraphics{./images/Image00237.jpg}
 \captionsetup{justification=centering}
 \caption{胃双对比造影片}
 \label{fig5-1-5}
  \end{figure} 

\textbf{【X线表现】}
 胃的双对比造影显示粘膜皱襞的细微结构即胃小区、胃小沟。正常胃小区为1~3mm大小,呈圆形、椭圆形或多角形大小相似的小隆起,其由于钡剂残留在周围浅细的胃小沟而显示出,呈细网眼状。正常的胃小沟粗细一致,轮廓整齐,密度淡而均匀,宽约1mm以下。

\textbf{【X线诊断】}  正常胃小区。

\textbf{【临床经验】}
 应当强调,X线征象的显示情况与检查方法有密切的关系。近年来,由于开展了气钡双重造影,对于龛影形态及胃粘膜皱襞的显示提供了良好的条件。临床工作中,只有把充盈像、粘膜皱襞像及粘膜像结合起来,才能比较确实地反映出龛影的病理形态。在良、恶性溃疡鉴别诊断时,良性胃溃疡多数表现为龛周胃小沟纤细,胃小区多数显示不清,少数显示形态不规则。另有见龛周胃小沟粗细不均,胃小区显示清晰,但形态不规则,呈多样性改变。恶性胃溃疡龛周胃小沟、胃小区破坏,癌组织代替了正常粘膜层,呈多样性改变。如结节样、磨砂玻璃样以及条索状,部分病例在靠近正常粘膜区,胃小区尚可辨认,但胃小沟粗细不均、紊乱、破坏。所以我们认为,龛周胃小区改变呈萎缩型或增生型者为良性溃疡;龛周胃小区呈破坏型代之以结节状、磨砂玻璃状、不规则条状皱襞改变者为恶性溃疡。

\begin{figure}[!htbp]
 \centering
 \includegraphics{./images/Image00238.jpg}
 \captionsetup{justification=centering}
 \caption{上消化道钡餐造影片}
 \label{fig5-1-6}
  \end{figure} 

\textbf{【X线表现】}
 十二指肠全程称十二指肠曲,因其成半环形又称为十二指肠环。一般分为球部、降部、水平部和升部。球部:充盈时呈边缘整齐的三角形,尖部指向右上后方,底部平整,两侧有对称的隐窝,幽门开口于球底中央。球尖顶到降部之间的一小段,X线上称为球后部,其长短不一,一般可达4~5cm,短时几乎不存在。粘膜皱襞可呈纵行,有4~5条,也可呈横行或花纹状,在双重造影时,球部粘膜可呈细网状或小点状,为粘膜绒毛及绒毛间沟充钡所致。球部充盈不全时,其边缘可不规则,为粘膜皱襞所致,易误为异常。因球部及球后部向右后方,所以,右前斜位便于观其全貌,左前斜位便于球部前后壁的显示。降部、水平部、升部:充盈后内外缘对称,因粘膜皱襞的影响,两侧缘呈锯齿状,尤以外缘明显,粘膜皱襞呈环形或羽毛状,收缩时则成纵行。蠕动呈波浪状前进,并可见逆蠕动,不能误为异常。降部宽2~3cm。十二指肠双重造影时,管径可增加一倍,羽毛状粘膜皱襞消失,代之以环形或龟背状花纹,或二者兼有。降部内缘可较平直或略凸,中段可见一肩样突起,称为岬部,其下方较平直,可见纵行皱襞。十二指肠乳头在岬部下方,呈圆形或类圆形,边界清晰,直径一般不超过1.5cm。乳头开口处可存钡,表现为点状,为正常现象。在乳头影上方有时可见一直径数毫米的圆形透亮区,为副乳头。

\textbf{【X线诊断】}  十二指肠正常X线表现。

\begin{figure}[!htbp]
 \centering
 \includegraphics{./images/Image00239.jpg}
 \captionsetup{justification=centering}
 \caption{小肠钡餐造影片}
 \label{fig5-1-7}
  \end{figure} 

\textbf{【X线表现】}
 平片检查,正常成人的小肠内虽有气体,但与食糜混合存在,而不能显示。长期卧床、幼儿及肠紧张的老年人,小肠内有分散的气团,多见于腹中部,为正常表现。另外,患者由卧位改成立位检查时,十二指肠球部可有积气,不能误为异常。造影检查,小肠长度平均为280cm,其长度与体重关系明显,与身长关系不明。空回肠两端较固定,其余部分活动度较大。空肠居于左上腹及中腹部,回肠位于右下腹及盆腔。一般上部肠曲多横行,下部肠曲多纵行。空肠管径较大,为2.5~3cm,回肠管径1.5~2.5cm。空肠粘膜呈细羽毛状,其长短、粗细、形态和方向随肠壁肌张力而变化。收缩时呈纵行状,舒张时呈环形,粘膜面仅有少量钡餐附着时,则呈雪花状。回肠粘膜皱襞则稀疏、低平而不明显,其末端常呈纵行皱襞。在小儿,由于淋巴组织丰富,淋巴集结可呈卵石状,多见于回肠。小肠运动主要为蠕动,表现为节段性充盈与排空。空肠蠕动迅速有力,回肠慢而弱,但分节运动较明显,表现为节律性收缩与舒张。小肠的运动受胃内钡剂排出状况影响,胃蠕动强、排出量大时,小肠的运动也增强。常规口服钡餐造影时,钡剂到达回盲瓣的时间一般为2~6小时,7~9小时钡剂从小肠全部排空。老年人排空时间延缓,可达11小时。如果少于1小时钡剂到达盲肠,为运动增快,超过6小时则为运动过缓。为了便于X线检查的描述,按小肠位置将其分为六组:①十二指肠。②上部空肠,位于左上腹部。③下部空肠,位于左腹部。④上部回肠,位于右中腹部。⑤中间回肠,位于右中下腹部。⑥下部回肠,位于盆腔内。

\textbf{【X线诊断】}  小肠正常X线表现。

\begin{figure}[!htbp]
 \centering
 \includegraphics{./images/Image00240.jpg}
 \captionsetup{justification=centering}
 \caption{结肠钡剂造影片}
 \label{fig5-1-8}
  \end{figure} 

\begin{figure}[!htbp]
 \centering
 \includegraphics{./images/Image00241.jpg}
 \captionsetup{justification=centering}
 \caption{结肠气钡双重造影}
 \label{fig5-1-9}
  \end{figure} 

\textbf{【X线表现】}
 盲肠位于右髂窝内,移动度较大,故位置不固定,可高至肝下或低至盆腔,甚至到左下腹部,但一般移动范围在10cm左右。回盲瓣开口于盲肠后内侧壁,上唇较长约2cm。下唇约0.6cm,瓣口为圆形、椭圆形或呈横裂口。阑尾一般位于盲肠下内侧,钡剂造影显示率为60%,充盈时光滑整齐,活动度大,有时可见粪石形成的充盈缺损,阑尾多与盲肠同时排空或稍延缓。横结肠和乙状结肠的系膜较长,因此,活动范围较大,其余部分位置较固定。直肠壶腹部内径最大,盲肠次之、盲肠向远端逐渐变窄,乙状结肠与直肠移行处最窄,为2~3cm,勿误为病理表现。常规钡剂灌肠时,因生理括约肌的作用,在回盲瓣的对侧、升结肠、横结肠近端和远端、降结肠下部、乙状结肠等部位,可见肠腔局限性狭窄,不能误为异常。

结肠的粘膜皱襞有横、纵、斜三个方向相互交错。盲肠、升结肠及横结肠的粘膜皱襞较显著,降结肠及其远段则稀疏。环肌收缩时粘膜呈纵行皱襞。

直肠没有结肠袋,但直肠壶腹的前壁及侧壁可见半圆襞形成的切迹。直肠后壁与骶骨之间称骶骨前间隙或称直肠后间隙,测量方法是第3~5骶骨前缘到直肠后壁的最短距离,而以第5骶骨处测量较准确。约95%的正常人此间隙小于或等于0.5cm,大于1.5cm时可疑异常,大于2cm者为病理性增大。

双重造影时,结肠的轮廓呈连续、均匀的线条,粗约1mm。其微小皱襞称无名线,此乃结肠的基本解剖单位,切线位表现为微细的刺状突出,深约0.2mm。正面观为0.1~0.2mm,并以0.6~1mm的间距与肠壁垂直分布,或交织呈网状。良好的双重造影片上,无名线的显示率可达90%。在结肠排空像的边缘有时可见深0.5~2mm、粗1mm、以3~5mm间距分布的尖刺影,称边缘锯齿征,或称结肠假溃疡征,是钡剂嵌于结肠Lieberkuhns腺管腺窝所致,出现率为5%~10%。复查时可消失,为正常表现。

\textbf{【X线诊断】}  正常结肠造影片。

\begin{figure}[!htbp]
 \centering
 \includegraphics{./images/Image00242.jpg}
 \captionsetup{justification=centering}
 \caption{经内镜逆行胰胆管造影片}
 \label{fig5-1-10}
  \end{figure} 

\textbf{【X线表现】}
 胆囊大小、形态、位置因人的体质及体位不同而不同,一般分为梨形、圆形和长形三种,最常见为梨形,长7~10cm,宽3~4cm,形态上胆囊可分为底部、体部、漏斗部和颈部。胆管分肝内胆管和肝外胆管两部分,肝内胆管由左、右肝管及其分支组成,肝外胆管由肝总管、胆囊管和胆总管组成。肝总管长3~4cm,宽5~6mm;胆囊管长3~4cm,宽2~3mm;胆总管长7~8cm,宽5~6mm。胆总管穿过十二指肠壁,终止于十二指肠大乳头,构成肝胰壶腹(Oddi)括约肌,宽12mm,长约数毫米,在其上方略为膨大成为肝胰壶腹,胰管汇合于此。

\textbf{【X线诊断】}  胆道系统正常X线表现。

\begin{figure}[!htbp]
 \centering
 \includegraphics{./images/Image00243.jpg}
 \captionsetup{justification=centering}
 \caption{T管造影片}
 \label{fig5-1-11}
  \end{figure} 

\textbf{【X线表现】}
 胰腺管分为主胰管和副胰管,主胰管从十二指肠大乳头开始,多为从右下斜行向左上,或呈横行、乙字形走行于第12胸椎至第2腰椎水平之间。主胰管分为头部、体部和尾部,全长14~18cm;宽:头部4mm,体部3mm,尾部2mm。副胰管于主胰管的头、体交界处与主胰管汇合,大致呈水平走向于十二指肠壁开口于十二指肠小乳头。

\textbf{【X线诊断】}  肝内外胆管及胰腺管X线解剖。

\section{食管病变}

\subsection{食管金属异物}

\begin{figure}[!htbp]
 \centering
 \includegraphics{./images/Image00244.jpg}
 \captionsetup{justification=centering}
 \caption{颈胸部正侧位片}
 \label{fig5-2-1}
  \end{figure} 

\textbf{【病史摘要】}
 男性,3岁。玩耍时不慎将1元硬币吞下,烦躁、哭闹1小时。自述吞咽不适。

\textbf{【X线表现】}  第7颈椎水平见一直径约2.0cm大小圆形不透光异物影。

\textbf{【X线诊断】}  食管入口处金属异物。

\textbf{【评  述】}
 依据患儿有明确的误吞金属异物病史,故常规的透视和颈胸部食管正侧位摄片即可观察到异物的形态,确定异物的位置,不需钡餐造影,诊断一般不会发生困难。需注意的是颈胸部的钙化影和气管内异物有时与食管内不透光异物相似,食管异物在侧位片上,位于气管之后,长形异物与食管纵轴一致;扁平形异物正位呈片状,侧位呈条状,而气管异物恰与此相反。异物最易滞留于食管生理狭窄和压迹处,故应重点观察,尤以食管入口(管径最小)为主,对于滞留于非好发部位的异物,应警惕食管器质性病变的可能性。有时食管内异物在患者的强力吞咽动作下,食管的生理狭窄和压迹处也可以充分扩张,使食管异物通过全食管抵达胃部,甚至肠道,所以在临床工作中,如果怀疑有异物存在,颈胸部X线检查未发现异物时,应进一步检查胃肠道,观察异物是否自行咽下,这是我们要注意的地方。

\subsection{食管透光性异物}

\begin{figure}[!htbp]
 \centering
 \includegraphics{./images/Image00245.jpg}
 \captionsetup{justification=centering}
 \caption{钡棉造影检查}
 \label{fig5-2-2}
  \end{figure} 

\textbf{【病史摘要】}
 男性,50岁。1小时前喝鱼汤时误将鱼刺咽下,感咽部疼痛,吞咽有异物感。

\textbf{【X线表现】}
 钡棉透视示钡棉滞留于食管上段平第6颈椎水平无法下行,未显示异物的形态。

\textbf{【X线诊断】}  食管上段透光异物(鱼刺)。

\textbf{【评  述】}
 对于较小的食管异物或是不透X线的异物时,应行钡棉造影检查,钡棉往往能停挂在异物处,嘱咐患者反复吞咽甚至饮水钡棉仍能停留在原处,称为挂絮征象,可以对细小异物做出诊断,也是对不透X线异物检查的有效方法,小的透光异物,如鱼刺、小骨片等一般常规透视和摄片检查不易发现,简单的钡餐透视亦不能显示。过小、过细的骨和鱼刺或嵌入咽或食管较深外露于粘膜面较小的异物不易显示挂絮征象,异物损伤了咽或食管的粘膜,患者的自觉症状也难以与异物滞留鉴别,拟建议内镜检查。

钡棉造影检查中需要重点注意的是:①当怀疑异物在主动脉弓水平附近而又需钡棉检查才能确定时,此时应以少量多次吞服钡棉为好,如一次性吞服大量钡剂有可能会牵引异物而致使食管穿孔,甚至累及大血管而致大出血,危及患者生命。②如患者吞服异物时间较长,在透视或摄片中见到异物处食管周围软组织肿胀甚至出现气液平面,则提示食管异物处有炎症感染或脓肿形成,此时吞钡检查会出现钡剂外溢现象且不能排空。

\subsection{反流性食管炎}

\begin{figure}[!htbp]
 \centering
 \includegraphics{./images/Image00246.jpg}
 \captionsetup{justification=centering}
 \caption{反流性食管炎}
 \label{fig5-2-3}
  \end{figure} 

\textbf{【病史摘要】}
 男性,55岁。胸骨后及心窝处烧灼感及疼痛,进食尤其是进热食疼痛加剧,卧位或弯腰时加重,有轻度吞咽困难。

\textbf{【X线表现】}
 右前斜卧位片示:食管内大量钡剂反流,中下段粘膜增粗、紊乱,内见小颗粒征,粘膜未见中断破坏。

\textbf{【X线诊断】}  反流性食管炎。

\textbf{【评  述】}
 X线为检查食管炎症重要的方法,造影检查与内镜证实的符合率可达90%以上,尤其对于中晚期病例。检查中应充盈法、粘膜法和低张双对比造影法相结合,还要应用多种体位及增加腹压等措施。早期可仅见功能异常,表现为吞咽激发的原发性蠕动到主动脉弓水平处终止或减弱,胃食管反流致中下段痉挛性狭窄,狭窄段可有蠕动,钡剂通过时可扩张,通过后又重复出现,但形态不固定、与癌性浸润不同。或者粘膜呈颗粒状,颗粒为1~2mm。表浅溃疡则呈小针刺状龛影。在后期,因瘢痕收缩,而致永久性无明显分界的狭窄及短缩,狭窄段一般4~5cm,多数规则、光滑,也可因瘢痕收缩牵引而不规则,呈假憩室状,低张双重造影显示较好。食管短缩者可见牵引性裂孔疝。发现痉挛性狭窄时,应再做双重造影,以显示粘膜改变。反流性食管炎主要应与食管癌鉴别,食管炎时粘膜改变为渐进性,而食管癌有粘膜中断、破坏、融合及管壁僵直等表现,且边界清楚。难于鉴别者需内镜和病理证实。

\subsection{食管结核}

\begin{figure}[!htbp]
 \centering
 \includegraphics{./images/Image00247.jpg}
 \captionsetup{justification=centering}
 \caption{食管结核}
 \label{fig5-2-4}
  \end{figure} 

\textbf{【病史摘要】}
 男性,61岁。因突发大量呕血入院。近半年来感低热,胸骨后疼痛,有时为背痛,多呈持续性,吞咽时加重,体重减轻。

\textbf{【X线表现】}
 食管钡透示:食管中下段管腔狭窄明显,粘膜纹理粗乱不规则,管壁轮廓不规则呈锯齿状,可见小针刺状龛影,管壁僵硬不明显,仍有一定的扩张度,钡剂通过稍受阻。

\textbf{【X线诊断】}  食管中下段结核(溃疡型)。

\textbf{【评  述】}
 食管结核在临床极为少见。患者多以吞咽困难、吞咽痛或胸骨后疼痛为主诉就诊,缺乏典型的结核中毒症状。有的患者以呕血为首发症状,甚至表现为内科治疗无法控制的消化道大出血。食管结核的病理类型可分为3种:①溃疡型。②增殖型。③颗粒型。

食管结核的钡剂造影检查可以发现下列征象:①溃疡型几乎都发生在食管中段,主要表现为食管管腔溃疡,可见龛影,但也并非所有患者都能见到溃疡所形成的龛影这一征象。由于瘢痕收缩及周围组织粘连而使管腔轻度狭窄或正常,粘膜纹理粗乱不规则,管壁轮廓可不规则呈锯齿状,但管壁僵硬不明显,仍有一定的扩张度,钡剂可顺利通过。②增殖型多见于食管中段,其次为下段。X线检查多显示程度不等的管腔狭窄,为侧壁局限性充盈缺损,大小不一,管壁有一定弹性,钡剂通过缓慢,而无梗阻。在充盈缺损附近有软组织肿块影,为增厚的管壁或肿大的淋巴结,病变区域的粘膜纹理可以正常,或变形甚至完全消失。

食管结核主要应与食管癌进行鉴别,主要鉴别点为:①食管结核多发生于青壮年,年龄较轻,低于45岁,女性多见;而恶性肿瘤发病多在50岁以上,男性多见。②食管结核患者多有肺结核病史或结核接触史,胸部X线检查提示肺部有陈旧性结核或有活动性结核病灶。③食管结核临床症状轻,由于结核性食管狭窄引起的吞咽困难进展较缓慢,呈非进行性吞咽困难,与食物性状无关,病程常较短,抗结核药物治疗有效;食管恶性肿瘤引起的吞咽困难及胸痛呈进行性加重,常在短时期内(3个月至半年)出现重度吞咽困难,且一般情况恶化快。其病程较长,常伴消瘦症状。④食管结核皮肤结核菌素试验(PPD皮试)阳性、血清结核抗体阳性。⑤X线钡剂造影检查:食管结核食管腔有充盈缺损和溃疡,或粘膜呈虫蚀样改变,管壁稍僵硬,纵隔淋巴结结核压迫食管所致充盈缺损,多呈弧形,局部粘膜平整,附近有软组织肿块影或病变周围结核钙化影;而食管癌管壁不整、僵硬,粘膜明显破坏,充盈缺损明显且不规则。

\subsection{化学性食管炎}

\begin{figure}[!htbp]
 \centering
 \includegraphics{./images/Image00248.jpg}
 \captionsetup{justification=centering}
 \caption{化学性食管炎}
 \label{fig5-2-5}
  \end{figure} 

\textbf{【病史摘要】}
 男性,20岁。因误服少量烧碱1小时入院,自述吞咽唾液时胸骨后疼痛伴吞咽困难。

\textbf{【X线表现】}
 食管钡透检查示食管上段管壁欠光整,边缘毛糙,食管蠕动较正常减弱。

\textbf{【X线诊断】}  化学性食管炎(烧碱)。

\textbf{【评  述】}
 化学腐蚀剂分为酸性和碱性两类。食管粘膜接触了化学腐蚀剂后,在病理上会产生一系列的变化:在短时间内(数小时至24小时内),食管壁会产生急性炎症反应,导致食管粘膜水肿、渗出、表面糜烂及激惹性的痉挛收缩而出现食管的早期明显狭窄或梗阻,如临床处理及时,在数天后水肿消退且同时伴随组织修补过程,进一步则进入瘢痕形成时期。食管受损的范围及程度与化学腐蚀剂的性质、浓度、剂量及服食速度有关。在急性期,主要表现为食管痉挛性收缩,管腔狭窄,病变以上管腔稍有扩大,病变部食管壁边缘不光滑,呈不规则或串珠状改变,粘膜像可显示粘膜增粗或消失;在恢复期,上述征象会有改善。但在病变后期,由于纤维组织增生及瘢痕形成,食管腔会显示连续性的进一步狭窄或间断性狭窄,边缘尚光整或稍不规则,粘膜消失,狭窄段以上食管扩张。

需要重点注意的是:一般需在临床紧急处理后,待病情稳定,再行食管钡餐检查。如怀疑有食管穿孔的可能性,则要求停用钡剂造影而改用碘油造影。

\subsection{食管静脉曲张}

\begin{figure}[!htbp]
 \centering
 \includegraphics{./images/Image00249.jpg}
 \captionsetup{justification=centering}
 \caption{食管静脉曲张}
 \label{fig5-2-6}
  \end{figure} 

\textbf{【病史摘要】}
 男性,61岁。因突发呕血2小时入院,患肝硬化、脾大6年,腹水征阳性。

\textbf{【X线表现】}
 食管钡透示食管上、中、下段粘膜增粗、紊乱,其间可见串珠状或蚯蚓状充盈缺损,食管管壁边缘凹凸不平呈锯齿状,钡剂通过缓慢。

\textbf{【X线诊断】}  食管静脉曲张(重度)。

\textbf{【评  述】}
 轻度静脉曲张局限于食管下段,粘膜皱襞略增粗,管腔边缘可呈轻微的锯齿状,管壁张力无明显异常,此时如检查方法不当或观察不仔细可漏诊;中度静脉曲张,病变累及中段,粘膜皱襞明显增粗,呈串珠状或蚯蚓状,食管边缘呈明显的锯齿状,管壁张力欠佳,钡剂通过迟缓;重度静脉曲张,病变累及食管上段,甚至膈上全部食管,管腔明显扩张,正常粘膜被大小、形态不一的圆形、环形充盈缺损取代,形成链状,食管轮廓更加不整,但管壁柔软,钡剂通过更加迟缓。钡剂检查时,钡剂不宜过多,以避免对曲张的静脉形成物理性挤压作用;钡剂宜一次吞下,防止多次吞咽产生的气泡伪影,干扰诊断。

食管静脉曲张表现典型,如检查方法得当,诊断并不困难。需鉴别者有:①气泡影:随钡剂吞入的小气泡随检查时间推移而变动位置或消失,静脉曲张形态可变化但持续存在。②食管癌:虽然下段食管癌可呈息肉状改变,但其病变局限,边界分明,管壁僵直,粘膜中断、破坏,都具特征性,而与静脉曲张不同。

\subsection{食管功能性憩室}

\begin{figure}[!htbp]
 \centering
 \includegraphics{./images/Image00250.jpg}
 \captionsetup{justification=centering}
 \caption{食管功能性憩室}
 \label{fig5-2-7}
  \end{figure} 

\textbf{【病史摘要】}  女性,30岁。因咽部不适,吞咽时有异物感。

\textbf{【X线表现】}
 食管钡透示食管上段主动脉弓下见囊袋状突起,食管壁柔软,钡剂下行顺畅。

\textbf{【X线诊断】}  食管功能性憩室。

\textbf{【评  述】}
 食管钡透时,可于主动脉弓压迹与左主支气管压迹之间,食管显示略膨出,注意不要误认为器质性憩室。

\subsection{食管憩室}

\begin{figure}[!htbp]
 \centering
 \includegraphics{./images/Image00251.jpg}
 \captionsetup{justification=centering}
 \caption{食管中段憩室}
 \label{fig5-2-8}
  \end{figure} 

\begin{figure}[!htbp]
 \centering
 \includegraphics{./images/Image00252.jpg}
 \captionsetup{justification=centering}
 \caption{食管憩室}
 \label{fig5-2-9}
  \end{figure} 

\textbf{【病史摘要】}  女性,35岁。胸背部不适,吞咽时有哽噎感半年。

\textbf{【X线表现】}
 右前斜位及左前斜位示食管中段囊性突起影,内有钡剂充盈,体位改变后,钡剂部分流出。

\textbf{【X线诊断】}  食管中段憩室。

\textbf{【评  述】}
 食管憩室是食管管壁的囊袋状突出,根据发生的部位,分为咽食管憩室、食管中段憩室、膈上食管憩室。X线检查对憩室的诊断起决定作用。因绝大多数的憩室起自食管的前壁或右侧壁,因此,左前斜位或右前斜位显示较好;有时需做俯卧位,以便钡剂进入憩室。食管憩室吞钡检查表现为囊袋状突出影,边缘光滑整齐,口部较小或较宽,大小可变,有时可见粘膜皱襞伸入。咽食管憩室较大时第6颈椎前软组织增宽,其内可见液平;因常有滞留物(食物或粘液等),充钡时密度不均或呈分层状,大的憩室可压迫食管使其前移。食管中段憩室,多位于气管分叉部附近的前壁或侧前壁,憩室顶端可呈牵幕状,颈部较宽。憩室伴有炎症时,其边缘不规则,邻近食管可有痉挛。憩室穿孔时,可见造影剂流入纵隔或气管、支气管。需要重点注意的是:食管中段憩室应与主动脉和左主支气管压迹之间的食管膨出相鉴别;膈上食管大憩室应注意与食管裂孔疝鉴别,憩室囊袋状结构影与食管相连,而食管裂孔疝的膈上疝囊则通过裂孔与胃相连。

\subsection{食管颈椎增生压迹}

\begin{figure}[!htbp]
 \centering
 \includegraphics{./images/Image00253.jpg}
 \captionsetup{justification=centering}
 \caption{食管颈椎增生压迹}
 \label{fig5-2-10}
  \end{figure} 

\textbf{【病史摘要】}
 男性,69岁。颈椎部疼痛,伴左侧前臂麻木,伴有吞咽时哽噎感。

\textbf{【X线表现】}
 食管上段钡透侧位片示:食管上段平4、5椎体水平后缘见弧形压迹,食管壁柔软,颈椎生理弧度僵直,第4、5颈椎体前缘见唇样骨赘形成。

\textbf{【X线诊断】}  食管颈椎增生压迹。

\textbf{【评  述】}
 食管为后纵隔的肌性器官,两端固定,中间可以移动。食管外压性改变可以是由于脊柱椎体骨质过度增生对食管后方产生局部压迫。临床上多有原发疾病的症状,伴有不同程度的吞咽困难或吞咽受阻感。

\subsection{贲门失迟缓症}

\begin{figure}[!htbp]
 \centering
 \includegraphics{./images/Image00254.jpg}
 \captionsetup{justification=centering}
 \caption{贲门失迟缓症}
 \label{fig5-2-11}
  \end{figure} 

\textbf{【病史摘要】}
 女性,45岁。间歇性胸骨后疼痛,吞咽困难2~3年,近年来自觉胸闷心慌,吞咽困难呈持续性,有时伴有呕吐。

\textbf{【X线表现】}
 食管显著扩张,管径5cm左右,下段扩张,呈萝卜根状,腔内多量钡剂潴留,中下段食管蠕动消失。狭窄段食管管壁光滑,柔软(图A、B)。

\textbf{【X线诊断】}  贲门失迟缓症(早期)。

\textbf{【评  述】}
 本病病因不明,一般认为是由于迷走神经的退行性变所致。临床一般见于20~40岁,女性较多。病程长,可数月至数年,常见症状为吞咽困难,胸骨后有阻塞感,以进食固体性食物时明显,症状时轻时重,与精神因素有一定关系。进食较快或梗阻严重时可出现呕吐。严重的食管失迟缓症,胸部X线片可因高度扩张的食管,使纵隔增宽,其内常有液平,而不致误为纵隔肿瘤。一般需钡餐造影确诊。早期,食管轻度扩张,以下半部明显。食管正常蠕动减弱或消失,代之以紊乱的环肌收缩,食管下端呈漏斗状进入膈下,狭窄段为1~4cm,边缘光滑整齐。呼气时狭窄段管腔可略增宽,吸气时变窄,因此,钡剂可随呼吸断续通过。狭窄段的粘膜细而平行,有利于与浸润型食管癌鉴别。晚期,食管显著扩张、延长、迂曲,食管的不规则收缩减弱或消失,或在服钡的瞬间看到几个蠕动波。第一口钡剂有时可少量通过狭窄段,之后连续服钡,需达一定量,常到主动脉弓水平或更高,借助钡剂的重力,才可经狭窄段喷射进入胃内。食管下段呈S形弯曲,下端呈鸟嘴状,边缘光滑、对称。深呼吸时膈肌裂孔迟缓,狭窄段可略变宽(图C、D)。这种随膈肌裂孔张缩而出现的变化,说明管壁柔软,有助于与食管癌鉴别。

\subsection{食管裂孔疝}

\begin{figure}[!htbp]
 \centering
 \includegraphics{./images/Image00255.jpg}
 \captionsetup{justification=centering}
 \caption{食管裂孔疝}
 \label{fig5-2-12}
  \end{figure} 

\textbf{【病史摘要】}
 女性,35岁。胸骨后不适、烧灼、疼痛2年余,饱食后平卧位症状加重,疼痛向肩部放射。

\textbf{【X线表现】}
 左侧膈上见大小约3.5cm×4.0cm的疝囊影,内见粗大弯曲粘膜皱襞,下方见较宽粘膜皱襞通过裂孔与胃相连,贲门位于膈上疝囊内。

\textbf{【X线诊断】}  食管裂孔疝。

\textbf{【评  述】}
 腹腔内脏器移位于胸腔,称为膈疝。腹腔内脏器经食管裂孔疝入胸腔者,称食管裂孔疝,约占膈疝的70%。一般将食管裂孔疝分为四型:①可复型食管裂孔疝。②牵引型食管裂孔疝。③食管旁食管裂孔疝。④先天性短食管型裂孔疝。X线检查为食管裂孔疝的可靠的检查方法。

食管裂孔疝常用检查方法是:①仰卧头低足高位大量服钡使胃过度充盈,之后从右前斜位转至左前斜位,或患者仰卧直腿抬高同时腹部加压。②卧位Valsalva呼吸实验。③胃充满后侧立位弯腰,有利于疝囊的显示。

食管裂孔疝的X线表现有:①膈上疝囊:为疝入胸腔的小部分胃构成,除食管旁型裂孔疝外,皆包括胃食管前庭部。疝囊呈圆柱状或漏斗状,疝囊上方可见下食管括约肌形成的收缩区,称A环。②食管胃环:为食管粘膜与胃粘膜交界部,正常位于膈下,不能显示,裂孔疝时,疝入胸腔,变为疝囊两侧对称性、局限性切迹,称B环。③膈上出现胃粘膜:表现为粗大迂曲的皱襞。④胃小区:个别患者双重造影时,裂孔上出现胃小区。⑤鸟嘴征:仰卧位时,钡剂使贲门轻度张开,形似鸟嘴状,常与其他裂孔疝之X线征象并存。⑥孔征:膈上胃囊充气时,轴位投影于胃底,其形态似孔,称孔征。⑦食管旁型食管裂孔疝:贲门仍位于膈下,疝囊在食管左前方,较大时可压迫食管。根据以上表现,裂孔疝诊断不难。在鉴别诊断方面,应注意不可将食管膈壶腹误为裂孔疝,前者为生理性表现,位于膈上4~5cm一段食管,扩大呈椭圆形,粘膜为纵行纤细的食管粘膜,无下食管括约肌收缩环及疝囊表现。膈上憩室,扩大的囊腔与食管有窄颈相连,其下有一段正常食管通过食管裂孔与贲门相连,胃底正常。

\subsection{食管平滑肌瘤}

\begin{figure}[!htbp]
 \centering
 \includegraphics{./images/Image00256.jpg}
 \captionsetup{justification=centering}
 \caption{食管平滑肌瘤}
 \label{fig5-2-13}
  \end{figure} 

\textbf{【病史摘要】}
 女性,45岁。近2年来进食时有哽噎感,无异物感及疼痛,既往体健,无消瘦。

\textbf{【X线表现】}
 食管钡透检查,示食管中下段椭圆形充盈缺损,见环形征,边缘光滑,食管粘膜未见明显中断、破坏,管腔无明显狭窄,管壁柔软无僵硬。

\textbf{【X线诊断】}  食管中下段平滑肌瘤。

\textbf{【评  述】}
 食管平滑肌瘤是最常见的食管良性肿瘤,占食管良性肿瘤的2/3。食管平滑肌瘤起自平滑肌层或粘膜肌层,位于壁内粘膜下,呈膨胀性生长。平滑肌瘤可发生于食管的各段,以中下段多见。钡餐检查最常见的征象为充盈缺损,呈圆形、椭圆形,边界清楚,较多钡剂通过后,肿瘤周围仍有钡剂存留,形成所谓环形征;肿瘤表面粘膜可变宽或展平,少数病例可见溃疡龛影。钡剂通过肿瘤部位时,在正位,钡剂自肿瘤两侧分流,管腔可变宽;切线位时,钡剂偏流而过。食管平滑肌瘤和壁内其他良性肿瘤的X线征象相似,难以鉴别。恶性肿瘤中食管平滑肌肉瘤罕见,可分息肉型及浸润型,前者多表现为不规则的分叶状或表面有大小不等的息肉状充盈缺损,易发生溃疡;后者与食管癌相似。食管平滑肌瘤则极少发生溃疡,其典型表现为规则的圆形或类圆形充盈缺损。平滑肌瘤与食管癌的鉴别,主要是癌瘤不规则,粘膜破坏,及浸润性生长而致管腔狭窄、僵硬等。

\subsection{食管癌(早期)}

\begin{figure}[!htbp]
 \centering
 \includegraphics{./images/Image00257.jpg}
 \captionsetup{justification=centering}
 \caption{食管癌(早期)}
 \label{fig5-2-14}
  \end{figure} 

\textbf{【病史摘要】}  男性,65岁。咽部不适伴胸骨后轻微疼痛6个月余。

\textbf{【X线表现】}
 食管钡透示:食管上段平第6、7胸椎水平局限性粘膜皱襞扭曲、中断。食管管壁边缘毛糙,呈轻度缩窄,食管蠕动较差。

\textbf{【X线诊断】}  食管上段早期癌。

\textbf{【评  述】}
 本例经手术证实为早期食管癌。早期食管癌病变表浅,X线改变比较轻微,由于造影检查时食管粘膜皱襞显示不清,诊断往往困难。所以早期食管癌的X线检查,应重点注意食管的蠕动、管壁的扩张情况,并多轴位的双重造影像及粘膜像结合诊断。早期食管癌的X线表现主要为:①食管粘膜皱襞的改变:粘膜皱襞增粗、迂曲,有1~2条粘膜皱襞中断,边缘毛糙。②形成小溃疡:在紊乱粗糙的粘膜面上出现小溃疡,可单发或多发,大小不等,一般在0.2~0.4cm,局部管壁轻度痉挛。③局限性小充盈缺损:直径多在0.5cm左右,最大不超过2cm,边缘毛糙,局部粘膜紊乱,少数病例在充盈缺损的病灶中有米粒样龛影。④管壁局限性僵硬:少数病例出现局限性舒展度减低,偏侧性管壁僵直。钡剂在此处通过减慢,呈滞留现象,或出现痉挛性收缩波。在粘膜像阴性情况下,这些征象可作为早期食管癌的定位征象。

\subsection{进展期食管癌(浸润型)}

\begin{figure}[!htbp]
 \centering
 \includegraphics{./images/Image00258.jpg}
 \captionsetup{justification=centering}
 \caption{进展期食管癌(浸润型)}
 \label{fig5-2-15}
  \end{figure} 

\textbf{【病史摘要】}
 男性,56岁。进行性吞咽困难5个月余,近1个月来感胸骨后疼痛,只能进流质,并有呕吐症状。

\textbf{【X线表现】}
 食管钡透示:食管中下段管腔环形狭窄,钡剂下行受阻,狭窄段呈漏斗状,管壁僵硬,蠕动消失,狭窄段以上食管扩张明显。

\textbf{【X线诊断】}  进展期食管癌(浸润型)。

\subsection{进展期食管癌(溃疡型)}

\begin{figure}[!htbp]
 \centering
 \includegraphics{./images/Image00259.jpg}
 \captionsetup{justification=centering}
 \caption{进展期食管癌(溃疡型)}
 \label{fig5-2-16}
  \end{figure} 

\textbf{【病史摘要】}
 女性,48岁。进行性吞咽困难3个月,近期进食流质时出现哽噎,消瘦。

\textbf{【X线表现】}
 食管钡透示:食管中下段管腔局限性狭窄,粘膜皱襞中断破坏,并见一较大龛影,与食管纵轴一致,切线位位于食管轮廓之内。

\textbf{【X线诊断】}  进展期食管癌(溃疡型)。

\subsection{进展期食管癌(增生型)}

\begin{figure}[!htbp]
 \centering
 \includegraphics{./images/Image00260.jpg}
 \captionsetup{justification=centering}
 \caption{进展期食管癌(增生型)}
 \label{fig5-2-17}
  \end{figure} 

\textbf{【病史摘要】}
 男性,45岁。进行性吞咽困难3个月余,近期进食固体类食物时下咽困难,流质尚可,既往体健。

\textbf{【X线表现】}
 食管钡透示:食管中上段管腔内见不规则充盈缺损,管腔呈偏心性狭窄,充盈缺损基底部管壁僵硬,蠕动消失。

\textbf{【X线诊断】}  进展期食管癌(增生型)。

\textbf{【评  述】}
 本病一般分为三型:浸润型、溃疡型、增生型。进展期食管癌侵及肌层后,进展加快,X线征象也日益显著,主要表现为:①粘膜皱襞增粗、紊乱、中断、破坏,代之以肿瘤形成的不规则影。②病变区管腔不规则、偏心性狭窄,管壁僵硬,伴有梗阻征,其近端食管扩张。③腔内不规则的充盈缺损,其表面常有破坏形成的龛影。④不规则的龛影,位于食管轮廓之内。上述征象常混合存在。

食管癌的类型不同,X线表现也各具特征:①浸润型:管腔呈环形狭窄,范围局限,一般为3~5cm,严重时呈漏斗状,管壁僵硬,边缘多较光滑,上段食管扩张明显。②增生型:以腔内不规则的充盈缺损及管腔偏心性不规则的狭窄为特征,充盈缺损表面常有不规则的溃疡,为肿瘤坏死所致。③溃疡型:以边界清楚、形态不规则的龛影为特征。龛影多较长,与食管纵轴一致,在切线位多在食管轮廓线内,较深时可超出食管轮廓以外。溃疡周围隆起明显者,可见环堤征。增生型食管癌需注意与良性肿瘤中最多见的平滑肌瘤相鉴别,后者切线位也可表现为管腔内圆形或椭圆形充盈缺损,致食管管腔狭窄,但其边缘一般光滑、肿瘤区粘膜皱襞可消失而周围粘膜皱襞正常,管壁柔软,正位可显示钡剂环绕形成的环形征,管腔可变宽,管腔外可见软组织肿块影是其特征。增生型食管癌,特别是较大者与恶性癌肉瘤X线区分有一定难度,可以作为参考的是癌肉瘤虽然肿瘤较大,与食管癌相比较,患者临床梗阻症状一般较轻,癌肉瘤多发生在食管中下段,管腔外往往可显示有软组织块影,而食管癌最常见发生在食管中上段,管腔外较少形成软组织肿块影,两者表现不同,有时也需结合内镜病理活检方可鉴别。

\subsection{食管平滑肌肉瘤}

\begin{figure}[!htbp]
 \centering
 \includegraphics{./images/Image00261.jpg}
 \captionsetup{justification=centering}
 \caption{食管平滑肌肉瘤}
 \label{fig5-2-18}
  \end{figure} 

\textbf{【病史摘要】}
 男性,62岁。自述吞咽困难3个月,胸骨后感疼痛,既往体健。

\textbf{【X线表现】}
 食管钡透示:食管中下段管腔呈梭形扩张,内见多枚大小不等类圆形充盈缺损,食管壁尚光滑,食管粘膜皱襞消失。

\textbf{【X线诊断】}  食管平滑肌肉瘤。

\textbf{【评  述】}
 本病少见。好发于食管中下段,多呈息肉状突入管腔,少数为浸润性生长,肿瘤常限于粘膜或粘膜下层,个别为环形浸润,很少转移,预后较好。临床表现不具特征性,常因吞咽困难就诊。X线表现典型者为食管中下段腔内息肉状充盈缺损,基底小,可有蒂,局部管腔扩张。少数不典型者,如呈环形浸润性生长者与增生型食管癌难于诊断,需结合内镜病理活检方可鉴别。

\section{胃部病变}

\subsection{胃憩室}

\begin{figure}[!htbp]
 \centering
 \includegraphics{./images/Image00262.jpg}
 \captionsetup{justification=centering}
 \caption{胃底憩室}
 \label{fig5-3-1}
  \end{figure} 

\textbf{【病史摘要】}
 女性,31岁。上腹部不适1周,无嗳气、反酸,无腹胀。体格检查:上腹部无明显压痛,肝、脾未及,心肺阴性。

\textbf{【X线表现】}
 胃钡透示:胃底部见囊袋状突起,边缘光滑整齐,内见钡剂残留,颈部狭窄,胃底粘膜伸入囊袋。

\textbf{【X线诊断】}  胃底憩室。

\textbf{【评  述】}
 本病一般为单发,80%位于贲门下方小弯侧的后壁,其次为幽门前区。憩室呈圆形或囊袋状,颈部狭窄、光滑。缺乏肌层者无收缩力,常因食物残留而致憩室炎。胃周粘连牵拉所致者,口部较宽,很少有食物残留。憩室有完整的粘膜层。胃憩室多无症状。伴发憩室炎时,可有腹痛、腹胀、恶心、呕吐及出血表现。X线表现主要有:①憩室外形光滑,如囊袋状影突出于胃腔之外,但基底部与胃相连,颈部略细。②粘膜像可见胃粘膜皱襞自颈部与憩室相连。③如合并憩室炎,其外形变得不规则,并可见局部胃壁痉挛变形等改变。根据憩室上述X线表现特点,尤其要注意粘膜皱襞形态,易与胃穿透性溃疡鉴别。憩室呈光滑的囊袋状,有正常的粘膜伸入其内,而没有粘膜皱襞纠集等表现;穿透性溃疡无粘膜皱襞伸入其中是与胃憩室区别之要点。

\subsection{胃底静脉曲张}

\begin{figure}[!htbp]
 \centering
 \includegraphics{./images/Image00263.jpg}
 \captionsetup{justification=centering}
 \caption{胃底静脉曲张}
 \label{fig5-3-2}
  \end{figure} 

\textbf{【病史摘要】}
 男性,61岁。患肝硬化多年,突发呕血1天伴黑便。体格检查:肝、脾大,腹水征阳性,腹壁静脉曲张,功能异常。

\textbf{【X线表现】}
 胃钡透示:胃底粘膜皱襞增宽迂曲,呈蚯蚓状,边缘光滑,未见明显中断破坏。

\textbf{【X线诊断】}  胃底静脉曲张。

\textbf{【评  述】}
 本病患者除具有门脉高压症的表现外(如肝脾肿大、脾功能亢进、肝功能异常、腹水、腹壁静脉曲张等),主要表现为呕血及黑便。X线检查具有安全、简便、准确的特点,易为患者接受。胃底静脉曲张常与食管静脉曲张并存,但也可单独存在,双对比造影可提高其显示率。胃底静脉曲张一般可分为两型:①泡沫型:胃贲门区及胃底粘膜呈葡萄状或息肉状透亮区,直径1~2cm,周围见薄层钡剂环绕,形如泡沫状。②肿块型:胃贲门区及胃底呈分叶状或团块状边缘光滑的充盈缺损。除上述表现外,胃底静脉曲张时,胃底与膈肌间距可增大,胃贲门角增大。因多伴有脾肿大,可造成胃的压迫性移位。胃底静脉曲张主要应与胃癌鉴别,静脉曲张形成的肿块影边缘光滑锐利,胃壁柔软(可借助气钡双重造影、呼吸动作或心脏搏动观察),贲门及腹段食管不被侵犯,结合临床病史,可以鉴别,个别病例可借助内镜检查或选择性血管造影帮助鉴别。

\subsection{胃内异物}

\begin{figure}[!htbp]
 \centering
 \includegraphics{./images/Image00264.jpg}
 \captionsetup{justification=centering}
 \caption{胃内异物(胃柿石)}
 \label{fig5-3-3}
  \end{figure} 

\textbf{【病史摘要】}
 女性,35岁。上腹部不适伴疼痛2个月余,可自行缓解,近期有大量进食柿子史。

\textbf{【X线表现】}
 胃钡透示:胃窦部见类圆形充盈缺损影,大小4cm×3.5cm左右,边缘稍毛糙,表面见凹凸不平的不规则钡斑,体位改变后,充盈缺损位置改变。

\textbf{【X线诊断】}  胃内异物(胃柿石)。

\textbf{【评  述】}
 柿子、豆类、毛发、绒线及粘液性物质进入胃内,因机械作用而相互缠绕成团,形成胃石。在产柿地区,胃柿石是最常见的胃石。因进食大量不成熟柿子后,与胃酸起作用,凝结成块,而成胃石。X线表现胃内可见类圆形充盈缺损影,体积可以很大,也可分成数块,表面凹凸不平,呈现不规则的钡斑。充盈缺损可在胃内移动,压迫或变动体位,其位置有变化。此外,周围胃壁柔软,蠕动正常,这些特点可与胃肿瘤相鉴别。特别注意的是要结合临床病史做出最后诊断。

\subsection{幽门肌肥厚症}

\begin{figure}[!htbp]
 \centering
 \includegraphics{./images/Image00265.jpg}
 \captionsetup{justification=centering}
 \caption{幽门肌肥厚}
 \label{fig5-3-4}
  \end{figure} 

\textbf{【病史摘要】}
 女性,40岁。右上腹部饱胀数月,无明显疼痛,时有恶心、呕吐。体格检查:中等体质,腹软,肝、脾未及,右上腹无明显压痛,心、肺阴性。

\textbf{【X线表现】}
 胃钡透示:胃幽门管细而长,其纵行粘膜皱襞显示呈双轨征,十二指肠球基底部形成蘑菇征。

\textbf{【X线诊断】}  幽门肌肥厚。

\textbf{【评  述】}
 本病是胃幽门环肌高度肥厚所致。多见于成年人。腹痛、腹胀、恶心、呕吐等为常见症状。幽门肌肥厚X线表现主要有:①钡剂通过狭窄的幽门管,幽门管显示狭窄而延长,呈线条状。②由于肥大的幽门肌终止于十二指肠球基底部,造成肥大的环形肌肉压迫球部基底部形成蘑菇征。③钡剂进入狭窄的幽门管,当充盈不全时似一个细长的鸟嘴突向十二指肠球部。④狭窄之幽门管纵行的粘膜皱襞显影形成双轨征。X线诊断应与幽门痉挛鉴别,但幽门痉挛无以上X线征象,确诊并不困难。

\subsection{胃息肉}

\begin{figure}[!htbp]
 \centering
 \includegraphics{./images/Image00266.jpg}
 \captionsetup{justification=centering}
 \caption{胃息肉}
 \label{fig5-3-5}
  \end{figure} 

\textbf{【病史摘要】}
 女性,43岁。上腹部疼痛不适,无食欲减退、嗳气、反酸。体格检查:腹软,上腹部轻压痛,肝、脾未及,心、肺阴性。

\textbf{【X线表现】}
 胃钡透示:胃窦部圆形充盈缺损,边缘整齐锐利,表面光滑,周围胃壁柔软,无僵硬,蠕动正常。

\textbf{【X线诊断】}  胃息肉(胃窦部)。

\textbf{【评  述】}
 本病常为单发,也可多发,多发者称胃息肉病。典型X线表现呈圆形或类圆形充盈缺损突入于胃腔,有蒂或无蒂,直径一般小于2cm。多见于胃窦及胃体下部,幽门前区带蒂息肉可脱入十二指肠内。息肉表面光滑整齐。组织学检查可分为腺瘤性息肉及增生性息肉两类。腺瘤性息肉多位于胃窦部,常伴萎缩性胃炎,可分为腺瘤及乳头状瘤,后者可呈菜花状。增生性息肉是在慢性胃炎基础上发生的,很少超过1cm。腺瘤性息肉被认为是癌前期病变,可与胃癌同时存在。临床一般多无症状,少数可有上腹部不适疼痛。带蒂者可随蠕动或压迫而移位。在幽门前区者突入十二指肠时,表现为十二指肠球部的充盈缺损,充盈加压检查或双重造影法可显示其带蒂。乳头状腺瘤可不规则。息肉应与息肉样癌鉴别,息肉样癌的充盈缺损一般多大于2cm,形态不规则,表面粗糙,肿瘤与胃壁交界欠清,一般无蒂。

\subsection{胃窦炎}

\begin{figure}[!htbp]
 \centering
 \includegraphics{./images/Image00267.jpg}
 \captionsetup{justification=centering}
 \caption{胃窦炎伴幽门痉挛}
 \label{fig5-3-6}
  \end{figure} 

\textbf{【病史摘要】}
 女性,40岁。左上腹部不适3个月余,食欲减退,时有恶心、呕吐。体格检查:腹软,肝、脾未及,上腹压痛,心、肺阴性。

\textbf{【X线表现】}
 胃窦部粘膜皱襞增粗、紊乱,幽门管痉挛,钡剂通过幽门管稍受阻,胃窦壁轮廓见锯齿状影,胃窦壁柔软,蠕动增强。

\textbf{【X线诊断】}  胃窦炎伴幽门痉挛。

\textbf{【评  述】}
 胃窦炎为局限于胃窦部的慢性炎症,可为浅表性或萎缩性,十分常见。轻症无阳性X线表现。常见的异常征象有:胃窦部粘膜皱襞增粗、紊乱。正常的胃窦部粘膜皱襞较体部细小,胃窦炎时可增大。紊乱的粘膜皱襞,即使在半收缩状态也呈横行,使窦部轮廓呈光滑的锯齿状。增粗的粘膜皱襞可呈息肉状,并随蠕动、舒缩或压迫而变形。肌层受累者,窦部呈向心性狭窄,但仍可见呈纵行的粘膜皱襞。窦部因环形及纵行肌的收缩与增厚而变短、变窄,其界线呈渐进性或较清楚。肌层的痉挛或增厚可致幽门前区小弯侧呈弧形压迫。幽门管可变窄并伸长。粘膜下层的增厚,使粘膜活动度增加,易形成粘膜脱垂。胃小沟及胃小区增宽、增大。窦部痉挛及分泌功能增强为常见的功能异常。胃窦炎常致窦部狭窄而应与胃窦癌鉴别。胃窦炎的狭窄,形态可变、粘膜皱襞存在、轮廓也较整齐,而胃窦癌表现为胃窦狭窄壁僵硬,与正常胃段分界陡峭呈截断征象,粘膜皱襞破坏,典型者可呈肩胛征。

\subsection{慢性胃炎}

\begin{figure}[!htbp]
 \centering
 \includegraphics{./images/Image00268.jpg}
 \captionsetup{justification=centering}
 \caption{慢性胃炎}
 \label{fig5-3-7}
  \end{figure} 

\textbf{【病史摘要】}
 男性,65岁。上腹部疼痛不适,嗳气、反酸、餐后饱胀数月。体格检查:腹软,肝、脾未及,右上腹压痛,心、肺阴性。

\textbf{【X线表现】}
 胃钡透示:胃体、胃窦粘膜皱襞增粗、肥厚、紊乱,部分呈弯曲、交叉状,胃体、胃窦处胃壁毛糙,压迫后胃壁柔软,胃蠕动正常。

\textbf{【X线诊断】}  慢性胃炎。

\textbf{【评  述】}
 本病为成人常见病,病因尚未完全明确。病理上慢性胃炎可分为慢性浅表性胃炎和慢性萎缩性胃炎及慢性肥厚性胃炎。慢性胃炎时粘膜充血、水肿、炎性细胞浸润及纤维结缔组织增生。轻微者肉眼难以发现,较重者粘膜皱襞增粗、迂回呈脑回状;部分萎缩性胃炎粘膜层萎缩变薄,皱襞细小。慢性胃炎病程较长,可长期反复发作。一般临床表现为食欲不振、腹痛、腹胀、恶心、呕吐、嗳气等。萎缩性胃炎可有贫血、营养不良、腹泻等表现。慢性胃炎的X线表现主要为粘膜皱襞增粗、迂曲、走行异常、失去与小弯平行的特点,体部及窦部粘膜皱襞超过0.5cm,甚至超过1cm;充盈像,因粘膜皱襞增粗、迂曲而使小弯侧凹凸不平,但形态不变,蠕动正常,而不致误为肿瘤。除上述外,还可见分泌功能增强及蠕动增强等变化。部分萎缩性胃炎粘膜皱襞纤细、稀少,服钡或双重造影的气体稍多,胃呈轻度扩张时,皱襞即可变平,甚至大弯侧也可变得光滑。胃小区增大,多数大于3mm,而且粗糙不规则。慢性胃炎常与溃疡并存,而有相应X线征。

\subsection{腐蚀性胃、十二指肠炎}

\begin{figure}[!htbp]
 \centering
 \includegraphics{./images/Image00269.jpg}
 \captionsetup{justification=centering}
 \caption{腐蚀性胃、十二指肠炎}
 \label{fig5-3-8}
  \end{figure} 

\textbf{【病史摘要】}
 女性,35岁。因进食时吞咽困难、呕吐频繁伴胸痛入院,数月前有硫酸误服致上消化道灼伤史。

\textbf{【X线表现】}
 胃钡透示:胃、十二指肠高度狭窄、壁僵硬,粘膜皱襞消失,部分边缘可见针尖样突出的小溃疡。

\textbf{【X线诊断】}  腐蚀性胃、十二指肠炎。

\textbf{【评  述】}
 吞服酸性腐蚀剂类物质易损伤食管、胃、十二指肠,若腐蚀剂浓度高、量大、接触时间长,可引起食管、胃、十二指肠以及空肠的烧灼性炎症。病理改变主要为粘膜及粘膜下层坏死。溃疡形成,晚期纤维瘢痕形成导致不同程度各种各样的狭窄。X线表现早期改变为胃粘膜皱襞粗大、水肿,可有溃疡龛影。胃蠕动消失。晚期可见胃腔狭窄呈漏斗状,胃壁边缘不规则如锯齿状,胃幽门瘢痕性狭窄。由于有患者误服腐蚀剂病史,故诊断一般不难。

\subsection{胃粘膜脱垂}

\begin{figure}[!htbp]
 \centering
 \includegraphics{./images/Image00270.jpg}
 \captionsetup{justification=centering}
 \caption{胃粘膜脱垂}
 \label{fig5-3-9}
  \end{figure} 

\textbf{【病史摘要】}
 男性,45岁。因上腹部不适,嗳气、反酸入院。体格检查:腹软,肝、脾未及,右上腹压痛,心、肺阴性。

\textbf{【X线表现】}
 胃钡透示:幽门管变宽,内见条状平行胃粘膜皱襞,十二指肠球部呈伞状,基底部见类圆形充盈缺损影。

\textbf{【X线诊断】}  胃粘膜脱垂。

\textbf{【评  述】}
 胃粘膜进入十二指肠称为胃粘膜脱垂,常为可复性。常见症状为上腹部疼痛,可随体位改变而缓解。X线表现随脱垂的粘膜数量及程度而异,一般可见幽门管增宽,其内可见条形皱襞。十二指肠球内见圆形或椭圆形充盈缺损,位于正中或呈偏侧性,随窦部的加压或体位的改变而时隐时现。球底一般呈伞状。诊断时应与幽门前区带蒂肿瘤脱入十二指肠相鉴别。后者形态、大小固定,不随压迫变形,回纳后,幽门前区仍可见之。钡餐检查时,当怀疑有胃粘膜脱垂可能时,需要重点注意的是:①应充分利用立位检查或腹部加压检查。②尽可能使球部纵轴走行方向与X线方向垂直。③胃窦处于舒张状态时摄片。④诊断胃粘膜脱垂时,必须肯定球底部之阴影为胃粘膜皱襞,除外体位不当造成的假象。

\subsection{胃溃疡}

\begin{figure}[!htbp]
 \centering
 \includegraphics{./images/Image00271.jpg}
 \captionsetup{justification=centering}
 \caption{胃角溃疡}
 \label{fig5-3-10}
  \end{figure} 

\textbf{【病史摘要】}
 男性,55岁。上腹部不适数年,进餐后可缓解,近一周上腹部疼痛加重,具有周期性及节律性,伴恶心、呕吐、嗳气、反酸。体格检查:腹软,肝、脾未及,右上腹压痛明显,心、肺阴性。

\textbf{【X线表现】}
 胃钡透示:胃角处见一突出于胃腔外的乳头状影,基底部见狭颈征,龛周粘膜皱襞纠集。

\textbf{【X线诊断】}  胃角溃疡。

\textbf{【评  述】}
 本例经胃镜检查病理证实为胃体小弯侧溃疡。上消化道钡餐造影显示胃角处见一突出于胃腔外的乳头状影,基底部见狭颈征,龛周粘膜皱襞纠集,符合良性胃溃疡的X线表现,故诊断不难。需要鉴别的是胃小弯侧恶性溃疡,后者以壁龛及邻近胃壁变化为主要表现。龛影多数较浅而大,形态多不规则,具有特征性的为口部指压迹征和裂隙征,与良性溃疡平坦的口部出现的狭颈征、项圈征对比分明。

\subsection{幽门管溃疡}

\begin{figure}[!htbp]
 \centering
 \includegraphics{./images/Image00272.jpg}
 \captionsetup{justification=centering}
 \caption{幽门管溃疡}
 \label{fig5-3-11}
  \end{figure} 

\textbf{【病史摘要】}
 男性,45岁。上腹部疼痛伴嗳气、反酸2个月,近期疼痛加重伴呕吐。体格检查:腹软,肝、脾未及,右上腹压痛明显,心、肺阴性。

\textbf{【X线表现】}
 上消化道钡餐造影示:幽门管区见突出于胃腔外的三角形龛影,底部狭窄,呈项圈征。

\textbf{【X线诊断】}  幽门管溃疡。

\textbf{【评  述】}
 本病为常见病,发病机制不甚明了,好发年龄为20~50岁。胃溃疡常单发,多在小弯及胃角处,其次为胃窦部,其他部位少见。病理改变主要为胃壁溃烂缺损,形成壁龛。溃疡先从粘膜开始并逐渐侵及粘膜下层,常深达肌层。X线检查是胃溃疡的重要检查方法,尤其是气钡双重造影,可显示小而表浅的溃疡。溃疡病的X线表现,可分为直接与间接征象,前者为X线诊断的主要依据。

1.直接征象 龛影为溃疡充钡后在X线上的反映,是溃疡的直接征象。在正位像上,呈圆形或类圆形影;如果溃疡内的钡剂较少,仅四周壁附薄层钡剂,则呈环形,即所谓环形龛影。在切线位上,龛影突出于胃轮廓之外,多呈乳头状,或为半圆形及锥形。边缘光滑整齐,底部平整。在切线位上还可显示:①粘膜线:为溃疡口部宽1~2mm的透光线影,见于口部的上缘、下缘或横贯整个口部。②狭颈征:龛影口部明显狭小,使龛影犹如一个狭长的颈。③项圈征:龛影口部的透明带,宽0.5~1cm,犹如一项圈。粘膜线、狭颈征、项圈征皆为溃疡周围炎性水肿所致。④粘膜纠集,溃疡周围的粘膜皱襞,因瘢痕收缩向壁龛均匀性纠集,直达龛影,呈星芒状。

2.间接征象 下述X线表现常见于胃溃疡,也可因胃癌所致,不具有特异性,但在综合分析时有一定价值。①胃小弯短缩:是小弯侧溃疡纤维组织增生,牵拉幽门及贲门靠近。②胃大弯侧指状切迹:胃小弯侧溃疡,因环形肌痉挛性收缩,在溃疡的对侧可见一指状切迹,立位时明显。③幽门梗阻及狭窄:幽门及其邻近部的溃疡可致幽门持久性痉挛,或因瘢痕形成而使幽门梗阻。X线可见空腹胃潴留液增多,幽门管狭小,钡剂通过困难。④胃液分泌增多:在无幽门梗阻的情况下,出现少至中等量的胃内空腹潴留液,使钡剂不易附着于胃壁而难以显示粘膜皱襞。⑤胃蠕动的变化:蠕动增强或减弱,张力增高或降低,排空加速或延缓。⑥局限性压痛:龛影部位常有明显的局限性压痛。

\subsection{穿透性溃疡}

\begin{figure}[!htbp]
 \centering
 \includegraphics{./images/Image00273.jpg}
 \captionsetup{justification=centering}
 \caption{胃角穿透性溃疡}
 \label{fig5-3-12}
  \end{figure} 

\textbf{【病史摘要】}
 男性,48岁。胃溃疡病史3年,近一周来上腹部疼痛加剧伴恶心、呕吐。体格检查:贫血貌,上腹部拒按,压痛明显,心、肺阴性。

\textbf{【X线表现】}
 上消化道钡餐检查示:胃小弯侧腔外见一1.8cm×2.0cm大小囊袋状影,轮廓尚光整,颈部狭长,狭颈征明显。

\textbf{【X线诊断】}  胃角穿透性溃疡。

\textbf{【评  述】}
 本例患者经手术证实为穿透性溃疡。穿透性溃疡为胃溃疡的特殊类型,其特点为龛影大而深,其深度与大小均超过1.0cm,形如囊袋状,狭颈征显著。需注意此征象与较大的胃憩室相鉴别:胃憩室发生部位以胃底贲门区后壁为多见,憩室内可显示胃粘膜皱襞影;穿透性溃疡X线表现如周围较广泛的水肿带以及粘膜皱襞向溃疡口部纠集征象与胃憩室不同。胃溃疡根据以上典型表现,诊断一般不难。但有时因瘢痕组织的不规则增生或溃疡比较扁平者易与恶性溃疡混淆。良性溃疡和恶性溃疡的鉴别诊断,应从龛影的形态、溃疡的位置、溃疡的口部、周围粘膜皱襞的情况、邻近胃壁的柔软与蠕动等多方面综合分析,详见下表。

胃良、恶性溃疡的X线鉴别诊断

\includegraphics{./images/Image00274.jpg}

\subsection{胃平滑肌瘤}

\begin{figure}[!htbp]
 \centering
 \includegraphics{./images/Image00275.jpg}
 \captionsetup{justification=centering}
 \caption{胃平滑肌瘤}
 \label{fig5-3-13}
  \end{figure} 

\textbf{【病史摘要】}
 女性,42岁。吞咽困难半年余,无明显疼痛,无消瘦。体格检查:腹软,肝、脾未及,腹部无压痛,心、肺阴性。

\textbf{【X线表现】}
 上消化道钡餐造影示:胃贲门下部见一类圆形充盈缺损,边缘光滑清晰,中央部可见小钡斑。

\textbf{【X线诊断】}  胃平滑肌瘤。

\textbf{【评  述】}
 本例患者经手术治疗病理证实为胃平滑肌瘤。胃平滑肌瘤来源于中胚层组织,大多在5cm以下,可分为胃内型、胃壁型、胃外型。X线表现主要有:①胃内隆起性病变:正面呈圆形、椭圆形,位于大、小弯者显示其侧面像为半圆形。充盈像时呈边缘光滑的充盈缺损。②粘膜皱襞:肿瘤表面被附粘膜,可见粘膜皱襞通过肿物征象,粘膜被抬起形成桥形皱襞,或被推开形成粘膜皱襞的躲避、迂回征象。肿瘤较大时,皱襞受压变薄、变平以致消失。③中心凹陷:肿瘤表面,尤其顶部常形成小凹陷,造影时出现小钡斑,即所谓的中心性凹陷。④肿瘤触诊:平滑肌瘤较硬,压之无变形。⑤钙化:平滑肌瘤可发生钙化,X线检查可见钙化斑。⑥周围改变:平滑肌瘤对周围无浸润,胃轮廓无僵硬。⑦恶性变:平滑肌瘤可恶变成为肉瘤。一般肿瘤较大、形态不规则、中心溃疡大而深又不规则时,应考虑恶变的可能性。

\subsection{胃淋巴瘤}

\begin{figure}[!htbp]
 \centering
 \includegraphics{./images/Image00276.jpg}
 \captionsetup{justification=centering}
 \caption{胃淋巴瘤}
 \label{fig5-3-14}
  \end{figure} 

\textbf{【病史摘要】}
 女性,40岁。上腹部不适3个月,食欲不振、消瘦,近期出现低热。体格检查:腹软,肝、脾未及,腹部无压痛,心、肺阴性,右侧锁骨上窝触及一类圆形肿块。

\textbf{【X线表现】}
 上消化道钡餐造影示:胃底大弯侧见不规则充盈缺损影,粘膜皱襞不规则增粗,胃壁柔韧度减弱,胃蠕动及收缩存在。CT检查示:胃底部胃壁局限性增厚,增强后均匀中度强化,壁柔软。

\textbf{【X线诊断】}  胃淋巴瘤。

\textbf{【评  述】}
 本例患者经手术治疗病理证实为胃淋巴瘤。胃肠道是淋巴结外淋巴瘤的最多见部位,最好累及胃。胃淋巴瘤可以是全身淋巴瘤的一部分,也可以是唯一的原发部位,多见于非霍奇金淋巴瘤。按形态学分类为:肿块型、溃疡型、浸润型和结节型。好发部位是胃体小弯侧和后壁。临床表现有上腹部疼痛、消瘦及上腹部肿块,可伴有全身淋巴瘤的其他表现。胃淋巴瘤X线常表现为局限性或广泛浸润性表现,前者为粘膜皱襞不规则、粗大,胃壁柔韧度消失,位于胃窦时呈漏斗状狭窄;后者为巨大粘膜皱襞的改变,排列紊乱,胃腔缩窄或变形,但其缩窄与变形程度不及浸润型胃癌。胃淋巴瘤缺乏特征性的X线表现,因此常不易与胃癌及其他肿瘤鉴别。但如下特征有助于本病的诊断:病变虽然广泛,但胃蠕动与收缩仍然存在,胃部病灶明显但临床一般情况较好,胃粘膜较广泛增粗,形态比较固定,临床有其他部位淋巴瘤的表现。

\subsection{早期胃癌(Ⅰ型)}

\begin{figure}[!htbp]
 \centering
 \includegraphics{./images/Image00277.jpg}
 \captionsetup{justification=centering}
 \caption{早期胃癌(Ⅰ型)}
 \label{fig5-3-15}
  \end{figure} 

\textbf{【病史摘要】}
 男性,61岁。上腹部不适、食欲不振、嗳气、反酸近1年,无黑便,无呕吐。体格检查:腹软,肝、脾未及,上腹部轻压痛,心、肺阴性。

\textbf{【X线表现】}
 上消化道钡餐造影示:胃窦部见一椭圆形充盈缺损影,边缘尚光整,周围粘膜皱襞中断破坏,基底部较宽,表面尚平坦,未见明显糜烂点,局部胃壁稍僵硬。

\textbf{【X线诊断】}  早期胃癌(Ⅰ型)。

\textbf{【评  述】}
 本例患者经手术病理证实结果为胃窦部早期胃腺癌,未侵犯胃粘膜肌层,未见明显转移灶。患者X线表现示胃窦部隆起型病变,轮廓尚光整,边缘稍粗糙,周围粘膜皱襞见中断破坏,局部胃壁稍僵硬,故诊断为早期胃癌(Ⅰ型)。Ⅰ型早期胃癌即表面隆起型,为肿瘤向胃腔内突出高度超过周围粘膜的5mm。早期胃癌发展缓慢,短者1~2年,长着可10余年无明显变化。但一般隆起型发展较快,而溃疡型发展较慢。胃癌向深层侵犯较快,而在粘膜内浸润较慢。隆起型早期胃癌,大小不一,直径多大于2cm,圆形、类圆形或不规则形,边界清楚,基底部较宽,极个别者可有蒂。肿瘤表面粗糙,常伴出血及糜烂。值得注意的是,由于早期胃癌病变范围较小,故X线检查可发现其存在,但最终诊断需要密切结合内镜与活检结果方能明确。

\subsection{早期胃癌(Ⅱa型)}

\begin{figure}[!htbp]
 \centering
 \includegraphics{./images/Image00278.jpg}
 \captionsetup{justification=centering}
 \caption{胃窦部早期胃癌(Ⅱa型)}
 \label{fig5-3-16}
  \end{figure} 

\textbf{【病史摘要】}
 女性,41岁。上腹部不适伴嗳气、反酸、食欲减退1个月。体格检查:腹软,上腹部轻度压痛,未扪及包块,心、肺阴性。

\textbf{【X线表现】}
 上消化道钡餐造影示:胃窦部见一形态不规则的平盘状充盈缺损影,表面凹凸不平,见小钡斑(箭头)。

\textbf{【X线诊断】}  胃窦部早期胃癌(Ⅱa型)。

\textbf{【评  述】}
 本例经手术病理证实为胃窦部早期胃癌Ⅱa型。胃双对比造影可显示粘膜面的细微结构而对早期胃癌的诊断具有重要价值。早期胃癌的X线表现主要为:①Ⅰ型(隆起型):肿瘤与周围粘膜有明显的分界,形态多不规则,呈息肉状、分叶状、菜花状等。表面不光滑,因有表层坏死形成粘膜缺损,双对比造影可见不规则钡斑。其基底部与正常粘膜分界清楚,侧面观可为广基型、无蒂型、有蒂型,有蒂者肿瘤在2cm以上。②Ⅱa型(表面隆起型):肿瘤形态不规则,呈平坦的息肉状、花坛状、平盘状等。表面有不规则凹凸而显示为不规则钡斑,基底部多为广基型。③Ⅱb型(表面平坦型):双对比造影主要表现为胃小区的细微变化,如胃小区粗大、紊乱,呈不规则之颗粒状形态。④Ⅱc型(表面凹陷型):肿瘤表现为形态不规则之表浅溃疡,呈楔形、星芒状等,边缘清楚、锐利,病变一般较小,病变周围伴有粘膜皱襞纠集现象,其粘膜皱襞尖端有明显的病理变形,如杵状增粗、笔尖样变细、阶梯状变薄、皱襞融合等。⑤Ⅲ型(凹陷型):为一深溃疡,其深溃疡本身不是癌,只于溃疡口边缘有癌浸润。X线表现其深溃疡形态很像良性溃疡,难以鉴别,只于溃疡口边缘显示轻微毛糙不平为其特征。由于早期胃癌的病变范围较小,因而X线双重造影检查的重点在于发现它的存在,最后的诊断需要密切结合内镜与活检方能明确。

\subsection{早期胃癌(Ⅱc型)}

\begin{figure}[!htbp]
 \centering
 \includegraphics{./images/Image00279.jpg}
 \captionsetup{justification=centering}
 \caption{胃窦小弯侧早期胃癌(Ⅱc型)}
 \label{fig5-3-17}
  \end{figure} 

\textbf{【病史摘要】}
 男性,45岁。上腹部不适、疼痛3个月余,近1个月疼痛加重。体格检查:腹软,剑突下压痛,未扪及包块,心、肺阴性。

\textbf{【X线表现】}
 上消化道钡餐造影示:胃窦小弯侧见一小不规则钡斑,表面凹凸不平,周围粘膜皱襞纠集。

\textbf{【X线诊断】}  胃窦小弯侧早期胃癌(Ⅱc型)。

\textbf{【评  述】}
 本例经手术病理证实为胃窦部小弯侧粘膜下癌,未突破粘膜肌层。胃癌是我国最常见的恶性肿瘤之一,好发年龄为40~60岁,可发生在胃的任何部位,但以胃窦、胃小弯及贲门区常见。目前,国内外均采用日本内镜学会提出的早期胃癌的定义及分型。早期胃癌是指癌限于粘膜及粘膜下层,而不论其大小或有无转移。依据肉眼形态分为三个基本型与三个亚型:Ⅰ型,隆起型,癌肿隆起高度大于5mm,呈息肉状。Ⅱ型,浅表型,癌灶比较平坦,不形成明显隆起或凹陷。本型根据其癌灶凹凸程度不同又分三个亚型:Ⅱa型,浅表隆起型,癌灶隆起高度小于5mm。Ⅱb型,浅表平坦型,与周围粘膜几乎同高,无隆起或凹陷。Ⅱc型,浅表凹陷型,癌灶凹陷深度小于5mm。Ⅲ型,凹陷型,癌灶深度大于5mm,形成溃疡,癌组织不超过粘膜下层。除上述三型外,尚有混合型。根据胃窦小弯侧不规则形表浅凹陷形成边缘粗糙的钡斑,其周围粘膜皱襞纠集呈杵状增粗和融合,拟诊断为早期胃癌,浅表凹陷型(Ⅱc型)。需要鉴别的是良性溃疡病变,其钡斑密度均匀、边缘光整,多呈圆形、椭圆形,溃疡周围粘膜皱襞纠集一般比较均匀规则,呈自远而近逐渐变细,与癌可形成鲜明的对照。

\subsection{早期胃癌(Ⅱa+Ⅱc型)}

\begin{figure}[!htbp]
 \centering
 \includegraphics{./images/Image00280.jpg}
 \captionsetup{justification=centering}
 \caption{胃体部早期胃癌(Ⅱa+Ⅱc型)}
 \label{fig5-3-18}
  \end{figure} 

\textbf{【病史摘要】}
 女性,45岁。上腹部疼痛不适伴嗳气、反酸5个月,近1个月疼痛加重,食欲减退。体格检查:腹软,上腹部压痛,未扪及包块,心、肺阴性。

\textbf{【X线表现】}
 上消化道钡餐造影示:胃体上部见隆起型小充盈缺损影,边缘欠光整,表面凹凸不平,见不规则钡斑影,周围粘膜皱襞中断破坏,胃小弯侧上段胃壁僵硬。

\textbf{【X线诊断】}  胃体部早期胃癌(Ⅱa+Ⅱc型)。

\textbf{【评  述】}
 本例经手术病理证实为胃体上部早期胃腺癌(Ⅱa+Ⅱc型),粘膜肌层未侵犯,周围淋巴结未见转移。早期胃癌除上述三个基本型及亚型外,病灶若具有两种形态者,称之为混合型,一般表述时将占优势的一型记录在前,如本例为Ⅱa+Ⅱc型,表示隆起型病灶的中央存在糜烂的深凹陷。

\subsection{胃癌(息肉型)}

\begin{figure}[!htbp]
 \centering
 \includegraphics{./images/Image00281.jpg}
 \captionsetup{justification=centering}
 \caption{胃窦部胃癌(息肉型)}
 \label{fig5-3-19}
  \end{figure} 

\textbf{【病史摘要】}
 男性,52岁。上腹部疼痛2年,无节律性,近期疼痛加重。体格检查:上腹部压痛,未扪及明显包块,腹软,肝、脾未及,心、肺阴性。

\textbf{【X线表现】}
 上消化道钡餐造影示:胃窦部近小弯侧见一充盈缺损,呈分叶状,轮廓欠光整,周围粘膜皱襞中断破坏,局部胃壁较僵硬,十二指肠水平段见囊袋状影。

\textbf{【X线诊断】}  胃窦部胃癌(息肉型);十二指肠水平段憩室。

\textbf{【评  述】}
 本例患者经手术病理证实为胃体近小弯侧胃腺癌,侵犯胃粘膜肌层。息肉型胃癌为常见病,好发于胃窦部,其次是胃底。早期癌肿突向胃腔,高约5mm,轮廓大多不规则,可广基底或呈狭蒂。中后期,癌肿进一步增大,表面高低不平如菜花样,与胃壁边界明确。临床多见于40岁以上男性,早期无症状,或类似溃疡病的症状。中后期,症状加剧,有中上腹痛,上消化道出血,扪及肿块,以及癌肿所在部位所产生的一些继发症状,如梗阻、呕吐等。

X线特点主要为:早期隆起型胃癌在适当加压或双重造影检查时,可见小的轮廓不规则的充盈缺损。至中晚期,一般钡餐检查,即可显示出轮廓不光整的充盈缺损,基底广、边界明确、直径3~4cm以上。缺损区邻近粘膜纹中断、破坏,胃壁僵硬。早期隆起型胃癌主要应与胃内良性肿瘤相鉴别,胃癌充盈缺损边缘不光整,粘膜破坏,胃壁僵硬。中晚期胃癌应与胃内其他恶性肿瘤相鉴别。早期隆起型胃癌如果病灶较小,常规钡餐检查容易漏诊,应注意适当加压方能显示出病灶。内镜检查有利于发现早期病灶,并能提供病理依据,便于明确诊断。

\subsection{贲门癌}

\begin{figure}[!htbp]
 \centering
 \includegraphics{./images/Image00282.jpg}
 \captionsetup{justification=centering}
 \caption{贲门癌}
 \label{fig5-3-20}
  \end{figure} 

\textbf{【病史摘要】}
 男性,56岁。上腹部疼痛、吞咽不适伴嗳气、反酸半年余,近1个月进食干性食物时吞咽困难加重伴呕吐。体格检查:腹软,上腹部轻度压痛,未扪及包块,心、肺阴性。

\textbf{【X线表现】}
 上消化道钡餐造影示:胃底贲门区见类圆形充盈缺损影,边缘毛糙,轻度分叶,周围粘膜皱襞破坏,胃体小弯侧上方胃壁僵硬。

\textbf{【X线诊断】}  贲门癌侵及胃体小弯上段(进展期)。

\textbf{【评  述】}
 本例患者经手术病理证实为胃底贲门腺癌,侵及胃体小弯侧上段。贲门癌为源于贲门中心周围2.0~2.5cm以内的胃癌。由于其位置比较特殊而易漏诊,主要原因为贲门区位于肋弓内不能触及肿块,贲门胃底部粘膜皱襞粗大使较小的病变难以识别。因此,检查贲门癌时应采用气钡双重造影,产气量越大越可形成良好的对比。一般先于立位观察,再采用仰卧、俯卧及左右斜位观察以免漏诊。

贲门癌的典型X线征象为:①贲门区肿物,可位于贲门开口上方或下方。②钡剂通过贲门时受阻,或在肿瘤之上绕过形成钡剂分流现象,有时呈喷射状入胃。③胃底增厚,呈多个弧形影,胃底与膈面距离加大(>1.5cm有诊断价值)。④贲门下方之胃小弯胃壁僵硬。⑤可合并龛影及出现环堤征。⑥食管下段受侵犯,出现狭窄、僵硬、变形等。

需要鉴别的是贲门失迟缓症,后者X线表现的食管下端狭窄对称、边缘光滑、壁柔软,管腔大小可变,腔内可见细而平行的粘膜皱襞,特别是无贲门癌胃泡内组织块影是鉴别要点。而发生于胃底的平滑肌瘤和平滑肌肉瘤,胃泡内也可见轮廓光整或分叶状软组织肿块影,但两者X线不仅有腔内的软组织肿块影,而且向胃腔外生长,还有胃壁改变,腔外较大肿块可推压邻近器官,再者平滑肌瘤与平滑肌肉瘤很少有侵犯食管下端,这些都与贲门癌不同。贲门区解剖结构特殊,发生于此的溃疡、静脉曲张及其他良恶性肿瘤X线表现与贲门癌有时差异也不显著,需密切结合临床病史,必要时做胃镜协助诊断。

\subsection{胃窦癌}

\begin{figure}[!htbp]
 \centering
 \includegraphics{./images/Image00283.jpg}
 \captionsetup{justification=centering}
 \caption{胃窦癌}
 \label{fig5-3-21}
  \end{figure} 

\textbf{【病史摘要】}
 女性,50岁。上腹部饱胀、疼痛2年,近2个月来疼痛加重,伴恶心、呕吐。体格检查:消瘦贫血貌,上腹部压痛并触及固定包块,心、肺阴性。

\textbf{【X线表现】}
 上消化道钡餐造影示:胃窦部狭窄,胃窦大弯侧见不规则充盈缺损,呈现肩胛征,胃窦部粘膜皱襞破坏紊乱,胃壁僵硬蠕动消失,胃内粘液潴留较多。

\textbf{【X线诊断】}  胃窦癌(进展期)。

\textbf{【评  述】}
 本例患者经手术病理证实为胃窦癌。胃窦部为胃癌另一好发部位,易发生浸润型胃癌,极易引起胃窦狭窄,狭窄的胃窦呈漏斗状或山峰状,出现肩胛征或袖口征,前者指狭窄的胃窦与其近端舒张的胃壁相连处呈肩胛状,后者则表现为狭窄近端随蠕动推进套在僵硬段上呈袖口状。此外,胃窦癌易于侵犯幽门而形成幽门梗阻,致胃排空延迟、胃残留物及滞留液增多,故必须做好检查前准备,清除和减少胃内滞留物。通常采用延长禁食时间和插胃管洗胃等方法,尚可使用辅助药物或针刺来改变胃窦张力和蠕动,以利于清晰显示狭窄段情况。胃窦癌须注意与胃窦炎或溃疡引起的良性狭窄相鉴别。鉴别的要点为良性狭窄病变段与正常胃分界呈渐进性,狭窄形态可变,可以收缩与扩张,粘膜皱襞存在、排列不整齐,其与胃窦癌形成的胃窦狭窄X线征象截然不同,鉴别不难。

\subsection{溃疡型胃癌}

\begin{figure}[!htbp]
 \centering
 \includegraphics{./images/Image00284.jpg}
 \captionsetup{justification=centering}
 \caption{胃小弯侧溃疡型胃癌}
 \label{fig5-3-22}
  \end{figure} 

\textbf{【病史摘要】}
 男性,65岁。上腹部疼痛不适2年,近1个月来疼痛加剧,消瘦明显。体格检查:消耗面容,上腹部可触及固定硬质包块,压痛明显,心、肺阴性。

\textbf{【X线表现】}
 上消化道钡餐造影示:胃角处胃腔内见一不规则龛影,龛影周围显示有不规则透亮环堤,其内可见指压迹征和裂隙征,局部胃壁僵硬,蠕动消失。

\textbf{【X线诊断】}  胃小弯侧溃疡型胃癌(进展期)。

\textbf{【评  述】}
 本例患者经手术后病理证实为胃小弯侧腺癌。溃疡型胃癌是进展期胃癌中的最多见类型,其X线表现以壁龛及邻近胃壁变化为主要表现。龛影多数较浅而大,形态多不规则,具有特征性的为口部指压迹征和裂隙征,与良性溃疡平坦的口部出现的狭颈征、项圈征对比分明。切线位显示龛影位于胃轮廓线以内或与之相平。龛影周围一圈不规则充盈缺损为环堤,环堤大小不一,高低不平,与正常胃壁界限清楚,其病理基础为癌肿破溃后留下的一圈隆起的边缘。若龛影骑跨于胃小弯前后壁,与周围的半弧形环堤构成了半月综合征。邻近粘膜皱襞亦可有聚拢表现,但近环堤处粘膜中断且有指压迹改变。

\subsection{浸润型胃癌}

\begin{figure}[!htbp]
 \centering
 \includegraphics{./images/Image00285.jpg}
 \captionsetup{justification=centering}
 \caption{浸润型胃癌}
 \label{fig5-3-23}
  \end{figure} 

\textbf{【病史摘要】}
 女性,48岁。上腹部疼痛2年,嗳气、反酸,近1个月来疼痛加剧,出现黑便,食欲减退。体格检查:消瘦,肝、脾未及,腹壁紧张,全腹压痛,心、肺阴性。

\textbf{【X线表现】}
 上消化道钡餐造影示:全胃胃腔缩小,胃壁僵硬,无蠕动,粘膜皱襞增粗、紊乱,呈脑回状改变,形态固定不变。

\textbf{【X线诊断】}  广泛浸润型胃癌(皮革胃)。

\textbf{【评  述】}
 本例患者经手术病理证实为高度恶性晚期进展期胃癌,X线表现呈皮革状胃,癌肿全胃广泛浸润。浸润型胃癌根据癌肿浸润范围不同,X线表现可分为局限浸润型和弥漫浸润型。局限型可发生于胃的任何部位,以胃窦部多见。X线表现为:癌肿浸润胃壁全周或半周时,胃腔显示为局限性、固定性狭窄与僵硬,严重时可呈管状、漏斗状狭窄。胃壁局限性僵硬、蠕动消失。胃粘膜皱襞增粗、紊乱,部分呈脑回状,形态固定不变。广泛型为癌肿浸润胃大部或全部,胃腔明显缩小,粘膜皱襞平坦、消失,胃壁僵硬、蠕动消失,犹如皮革囊状,称皮革胃。因幽门受侵而失去正常功能,由于钡剂的重力作用,可见造影时幽门处于开放状态,有造影剂源源不断地进入十二指肠。浸润型胃癌有时应和胃淋巴瘤相鉴别。胃淋巴瘤主要为粘膜下浸润性生长,肌层较少受浸润,且又无明显的纤维组织增生,故胃壁虽可增厚,但胃腔缩小一般不明显,胃壁僵硬也不显著,往往可见有蠕动波为鉴别之要点。

\subsection{残胃癌}

\begin{figure}[!htbp]
 \centering
 \includegraphics{./images/Image00286.jpg}
 \captionsetup{justification=centering}
 \caption{残胃癌}
 \label{fig5-3-24}
  \end{figure} 

\textbf{【病史摘要】}
 男性,65岁。5年前因溃疡型胃癌行胃大部分切除术、Billroth
Ⅰ式吻合术,近期上腹部疼痛加重伴嗳气、反酸,并有恶心、呕吐。体格检查:消瘦,皮肤、巩膜无黄染,左上腹部压痛明显,肝、脾无增大,心、肺阴性。

\textbf{【X线表现】}
 上消化道钡餐造影示:残胃吻合口下部小弯侧见不规则龛影,龛影周围显示有不规则透亮环堤,其内可见指压迹征,局部胃壁僵硬,蠕动消失。

\textbf{【X线诊断】}  残胃癌。

\textbf{【评  述】}
 本例患者经手术病理证实为残胃癌。残胃癌的诊断标准不一,国外多主张因良性疾患行胃部分切除术、胃肠吻合术后3年以上,残胃生癌者为残胃癌。我国主张良性胃疾患胃部分切除术后3年以上,胃癌行部分切除术后5年以上,残胃生癌者称残胃癌。国内外材料一致认为,胃空肠吻合术后残胃癌发生率高于胃十二指肠吻合术者。残胃癌病因不明,可能与碱性肠液刺激及术后引起吻合口慢性刺激有关。早期残胃癌临床症状不具特征性。中晚期残胃癌常见症状是中上腹疼痛、食欲减退和出血。残胃癌好发于胃残端部,其次为贲门区和大小弯前后壁交界部。因术后粘连及变形,残胃癌的X线诊断困难。常见的表现为吻合口狭窄、排空迟缓、残胃扩张;胃腔狭窄变形,胃壁僵直丧失舒缩功能;粘膜破坏;充盈缺损或不规则龛影等。残胃癌应与炎症性粘膜肿胀、缝线引起的异物反应或肉芽肿等鉴别,手术缝合时,可使小弯侧结节状改变也应注意,鉴别困难时,应结合纤维胃镜检查及组织学检查。

\section{十二指肠病变}

\subsection{十二指肠球部溃疡}

\begin{figure}[!htbp]
 \centering
 \includegraphics{./images/Image00287.jpg}
 \captionsetup{justification=centering}
 \caption{十二指肠球部溃疡}
 \label{fig5-4-1}
  \end{figure} 

\textbf{【病史摘要】}
 男性,55岁。上腹部节律性疼痛伴反酸2年,餐后疼痛缓解。体格检查:上腹部剑突下压痛,肝、脾无肿大,心、肺阴性。

\textbf{【X线表现】}
 上消化道钡餐造影示:十二指肠球基底部见一龛影,黄豆大小,边缘光整,周围粘膜纠集,球部变形。

\textbf{【X线诊断】}  十二指肠球部溃疡。

\textbf{【评  述】}
 本例经胃镜证实为十二指肠球部溃疡。十二指肠溃疡为常见病,其发生率高于胃溃疡。十二指肠溃疡好发于青壮年,40岁以下占80%。男性多于女性。十二指肠溃疡病因复杂,尚未完全阐明。十二指肠溃疡85%发生于球部,其次在球后部。发生于球部者,前壁较多,占50%,其次为后壁及球部的大小弯。单发为主,也可多发。临床主要征象为周期性、节律性右上腹痛,多在餐后3~4小时发生,进餐后可缓解。

X线表现主要有:①球部龛影,为球部溃疡的直接征象,正面观呈圆形或椭圆形,少数呈线状,需双重造影显示。充盈加压时,溃疡周围的水肿、增生表现为外缘模糊的透光带。切线位上龛影突出于轮廓线以外,呈锥形或乳头状,以充盈像显示较好。②球部变形,多由溃疡所致,少数可因胆系或胰腺等邻近脏器疾病所致,因此发现球部变形时需除外其他原因后方可诊断溃疡。球部呈二叶状、山字形、花瓣状畸形为瘢痕收缩的结果,球部的大、小弯侧可见袋状突出,称假性憩室,也因瘢痕收缩所致。球部整体性痉挛及严重的瘢痕收缩皆可致明显缩窄,此时常伴幽门梗阻,平滑肌松弛剂的应用有助于痉挛与瘢痕挛缩的鉴别。③激惹征,表现为钡剂迅速经过球部而不能满意充盈,为炎症刺激所致。

\subsection{十二指肠复合性溃疡}

\begin{figure}[!htbp]
 \centering
 \includegraphics{./images/Image00288.jpg}
 \captionsetup{justification=centering}
 \caption{十二指肠复合性溃疡}
 \label{fig5-4-2}
  \end{figure} 

\textbf{【病史摘要】}
 男性,35岁。上腹部不适、嗳气、反酸1年,近日夜间疼痛加重,饥饿时加重,进食后缓解。体格检查:上腹部剑突下压痛,肝、脾无增大,心、肺阴性。

\textbf{【X线表现】}
 上消化道钡餐造影示:十二指肠球部及球后部见大小不等龛影,球部变形呈二叶形,十二指肠上曲狭窄,钡剂下行受阻。

\textbf{【X线诊断】}  十二指肠复合性溃疡。

\textbf{【评  述】}
 本例患者经胃镜检查证实为十二指肠复合性溃疡,即十二指肠球部及球后部溃疡。十二指肠球后部是指球部与降部之间的肠管。该部溃疡以龛影为主,可合并局限性偏心性狭窄,十二指肠激惹征较为明显,局部压痛可同时存在。由于球部的重叠,十二指肠球后部溃疡较难显示。检查的方法包括以下三方面:①常规钡餐,利用各种体位及结合加压来显示病变,可以较好地了解球部充盈及排空情况以及十二指肠蠕动状况。②低张气钡造影,能更清晰地显示细微结构及病变情况。③内镜应用,钡餐结合内镜所见来提高诊断率已经成为一种必不可少的手段。

\subsection{肠系膜上动脉压迫综合征}

\begin{figure}[!htbp]
 \centering
 \includegraphics{./images/Image00289.jpg}
 \captionsetup{justification=centering}
 \caption{肠系膜上动脉压迫综合征}
 \label{fig5-4-3}
  \end{figure} 

\textbf{【病史摘要】}
 女性,35岁。进食后上腹部饱胀恶心、呕吐,俯卧位后症状缓解。体格检查:瘦长体形,肝、脾未及,腹部无明显压痛,心、肺阴性。

\textbf{【X线表现】}
 上消化道钡餐造影示:无力型胃,胃角位于髂嵴连线下方2.5cm左右,蠕动缓慢。十二指肠水平段钡剂受阻,水平段以上肠管扩张,蠕动亢进,并见逆蠕动发生,受阻处十二指肠见管状压迹。

\textbf{【X线诊断】}  肠系膜上动脉压迫综合征;胃下垂。

\textbf{【评  述】}
 本例患者经腹部CT扫描及CTA检查,示肠系膜上动脉自腹主动脉分出后,夹角过小,压迫十二指肠水平段,引起十二指肠郁积,故确诊为肠系膜上动脉压迫综合征。肠系膜上动脉压迫综合征多见于中年体弱和瘦长体形者,女性多于男性。其主要原因为肠系膜上动脉根部紧张度增强或先天性原因使肠系膜上动脉与腹主动脉间夹角变小,引起十二指肠水平段受压,使受压部以上肠管扩张而出现郁积。临床上主要表现为食后上腹部饱胀、恶心、呕吐,且呕吐物中带有胆汁,俯卧位时症状缓解或消失。X线表现主要是立位检查时钡剂通过十二指肠水平段受阻,十二指肠降段以上肠管扩张,蠕动亢进,可见钡剂如钟摆样来回运动,水平段受压处有一光滑整齐的纵形压迹,称为笔杆状压迹,使肠管紧贴脊柱,粘膜变平,当患者俯卧位时,该压迹可消失。诊断本病时应谨慎,因正常瘦长体形的人也可出现十二指肠水平段钡剂暂时性停留和少量逆蠕动,但无肠管扩张及胃排空延迟。此外,还需与器质性病变所致的梗阻相鉴别,若梗阻端形态显示良好,鉴别应无困难。

\subsection{十二指肠憩室}

\begin{figure}[!htbp]
 \centering
 \includegraphics{./images/Image00290.jpg}
 \captionsetup{justification=centering}
 \caption{十二指肠降部憩室}
 \label{fig5-4-4}
  \end{figure} 

\textbf{【病史摘要】}
 男性,45岁。上腹部疼痛不适月余。体格检查:上腹部剑突下压痛,肝、脾无增大,心、肺阴性。

\textbf{【X线表现】}
 上消化道钡餐造影示:胃窦部粘膜皱襞增粗、紊乱,胃壁柔软,十二指肠降部见一囊袋状影突出于肠壁,内见钡剂充盈,可见十二指肠粘膜纹理伸入其中。

\textbf{【X线诊断】}  胃窦炎;十二指肠降部憩室。

\textbf{【评  述】}
 本病比较常见,大多数患者无明显症状,多见于中老年人。发生部位多位于降段内后壁,其次为十二指肠水平段。若合并憩室炎可引起糜烂、溃疡和出血,壶腹部附近憩室尚可引起胆管炎或胰腺炎等。十二指肠憩室发生的原因可能与肠壁生长发育过程中的局部缺陷与薄弱有关,随着年龄增长而加剧退变,在肠内压异常增加或肠肌收缩不协调时,薄弱点向腔外突出而形成憩室。十二指肠憩室X线表现主要是充钡后憩室呈圆形、椭圆形或三角形囊袋状突出物,轮廓光整,颈部较狭窄,并可见十二指肠粘膜皱襞伸入其中。憩室大小不一,较大者立位可见囊内气、液、钡分层现象,较小者可呈短管状,一般钡透不易发现,需行低张气钡双重造影才不至于漏诊。憩室轮廓不规则、压痛、邻近十二指肠有肠激惹征象者应考虑合并憩室炎。此外,憩室尚需与溃疡鉴别,后者常伴有狭窄痉挛,龛影内无粘膜皱襞伸入。

\subsection{十二指肠腺瘤}

\begin{figure}[!htbp]
 \centering
 \includegraphics{./images/Image00291.jpg}
 \captionsetup{justification=centering}
 \caption{十二指肠腺瘤}
 \label{fig5-4-5}
  \end{figure} 

\textbf{【病史摘要】}
 男性,45岁。上腹部疼痛不适,有嗳气、反酸。体格检查:上腹部剑突下压痛,肝、脾无增大,心、肺阴性。

\textbf{【X线表现】}
 上消化道钡餐造影示:十二指肠降部下段外侧可见分叶状充盈缺损(箭头),其基底部与肠壁形成切迹,肠壁略凹陷,周围粘膜皱襞正常,未见明显中断、破坏。

\textbf{【X线诊断】}  十二指肠降部腺瘤。

\textbf{【评  述】}
 本例经手术病理证实为十二指肠降部腺瘤。十二指肠良性肿瘤约占小肠良性肿瘤的20%。以腺瘤、平滑肌瘤、脂肪瘤多见。发生部位以球部最多,占50%以上,降部次之,升部最少。肿瘤多向肠腔内呈息肉状生长,少数向肠腔外生长。临床上多见于老年人,因肿瘤多较小而少有症状。食欲不振、恶心、上腹部疼痛及出血为常见症状。十二指肠腺瘤与消化道其他部位的腺瘤相似,X线表现为圆形、椭圆形或分叶状充盈缺损,边缘光滑,局部肠壁柔软,粘膜皱襞无破坏,一般以单发为多见,少数可多发,可带蒂,此时可见肿瘤随肠蠕动而移动。腺瘤多发时要与布氏腺增生鉴别。布氏腺增生比较罕见,多发生在球部,亦可延及降部。病因不明,通常认为是一种炎症。病理上有多发型和单发型两种,前者为广泛结节状粘膜增生,后者与单发腺瘤相似,可带蒂。X线表现为十二指肠球部粘膜紊乱,皱襞增粗,其中可见多枚黄豆或绿豆大小的充盈缺损,形态固定。单发者为单个充盈缺损,与腺瘤无法鉴别,通常十二指肠没有激惹和变形。

一般临床工作中,如发现十二指肠单发带蒂肿瘤应首先考虑腺瘤的诊断,而十二指肠多发结节状充盈缺损则应首先考虑布氏腺增生的诊断。最终诊断要结合内镜或手术病理诊断。

\subsection{十二指肠平滑肌瘤}

\begin{figure}[!htbp]
 \centering
 \includegraphics{./images/Image00292.jpg}
 \captionsetup{justification=centering}
 \caption{十二指肠平滑肌瘤}
 \label{fig5-4-6}
  \end{figure} 

\textbf{【病史摘要】}
 男性,35岁。上腹部疼痛不适月余,无嗳气、反酸,无恶心、呕吐。体格检查:上腹部无压痛,肝、脾无增大,心、肺阴性。

\textbf{【X线表现】}
 上消化道钡餐造影示:十二指肠下曲见类圆形充盈缺损影,边缘光整,内见小钡斑影,十二指肠腔未见狭窄,肠壁未见僵硬,蠕动正常。

\textbf{【X线诊断】}  十二指肠平滑肌瘤。

\textbf{【评  述】}
 本例患者经手术治疗病理证实为十二指肠平滑肌瘤。本例患者十二指肠病变X线征象符合良性肿瘤的表现,发生于十二指肠的良性肿瘤较少见,以平滑肌瘤及腺瘤多见。平滑肌瘤来源于中胚层组织。

X线表现主要为:①小肠局限性肿物,瘤体一般<5cm。②肿物呈球形或分叶状,周界规则,切线位上呈半圆形充盈缺损。③向腔内生长者,肿瘤一般体积都较大,无蒂,较固定,活动度差,局部管腔狭窄,可致肠梗阻;腔外生长者多无临床症状;如同时向腔内及腔外生长,尚可见肠管受压甚至移位。④瘤体中心因血供缺乏,往往容易发生坏死,出现龛影或表面糜烂,X线表现为钡斑。需要指出的是,平滑肌瘤和平滑肌肉瘤皆为粘膜下肿瘤,均具有粘膜下肿瘤的特征,故两者X线表现有时很相似,鉴别有一定难度。但平滑肌肉瘤瘤体体积常大于5cm,形态不规则,表面常凹凸不平,并且常较早出现肝脏、淋巴结转移,这有利于两者的鉴别诊断。

\subsection{十二指肠腺癌}

\begin{figure}[!htbp]
 \centering
 \includegraphics{./images/Image00293.jpg}
 \captionsetup{justification=centering}
 \caption{十二指肠降部腺癌}
 \label{fig5-4-7}
  \end{figure} 

\textbf{【病史摘要】}
 男性,65岁。上腹部疼痛不适伴呕吐3个月余,近期有黑便。体格检查:腹软,中上腹部压痛,肠肠鸣音正常,肝、脾未及,心、肺阴性。

\textbf{【X线表现】}
 上消化道钡餐造影示:十二指肠降部管腔明显环形狭窄,粘膜破坏,管壁僵硬,蠕动消失,近端肠管扩张。

\textbf{【X线诊断】}  十二指肠降部腺癌。

\textbf{【评  述】}
 本例经手术治疗病理证实为十二指肠降部粘液腺癌。十二指肠腺癌占小肠腺癌的40%~50%,好发于60~70岁,男女之比约为1.2∶1。按癌瘤发生的部位可分为乳头上部癌、乳头周围癌和乳头下部癌,其中以乳头周围癌最多见,约占65%,乳头上部癌约占20%,乳头下部癌约占15%。按肿瘤的大体形态可分为息肉型、浸润溃疡型、缩窄型和弥漫型。临床表现与肿瘤的类型及部位有关。主要症状有:上腹部隐痛、烧灼样痛或钝痛:酷似十二指肠溃疡,但进食及制酸药均不能缓解疼痛。黄疸:乳头周围癌75%~80%可发生黄疸。肠梗阻:息肉型或缩窄型癌容易导致肠腔狭窄或堵塞,导致部分或完全性十二指肠梗阻;乳头上部癌导致的完全性肠梗阻,呕吐物内不含胆汁,易被误诊为幽门梗阻。出血:十二指肠癌患者的大便隐血试验阳性者占60%~80%,出血明显者可有黑便,大出血时可发生呕血。腹块:右上腹出现肿块者占10%~25%。

根据肿瘤的X线表现可分为息肉型、溃疡型及浸润型:①息肉型:表现为息肉样隆起病变,形态不规则呈分叶状,粘膜破坏消失。肠腔可呈扩张状,钡剂分流,如果肿块较大可填塞十二指肠,钡剂受阻,近端肠腔扩张。同时也可伴有溃疡,肠壁僵硬等。②溃疡型:表现为粘膜破坏,出现不规则的腔内龛影,或部分腔内部分位于腔外。溃疡口部可有环堤、裂隙征及指压痕等恶性溃疡的征象。同时也可伴有局部肠壁僵硬,出现不规则的隆起性改变。③浸润型:X线表现为肠壁受到肿瘤浸润而僵硬,蠕动消失,肠腔狭窄,近端肠腔扩张,粘膜破坏,可伴有溃疡及不规则隆起性病变。本例患者为发生在十二指肠乳头上部的浸润型腺癌,X线表现比较明确,手术病理予以证实。但十二指肠癌如发生在乳头区则需与胰头癌相鉴别。十二指肠癌可推移相对正常的胰头或钓突结构向前内侧移位,肿块密度不均匀伴溃疡形成,十二指肠内外侧壁都呈不规则增厚和肠腔狭窄等有助于与胰头癌鉴别。但当肿瘤侵犯胰头时,两者的鉴别极为困难。

\subsection{十二指肠平滑肌肉瘤}

\begin{figure}[!htbp]
 \centering
 \includegraphics{./images/Image00294.jpg}
 \captionsetup{justification=centering}
 \caption{十二指肠平滑肌肉瘤}
 \label{fig5-4-8}
  \end{figure} 

\textbf{【病史摘要】}
 男性,55岁。上腹部疼痛半年余,无嗳气、反酸,无明显节律性。近期疼痛突然加剧,伴恶心、呕吐。体格检查:腹部拒按,上腹部触及包块,肝、脾未及,心、肺阴性。

\textbf{【X线表现】}
 上消化道钡餐造影示:十二指肠肠曲扩大,上曲内缘呈弧形压迹,肠腔伴有狭窄,其边缘皱襞有不规则破坏,并可见一不规则线状钡影呈水平状伸向十二指肠肠曲内,其上方见一小憩室。

\textbf{【X线诊断】}  十二指肠降部平滑肌肉瘤,肿瘤液化坏死与肠腔相通。

\textbf{【评  述】}
 本例患者经手术治疗病理证实为十二指肠降部平滑肌肉瘤,肿瘤液化坏死与肠腔相同。平滑肌肉瘤发生于十二指肠较少见。其病理变化与肿瘤生长方式有关,如肿瘤向肠腔内生长,呈半球状突入肠腔,可略有分叶,广基底,粘膜面糜烂或呈不规则溃疡,如肿瘤向肠腔外生长,则压迫十二指肠移位。临床可扪及腹块,质硬;可伴上消化道出血。

X线表现主要有:①腔内充盈缺损,略带有分叶改变,局部粘膜纹消失,可伴有不规则龛影,肠腔扩张,钡流改道。②肠腔被压迫移位,导致十二指肠肠曲变形,其形态改变依据肿瘤的部位和大小而定。上述两方面变化,有时混合存在。平滑肌瘤和平滑肌肉瘤皆为粘膜下肿瘤,均具有粘膜下肿瘤的特征,故两者X线表现有时很相似,鉴别有一定难度。但本例患者瘤体体积较大,形态不规则,表面凹凸不平,肿瘤液化坏死,并出现与肠腔相通的窦道,肠壁较僵硬,局部蠕动消失,与平滑肌瘤的表现相异,故诊断为平滑肌肉瘤。

\subsection{十二指肠淋巴瘤}

\begin{figure}[!htbp]
 \centering
 \includegraphics{./images/Image00295.jpg}
 \captionsetup{justification=centering}
 \caption{十二指肠淋巴瘤}
 \label{fig5-4-9}
  \end{figure} 

\textbf{【病史摘要】}
 男性,40岁。上腹部疼痛不适伴低热,无嗳气、反酸,偶有呕吐。体格检查:上中腹部压痛,中腹部似触及包块,肝、脾未及,心、肺阴性。

\textbf{【X线表现】}
 上消化道钡餐造影示:十二指肠降部中上段(箭头)肠腔狭窄,见不规则充盈缺损及小龛影,粘膜中断,肠壁略显僵硬。

\textbf{【X线诊断】}  十二指肠降部淋巴瘤。

\textbf{【评  述】}
 本例患者经内镜检查并经病理证实为十二指肠降部非霍奇金淋巴瘤。原发性十二指肠恶性淋巴瘤(primary
malignant lymphoma of
duodenum),是指原发于十二指肠肠壁淋巴组织的恶性肿瘤,原发性十二指肠恶性淋巴瘤好发于40岁左右,较其他恶性肿瘤发病年龄轻,男女发病之比为1:1~3:1。该病的临床表现无特异性,可因肿瘤的类型和部位而异,主要表现为上腹痛、腹块、弛张热等。病理巨检表现有浸润型、息肉型、溃疡型,可混合存在。

X线平片检查有时可显示十二指肠梗阻的X线表现,或软组织块影。胃肠道钡餐双重对比造影对十二指肠肿瘤的诊断准确率达42%~75%,其影像表现有:①十二指肠粘膜皱襞变形、破坏、消失,肠壁稍僵硬。②肠壁充盈缺损、龛影或环状狭窄。③肠管可有局限性囊样扩张,呈动脉瘤样改变。④肠壁增厚,肠管变小,呈多发性结节状狭窄。十二指肠低张造影,更有利于观察粘膜皱襞的细微改变,使其诊断准确率提高到93%左右。肠穿孔是本病的主要并发症,有15%~20%的十二指肠恶性淋巴瘤患者会发生肠穿孔,比其他恶性肿瘤发生率高。此多为肿瘤侵犯肠壁发生溃疡、肠坏死,或肿瘤继发感染而引致。本例患者十二指肠降部肠腔狭窄,肠壁见充盈缺损,内见小龛影,周围粘膜皱襞破坏,肠壁略显僵硬,故符合十二指肠淋巴瘤的诊断,最终需经内镜检查或手术行活检以获病理确诊。

\subsection{十二指肠类癌}

\begin{figure}[!htbp]
 \centering
 \includegraphics{./images/Image00296.jpg}
 \captionsetup{justification=centering}
 \caption{十二指肠类癌}
 \label{fig5-4-10}
  \end{figure} 

\textbf{【病史摘要】}
 男性,39岁。上腹部疼痛不适3个月余,无恶心、呕吐。体格检查:腹软,中腹部压痛,未触及包块,肝、脾未及,心、肺阴性。

\textbf{【X线表现】}
 上消化道钡餐造影示:十二指肠降部及水平部见不规则充盈缺损,局部粘膜破坏,肠壁稍僵硬,肠腔稍窄。

\textbf{【X线诊断】}  十二指肠占位:淋巴瘤?腺癌?类癌?

\textbf{【评  述】}
 本例患者经手术病理证实为十二指肠降部恶性神经内分泌癌,即类癌。十二指肠类癌是很特殊的一种类癌,好发部位依次为十二指肠第二段、第一段、第三段。年龄22~84岁,平均55岁。男女发病率差别不大。常合并Von
Recklinghausen's病、Zollinger-Ellison综合征和多发性内分泌肿瘤(MEN)。十二指肠和壶腹部还可发生杯状细胞类癌(腺类癌)和小细胞神经内分泌癌。杯状细胞类癌又称腺类癌或粘液类癌。主要病理改变为息肉状病变,大小不等,单发或多发,小的表现为粘膜下结节,大的则明显突向腔内,以致肠腔阻塞。局部常伴有腔外肿块。可合并有其他小肠、结肠或肺、气管类癌。临床症状无特征性,可扪及腹块。

类癌瘤较小时常被漏诊,发展到一定大小后,X线即可表现为:①病变范围内大小不等的结节状透亮区,或较大的充盈缺损,缺损区伴有肠腔外肿块,为本病重要表现之一。②病变区粘膜纹粗大,可伴有肠腔狭窄。③病变段肠曲固定,移动度消失。④病变可多发,同时可见于胃肠道的其他部分,甚至肺、支气管,亦有类癌瘤存在。类癌的鉴别诊断极为困难,因其表现多样化,仅凭影像学表现很难与其他肠道的良恶性肿瘤鉴别。以往主要依赖消化道钡剂造影,病变检出的阳性率较低。但CT广泛使用后,尤其是多层螺旋CT的使用大大提高了肿瘤的发现比例。CT不但可发现原发病灶,还可显示肿瘤对邻近组织的侵犯情况,观察肝脏的转移灶,肠系膜的侵犯,后腹膜及邻近淋巴结的转移。

\section{小肠病变}

\subsection{空肠憩室}

\begin{figure}[!htbp]
 \centering
 \includegraphics{./images/Image00297.jpg}
 \captionsetup{justification=centering}
 \caption{空肠憩室}
 \label{fig5-5-1}
  \end{figure} 

\textbf{【病史摘要】}
 男性,45岁。上腹部疼痛半年余伴腹胀。体格检查:腹软,腹部无明显压痛,未扪及包块,肝、脾未及,心、肺阴性。

\textbf{【X线表现】}
 全消化道钡餐造影示:空肠见一卵圆形袋状阴影,边缘整齐光滑,以宽窄不等的开口通向肠腔,内见钡剂进出。

\textbf{【X线诊断】}  空肠憩室。

\textbf{【评  述】}
 憩室是由于钡剂经过胃肠道管壁的薄弱区向外膨出形成的囊袋状影像,或是由于管腔外邻近组织病变的粘连、牵拉造成管壁全层向外突出的囊袋状影像,其内及附近的粘膜皱襞形态正常,称之为憩室。小肠憩室好发于上段空肠,少数在回肠。正常空肠上段的终末血管粗大,肠系膜缘血管进入处的肠壁结构较薄弱,容易成为憩室的好发部位。憩室可为单发,多为多发性,多个憩室集中于某段空肠。多发性憩室数目由2~40个不等;直径由数毫米到数厘米。憩室均沿小肠系膜侧肠壁终末血管区分布,形状呈圆形或卵圆形的袋状结构向肠壁外膨出,并以宽径或窄径基底部向肠腔开口。

小肠气钡双重造影检查憩室的X线表现主要有:显影的憩室在小肠系膜侧呈圆形或卵圆形袋状阴影,边缘整齐光滑,以宽窄不等的开口通向肠腔。较大的憩室腔内可显示气体、液体和钡剂的3层平面,如遇开口宽大的憩室可见造影剂在憩室和肠腔之间自由进出,此为本症特有的X线造影表现。小肠憩室发生憩室粘膜出血、憩室穿孔、气腹和小肠壁气囊肿或肠梗阻时,应与消化性溃疡出血及穿孔、机械性肠梗阻等相鉴别。

\subsection{小肠蛔虫症}

\begin{figure}[!htbp]
 \centering
 \includegraphics{./images/Image00298.jpg}
 \captionsetup{justification=centering}
 \caption{小肠蛔虫症}
 \label{fig5-5-2}
  \end{figure} 

\textbf{【病史摘要】}
 男性,35岁。中腹部疼痛不适1周。体格检查:腹软,腹部无压痛,肝、脾未及,心、肺阴性。

\textbf{【X线表现】}
 全消化道钡餐检查示:回肠内可见边缘光滑之细长条状弯曲的充盈缺损影,中央可见细线状钡影,周围粘膜皱襞正常。

\textbf{【X线诊断】}  小肠蛔虫症。

\textbf{【评  述】}
 本病相对少见。根据小肠肠腔内边缘光滑之细长条状弯曲的充盈缺损影,特别是其中央可见与充盈缺损纵轴相一致的细线状钡影,为钡剂进入虫体腔内所致,小肠蛔虫症的诊断可以确定。而小肠腔内的各类占位性及其他病变均不能表现出以上的X线形态特征。需要注意的是小肠蛔虫的X线检查应仔细,常常需要加压观察,尤其是位置隐蔽、虫体较小时容易漏诊。

\subsection{小肠Crohn病}

\begin{figure}[!htbp]
 \centering
 \includegraphics{./images/Image00299.jpg}
 \captionsetup{justification=centering}
 \caption{小肠Crohn病}
 \label{fig5-5-3}
  \end{figure} 

\textbf{【病史摘要】}
 男性,35岁。下腹部疼痛伴腹泻1年余,时有发热,近1个月疼痛加剧,食欲减退。体格检查:右下腹部压痛,未扪及包块,肝、脾未及,心、肺阴性。

\textbf{【X线表现】}
 全消化道钡餐造影示:回肠末端边缘不整,管壁略僵硬,边缘呈锯齿状改变,粘膜紊乱,内见卵石样或息肉样充盈缺损影。

\textbf{【X线诊断】}  小肠Crohn病。

\textbf{【评  述】}
 小肠Crohn病,又称克罗恩病、局限性肠炎、肉芽肿性肠炎。1932年由Crohn和Oppenheimer最早描述。病因不明。发病年龄呈双峰特征:15~30岁和55~80岁高发,女性比男性发病率高20%~30%。临床症状多样化,如腹痛、腹泻、便秘、肠梗阻、便血、低热、消瘦、贫血、胃肠外症状等。本病从口至肛门的全胃肠道的任何部位均可受累,病变呈跳跃式或节段性分布。小肠和结肠同时受累最为常见,占40%~60%;限于小肠,主要是末端回肠发病的占30%~40%。病理改变主要为:特征性肠系膜侧纵行线状溃疡;在纵横交错的溃疡之间出现粘膜隆起,形成卵石征;纤维化致肠壁增厚,肠腔狭窄;瘘管形成;周围淋巴结肿大。

X线表现主要为:早期为肠粘膜纹理增粗,甚至有卵石样充盈缺损,或锯齿状或尖刺状龛影,病变段肠管形态固定,蠕动不明显,肠间距增宽。后期则有不规则的线样狭窄,范围不一,多为1~2cm或更长,间断发病,可合并肠粘连或肠梗阻表现。

此病主要与小肠结核鉴别,两者的X线表现非常相似,有时区别十分困难。肠结核常伴有回盲瓣病变,因结核病变使回盲瓣变形、开放,造影剂自由通过,而Crohn病使回盲部形成狭窄,可助鉴别。

\subsection{小肠结核}

\begin{figure}[!htbp]
 \centering
 \includegraphics{./images/Image00300.jpg}
 \captionsetup{justification=centering}
 \caption{小肠结核}
 \label{fig5-5-4}
  \end{figure} 

\textbf{【病史摘要】}
 男性,45岁。右下腹疼痛、恶心、呕吐伴食欲减退半年余,近半个月出现腹泻伴发热。2年前有肺结核病史,经治疗呼吸道症状消失。体格检查:右下腹压痛,腹肌紧张,未扪及包块,肝、脾未及,心、肺阴性。

\textbf{【X线表现】}
 全消化道钡餐造影示:末端回肠狭窄伴瘘管形成,盲肠狭窄,盲肠及回肠末端上移靠拢形成一字征。

\textbf{【X线诊断】}  回盲部肠结核(溃疡型)。

\textbf{【评  述】}
 肠结核好发于回盲部,但也见于十二指肠、空肠和回肠。肠结核分为溃疡型和增殖型两型,溃疡型多见,也见两型同时存在。早期是肠壁集合淋巴结与Peyer氏淋巴丛肿胀,以后融合成干酪性病灶、粘膜破溃,形成与长轴垂直的溃疡;病变严重者,愈合后形成大量瘢痕组织引起肠腔环形狭窄。也有些病例在结核初期,就有肠壁粘膜下层的结核性肉芽组织增生与纤维化,从而粘膜面产生许多大小不一的隆起性结节,肠壁变硬,早期就有肠腔狭窄。本病常见的症状为腹痛、腹泻,或腹泻、便秘交替出现。右下腹块与不全性梗阻症状与体征。

X线特点主要有:①早期表现为受累肠曲有激惹现象,回肠末端可以始终不充盈,或呈细线状。②溃疡形成时可见肠管边缘呈锯齿状,或呈斑点状龛影。③增生显著者,则表现为回盲部粘膜增粗,犹如多发性、大小不一的息肉样充盈缺损,甚至类似于肿瘤样表现。④愈合后常遗有环形肠腔狭窄与狭窄上肠曲扩张。

本例患者回盲部X线表现结合患者有肺结核病史,溃疡型肠结核诊断明确。肠结核需与肿瘤、克罗恩病鉴别,增生型肠结核的病变多为移行性,多发性小息肉样充盈缺损,粘膜增粗、紊乱,激惹征,回盲瓣受累机会高,而与肿瘤不同。肠结核与克罗恩病鉴别困难,而克罗恩病常见的纵行溃疡以及对侧假性憩室样囊袋状膨出和周围卵石样充盈缺损、偶见瘘管形成与溃疡型肠结核表现不同。

\subsection{小肠腺瘤}

\begin{figure}[!htbp]
 \centering
 \includegraphics{./images/Image00301.jpg}
 \captionsetup{justification=centering}
 \caption{小肠腺瘤}
 \label{fig5-5-5}
  \end{figure} 

\textbf{【病史摘要】}
 男性,45岁。上腹部不适3个月,无嗳气、反酸,无恶心、呕吐。体格检查:腹软,无压痛,肝、脾未及,心、肺阴性。

\textbf{【X线表现】}
 全消化道钡餐造影示:空肠内见一椭圆形充盈缺损影,边缘光整,基底部见带蒂,周围粘膜皱襞未见异常,肠蠕动正常。

\textbf{【X线诊断】}  空肠占位,考虑腺瘤可能性大。

\textbf{【评  述】}
 本例患者经手术治疗病理证实为空肠息肉状腺瘤。小肠腺瘤是发生于小肠粘膜上皮或肠腺体上皮的良性肿瘤,体积小、带蒂,呈息肉样生长,故又称肠息肉。小肠腺瘤多发生于十二指肠和回肠,空肠较少。一般来自肠粘膜上皮或腺上皮,多向肠腔内突出,表面覆盖粘膜和粘膜下组织。

根据组织学结构小肠腺瘤有3种类型:①管状腺瘤,亦称腺瘤样息肉或息肉状腺瘤,以发生于十二指肠最多,多是单发,也可多发,此种腺瘤呈息肉状,大多有蒂。②绒毛状腺瘤,亦称乳头状腺瘤。较管状腺瘤少见,最多发生于十二指肠内,体积较管状腺瘤大。③混合性腺瘤。小肠容受性好,内容物常为液体,而且腺瘤一般生长较慢,故小肠腺瘤可在较长时间内无症状。小肠腺瘤X线表现多为腔内的圆形充盈缺损,大小不一,轮廓光整、边缘光滑,如有带蒂,则可以移动。扪之柔软,易变形。本例患者钡餐X线表现为空肠内椭圆形充盈缺损,边缘光整,基底部见带蒂,周围粘膜皱襞未见异常,肠蠕动正常,故考虑小肠腺瘤可能性大。本病需与增生型肠结核及小肠癌鉴别。增生型肠结核X线钡剂检查表现为回盲部粘膜增粗,犹如多发性、大小不一的息肉样充盈缺损,盲肠收缩上移,回肠末端与其靠拢形成的一字征为其特点,小肠腺瘤不具此征;小肠癌好发于十二指肠、空肠与回肠下段,多呈环形生长,X线表现可显示局限性的不规则环形狭窄及狭窄前扩张,局部粘膜纹理破坏与不规则的结节样充盈缺损,很少见有龛影,局部肠壁僵硬,可扪及肿块,此与小肠腺瘤X线表现不同。

\subsection{小肠平滑肌瘤}

\begin{figure}[!htbp]
 \centering
 \includegraphics{./images/Image00302.jpg}
 \captionsetup{justification=centering}
 \caption{空肠平滑肌瘤}
 \label{fig5-5-6}
  \end{figure} 

\textbf{【病史摘要】}
 女性,35岁。因右下腹痛伴腹胀半年余,时有恶心、呕吐。体格检查:右下腹可扪及鸡蛋大小包块,可移动,肝、脾未及,心、肺阴性。

\textbf{【X线表现】}
 全消化道钡餐造影示:空肠内见一分叶状软组织肿块,局部肠壁凹陷。空肠肠襻折曲成角,下方见光滑弧形压迹,粘膜皱襞未见异常。

\textbf{【X线诊断】}  空肠占位,空肠平滑肌瘤可能性大,小肠腺癌待排。

\textbf{【评  述】}
 本例患者经手术治疗病理证实为空肠平滑肌瘤。小肠平滑肌瘤是最常见的小肠良性肿瘤,源自小肠固有肌层,少数来自粘膜肌层,为一肠壁间肿瘤,在小肠良性肿瘤中其发病率居第二位,仅次于腺瘤。在小肠各段的分布中以空肠为最多,回肠次之,肿瘤多为单发,大小不一,常为圆形或椭圆形,有时呈分叶状或结节状。根据肿瘤在肠壁间的部位及其生长方式,可分为四种类型:腔内型、壁内型、腔外型、腔内外型。主要临床表现为消化道出血、腹痛、腹块、肠梗阻,及并发内瘘。

X线表现主要为:①边界清楚的圆形或结节样肿块。②脐样或牛眼样龛影。③肠管3字征。④粘膜部分消失、部分呈弧形或横形展开。⑤局部钡剂不同程度受阻;局部肠腔狭窄;肠管或周围器官受压移位;近端肠腔不同程度扩张。

小肠平滑肌瘤需与小肠腺癌鉴别。小肠腺癌X线表现为肠腔内不规则的分叶状或菜花状充盈缺损伴溃疡形成,周围粘膜中断、破坏,这些都与小肠平滑肌瘤X线表现有区别。本例患者空肠占位基底部较宽,虽肿块呈分叶状,但未见溃疡龛影,周围粘膜皱襞未见中断破坏,故首先考虑空肠平滑肌瘤可能性大。

\subsection{小肠淋巴瘤}

\begin{figure}[!htbp]
 \centering
 \includegraphics{./images/Image00303.jpg}
 \captionsetup{justification=centering}
 \caption{小肠淋巴瘤}
 \label{fig5-5-7}
  \end{figure} 

\textbf{【病史摘要】}
 女性,35岁。右下腹疼痛伴弛张热6个月余,近期恶心、呕吐、腹胀。体格检查:右下腹压痛,未触及包块,肝、脾未及,心、肺阴性。

\textbf{【X线表现】}
 全消化道钡餐造影示:局部回肠肠腔瘤样扩张,边缘凹凸不平,肠腔内见较大不规则充盈缺损,周围粘膜皱襞破坏、消失,肠壁稍僵硬。

\textbf{【X线诊断】}  回肠占位,回肠淋巴瘤可能性大,回肠腺癌待排。

\textbf{【评  述】}
 本例患者经手术治疗病理证实为小肠淋巴瘤,非霍奇金型。小肠淋巴瘤起源于粘膜下层淋巴组织,病变沿肠壁向纵深方向发展。向外侵及浆膜层、肠系膜及其淋巴结,向内浸润粘膜,使之变平、僵硬。肠腔可以狭窄,也可以因为肌间神经丛受损而发生麻痹性扩张,病变肠区范围可以较肿瘤为广,而且界限不明确。临床表现主要有:腹痛伴有恶心、呕吐,腹块,腹泻、腹胀。钡餐造影主要表现为:病变广泛,小肠正常粘膜皱襞大部分或全部消失,肠腔内可见到无数小的息肉样充盈缺损,肠腔宽窄不一,沿肠壁可见到锯齿状切迹。

小肠淋巴瘤X线表现无明显特征性,需与克罗恩病、肠结核以及小肠癌相鉴别。克罗恩病可有节段性狭窄、卵石征或假息肉的征象,有时难与恶性淋巴瘤相鉴别。但克罗恩病一般病史较长,可有腹部肿块,往往因局部炎症穿孔形成内瘘,钡剂检查可见内瘘病变,节段性狭窄较光滑,近段扩张较明显,线性溃疡靠肠系膜侧,并有粘膜集中,肠襻可聚拢,呈车轮样改变。小肠恶性淋巴瘤一般无内瘘形成,临床表现重,X线下狭窄段不呈节段性分布,边缘不光滑,结节大小不一,溃疡和空腔较大而不规则。增殖型小肠结核X线表现为单发或多发的局限性肠腔狭窄,边缘较恶性淋巴瘤光滑,近端扩张亦较明显;溃疡型小肠结核龛影一般与肠管纵轴垂直,恶性淋巴瘤的溃疡部位不定,龛影较大而不规则。小肠癌病变往往局限,很少能触及包块,即使有亦是较小的局限的包块,X线钡餐检查仅为一处局限性肠管狭窄、粘膜破坏,这与小肠淋巴瘤范围较广不同。

\subsection{小肠腺癌}

\begin{figure}[!htbp]
 \centering
 \includegraphics{./images/Image00304.jpg}
 \captionsetup{justification=centering}
 \caption{小肠腺癌}
 \label{fig5-5-8}
  \end{figure} 

\textbf{【病史摘要】}
 女性,71岁。右下腹痛、进行性消瘦1年余,近期有黑便。体格检查:消瘦,右下腹触及鸡蛋大小包块,质硬,无移动,压痛明显,肝、脾未及,心、肺阴性。

\textbf{【X线表现】}
 全消化道钡餐造影示:空肠近端不规则充盈缺损,周围粘膜皱襞破坏消失,管壁僵硬,蠕动消失。空肠近端管腔不规则狭窄,狭窄端以上肠管明显扩张。

\textbf{【X线诊断】}  空肠近端占位,小肠腺癌。

\textbf{【评  述】}
 本例患者经手术治疗病理证实为空肠腺癌,侵及浆膜层。小肠恶性肿瘤主要为腺癌,多见于回肠,其次为空肠。可分为息肉型、溃疡型、弥漫型、溃疡浸润型四型。临床表现主要为腹部肿块、腹痛、肠梗阻、消瘦、消化道出血。

X线钡餐造影主要表现为:①肿块型腺癌,肠腔内见不规则的分叶状或菜花状充盈缺损,并常可引起套叠,若有溃疡形成,则显示不规则腔内龛影。②浸润狭窄型腺癌,肠腔呈环形向心性狭窄,狭窄段的近、远侧两端有病变突出于肠腔内,使病变段肠腔呈苹果核样形态,核心则为癌溃疡。③病变近侧的肠腔常有不同程度的扩张,有时在病变的一端或两端可出现反压迹征,这是由于病变区肠管与其上下的正常肠管截然分界,钡剂不能通过病变区,此时蠕动频繁增强的正常肠管覆盖在肿块上而造成。④病变部位粘膜皱襞破坏消失,管壁僵硬,蠕动消失。本例患者空肠近端X线表现符合肿块型腺癌诊断,其与小肠良性肿瘤、平滑肌瘤、腺瘤等疾病形成的边界光滑整齐的充盈缺损表现不同。

X线诊断小肠腺癌需注意与淋巴瘤和平滑肌肉瘤相鉴别,淋巴瘤一般侵犯范围较广,肿瘤沿肠壁侵犯,也可侵犯肠系膜,系膜肿大淋巴结侵犯、压迫肠管形成狭窄,但不易引起梗阻,部分淋巴肉瘤局部肠管不狭窄反而扩张。而平滑肌肉瘤则生长迅速,一般瘤体较大,常伴有巨大溃疡,肿瘤呈肠外生长,附近肠曲受压推移,但也不易形成梗阻。

\subsection{小肠类癌}

\begin{figure}[!htbp]
 \centering
 \includegraphics{./images/Image00305.jpg}
 \captionsetup{justification=centering}
 \caption{小肠类癌}
 \label{fig5-5-9}
  \end{figure} 

\textbf{【病史摘要】}
 女性,65岁。左下腹疼痛不适,无恶心、呕吐,近期出现腹泻伴皮肤潮红。体格检查:下腹部压痛,触及包块,质稍硬,肝、脾未及,心、肺阴性,尿液检查:5-羟吲哚醋酸增高。

\textbf{【X线表现】}
 全消化道钡餐造影示:局部回肠狭窄,呈息肉样充盈缺损,周围粘膜皱襞破坏、消失。

\textbf{【X线诊断】}  回肠占位性病变,腺癌可能,类癌待排。

\textbf{【评  述】}
 本例患者经手术治疗病理证实为回肠类癌。小肠类癌来源于肠壁腺泡的细胞,是一种能产生小分子多肽类或肽类激素的肿瘤。小肠类癌以回肠多见,其在粘膜下生长,多为1~3cm的粘膜下结节,呈广基息肉状。传统的观念认为类癌属于低度恶性肿瘤。可将类癌分为三类:①典型的类癌。②不典型类癌。③低分化神经内分泌癌(小细胞癌)。常见的症状为皮肤潮红、腹泻、喘息、右心瓣膜病、糙皮病等症状。

X线钡剂造影主要表现为:由于小肠类癌系粘膜下肿瘤,当肿瘤较小时,X线钡剂造影不易发现。肿瘤较大长入肠腔或浸润肠壁引起肠管狭窄时,可显示肠腔内息肉样充盈缺损或出现肠套叠征象,病变增大侵及肠系膜则可显示肠外肿块推移邻近肠襻,肠系膜的牵拉使肠襻呈辐辏状排列,肠壁扭曲、肠腔狭窄,甚至梗阻,严重者可引起肠系膜上动脉闭锁,而导致小肠缺血坏死。小肠类癌X线表现无特异性,故与小肠腺癌鉴别诊断困难,因此X线诊断该疾病时必须密切结合临床症状和实验室检查。本例患者X线表现为回肠息肉样充盈缺损,周围粘膜皱襞中断、破坏,结合患者临床上出现皮肤潮红、腹痛、腹泻等类癌综合征的表现,尿液检查5-羟吲哚醋酸增高,故应该考虑小肠类癌诊断的可能性。

\subsection{转移性小肠肿瘤}

\begin{figure}[!htbp]
 \centering
 \includegraphics{./images/Image00306.jpg}
 \captionsetup{justification=centering}
 \caption{回肠转移性肿瘤}
 \label{fig5-5-10}
  \end{figure} 

\textbf{【病史摘要】}
 男性,67岁。结肠癌术后2年,近1个月来右下腹疼痛不适,无节律性,时有腹胀伴恶心、呕吐。体格检查:右侧腹部见手术瘢痕,右下腹压痛,扪及包块,质硬,无移动,肝、脾未及,心、肺阴性。

\textbf{【X线表现】}
 全消化道钡餐造影示:末端回肠可见腔内不规则充盈缺损,中央部见不规则龛影,肠粘膜皱襞破坏。

\textbf{【X线诊断】}  回肠末段转移性肿瘤。

\textbf{【评  述】}
 本例患者经手术治疗病理证实为回肠末端腺癌(转移性)。转移性小肠肿瘤临床罕见,常发生于恶性肿瘤晚期或广泛转移者,尤其是来源于其他消化道恶性肿瘤者。转移灶多见于回肠,尤其是末端回肠,其次为空肠,十二指肠较少见。可单发也可多发,而鳞癌两者均可见到。组织学分类以腺癌及鳞癌居多,其次为恶性黑色素瘤。恶性肿瘤可通过血行、淋巴、腹腔内种植侵犯小肠,尤以血行和腹腔内种植更常见。

小肠气钡双对比造影检查对检出小肠转移瘤有较重要价值,具体表现可有:①局限性向心性狭窄,粘膜破坏,皱襞消失,肠壁光滑僵硬。②孤立性隆起性病变,充盈缺损。③溃疡形成,不规则较大龛影,常伴有轻度狭窄和结节样病变。④瘘管形成,钡剂外溢。⑤冰冻征,见于广泛的腹腔转移和恶性弥漫性腹膜间皮瘤。⑥多发性结节样肠壁压迹。可见有肠梗阻征象,偶有气腹。由于患者都有明确的恶性肿瘤病史,故结合其X线表现诊断一般较明确。

\section{结肠病变}

\subsection{结肠多发性憩室}

\begin{figure}[!htbp]
 \centering
 \includegraphics{./images/Image00307.jpg}
 \captionsetup{justification=centering}
 \caption{结肠多发性憩室}
 \label{fig5-6-1}
  \end{figure} 

\textbf{【病史摘要】}
 男性,66岁。大便习性改变伴腹泻近1个月。体格检查:腹软,无明显压痛,未扪及包块,肝、脾未及,心、肺阴性。

\textbf{【X线表现】}
 全消化道钡餐造影示:盲肠、升结肠、横结肠、降结肠及乙状结肠见多发大小不等乳头状的囊袋状影,阴影凸向肠壁的腔壁线之外,以降结肠段明显,各段结肠未见明显狭窄及其他异常。

\textbf{【X线诊断】}  结肠多发性憩室。

\textbf{【评  述】}
 结肠憩室国外常见,国内少见,好发于40岁以上,男性多于女性。可见于结肠各部分,而乙状结肠、降结肠最多见。结肠憩室一般无明显症状,或仅有轻微不适、便秘等。结肠憩室以钡剂灌肠造影检查较好,尤其是低张双重造影更有利于憩室的显示。常见表现为:突出于肠腔之外的圆形或类圆形阴影,位于结肠袋的顶端,大小不一,口部常较细小,其表现与憩室的大小、充盈状况及粪便多少等因素相关,如憩室内完全为钡剂充盈,则呈圆形或类圆形影;其内有粪便,钡剂涂布于粪团周围的粘膜上,造影表现为环状;憩室完全由粪便充填,钡剂只能充盈于憩室颈部,表现为柱状、杯口状等。憩室的正面观在充盈像上难于发现,需排除钡剂后或双重造影显示。结肠憩室需与溃疡性结肠炎鉴别,后者由于浅小溃疡使结肠壁显示多发的细小毛刺状突出,较大溃疡,结肠壁可见揿扣状壁龛,肠腔表面显示颗粒状粘膜,结肠腔壁线粗糙不光整,病史较长者往往结肠袋消失,管腔变窄。

\subsection{先天性巨结肠}

\begin{figure}[!htbp]
 \centering
 \includegraphics{./images/Image00308.jpg}
 \captionsetup{justification=centering}
 \caption{先天性巨结肠}
 \label{fig5-6-2}
  \end{figure} 

\textbf{【病史摘要】}
 女性,21岁。自幼诊断为先天性巨结肠,近期腹胀、便秘加重。体格检查:发育尚正常,腹部无明显压痛,未及包块,肝、脾未及,心、肺阴性。

\textbf{【X线表现】}
 气钡双重造影示,乙状结肠中段肠腔狭窄,近段结肠扩张明显,可见横向平行的粗大的粘膜皱襞,钡剂下行困难。

\textbf{【X线诊断】}  先天性巨结肠。

\textbf{【评  述】}
 本病是由于直肠或结肠远端的肠管持续痉挛,粪便淤滞在近端结肠,使结肠肥厚、扩张,是小儿常见的先天性肠道畸形。主要临床表现为顽固性便秘、腹胀、营养不良、发育迟缓等。钡剂灌肠的目的在于显示狭窄段及狭窄-扩张移行段结肠,不必充满整个结肠。侧位和前后位照片中可见到典型的痉挛肠段和扩张肠段,排钡功能差,24小时后仍有钡剂存留,若不及时灌肠洗出钡剂,可形成钡石,合并肠炎时扩张肠段肠壁呈锯齿状表现。新生儿先天性巨结肠要与其他原因引起的肠梗阻如结肠闭锁、胎便性便秘、新生儿腹膜炎等鉴别。较大的婴幼儿、儿童应与直肠肛门狭窄、管腔内外肿瘤压迫引起的继发性巨结肠、结肠无力(如甲状腺功能低下患儿引起的便秘)、习惯性便秘以及儿童特发性巨结肠(多在2岁以后突然发病,为内括约肌功能失调)等相鉴别。并发小肠结肠炎时与病毒、细菌性肠炎或败血症肠麻痹鉴别。对短段型先天性巨结肠,尤其是超短段型先天性巨结肠,难与特发性巨结肠鉴别。

\subsection{溃疡性结肠炎}

\begin{figure}[!htbp]
 \centering
 \includegraphics{./images/Image00309.jpg}
 \captionsetup{justification=centering}
 \caption{溃疡性结肠炎}
 \label{fig5-6-3}
  \end{figure} 

\textbf{【病史摘要】}
 女性,35岁。左下腹部疼痛,粘液血便近半年,腹泻、腹胀,食欲减退。体格检查:消瘦,腹部稍膨隆,左下腹压痛明显,肝、脾未及,心、肺阴性。

\textbf{【X线表现】}
 钡剂灌肠检查示:降结肠及横结肠脾曲段结肠袋消失,肠壁粗糙,边缘见多发锯齿状突起,粘膜面网状结构消失而见大小不等的点状致密影。

\textbf{【X线诊断】}  溃疡性结肠炎。

\textbf{【评  述】}
 本例患者钡剂灌肠检查X线征象典型,结肠镜所见证实为溃疡性结肠炎。溃疡性结肠炎原因不明,常发生于青壮年。本病首先侵犯直肠,继而沿长轴向上发展,逐一波及乙状结肠、降结肠、横结肠,甚至全部结肠,但仍以左半结肠为主。病变主要在粘膜与粘膜下层,溃疡很浅,底在肌层,可以自行愈合,溃疡与溃疡之间的肠粘膜面,可由于大量增生而形成许多炎症性息肉。病变愈合后,粘膜下层的纤维组织增生,可使肠腔普遍性变窄,肠管缩短,而呈光滑的直筒状外观。临床上有发作与缓解交替出现的肠炎症状,病程较长。钡剂灌肠检查可见从直肠开始就有刺激性痉挛收缩,左半结肠肠袋变浅,边缘可有许多尖刺状突起,而呈锯齿状。肠粘膜息肉样增生可表现为许多赤豆大小的充盈缺损。上述X线表现,以粘膜像或双对比造影像显示为佳。晚期纤维化之肠管,可呈铅管样结肠。溃疡性结肠炎主要与结肠克罗恩病及肠结核鉴别。结肠克罗恩病好发于右半结肠,病变呈跳跃式,且往往累及末端回肠。结肠结核病变可呈连续性,但往往大多数为末端回肠、盲肠、升结肠受累,发生于结肠其他部位者少见,与溃疡性结肠炎不同。

\subsection{结肠息肉}

\begin{figure}[!htbp]
 \centering
 \includegraphics{./images/Image00310.jpg}
 \captionsetup{justification=centering}
 \caption{结肠多发息肉}
 \label{fig5-6-4}
  \end{figure} 

\textbf{【病史摘要】}
 男性,35岁。下腹部疼痛伴便血半个月,食欲减退。体格检查:腹软,腹部无明显压痛,未扪及包块,肝、脾未及,心、肺阴性。

\textbf{【X线表现】}
 气钡双重造影示:直肠内见多发轮廓光整的充盈缺损,基底部位于肠壁,肠腔壁柔软,光滑整齐。

\textbf{【X线诊断】}  直肠及乙状结肠多发息肉。

\textbf{【评  述】}
 本例患者经结肠镜检查病理证实为直肠及乙状结肠腺瘤样息肉。凡从粘膜表面突出到肠腔的息肉状病变,在未确定病理性质前均称为息肉,按病理可分为:腺瘤样息肉,炎性息肉,错构瘤型息肉。结肠息肉多见于40岁以上成人,男性稍多。大部分病例并无引人注意的症状。仅在体格检查或尸体解剖时偶然发现,部分病例可以具有如便血、粪便改变、腹痛及息肉脱垂等症状。适当的检查方法对提高诊断效率,特别是较小的息肉诊断最为关键,理想的检查是要获得良好的粘膜像与气钡双重造影,结合多轴面透视观察,适当加压,才能充分显示病变。

钡灌肠检查表现主要为:肠腔内轮廓光整的充盈缺损,多发性息肉表现为多个大小不等充盈缺损,带蒂的息肉可显示其长蒂,有一定的活动度。而息肉病表现为直肠、乙状结肠及结肠其他部位有大大小小的充盈缺损,在粘膜相上出现无数轮廓光整葡萄状的块影,充满肠腔。结肠息肉有时应注意与肠腔内气泡和粪块相鉴别,粪块和气泡转换体位时形态、位置均会有改变,尤其重复检查对鉴别帮助最大,因为粪块和气泡不会在多次检查中位于同一部位。结肠息肉来源于粘膜上皮,不累及肌层,故局部肠壁及结肠袋一般正常,此与溃疡性结肠炎形成的假性息肉所致的肠壁及结肠袋的改变不同,应注意区别。息肉的恶变,文献报道,息肉大小在良、恶性鉴别上有肯定意义,息肉直径大于2cm者恶变概率在50%,小于5mm者恶变概率不到0.1%。带蒂息肉恶变概率较小,大于1cm的息肉,基底部出现不规则凹陷和回缩可考虑为恶变征象。

\subsection{回盲型肠套叠}

\begin{figure}[!htbp]
 \centering
 \includegraphics{./images/Image00311.jpg}
 \captionsetup{justification=centering}
 \caption{回盲部肠套叠}
 \label{fig5-6-5}
  \end{figure} 

\textbf{【病史摘要】}
 男性,6岁。因腹胀、恶心、呕吐伴肛门停止排气1天入院。体格检查:右下腹痛,拒按,触及腊肠状腹块,肝、脾未及,心、肺阴性。大便隐血(++)。

\textbf{【X线表现】}
 钡剂灌肠示结肠肝曲处见钡剂受阻,呈杯口样充盈缺损,其内可见弹簧状纹理。灌注空气,示钡剂进入升结肠、盲肠及回肠末端,肠套叠复位。

\textbf{【X线诊断】}  回盲部肠套叠。

\textbf{【评  述】}
 本例患者经钡剂灌肠检查及空气灌注整复,确诊为回盲型肠套叠。肠套叠是指一段肠管套入与其相连的肠腔内,并导致肠内容物通过障碍。有原发性和继发性两类。原发性肠套叠多发生于婴幼儿,继发性肠套叠则多见于成人。成人肠套叠多发生在回盲部,且继发于肿瘤、息肉等。肠套叠可发生在小肠或大肠的任何部位,按套入肠的顶端和外鞘、颈部肠段的不同分为5型:小肠型,回盲型,回结型,结肠型,空肠胃套叠。上述类型中,回盲型肠套叠发病率最高。回盲型肠套叠系套入部位于盲肠内,造成充盈缺损而导致盲肠变形。急性肠套叠临床表现主要为急性肠梗阻症状、便血,并可扪及腊肠状腹块,慢性肠套叠表现为慢性不全性梗阻,同时伴有便血、腹块。本病常用空气或钡剂灌肠法检查,在不全性梗阻的病例中可使用口服法检查,但应特别慎重,否则有可能加重梗阻而使症状加重。

X线影像主要为:①钡剂在套叠部,先入套入部,或称套叠中央管,其表现较具特征,即该套入部肠腔明显变窄,由于该套入部充盈钡剂程度不同,表现各异,充盈多时,可见皱襞呈纵形平整的条索,充盈不足时,仅呈窄细的线形,远端肠扩大,呈杯口状或螺旋状环绕套入中央管。②由于成人慢性肠套叠以回盲结型多见,回肠末端及其系膜被卷入升结肠内,受系膜的牵拉,使整个套叠部向内下移位,遇有局部痉挛、激惹等使上述套叠结构显示不清时,如果见升结肠、肝曲有向内、下移位现象,应考虑回盲部套叠所致,并排除回盲部结核和肿瘤。③钡剂通过套叠部时间延长呈半梗阻状态。④套叠头部常呈分叶状,钡剂仅从其中之一通过至远端结肠。肠套叠X线征象典型,诊断一般不难,但引起套叠的肠壁实质性占位有时确诊并不容易,尤其是肿瘤较大,加之肠套叠套鞘与套入部形成密集的弹簧状及发状粘膜皱襞的遮盖,肿瘤形态不易观察,检查及诊断应多时相、多体位、密切结合临床间断观察。

\subsection{阑尾周围脓肿}

\begin{figure}[!htbp]
 \centering
 \includegraphics{./images/Image00312.jpg}
 \captionsetup{justification=centering}
 \caption{阑尾周围脓肿}
 \label{fig5-6-6}
  \end{figure} 

\textbf{【病史摘要】}
 男性,35岁。转移性右下腹疼痛1周,伴发热、恶心、呕吐。体格检查:右下腹压痛明显,扪及质软包块,肝、脾未及,心、肺阴性。血常规血白细胞计数12×10\textsuperscript{9}
/L,中性粒细胞比例85%。

\textbf{【X线表现】}
 钡剂灌肠检查示:盲肠下端管腔狭窄,边缘不整齐,见弧形压迹影,钡剂通过有激惹征象,周围粘膜皱襞未见明显中断、破坏。

\textbf{【X线诊断】}  阑尾区占位性病变,考虑阑尾周围脓肿可能性大。

\textbf{【评  述】}
 本例患者经手术治疗,术后病理证实为阑尾脓肿。急性阑尾炎化脓坏疽或穿孔,如果此过程进展较慢,大网膜可移至右下腹部将阑尾包裹、粘连形成炎性肿块或阑尾周围脓肿。细菌感染和阑尾腔的阻塞是阑尾炎发病的两个主要因素。由早期炎症加重而致,或由于阑尾管腔梗阻,内压增高,远端血运严重受阻,感染形成和蔓延迅速,以致数小时内即成化脓性甚至蜂窝织炎性感染。阑尾肿胀显著,浆膜面高度充血并有较多脓性渗出物,部分或全部为大网膜所包裹。临床表现:患者多有右下腹疼痛,或者转移性右下腹疼痛病史,可有发热、恶心、呕吐等表现。亦可有轻微腹泻等表现。少数患者可因大网膜压迫肠管,造成不全肠梗阻症状。钡灌肠能很好地观察结肠及回盲部的充盈情况和粘膜有无异常,为首选方法。钡剂造影检查可见右下腹包块与肠管粘连,不能分开;盲肠变形,边缘不规则,但粘膜皱襞无破坏,局部有压痛;盲肠有激惹征象,钡剂通过快,盲肠也可处于痉挛状态;盲肠局部可出现压迹,末端回肠可同时向上推移。若脓肿与盲肠相通,可使之显影,显示为肠道外不规则窦腔。根据上述阑尾脓肿的X线特点,结合临床,多数诊断当无困难,但少数病例由于临床表现复杂,需与下列回盲部病变鉴别:包括回盲部良、恶性肿瘤及炎性病变,有些表现与脓肿相似,但均有相应的临床及X线特点可资鉴别。如结肠癌时的肠腔狭窄、充盈缺损,形态恒定,管壁僵硬,粘膜破坏,无弧形压迹,能触及肠腔内包块,临床可有粘液血便等。炎性病变可见肠腔狭窄、短缩,牵拉移位及激惹等,且有弧形压迹及包块,与阑尾周围脓肿表现不同。

\subsection{结肠癌}

\begin{figure}[!htbp]
 \centering
 \includegraphics{./images/Image00313.jpg}
 \captionsetup{justification=centering}
 \caption{横结肠浸润性结肠癌}
 \label{fig5-6-7}
  \end{figure} 

\textbf{【病史摘要】}
 男性,55岁。腹痛、腹胀、便秘2个月余,粘液脓血便。体格检查:消瘦,左上腹扪及包块,质硬,肝、脾未及,心、肺阴性。大便常规隐血(++)。

\textbf{【X线表现】}
 气钡双重造影示:横结肠管腔向心性狭窄,粘膜皱襞中断、破坏,病变与正常肠壁分界清楚。

\textbf{【X线诊断】}  横结肠浸润性结肠癌(进展期)。

\textbf{【评  述】}
 本例经手术治疗病理证实为横结肠腺癌。结肠癌是发生于结肠部位的常见的消化道恶性肿瘤。好发部位为直肠及直肠与乙状结肠交界处,以40~50岁年龄组发病率最高。浸润性结肠癌以向肠壁各层呈浸润生长为特点。病灶处肠壁增厚,表面粘膜皱襞增粗、不规则或消失变平。早期多无溃疡,后期可出现浅表溃疡。如肿瘤累及肠管全周,可因肠壁环状增厚及伴随的纤维组织增生使肠管狭窄,即所谓的环状缩窄型,此时在浆膜局部可见到缩窄环。切面肿瘤边界不清,肠壁因肿瘤细胞浸润而增厚。临床常见症状为排便习惯改变,血性便及肠梗阻。肠梗阻可表现为突然发作的急性完全性梗阻,但多数为慢性不完全性梗阻,腹胀很明显,大便变细形似铅笔,症状进行性加重最终发展为完全性梗阻。钡剂灌肠检查可见癌肿部位的肠壁僵硬,扩张性差,蠕动至病灶处减弱或消失,结肠袋形态不规则或消失,肠腔狭窄,粘膜皱襞紊乱、破坏或消失、充盈缺损等。结肠进展期各型癌肿X线征象均较明确,诊断不难。

结肠癌有时需注意和结肠其他少见肿瘤的鉴别,平滑肌瘤以累及直肠为多,X线表现为粘膜下肿瘤的特点,其大部分位于肠腔外是其特征。淋巴瘤少见,发生多位于盲肠或直肠,常常累及末端回肠,环状浸润范围较长,可表现为向心性狭窄,但很少出现梗阻,弥漫型可累及长段或全结肠与结肠癌不同。类癌大多发生于直肠,其次为盲肠。X线表现为不规则伞状充盈缺损,另一种为不规则环状狭窄,与结肠癌不易鉴别。发生在直肠的类癌,直肠镜检查优于X线检查。

\section{急腹症}

\subsection{胃穿孔}

\begin{figure}[!htbp]
 \centering
 \includegraphics{./images/Image00314.jpg}
 \captionsetup{justification=centering}
 \caption{消化道穿孔}
 \label{fig5-7-1}
  \end{figure} 

\textbf{【病史摘要】}
 男性,45岁。进食后突发持续性剧烈腹痛1小时,伴恶心、呕吐,既往有胃溃疡病史。体格检查:全腹压痛、反跳痛与肌紧张,肠鸣音减弱,体温38.2℃,血常规白细胞计数11.7×10\textsuperscript{9}
/L,中性粒细胞比例85%。

\textbf{【X线表现】}
 立位腹部平片示两侧膈下见新月形游离气体影;左侧卧位片示右侧胸腹壁下见半月形游离气体影。

\textbf{【X线诊断】}  消化道穿孔,结合胃溃疡病史,诊断胃溃疡急性穿孔。

\textbf{【评  述】}
 本例患者经手术治疗证实为胃体小弯侧溃疡穿孔。胃肠腔外气体的来源最多见于消化道穿孔,气体游离于腹腔内,其次是腹腔内产气细菌性脓肿及外科腹部手术或外伤后空气进入腹腔。膈下游离气体是胃肠道穿孔的最重要的X线表现。立位X线检查时,腹腔内游离气体上升于膈下,呈镰刀样或半月形透明阴影。右侧比左侧多见,但亦有单独出现在左侧的,此时要注意与结肠脾曲及单纯胃泡相区别。若有疑问可进一步做左侧卧位水平X线投照及半立位侧水平X线片检查,观察肝脏上方及剑突下有无游离气体。胃后壁溃疡穿孔时,气体进入小网膜囊,于上腹部或左上腹部存在透明气影,它不随体位变化而移动。

膈下游离气体应与假性气腹相鉴别,以免误诊。可出现假性气腹的有:①横膈下脂肪垫,肥胖患者在透视或平片X线检查时,于膈下有时可见条状或带状不规则透亮阴影,很似膈下游离气体。但透亮度一般比气腹低,变换体位时,此透亮影固定不变,无移动性。②膈下脓疡,于膈下可见一局限性包裹性气影,可有液平面。此外尚有患侧膈肌升高、运动减弱或消失,有胸膜反应或胸腔积液等X线征象。③间位结肠或间位小肠,仔细观察可见到结肠袋间隔或小肠的环形皱襞阴影,可与膈下游离气体相鉴别。④两侧弥漫性阻塞性肺气肿或下肺野局限性肺气肿时,其气肿的肺组织投影于膈下区域,有时很像膈下游离气体,要注意鉴别。⑤内脏转位,内脏反位患者,胃泡气影位于右侧膈下,同时可见到其他脏器的反位现象,区别并不困难。气腹除见于上述原因外,还可见于下列原因:人工气腹、腹腔穿刺后、输卵管通气术后、阴道冲洗后、肠壁气囊肿破裂等,故诊断消化道穿孔需密切结合临床资料,综合分析诊断。

\subsection{小肠机械性肠梗阻}

\begin{figure}[!htbp]
 \centering
 \includegraphics{./images/Image00315.jpg}
 \captionsetup{justification=centering}
 \caption{小肠机械性肠梗阻}
 \label{fig5-7-2}
  \end{figure} 

\textbf{【病史摘要】}
 女性,65岁。阵发性腹痛3天并逐渐加重,伴恶心、呕吐、肛门停止排便排气1天。体格检查:腹部膨隆,脐周压痛,肠鸣音亢进,可闻及气过水声。

\textbf{【X线表现】}
 立位腹部平片示:肠腔气体郁积,见多发宽窄不等、阶梯状排列的气液平面。

\textbf{【X线诊断】}  低位单纯性小肠机械性梗阻。

\textbf{【评  述】}
 肠梗阻为常见的急腹症,X线检查是诊断的可靠方法之一。本例患者立位腹部平片梗阻征象明确,诊断成立。小肠高位机械性肠梗阻时,梗阻近端之肠管内大量液体滞留,而气体多反流入胃内,故X线征象不多,平片诊断常很困难。最好采用口服有机碘溶液造影检查。低位性小肠梗阻时,梗阻近端的小肠积气扩张,小肠呈线团状或鱼骨状粘膜皱襞形态,主要见于空肠。回肠环形粘膜皱襞较少,特别远端回肠更少。当肠管明显扩张时,回肠粘膜皱襞可完全消失。梗阻远端肠曲收缩,结肠内很少或没有气体存留。于立位照片时,腹部可见多个呈阶梯状液平面,似倒U形。透视可见液平面上下不规则地移动。如机械性肠梗阻持续时间长,可继发肠麻痹(反射性肠淤张)。此时前者的征象可以完全被掩盖,对诊断及识别病变的真正过程造成一定困难,需全面检查并结合临床仔细分析才能得出正确结论。

\subsection{麻痹性肠梗阻}

\begin{figure}[!htbp]
 \centering
 \includegraphics{./images/Image00316.jpg}
 \captionsetup{justification=centering}
 \caption{小肠麻痹性肠梗阻}
 \label{fig5-7-3}
  \end{figure} 

\textbf{【病史摘要】}
 女性,56岁。1周前因胃癌行毕Ⅰ式胃大部分切除术,现自述腹胀、肛门无排便排气。体格检查:腹部见手术吻合钉影,腹部膨隆,压痛,未及包块,肝、脾未及,心、肺阴性。

\textbf{【X线表现】}
 立位腹部平片示:残胃、小肠、结肠均积气,肠腔扩张不明显,可见多发小液平。

\textbf{【X线诊断】}  小肠麻痹性肠梗阻。

\textbf{【评  述】}
 本例患者有近期胃大部分切除术史,现临床出现肠梗阻症状,结合立位腹部平片表现,麻痹性肠梗阻诊断不难。麻痹性肠梗阻常见于腹部手术后、腹部炎症、腹膜炎、胸腹部外伤及感染等。临床症状表现为疼痛、呕吐、腹胀、肛门停止排便排气、腹软、肠鸣音减弱或消失。麻痹性肠梗阻由于没有肠管的器质性狭窄,而是肠管处于麻痹状态,引起肠内容物的通过和吸收障碍。其X线特点为胃、大肠、小肠呈均等的积气扩张,并有液平面,液平面较宽,但小于机械性小肠梗阻。多次复查肠管形态改变不明显。如果不合并有腹膜炎,则扩张的肠曲相互靠近,肠间隙正常。如果同时合并腹腔内感染,则肠间隙可增宽,腹脂线模糊。

\subsection{小肠绞窄性肠梗阻}

\begin{figure}[!htbp]
 \centering
 \includegraphics{./images/Image00317.jpg}
 \captionsetup{justification=centering}
 \caption{小肠绞窄性肠梗阻}
 \label{fig5-7-4}
  \end{figure} 

\textbf{【病史摘要】}
 男性,48岁。突然出现腹部剧痛伴恶心、呕吐、肛门停止排便排气1天。体格检查:腹部膨隆,有压痛,可见肠形,听诊肠鸣音亢进,有气过水声,血压110/70mmHg。

\textbf{【X线表现】}
 立位腹部平片示:小肠积气,扩张明显,中腹部见多个跨度卷曲肠襻,呈花瓣型。

\textbf{【X线诊断】}  小肠绞窄性肠梗阻。

\textbf{【评  述】}
 绞窄性肠梗阻是由于肠系膜血管发生狭窄,致使血循环发生障碍,引起小肠坏死。常见的原因是小肠扭转、粘连带压迫和内疝等。肠系膜过长、肠管功能紊乱以及肠内容物增加均易造成小肠扭转。绞窄性肠梗阻时,早期即出现严重的临床症状:休克、呕吐、便血、脉快而弱等。

X线改变主要有:①假肿瘤征:充满液体的嵌闭肠曲呈圆形肿块,边缘清楚,不可活动,多位于下腹部,可压迫周围肠曲或膀胱引起移位。立位时可见液平面,但若是完全性绞窄性梗阻,则绞窄肠曲内多无气体。②咖啡豆状征:这是较具特征性的改变;当肠系膜绞窄时,系膜因痉挛水肿而挛缩变短,于是以肠系膜为中心,牵拉闭襻梗阻肠曲的两端使之纠集变位,出现各种排列状态,如C字形、8字形、花瓣征、香蕉征等。③阻塞近端肠管大量积液扩张并有液平面。因肠管麻痹,气体多反流至胃,形成小肠内气体较少,液平面较长,其上气柱低而扁。且活动度低。④空回肠换位征。本例患者出现典型的花瓣征,结合临床症状,诊断小肠绞窄性肠梗阻。绞窄性肠梗阻的诊断非常重要,因为明确绞窄性肠梗阻诊断后,外科需立即急诊手术治疗,否则病死率极高。因此,当已确定小肠梗阻时,还必须检查分析是否有绞窄性肠梗阻可能,并结合临床症状、体征和发病过程,再排除与其相似的疾病,可做出初步诊断。

\subsection{乙状结肠扭转}

\begin{figure}[!htbp]
 \centering
 \includegraphics{./images/Image00318.jpg}
 \captionsetup{justification=centering}
 \caption{乙状结肠扭转}
 \label{fig5-7-5}
  \end{figure} 

\textbf{【病史摘要】}
 男性,55岁。突发腹部持续性剧烈疼痛半天,肛门停止排气、排便。有右股骨颈骨折内固定手术史。体格检查:腹部膨隆,左下腹压痛明显,肝、脾未及,心、肺阴性。

\textbf{【X线表现】}
 钡剂灌肠检查示:乙状结肠中段阻塞,近端狭窄呈鸟嘴状,鸟嘴尖端指向左侧,远端直肠、乙状结肠扩张明显。

\textbf{【X线诊断】}  乙状结肠扭转。

\textbf{【评  述】}
 本例患者经手术治疗明确诊断为乙状结肠扭转。乙状结肠较长,而乙状结肠系膜附着处又短窄,近侧和远侧两侧肠管接近,肠襻活动度大,这是容易发生扭转的解剖基础。乙状结肠扭转可以呈顺时针或逆时针方向。扭转对肠管血循环的影响程度,主要决定于扭转的多少和松紧程度,如扭转180°时,肠系膜血循环可无绞窄,仅位于乙状结肠壁后面的直肠受压而出现单纯性肠梗阻。扭转超过360°时,必将造成绞窄性闭襻性肠梗阻。X线检查腹部平片可见腹部偏左明显充气的巨大孤立肠襻自盆腔达中上腹部,甚至可达膈下,占据腹腔大部形成所谓弯曲管征。在巨大乙状结肠肠襻内,常可看到两个处于不同平面的液气面。左、右半结肠及小肠有不同程度的胀气。钡剂灌肠造影可见钡剂在直肠与乙状结肠交界处受阻,钡柱尖端呈锥形或鸟嘴形,且灌肠之容量往往不及500ml(正常可灌入2000ml以上),并向外流出,即可证明在乙状结肠处有梗阻。此项检查仅适用于一般情况较好的早期扭转病例,当有腹膜刺激征或腹部压痛明显者禁忌钡灌肠检查,否则有发生肠穿孔的危险。

\section{胆道疾病}

\subsection{先天性胆总管囊肿}

\begin{figure}[!htbp]
 \centering
 \includegraphics{./images/Image00319.jpg}
 \captionsetup{justification=centering}
 \caption{先天性胆总管囊肿}
 \label{fig5-8-1}
  \end{figure} 

\textbf{【病史摘要】}
 男性,9岁。反复发作性右上腹痛伴皮肤发黄、瘙痒,近期疼痛加剧并伴发热。体格检查:皮肤、巩膜黄染,腹软,右上腹可触及一鸡蛋大小肿块,质软,可推动,轻度压痛,心、肺阴性。

\textbf{【X线表现】}
 经皮肝穿胆管造影示:胆总管扩张明显,呈囊状,壁光滑,其扩张段直径与胆道其余部分失去比例关系。

\textbf{【X线诊断】}  先天性胆总管囊肿。

\textbf{【评  述】}
 本例患者经手术证实为先天性胆总管囊肿。先天性胆总管囊肿又称为胆总管囊性扩张,病因尚不明确,多见于女性、儿童。胆总管囊性扩张范围不一定,可涉及胆总管某一部分或全部,囊肿大小不等,多位于胆总管中段。胆总管囊肿的典型三联症是腹痛、黄疸和腹部包块,其中以腹部疼痛最为明显。口服或静脉胆管造影多不显影,囊肿穿刺造影虽可显示囊肿大小和位置,但有一定的危险性,即有可能继发胆汁性腹膜炎。内镜逆行胰胆管造影(ERCP)则能直接显示整个胆道系统,尤其是对了解胰管与胆管的关系,肝内胆管有无结石和狭窄等提供直接证据。先天性胆总管囊肿,有时需注意与胆总管下端肿瘤或结石引起的肝内外胆管扩张相鉴别,前者胆总管呈球形或梭形局限性扩张,肝内胆管扩张但并不广泛,以靠近肝门近端周围肝管扩张为其特点;后者形成梗阻所致梗阻以上肝内外胆管扩张广泛而均匀,胆总管下端可见圆形或类圆形低密度充盈缺损结石影为其特点。

\subsection{先天性胆囊畸形}

\begin{figure}[!htbp]
 \centering
 \includegraphics{./images/Image00320.jpg}
 \captionsetup{justification=centering}
 \caption{先天性胆囊畸形}
 \label{fig5-8-2}
  \end{figure} 

\textbf{【病史摘要】}
 男性,42岁。右上腹痛1年,向肩背部放射,疼痛时伴恶心、呕吐。体格检查:腹软,右上腹轻压痛,未扪及包块,皮肤、巩膜无黄染,心、肺阴性。

\textbf{【X线表现】}
 胆囊造影检查示:胆囊呈葫芦形,上部见囊壁局部缩窄,胆囊壁光整,肝内胆管及肝总管、胆总管未见充盈缺损及扩张,造影剂经十二指肠弥散通畅(图A)。

\textbf{【X线诊断】}  胆囊炎;先天性胆囊畸形(葫芦形)。

\textbf{【评  述】}
 一般的胆囊先天性异常无临床症状且不影响生理功能,仅在影像检查时偶然发现。如葫芦形胆囊(图A);错位胆囊,包括肝内胆囊(图B)及左位胆囊(图D);巨胆囊(图C)等。此类异常的诊断有赖于对影像的正确认识,最好有两种以上的检查印证,如B超、胆囊造影、MRCP等。其诊断的意义在于与病理状态的鉴别及了解有无合并其他病变。

\subsection{胆道蛔虫症}

\begin{figure}[!htbp]
 \centering
 \includegraphics{./images/Image00321.jpg}
 \captionsetup{justification=centering}
 \caption{胆道蛔虫症}
 \label{fig5-8-3}
  \end{figure} 

\textbf{【病史摘要】}
 男性,23岁。右上腹部剧痛伴恶心、呕吐。体格检查:右上腹压痛明显,肝、脾未及,心、肺阴性。

\textbf{【X线表现】}
 胆总管及右肝管内见一边缘光整、稍呈弯曲的条状透亮阴影,右肝管稍扩张,胆总管未见扩张。

\textbf{【X线诊断】}  胆道蛔虫症。

\textbf{【评  述】}
 依据胆管内显示边缘平滑并呈弯曲的条状透亮阴影,形状与蛔虫相似,两侧的造影剂呈现出双轨征,胆道蛔虫症可以确诊。胆道蛔虫症是肠蛔虫病的常见并发症,也是常见的急腹症。X线检查以胃肠道钡剂造影和直接胆管造影为主。平片检查价值有限,静脉胆道造影作用也有限。超声能清楚显示进入胆管的蛔虫,并能在超声导向下做取虫治疗。胆道造影检查常表现为胆管内发辫状或长圆柱状充盈缺损,为蛔虫的直接征象。充盈缺损影纵轴与胆管方向一致,多为一条,也有数目较多者。蛔虫不仅位于肝外胆管,也可伸入肝内胆管。蛔虫死亡解体后的残体以及所形成的结石也形成充盈缺损,形态各异,注意同单纯胆石鉴别。除充盈缺损外,还可显示胆管扩张性改变。

\subsection{胆总管结石}

\begin{figure}[!htbp]
 \centering
 \includegraphics{./images/Image00322.jpg}
 \captionsetup{justification=centering}
 \caption{胆道多发性结石}
 \label{fig5-8-4}
  \end{figure} 

\textbf{【病史摘要】}
 男性,55岁。胆囊结石行胆囊切除术后,反复发作性右上腹胀痛、不适近3年,无明显发热、恶心、呕吐,近1周巩膜、皮肤黄染,伴有呕吐,食欲减退。体格检查:皮肤、巩膜黄染,上腹部压痛,未扪及包块,肝、脾无增大,心、肺阴性。

\textbf{【X线表现】}
 胆总管内见两枚大小不一的充盈缺损,胆总管及肝内胆管均显示扩张(图A),胆囊未见。

\textbf{【X线诊断】}  胆总管多发结石。

\textbf{【评  述】}
 本例患者经十二指肠大乳头括约肌切开网篮取出两枚完整结石。胆管结石常见。结石可位于肝胆管的任何部位,如胆总管下端(图B);肝内胆管(图C);肝总管(图D),以胆总管结石为多见。胆总管结石大多由胆色素及胆固醇组成,一般含钙盐较少,通常透X线,故胆区平片观察结石价值不大。B超检查可发现胆管内结石及胆管扩张影像,故胆管结石一般首选B超检查,必要时可加行ERCP或PTC。

PTC的X线特征有:①肝总管或左右肝管处有环形狭窄,狭窄近端胆管扩张,其中可见结石阴影。②左右肝管或肝内某部分胆管不显影。③左右叶肝内胆管呈不对称性、局限性、纺锤状或哑铃状扩张。ERCP可选择胆管造影,对肝内胆管结石具有较高的诊断价值,可清晰显示肝内胆管结石,确定结石的部位、大小、数量,肝内胆管的狭窄或远端扩张。CT扫描对于肝内胆管结石的诊断意义较大。胆总管结石由于较大而容易被发现,而胰腺钓突内结石则较小,尤其是含钙量少时只表现为小致密点,因为CT密度分辨率较高,则可显示。胆总管扩张时,胆总管的横断面呈边界清楚的圆形或椭圆形低密度影,自上而下逐渐变小。核磁共振胰胆管造影(MRCP)是不同于ERCP的全新的检查方法,属无创性检查,不需要做十二指肠镜即可诊断肝内、外胆管结石。对肝内胆管结石有较大诊断价值,但价格较贵,不易普及。总之,B超、ERCP、胆道镜等方法诊断价值较大,简便易行,是诊断肝内胆管结石的首选方法。尤其是ERCP和胆道镜,对肝内胆管结石诊断的准确性高于B超。在B超检查发现肝内胆管结石后,应常规进行上述方法的检查。

胆管结石需与胆管肿瘤鉴别。胆管良性肿瘤极为少见。多见的胆管癌阻塞端常有破坏、狭窄、僵直及不规则充盈缺损。胆管结石的阻塞端多为圆形充盈缺损,典型者则显示杯口状充盈缺损是其特征,无破坏、狭窄及僵直改变。胆管癌扩张的肝内胆管往往呈软藤状,而结石扩张的肝内胆管则显示枯枝状,两者表现不同。

\subsection{慢性胆囊炎、胆结石}

\includegraphics{./images/Image00323.jpg}

\begin{figure}[!htbp]
 \centering
 \includegraphics{./images/Image00324.jpg}
 \captionsetup{justification=centering}
 \caption{胆囊结石}
 \label{fig5-8-5}
  \end{figure} 

\textbf{【病史摘要】}
 女性,35岁。右上腹疼痛近1年,伴发作时恶心、呕吐。体格检查:腹软,右上腹压痛,未扪及包块,皮肤、巩膜无黄染,心、肺阴性。

\textbf{【X线表现】}
 腹部平片示:胆囊内见多发大小不等结节样充盈缺损,边缘密度较高,中央密度较低,边缘光整(图A)。

\textbf{【X线诊断】}  胆囊结石。

\textbf{【评  述】}
 胆囊炎、胆结石临床常见。结石可发生在胆囊任何部位,如胆囊底部(图B);胆囊底、体颈部(图C);胆囊管部(图D)。胆囊阳性结石为10%~20%,结石密度不均匀,可为年轮状致密影,中央透光而周围呈不同厚度的环形影,形如石榴子状。急性胆囊炎依据患者的病史及症状、实验室检查,即可做出诊断。X线检查对其诊断有一定限度。慢性胆囊炎由于胆囊壁增厚、瘢痕收缩以及周围组织粘连,并经常与胆囊结石并发,故X线征象典型。阳性胆囊结石平片诊断中需与肾结石、肾上腺钙化、肠系膜淋巴钙化、胰腺结石、肝包囊虫钙化等进行鉴别,一般通过改变体位不难区别。阴性结石在造影中形成充盈缺损,这需与胆囊良性肿瘤、胆固醇息肉、胆囊腺肌增生症、胆囊癌以及结肠内气体重叠干扰等进行鉴别。胆囊癌为胆囊腔内分叶状不规则充盈缺损,囊壁僵硬,有内陷,轮廓不光整,特别是患者的临床表现以及胆囊区可触及的质硬肿块为鉴别的重要参考,而胆石在不同体位时可以移动,是重要鉴别点。

\subsection{胆管癌}

\begin{figure}[!htbp]
 \centering
 \includegraphics{./images/Image00325.jpg}
 \captionsetup{justification=centering}
 \caption{胆管癌}
 \label{fig5-8-6}
  \end{figure} 

\textbf{【病史摘要】}
 男性,65岁。因胆囊炎、胆结石行胆囊切除术后5年,近1个月来右上腹部疼痛伴黄疸。体格检查:体瘦,皮肤、巩膜黄染,腹壁紧张,右上腹压痛,未扪及包块,肝、脾未及,心、肺阴性。

\textbf{【X线表现】}
 ERCP造影检查示:肝总管不规则狭窄、扭曲,梗阻近侧端胆管正常,肝内胆管扩张(图A)。

\textbf{【X线诊断】}  肝总管胆管癌。

\textbf{【评  述】}
 胆管癌可发生于胆管的各个部位,如胆总管下段(图B)。近50%肝外阻塞的患者是由非结石性病因引起的,其中以恶性肿瘤最多见。这些恶性肿瘤大多数发生于远端胆总管所在的胰头部,少数发生于壶腹部、胆管、胆囊和肝内。由转移性肿瘤和淋巴结阻塞胆管的现象极为少见。发生在胆管的一些良性乳头状瘤或绒毛状腺瘤也可阻塞胆管。早期肿瘤较小时,多无临床症状。随着胆管阻塞的症状和体征进行性加重,可见黄疸、不同程度的腹部不适、厌食、体重下降、皮肤瘙痒、腹部可触及包块或胆囊等,但寒战、高热少见。

X线所见:早期多为偏侧性充盈缺损而造成胆管狭窄,其范围多在1cm以下,边缘光滑者应考虑为良性肿瘤,边缘不规则者多为癌,同时伴有狭窄上端胆管扩张;晚期则胆管不显影。

胆管肿瘤需与胆管结石鉴别。胆管良性肿瘤极为少见。多见的胆管癌阻塞端常有破坏、狭窄、僵直及不规则充盈缺损。胆管结石的阻塞端多为圆形充盈缺损,典型者则显示杯口状充盈缺损是其特征,无破坏、狭窄及僵直改变。胆管癌扩张的肝内胆管往往呈软藤状,而结石扩张的肝内胆管则显示枯枝状,两者表现不同。

结节型胆管癌影像学有时需与胆管良性肿瘤如乳头状腺瘤相鉴别,后者少见,其在胆管内可形成广基底或带蒂的充盈缺损,轮廓光整,胆管壁光滑无内陷。而浸润型胆管癌所致胆管不规则狭窄,管壁粗糙僵硬与硬化型胆管炎累及范围较长、管腔狭窄、管壁光滑的影像也不同。

\section{胰腺病变}

\subsection{慢性胰腺炎}

\begin{figure}[!htbp]
 \centering
 \includegraphics{./images/Image00326.jpg}
 \captionsetup{justification=centering}
 \caption{慢性胰腺炎}
 \label{fig5-9-1}
  \end{figure} 

\textbf{【病史摘要】}
 男性,55岁。反复发作性上腹部痛2年,伴恶心、呕吐,诊断为慢性胰腺炎,行药物及饮食治疗,近1周出现上腹部疼痛加剧,伴腹泻、恶心、呕吐、食欲减退。体格检查:上腹部压痛明显,未扪及包块,心、肺阴性。实验室检查:血清淀粉酶650U。

\textbf{【X线表现】}
 主胰管扩张、迂曲,主胰管周围胰管分支扩张、粗细不均(箭头),胰管内未见明显充盈缺损。

\textbf{【X线诊断】}  慢性胰腺炎。

\textbf{【评  述】}
 慢性胰腺炎病因是多方面的。70%~80%的病例与长期酗酒有关。乙醇作用可减少胰液的分泌,使胰液中的蛋白质成分增加,在小胰管中沉积,引起填塞、慢性炎症和钙化。本例患者内镜逆行胰胆管造影检查见主胰管扩张改变,结合临床表现及实验室检查,慢性胰腺炎诊断可以确诊。慢性胰腺炎部分患者X线平片在胰腺区可见不规则斑点状钙化阴影。内镜逆行胰胆管造影(ERCP)主要表现为主胰管及其分支规则、均匀性扩张,也可表现为扩张与狭窄交替,扭曲呈串珠状改变。

主胰管内形成结石或囊肿以及纤维化,可出现杯口及截断性梗阻,往往需与胰腺癌加以鉴别,后者主胰管因癌症浸润截然中断,阻塞部断端锐利,并伴有不规则僵硬征象,邻近胰管小分支消失。而前者阻塞部断端表现圆钝而光滑,周围可见胰管小分支显影。主胰管狭窄胰腺癌可呈直线状或尖端变细如鼠尾状明显僵直。胰腺炎狭窄可为局限性或为范围较长,但边缘一般比较光滑,狭窄部与扩张部移行性过度,周围胰管分支可见,两者征象表现不同应能鉴别。但实践中胰腺炎特别是局限性胰腺炎与胰腺癌的鉴别仍有许多困难,临床工作中应采用影像学综合检查明确诊断。近年来由于CT、MRI检查的运用,诊断符合率明显增加。

\subsection{胰头癌}

\begin{figure}[!htbp]
 \centering
 \includegraphics{./images/Image00327.jpg}
 \captionsetup{justification=centering}
 \caption{胰头癌}
 \label{fig5-9-2}
  \end{figure} 

\textbf{【病史摘要】}
 女性,55岁。上腹部疼痛不适伴黄疸近2个月,恶心、呕吐,食欲减退。体格检查:体瘦,皮肤、巩膜黄染,上腹部压痛,未扪及包块,肝、脾肋下未及,心、肺阴性。

\textbf{【X线表现】}
 上消化道钡餐造影示:胃窦、十二指肠球部大弯侧、十二指肠肠曲内侧缘粘膜皱襞紊乱、固定,呈锯齿状改变。

\textbf{【X线诊断】}  胰头癌侵犯十二指肠、胃窦。

\textbf{【评  述】}
 本例患者经手术治疗病理证实为胰头癌,侵及十二指肠及胃窦。胰腺癌是胰腺最常见的肿瘤。多发生于40岁以上的中老年人。临床表现主要为腹部胀痛不适、胃纳减退、体重减轻、黄疸和腰背部疼痛。胰腺癌发生于胰头部最多,占60%~70%;胰体癌其次,胰尾癌更次之。胰头癌因常常早期侵犯胆总管下端、引起梗阻性黄疸而发现较早。

X线平片检查不能显示胰腺,故没有价值。胃肠道钡餐造影检查在胰头癌肿块较大、侵犯十二指肠时做低张十二指肠钡剂造影检查,可见十二指肠内缘反3字形压迹,并有内缘肠粘膜破坏。胰体、胰尾癌进展期可侵犯十二指肠水平段,致局限性肠管狭窄、僵硬、粘膜破坏、钡剂通过受阻。CT因其无创、分辨率高,是首选的检查方法。其主要表现为胰腺局部增大,肿块形成,增强扫描时肿块密度增加不明显,胰管扩张,胰头癌时胆总管扩张呈所谓双管征。MRI检查除能横断位成像外,还能做MRCP检查,有其独特的价值。内镜逆行胰胆管造影是显示胰管影像的可靠方法,故诊断胰头癌相当有价值,胰头癌内镜逆行胰胆管造影典型改变以主胰管的狭窄阻塞;断端僵直、锐利;边缘受压、远端扩张为主要表现。

胰头癌侵犯胆总管特别是胆总管下端需注意与肝胰壶腹癌鉴别。肝胰壶腹癌多见息肉型,乳头状息肉突入胆管腔内或侵犯一侧管壁使管腔变形与胰头癌围管浸润的特点,X线征象不同,有时两者表现也极为相似,影像学检查难以区分,必要时应做内镜检查及临床综合分析判断。

\subsection{壶腹癌}

\begin{figure}[!htbp]
 \centering
 \includegraphics{./images/Image00328.jpg}
 \captionsetup{justification=centering}
 \caption{壶腹癌}
 \label{fig5-9-3}
  \end{figure} 

\textbf{【病史摘要】}
 男性,61岁。上腹部不适1年,近1周皮肤黄染,恶心、呕吐、食欲减退。体格检查:上腹部轻压痛,未扪及明显包块,肝右侧肋下触及,脾脏不大,心、肺阴性。

\textbf{【X线表现】}
 上消化道钡餐造影示:十二指肠下曲乳头部可见2cm×2cm大小的隆起性病变,边缘比较光滑,内侧缘肠壁稍显僵直。

\textbf{【X线诊断】}  壶腹癌(肿瘤型)。

\textbf{【评  述】}
 本例患者经手术治疗,病理证实为壶腹部低分化腺癌,侵及肠壁全层。壶腹癌为腺癌,位置在胰管与胆总管接合处,称为乏特氏乳头(papilla
of vater)。其发生率占所有胆管癌的8%,占所有壶腹周围癌的10%。

低张十二指肠钡餐造影X线表现为:①直接征象:肿瘤型壶腹癌壶腹的正常形态消失,代之以局限性不规则的充盈缺损,当浸润粘膜时,可见周围横行皱襞中断、破坏。溃疡型壶腹癌可见壶腹部边缘不规则的局限性充盈缺损,其内可见形态不规则的溃疡形成,多伴有粘膜皱襞的中断、破坏。②间接征象:可见因胆总管扩张或胆囊扩大,在十二指肠上部可见光滑的压迹。内镜逆行胰胆管造影(ERCP)对于壶腹部癌有明显诊断作用。其主要表现为胆总管末端可见局限性不规则充盈缺损,与低张十二指肠造影所见的充盈缺损部位一致,其上部胆总管扩张。

\protect\hypertarget{text00011.html}{}{}


\chapter{特殊人群用药}

\section{妊娠期和哺乳期妇女用药}

\subsection{妊娠期临床用药}

妊娠期由于母体变化、胎儿胎盘的存在及激素的影响,药物代谢和转运与非妊娠时期有很大差别。在全妊娠过程中,母体、胎盘、胎儿三者相互关联组成一个生物学、药物代谢动力学的组合单位。除极少数药物(例如胰岛素、肝素)不通过胎盘到胎儿,大多数药物均能通过胎盘进入胎儿体内。因此,孕妇用药必须了解药物的药代动力学,了解药物经胎盘到胎儿体内对胎儿及新生儿的药理作用,选择安全有效的药物,适时、适量地用药。

\subsubsection{药物在胎盘的转运与代谢}

胎盘由羊膜、属于子体部分的绒毛膜和属于母体部分的底蜕膜构成,将母血与胎儿血分开,称“胎盘屏障”。胎盘通透性与一般的血管生物膜相似,相当多的药物能够通过“胎盘屏障”进入胎儿体内。感染、缺氧常能破坏“胎盘屏障”,能使正常情况下不易通过“胎盘屏障”的抗生素容易通过。
\paragraph{影响药物通过胎盘的因素}

药物多以被动转运方式经胎盘转运,其速度受以下因素影响。①药物脂溶性高低。脂溶性药物,如安替比林及硫喷妥钠,能很快地以扩散方式通过胎盘。②药物分子的大小。较小分子量药物比大分子量药物扩散速度快。③药物离子化程度。④与蛋白结合能力。药物与蛋白质结合能力的高低与通过胎盘的药量成反比。⑤胎盘血流量。合并先兆子痫、糖尿病等全身性疾病的孕妇,麻醉或脐带受压迫时引起子宫胎盘血流量的改变,也可以使胎盘输送功能受到不同程度的影响,减缓药物转运。
\paragraph{药物在胎盘的代谢}

有些药物需要在胎盘经过代谢转化,才能成为容易输送的物质。胎盘有无数有活力的酶系统,具有生物合成及降解药物的功能。有些药物通过胎盘代谢降低活性,有些药物则增加活性。如天然或人工合成的肾上腺皮质激素,皮质醇及泼尼松通过胎盘转化为失活的11-酮衍化物;地塞米松通过胎盘则不需要经过代谢就能进入胎儿体内。因此,为了治疗孕妇疾病可用泼尼松,治疗胎儿疾病宜应用地塞米松。胎盘能代谢的仅限于几类酶所作用的物质,主要承担甾体类及多环碳氢化合物的代谢。

\subsubsection{母体药代动力学}
\paragraph{药物吸收}

妊娠期因孕激素影响胃肠系统的张力及活动力减弱,胃酸分泌减少,使口服药物的吸收延缓,达峰时间延长。但难溶性药物(如地高辛)因药物通过肠道的时间延长而生物利用度提高。

妊娠妇女由于肺潮气量和每分通气量明显增加,心排出量和肺血流量也增加,可使呼吸道吸入给药经肺泡摄取的药量增加。在妊娠妇女吸入麻醉时麻醉药的剂量通常应减少。
\paragraph{药物分布作用}

药物吸收后进入较非孕期增多的血浆、体液及脂肪组织中,使药物的分布容积增大,血药浓度低于非妊娠期。
\paragraph{药物与蛋白结合}

妊娠期虽然生成白蛋白的速度加快,但因血浆容积增加,形成生理性血浆蛋白低下。同时妊娠期很多蛋白结合部位被内泌素等物质所占据,所以使妊娠期药物蛋白结合能力下降,游离药物增多,药效和不良反应增强。
\paragraph{肝的代谢作用}

肝微粒体酶降解的药物可能减少,妊娠期高雌激素水平使胆汁在肝脏郁积,药物从胆汁排出减慢,从而使药物在肝脏清除减慢。
\paragraph{药物排出}

肾血流量及肾小球滤过率均增加,肾排泄药物或代谢产物加快,使主要以原形从尿中排出的药物消除加快,血药浓度不同程度降低(妊高征除外)。晚期妊娠期仰卧位时肾血流量减少而使由肾排出的药物作用延长,孕妇可采用侧卧位以促进药物的消除。

\subsubsection{药物在胎儿体内的转运与代谢}

胎儿体内的药物大部分经胎盘转运而来,也有少量药物经羊膜转运进入羊水中,而被胎儿吞饮经胃肠道吸收,或直接经皮肤吸收。
\paragraph{肝脏中的代谢}

药物通过胎盘经脐静脉进入胎儿血循环中。胎儿肝脏中酶的水平为成年人的30%~50%,胎儿对药物代谢能力较成年人低,所以胎儿体内药物浓度较母体高。因胎儿肝细胞缺乏催化葡萄糖醛酸苷类生成的酶,对药物解毒能力很差,如巴比妥、水杨酸类和激素等,易在胎儿体内达到毒性浓度。
\paragraph{肝外的代谢}

胎儿肝脏以外的代谢部位为肾上腺,胎儿肾上腺有很高活性的细胞色素P-450,在胎儿肾上腺内代谢的酶作用物质可能与肝脏是相同的。
\paragraph{排泄}

胎儿的肾小球滤过率甚低,肾排泄药物功能极差。许多药物在胎儿体内排泄缓慢,容易造成蓄积,如氯霉素、四环素等药物在胎儿体内排泄速度较母体明显减慢。胎儿进行药物消除的主要方式是将药物或其代谢物经胎盘返运回母体,由母体消除。

\subsubsection{胎儿治疗学}

胎儿治疗学指妊娠期孕妇用药,其目的不为治疗孕妇,而是为了给胎儿用药。胎儿治疗学所选用药物应注意其药代动力学,必须是经胎盘转运到胎儿,未经胎盘代谢,保持原有药效作用。已证实有效的治疗药物,如预计要早产的孕妇,妊娠期用肾上腺皮质类固醇促使胎儿肺提前成熟,选用肾上腺皮质类固醇时用地塞米松而不用泼尼松。

\subsubsection{妊娠期合理用药的条件}

鉴于许多药物可以通过胎盘,故在用药前应考虑以下几点。

(1)采用疗效肯定、不良反应小且对于药物代谢有清楚说明的药物,避免使用尚难确定有无不良影响的新药。

(2)已证明药物对灵长目动物胚胎是无害的。但没有任何一种药物对胎儿的发育是绝对安全的。

(3)用药时需清楚地了解妊娠周期。因为很难确定何时是胚胎器官形成的最终时刻,所以用药最好能在妊娠足4个月以后开始,在怀孕的前3个月内应避免应用任何药物。

(4)用药需有明确指征。用可能对胎儿有影响的药物时,要权衡利弊后给药,只有药物对母亲的益处多于对胎儿的危险时才考虑在孕期用药。

\subsubsection{FDA颁布的药物对妊娠的危害等级标准}

(1)A级:在有对照组的研究中,妊娠3个月的妇女未见到对胎儿有危害的迹象(并也没有在其后的6个月有危害性的证据),可能对胎儿的影响甚微。

(2)B级:在动物繁殖性研究中(并未进行孕妇的对照研究),未见到对胎儿的不良影响。在动物繁殖性研究中发现有不良反应,但这些不良反应并未在妊娠3个月的妇女得到证实(也没有对其后6个月的危害性证据)。

(3)C级:动物研究证明对胎儿有危害性(致畸或杀死胚胎),但并未对对照组妇女进行研究,或没有对妇女和动物平行地进行研究。本类药物只有在权衡了对孕妇的好处大于对胎儿的危害后方可应用。

(4)D级:对胎儿的危害性有明确的证据,尽管有危害性,但孕妇用药后有绝对好处。例如孕妇受到死亡的威胁或患有严重疾病,应用其他药物虽然安全但无效,因此需要用此类药物。

(5)X级:在动物或人的研究中均表明它可造成胎儿异常,或根据经验认为对人或动物是有危害性的,给孕妇应用这类药显然无益。本类药物禁用于妊娠或即将妊娠的患者。

\subsection{哺乳期临床用药}

大部分药物均能从乳汁排出并能测出药物浓度,一般药物由乳汁排出的浓度低,不超过母体1d内药量的1%。如果哺乳期需要用药,而且是一种比较安全的药,应在婴儿哺乳后(即下次哺乳前3~4h)用药。个别药物在乳汁中可达到较高浓度,如甲硝唑、异烟肼、红霉素及磺胺类等药物,它们在乳汁中的浓度可达到乳母血药浓度的50%。有时也可利用药物进入乳汁来治疗乳儿疾病。如用苯海拉明治疗婴儿皮肤过敏性疾患时,可让母亲服用常用量(25~50mg),通过哺乳,乳儿可获得治疗量的药物。

\section{老年人用药}

老年人由于年龄的增长,其生理功能处于逐渐衰退的状况,肌体对于药物的吸收、生物转化和排泄功能等各项指标都在下降,对药物处置能力及药物的反应性相应降低。在用药过程中由于多种疾病的存在使药物的体内过程复杂化,且多种疾病的并存往往需要同时使用多种药物治疗,由此产生的药物相互作用不仅影响老年人的药物治疗效果,同时药物不良反应所带来的用药风险性也随之增加。

\subsection{老年人药代动力学改变}

\subsubsection{吸收}

老年人胃肠吸收功能减退,药物吸收减少。但由于胃肠蠕动减慢,药物在胃肠中停留时间及与肠道吸收表面接触时间均延长,故对大多数药物(被动转运吸收的药物)总吸收的影响不明显,老年人和成年人相比无明显差异。但对靠主动转运来吸收的药物(如铁、木糖、钙以及维生素B{1}
、B{2} 、B{12}
、C等),由于老年人吸收这些药物所需的酶和糖蛋白等载体分泌减少,故吸收机能减弱。由于药物在胃肠内滞留时间延长,对胃肠道刺激增加,胃肠道反应增加。

\subsubsection{分布}

老年人机体组成成分发生改变,细胞内液减少,身体总水量减少,脂肪组织增加。故水溶性药物分布容积减少,血药浓度增加,如吗啡、乙醇、水杨酸盐、青霉素等;脂溶性药物分布容积增大,作用持续较久,半衰期延长,易在体内蓄积中毒。如老年人使用利多卡因时毒性反应明显增加,70岁以上者发病率为80%。

老年人血浆白蛋白含量减少,病情严重或极度虚弱的老年人下降尤为明显。应用血浆白蛋白结合率高的药物时,血中游离型药物浓度增大,易出现不良反应。如华法林,老年人用成人剂量时不良反应大,有引起出血的危险。此类药物还有普萘洛尔、苯妥英钠、安定、保泰松、地高辛和水杨酸盐,用时应注意减量。

\subsubsection{代谢}

老年人肝细胞减少,肝微粒酶的活性降低,肝血流量减少,使代谢能力下降,药物代谢减慢,造成药物蓄积,引起不良反应。对肝清除率高、首过效应明显的药物影响尤为显著。如老年人服用利多卡因、咖啡因、氨基比林、普萘洛尔等,要注意减少用量,或延长服药的间隔时间。

\subsubsection{排泄}

老龄所致的最大药代动力学改变在于药物的排泄,是老年人发生药物中毒反应的最重要因素。人的年龄达到40岁后,肾小球滤过率和肾小管排泄能力按每年1%的速度降低。因此,老年人药物清除率降低,即使无肾脏疾病,使用主要经肾排泄的药物时,易在体内蓄积而造成中毒。如地高辛、氨基糖苷类抗生素、青霉素G、苯巴比妥、西咪替丁及磺酰脲类降糖药等,都可因肾功能减退而排泄减少,半衰期显著延长,并有蓄积中毒的危险,均应相应减少用量或延长给药间隔时间。

\subsection{与增龄相关的系统改变}

\subsubsection{药物的相互作用}

老年人多病,常多药并用,药物的相互作用不仅影响老年人的药物治疗效果,同时使不良反应发生率上升,用药风险性随之增加。

\subsubsection{疾病因素}

老年人某些疾病的发病率亦迅速增加,使药物在体内过程复杂化。

(1)神经系统:衰老时中枢神经有某种病理变化的缓慢发展,对用药的影响表现在:①因记忆力差,引起服药的差错增多,对需要有稳态血药浓度的药物易因漏服而出现症状或因过量而出现不良反应;②应慎用对神经有毒性的药物,以防毒性叠加;③对药物的反应性有变化,如服用地西泮有引起脑功能失调的报道。

(2)心血管系统:老年人应激时调节最大心律的能力下降;平均收缩压较高,对血压调节功能降低,易出现体位性低血压。应慎用降压药和利尿药,避免引起体位性低血压;应注意控制甲状腺功能亢进、感染(特别是肺部)等疾病和输液用量,避免加重充血性心力衰竭。

(3)肾脏:主要由肾脏消除的药物应调整剂量,同时应关注体液和电解质平衡的紊乱。

(4)消化系统:肝脏清除药物减慢,主要经肝脏消除的药物必要时需调整剂量;慎用易引起便秘的药物。

(5)血液系统:造血组织的总量有所减少,但血液成分的变化不明显。对用药的影响主要为慎用有骨髓抑制不良反应的药物。

\subsection{老年人在临床治疗中需特别注意的常用药物}

\subsubsection{抗菌药物}

(1)青霉素类:主要经肾清除,老年人肾功能减退引起其消除半衰期延长,血药浓度增高,易出现神经精神症状,如幻觉、抽搐、昏睡、知觉障碍等。当控制感染需较大剂量青霉素类时,必须减少剂量或延长给药间隔时间。肌酐清除率可以作为其可靠的衡量指标。老年人处理电解质平衡的能力低,要注意避免处方含钠青霉素类而致钠过多,而处方羧苄西林或替卡西林时应注意有无血钾过低。

(2)头孢菌素类:所有头孢菌素都会抑制肠道菌群产生维生素K,因此具有潜在的致出血作用。服用阿司匹林、华法林等抗凝药物的老年人在给予头孢菌素类药物抗感染时,尤其需密切监测凝血酶原时间的变化,以免发生出血等严重不良反应。

(3)氨基糖苷类:均有不可逆的耳毒性和不同程度的肾毒性,肌酐清除率降低,使药物排泄受到一定限制;对耳毒药物更为敏感,更易发生上述毒性反应。65岁以上老年人应慎用此类药物,临床上对确需使用氨基糖苷类药物的老年患者应考虑采用每日1次的给药方案,以减小其耳肾毒性。同时注意避免与呋塞米、依他尼酸、顺铂等其他耳、肾毒性药物联合应用。

(4)喹诺酮类:该类药物具有脂溶性,脑脊髓中浓度高,并抑制脑内抑制性递质γ-氨基丁酸与其受体的结合,从而增加中枢神经系统的兴奋性。老年人存在不同程度的脑萎缩或脑动脉硬化,且肾清除药物的能力降低,因此老年人静脉滴注喹诺酮类药物,引起精神紊乱或中枢神经系统兴奋等不良反应的发生率较年轻人高。

\subsubsection{地高辛}

地高辛是临床上治疗充血性心力衰竭的常用药物,但治疗窗窄,中毒反应严重。地高辛中毒的发生率随年龄增加而增高,因此,老年人使用地高辛时,需监测地高辛血药浓度,且老年人的地高辛血药浓度的治疗范围可适当降低(<2.0ng/mL)。

\subsubsection{镇静催眠药}

老年人感觉较为迟钝,智力反应减低,应用镇静催眠药更易发生不良反应。老年人使用巴比妥类药物会发生兴奋激动,不宜常规应用。老年人对地西泮的中枢抑制作用比年轻人更敏感,应用时需谨慎,给药的时间间隔要加长。

\subsubsection{氨茶碱}

氨茶碱是慢性支气管炎和心源性哮喘患者的常用药,被肝脏的混合功能酶代谢。老年人肝功能都有不同程度的降低,半衰期因此延长。所以老年人服用氨茶碱后容易出现氨茶碱中毒,表现出烦躁、呕吐、忧郁、记忆力减退、定向力差、心律失常、血压急剧下降等现象乃至死亡。静脉注射速度过快或浓度太高可引起心悸、惊厥等严重反应。因此,对于急性心肌梗死、低血压、甲状腺功能亢进的患者禁用。老年人应用氨茶碱一定要慎重,开始用药要小剂量试用,询问氨茶碱的用药史。一旦发现有胃部不适或兴奋失眠,可用安定、复方氢氧化铝等药物来对抗或停药。氨茶碱主要通过肝药酶CYP1A2代谢,当与CYP1A2酶抑制剂(如环丙沙星等喹诺酮类抗菌药物)联合用药时,适当减少茶碱给药剂量或调整给药间隔,并密切监测茶碱血药浓度,以避免茶碱血药浓度过高而引起不良反应。

\subsubsection{HMG-CoA还原酶抑制剂(他汀类)}

他汀类药物是目前最强有力的调脂药物,肌病和横纹肌溶解是他汀类的最严重不良反应。他汀类通过CYP3A4药酶代谢(普伐他汀类除外),如果联用CYP3A4抑制剂,如大环内酯类的红霉素或克拉霉素、唑类抗真菌药(伊曲康唑、氟康唑、酮康唑等)、贝特类调脂药,可潜在地引起肌病和随后的横纹肌溶解。应控制剂量,对高龄老人慎用或减量,尽量避免与CYP3A4酶抑制剂或贝特类药物联用,如必须与CYP3A4酶抑制剂合用可选择普伐他汀。用药期间定期检测肝肾功能及血清肌酸激酶,他汀类药物也是可以安全使用的。

\subsection{老年人临床合理用药原则}

\subsubsection{不应当随意加服药物}

如因情绪不稳定、过度紧张、过度疲劳、睡前用脑过度而影响睡眠,出现失眠的情况,可以通过改变生活方式、心理慰藉来改善睡眠障碍,不应滥用安眠药。

\subsubsection{减少用药数量和剂量}

《中国药典》规定60岁以上老年人用药剂量为成年人的3/4,中枢神经系统抑制药应当以成年剂量的1/2或1/3作为起始剂量。

\subsubsection{加强宣传教育}

加强对老年人合理用药的宣传教育,应告知其按医师嘱咐合理用药。

\subsubsection{必要的血药浓度监测指导用药}

一些安全范围很窄的药物,应当做血药浓度监测,以调整用药剂量或更换药物治疗,并做到给药方案的个体化,如抗心律失常药普鲁卡因胺。

\subsubsection{合理使用抗生素}

老年人抗生素应用频率很高,但由于老年人肾功能呈生理性退行性改变,药物排出减少,血药浓度易在体内增高,易产生不良反应,一般用正常治疗剂量的2/3~1/2为宜,包括头孢菌素、青霉素等β-内酰胺类,应尽量减少用毒性大的抗菌药物如万古霉素及氨基糖苷类等品种。

\subsubsection{传统医药的应用}

老年人常服用补虚扶正中成药,但也不应太过或随意服用,一般提倡应用调补药品。但所用补药剂量不应过大。老年人用中药也宜从小剂量开始,因人、因时、因地不同而辨证论治。

\subsubsection{中西药的相互作用问题}

临床配合应用中西药物的现象很多,但对其相互作用研究不多。如在中药汤药中,习惯用甘草调和诸药,如果患者同时用呋塞米等利尿药,血钾浓度可能下降;并用降糖药者,其效用可能减低。

\section{儿科用药}

从胎儿到青春期(14岁)为儿科范围,药物治疗是儿科防病治病的主要手段。因小儿正处于生长发育的重要时期,所以用药时须特别注意。小儿时期具体包括新生儿期(出生至生后28d)、婴儿期、幼儿期、学龄前期、学龄期和少年期,应注意不同年龄分期,结合儿童的具体情况,如营养状况、体质等,根据药物的性质、用药方式作调整,才能取得满意效果。给药途径取决于病情的轻重缓急、用药目的及药物本身性质。一般情况下,有小儿剂型的药物不使用成人剂量分成几等份后服用。

\subsection{小儿药代动力学特点}

小儿时期,其器官和组织均处在不断发育和成熟的过程,新生儿期尤其是一个特殊阶段。为保证用药安全,应根据小儿生理的特征及药物在体内的药动学和药效学特点合理选择药物。

\subsubsection{吸收}

新生儿胃排空时间长,通过胃肠道吸收的药物比成人慢,肠壁相对长而薄,通透性高,可使一些药物的吸收增加。各种消化液分泌量少,胃液及酶的浓度小,消化能力弱。胃酸pH值较高,对遇酸不稳定的青霉素分解少,吸收好;对弱酸性药物,由于在胃液中解离增加,吸收少。

\subsubsection{分布}

小儿的体液总量和细胞外液量较成人比例高,可影响药物的分布。新生儿体脂含量低,脂溶性药物不能充分与之结合,因而分布容积小,游离药物浓度高,易出现中毒。同时,新生儿脑组织占身体比重较大,血脑屏障发育不完善,使脂溶性药物易进入大脑,出现神经系统反应。新生儿药物血浆蛋白的结合力低于成年人,营养不良和低蛋白血症的新生儿更低,对某些药物的敏感性增加。因此,在成人被认为是安全的、很低的血浆药物浓度,对新生儿可能引起不良反应。

\subsubsection{代谢}

新生儿肝容积与体重的比例较成人大,部分酶类较成人多,使一些完全在肝脏代谢的药物血浆半衰期缩短。同时,新生儿肝内混合功能氧化酶和化合酶代谢药物的活性比成人低得多,又使很多药物代谢缓慢,血浆半衰期延长。除了代谢程度,新生儿药物代谢与成人相比还有本质上的差别,如新生儿使用茶碱,将有一部分代谢为咖啡因,还需考虑咖啡因的药理作用。所以,对新生儿用药时,应考虑品种和剂量的选择,以防药物蓄积中毒。

\subsubsection{排泄}

新生儿的肾小球滤过功能和肾小管的分泌功能均不足,对主要在肾脏排泄的药物清除慢,引起中毒的危险,药物剂量需进行调整。

\subsection{小儿用药的特殊反应}

\subsubsection{药物敏感性和耐受性改变}

新生儿对酸、碱和水电解质平衡的调节能力差,过量水杨酸类药物可致酸中毒,利尿剂可致缺钠或缺钾,氯丙嗪易引起麻痹性肠梗死,氯霉素致灰婴综合征和再生障碍性贫血,长期使用皮质激素易引起胰腺炎等。

儿童对铁盐耐受性很差,成年人可耐受50g之多,而婴儿口服1g即可引起严重中毒反应,2g以上可致死,原因是可溶性铁盐引起婴幼儿肠道黏膜的损伤,甚至严重呕吐、腹泻、胃肠出血导致失水、休克。

小儿应用解热镇痛药后,可因体温骤降、出汗引起虚脱。应注意的是,阿司匹林、吲哚美辛可收缩血管,使新生儿动脉导管迅速关闭,致肺动脉高压,使新生儿病死率增加。解热镇痛药之间有交叉过敏反应,如对阿司匹林过敏,应用吲哚美辛、萘普生等也可能过敏。所以,在用药的过程中,要密切观察,防止少数患儿因过敏致死。

儿童因苯巴比妥过敏反应较多,故很少用。使用苯妥英钠常见的不良反应是癫痫
发作频率增加,如果此时未检测血药浓度,则往往认为是剂量不足,再增加剂量则症状更显著,所以使用也相对较少。通常用丙戊酸钠,其不良反应发生率较低,但有肝毒性,2岁以下儿童在合用其他抗癫痫
药时较易发生,用药期间应注意监测肝功能。

\subsubsection{溶血反应}

主要发生在红细胞G-6-PD缺乏的新生儿,使用维生素K、磺胺类、噻嗪类、利尿类、萘啶类、呋喃唑酮等有氧化性的药物时,可使红细胞膜发生破裂,引起溶血。

已知在妊娠后期,临床应用容易引起新生儿溶血或黄疸的药物包括较大剂量的解热镇痛药,如非那西丁、阿司匹林、氨基比林、安替比林、辛克芬;抗疟疾药,如奎宁、伯氨喹等;抗微生物药物,如头孢菌素类、青霉素、新生霉素、金霉素、氯霉素、四环素;中枢神经系统抑制剂,如吩噻嗪类、地西泮、苯巴比妥、苯妥英、乙醇、氯仿、水合氯醛;洋地黄毒苷类;性激素类等。其中,可引起免疫性溶血的药物包括青霉素、头孢菌素、磺胺药、异烟肼、奎宁、甲基多巴、非那西丁等。

\subsubsection{核黄疸}

新生儿本来就有黄疸的因素,在使用一些与胆红素竞争血浆蛋白的药物时,血中游离的胆红素升高,进入脑内与基底核结合导致胆红素脑病或核黄疸。维生素K{3}
、磺胺类、新生霉素都能影响胆红素代谢,加重新生儿黄疸,故慎用。

\subsubsection{神经系统反应}

新生儿血脑屏障发育不成熟,药物易透过血脑屏障直接作用于脆弱的中枢神经系统,引起神经系统反应。如阿片类药物易引起呼吸抑制;抗组胺药、苯丙胺、氨茶碱、阿托品可致昏迷及惊厥;皮质激素易引起手足抽搐;氨基糖苷类抗生素易引起听神经损伤等。

\subsubsection{牙色素沉着}

四环素、多西环素、米诺环素等可沉积于骨组织和牙齿,引起永久性色素沉着,如牙齿发黄,四环素还可抑制骨的生长发育。故妊娠4个月后,哺乳期妇女和8岁以下儿童除眼科局部用药外,不得应用四环素。
\chapter{呼吸系统疾病}

\chapterabstract{本章主要介绍呼吸系统常见疾病(慢性阻塞性肺疾病、肺炎、硅肺、慢性肺源性心脏病、鼻咽癌、肺癌)的病理形态特点和临床病理联系,对这些常见疾病的病因及发病机制也进行了简单的介绍。要求掌握慢性支气管炎、肺气肿、肺心病的病理变化;大叶性肺炎的病理变化、病变性质及并发症;小叶性肺炎的病变性质及病变特征;肺癌分型及病理变化。熟悉大叶性肺炎的主要临床病理联系;小叶性肺炎的病因、发病机制及临床病理联系;了解慢性支气管炎、肺气肿、肺心病、大叶性肺炎的病因及发病机制;病毒性肺炎(包括SARS)和支原体性肺炎的病因、发病机制、病理变化及临床病理联系;了解硅肺的发病机制及病理变化。}

呼吸系统是通气和换气的器官。终末细支气管以上为气体传导部分,呼吸性细支气管、肺泡管、肺泡囊为换气部分。传导性气道管壁被覆纤毛柱状上皮。肺泡由I型和Ⅱ型肺泡上皮细胞覆盖。黏液-纤毛排送系统是呼吸道的主要防御功能之一。肺泡巨噬细胞又称为尘细胞,是肺内重要的防御细胞,不仅具有吞噬功能,还可摄取和处理抗原、增强淋巴细胞的免疫活性。此外,呼吸道分泌物中的溶菌酶、补体系统、干扰素和分泌型IgA等也具有增强局部免疫力的作用。

呼吸系统与外界相通,极易受外界环境中有害物质的作用诱发疾病。肺又是全身血液循环必经之处,因此许多疾病常可以并发肺部病变。一些自身免疫或代谢性全身疾病,如系统性红斑狼疮、类风湿关节炎等都可累及肺部,因而呼吸系统疾病比较常见。由于大气污染、吸烟、人口老龄化及其他因素使慢性阻塞性肺疾病、肺癌、肺部弥散性间质纤维化、慢性肺源性心脏病等的发病率、死亡率日趋增多。

\section{慢性阻塞性肺疾病}

慢性阻塞性肺疾病(chronic obstructive pulmonary
disease,COPD)是一组由多种原因引起的以持续气流受限为特征的慢性阻塞性气道疾病的总称,以呼气性呼吸困难为特征,病人常因肺功能不全、肺动脉高压等死亡。属于这组疾病的有慢性支气管炎、肺气肿、支气管扩张症和支气管哮喘等疾病,以华北及东北地区多见。因篇幅有限,在此只介绍前三种疾病。

\subsection{慢性支气管炎}

慢性支气管炎(chronic
bronchitis)是气管、支气管黏膜及周围组织的慢性非特异性炎症。临床以咳嗽、咳痰、喘息为主要症状,每年发病持续三个月,连续两年或两年以上。以老年男性多见,冬春季节高发。病情缓慢发展,易并发阻塞性肺气肿,肺动脉高压和肺源性心脏病。是严重威胁人体健康的常见疾病。

\subsubsection{病因及发病机制}

\paragraph{吸烟}
吸烟为发病的主要因素。香烟中的焦油、尼古丁和氰氢酸等可损伤呼吸道黏膜上皮细胞,导致气道净化功能下降并能刺激黏膜下感受器,使副交感神经功能亢进,引起支气管平滑肌收缩,导致气道阻力增加以及腺体分泌增多。杯状细胞增生、支气管黏膜充血水肿、黏液积聚容易诱发感染。此外,香烟烟雾还可使毒性氧自由基产生增多,诱导中性粒细胞释放各类蛋白水解酶,破坏肺弹力纤维,诱发肺气肿的发生。研究表明,吸烟者慢性支气管炎的患病率较不吸烟者高2~8倍,烟龄越长,烟量越大,患病率亦越高。

\paragraph{空气污染}
空气中二氧化硫、二氧化氮、氯气及臭氧等对气道黏膜上皮均有刺激和细胞毒作用。二氧化硅、煤尘、蔗尘、棉屑等损伤支气管黏膜,使肺清除功能遭受损害,引起细菌感染。

\paragraph{感染因素}
感染是慢性支气管炎发生和发展的重要因素之一。细菌、病毒和支原体感染为本病急性发作的主要原因。病毒感染以流感病毒、鼻病毒、腺病毒和呼吸道合胞病毒常见;细菌感染以肺炎链球菌、流感嗜血杆菌及葡萄球菌多见,常继发于病毒或支原体感染。

\paragraph{过敏因素}
喘息型慢性支气管炎患者多有过敏史,痰液中嗜酸性粒细胞数量和组胺含量和血中IgE具有增多的趋向。部分患者血清中类风湿因子阳性以及T淋巴细胞亚群分布异常等,提示过敏因素与本病的发生有关。

\paragraph{其他}
慢性支气管炎急性发作在冬季较多见。寒冷空气可刺激腺体分泌黏液增加和纤毛运动,减弱、削弱气道的防御功能,还可引起支气管平滑肌痉挛、黏膜血管收缩、局部血循环障碍。大多患者具有自主神经功能失调的现象,部分患者副交感神经功能亢进,气道反应性较正常人增高。老年人肾上腺皮质功能减退、细胞免疫功能受损、溶菌酶活性降低、营养低下、维生素A、维生素C不足等均可使气道黏膜血管通透性增加和上皮修复功能减退。遗传因素是否与慢性支气管炎的发病有关迄今尚未证实。

\subsubsection{病理变化}

\paragraph{呼吸道上皮的损伤与修复}
由于炎性渗出和黏液分泌增多,使黏膜上皮的纤毛因负荷过重而发生粘连、倒伏,甚至脱失。上皮细胞变性、坏死。病变严重或持续过久,可发生鳞状上皮化生(图\ref{fig7-1})。

\begin{figure}[!htbp]
 \centering
 \includegraphics{./images/Image00110.jpg}
 \captionsetup{justification=centering}
 \caption{慢性支气管炎(HE染色,中倍)\\ {\small 支气管黏膜上皮变性、坏死,鳞状上皮化生}}
\label{fig7-1}
  \end{figure}

\paragraph{呼吸道腺体的病变}
为支气管炎的形态学特征。表现为黏膜上皮层内杯状细胞增多;黏液腺泡增生、肥大;浆液腺泡黏液化,黏液分泌增多。

\paragraph{管壁组织的损害}
急性发作时,黏膜层及黏膜下层充血、水肿,淋巴细胞、浆细胞及中性粒细胞浸润。炎症反复发作可破坏平滑肌、弹力纤维和软骨。支气管黏膜发生溃疡,管壁结缔组织增生,管腔狭窄,管腔内黏液栓潴留致气道阻塞。局部管壁塌陷、扭曲、变形。

\subsubsection{病理临床联系}

慢支患者缓慢起病,病程长,反复急性发作而病情加重。主要症状为咳嗽、咳痰,或伴有喘息。急性加重系指咳嗽、咳痰、喘息等症状突然加重,主要原因是呼吸道感染。由于黏液的分泌增多,痰液和炎症刺激支气管黏膜而引起咳嗽、咳痰。痰呈白色泡沫状,并发感染时可呈脓性。肺部听诊可闻及干湿性啰音。由于支气管黏膜肿胀、痰液阻塞和平滑肌痉挛,可出现哮喘样发作。

\subsubsection{并发症}

慢性支气管炎反复发作,病变逐渐加重。炎症向肺泡及支气管壁周围扩展,导致细支气管周围炎,还可发生纤维闭塞性支气管炎,以后引起阻塞性肺气肿、支气管扩张症,最终导致肺源性心脏病。长期炎症刺激可引起气管和支气管黏膜上皮发生鳞状上皮化生,进而发生不典型增生,最终恶变为鳞状细胞癌。

\subsection{支气管扩张症}

支气管扩张症(bronchiectasis)是由于支气管及其周围肺组织慢性化脓性炎症和纤维化,使支气管壁的肌肉和弹性组织破坏,导致支气管变形及持久扩张。典型的症状有慢性咳嗽、咳大量脓痰和反复咯血。主要致病因素为支气管感染、阻塞和牵拉,部分有先天遗传因素。患者多有麻疹、百日咳或支气管肺炎等病史。随着人民生活的改善,麻疹、百日咳疫苗的预防接种,以及抗生素的临床应用,本病的发病率大为减少。

\subsubsection{病因}

\paragraph{感染}
感染是引起支气管扩张的最常见原因。儿童时期麻疹、百日咳、流行性感冒(某些腺病毒感染)或严重的肺部感染如肺炎克雷白杆菌、葡萄球菌、流感病毒、真菌、分枝杆菌以及支原体感染,使支气管各层组织尤其是平滑肌纤维和弹性纤维遭到破坏,黏液纤毛清除功能降低,削弱了管壁的支撑作用,可继发支气管扩张。

\paragraph{支气管阻塞}
支气管由于受肿瘤、肿大淋巴结的压迫,或因腔内异物而发生不完全阻塞,使阻塞处以下的支气管腔内压力不断增大,管壁受损、管腔扩张。支气管完全阻塞时,其所属肺泡腔内空气被吸收而萎缩,该部位支气管壁受胸腔负压的牵拉而扩张。

\paragraph{先天性和遗传性疾病}
先天性较少见,是由于先天性支气管发育不良,存在先天性缺陷或遗传性疾病,使肺的外周不能进一步发育,导致已发育支气管扩张,如支气管软骨发育不全(Williams-Camplen综合征)。有的病人支气管扩张在出生后发生,但也有先天异常因素存在,如Kartagener综合征,患者除支气管扩张外可伴有内脏异位和胰腺囊性纤维化病变。支气管扩张症也可见于Young综合征,特征为阻塞性精子缺乏,慢性鼻窦炎,反复肺部感染和支气管扩张。部分支气管扩张病人显示免疫球蛋白缺陷,易于反复细菌感染。

\subsubsection{病理变化}

肉眼观:支气管扩张多发生于肺段及段以下支气管(Ⅲ~Ⅳ级支气管及细支气管)。常见于一个肺段,也可在双侧多个肺段发生。左肺较右肺多见,特别见于左肺下叶。扩张部支气管腔明显扩大,形态可分为圆柱状、囊状两种,亦常混合存在。柱状扩张者管壁破坏较轻,随着病变发展,破坏严重,乃出现囊状扩张(图\ref{fig7-2})。严重者肺组织呈蜂窝状。扩张的管腔内充满黄绿色黏稠脓性或血性渗出物。管壁黏膜萎缩或增生、肥厚,形成纵行皱襞。

\begin{figure}[!htbp]
 \centering
 \includegraphics{./images/Image00111.jpg}
 \captionsetup{justification=centering}
 \caption{支气管扩张症\\ {\small 支气管呈圆柱状扩张,直达胸膜}}
\label{fig7-2}
  \end{figure}

镜下观:支气管呈慢性化脓性炎症,并伴有不同程度的组织破坏及管壁纤维化、瘢痕化。支气管扩张症易发生反复感染,其炎症可蔓延到邻近的肺实质,引起不同程度的肺炎、小脓肿或肺小叶不张以及慢性支气管炎的病变,久之可形成肺纤维化和阻塞性肺气肿,上述变化又加重支气管扩张。

\subsubsection{临床病理联系}

支气管扩张症的典型症状为慢性咳嗽、大量脓痰和反复咯血。咳嗽、咳痰主要是慢性炎症的刺激、黏液分泌增多及继发化脓菌感染所致。咳嗽和痰量与体位改变有关,尤其是清晨起床可咳出大量脓性痰,若有厌氧菌感染,则有臭味。咯血为小血管炎性破坏及咳嗽所致。有些患者仅表现为反复咯血,平时无咳嗽、脓痰等呼吸道症状,临床上称为“干性支气管扩张”。并发胸膜炎时,可出现胸痛。慢性重症支气管扩张症,肺组织广泛纤维化,病人可出现气急、发绀、杵状指(趾)。

\subsubsection{并发症}

支气管扩张症常见的并发症有肺脓肿、脓胸、脓气胸等。病灶内细菌经血道播散可到达远处器官,最常见的是引起脑膜炎、脑脓肿;由于抗菌药物的运用,此种情况已较少见。严重的支气管扩张致肺组织广泛纤维化,破坏肺血管床或形成支气管动脉与肺动脉分支吻合,则可导致肺动脉高压,引起肺心病。此外,在鳞状上皮化生的基础上可发生鳞状细胞癌。

\subsection{慢性阻塞性肺气肿}

慢性阻塞性肺气肿(chronic obstructive pulmonary
emphysema)是由于慢性支气管炎等引起呼吸性细支气管以远的末梢肺组织因残气量增多而呈持久性扩张,并伴有肺泡间隔破坏,以致肺组织弹性减弱,容积增大的阻塞性肺病。

\subsubsection{病因和发病机制}

肺气肿是支气管和肺疾病常见的并发症,与吸烟、空气污染、小气道感染、尘肺等关系密切,尤其是慢性阻塞性细支气管炎是引起肺气肿的重要原因。发病机制与下列因素有关:

\paragraph{阻塞性通气障碍}
慢性细支气管炎时,由于小气道的狭窄、阻塞或塌陷,导致阻塞性通气障碍,使肺泡内残气量增多。细支气管周围炎症使肺泡壁破坏、弹性减弱,末梢肺组织残气量不断增多而发生扩张,肺泡孔扩大,肺泡间隔断裂,扩张的肺泡互相融合形成气肿囊腔。此外,炎症损伤细小支气管壁软骨,细支气闭塞时,吸入的空气可经存在于细支气管和肺泡之间的Lambert孔进入闭塞远端的肺泡内(即肺泡侧流通气),而呼气时,Lambert孔闭合,空气不能排出,导致肺泡内储气量增多、肺泡内压增高。

\paragraph{弹性蛋白酶增多、活性增高}
与肺气肿发生有关的内源性蛋白酶主要是中性粒细胞和单核细胞释放的弹性蛋白酶。慢性支气管炎伴有肺感染尤其是吸烟者,肺组织内渗出的中性粒细胞和单核细胞较多,可释放多量弹性蛋白酶。此酶能降解肺组织中的弹性硬蛋白、结缔组织基质中的胶原和蛋白多糖,破坏肺泡壁结构。

\paragraph{遗传性α{1}-抗胰蛋白酶(α{1}-AT)缺乏}  α{1}
-抗胰蛋白酶由肝细胞产生,是一种分子量为45 000~56
000的糖蛋白,它能抑制蛋白酶、弹性蛋白酶、胶原酶等多种水解酶的活性。该酶缺失则增强了弹性蛋白酶的损伤作用。遗传性α{1}
-抗胰蛋白酶缺乏是引起原发性肺气肿的原因,α{1}
-抗胰蛋白酶缺乏的家族,肺气肿的发病率比一般人高15倍,主要是全小叶型肺气肿。

\subsubsection{病理变化}

\paragraph{肉眼观}
肺显著膨大,边缘钝圆,色泽灰白,表面常可见肋骨压痕,肺组织柔软而弹性差,指压后的压痕不易消退,触之捻发音增强。表面可见多个大小不一的大泡(图\ref{fig7-3})。

\begin{figure}[!htbp]
 \centering
 \includegraphics{./images/Image00112.jpg}
 \captionsetup{justification=centering}
 \caption{肺气肿}
 \label{fig7-3}
  \end{figure} 

\paragraph{镜下观}
肺泡扩张,间隔变窄,肺泡孔扩大,肺泡间隔断裂,扩张的肺泡融合成较大的囊腔。肺毛细血管床明显减少,肺小动脉内膜呈纤维性增厚。小支气管和细支气管可见慢性炎症。根据扩张部位又可分为小叶中央型、小叶周围型和全小叶型肺气肿(图\ref{fig7-4})。

(1)小叶中央型肺气肿:是临床最常见的一型。病变累及肺小叶的中央部分,呼吸性细支气管病变最明显,呈囊状扩张。而肺泡管、肺泡囊变化则不明显。

(2)全小叶型肺气肿:病变累及肺小叶的各个部位,从终末呼吸细支气管直至肺小叶和肺泡均呈弥漫性扩张,遍布于肺小叶内。如果肺泡间隔破坏较严重,气肿囊腔可融合成直径超过1
cm的大囊泡,形成大泡性肺气肿。

(3)小叶周围型肺气肿:病变主要累及肺小叶远端部位的肺泡囊,而近端部位的呼吸性细支气管和肺泡管基本正常。常合并有小叶中央型和全小叶型肺气肿。

肺气肿的气肿囊泡为扩张的呼吸性细支气管,在近端囊壁上常可见呼吸上皮(柱状或低柱状上皮)及平滑肌束的残迹。全小叶型肺气肿的气肿囊泡主要是扩张变圆的肺泡管和肺泡囊,有时还可见到囊泡壁上残留的平滑肌束片断,在较大的气肿囊腔内有时还可见含有小血管的悬梁。

\begin{figure}[!htbp]
 \centering
 \includegraphics{./images/Image00113.jpg}
 \captionsetup{justification=centering}
 \caption{慢性阻塞性肺气肿类型模式图}
 \label{fig7-4}
  \end{figure} 

\subsubsection{临床病理联系及转归}

肺气肿患者的主要症状是气短,轻者仅在体力劳动时发生,随着气肿程度加重,气短逐渐明显,甚至休息时也出现呼吸困难,并常感胸闷。每当合并呼吸道感染时,症状加重,并可出现缺氧、酸中毒等一系列症状。患者胸廓前后径增大,呈桶状胸。胸廓呼吸运动减弱。叩诊呈过清音,心浊音界缩小或消失,肝浊音界下降。语音震颤减弱。听诊时呼吸音减弱,呼气延长,用力呼吸时两肺底部可闻及湿啰音和散在的干啰音。剑突下心音增强,肺动脉瓣第二音亢进。

肺气肿严重时可引起肺源性心脏病及衰竭。肺大泡破裂后引起自发性气胸,并可导致大面积肺萎陷。由于外呼吸功能严重障碍,导致呼吸衰竭及肺性脑病。呼吸衰竭时发生的低氧血症和高碳酸血症会引起各系统的代谢功能严重紊乱。中枢神经系统对缺氧最为敏感,随着缺氧程度的加重,可出现一系列中枢神经系统功能障碍,由开始的大脑皮层兴奋性增高而后转入抑制状态。病人表现由烦躁不安、视力和智力的轻度减退,逐渐发展为定向和记忆障碍,精神错乱、嗜睡、惊厥以至意识丧失。

\section{肺源性心脏病}

肺源性心脏病(cor
pulmonale,)简称肺心病,主要由于支气管肺组织、胸廓或肺动脉血管病变所致肺循环阻力增加,肺动脉高压,导致右心室肥厚、扩张而引起的心脏病。根据起病缓急和病程长短,可分为急性和慢性两类。临床上后者多见,除原有肺、胸疾病的各种症状和体征外,主要是逐步出现肺、心功能衰竭以及其他器官损害的征象。

\subsection{病因}

\paragraph{支气管、肺疾病}
以慢支并发阻塞性肺气肿最为多见,占80%~90%,其次为支气管哮喘、支气管扩张、重症肺结核、尘肺、慢性弥漫性肺间质纤维化、结节病、过敏性肺泡炎、嗜酸性肉芽肿等。

\paragraph{胸廓运动障碍性疾病}
较少见,严重的脊椎后、侧凸,脊椎结核,类风湿关节炎,胸膜广泛粘连及胸廓形成术后造成的严重胸廓或脊椎畸形,以及神经肌肉疾患如脊髓灰质炎,可引起胸廓活动受限、肺受压、支气管扭曲或变形,导致肺功能受限,气道引流不畅,肺部反复感染,并发肺气肿,或纤维化、缺氧、肺血管收缩、狭窄,使阻力增加,肺动脉高压,发展成肺心病。

\paragraph{肺血管疾病}
甚少见。累及肺动脉的过敏性肉芽肿病,广泛或反复发生的多发性肺小动脉栓塞及肺小动脉炎,以及原因不明的原发性肺动脉高压症,均可使肺小动脉狭窄、阻塞,引起肺动脉血管阻力增加、肺动脉高压和右心室负荷加重,发展成肺心病。

\subsection{发病机制}

上述任何因素引起肺心病的关键环节都是肺动脉高压,其发病机制如下:

\paragraph{肺毛细血管床显著减少}
慢性肺气肿或肺广泛纤维化,使肺泡壁毛细血管受压、扭曲变形,甚至管腔狭窄或闭锁,肺毛细血管床总横断面积减少,从而肺循环阻力增加,肺动脉压升高。

\paragraph{肺内血管分流}
在慢性肺部疾病时,肺泡壁毛细血管受压闭塞,或因肺组织广泛纤维化,肺循环的正常途径受阻,使肺动脉和支气管动脉之间的吻合支开放,压力高的支气管动脉血流入压力低的肺动脉系统,引起肺动脉压升高。

\paragraph{肺通气、换气功能障碍}
严重的慢性肺部疾病可引起肺通气、换气功能障碍,导致缺氧、高碳酸血症和呼吸性酸中毒,使小动脉痉挛收缩,并刺激血管平滑肌细胞,使之增生、肥大,血管壁增厚,引起肺循环阻力增大,肺动脉压升高。慢性缺氧还可产生继发性红细胞增多、血液黏稠度增加,血流阻力随之增高。缺氧还引起肾动脉收缩,肾血流量减少,醛固酮分泌增加,从而引起水、钠潴留,血容量增多,更使肺动脉压升高。临床上缺氧和高碳酸血症得到纠正后,肺动脉压可明显降低,部分病人甚至可恢复到正常范围。

\subsection{病理变化}

肺心病是多种慢性肺部疾病的晚期并发症,形成肺血管改变和右心室肥厚扩张。因此,肺心病的病理变化包括晚期肺部疾病、肺血管和心脏三种病变。肺部疾病详见有关章节,此处仅叙述肺血管和心脏病变。

\subsubsection{肺血管病变}

肺小动脉及其分支的病变在肺动脉高压的形成中起着重要作用,表现为:①肺小动脉中膜平滑肌增生肥大,细胞外基质合成增多,致肺小动脉中膜肥厚,使肺小动脉管壁增厚、变硬,管腔狭窄,肺循环阻力增加。②无肌细动脉肌化:持续缺氧可以刺激肺泡毛细血管前的无肌细动脉管壁平滑肌增生。③肺小、细动脉内膜下胶原纤维增生,并出现纵行肌束。上述改变使肺小、细动脉管壁增厚,管腔狭窄(图\ref{fig7-5})。④肺小动脉炎:若肺部炎症累及邻近的肺小动脉,引起血管的急、慢性炎,致使病变处血管管壁增厚、管腔狭窄或纤维化。⑤肺泡壁毛细血管床数量显著减少。

\begin{figure}[!htbp]
 \centering
 \includegraphics{./images/Image00114.jpg}
 \captionsetup{justification=centering}
 \caption{慢性肺源性心脏病肺小动脉硬化\\ {\small 镜下见肺细动脉管壁增厚,管腔狭窄(HE染色,中倍)}}
\label{fig7-5}
  \end{figure}

\subsubsection{心脏病变}

肉眼观:心脏体积明显增大,重量增加,平均为326 g,最重者可达785
g。右心室壁显著肥厚,后期心腔扩张。心尖钝圆,肺动脉圆锥隆起(图\ref{fig7-6}A),肺动脉瓣下2
cm处右心室肌壁厚度超过0.5 cm。

镜下观:右心室心肌纤维肥大(图\ref{fig7-6}B),可见肌浆溶解、变性、坏死,间质水肿和结缔组织增生。

\begin{figure}[!htbp]
 \centering
 \includegraphics{./images/Image00115.jpg}
 \captionsetup{justification=centering}
 \caption{慢性肺源性心脏病}
 \label{fig7-6}
  \end{figure} 

\subsection{临床病理联系}

肺心病进展缓慢,开始主要表现为原来肺部疾病的症状。随着病变加重,肺动脉压升高,右心负荷增加,患者出现心慌、气急、发绀及下肢水肿、肝肿大等右心力衰竭的症状和体征。重症肺心病,由于呼吸功能衰竭所致缺氧、二氧化碳潴留可引起肺性脑病,患者表现为头痛、烦躁、精神错乱、意识不清和昏迷等,是肺心病患者重要致死原因。

\section{肺炎}

肺炎(pneumonia)通常是指肺的急性渗出性炎症,为呼吸系统的多发病、常见病。在我国各种致死病因中,肺炎占第5位。肺炎按病因可分为感染性肺炎,如细菌性肺炎、病毒性肺炎、支原体肺炎、真菌性肺炎及其他病原体包括立克次体、肺炎衣原体、寄生虫(如弓形体、卡氏肺孢子虫、肺包虫、肺吸虫)等引起的肺炎等。理化因素引起的肺炎,如放射性、吸入性、类脂性肺炎以及变态反应性(如过敏性和风湿性)肺炎等。由于致病因子和机体反应性的不同,炎症发生的部位、累及范围和病变性质也往往不同。炎症发生于肺泡内者称肺泡性肺炎(大多数肺炎为肺泡性),累及肺间质者称间质性肺炎。病变范围以肺小叶为单位者称小叶性肺炎,累及肺段者称节段性肺炎,波及整个或多个大叶者称大叶性肺炎。按病变性质可分为浆液性、纤维素性、化脓性、出血性、干酪性、肉芽肿性或机化性肺炎等不同类型。

\subsection{大叶性肺炎}

大叶性肺炎(lobar
pneumonia)主要是由肺炎链球菌感染引起的肺组织的急性纤维素性渗出性炎症。病变起始于肺泡,并迅速扩展至整个或多个大叶。多见于青壮年,好发于冬、春季节。临床表现为骤然起病、寒战高热、胸痛、咳嗽、咳铁锈色痰、呼吸困难,并有肺实变体征及白细胞增高等。典型病程7~10天。

\subsubsection{病因和发病机制}

95%以上的大叶性肺炎由肺炎链球菌引起,尤以Ⅲ型者毒力最强。此外,肺炎杆菌、金黄色葡萄球菌、溶血性链球菌、流感嗜血杆菌也可引起。本病主要经呼吸道感染,受寒、疲劳、醉酒、感冒、麻醉、糖尿病、肝肾疾病等均为其诱因。此时,呼吸道防御功能被削弱,机体抵抗力降低,易发生细菌感染。细菌侵入肺泡后在其中繁殖,引起肺泡壁水肿,继而白细胞渗出和红细胞漏出,特别是形成的浆液性渗出物有利于细菌繁殖,并使细菌通过肺泡间孔(Cohn孔)或呼吸性细支气管迅速向邻近肺组织蔓延,从而波及肺段或整个大叶。在肺大叶之间的蔓延则系带菌渗出液经叶支气管播散所致。

\subsubsection{病理变化及与临床联系}

病变一般多见于左肺下叶,也可同时或先后发生于两个以上肺叶。由于毛细血管通透性增高,大量纤维蛋白原渗出于肺泡,使肺组织大面积广泛实变。按自然病程可分为四期。

\paragraph{充血水肿期}
发病后1~2天,肺叶充血、水肿,暗红色,切开时有血性浆液自切面流出。镜下观:肺泡壁毛细血管扩张充血,肺泡腔内有大量浆液性渗出物,混有少数红细胞、中性粒细胞和巨噬细胞,并含有大量细菌(图\ref{fig7-7})。

临床上出现高热、寒战、白细胞增多等毒血症症状,听诊可闻及湿性啰音,X线检查病变处呈现淡薄、均匀的阴影,渗出物中可检出大量肺炎双球菌。

\begin{figure}[!htbp]
 \centering
 \includegraphics{./images/Image00116.jpg}
 \captionsetup{justification=centering}
 \caption{大叶性肺炎(充血水肿期)(HE染色,中倍)\\ {\small 肺泡壁毛细血管高度充血水肿,肺泡腔内充满浆液,其中混有少量红细胞和白细胞}}
\label{fig7-7}
  \end{figure}

\paragraph{红色肝样变期}
1~2天后,即有大量纤维蛋白原渗出。肉眼观:肺叶肿大,颜色暗红,质实如肝,切面颗粒状,为充塞于肺泡腔内的纤维素性渗出物突出于切面所致。病变肺叶的胸膜面常有纤维素性渗出物覆盖。镜下观:肺泡壁毛细血管显著充血,肺泡腔内充满混有红细胞、中性粒细胞、巨噬细胞的纤维素性渗出物,纤维素可穿过肺泡间孔与相邻肺泡中的纤维素相互连接成网状,有利于中性粒细胞和巨噬细胞的吞噬作用,防止细菌扩散(图\ref{fig7-8})。

\begin{figure}[!htbp]
 \centering
 \includegraphics{./images/Image00117.jpg}
 \captionsetup{justification=centering}
 \caption{大叶性肺炎红色肝样变期(HE染色,高倍)\\ {\small 肺泡壁毛细血管高度扩张、充血,肺泡腔中含大量纤维素、红细胞及少量白细胞}}
\label{fig7-8}
  \end{figure}

临床上,病人高热稽留不退,呼吸急促。由于红细胞破坏与崩解,被巨噬细胞吞噬,形成含铁血黄素,经痰液排出,使痰呈铁锈色。由于病变累及胸膜,病人常感胸痛。若病变范围较大,实变区的大量静脉血未能氧合便流入左心,引起血氧分压和氧饱和度降低,病人可出现发绀、呼吸困难等缺氧表现。胸部叩诊呈浊音,听诊闻及管性呼吸音和胸膜摩擦音,X线检查见大片致密阴影。渗出物中仍可检出肺炎双球菌。

\paragraph{灰色肝样变期}
在发病4~6天后,肺泡腔内纤维素性渗出物及中性粒细胞继续增加。肉眼观:病变肺叶质实如肝,明显肿胀,重量增加,呈灰白色。如血管损伤较重、出血较多,外观可呈红色。胸膜面仍有纤维素性渗出物覆盖。镜下观:肺泡腔内充满大量纤维素及中性粒细胞,红细胞大部分破坏溶解,被咳出或被吸收。由于肺泡腔内渗出物的压力,肺泡壁毛细血管受压而处于贫血状态(图\ref{fig7-9})。

\begin{figure}[!htbp]
 \centering
 \includegraphics{./images/Image00118.jpg}
 \captionsetup{justification=centering}
 \caption{大叶性肺炎灰色肝样变期}
 \label{fig7-9}
  \end{figure} 

临床上,痰呈脓性,因肺泡壁毛细血管受压,流经病变区的血流量减少,肺静脉血氧合不足的情况反而减轻,故缺氧状况有所改善。听诊及X线检查所见与红色肝样变期表现基本相同。由于特异性抗体的产生或吞噬作用的加强,此期肺炎双球菌大多已被消灭,故不易检出。

\paragraph{溶解消散期}
经5~10天,炎症消退。肉眼观:肺叶质地变软,色转灰红,切面颗粒状外观消失。细菌被吞噬细胞吞噬清除,渗出物被溶解,或经淋巴管吸收或被咳出。肺泡腔逐渐排空,重新充气。大叶性肺炎时,肺组织常无坏死,肺泡壁结构也未遭破坏,愈复后,肺组织可完全恢复其正常结构和功能。胸膜渗出物可完全吸收,否则可遗留胸膜增厚或粘连(图\ref{fig7-10})。

临床上病人体温下降,症状消退,由于渗出物的液化排出,肺部又可闻及湿性啰音。X线检查见病变处阴影密度减低,透亮度逐渐增加。2~3周后肺实变体征完全消失。

大叶性肺炎的一个重要特点是在整个病程中,肺泡壁结构通常未遭破坏,愈合后肺组织可完全恢复正常结构和功能。支气管的炎症病变轻微,仅有充血、点状出血和小支气管黏膜上皮脱落等,晚期可完全恢复正常。

\begin{figure}[!htbp]
 \centering
 \includegraphics{./images/Image00119.jpg}
 \captionsetup{justification=centering}
 \caption{大叶性肺炎消散期\\ {\small 肺泡腔内的渗出物逐渐溶解、吸收,肺泡壁部分毛细血管重新开放(HE染色,低倍)}}
\label{fig7-10}
  \end{figure}

\subsubsection{结局和并发症}

不伴有并发症的大叶性肺炎经过一般治疗,病人在发病后7~10天,体温下降,症状好转,趋向痊愈。需要指出的是,病变的发展是一个连续过程,因而上述分期不是绝对的,特别是抗生素广泛用于临床以来,上述典型病程已很少见到。抗生素的及时应用能减轻病情、缩短病程、提早康复。少数病例可出现以下并发症:

\paragraph{感染性休克}
是最严重的并发症。病人出现高热,血压下降,四肢厥冷,多汗,口唇青紫等休克症状,称休克型或中毒型肺炎。如果抢救不及时,病死率较高。

\paragraph{肺肉质变}
少数病例肺泡腔内渗出物未被及时溶解、清除,由肺泡壁增生的肉芽组织替代,发生机化,使局部肺组织形成肉样组织,称肺肉质变。

\paragraph{败血症}
严重感染时,细菌侵入血液繁殖,形成败血症,可引起心内膜炎、脑膜炎及关节炎等。

\paragraph{肺脓肿、脓胸}
由于抗生素的早期应用,临床已少见。

\subsection{小叶性肺炎}

小叶性肺炎(lobular
pneumonia)主要由化脓菌感染引起,病变起始于细支气管,并以细支气管为中心、向周围或末梢肺组织发展,形成以肺小叶为单位、呈灶状散布的肺化脓性炎。因其病变以支气管为中心故又称支气管肺炎(bronchopneumonia)。主要发生于小儿、年老体弱者或久病卧床者,冬、春季节发病率增高。

\subsubsection{病因和发病机制}

小叶性肺炎主要由细菌感染引起,最常见的细菌为致病力较弱的肺炎球菌,其次为葡萄球菌、链球菌、肺炎球菌、流感嗜血杆菌、铜绿假单胞菌和大肠埃希菌等。这些细菌通常是口腔或上呼吸道内致病力较弱的常驻寄生菌,往往在某些诱因影响下,如患传染病、营养不良、恶病质、慢性心力衰竭、昏迷、麻醉、手术后等,使机体抵抗力下降,呼吸系统的防御功能受损,细菌得以入侵、繁殖,发挥致病作用,引起支气管肺炎。因此,支气管肺炎常是某些疾病的并发症,如麻疹后肺炎、手术后肺炎、吸入性肺炎、坠积性肺炎等。有时成为病人的直接死亡原因,故又有临终性肺炎之称。

\subsubsection{病理变化}

以细支气管为中心的肺组织化脓性炎是小叶性肺炎的特征。

肉眼观:肺组织内散布一些以细支气管为中心的化脓性炎症病灶,常散布于两肺各叶,尤以背侧和下叶病灶较多。病灶大小不等,直径多在1
cm左右(相当于肺小叶范围),形状不规则,色暗红或带黄色(图\ref{fig7-11}A)。严重者,病灶互相融合甚或累及全叶,形成融合性支气管肺炎。

镜下观:病灶中支气管、细支气管及其周围的肺泡腔内充满脓性渗出物,纤维蛋白一般较少(图\ref{fig7-11}B)。病灶周围肺组织充血,可有浆液渗出、肺泡过度扩张等变化。由于病变发展阶段的不同,各病灶的病变表现和严重程度亦不一致。有些病灶完全化脓,支气管和肺组织遭破坏,而另一些病灶内则仅可见浆液性渗出,有的还停留于细支气管及其周围炎阶段。

\begin{figure}[!htbp]
 \centering
 \includegraphics{./images/Image00120.jpg}
 \captionsetup{justification=centering}
 \caption{小叶性肺炎}
 \label{fig7-11}
  \end{figure} 

\subsubsection{临床病理联系}

因小叶性肺炎多为其他疾病的并发症,其临床症状常为原发性疾病所掩盖。由于支气管黏膜的炎症刺激而引起咳嗽,痰呈黏液脓性。因病变常呈灶性散布,肺实变体征一般不明显。病变区细支气管和肺泡内含有渗出物,听诊可闻湿啰音。X线检查可见肺野内散在不规则小片状或斑点状模糊阴影。

\subsubsection{结局和并发症}

小叶性肺炎若治疗及时,多数病例预后良好。如病人为年老、体弱、婴幼儿或作为其他疾病的并发症,则预后较差。常见的并发症有肺脓肿、脓胸、支气管扩张症。严重的小叶性肺炎,病变范围广泛者可并发呼吸功能及心功能不全。

\subsection{间质性肺炎}

间质性肺炎(interstitial
pneumonia)指发生于肺间质的炎症,以淋巴细胞、巨噬细胞浸润为特征。肺间质包括肺泡壁、肺小叶间隔及细支气管周围组织。间质性肺炎的病变及临床症状与大叶性肺炎、小叶性肺炎均不相同,主要由肺炎支原体和病毒引起,其肺部的病理变化大致相似。

其基本病理变化为:

肉眼观:病变常位于一侧肺叶,偶有累及两肺者,以肺下叶较多见。病灶多呈斑片状,红黄色。镜下观:肺泡壁增厚,充血,水肿,常有多量淋巴细胞、巨噬细胞浸润,偶见浆细胞。肺泡腔内渗出物不明显,仅见少量浆液及少数巨噬细胞(图\ref{fig7-12})。

\begin{figure}[!htbp]
 \centering
 \includegraphics{./images/Image00121.jpg}
 \captionsetup{justification=centering}
 \caption{间质性肺炎(HE染色,中倍)\\ {\small 肺间质内有多量巨噬细胞和淋巴细胞浸润,肺泡腔内渗出物少}}
\label{fig7-12}
  \end{figure}

因病原不同,本病病变又各具特点,现分述如下:

\subsubsection{支原体肺炎}

支原体肺炎(mycoplasmal
pneumonia)是由肺炎支原体引起的一种间质性肺炎,发病率占各种类型肺炎的5%~10%。支原体存在于病人口鼻分泌物中,经飞沫传播,引起散发性呼吸道感染或者小流行。

肺炎支原体侵入呼吸道后在支气管黏膜上皮表面繁殖,使纤毛肿胀,活动减弱甚至消失,在免疫功能下降时,引起局部炎症。支原体肺炎多发生于20岁以下的青少年,50岁以上的成人由于隐性感染获得一定免疫力,因而其发病率随年龄增长而降低。

\paragraph{病理变化}
肺炎支原体侵犯整个呼吸道黏膜,引起气管炎、支气管炎和肺炎,甚至全呼吸道炎。肺部病变以下叶多见。肉眼观:病灶无明显实变,肺呈暗红色。切面肺普遍充血、水肿和不同程度的出血。镜下观:呈间质性肺炎改变。

\paragraph{临床病理联系}
本病一般起病较急,多有发热、乏力、咽痛、咳嗽等。由于支气管受炎症刺激,病人突出的症状为剧烈的咳嗽,由于肺泡腔内渗出物不多,痰量少,故常为干咳。肺实变体征不明显。X线检查示肺部病变多样化,可显示肺纹理增加、网织状阴影或斑点片状模糊阴影。

\subsubsection{病毒性肺炎}

引起病毒性肺炎(viral
pneumonia)的病毒种类较多,在成人多为流感病毒,在儿童及幼儿多为呼吸道合胞病毒,其他诸如腺病毒、麻疹病毒、巨细胞病毒等亦可致病。本病主要经呼吸道飞沫传播,在机体免疫力低下时引起肺部病变,少数则是病毒血症的结果。一般为散发性,偶可引起流行。

\paragraph{病理变化}
病毒性肺炎的病变常不一致,除上述典型的间质性肺炎外,还可出现下列病变:在严重病例,肺泡亦受累,肺泡腔内炎性渗出物增多,除浆液外,尚有纤维素、红细胞及巨噬细胞。某些病例渗出现象明显,渗出物浓缩并受空气挤压,在肺泡表面形成红染的膜状物,称为透明膜(图\ref{fig7-13}),这种改变可见于流感病毒、麻疹病毒及腺病毒引起的肺炎。有些病毒性肺炎可见支气管上皮、肺泡壁上皮细胞增生,并有多核细胞形成,在增生的上皮和巨噬细胞内可查见病毒包涵体,具有诊断意义。病毒包涵体常呈球形,约红细胞大小,呈嗜酸性染色,均质或细颗粒状,周围常有清晰的透明晕。包涵体可位于细胞核内(如腺病毒)或胞浆中(如呼吸道合胞病毒)或两者均有(如麻疹病毒)。严重的病例还可继发细菌感染,表现为间质性支气管肺炎。

\begin{figure}[!htbp]
 \centering
 \includegraphics{./images/Image00122.jpg}
 \captionsetup{justification=centering}
 \caption{病毒性肺炎(HE染色,中倍)\\ {\small 肺泡壁充血,巨噬细胞和淋巴细胞浸润,肺泡腔内渗出物形成透明膜}}
\label{fig7-13}
  \end{figure}

\paragraph{临床病理联系}
病毒血症可引起发热及全身中毒症状。由于支气管、细支气管炎症刺激可引起剧烈咳嗽。若肺泡腔内渗出物少,肺部啰音及实变体征不明显。严重病例或继发细菌感染时,肺部出现实变体征,伴有严重的全身中毒和缺氧症状,甚至导致心、肺功能不全,预后不良。

{【附】SARS}

严重急性呼吸道综合征(severe acute respiratory
syndrome,SARS)是由冠状病毒(SARS病毒)引起的一种新的呼吸系统传染性疾病。中国广东省首先发现,最早的病例出现在2002年11月中旬。目前已有多个国家报告发现了SARS病例。本病主要通过近距离空气飞沫和密切接触传播,临床主要表现为肺炎,有比较强的传染力。人群普遍易感,医护人员是本病的高危人群。潜伏期为2~12天,通常为4~5天。传染性主要在急性期(发病早期),尤以刚发病时最强。当病人被隔离及采取抗病毒、提高机体免疫力等治疗措施后,机体开始识别病毒并出现针对SARS的特异性免疫反应来抵抗和中和病毒。随着疾病的康复,SARS病毒逐渐被机体所清除,其传染性也随之消失。SARS起病急,以发热为首发症状,体温一般超过38℃,偶有畏寒;可伴有头痛、关节酸痛、肌肉酸痛、乏力、腹泻;常无上呼吸道其他症状;可有咳嗽,多为干咳、少痰,偶有血丝痰;可有胸闷,严重者出现呼吸加速,气促,或明显呼吸窘迫。肺部体征不明显,部分病人可闻少许湿啰音,或有肺实变体征。实验室检查发现:外周血白细胞计数一般不升高,或降低;常有淋巴细胞计数减少。胸部X线检查为肺部有不同程度的片状、斑片状浸润性阴影或呈网状改变,部分病人进展迅速,呈大片状阴影,常为双侧改变,阴影吸收消散较慢。肺部阴影与症状体征可不一致等。

一、病理变化

SARS死亡病例尸检显示该病以肺和免疫系统的病变最为突出,心、肝、肾、肾上腺等实质性器官也不同程度受累。

1.
肺部病变 肉眼观:双肺呈斑块状实变,严重者双肺完全性实变;表面暗红色,切面可见肺出血灶及出血性梗死灶。镜下观:以弥漫性肺泡损伤为主,肺组织重度充血、出血和肺水肿,肺泡腔内充满大量脱落和增生的肺泡上皮细胞及渗出的单核细胞、淋巴细胞和浆细胞。部分肺泡上皮细胞胞质内可见典型的病毒包涵体,电镜证实为病毒颗粒。肺泡腔内可见广泛透明膜形成,部分病例肺泡腔内渗出物出现机化,呈肾小球样机化性肺炎改变。肺小血管呈血管炎改变,部分管壁可见纤维素样坏死伴血栓形成,微血管内可见纤维索性血栓。

2.
脾和淋巴结病变 脾体积略缩小,质软。镜下见脾小体高度萎缩,脾动脉周围淋巴鞘内淋巴细胞减少,红髓内淋巴细胞稀疏,白髓和被膜下淋巴组织大片灶状出血坏死。肺门淋巴结及腹腔淋巴结固有结构消失,皮髓质分界不清,皮质区淋巴细胞数量明显减少,常见淋巴组织呈灶状坏死。心、肝、肾、肾上腺等器官均有不同程度变性、坏死和出血等改变。

二、结局及并发症

从目前掌握的SARS的传染过程来看,SARS病人的传染性主要在急性期(发病早期),尤以刚发病时为强。随着疾病的康复,SARS病毒逐渐被机体所清除,其传染性也随之消失。所以,SARS病人康复出院后是不会传染他人的。

不足5%的严重病例可因呼吸衰竭而死亡,其并发症及后遗症有待进一步观察确定。

\section{硅肺}

在职业活动中,因长期吸入有害粉尘,引起以肺广泛纤维化为主要病变的疾病,统称尘肺(pneumoconiosis)。尘肺是我国一种法定职业病。硅肺(silicosis)又称矽肺,是尘肺中最常见的类型,也是危害最严重的一种职业病。是人体在生产环境中长期吸入大量含游离二氧化硅(SiO{2}
)粉尘微粒所引起的以肺纤维化为主要病变的全身性疾病。该病发展缓慢,即使在脱离硅尘作业后,病变仍然继续发展。病人多在接触硅尘10~15年后才发病。若因吸入高浓度、高游离二氧化硅含量的硅尘,经1~2年后发病者,称速发型硅肺。硅肺的早期即有肺功能损害,但因肺的代偿能力很强,病人往往无症状;随着病变的发展,尤其是合并肺结核和肺心病时,则逐渐出现不同程度的呼吸和心功能障碍。

\subsection{病因和发病机制}

游离二氧化硅是硅肺的致病因子。硅肺的发生、发展与硅尘中游离二氧化硅的含量,生产环境中硅尘的浓度、分散度,从事硅尘作业的工龄及机体防御功能等因素有关。一般来说,直径大于5
μm的硅尘往往被阻留在上呼吸道,并可被呼吸道的防御装置清除。直径小于5
μm的硅尘才能被吸入肺泡,并进入肺泡间隔,引起病变,其中尤以1~2
μm的硅尘微粒引起的病变最为严重。

吸入肺泡内的硅尘微粒被肺巨噬细胞吞噬,沿肺淋巴流经细支气管周围、小血管周围、小叶间隔和胸膜再到达肺门淋巴结。当淋巴道阻塞后,硅尘沉积于肺间质内引起硅肺病变。若局部沉积的硅尘量多,引起肺巨噬细胞局灶性聚积,可导致硅结节形成;若硅尘散在分布,则引起弥漫性肺间质纤维化。

硅肺的发病机制尚未完全阐明。一般认为,游离二氧化硅颗粒进入肺泡后,被聚集在肺淋巴管起始部位的肺巨噬细胞所吞噬,游离二氧化硅对巨噬细胞有极强的毒性作用,可致其自溶死亡,二氧化硅被吞噬后,被包裹在吞噬细胞溶酶体中,由于石英表面的羟基和巨噬细胞溶酶体膜脂蛋白结构上的氢原子受体(氧、氮及硫原子)间形成氢键,引起细胞膜的改变和通透性的变化,导致巨噬细胞溶酶体崩解,并释放出酸性水解酶进入细胞内,继而导致巨噬细胞死亡,并再次将石英粒子释放,形成恶性循环,造成更多的细胞受损,受损的巨噬细胞释放出非脂类“致纤维化因子”,刺激成纤维细胞,合成胶原纤维增多,形成以胶原纤维为中心的病灶结节-硅结节,硅结节向全肺扩展并相互融合,造成双肺弥漫性损害。纤维化不仅局限于肺内,也存在于巨噬细胞所迁移到的淋巴结内。在许多硅肺病人中已发现血清γ-球蛋白水平增高,自身抗体的存在,以及在硅肺病变中存在γ-球蛋白,故认为硅肺发生与免疫发病有关。

\subsection{病理变化}

硅肺的基本病理变化是肺组织内硅结节形成和弥漫性间质纤维化。硅结节是硅肺的特征性病变,结节境界清楚,直径2~5
mm,呈圆形或椭圆形,灰白色,质硬,触之有砂样感。随着病变的发展,结节可融合成团块状,在团块的中央,由于缺血、缺氧而发生坏死、液化,形成硅肺性空洞。硅结节的形成过程大致分为三个阶段:①细胞性结节,由吞噬硅尘的巨噬细胞局灶性聚积而成,巨噬细胞间有网状纤维,这是早期的硅结节(图\ref{fig7-14}A);②纤维性结节,由成纤维细胞、纤维细胞和胶原纤维构成(图\ref{fig7-14}B);③玻璃样结节,玻璃样变从结节中央开始,逐渐向周围发展,往往在发生玻璃样变的结节周围又有新的纤维组织包绕。

镜下,典型的硅结节是由呈同心圆状或旋涡状排列的、已发生玻璃样变的胶原纤维构成。结节中央往往可见内膜增厚的血管。用偏光显微镜观察,可以发现沉积在硅结节和肺组织内呈双屈光性的硅尘微粒。除硅结节外,肺内还有不同程度的弥漫性间质纤维化,范围可达全肺2/3以上。此外,胸膜也因纤维组织弥漫增生而广泛增厚,甚至可厚达1~2
cm。肺门淋巴结内也有硅结节形成和弥漫性纤维化及钙化,淋巴结因而肿大、变硬。

\begin{figure}[!htbp]
 \centering
 \includegraphics{./images/Image00123.jpg}
 \captionsetup{justification=centering}
 \caption{硅结节}
 \label{fig7-14}
  \end{figure} 

\subsection{硅肺的分期}

根据肺内硅结节的数量、分布范围和直径大小,可将硅肺分为三期。

Ⅰ期硅肺:硅结节主要局限在淋巴系统。肺组织中硅结节数量较少,直径一般在1~3
mm,主要分布在两肺中、下叶近肺门处。X线检查,肺野内可见一定数量的类圆形或不规则小阴影,其分布范围不少于两个肺区。此时,肺的重量、体积和硬度无明显改变。胸膜上可有硅结节形成,但胸膜增厚不明显。

Ⅱ期硅肺:硅结节数量增多、体积增大,可散于全肺,但仍以肺门周围中、下肺叶较密集,总的病变范围不超过全肺;X线表现为肺野内有较多量直径不超过1
cm的小阴影,分布范围不少于四个肺区。此时,肺的重量、体积和硬度均有增加,胸膜也增厚。

Ⅲ期硅肺:硅结节密集融合成块。X线表现有大阴影出现,其长径不小于2
cm,宽径不小于7
cm。此时,肺的重量和硬度明显增加。解剖取出新鲜肺标本可竖立不倒,切开时阻力甚大,并有砂粒感。浮沉试验,全肺入水下沉。团块状结节的中央可有硅肺空洞形成。结节之间的肺组织常有明显的灶周肺气肿,有时肺表面还可见到肺大泡。

\subsection{硅肺的常见并发症}

\paragraph{硅肺结核病}
硅肺合并结核病时称为硅肺结核病。Ⅲ期硅肺的合并率达60%~70%。硅肺病人易合并肺结核可能是因游离二氧化硅对巨噬细胞的毒性损害以及肺间质弥漫性纤维化,导致肺的血液循环和淋巴循环障碍,从而降低了肺组织对结核杆菌的防御能力的缘故。硅肺结核病时,硅肺病变和结核病变可分开存在,也可混合存在。硅肺结核病的病变比单纯硅肺和单纯肺结核的病变发展更快,累及范围更广,更易形成空洞。硅肺结核性空洞的特点是数目多、直径大、空洞壁极不规则。较大的血管易被侵蚀,可导致病人大咯血死亡。

\paragraph{肺感染}
由于硅肺病人抵抗力低,又有慢性阻塞性肺疾病,小气道引流不畅,故易继发细菌或病毒感染。尤其在有弥漫性肺气肿的情况下,肺感染可诱发呼吸衰竭而致死。

\paragraph{慢性肺源性心脏病}
有60%~75%的硅肺病人并发肺心病。这是因为肺间质弥漫性纤维化,肺毛细血管床减少,肺循环阻力增加。同时,硅结节内小血管常因闭塞性血管内膜炎,管壁纤维化,使管腔狭窄乃至闭塞,血管也扭曲、变形,尤以肺小动脉的损害更为明显,加之因呼吸功能障碍造成的缺氧,引起肺小动脉痉挛,均可导致肺循环阻力增加、肺动脉高压和右心室肌壁肥厚,心腔扩张。重症病人可因右心衰竭而死亡。

\paragraph{肺气肿和自发性}
气胸晚期硅肺病人常有不同程度的弥漫性肺气肿,主要是阻塞性肺气肿,有时在脏层胸膜下还可出现肺大泡。气肿囊腔破裂引起自发性气胸。

\section{呼吸系统常见恶性肿瘤}

\subsection{鼻咽癌}

鼻咽癌(nasopharyngeal
carcinoma,NPC)是发生于鼻咽部上皮组织的恶性肿瘤,在我国较为常见。尤其多见于广东、广西、福建等南方地区,有明显的地区多发性。男性患者为女性的2倍,患者多在40~50岁。

\subsubsection{病因及发病机制}

鼻咽癌的病因迄今尚未完全阐明,可能与以下因素相关。

\paragraph{EB病毒}
资料显示100%鼻咽癌患者中有EB病毒的基因组,癌细胞内存在EBV-DNA及EB核抗原(EBNA),患者血清内还有高效价的抗EB病毒抗体,但EB是鼻咽癌的致病启动因素还是其他致癌物质的辅助作用因素尚需进一步研究。

\paragraph{环境致癌物质}
某些环境化学致癌物如亚硝胺、多环芳烃类化合物、微量元素镍等可能与鼻咽癌的发生有关。

\paragraph{遗传因素}
鼻咽癌发病有明显的地区性差异,高发区居民移居他地或国外,其后裔的发病率仍远远高于当地居民。部分鼻咽癌患者还有家族发病倾向,因此在其发病中可能有遗传性易感因素。

\subsubsection{病理变化}

鼻咽癌最常发生于鼻咽顶部,其次为侧壁及咽隐窝。有时还可同时在顶部及侧壁发生。

肉眼观:鼻咽癌呈结节型、菜花型、浸润型及溃疡型四种形态,其中以结节型最常见。早期局部黏膜仅显粗糙、增厚或稍稍隆起,临床检查时易被忽略。有时原发部位未发现肿瘤时已发生颈部淋巴结转移。

绝大多数鼻咽癌起源于鼻咽黏膜柱状上皮的储备细胞,该细胞是一种原始多能性的细胞,既可向柱状上皮方向分化,又可向鳞状上皮方向分化。因此,鼻咽癌的组织学分类较为复杂,迄今还没有统一的鼻咽癌病理学分类。一般来说,可分为两类。

\paragraph{鳞状细胞癌}
按细胞分化程度,可分为角化型和非角化型。

角化型鳞状细胞癌极少见,主要发生于老年患者,此型较少见,一般认为其发生与EB病毒无关。非角化型鳞状细胞癌最为多见,癌细胞呈多角形、卵圆形或梭形,无细胞角化现象,其发生与EB病毒感染关系密切。非角化型癌还可进一步分为分化型和未分化型。分化型即低分化鳞状细胞癌。未分化型又可分为分化极差的鳞状细胞癌和泡状核细胞癌。

\paragraph{腺癌}
高分化腺癌少见,癌细胞排列成腺泡状或管状。低分化腺癌癌细胞呈不规则条索状或片状排列,有时可见到腺腔结构或围成腺腔的倾向。

在鼻咽癌的组织学分型中,非角化型鳞状细胞癌最为常见,其次为未分化型的泡状核细胞癌。低分化腺癌较少,高分化鳞状细胞癌及腺癌最少。

\subsubsection{扩散及转移}

\paragraph{直接蔓延}
鼻咽部解剖解构复杂,肿瘤向上可侵犯颅内,向下扩展到达口咽,向下后方则侵犯梨状隐窝、会厌及喉腔上部,向外侧扩展可侵犯耳咽管至中耳,向后扩展则穿过鼻咽后壁侵犯上段颈椎,向前扩展则进入鼻腔甚至侵入眼眶。

\paragraph{淋巴道转移}
鼻咽黏膜固有层有丰富的淋巴管,故本癌可早期经淋巴道转移。颈淋巴结转移常为同侧,其次为双侧,极少只呈对侧转移。

\paragraph{血道转移}
常转移至肝、肺、骨,其次是肾、肾上腺及胰腺等处。

\subsubsection{临床病理联系}

鼻咽癌临床表现多样,常有鼻衄、鼻塞、耳鸣、听力减退、颈部肿块、复视及偏头痛等症状。当症状明显时多已进入进展期或晚期,治愈率极低,故早期诊断极为重要。

\begin{framed}
{案例7-1}

{【病例摘要】}

患者,男,65岁,因咳嗽、咳痰、痰中带血3个月入院。患者三月前开始出现刺激性咳嗽,自服止咳药未好转,痰中可出现血丝,近一月来症状加重。自发病以来患者体重下降8
kg。既往40余年吸烟史,平均每日1.5包,无酗酒史。入院后体检呈消瘦貌,神萎,血常规示中度贫血。胸片示肺门处一3cm×4cm占位影,怀疑支气管肺癌。

{【问题】}

(1)该患者怀疑支气管肺癌的依据为哪些?

(2)后行支气管镜检查,确诊为小细胞肺癌,试描述肿瘤镜下的组织学特征。
\end{framed}

\subsection{肺癌}

肺癌(lung
carcinoma)又称支气管肺癌,是最常见的恶性肿瘤之一。每年全世界有超过130万人被确诊患有肺癌,超过110万人死于肺癌。我国肺癌的发病率在20世纪70年代至90年代上升一倍多之后,近10年里继续呈明显上升趋势,目前肺癌已成为我国危害最大的癌症。肺癌多发生于45岁以上的中老年人,在55~75岁患病率最高,男女性别比例为2∶1。近年来,由于女性吸烟者的不断增加,女性比例相应上升。

\subsubsection{病因与发病机制}

肺癌的病因较复杂,其发生与下列因素有关。

\paragraph{吸烟}
吸烟是肺癌发生的重要危险因素,大约有3/4的肺癌患者有重度吸烟史。吸纸烟者肺癌的死亡率比不吸烟者高10~13倍。吸烟的量越多、吸烟的时间越长、开始吸烟的年龄越早,肺癌的死亡率越高。戒烟后则随戒烟时间的延长,肺癌发生率逐渐降低。卷烟燃烧的烟雾中含有超过1
200种化学物质,其中多环芳烃、3,4-苯并芘、放射性元素及砷等多种物质均具有致癌作用。3,4-苯并芘等多环芳烃碳氢化合物在人体内的芳烃羟化酶(AHH)的作用下转变为环氧化物而成为终致癌物,可导致细胞基因突变。由于不同人体内AHH的酶活性不同,因此吸烟致癌存在着个体差异。

\paragraph{物理化学致癌因子}
目前比较公认的致癌因子有烟草燃烧的产物、石棉、砷、铬、镍、铍、煤焦油、沥青、烟尘、芥子气、二氯甲醚等。如果长期接触这些物质,可以诱发肺癌。我国云南锡矿工人的肺癌发生率高达435.44/10万,井下作业较地面作业工人肺癌发病率高20倍。可能与工作中长期接触化学致癌物质和放射性物质有关。

\paragraph{大气污染}
煤、汽油、柴油等燃烧后的废气或烟尘、行驶机动车的排气均可造成空气污染。被污染的空气中含有3,4-苯并芘、二乙基亚硝胺和砷等致癌物。调查表明,工业发达国家肺癌发病率比工业落后国家高、城市比农村高、大城市比中小城市高。

\paragraph{基因改变}
各种致癌因素可引起细胞的基因变化而导致细胞发生癌变。目前已知在肺癌中有多种癌基因的突变或肿瘤抑制基因的失活,其中KRAS、c-Myc、P53、Rb和bcl-2基因都是研究的热点。有关遗传或基因因素在肺癌发生过程中的作用,有待于进一步研究探索。

\subsubsection{病理变化}

\paragraph{大体类型}
根据肿瘤的发生部位可把肺癌分为三种类型:中央型、周围型和弥漫型,与临床X线的肺癌分型相一致。

(1)中央型:此型最常见,多起源于主支气管或叶支气管等大支气管,肿瘤位于肺门部,常破坏支气管壁向周围肺组织浸润、扩展。晚期形成巨块,常包绕癌变的支气管(图\ref{fig7-15}A)。

(2)周围型:此型发生率仅次于中央型,多起源于肺段以下的末梢支气管或肺泡。常在靠近胸膜的肺周边部形成孤立的癌结节。肉眼形态多为结节型(图\ref{fig7-15}B)。

(3)弥漫型:此型少见。肉眼观察:多数呈播散性的粟粒性结节,弥漫侵犯部分肺大叶或全肺叶,似肺炎或播散性肺结核(图\ref{fig7-15}C)。

\begin{figure}[!htbp]
 \centering
 \includegraphics{./images/Image00124.jpg}
 \captionsetup{justification=centering}
 \caption{肺癌大体形态}
 \label{fig7-15}
  \end{figure} 

\paragraph{组织学类型}
世界卫生组织(WHO)最新分类中把肺癌分为鳞状细胞癌、腺癌、大细胞癌、腺鳞癌、神经内分泌癌、肉瘤样癌、其他类型癌和唾液腺来源的癌等8种类型。不同组织学类型在临床表现、治疗手段的选择及预后上均不相同。

(1)鳞状细胞癌:是肺癌最常见类型之一,绝大多数为中老年患者,多有吸烟史。多来自段以上或主支气管,肉眼属中央型,纤支镜检查易被发现,痰脱落细胞学检查阳性率高。高分化鳞癌多有角化珠形成,低分化鳞癌仅有少量细胞角化。

(2)腺癌:也是原发性肺癌中最常见类型之一,且近年来发病率有不断上升的趋势。肺腺癌多数为周围型,女性患者较多,患者不吸烟但多有被动吸烟史。腺癌常位于肺周边部呈孤立结节,边界清楚,常累及胸膜。高分化癌可见癌组织形成腺管或乳头,并有黏液分泌。

(3)神经内分泌细胞癌:主要包含小细胞癌、大细胞神经内分泌癌和类癌。小细胞癌为仅低于肺鳞癌及腺癌的相对常见的一型肺癌。其发生率占原发性肺癌的15%~20%。发病年龄较鳞癌低,好发于中年男性,与吸烟及职业性接触有一定关系。肿瘤恶性度极高,生长迅速。多有早期转移,一般不适合手术切除,但对化疗及放疗敏感。本型癌细胞小呈短梭形(燕麦型,图\ref{fig7-16})或小圆形(淋巴细胞样),核浓染,胞浆稀少形似裸核。癌细胞常密集成群,有时围绕小血管排列成假菊形团样结构。电镜下可见一部分癌细胞胞浆含有神经分泌颗粒,现认为该肿瘤起源于APUD系统,可伴有异位内分泌综合征。

\begin{figure}[!htbp]
 \centering
 \includegraphics{./images/Image00125.jpg}
 \captionsetup{justification=centering}
 \caption{燕麦细胞癌(HE染色,高倍)\\ {\small 癌细胞小呈短梭形或小圆形,常密集成群,围绕小血管排列成假菊形团样结构}}
\label{fig7-16}
  \end{figure}

(4)大细胞癌:肺大细胞癌属于未分化癌,恶性度高,癌生长迅速,早期发生转移。

(5)腺鳞癌:此型肺癌含有腺癌细胞及鳞癌细胞两种成分,属于混合性癌。现认为此型肺癌发生自支气管上皮的具多向分化潜能的干细胞。

(6)肉瘤样癌:为近年WHO新列出的一种肺癌分类,癌分化不成熟,恶性度高,有多形性、梭形细胞性、巨细胞癌及癌肉瘤等多种亚型。

\subsubsection{扩散与转移}

\paragraph{直接蔓延}
中央型肺癌常直接侵及肿瘤周围组织如纵隔、心包及周围血管,或沿支气管向同侧甚至对侧肺组织蔓延。周围型肺癌可直接侵犯胸膜,在胸壁生长。

\paragraph{转移}
肺癌沿淋巴道转移时首先转移至肺门淋巴结,再扩散至纵隔、锁骨上、腋窝和颈部淋巴结。周围型肺癌的癌细胞可到达胸膜下淋巴丛,引起胸膜腔的血性渗出液。血行转移常见于肝、脑、肾上腺、骨及肾等处。

\subsubsection{临床病理联系}

肺癌早期因症状不明显易被忽视。患者可有咳嗽、痰中带血丝及胸痛等症状。肿瘤压迫或阻塞支气管可引起远端肺组织的化脓性炎、脓肿形成。癌组织侵及胸膜引起癌性胸膜炎、积液。侵犯纵隔内压迫上腔静脉引起面颈部水肿及颈、胸部静脉曲张(上腔静脉综合征)。肺尖部肺癌易侵犯交感神经引起病侧眼睑下垂、瞳孔缩小和胸壁皮肤无汗等交感神经麻痹综合征(Horner综合征)。有异位内分泌作用的肺癌,尤其是小细胞肺癌可因5-羟色胺分泌过多而引起类癌综合征,表现为支气管哮喘、心动过速、水样腹泻和皮肤潮红等。

\begin{center}
    \textbf{知识链接}
\end{center}
\chapterabstract{肺癌生物治疗是一种利用细胞生物学与分子生物学手段调节机体免疫系统功能或肿瘤生长,从而达到抑瘤目的的治疗方法,是继手术、放疗、化疗模式之后新兴的治疗手段,它具有常规治疗方法无可比拟的优势,并显示出良好的临床应用前景。具体治疗包括树突状细胞疫苗、相关肿瘤抗原疫苗、肿瘤细胞疫苗、过继性细胞免疫治疗和分子靶向治疗等。肺癌的生物治疗不仅开辟了肺癌全新治疗模式,同时也丰富了肿瘤生物治疗范围。}

\section*{复习与思考}

{一、名词解释}

COPD慢性阻塞性肺气肿 小叶中央型肺气肿 全小叶型肺气肿 肺心病 大叶性肺炎 小叶性肺炎 肺肉质变 硅结节 燕麦细胞癌

{二、问答题}

1. 试述慢性支气管炎的病变特点。

2. 试述支气管扩张症的发病机制。

3. 试述慢性肺源性心脏病的病变及临床病理联系。

4. 试述大叶性、小叶性、间质性肺炎的病变特点及大、小叶性肺炎的鉴别点。

5. 硅肺的病理特点是什么?硅结节是如何形成的?

6. 鼻咽癌的主要组织学类型及其扩散途径是什么?

7. 肺癌的常见病理类型有哪些?

8. 右肺上叶有一直径约1.5
cm的球形病灶,试考虑有哪些病变的可能及其病理特点。

{三、临床病理分析}

病史:患者男性,64岁。慢性咳嗽、咳痰28年,痰通常为白色泡沫样,有时发热伴脓痰。近五年来爬坡即感气急,近两年来稍活动即感气急,时有心悸,面部与下肢水肿。入院前一周开始发热,近三日来高达39℃,气急加重,指唇出现青紫,下肢水肿。

既往史:吸烟30年,日吸一包以上。

体检:体温38.6℃,脉搏100次/分,呼吸24次/分,血压正常,神志清但迟钝。口唇轻度青紫,下肢轻度水肿,颈静脉稍充盈,胸廓呈桶状。腹部略膨胀,肝剑突下二横指,质中,轻度压痛。扣诊,心界扣不出。听诊:两肺可闻及广泛湿啰音,肺动脉瓣第二音亢进。X线胸片,两肺透明度增高,肺纹理增强,两肺下叶有小片状模糊炎性阴影,横膈低平,心影扩大,肺动脉圆锥突起。

讨论题:

1. 试分析患者可能发生的疾病。

2. 试叙述疾病的发生发展过程。

3. 试阐明疾病的病理变化。

\end{document}
