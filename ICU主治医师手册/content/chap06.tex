\chapter{急性加重期慢性阻塞性肺疾病}

\section{前沿学术综述}

慢性阻塞性肺疾病(chronic obstructive pulmonary
disease,COPD)是一种常见病、多发病,病死率高,其主要特征为肺功能缓慢减退及进行性气流受限,不完全可逆,严重影响患者的劳动能力和生活质量,目前尚无完全治愈的方法。2002年世界卫生组织(WHO)公布的资料显示,目前COPD居世界上所有死亡原因的第5位,预计到2020年,COPD将成为第3大死亡原因,而COPD患者在漫长的病程中,每年平均急性加重2~3次,成为COPD患者住院和死亡的最重要原因。COPD急性加重会导致很多负面效应,如降低病人的生活质量、损伤肺功能和增加社会经济成本。有资料显示,住重症医学科的COPD患者病死率可达15%~24%,年龄>65岁的患者的病死率更高达30%
\protect\hyperlink{text00012.htmlux5cux23ch1-11}{\textsuperscript{{[}1{]}}}
。因此,如何对急性加重期COPD进行有效的管理具有非常重要的现实意义。近年来,针对慢性阻塞性肺疾病急性发作的呼吸支持和药物治疗技术均取得了显著进展,概述如下。

\subsubsection{呼吸支持技术}

呼吸支持技术是治疗慢性阻塞性肺疾病急性发作必不可少的手段,其主要作用是提供生命支持,为原发病的治疗争取时间。常用呼吸支持技术包括无创正压通气技术(noninvasive
positive pressure ventilation,NPPV)和有创正压通气技术(invasive
positive pressure
ventilation,IPPV)。近年来,有多项随机对照研究显示,对于慢性阻塞性肺疾病急性发作患者,早期无创正压通气的治疗能在短期内明显缓解呼吸困难症状,提高动脉血pH值,降低动脉血二氧化碳分压,并能降低气管插管率、住院时间和住院病死率
\protect\hyperlink{text00012.htmlux5cux23ch2-11}{\textsuperscript{{[}2{]}}}
。对于无创正压通气禁忌或使用无创正压通气失败的严重呼吸衰竭患者,应及早气管插管改用有创正压通气。此外,以无创正压通气辅助有创正压通气撤机,即早期拔管改用无创正压通气的有创-无创序贯通气策略,可使患者的机械通气时间明显缩短,呼吸机相关性肺炎的发生率和住院病死率也显著降低
\protect\hyperlink{text00012.htmlux5cux23ch3-11}{\textsuperscript{{[}3{]}}}
\textsuperscript{,}
\protect\hyperlink{text00012.htmlux5cux23ch4-11}{\textsuperscript{{[}4{]}}}
,显著地改善了慢性阻塞性肺疾病急性发作的治疗效果。

\subsubsection{支气管扩张剂}

吸入型短效β\textsubscript{2}
肾上腺素受体激动剂是慢性阻塞性肺疾病急性发作最常用的支气管扩张剂,主要用于短期内控制症状,包括沙丁胺醇(albuterol)和特布他林(terbutaline)。M胆碱受体阻滞剂是另一类支气管扩张剂,如异丙托溴铵(ipratropium
bromide)。而最近上市的一种高选择性抗胆碱能药噻托溴铵(tiotropium
bromide)具有血药浓度维持时间长、副作用小等优点
\protect\hyperlink{text00012.htmlux5cux23ch5-11}{\textsuperscript{{[}5{]}}}
。β\textsubscript{2}
肾上腺素受体激动剂和M胆碱受体阻滞剂均可以达到有效扩张支气管的作用,尽管没有充足的证据证实两者的联用会进一步扩张支气管
\protect\hyperlink{text00012.htmlux5cux23ch6-11}{\textsuperscript{{[}6{]}}}
,但若β\textsubscript{2}
肾上腺素受体激动剂达到最大剂量后仍未显效,可考虑联用M胆碱受体阻滞剂。

几项小样本的临床研究显示,长效与短效β\textsubscript{2}
肾上腺素受体激动剂(如福莫特罗)相比,长效β\textsubscript{2}
肾上腺素受体激动剂同样可以有效扩张支气管
\protect\hyperlink{text00012.htmlux5cux23ch7-11}{\textsuperscript{{[}7{]}}}
\textsuperscript{,}
\protect\hyperlink{text00012.htmlux5cux23ch8-11}{\textsuperscript{{[}8{]}}}
。另有研究显示,福莫特罗和噻托溴铵都可以显著增加第一秒用力呼气量、用力肺活量和静息状态下深吸气量,且两者合用较单用肺功能改善更为明显
\protect\hyperlink{text00012.htmlux5cux23ch9-11}{\textsuperscript{{[}9{]}}}
。

支气管扩张剂的吸入有赖于相应的雾化装置,研究显示,定量雾化吸入器和喷射雾化器(jet
neubilizer)是支气管扩张剂有效舒张支气管的重要工具,但雾化吸入器简单、便携、便宜。为增加药物的吸入效率,可将雾化吸入器与储雾罐(spacer)合用。

氨茶碱是治疗慢性阻塞性肺疾病急性发作患者的二线用药。严重慢性阻塞性肺疾病急性发作患者若对短效支气管扩张剂无效时可以加用口服或静脉使用的氨茶碱以缓解气道痉挛。在确定使用茶碱类药物之前要考虑到其副作用,使用时应注意监测血药浓度,防止茶碱中毒。最近研究指出,低浓度茶碱(5~10mg/L)在COPD治疗中既能发挥抗炎作用,又因其血药浓度低,中毒等副作用少,有望成为COPD的长期治疗手段之一
\protect\hyperlink{text00012.htmlux5cux23ch10-11}{\textsuperscript{{[}10{]}}}
。

\subsubsection{糖皮质激素}

已有大量的研究证实,经口服或静脉使用糖皮质激素具有扩张支气管和减轻COPD急性期炎症反应的作用,能迅速帮助慢性阻塞性肺疾病急性发作患者恢复肺功能和缓解急性期症状。临床常用的糖皮质激素有甲基泼尼松龙(甲基强的松龙)、地塞米松和氢化泼尼松等。研究显示,甲基泼尼松龙与地塞米松相比,更能明显地改善肺功能
\protect\hyperlink{text00012.htmlux5cux23ch11-11}{\textsuperscript{{[}11{]}}}
。小样本随机对照研究显示,甲基泼尼松龙长期治疗(10天)与短期治疗(3天)相比,能更有效地改善肺功能和缓解呼吸困难等症状
\protect\hyperlink{text00012.htmlux5cux23ch12-11}{\textsuperscript{{[}12{]}}}
,但应用大剂量糖皮质激素(每天甲基泼尼松龙>80mg)可能会延长住院时间
\protect\hyperlink{text00012.htmlux5cux23ch13-11}{\textsuperscript{{[}13{]}}}
。2011年全球慢性阻塞性肺疾病防治倡议(the global initiative for chronic
obstruction lung
disease,GOLD)推荐慢性阻塞性肺疾病急性发作患者每日口服氢化泼尼松30~40mg,疗程7天~10天。最近研究显示口服糖皮质激素可增加慢性阻塞性肺疾病急性发作患者无创通气治疗的成功率并缩短通气支持时间
\protect\hyperlink{text00012.htmlux5cux23ch14-11}{\textsuperscript{{[}14{]}}}
。

有关评价雾化吸入糖皮质激素疗效的研究近年逐渐增多。一项随机对照研究显示,糖皮质激素雾化吸入与口服给药均能显著改善气流受限(第1秒用力呼气量平均增加100~160ml),缓解急性期症状,但雾化治疗的副作用更少。目前尚没有研究显示雾化吸入与静脉或口服用药在肺功能改善方面的显著性差异
\protect\hyperlink{text00012.htmlux5cux23ch15-11}{\textsuperscript{{[}15{]}}}
\textsuperscript{,}
\protect\hyperlink{text00012.htmlux5cux23ch16-11}{\textsuperscript{{[}16{]}}}
。

\subsubsection{抗生素}

COPD急性发作大约有80%由支气管肺部感染所造成,合理使用抗生素是必要的。一般认为出现下列情况应给予抗生素治疗:出现脓痰伴呼吸困难加重或痰量明显增加;需正压机械通气(包括有创和无创通气)的严重慢性阻塞性肺疾病急性发作。

慢性阻塞性肺疾病急性发作的严重程度不同,其病原微生物的类型亦随之改变。病情较轻患者主要的病原菌以流感嗜血杆菌、卡他莫拉菌和肺炎链球菌多见;病情严重、需接受机械通气治疗者则以肠道革兰阴性杆菌和铜绿假单胞菌比较多见。2006年在法国进行的一项研究也得到了相同的结果
\protect\hyperlink{text00012.htmlux5cux23ch17-11}{\textsuperscript{{[}17{]}}}
。因此,有严重肺功能损害的患者可能从抗生素治疗中获益更多。

抗菌药物应根据患者临床情况、痰液性质、当地病原菌感染趋势及细菌耐药情况合理选用。一般抗生素治疗3~10天。但有20%~30%患者对经验性治疗没有反应,此时需重新评估急性发作的原因(如心力衰竭、肺栓塞等)和重新进行病原学检查。

\subsubsection{新一代治疗药物}

最近很多学者尝试从细胞和分子水平阻止COPD气道阻塞的进展。新一代治疗药物主要以抗炎为基本作用机制,初步显示了一定的临床疗效,但仍需大量的基础和临床研究来证实。主要包括以下几类:①磷酸二酯酶4抑制剂
\protect\hyperlink{text00012.htmlux5cux23ch18-11}{\textsuperscript{{[}18{]}}}
;②炎症介质抑制剂
\protect\hyperlink{text00012.htmlux5cux23ch19-11}{\textsuperscript{{[}19{]}}}
;③抗氧化类药物
\protect\hyperlink{text00012.htmlux5cux23ch20-11}{\textsuperscript{{[}20{]}}}
;④抗蛋白酶类药物
\protect\hyperlink{text00012.htmlux5cux23ch21-11}{\textsuperscript{{[}21{]}}}
等。

\subsubsection{氦氧混合气}

氦氧混合气是一种低密度的混合气体,能减少气体湍流的发生,降低气道阻力,从而减少呼吸功耗
\protect\hyperlink{text00012.htmlux5cux23ch22-11}{\textsuperscript{{[}22{]}}}
。回顾性研究显示,早期使用氦氧混合气治疗,可以显著缩短哮喘患者住院时间,降低气管插管率及病死率。但荟萃分析显示,目前仍没有充足的证据证实氦氧混合气可作为慢性阻塞性肺疾病急性发作的有效治疗手段
\protect\hyperlink{text00012.htmlux5cux23ch23-11}{\textsuperscript{{[}23{]}}}
。

\subsubsection{黏液溶解剂}

荟萃分析显示
\protect\hyperlink{text00012.htmlux5cux23ch24-11}{\textsuperscript{{[}24{]}}}
,不同黏液溶解剂对慢性阻塞性肺疾病急性发作患者的肺功能均没有改善作用,亦不能明显缩短急性加重期时间。

\subsubsection{其他治疗}

通过器械和手动的胸部物理治疗并不能改善慢性阻塞性肺疾病急性发作患者的症状和肺功能
\protect\hyperlink{text00012.htmlux5cux23ch24-11}{\textsuperscript{{[}24{]}}}
。但临床上,对于痰液较多或存在肺不张的患者可给予胸部物理治疗。此外,加强营养支持对于慢性阻塞性肺疾病急性发作患者的治疗也非常重要。

\section{临床问题}

\subsubsection{何谓慢性阻塞性肺疾病急性发作?预后如何?}

目前,慢性阻塞性肺疾病(COPD)急性发作的定义尚无统一标准,但在大多数研究和指南中,COPD患者急性发作的定义主要包括以下3个方面:呼吸困难加重,痰液增多,出现脓痰。当患者出现这3种表现中的1种或几种时,即认为急性发作。另外,COPD患者急性发作时还会出现发热、胸闷、喘息等症状。

COPD急性发作的频率平均为2~3次/年。但这一数据要低于实际的急性发作次数。可能原因是由于患者对疾病状态耐受,而在急性发作后并不去就诊。慢性阻塞性肺疾病急性发作患者住院病死率大约为10%,住院后180天、1年和2年的病死率分别为13.4%、22%和35.6%
\protect\hyperlink{text00012.htmlux5cux23ch25-11}{\textsuperscript{{[}25{]}}}
。住重症医学科的重症患者病死率可达到15%~24%,年龄>65岁的患者的病死率更高达30%
\protect\hyperlink{text00012.htmlux5cux23ch1-11}{\textsuperscript{{[}1{]}}}
。此外,频繁急性发作可使患者肺功能进一步恶化
\protect\hyperlink{text00012.htmlux5cux23ch26-11}{\textsuperscript{{[}26{]}}}
、外周骨骼肌的功能受损
\protect\hyperlink{text00012.htmlux5cux23ch27-11}{\textsuperscript{{[}27{]}}}
,严重影响患者的生活质量
\protect\hyperlink{text00012.htmlux5cux23ch28-11}{\textsuperscript{{[}28{]}}}
。

\subsubsection{慢性阻塞性肺疾病急性发作常见的诱发因素有哪些?}

慢性阻塞性肺疾病(COPD)急性发作的主要原因包括支气管-肺部感染、大气污染、肺栓塞、肺不张、胸腔积液、气胸、左心功能不全等,另外还有1/3急性发作无明显的诱因,其中支气管-肺部感染为最常见诱因。50%慢性阻塞性肺疾病急性发作患者在稳定期其呼吸道已存在病原菌定植,研究显示这种病原菌定植与急性发作有关
\protect\hyperlink{text00012.htmlux5cux23ch29-11}{\textsuperscript{{[}29{]}}}
。

\subsubsection{如何对慢性阻塞性肺疾病急性发作的严重程度进行评估?}

对慢性阻塞性肺疾病(COPD)急性发作患者的严重度进行评估,主要依据患者的病史、症状和体征、肺功能、动脉血气指标、X线胸片和其他的辅助检查。应特别注意患者本次发病时呼吸困难和咳嗽的频率及严重程度,另外还有痰液的性状和日常生活受限的情况。当患者出现以下情况时提示严重的急性发作:胸腹矛盾运动;辅助呼吸肌的参与;意识状态的恶化;出现右心功能不全或休克等。

第一秒用力呼气量<1L,或呼气峰流速<100L/分钟提示存在严重的急性发作,但严重患者的稳定期亦会出现这种改变。此外,由于急性发作患者有时不能配合简单的肺功能检查,所以FEV\textsubscript{1}
和PEF并不是可靠的评价指标。

动脉血气分析是非常重要的评价疾病严重程度的指标,对合理的氧疗和机械通气治疗有指导意义。但对动脉血气指标进行分析时,需结合患者稳定期的水平进行考虑。

常规X线胸片检查能帮助临床医生明确COPD急性发作的诱因,排除与COPD急性发作有相似临床表现的其他疾病,如肺栓塞、气胸、肺水肿等。

\subsubsection{慢性阻塞性肺疾病急性发作患者何时需要转入重症医学科治疗?}

当慢性阻塞性肺疾病患者出现严重的急性发作症状时,需转入重症医学科治疗。指征包括:①严重呼吸困难对初始治疗反应差;②出现意识不清、昏迷;③给予充分的氧疗和无创通气后仍存在持续或进行性加重的低氧血症(动脉血氧分压<40mmHg)、严重或进行性恶化的高碳酸血症(动脉血二氧化碳分压>60mmHg)、严重或进行性恶化的呼吸性酸中毒(动脉血pH<7.25);④需要进行有创通气;⑤需要血管活性药物治疗的血流动力学不稳定者。

\subsubsection{慢性阻塞性肺疾病急性发作患者发生呼吸衰竭的主要机制是什么?}

慢性阻塞性肺疾病(COPD)的慢性炎性反应常常累及全肺,在中央气道(内径>2~4mm)主要改变为杯状细胞和鳞状细胞化生、黏液腺分泌增加、纤毛功能障碍;外周气道(内径<2mm)的主要改变为管腔狭窄,导致气道阻力增加,延缓肺内气体排出,导致呼气不畅、功能残气量增加;其次,肺实质组织(呼吸性细支气管、肺泡、肺毛细血管)广泛破坏导致肺弹性回缩力下降,使呼出气流的驱动压降低,造成呼气气流缓慢。这两个因素使COPD患者呼出气流受限,在呼气时间内肺内气体呼出不完全,形成动态肺过度充气(dynamic
pulmonary
hyperinflation,DPH)。由于动态肺过度充气的存在,肺动态顺应性降低,肺压力容积曲线趋于平坦,在吸入相同容量气体时需要更大的压力驱动,从而使吸气负荷增加。

动态肺过度充气时呼气末肺泡内残留的气体过多,呼气末肺泡内呈正压,称为内源性呼气末正压(intrinsic
positive end-expiratory
pressure,PEEPi)。由于内源性呼气末正压存在,患者必须首先产生足够的吸气压力以克服内源性呼气末正压才可能使肺内压低于大气压而产生吸气气流,这也增大了吸气负荷。肺容积增大造成胸廓过度扩张,并压迫膈肌使其处于低平位,造成曲率半径增大,从而使膈肌收缩效率降低,辅助呼吸肌也参与呼吸。但辅助呼吸肌的收缩能力差,效率低,容易发生疲劳,而且增加了氧耗量。COPD急性加重时,上述呼吸力学异常进一步恶化,氧耗量和呼吸负荷显著增加,超过呼吸肌自身的代偿能力使其不能维持有效的肺泡通气,从而造成缺氧及二氧化碳潴留,严重者发生呼吸衰竭。

\subsubsection{慢性阻塞性肺疾病急性发作患者应用无创正压通气的时机如何掌握?}

无创正压通气与有创正压通气通过提供正压通气,增加肺泡通气量,促进二氧化碳排出。在慢性阻塞性肺疾病急性加重早期,患者神志清楚,咯痰能力尚可,痰液引流问题并不十分突出,而呼吸肌疲劳是导致呼吸衰竭的主要原因,此时予以无创正压通气早期干预可减少呼吸功耗,缓解呼吸肌疲劳;若痰液引流障碍或有效通气不能保障,则需建立人工气道行有创正压通气,可以有效地引流痰液和提供较无创正压通气更有效的正压通气;一旦支气管-肺部感染或其他诱发急性加重的因素有所控制,自主呼吸功能有所恢复,痰液引流问题已不是主要问题时,可撤离有创正压通气,改用无创正压通气以辅助通气,继续缓解呼吸肌疲劳。

\subsubsection{无创正压通气治疗慢性阻塞性肺疾病急性发作患者的适应证有哪些?}

慢性阻塞性肺疾病急性加重患者应用无创正压通气应具备的基本条件包括:意识清楚,咯痰能力较强,血流动力学稳定,具有较好的主动配合能力。至于具体适应证,应综合分析病情后决定:①病情较轻(动脉血pH>7.35,动脉血二氧化碳分压>45mmHg)的患者,应用无创正压通气可在一定程度上缓解呼吸肌疲劳,预防呼吸功能不全进一步加重,必要时可考虑应用;②出现轻中度呼吸性酸中毒(7.25<动脉血pH<7.35)及明显呼吸困难(辅助呼吸肌参与、呼吸频率>25次/分)的患者,推荐应用无创正压通气;③出现严重呼吸性酸中毒(动脉血pH<7.25)患者,可在严密观察的前提下短时间(1~2小时)试用无创正压通气,疗效不佳及时改为有创正压通气治疗;④对于伴有严重意识障碍的患者不宜行无创正压通气;⑤不具备有创正压通气条件或患者及(或)家属拒绝有创正压通气时,可考虑试用无创正压通气。

\subsubsection{无创正压通气治疗慢性阻塞性肺疾病急性发作患者的禁忌证是什么?}

无创正压通气治疗慢性阻塞性肺疾病急性加重患者的禁忌证包括:①气道保护能力差、误吸危险性高;②气道分泌物多且排出障碍;③心跳或呼吸停止;④面部、颈部和口咽腔创伤、烧伤、畸形或近期手术;⑤上呼吸道梗阻;⑥血流动力学不稳定;⑦危及生命的低氧血症;⑧合并严重的上消化道出血或频繁剧烈呕吐。

\subsubsection{无创正压通气治疗慢性阻塞性肺疾病急性发作的通气模式应如何选择?}

常用于无创正压通气模式有以下几种:压力控制通气、持续气道内正压、双水平正压通气(BiPAP)及比例辅助通气,其中以双水平正压通气模式最为常用有效。

如何为患者设定个体化的合理呼吸机治疗参数十分重要。压力和潮气量过低可导致治疗失败,但过高也可能导致漏气和患者耐受性下降。一般采取适应性调节的方式:呼气相压力从2~4cm
H\textsubscript{2}
O开始,逐渐上调压力水平,以尽量保证患者每一次吸气动作都能触发呼吸机送气;吸气相压力从4~8cm
H\textsubscript{2}
O开始,待患者耐受后再逐渐上调,直至达到满意的通气水平或患者可能耐受的最高通气支持水平。

\subsubsection{慢性阻塞性肺疾病急性发作患者进行有创正压通气的适应证有哪些?}

慢性阻塞性肺疾病急性加重患者如出现以下情况应考虑应用有创正压通气治疗:①危及生命的低氧血症[动脉血氧分压<50mmHg或动脉血氧分压/吸入氧浓度<200mmHg];②动脉血二氧化碳分压进行性升高伴严重的酸中毒(动脉血pH≤7.20);③严重的神志障碍(如昏睡、昏迷或谵妄);④严重的呼吸窘迫症状(如呼吸频率>40次/分、矛盾呼吸等)或呼吸抑制(如呼吸频率<8次/分);⑤血流动力学不稳定;⑥气道分泌物多且引流障碍,气道保护能力丧失;⑦无创正压通气治疗失败的严重呼吸衰竭患者。

\subsubsection{慢性阻塞性肺疾病急性发作患者有创通气时,应选用哪类人工气道?}

慢性阻塞性肺疾病急性加重期患者行有创正压通气治疗时,应首选经口气管插管建立人工气道。与经鼻气管插管相比,经口气管插能明显降低鼻窦炎和呼吸机相关性肺炎的发生。

原则上应尽量避免气管切开,因为气管切开后可能导致气管狭窄,对于可能因反复呼吸衰竭而需要多次接受机械通气的慢性阻塞性肺疾病患者而言,多次实施气管切开非常困难;若需行气管切开,可首选经皮扩张气管切开术
\protect\hyperlink{text00012.htmlux5cux23ch30-11}{\textsuperscript{{[}30{]}}}
。

\subsubsection{有创正压通气治疗慢性阻塞性肺疾病急性发作时,应如何选择通气模式?}

在有创通气早期,为了使呼吸肌得到良好的休息,可采用控制通气模式,但需尽量缩短控制通气的时间,以避免大量镇静剂的使用和肺不张、通气/血流比例失调及呼吸肌废用性萎缩的发生。一旦患者的自主呼吸有所恢复,宜尽早采用辅助通气模式,保留患者的自主呼吸,使患者的呼吸肌功能得到锻炼和恢复,为撤机做好准备。常用的通气模式包括辅助控制模式、同步间歇指令通气和压力支持通气,也可试用一些新型通气模式,如比例辅助通气等,其中同步间歇指令通气+压力支持通气和压力支持通气已有较多的实践经验,临床最为常用。

对接受有创正压通气的慢性阻塞性肺疾病急性加重期患者可采取限制潮气量(6~8ml/kg)和呼吸频率(10~15次/分)、增加吸气流速(40~60L/分钟)等措施以促进呼气,改善动态肺过度充气的发生,同时给予合适水平的外源性呼吸末正压(大约为内源性呼气末正压的80%)以防止气道的动态塌陷、降低呼吸功耗。此外,还需注意,在参数调节时要加强动脉血二氧化碳分压的监测,应尽量避免动脉血二氧化碳分压下降过快所致的严重碱中毒。

\subsubsection{如何把握慢性阻塞性肺疾病患者撤离有创正压通气的时机?}

当慢性阻塞性肺疾病患者满足以下条件时,可考虑撤离有创正压通气:

(1)引起呼吸衰竭的诱发因素得到有效控制,这是撤机的先决条件,应仔细分析可能的诱发因素并加以处理。

(2)患者意识清楚,可主动配合。

(3)患者自主呼吸能力有所恢复,具有呼吸道自洁能力。

(4)通气及氧合功能良好:动脉血氧分压/吸入氧浓度>250mmHg,呼气末正压<5~8cm
H\textsubscript{2} O,动脉血pH>7.35,动脉血二氧化碳分压达缓解期水平。

(5)血流动力学稳定,无活动性心肌缺血,未使用升压药治疗或升压药剂量较小。

当患者满足上述条件后,可逐渐降低部分通气支持模式的支持力度,直至过渡到完全自主呼吸。常用的部分支持通气模式包括同步间歇指令通气+压力支持通气和压力支持通气模式。在运用同步间歇指令通气+压力支持通气模式撤机时,可逐渐降低同步间歇指令通气的指令频率,当调至2~4次/分钟后不再下调,然后再降低压力支持通气的压力支持水平,直至能克服气管插管阻力的压力(5~7cm
H\textsubscript{2}
O),稳定4~6小时后可脱机。单独运用压力支持通气模式撤机时,压力支持水平的调节可采取类似方法。与其他撤机方式相比,同步间歇指令通气可能会增加撤机的时间,不宜单独应用。

自主呼吸试验(spontaneous breathing
trial,SBT)是指导撤机的常用方法。有研究显示,能耐受30~120分钟SBT的患者,其成功撤机的比例可达80%左右,但仍有约15%患者在48小时内重新气管插管。因此,自主呼吸试验只可作为慢性阻塞性肺疾病患者撤机前的参考。

\subsubsection{什么是有创-无创序贯通气?有何临床意义?}

有创-无创序贯机械通气是指接受有创正压通气的急性呼吸衰竭患者,在未达到拔管撤机标准之前即撤离有创正压通气,继之以无创正压通气,从而减少有创正压通气时间和与有创正压通气相关的并发症。国内外已有多项随机对照证实采用有创-无创序贯通气可显著提高慢性阻塞性肺疾病急性加重期患者的撤机成功率,缩短有创正压通气和住重症医学科的时间,降低院内感染率,并增加患者存活率
\protect\hyperlink{text00012.htmlux5cux23ch31-11}{\textsuperscript{{[}31{]}}}
。

\subsubsection{成功实施有创-无创序贯通气应注意哪些要点?}

(1)病例选择 首先,适合有创-无创序贯通气的病例应具备应用无创正压通气(NPPV)的基本条件;其次,由于无创正压通气的通气支持水平有限,对于基础肺功能很差而需较高呼吸支持水平的病例也不适合。因此,在有创-无创序贯通气的随机对照研究中,均有较明确的病例入选与排除标准。国内进行的一项研究中
\protect\hyperlink{text00012.htmlux5cux23ch4-11}{\textsuperscript{{[}4{]}}}
,要求入选患者年龄不超过85岁,近1年内生活能基本自理,存在以下情况之一则应予排除:①严重的心、脑、肝、肾衰竭;②严重营养不良;③严重且难以纠正的电解质紊乱;④上气道或面部损伤导致无法佩戴鼻或面罩;⑤出现肺部感染控制窗时咳嗽反射极弱或咯痰无力;⑥不能配合无创正压通气。

(2)有创正压通气与无创通气切换点的把握 切换点的把握是实施序贯通气的另一个关键因素。由于慢性阻塞性肺疾病(COPD)急性加重主要是由支气管-肺部感染引起,慢性阻塞性肺疾病急性加重(AECOPD)患者建立有创人工气道有效引流痰液并合理应用抗生素后,在有创正压通气5~7天时,支气管-肺部感染多可得到控制,临床上表现为痰液量减少、粘度变稀、痰色转白、体温下降、白细胞计数降低、X线胸片上支气管-肺部感染影消退,这一肺部感染得到控制的阶段称为“肺部感染控制窗(pulmonary
infection control
window,PIC窗)”。肺部感染控制窗是支气管-肺部感染相关的临床征象出现好转的一段时间,出现肺部感染控制窗后若不及时拔管,则很有可能随插管时间延长并发呼吸机相关性肺炎。出现肺部感染控制窗时患者痰液引流问题已不突出,而呼吸肌疲劳仍较明显,需要较高水平的通气支持,此时撤离有创正压通气,继之无创正压通气,既可进一步缓解呼吸肌疲劳,改善通气功能,又可有效地减少呼吸机相关性肺炎,改善患者预后。国外的研究显示,对肺部感染不显著的COPD患者,有创正压通气治疗早期可行T管撤机试验。对支气管-肺部感染明显的患者,以肺部感染控制窗的出现作为切换点,更符合慢性阻塞性肺疾病急性加重期的治疗规律。而对存在拔管后易发生呼吸衰竭的慢性阻塞性肺疾病急性加重期患者,如曾出现撤机试验失败的患者撤机后尽早行无创正压通气序贯通气,有助于避免再插管和降低患者病死率
\protect\hyperlink{text00012.htmlux5cux23ch32-11}{\textsuperscript{{[}32{]}}}
。

(3)无创正压通气的规范地操作 由于患者提前拔管后还合并有较明显的呼吸肌疲劳和呼吸功能不全,往往还需要较长时间的无创正压通气。因此,规范地操作无创正压通气能保证患者从无创正压通气获得最佳呼吸支持,是成功实施有创-无创序贯通气治疗另一重要方面。

\subsubsection{侵袭性肺曲霉病(invasive pulmonary aspergillosis,IPA)在慢性阻塞性肺疾病患者中流行病学情况?}

侵袭性肺曲霉病常见于免疫抑制的患者,预后极差,病死率达50%~100%。近年来慢性阻塞性肺疾病(COPD)合并侵袭性肺曲霉病的病例报道逐渐增加(>10%),入住重症医学科的重症COPD可能与侵袭性肺曲霉病发病具有相关关系,同时,在入住重症医学科的侵袭性肺曲霉病患者中,COPD患者占有极大比例,且病死率极高(67%~100%)。COPD合并侵袭性肺曲霉病的高危因素主要包括激素的应用、广谱抗生素的应用、存在气道基础疾病和曲霉定植
\protect\hyperlink{text00012.htmlux5cux23ch33-11}{\textsuperscript{{[}33{]}}}
。

\subsubsection{侵袭性肺曲霉病在慢性阻塞性肺疾病患者中的诊断?}

Bulpa等
\protect\hyperlink{text00012.htmlux5cux23ch34-11}{\textsuperscript{{[}34{]}}}
依据其回顾性研究结果,提出了慢性阻塞性肺疾病(COPD)患者合并IPA的最新诊断标准。该标准分为确诊、临床诊断、拟诊和定植4个级别。

(1)曲霉定植:COPD患者下呼吸道标本曲霉培养阳性,但不伴有呼吸困难、气道痉挛加重和新发肺部浸润影。

(2)拟诊:接受激素治疗的重症COPD,即慢性阻塞性肺疾病全球倡议(GOLD)分级为Ⅲ或Ⅳ级的患者,近期出现呼吸困难加重,3个月内的胸部影像学检查具有提示意义,而下呼吸道标本或血标本曲霉培养或显微镜检查阴性。

(3)临床诊断:在拟诊基础上下呼吸道标本曲霉培养或显微镜检阳性,血清曲霉抗体阳性,血清半乳甘露聚糖检测连续2次阳性。

(4)确诊:对3个月内新发的肺部病变进行针吸活检或尸检,经组织病理学或细胞学检查显示曲霉菌丝生长和组织破坏,并伴有下呼吸道标本曲霉培养、血清曲霉抗原或抗体检测及分子生物学、免疫学方法和(或)培养3项中任一项阳性。

该标准强调,在针对培养阳性的敏感细菌使用抗生素的剂量、途径和抗菌谱均恰当时,患者呼吸困难和气道痉挛症状无好转,或影像学具有提示意义的肺部征象,方可归为上述临床特点。

\subsubsection{慢性阻塞性肺疾病患者合并侵袭性肺曲霉菌病如何治疗?}

临床疑诊是早期治疗的依据。病理组织学证据是侵袭性肺曲霉菌病(IPA)诊断的金标准,但由于这一诊断方法的实施困难,在临床实际工作中并不实用。通常侵袭性肺曲霉菌病的早期诊断建立在患者的临床特点和实验室检查结果之上。侵袭性肺曲霉菌病的早期临床疑诊困难导致延误治疗时机,这是造成患者预后不良的主要原因之一。重视临床疑诊最有价值。据报道,减少诊断时间和早期开始治疗,82%的侵袭性肺曲霉菌病患者可以存活。因此,将侵袭性肺曲霉菌病作为重症慢性阻塞性肺疾病(COPD)患者的鉴别诊断之一,早期疑诊,尽早行纤维支气管镜和胸部CT检查是最佳诊断方法,如果存在相应病变,则立即开始治疗,并继续行相关检查以进一步诊断侵袭性肺曲霉病
\protect\hyperlink{text00012.htmlux5cux23ch35-11}{\textsuperscript{{[}35{]}}}
。

两性霉素B脱氧胆酸盐是治疗侵袭性肺曲霉病的主要药物,它具有广谱抗真菌活性、低廉的价格和长期应用的历史,但因多种不良反应(肾衰竭等)而限制其应用。脂质体形式的两性霉素B是两性霉素B的替代产品,它具有更好的肾安全性,但其缺点是费用昂贵,而且治疗需要延续到症状消失后15天。一项研究结果表明,在43例接受两性霉素B治疗的侵袭性肺曲霉病患者中,尽管11例联合使用吸入两性霉素B、5-氟胞嘧啶和伊曲康唑,但仅有3例(7%)存活
\protect\hyperlink{text00012.htmlux5cux23ch34-11}{\textsuperscript{{[}34{]}}}
。伊曲康唑和两性霉素B治疗侵袭性肺曲霉病同样有效,但伊曲康唑与多种药物存在相互作用,生物利用度差是限制其使用的主要原因。随着新的抗真菌药物出现,伊曲康唑可能将局限用于巩固治疗。目前这些药物对COPD合并侵袭性肺曲霉病患者的疗效尚不明确。联合治疗也许能够提高疗效。联合治疗具有增强杀菌活性、降低药物剂量并减少其不良反应、预防快速出现的耐药等优点。实验室研究结果提示,联合卡泊芬净和两性霉素B或伊曲康唑具有抗曲霉的相加和协同效应。体外实验、动物实验研究及少数临床报道提示,联合应用三唑类和棘白菌素类药物可降低侵袭性肺曲霉病的病死率,但还需要进一步的相关临床研究。此外,如果治疗过程中必需使用激素,则必须将其剂量限制在最低水平,因为大剂量激素是侵袭性肺曲霉病的危险因素。

侵袭性肺曲霉病的预后极差,早期诊断和早期治疗是改善预后的唯一方法,而抗真菌治疗宜早不宜迟。

\begin{center}\rule{0.5\linewidth}{\linethickness}\end{center}

参考文献

\protect\hyperlink{text00012.htmlux5cux23ch1-11-back}{{[}1{]}} .Afessa
B,Morales IJ,Scanlon PD,et al.Prognostic factors,clinical course
and hospital outcome of patients with chronic obstructive pulmonary
disease admitted to an intensive care unit for respiratory failure.Crit
Care Med,2002,30:1610-1615.

\protect\hyperlink{text00012.htmlux5cux23ch2-11-back}{{[}2{]}}
.Lightowler JV,Wedzicha JA,Elliott MW,et al.Non-invasive positive
pressure ventilation to treat respiratory failure resulting from
exacerbations of chronic obstructive pulmonary disease:Cochrane
systematic review and metaanalysis.BMJ,2003,326:185.

\protect\hyperlink{text00012.htmlux5cux23ch3-11-back}{{[}3{]}} .Ferrer
M,Esquinas A,Arancibia F,et al.Noninvasi veventilation during
persistent weaning failure:a randomized controlled trial.Am J Respir
Crit Care Med,2003,168:70-76.

\protect\hyperlink{text00012.htmlux5cux23ch4-11-back}{{[}4{]}}
.有创-无创序贯机械通气多中心协作组.以“肺部感染控制窗”为切换点行有创与无创序贯性通气治疗慢性阻塞性肺疾病所致严重呼吸衰竭的多中心前瞻性随机对照研究.中华结核和呼吸杂志,2006,29:14-18.

\protect\hyperlink{text00012.htmlux5cux23ch5-11-back}{{[}5{]}}
.ZuWallack AR,ZuWallack RL.Tiotropium bromide,a new,once-daily
inhaled anticholinergic bronchodilator for chronic obstructive pulmonary
disease.Expert Opin Pharmacother,2004,5:1827-1835.

\protect\hyperlink{text00012.htmlux5cux23ch6-11-back}{{[}6{]}} .McCrory
DC,Brown CD.Anti-cholinergic bronchodilators versus
beta2-sympathomimetic agents for acute exacerbations of chronic
obstructive pulmonary disease.Cochrane Database Syst
Rev,2002,(4):CD003900.

\protect\hyperlink{text00012.htmlux5cux23ch7-11-back}{{[}7{]}} .Cazzola
M,D'Amato M,Califano C,et al.Formoterol as dry powder oral
inhalation compared with salbutamol metered-dose inhaler in acute
exacerbations of chronic obstructive pulmonary disease.Clin
Ther,2002,24:595-604.

\protect\hyperlink{text00012.htmlux5cux23ch8-11-back}{{[}8{]}} .Cazzola
M,Califano C,Di Perna F,et al.Acute effects of higher than customary
doses of salmeterol and salbutamol in patients with acute exacerbation
of COPD.Respir Med,2002,96:790-795.

\protect\hyperlink{text00012.htmlux5cux23ch9-11-back}{{[}9{]}} .Di
Marco F,Verga M,Santus P,et al.Effect of formoterol,tiotropium,and
their combination in patients with acute exacerbation of chronic
obstructive pulmonary disease:A pilot study.Respir
Med,2006,100:1925-1932.

\protect\hyperlink{text00012.htmlux5cux23ch10-11-back}{{[}10{]}}
.Barnes PJ.Theophylline:new perspectives for an old drug.Am J Respir
Crit Care Med,2003,167:813-818.

\protect\hyperlink{text00012.htmlux5cux23ch11-11-back}{{[}11{]}} .Li
H,He G,Chu H,et al.A step-wise application of methylprednisolone
versus dexamethasone in the treatment of acute exacerbations of
COPD.Respirology,2003,8:199-204.

\protect\hyperlink{text00012.htmlux5cux23ch12-11-back}{{[}12{]}}
.Sayiner A,Aytemur ZA,Cirit M,et al.Systemic glucocorticoids in
severe exacerbations of COPD.Chest,2001,119:726-730.

\protect\hyperlink{text00012.htmlux5cux23ch13-11-back}{{[}13{]}}
.Vondracek SF,Hemstreet BA.Retrospective evaluation of systemic
corticosteroids for the management of acute exacerbations of chronic
obstructive pulmonary disease.Am J Health Syst
Pharm,2006,63:645-652.

\protect\hyperlink{text00012.htmlux5cux23ch14-11-back}{{[}14{]}} .Alía
I,de la Cal MA,Esteban A,et al.Efficacy of corticosteroid therapy in
patients with an acute exacerbation of chronic obstructive pulmonary
disease receiving ventilatory support.Arch Intern
Med,2011,171:1939-1946.

\protect\hyperlink{text00012.htmlux5cux23ch15-11-back}{{[}15{]}} .Gunen
H,Mirici A,Meral M,et al.Steroids in acute exacerbations of chronic
obstructive pulmonary disease:are nebulized and systemic forms
comparable?Curr Opin Pulm Med,2009,15:133-137.

\protect\hyperlink{text00012.htmlux5cux23ch16-11-back}{{[}16{]}}
.Maltais F,Ostinelli J,Bourbeau J,et al.Comparison of nebulized
budesonide and oral prednisolone with placebo in the treatment of acute
exacerbations of chronic obstructive pulmonary disease:a randomized
controlled trial.Am J Respir Crit Care Med,2002,1654:698-703.

\protect\hyperlink{text00012.htmlux5cux23ch17-11-back}{{[}17{]}}
.Burgel PR.Antibiotics for acute exacerbations of chronic obstructive
pulmonary disease(COPD).Med Mal Infect,2006,French.PMID:16839731

\protect\hyperlink{text00012.htmlux5cux23ch18-11-back}{{[}18{]}}
.Gamble E,Grootendorst DC,Brightling CE,et al.Antiinflammatory
effects of the phosphodiesterase-4 inhibitor cilomilast(Ariflo)in
chronic obst ructive pulmonary disease.Am J Respir Crit Care
Med,2003,168:976-982.

\protect\hyperlink{text00012.htmlux5cux23ch19-11-back}{{[}19{]}}
.Barnes PJ.Therapy of chronic obstructive pulmonary disease.Pharmacol
Ther,2003,97:87-94.

\protect\hyperlink{text00012.htmlux5cux23ch20-11-back}{{[}20{]}} .Rubio
ML,Martin-Mosquero MC,Ortega M,et al.Oral N2 acetylcysteine
attenuates elastase-induced pulmonary emphysema in
rats.Chest,2004,125:1500-1506.

\protect\hyperlink{text00012.htmlux5cux23ch21-11-back}{{[}21{]}}
.Ohbayashi H.Novel neut rophil elastase inhibitors as at reatment for
neutrophil-predominant inflammatory lung
diseases.Drugs,2002,5:910-923.

\protect\hyperlink{text00012.htmlux5cux23ch22-11-back}{{[}22{]}}
.Gainnier M,Arnal JM,Gerbeaux P,et al.Helium-oxygen reduces work of
breathing in mechanically ventilated patients with chronic obstructive
pulmonary disease.Intensive Care Med,2003,29:1666-1670.

\protect\hyperlink{text00012.htmlux5cux23ch23-11-back}{{[}23{]}}
.Rodrigo G,Pollack C,Rodrigo C,et al.Heliox for treatment of
exacerbations of chronic obstructive pulmonary disease.Cochrane
Database Syst Rev 2002,(2):CD003571.

\protect\hyperlink{text00012.htmlux5cux23ch24-11-back}{{[}24{]}}
.McCrory DC,Brown C,Gelfand SE,et al.Management of acute
exacerbations of COPD:a summary and appraisal of published
evidence.Chest,2001,119:1190-1209.

\protect\hyperlink{text00012.htmlux5cux23ch25-11-back}{{[}25{]}}
.Almagro P,Calbo E,de Echaguen AO,et al.Mortality after
hospitalization for COPD.Chest,2002,21:1441-1448.

\protect\hyperlink{text00012.htmlux5cux23ch26-11-back}{{[}26{]}}
.Donaldson GC,Seemungal TA,Bhowmik A,et al.Relationship between
exacerbation frequency and lung function decline in chronic obstructive
pulmonary disease.Thorax,2002,57:847-852.

\protect\hyperlink{text00012.htmlux5cux23ch27-11-back}{{[}27{]}}
.Spruit MA,Gosselink R,Troosters T,et al.Muscle force during an
acute exacerbation in hospitalised patients with COPD and its
relationship with CXCL8 and IGF-I.Thorax,2003,58:752-756.

\protect\hyperlink{text00012.htmlux5cux23ch28-11-back}{{[}28{]}}
.Seemungal TA,Donaldson GC,Paul EA,et al.Effect of exacerbation on
quality of life in patients with chronic obstructive pulmonary
disease.Am J Respir Crit Care Med,1998,157:1418-1422.

\protect\hyperlink{text00012.htmlux5cux23ch29-11-back}{{[}29{]}} .Sethi
S,Evans N,Grant BJ,et al.New strains of bacteria and exacerbations
of chronic obstructive pulmonary disease.N Engl J
Med,2002,347:465-471.

\protect\hyperlink{text00012.htmlux5cux23ch30-11-back}{{[}30{]}}
.Delaney A,Bagshaw SM,Nalos M.Percutaneous dilatational tracheostomy
versus surgical tracheostomy in critically ill patients:a systematic
review and meta-analysis.Crit Care,2006,10:R55.

\protect\hyperlink{text00012.htmlux5cux23ch31-11-back}{{[}31{]}} .Burns
KE,Adhikari NK,Keenan SP,et al.Use of non-invasive ventilation to
wean critically ill adults off invasive ventilation:meta-analysis and
systematic review.BMJ,2009,338:b1574.

\protect\hyperlink{text00012.htmlux5cux23ch32-11-back}{{[}32{]}}
.Ferrer M,Sellarés J,Valencia M,et al.Non-invasive ventilation
after extubation in hypercapnic patients with chronic respiratory
disorders:randomised controlled trial.Lancet,2009,374:1082-1088.

\protect\hyperlink{text00012.htmlux5cux23ch33-11-back}{{[}33{]}}
.贺航咏,詹庆元.慢性阻塞性肺疾病急性加重期合并侵袭性肺曲霉病的研究进展.中华结核与呼吸杂志,2009,32,6:463-466.

\protect\hyperlink{text00012.htmlux5cux23ch34-11-back}{{[}34{]}} .Bulpa
P,Dive A,Sibille Y.Invasive pulmonary aspergillosis in patients with
chronic obstructive pulmonary disease.Eur Respir J,2007,30:782-800.

\protect\hyperlink{text00012.htmlux5cux23ch35-11-back}{{[}35{]}}
.Vandewoude KH,Blot SI,Depuydt P,et al.Clinical relevance of
Aspergillus isolation from respiratory tract samples in critically ill
patients.Crit Care,2006,10(1):R31.

\protect\hypertarget{text00013.html}{}{}

