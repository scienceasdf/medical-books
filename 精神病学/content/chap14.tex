\chapter{人格障碍性心理障碍}

\section{人 格 障 碍}

\subsection{概述}

人格亦称个性(personality),是个体心理特征的总和,是一个人总的精神面貌。每个人都具有自己的人格特征。人格的形成与先天的生理特征及后天的生活环境均有较密切的关系。人格通过自我意识的协调和控制而具有整体性、稳定性和有别于他人的差异性。

人格障碍(personality
disorder)是指在发育过程中形成的,从童年、少年或青春期开始,延续到成年期的显著偏离常态的人格,表现为一种显著与个人表现的文化背景相偏离的持久的内心经历和行为模式,又称为人格异常、病态人格、变态人格等。

人格障碍一般具有以下特征:

1.常在个体发育的早期阶段作为体质因素和社会经历的双重结果而出现。

2.具有根深蒂固的和持久的行为模式,表现为对广泛的人际关系和社会处境产生固定的反应。

3.与特定文化背景中一般人的感知、思维、情感,特别是待人方式,有极为突出或明显的偏离。

4.常常伴有不同程度的主观的苦恼及社会功能与行为方面的问题。

人格障碍与人格改变(personality
changes)是两个不同的概念,后者是获得性的,通常出现在成年期,在严重的或持久的应激、极度的环境剥夺、严重的精神障碍、脑部疾病或损伤之后发生。人格障碍没有明确的起病时间,始于童年或青少年且持续终生。

\subsection{病因}

人格障碍形成的原因至今尚不完全清楚,通常认为是在大脑先天性缺陷的基础上,由于心理社会因素和其他环境有害因素影响的结果。

1.遗传因素 人格或个性心理特征受遗传因素影响是十分明显的,人格障碍患者亲属中人格障碍的发生率较高,双亲中脑电图异常率较高。

在双生子犯罪问题的研究中,339对同卵双生子中共犯罪率为55%;426例双卵双生子共犯罪率为17%,说明遗传因素在人格障碍中起到一定的作用。

2.心理因素 童年生活经历对个体人格的形成具有重要的作用。人格的形成与儿童时期家庭和环境的教育有关。父母关系不和睦家庭是正常家庭的2.5倍,单亲家庭人格障碍患病率是双亲家庭的5.9倍。父母的不良养育方式是人格障碍的危险因素,比如对子女的过分惩罚、羞辱、刻薄和吝啬,易使子女产生自卑感、无助感和不安全感,害怕失败和挫折,有较强的逆反心理,对周围环境容易产生敌对情绪,易记仇报复,使之成为人格障碍高危人群。此外,家庭经济收入过低和独生子女亦是人格障碍的危险因素。

3.环境因素 不良的生活环境、结交具有品行障碍的“朋友”及经常混迹于有恶习的社交圈子,对人格障碍的形成常起到重要作用;青少年由于批评能力低,行为自控能力差,受到大量淫秽、凶杀等内容的小说及影视文化的影响,容易通过模仿、观察或受教唆等习得不良行为,甚至出现越轨行为。

\subsection{临床类型及其临床表现}

人格障碍的临床表现比较复杂。根据CCMD-3可以分为以下类型:

\subsubsection{偏执型人格障碍}

偏执型人格障碍(paranoid personality
disorder)以猜疑、偏执为特点,始于成年早期,男性多于女性。主要特征有:①对挫折与批评过分敏感;②固执,狭隘,易记恨他人,缺乏宽容心;③多疑,过分警觉,常误解他人无意或友好的行为;④对自己估计过高,习惯于把失败归咎于他人,总感觉受压制、被迫害,甚至上告、上访,不达目的不肯罢休;⑤与现实环境不相称的好斗及顽固地维护个人的权利;⑥常有超价观念或病理性嫉妒。

\subsubsection{分裂样人格障碍}

分裂样人格障碍(schizoid personality
disorder)以观念、行为和外貌装饰的奇特、情感冷淡及人际关系缺陷为特点。主要特征有:①情感体验不鲜明,内心无愉快感;②性格明显内向(孤僻、被动、退缩),与家庭和社会疏远,对外界事物的情绪反应冷淡或平淡;③对他人表达情感的能力受限;④无论对批评或表扬都无所谓;⑤行为孤独,退缩;⑥过分沉湎于幻想和内省;⑦无交友兴趣,无亲密朋友;⑧性兴趣缺乏;⑨对社会常规与风俗适应不良。

\subsubsection{反社会型人格障碍}

反社会型人格障碍(antisocial personality
disorder)也称社交紊乱性人格障碍,以行为不符合社会规范,经常违法乱纪,对人冷酷,男性多于女性。病人往往在童年或青少年(18岁前)就出现品行问题,成年后习性不改,表现为行为不符合社会规范,甚至违法乱纪。其主要特征有:①对他人感受漠不关心,如经常不承担经济义务、拖欠债务、不抚养子女或赡养父母;②全面、持久地缺乏责任感,无视社会规范、规则与义务;③虽能建立人际关系,但都不能长久地保持;④对挫折的耐受性极低,微小刺激便可引起攻击,甚至引起暴力行为;⑤无内疚感,不能从经历中,特别是从惩罚中吸取教训;⑥常责怪他人,或当其与社会发生冲突时对行为做似是而非或强词夺理的解释;⑦不尊重事实,经常撒谎、欺骗他人,以获得个人利益。

【\textbf{典型案例}
】 {患者男性,32岁,经常违法乱纪,无视社会道德、法律,14岁时偷车,因攻击警察、从百货商店抢东西、自己心烦与室友打架等原因先后4次被监禁,在狱中仍好斗殴,事后无悔意,言行鲁莽,对人不尊重。曾被强制性就诊于精神病院,精神检查无定向力障碍、思维障碍等精神病性症状。无长期饮酒、药物依赖史。}

\subsubsection{冲动性人格障碍}

以情感爆发、明显行为冲动为特征,男性明显多于女性。主要特征为:①情绪不稳,易激惹,易与他人发生争执与冲突,冲动后虽对自己的行为懊悔,但不能防止再犯,间隙期正常;②人际关系时好时坏,要么与人关系极好,要么极坏,几乎没有持久的朋友。行为冲动,往往不计后果;伴有情感不稳定,行为无计划性;强烈的愤怒爆发常导致暴力或“行为爆炸”,当冲动行为被人批评或阻止时尤其如此。此型又可分为冲动型和边缘型两个亚型。

冲动型(impulsive personality
disorder):主要特征为情绪不稳定及冲动控制不良,伴有暴力或威胁性的行为爆发。当被人批评或劝阻其行为时,常出现暴力或威胁性行为。此亚型又称为暴发型或攻击型人格障碍。

边缘型(borderline personality
disorder):除情绪不稳定外,患者的自我形象、行为的目的性、内心的偏好(包括性偏好)常常是模糊或扭曲的。常有持久的空虚感,易卷入强烈的不稳定的人际关系中,导致连续的情感危机,为竭力避免被人遗弃,可出现一连串的自杀威胁或自伤行为。

\subsubsection{表演型人格障碍}

此型又称癔症型或心理幼稚型人格障碍(histrionic personality
disorder),以过分的感情用事或夸张言行吸引他人注意为特征。其主要特征有:①富于表演色彩,情绪表达做作、夸张;②暗示性高,易受他人或环境影响;③情感肤浅,易变,脆弱易受伤害;④自我为中心,追求刺激,渴望受到赞赏,热衷于引人注目的活动;⑤自我放任,外表及行为显出不恰当的挑逗性;⑥过分关心躯体的性感,以满足自己的要求。

\subsubsection{强迫型人格障碍}

强迫型人格障碍(obsessive compulsive personality
disorder)有一种普遍的模式,即过分关注整洁、完美主义精神和人际交往的限制,发病于成年早期。男性多于女性2倍,约70%的强迫症病人有强迫性人格障碍。主要特征有:①刻板固执,迂腐拘泥,墨守成规,缺乏应变能力,过分注重细节、规则、条款、秩序、组织或者日程,以至于忽视全局;②追求完美,过分注重工作细节,但又缺乏信心而反复核对;③道德感过强,过于自我克制,少有乐趣;④责任感过强,谨小慎微,过分关注安全,思想难以松弛;⑤可有强加的、令其讨厌的思想或冲动闯入;⑥不合情理地要求他人必须按自己的方式行事;⑦因循守旧,缺乏表达温情的能力。

\subsubsection{焦虑型人格障碍}

焦虑型人格障碍(anxious personality
disorders)又称回避型人格障碍,以一贯感到紧张、提心吊胆、不安全及自卑为特征,总是需要被人喜欢和接纳,对拒绝和批判过分敏感,因习惯性地夸大日常处境中的潜在危险,而有回避某些活动的倾向。其主要特征有:①自幼胆小,懦弱,易惊恐;②有持续和泛化的紧张忧虑;③因有自卑感,希望受到别人的欢迎与接受,但对批评或拒绝又过度敏感;④对日常生活中潜在的危险惯于夸大,可达到回避正常社交的程度;⑤除非得到保证被他人所接受和不会受到批评,否则拒绝与他人建立人际关系;⑥惯于夸大生活中潜在的危险因素,达到回避某种活动的程度,但无恐惧性回避。

\subsubsection{依赖型人格障碍}

依赖型人格障碍(dependent personality
disorder)以过分依赖为特征。其主要特征有:①缺乏独立性,总认为自己无依靠、无能力,缺乏精力;②情愿把自己置于从属地位,请求他人为自己的事情作决定,过分顺从他人的意志;③不愿意对所依赖的人提出合理的要求;④独处时常感到不安或无助,唯恐被人抛弃,宁愿忍辱负重;⑤当亲密的关系终止时,迫切地寻求另一个关心和帮助的来源,有被毁灭和无助的体验;⑥在逆境或不顺利时有将责任推给他人的倾向。

\subsubsection{其他或待分类的人格障碍}

包括被动性人格障碍、抑郁性人格障碍和自恋性人格障碍等。

\subsection{诊断}

人格障碍的诊断主要依靠病史的收集,除询问本人外,知情者提供的情况尤为重要。病史内容重点涉及:①日常生活安排;②人际关系;③心境或情绪反应;④性格特点;⑤处世态度及准则,包括对健康与疾病的态度;⑥生长发育史,从出生到16岁以前的情况,包括家庭教育与社会环境。

CCMD-3人格障碍的诊断标准:

【\textbf{症状标准}
】 个人的内心体验和行为特征(不限于精神障碍发作期)在整体上与其所处的文化背景明显不同,且至少有下列两种以上不同于常人的表现:

(1)认知(感知及解释任何事物,由此形成对自我及他人的态度和形象的方式)的异常偏离;

(2)情感(情感的表现方式、强度、变化度以及情感表现的合宜性)的异常偏离;

(3)人际关系的偏离异常;

(4)控制冲动以及满足个人需要的异常偏离。

【\textbf{严重标准}
】 特殊行为模式的异常偏离,足以影响到其个人、社会、职业等正常功能,使病人或其他人(如家人)感到痛苦或社会适应不良。

【\textbf{病程标准}
】 开始于童年、青春期或成年早期。年龄18岁以上,至少已持续2年。

【\textbf{排除标准}
】 人格特征的异常偏离并非躯体疾病或精神障碍的表现或后果。

在不同的文化中,需要建立一套独特的标准以适应社会常模、规则与义务。对于前节列举的大多数亚型,通常要求至少存在三条临床描述的特点或行为的确切证据时才能诊断。

\subsection{鉴别诊断}

1.需与各种精神疾病如神经症、心境障碍、精神分裂症相鉴别。

2.需与各种原因引起的持久性人格改变相鉴别,如由脑部疾病、损伤和脑功能紊乱所致的人格和行为障碍,灾难性经历后的持久性人格改变,以及精神科疾患后持久性人格改变等。

在某些情况下,人格障碍可以与精神科疾患共病,不同类型的人格障碍之间也可共患。此时应将两种或多种诊断均列出。

\subsection{病程和预后}

通常认为人格障碍因缺乏有效的治疗手段,无法治愈,常绵延终生。但近年来临床实践表明,通过药物和环境两方面积极的矫治,某些人格缺陷可以改善。另外,随着年龄的增长,无论何种类型的人格障碍,一般均可逐步趋向缓解。因此,应当克服无能为力的消极观念,积极采取相应措施,争取较好的预后。

\subsection{治疗}

由于人格障碍的病因不明,目前尚无确切有效的治疗手段。有经验表明,药物治疗和心理社会于预能收到一定的效果。

\subsubsection{药物治疗}

不同类型的人格障碍患者,其心理过程障碍的特点不同,可予以适量的药物对症治疗。

1.以认知障碍、偏执、古怪为主要表现的,如偏执型、分裂样人格障碍,可只用抗精神病药物。

2.以情感不稳定、冲动攻击为主要表现的,如社交紊乱型、情绪不稳型人格障碍,可选用心境稳定药、抗抑郁药和抗躁狂药。

3.以焦虑、强迫为主要临床表现的,如焦虑(回避)型、强迫型人格障碍,可选用抗焦虑药或抗抑郁药。

\subsubsection{心理社会干预}

人格障碍的心理治疗虽然很困难,但却是必不可少的,或者说是唯一可能有效的手段。由于大多数人格障碍能够以最基本的方式应付日常生活,一般人包括自身并不认为自己是患者,没有主动求医的愿望。在强制治疗时,容易出现抵制治疗、欺骗医务人员的现象,以致事倍功微。人格障碍最关键和最重要的问题是与周围环境格格不入,即社会化障碍。因此,心理社会干预的着眼点在于重建其心理和社会环境,创造对其关心、爱护和不受歧视的氛围,鼓励其积极参加公益性事业活动,培养其尊重他人和尊重自己的情趣,逐渐改造其不良人格。

1.个别心理治疗 治疗者首先要通过深入了解患者,与其建立良好的关系,取得其信任,以便于沟通。然后,逐渐帮助其认识个性的缺陷,并指出缺陷是可以改变的,鼓励他们树立信心,启发其自我认同感和人际同情心,检讨自己的缺陷并寻找改变的途径,努力改善与家人、同学、同事之间的关系。治疗者与患者要保持适当的情感距离,不可过分亲近,始终站在中立的立场,有了进步予以鼓励,存在问题要严肃指出,充分显示治疗者的权威和力度,促使其信服和听从指导意见。如遇到困境时可进行危机干预。

2.集体心理治疗 组织患者参加治疗性社区(或称治疗性团体)的活动。这是一种较好的生活和学习环境。通过与参加活动的其他成员相互交流,提高认识,有利于控制和改善偏离的行为,丢弃那些获得和习得的不良习惯,探索和寻求建立新的行为方式的方法和途径。实践证明,集体心理治疗的效果优于个别心理治疗。

\section{性心理障碍}

\subsection{概述}

性是人类生存的一种基本需要,性行为是人的本能活动之一。人类通过性行为使生殖功能得以保存和延续种类,同时由于人类性行为的社会化,性爱还体现为两性的感情结合即爱情,已成为社会稳定和发展的重要因素。人类性行为受社会文化的制约,不同国家、种族、社会团体对性行为历来有不同的价值观;即使同一文化的国家,在不同的历史发展阶段,对某种性行为的评价也有很大的不同。如何评价性行为的正常与异常,至今没有确切的标准,区别只是有条件的、相对的,通常包括以下几个方面:

1.以现实的社会道德规范为准则;

2.以生物学特点为准则;

3.以对他人或社会的影响为准则;

4.以对本人的影响为准则。

性心理障碍(psychosexual
disorder)是一组精神障碍,其主要特征是在两性行为方面的心理和行为明显偏离正常,并以这类性偏离作为性兴奋、性满足的主要或唯一方式。正常的异性性活动受到全部或某种程度的破坏、干扰或影响,但一般的精神活动并无其他明显异常。性心理障碍最常见的表现形式为性变态(sexual
deviation),包括性身份障碍、性偏好障碍、与性发育和性取向有关的心理及行为障碍,共同特征为性兴奋的唤起、性对象的选择以及两性行为方式等方面出现反复持久的异乎常态的表现。

\subsection{病因与发病机制}

19世纪早期,学者通常认为性变态是一种先天性异常,假设其具有生物学基础,但经长期研究未能找到确切的证据。目前多数学者认为,性变态是通过后天经验获得的。其心理学解释有精神动力学派和行为主义学派,但任何一个学派的理论都不能令人信服地解释和治疗多数患者。近年来一些学者将两派理论加以融合,提出整合理论模式。现分述如下:

\subsubsection{生物学因素}

一些学者发现,性变态可与某些疾病伴随发生,如颞叶病变、酒精中毒、颅脑外伤、精神分裂症、精神发育迟滞、老年性精神病以及内分泌疾病等,但迄今未能得到公认。

\subsubsection{个体心理发育因素}

1.精神动力学理论 该理论认为,性变态是在正常发育过程中异性恋发展遭到失败的结果。患者常为男性,源自儿童早期恋母情结形成的阉割焦虑和分离焦虑的威胁,在无意识中持续发生作用。当患者在现实环境触发因素作用下,解决两性问题遭遇困难或挫折。为缓解焦虑和心理冲突,在心理防卫机制作用下,性心理退行到儿童早期幼稚的发展阶段,使异性恋发展受挫,性生殖功能不能整合为成熟的状态,性冲动成为成熟的性心理和性行为方式。

2.行为主义学派理论 该理论认为,性变态是后天习得的行为模式。经验学习理论认为,具有敏感人格素质的人对周围环境中某种事物或某种情境,偶然与性兴奋性满足相结合,产生某种反应。此反应可因相同情景反复出现而强化,从而牢固地形成病理性联系。

3.整合理论模式 该理论主张对不同理论可部分地整合在一起加以应用。在社会化过程中所发展的对性的认知、信念、态度和行为方式,对性变态的发生有重要作用。在社会化过程中,良好的教育可引导儿童学会社会期待行为,反之,管教失当会诱发非社会化行为。由于性问题在家庭和社会中往往是不准讨论的,因此儿童的性偏离行为得不到及时纠正,被隐匿,甚至不断在性兴趣、性想象和不良性活动中得到强化,以致成为难以纠正的“沉疴”。

\subsubsection{心理社会因素}

萌发于儿童期,表现于成人期的性变态常受多种心理社会因素的影响,主要涉及以下几方面:①正常的异性恋活动受挫,如失恋、单恋、性活动的失败或抑制的痛苦经验;②遭遇重大负性生活事件,如人际关系、家庭或事业的失败与打击;③儿童期家庭环境中的不良刺激,如反性别着装或暗示、家庭性虐待等;④不良社会文化的影响,如淫秽出版物、色情物品、性骚扰或性诱惑等;⑤个体人格因素:某些性格特征突出者或某种人格障碍者,如害羞、内向、孤僻、不善交际、女子气等,有易患性变态的趋势。

\subsection{临床类型}

\subsubsection{性身份障碍}

性身份障碍主要指易性症(transsexualism),患者对自身性别的认定与解剖生理上的性别特征呈持续厌恶的态度,渴望像异性一样生活,成为异性队伍中的一员,希望通过激素治疗或外科手术使自己的性器官与异性一致。

诊断要点:①要求转换性别至少持续2年以上;②排除其他精神障碍如精神分裂症的伴发症状;③排除雌雄同体、遗传或性染色体异常等情况。

\subsubsection{性偏好障碍}

1.恋物症(fetishism) 以某些非生命物体作为性唤起及性满足的刺激物。所恋物品多为人体的延伸物,如衣服、鞋袜,或具有某种特殊质地的物品,如橡胶、皮革制品。只有当迷恋物品是性刺激的最重要来源或达到满意性反应的必备条件时,才能诊断恋物症。该症几乎仅见于男性。

恋物症患者所眷恋的妇女用品常有胸罩、内衣、内裤、手套、手绢、鞋袜、饰物等。患者接触所偏爱的物品时可导致性兴奋甚至达到性高潮,体验到性的快乐。因此他们采取各种手段甚至不惜冒险偷窃妇女用品并收藏起来,作为性兴奋的激发物。也有患者表现为对女性身体的某一部位如手指、脚趾、头发、指甲迷恋。

2.露阴症(exhibitionism) 一种反复发作或持续存在的倾向,即向陌生异性或公共场所的人群暴露生殖器,但与对方保持安全距离,并无进一步勾引或接近的意图。在露阴时通常出现性兴奋并继以手淫,几乎仅见于男性。这些人多数将露阴作为发泄性欲的唯一出路,如果目击者表现出震惊、恐惧时,其兴奋性常会增强,事后常自感悔恨,但冲动难以控制。大部分露阴症患者性功能低下或缺乏正常性功能,有的明确表示对性交不感兴趣。

3.窥阴(淫)症(scopophilia) 是一种反复出现或持续存在的窥视异性裸体、性器官或他人亲昵和性交行为的倾向,通常引起性兴奋和手淫。见于男性,常以此作为性满足的主要或唯一来源。

4.恋童症(paedophilia) 性偏好的对象为儿童,通常为青春期前或青春初期(一般为12~13岁或更小)的孩子。某些恋童症的迷恋对象仅为女孩,另一些则仅为男孩,也有些对两性儿童均有兴趣。在恋童症中,也包括那些对成人性伴侣保留性偏好的男性,但由于他们正常的性生活受挫,便习惯地转向以儿童作为替代物。有些男性对未成年子女进行性骚扰,偶尔也会接近其他儿童,这两种行为都是恋童症的指征。

5.性施虐和性受虐症(sexual sadism, sexual
masochism) 是将捆绑、施加痛苦或侮辱带入性生活的一种偏好。如果个体乐于承受这类刺激,便为受虐症;如果是施予者,便为施虐症。个体常常从施虐和受虐两种活动中可获得性兴奋。需要鉴别的是:一是在正常的性活动中,也常有轻度的施虐受虐刺激用来增强快感,只有那些以施虐受虐活动作为最重要的刺激来源或性满足的必备手段时才可作出诊断;二是施虐症有时很难与性接触中的残暴行为或与色欲无关的愤怒相区别,只有当暴力是性欲唤起的必备条件时,诊断才成立。

\subsubsection{性指向障碍}

性指向障碍有多种表现形式,常见形式为同性恋(homosexuality),是指对同性的人持续表现性爱倾向,可伴有或不伴有性行为,而同时对异性的人毫无性爱倾向,也可有减弱的性爱倾向或正常的性行为。

由于各国法律、文化、制度等差异,对同性恋的评判标准有很大差别,越来越多的国家倾向于不把它列为病态。

\subsubsection{其他性偏好障碍}

1.性摩擦症 反复多次在拥挤的公共场所,用生殖器接触摩擦异性身体以获得性刺激的行为,常见于男性。

2.恋兽性 指与动物反复发生性行为以取得性满足。

3.恋尸症 指与异性尸体发生性行为以取得性满足。

4.自虐症 用针刺阴茎或乳头,或将物体插入尿道或直肠,或用自我窒息等办法取得性满足的行为。

5.恋污物症 指以嗅、舔衣服或身体上的污物(尿、粪、汗等)以取得性满足的行为。

6.恋残体征 指喜与肢体残缺人发生性行为以取得性满足的行为。

\subsection{诊断}

诊断主要依据病史、生活经历和临床表现,排除性激素及有无染色体畸变继发的器质性障碍。性心理障碍的共同特点主要包括:

1.表现为性对象选择或性行为方式的明显异常,这种行为较固定和不易纠正,且不是境遇性的;

2.行为的后果对个人及社会可能带来损害,但不能自我控制;

3.患者本人具有对行为的辨认能力,自知行为不符合一般社会规范,迫于法律及舆论的压力,可出现回避行为;

4.除了单一的性心理障碍所表现的变态行为外,一般社会适应良好,无突出的人格障碍。

\subsection{治疗}

总体来说,性心理障碍尚无确切有效的、系统成熟的治疗方法。从20世纪初至今,不少学者先后试用以下一些方法,在部分病例中获得成功。

\subsubsection{药物及手术疗法}

1.阉割手术治疗露阴症、恋童症等,据称有20%的人得到改善。此法现已废弃。

2.激素对抗治疗 使用女性激素或人工合成的抗睾丸素制剂,能降低男性激素的生理效应,用以治疗露阴症等,有一定疗效。但其不良反应为易产生抑郁状态。

3.药物治疗 选用具有脑垂体性腺系统抑制效应的抗精神病药物,如Thioridazine用以治疗露阴症等取得初步成效,有待深入研究。

\subsubsection{心理疗法}

性心理障碍与性行为异常者多不主动就医,很少有强烈和持久的治疗愿望,心理干预比较困难。心理治疗只能对具有强烈治疗愿望者才能有较好的效果。常用的方法为行为疗法中的厌恶条件化疗法,即让患者想象性变态渴求体验场景时,当性兴奋达到高潮时,给予、诸如电击等痛苦的或不愉快的刺激,两者紧密结合,反复多次之后,形成厌恶条件反射,即当患者一想到某种性变态场景,同时就体验到恶性刺激的痛苦,从而戒除或避免出现异常性行为。

值得注意的是,在各种性心理障碍或性行为异常的心理治疗过程中,医患双方应遵循以下一些原则:①医患双方要取得共识,互相鼓励,建立信心;②消除不良情绪因素,从改变认知入手,逐渐矫正行为模式,恢复自然的性功能;③夫妻共同治疗;④注意性活动以外的其他因素的影响;⑤为患者保密。
