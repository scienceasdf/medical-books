\chapter{儿童期情绪与行为障碍}

\section{儿童情绪障碍}

儿童情绪障碍(children emotional
disorders)是发生在儿童少年期以紧张、焦虑、恐怖为主要临床表现的一组疾病。因儿童心理生理发育年龄特点,其临床表现与成人的神经症有明显的差别,主要表现在:

(1)有情绪障碍的儿童多数儿童至成年期表现正常,只有少数有神经性障碍(简称神经症);成年期起病的神经症患者的儿童期,并无明显的精神病理学预兆,两者之间并无连续性。

(2)很多情绪障碍儿童本身并无质的异常。

(3)理论假设的机理不同。

(4)儿童期多种情绪障碍之间界限不清楚。

\subsection{流行病学}

儿童情绪障碍的发生率在儿童精神障碍中占第二位,仅次于儿童行为障碍。Rutter等(1970年)的报道表明,各种情绪障碍在儿童少年中患病率为2.5%。

国内各地区报道的患病率差异很大,0.3%~21%都有过报道。一般来说,本症的患病率城市儿童比农村高,大城市比小城市高;年龄小者少,愈大就愈接近成年人的患病率。而性别差异在年龄小者不明显,在年龄大者则是女性比男性较多见。

\subsection{病因与发病机制}

\subsubsection{神经生化及生理因素}

情绪障碍的形成涉及范围极其广泛,产生的机制尤其复杂。关于情绪障碍的神经生化机制,以动物实验或成人临床研究为主,对儿童的研究极少。一般认为焦虑症与5-HT增高、多巴胺降低以及γ-氨基丁酸功能不足有关。也有实验证明焦虑症与外周去甲肾上腺素的释放增多有关,因此临床出现一系列症状,如心跳加快、皮肤苍白、多汗、口干等。而强迫症可能与5-HT功能增强有关,凡能抑制5-HT再摄取的药物,对强迫症均有一定疗效。正电子发射计算机断层扫描(PET)的研究提示,强迫症患者有前额叶眶部皮质及纹状体(主要是尾状核)的功能失调,因此提出OCD环路假说,即强迫症患者存在眶额回-纹状体-苍白球-丘脑功能环路异常(Perani,
1995)。近年来,功能核磁共振(fMRI)研究发现OCD患者前扣带回、前额叶、中颞叶均出现激活增加,进一步支持了OCD环路假说(Adler,
2000年)。

\subsubsection{遗传及易感素质因素}

有关遗传因素研究方面的报道显示,约15%的焦虑症患儿的父母和同胞也有焦虑,约50%的焦虑症患儿的双生子有类似的诊断。Sworth等(1980年)研究发现,17名强迫症患儿家庭中有7个母亲,4个父亲及1个姐姐也患有强迫性神经症。美国精神卫生研究所(NIMH)最近研究发现,强迫症先证者2级亲属中有20%符合强迫症诊断(Lenane等1990年),但受影响的家庭成员的强迫症首发症状常与家庭中最早发病者症状不同。这些研究表明,遗传因素对情绪障碍的发病有很大影响。

同时,易感素质在发病上具有重要影响。如幼儿时期胆怯、敏感、过分依赖、情感不稳定、反复无常、易受环境影响、易受暗示者易产生情绪障碍。

\subsubsection{家庭因素}

在家庭中父母的行为和对刺激的反应方式往往影响儿童情绪障碍的发生。孩子会通过父母进行榜样学习。若父母本身有刻板的行为、动作,比如过分爱清洁、怕脏、对物品摆放要求严格等,易使子女习得强迫性的行为模式,对强迫症的诱发起到极大影响;若父母胆小,遇事易紧张不安,对一些环境或物体等易产生恐惧反应,易使子女习得这些恐惧反应,促使恐惧症的发生。

\subsubsection{早年生活经验}

幼儿时体验到不稳定的家庭生活,或受惊吓、批评、侮辱,或与父母突然分离,或经历手术、不幸事故、亲人重病或死亡等精神创伤等也是情绪障碍的常见致病因素。这可能是因为在受到这些创伤之后,大脑皮质兴奋或抑制过程过度紧张,或相互冲突,形成孤立的病理惰性兴奋灶,引起过分而持久的情绪反应。

从精神分析理论角度来看,焦虑症可能是由于本能欲望不能得到满足而被压抑在无意识内,引起内在的冲突,而通过神经症症状使得内在冲突得以缓和。强迫症状是对威胁性冲动的外在表现,这类冲动是由于早年的内心冲突重新活化的结果。此外,随着依恋理论在精神病学领域的不断深入研究,目前认为婴儿期的非安全型依恋是日后发展为焦虑障碍的高危因素(Bernstein,
1996年),矛盾型依恋的婴儿在青少年期出现焦虑障碍的比例较高(Manassis,
1994年),父母自身的依恋类型对儿童焦虑障碍的发生也有影响。

\subsubsection{其他因素}

躯体疾病或过度紧张疲劳、学习负担过重、睡眠不足等对情绪障碍发病均有影响。这些因素可能与以上几种因素相互作用,进而对发病产生影响。

\subsection{常见类型}

儿童情绪障碍的常见类型有焦虑症、恐惧症、强迫症和癔症等,但有几种临床类型不易分型,且常有重叠,它们的临床表现、诊断和鉴别诊断分别叙述如下。

\subsubsection{焦虑症}

焦虑症(anxiety
disorder)是在儿童时期无明显原因下发生的发作性紧张、莫名恐惧与不安,常伴有自主神经系统功能的异常。这是一种较常见的情绪障碍。

1.临床表现 焦虑有三种表现形式:

(1)主观的焦虑体验:如患儿感到不安、烦躁、害怕、惶恐等。

(2)外显的不安行为:如年幼儿童表现为爱哭闹,出现睡眠障碍、排泄习惯紊乱等;学龄儿童表现为上课不安,坐不住,易和同学、老师发生冲突,学习效率低等。有的焦虑儿童还出现不敢当众讲话,纠缠父母寸步不离,拒绝上学,逃学等行为。

(3)生理反应现象:焦虑状态时的生理反应现象比较突出,会出现交感神经、副交感神经兴奋所产生的自主神经功能紊乱的症状,如胸闷、心悸、呼吸加速、血压升高、多汗、口干、头晕、恶心、腹部不适、四肢发凉、便秘、尿频、遗尿、睡眠不宁、早醒、多梦等。在焦虑发作时,交感神经活动增强,肾上腺皮质激素分泌增多,患儿可出现高度的激动状态,在某些紧张、恐惧的情况下,有时会发生昏厥现象。

不同的患儿,三方面的表现程度不一样或以其中的一种为主要的临床形式。

2.临床分型

(1)根据其发病形式、临床特点和病程,分为:

①惊恐发作(panic
attack):为急性焦虑发作,表现为突然发生强烈的烦躁不安、紧张、恐惧,伴有明显自主神经功能紊乱的症状。发作时间较短,多发生在具有焦虑倾向的儿童,如婴儿期对外界刺激过分敏感,容易引起惊哭、紧张和恐惧等现象,年龄大的儿童还会出现对新环境焦虑,出现适应障碍。

②广泛性焦虑症(generalized anxiety
disorder):焦虑程度较惊恐发作轻,但持续时间长。有的表现为上课怕老师提问,担心考试成绩不好,怕与同学相处等,也会伴有自主神经系统功能紊乱症状。

(2)根据发病原因和症状特征,分为:

①分离性焦虑(separation anxiety
disorder):多见于学龄前儿童。与亲人分离时深感不安而产生明显的焦虑情绪,甚至多数患儿常无根据地担心亲人会离开自己发生危险,将会发生意外的事故,会大祸临头,使自己与亲人失散,或自己被拐骗等,因此不愿意离开亲人,不去幼儿园或拒绝上学,即使勉强送去,也表现哭闹、挣扎,出现自主神经系统功能紊乱的症状如呕吐、腹痛等。病程可持续数月至数年。如一名6岁女孩,因某次父母争吵,开始出现焦虑症状,在幼儿园坐立不安,暗自流泪,下课总要纠缠老师打电话到母亲单位问母亲在干嘛,若一时联系不到母亲则大哭,怀疑妈妈被汽车压死了,剧烈腹痛,要回家找妈妈。后来干脆不去上学,也不让母亲上班。

②过度焦虑反应(overanxious
disorder):多见于学龄儿童和少年,女性较多。其病前个性胆小、多虑,缺乏自信心,对事物反应敏感,同时有自主神经系统症状。表现对未来过分担心、忧虑,有不切实际的烦恼。如担心完不成学业,担心考试成绩差,怕黑暗,怕孤独,常为一点小事影响情绪而惴惴不安、焦虑烦恼。

③社交性焦虑(social anxiety
disorder):常与社交恐惧症相伴。患儿每当与人接触或谈话时会紧张、害怕、局促不安,尤其是当接触陌生人或在新环境,表现持久而过分紧张不安、烦躁焦虑,并企图回避。如一个12岁女孩,从某农村中学转入县城中学读书,转学后出现怕与同学接触谈话,回避班级集体活动,开始是低头不看人,后来天天包一块方巾上课,常常逃课以回避与人交往。

3.诊断

(1)惊恐发作(惊恐障碍)(根据CCMD-3的诊断标准)

1)症状标准:惊恐发作需符合以下4项。

①发作无明显诱因、无相关的特定情境,发作不可预测。

②在发作间歇期,除害怕再发作外,无明显症状。

③发作时表现强烈的恐惧、焦虑及明显的自主神经症状,并常有人格解体、现实解体、濒死恐惧或失控感等痛苦体验。

④发作突然开始,迅速达到高峰,发作时意识清晰,事后能回忆。

2)严重标准:患者因难以忍受又无法解脱,而感到痛苦。

3)病程标准:在1个月内至少有3次惊恐发作,或在首次发作后继发害怕再发作的焦虑持续1个月。

4)排除标准:①排除其他精神障碍,如恐惧症、抑郁症或躯体形式障碍等继发的惊恐发作;②排除躯体疾病如癫痫
、心脏病发作、嗜铬细胞瘤、甲状腺功能亢进或自发性低血糖等继发的惊恐发作。

(2)分离性焦虑症(根据CCMD-3的诊断标准)

1)症状标准

①担心依恋对象可能遇到伤害,或害怕依恋对象一去不复返。

②担心自己会走失、被绑架、被杀害,或住院,以至与依恋对象离别。

③因不愿离开依恋对象而不想上学或拒绝上学。

④非常害怕一人独处,或没有依恋对象陪同绝不外出,宁愿呆在家里。

⑤没有依恋对象在身边时不愿意或拒绝上床就寝。

⑥反复做噩梦,内容与离别有关,以致夜间多次惊醒。

⑦与依恋对象分离前过分担心,分离时或分离后出现过度的情绪反应,如烦躁不安、哭喊、发脾气、痛苦、淡漠或退缩。

⑧与依恋对象分离时反复出现头痛、恶心、呕吐等躯体症状,但无相应躯体疾病。

2)严重标准:日常生活和社会功能受损。

3)病程标准:起病于6岁前,符合症状标准和严重标准至少已1个月。

4)排除标准:不是由于广泛发育障碍、精神分裂症、儿童恐惧症及具有焦虑症状的其他疾病所致。

(3)广泛性焦虑症(包括过度焦虑反应)(根据CCMD-3的诊断标准)

1)症状标准:以烦躁不安、整日紧张、无法放松为特征,并至少有下列2项。

①易激惹,常发脾气,好哭闹。

②注意力难集中,自觉脑子里一片空白。

③担心学业失败或交友遭到拒绝。

④感到易疲倦、精疲力竭。

⑤肌肉紧张感。

⑥食欲不振、恶心或其他躯体不适。

⑦睡眠紊乱(失眠、易醒、思睡却又睡不深等)。

⑧焦虑与担心出现在2种以上的场合、活动或环境中。

⑨明知焦虑不好,但无法自控。

2)严重标准:社会功能明显受损。

3)病程标准:起病于18岁前,符合症状标准和严重标准至少已6个月。

4)排除标准:不是由于药物、躯体疾病(如甲状腺功能亢进)及其他精神疾病或发育障碍所致。

(4)社交焦虑:参见后面社交恐惧症的诊断标准。

4.鉴别诊断

(1)恐惧症:患儿往往具有具体的恐惧对象,同时伴有回避行为、焦虑情绪及自主神经系统功能紊乱症状。而焦虑症患儿是莫名恐惧紧张,缺乏具体恐惧对象。

(2)强迫症:以强迫观念和强迫动作为主要症状,同时伴有焦虑情绪和适应困难。而焦虑症是以焦虑情绪为主要症状,一般不伴有强迫症状。

(3)精神分裂症:精神分裂症患儿可有坐立不安等焦虑症状,但是同时具有幻觉妄想、思维联想障碍、情感不协调等特征性症状,因此不难与焦虑症相鉴别。

(4)其他:某些药物、躯体疾病(如甲状腺功能亢进)也能引起焦虑反应,但根据相应的服药史、躯体症状和体征以及实验室检查即可与焦虑症进行鉴别。

\subsubsection{恐惧症}

恐惧症(phobia)是指对某些物体或某些特殊环境明知不存在对自身具有真实的危险,却产生异常强烈的恐惧,伴有焦虑情绪及自主神经系统功能紊乱症状,并有回避行为,以期达到解除恐怖所致的痛苦。正常儿童对一些物体和特殊情境,如黑暗、动物、昆虫、死亡、登高、雷电等会产生恐惧,但其恐惧的程度轻、时间短,时过境迁,恐惧心理很快消失,这是正常情绪反应,不属于恐惧症范畴。

1.临床症状 儿童恐惧症的临床症状有三个方面的表现:

(1)主观的恐惧体验:患儿往往对某种物体或某些特殊环境产生异常强烈、持久的恐惧,明知恐惧的对象对自身无危险,也不必要,但无法自制,内心极其痛苦。

(2)回避行为:患儿有逃离恐惧现场的回避行为。如一个9岁女孩每次见到毛毛虫就大声喊叫,呼吸急促,跳着跑着要逃离现场。

(3)自主神经系统功能紊乱:表现为呼吸急促、面色苍白或潮红、出汗、心慌、胸闷、血压上升、恶心、四肢震颤或软弱无力,重者可瘫软在地、昏厥、痉挛,或有饮食和睡眠障碍等。

2.临床类型 恐惧症临床常见类型主要有:

(1)动物恐惧(zoo phobia):如对鸟类、猫、狗、昆虫等产生恐惧。

(2)特殊环境恐惧(specific situational
phobia):如对黑暗、广场、高处、学校、火光、强声、雷电等环境的恐惧,持续时间不长,仅限于某种特殊环境。

(3)社交恐惧(social
phobia):指患儿与人交往时产生恐惧,出现回避行为以及自主神经系统功能紊乱等症状。

(4)广场恐惧(agoraphobia):指对公共场所恐惧,伴有焦虑、抑郁、强迫观念和人格解体,患儿表现为不敢离开家,不敢乘公共汽车,不敢进商店购物,影响工作与学习,内心极度烦恼。

(5)疾病恐惧(disease
phobia):对细菌、患病、出血、死人等产生强烈的恐惧。

(6)学校恐惧:强烈拒绝上学,长期旷课,对上学表现明显的焦虑和恐惧,并常诉述自己有病,但查不出其疾病所在,而在家可以学习,亦无其他不良行为的表现,这种现象称为学校恐惧症(school
phobia)。

3.诊断

(1)儿童恐惧症(根据CCMD-3的诊断标准)

1)症状标准:对日常生活中的一般客观事物和情境产生过分的恐惧情绪,出现回避、退缩行为。

2)严重标准:日常生活和社会功能受损。

3)病程标准:符合症状标准和严重标准至少已1个月。

4)排除标准:不是由于广泛性焦虑障碍、精神分裂症、心境障碍、癫痫
所致精神障碍、广泛发育障碍等所致。

(2)儿童社交恐惧症(包含社交焦虑症)(根据CCMD-3的诊断标准)

1)症状标准

①与陌生人(包括同龄人)交往时,存在持久的焦虑,有社交回避行为。

②与陌生人交往时,患儿对其行为有自我意识,表现出尴尬或过分关注。

③对新环境感到痛苦、不适、哭闹、不语或退出。

④与家人或熟悉的人在一起时,社交关系良好。

2)严重标准:显著影响社交(包括与同龄人)功能,导致交往受限。

3)病程标准:符合症状标准和严重标准至少已1个月。

4)排除标准:不是由于精神分裂症、心境障碍、癫痫
所致精神障碍、广泛性焦虑障碍等所致。

(3)学校恐惧症(根据Bery、Nichols和Pritchard提出的4条诊断标准)

①去学校产生严重困难。

②严重的情绪焦虑。

③家长明知患儿在家是因恐惧不去上学。

④缺乏明显的反社会行为。

4.鉴别诊断 学校恐惧症应与逃学儿童鉴别。前者大多学习成绩较好,有焦虑恐惧的情绪,但行为品德无问题;而逃学儿童无情绪问题,行为品德问题甚多,学习成绩较差,仔细观察可以鉴别。

\subsubsection{强迫症}

强迫症(obsessive compulsive
neurosis)是儿童时期以强迫观念和强迫动作为主要症状,伴有焦虑情绪和适应困难的一种心理障碍。

1.临床表现 强迫症状主要有强迫观念和强迫动作。

(1)强迫观念:指不自主重复出现的思想、观念、表象、冲动等。主要有以下几种情况:

①强迫性怀疑(obsessive
doubt):是指对自己刚说过的话或做过的事不断加以怀疑。如一患儿总怀疑课本、作业本没有带齐,因而总要反复整理书包,睡前要整理四五遍才能上床睡觉,上学出门前也要整理四五次,甚至还要求母亲再帮他整理一遍才能放心出门。

②强迫性回忆(obsessive
reminiscence):患儿对过去经历过的事情、看见过的场景、听过的音乐或别人说过的话反复回想。若在回忆过程中被外界刺激干扰,该回忆必须从头开始,否则内心烦躁、焦虑不安。

③强迫性对立观念(obsessive contradictory
idea):患儿每出现一种观念,立刻又出现完全对立的另一个观念。若对立内容涉及父母、老人、伟人时,患儿往往明知不对,但却控制不住,为此感到非常紧张、害怕、痛苦。如一个高中男孩,一方面很喜欢数理化,而且数理化成绩很好,但却忍不住责问自己“我为什么要喜欢数理化?为什么别人不喜欢数理化?”不断处于这种冲突中,难以控制,为此非常痛苦。

④强迫性穷思竭虑(obsessive
rumination):患儿在相当长的时间里固定思考某一件事情或某一问题,如“人为什么要死?”、“人死后会变成什么?”、“怎么才能让别人知道我是真死还是假死?”“死时疼不疼?”等等。患儿有时明知道这种思考是无意义的,也无法考证的,但却无法摆脱这些思考。

⑤强迫性意向(obsessive
idea):患儿感到有一种强有力的内在驱使力量,马上就要行动起来的冲动感,实际上并不能直接转变为行动。如站在火车站台上,火车将来时,总想要冲到铁轨上,但从没真正地行动。对此患儿非常恐惧,无法控制。

(2)强迫性动作(compulsion)

①强迫性洗涤(obsessive
washings):是强迫性动作最多见的症状。如患儿因对肮脏的厌恶、疾病的恐惧而反复洗手,一旦手稍微碰触到其他物体,则必须再洗。有的除了反复洗手外,还反复洗衣服、被褥等。有的患儿在强迫洗手时还带有仪式性动作,如洗手时要数1、2、3、4\ldots{}\ldots{}至50,若在数的过程中被打断,则又必须从头洗手和数数。

②强迫性仪式动作(obsessive
rituals):患儿行为上有一套先后次序的动作,在做某些事之前必须重复做一系列的动作或必须重复到一定的次数。在进行这套动作中认为没做好或中间被打断后又要重新开始,直到患儿觉得可以了才能停止。如一患儿进家门之前必须在家门口向前走7步,再向后走7步,反复来回三趟才拿出钥匙开门,且若在该过程中有人打岔如问他在干什么,则必须重头来过。

2.诊断 儿童强迫症目前尚无专用的诊断标准,临床上常参照成人的诊断标准。根据CCMD-3,强迫症的诊断标准如下:

(1)症状标准

1)符合神经症的诊断标准,并以强迫症状为主,至少有下列1项:

①以强迫思想为主,包括强迫观念、回忆或表象、强迫性对立观念、穷思竭虑、害怕丧失自控能力等。

②以强迫行为(动作)为主,包括反复洗涤、核对、检查或询问等。

③上述的混合形式。

2)患者称强迫症状起源于自己内心,不是被别人或外界影响强加的。

3)强迫症状反复出现,患儿认为没有意义,并感到不快,甚至痛苦,因此试图抵抗,但不能起效。

(2)严重标准:社会功能受损。

(3)病程标准:符合症状标准至少已3个月。

(4)排除标准

①排除其他精神障碍的继发性强迫症状,如抑郁症和精神分裂症、抑郁症或恐惧症等。

②排除脑器质性疾病,特别是基底节病变的继发性强迫症状。

3.鉴别诊断

(1)儿童精神分裂症:儿童精神分裂症早期症状可能有强迫状态,但其强迫观念的内容往往很荒谬,随后可能出现明显的情感淡漠、退缩离群、思维联想障碍、妄想、幻觉等症状。

(2)儿童孤独症:孤独症儿童常出现刻板的动作和仪式行为,但其往往具有严重交往障碍、语言功能障碍和智力发育障碍等,可与强迫症儿童鉴别。

(3)抽动秽语综合征:部分抽动秽语综合征儿童同时存在不自主的刻板重复的动作和行为,或仪式动作和行为,有的还出现强迫计数、重复语言,但抽动症极少同时具有强迫观念,且由于具有抽动病史,可以此鉴别。

\subsubsection{癔症}

癔症(hysteria)是传统用词,在ICD-10和DSM-IV中分别称为分离性和躯体形式障碍,而在CCMD-2-R中仍保留此名称。该症是由明显精神因素导致的精神障碍,主要表现为感觉或运动障碍,或意识状态改变,症状无器质性基础。

1.临床表现 癔症的症状表现多种多样,通常分为两大类:

(1)分离性癔症(dissociative
hysteria):表现为情感爆发和意识改变。如幼儿大哭大闹,四肢乱动,屏气,面色苍白或青紫,大小便失控等;较大儿童可烦躁、哭闹,并伴有冲动行为,如砸东西、拔头发、撕衣服,或在地上乱滚、四肢抽动等,有的还出现“晕倒”现象。整个发作时间长短不一,发作后有部分遗忘。

(2)转换性癔症(conversion
hysteria):表现为躯体功能障碍,包括感觉障碍如疼痛、感觉缺失、失明、失聪等和运动障碍如瘫痪、失语、抽搐、震颤、共济失调、步态不稳等。

癔症表现虽然多样,但却具有以下一些共同特点:

(1)症状无器质性病变基础,其症状不能用神经解剖、生理学、医学等知识解释。

(2)症状变化迅速、反复,亦不符合一般器质性疾病的规律,如患者双下肢突然瘫痪,不到半小时即痊愈,若遇精神因素又可重新发作。

(3)自我中心性:一般常在引人注目的时间或地点发作,围观时症状加重,症状表现有夸大性和表演性。

(4)暗示性:容易受周围环境的暗示发病、加重或好转,有的还在自我暗示情况下发病。因此,有时客观的发病原因消除但症状发作并不能消失。

2.诊断

(1)分离性癔症

1)诊断要点

①具有分离性癔症的临床特征。

②不存在可以解释症状的躯体障碍的证据。

③有心理致病的证据,表现在时间上与应激性事件、问题或紊乱的关系有明确的联系(即使患儿否认这一点)。

2)确诊依据

①不存在躯体障碍的证据。如果存在或可能存在躯体障碍,有任何疑问,或无从理解障碍为什么会发生,诊断应保留为“可能诊断”或“暂时诊断”。

②对患儿的心理社会背景及人际关系应有充分了解,从而有可能对障碍形成原因做出有说服力的推断。

(2)转换性癔症的诊断标准(DSM-Ⅳ)

①影响着自主运动或感觉功能,并提示是神经系统或其他一般躯体情况的一种以上症状。

②可以判断有心理因素伴随这些症状或缺陷,因为在症状的发生或恶化之前都有心理冲突或其他应激。

③这些症状或缺陷都不是有意识产生或伪装的(如人为性障碍或诈病)。

④在适当的调查了解后,可以发现这些症状或缺陷不可能用一般躯体情况或某种物质的直接效应来解释,也不像其文化所认可的行为或体验。

⑤这些症状或缺陷产生了临床上明显的痛苦、烦恼或在社交、职业或其他重要方面的功能缺损,或者要找内外科医生做出评价保证。

⑥这些症状缺陷不限于疼痛或性功能失调,可以排除是在躯体化障碍中发生的,也不可能归于其他精神障碍。

3.鉴别诊断

(1)癫痫
大发作:发作无精神诱因,发作前有先兆,发作时意识完全丧失,痉挛发作有一定规律,发作时间较短,常有大小便失禁,发作后完全遗忘,脑电图呈放电样。上述各点癔症多不具有,可以资鉴别。

(2)反应性精神病:反应性精神病者多不具有癔症的性格特征和易受暗示的特点,症状变化少,病程持续时间比癔症长,且反复发作者甚少。

(3)精神分裂症:癔症患儿有时可表现为情感、思维及行为紊乱现象,临床上应与精神分裂症鉴别。鉴别要点在于,癔症一般在强烈的精神因素作用下急骤起病,其症状与精神因素有密切联系,故其情感、思维及行为不像精神分裂症患儿所表现的那么荒谬离奇,使人难以理解。癔症病程呈反复发作的倾向,多数患儿可获良好的缓解。

\subsection{治疗}

儿童情绪障碍以综合性治疗为原则,根据发病有关因素和症状的特征,可采取心理治疗、环境或家庭治疗和药物治疗等。

\subsubsection{心理治疗}

儿童心理治疗的特点:

1.儿童不会因感到情绪问题而自行要求心理治疗,因此与患儿建立良好的关系非常重要。儿童的情绪问题会向父母倾诉或发泄,因此,一般由家长带来接受心理治疗。会谈开始时,要多观察,尤其是观察儿童与父母的关系,不要太快与儿童接近,让他们习惯了跟治疗者谈话之后,再慢慢亲密起来,或让其父母离开。最好不要穿白大褂,以免幼儿与打针的经历联系起来而害怕。要让患儿明白心理治疗可以帮助他们生活快乐,激发其治疗的愿望,同时也不能忽略与家长建立良好的治疗关系。

2.年幼儿童缺乏沟通能力,不善用言语表达自己的心情,因此要善于观察儿童的非言语表现,体会患儿的心境,并使用与儿童发展阶段接近的言语和交往方式。例如用木偶或娃娃来与儿童进行间接交流,借玩具的话影响患儿的情感或欲望。

3.儿童深受父母或保养员的影响,要充分利用他们对儿童的影响,通过养育方式的改变和与儿童关系的改善,而解决儿童的心理问题。家庭治疗往往比个别治疗收获显著。

4.儿童处于继续发展和转变的心理阶段,治疗者要熟悉儿童成长的曲线图,作为对儿童问题认识和治疗形式选择的参考依据。

5.儿童富于潜力,发展与成长很快,只要把阻碍儿童正常发展的障碍排除,儿童常能依靠其发展能力自行恢复、纠正、成长,不用复杂的理论与技巧就可获得效果。如对婴儿或幼儿给予良好的照管、爱护,就能使他们感到满足、安心,获得基本的安全感,其心身状况就自然恢复常态,无需太多考虑如何改善行动与情绪的问题。

心理治疗主要包括行为疗法,支持、认知疗法和暗示疗法等。进行心理治疗时需要根据儿童不同症状灵活选择应用。

1.行为疗法 如恐惧症、强迫症儿童可选择系统脱敏疗法、代币疗法、满灌疗法等。癔症患儿也可采取系统脱敏疗法,使原能诱发癔症的精神因素逐渐失去诱发的作用,减少发作或达到治愈。焦虑症、强迫症也可采用生物反馈疗法,有利于减轻焦虑和自主神经功能紊乱,改善睡眠,对治疗起到辅助作用。

2.支持、认知疗法 如焦虑症、抑郁症、强迫症患儿可采用支持、认知治疗,如倾听他们的诉说,给予他们适当的心理支持,消除他们的顾虑,帮助患儿接受和调整情绪感受;也要帮助患儿消除各种不利因素,帮助患儿提高适应环境的能力等。

3.暗示疗法 对癔症患儿可采用这种疗法。可通过言语暗示,如告诉患儿经过治疗会取得良好效果,与此同时配合针灸、穴位注射、电兴奋治疗等。

4.家庭治疗 主要是通过家庭会谈来了解家庭成员尤其是父母的个性心理特征、心理健康水平、教养方式;分析家庭中的行为互动模式、沟通模式、家庭关系及其可能对患儿产生的影响;帮助父母或其他养育者提高对患儿疾病的认识,了解产生疾病的因素;帮助父母认识其自身或家庭因素可能对患儿产生的不良影响,进而消除家庭环境或家庭教育中的不良因素,促进患儿恢复健康。

\subsubsection{药物治疗}

可根据主要症状选用药物。

1.针对焦虑症 苯二氮䓬
类药抗焦虑作用较好,常用地西泮(Diazepam,安定)每日1~2.5mg/kg、氯氮䓬
(Chlordiazepoxide,利眠宁)每日0.5mg/kg、阿普唑仑(Alprazolam)每日0.02~0.06mg/kg、氯硝西泮(Clonazepam)每日0.01~0.04mg/kg、劳拉西泮(Corazepam,罗拉)每日0.01~0.04mg/kg。常见不良反应为镇静作用和药物依赖。新一代抗焦虑药丁螺环酮(Baspirone)其抗焦虑作用与苯二氮䓬
类药相似,但不良反应小。成人抗焦虑每日量为20~30mg,儿童剂量尚待探索。另外,氟西汀(Fluoxetine)、文拉法辛(Venlafaxine,博乐欣)、氟伏沙明(Fluvoxamine,兰释)也可酌情用于儿童焦虑症的治疗。

2.针对恐惧症 可选用丙米嗪(Imipramine)12.5mg,每日2次;氯米帕明(Clomipramine,氯丙咪嗪)每次12.5mg,每日2次;多虑平(Doxepin)每次12.5mg,每日2次。另外,亦可用抗焦虑药物如地西泮、阿普唑仑等。剂量要根据病情适当调整。

3.针对强迫症 既往报道氯米帕明、丙米嗪、多虑平有一定疗效,其中以氯米帕明效果最好。氯米帕明常用剂量每日2~5mg/kg,最大剂量不超过每日250mg为宜。该药不良反应多,常见头晕、口干、肢体颤抖、嗜睡、疲劳感等。新一代抗抑郁药有氟西汀和氟伏沙明,据国外应用疗效优于氯米帕明。苯二氮䓬
类抗焦虑药亦可采用,对改善情绪、减轻焦虑有较好效果,但对强迫症状无多大效果。此外,可采用联合用药,当氯米帕明疗效不佳时,可加用碳酸锂(Lithium
Carbonate)或精神抑制药,也能起到一定疗效。新药可选用文拉法辛等治疗。

4.针对抑郁症 既往临床大多选用三环抗抑郁药,如丙米嗪、多虑平、氯米帕明等,剂量为每日2~5mg/kg,分2~3次口服。服药后3~4日见效。这些药不良反应大,常见有口干、便秘、视物模糊、低位性低血压、心电图改变,甚至引起癫痫
发作。因此,需密切观察,及时调整治疗。近年来氟西汀、文拉法辛、舍曲林(Sertraline)等新型抗抑郁药已相继应用于儿童抑郁症。这些药不良反应相对较轻,常见恶心、腹泻、失眠等。

5.针对癔症 可选用注射用水肌注,或10%葡萄糖酸钙静脉注射,暗示治疗。对于情感暴发,可给予地西泮5~10mg或奋乃静5mg肌内注射,使安静入睡。但须注意,儿童癔症不宜长期使用药物治疗,以免增强暗示作用,使其病情巩固。

\subsubsection{生物反馈疗法}

生物反馈疗法是建立在生物反馈学说理论基础上的一种治疗方法。它借助于专业的仪器或设备将人们通常情况下难以意识到的生理活动,如心电、脑电、肌电、皮电、皮温、呼吸、血压、胃电、胃肠压力等,记录、保存并转变为直观的、容易理解的视觉、听觉形式,受试者根据这些信号了解自身的生理变化,同时依据这些变化逐渐学会通过自己的行为改变,在一定程度上控制和纠正这些活动的过程。

例如:肌肉的紧张程度常常与人的整体生理警觉水平有关,因此精神紧张与肌肉紧张的消长是平行的。利用肌电生物反馈仪采集患儿伴随肌肉收缩或松弛产生的电活动,即肌电信号,并以各种图像的方式进行实时反馈,通过松弛肌肉的训练达到缓解精神紧张、焦虑的目的。

\section{儿童注意缺陷多动障碍}

注意缺陷多动障碍(attention deficit hyperactivity disorder,
ADHD)通常称为儿童多动症、注意缺陷障碍(attention deficit disorder,
ADD)。ICD-10中称为多动障碍(hyperactivity
disorder),是儿童期常见的行为问题,以注意力不集中、多动和冲动为主要核心症状,常导致明显学业与社交受损。

\subsection{流行病学}

ADHD遍布世界各国,患病率较高。以DSM-Ⅳ诊断系统来看,学龄期儿童患病率为5%~10%,成人为4%左右。国内报道患病率1.3%~13.6%,多数城市报道5%左右。男孩多见,学龄期儿童的男女发病率比约为9:1,青春期为(2~3):1。

\subsection{病因与发病机制}

ADHD是一种综合征,病因和发病机制尚不明了。大多认为是由于神经生物学基础与较强遗传因素,结合心理因素及社会问题等原因共同作用所致。发病可能涉及下列几种因素:

\subsubsection{遗传因素}

家系调查、双生子、寄养儿及分子遗传学等研究显示,ADHD具有较高的遗传性,遗传率0.6%~0.9%,患儿的一级亲属(父母和同胞兄弟姐妹)患ADHD风险显著增加。同卵双生子同病率(79%)明显高于异卵双生子(32%)。寄养儿的亲生父母ADHD患病率高于养父母。基因研究发现多巴胺传递基因(DAT、SLC6A3)和多巴胺受体基因(DRD4、DRD5)等可能与发病相关。但目前遗传方式不明。

\subsubsection{神经生理学因素}

研究资料显示,本症可能是由于大脑皮质的觉醒不足。如多动症儿童脑电图异常率高,慢波活动增加,顶枕区、右颞区α节律显著低于正常对照儿童。觉醒不足属于大脑皮质抑制功能障碍,从而诱发皮质下中枢活动释放,表现出多动行为。另外,患儿觉醒不足,对奖惩行为在一般心理水平下不起作用,所以多动儿童难以吸取经验教训,其行为问题难以矫正。

脑诱发电位(VEP)测试提示多动症儿童N100、P200、P300波幅较正常低,主动被动注意的ERP变异率小。VEP晚成分可以反映本症患儿注意及认知变化。对周围自主神经的研究也提示反应较对照组低。

\subsubsection{生物化学因素}

多巴胺(DA)、去甲肾上腺素(NE)及5-羟色胺(5-HT)等神经递质异常被认为与ADHD发生、发展密切相关,但确切机制仍未阐明。不少研究报道患儿的血或尿中NE和DA功能低下以及5-HT活动亢进。还有报道血小板内的单胺氧化酶及血浆中的多巴胺β-羟化酶在多动症伴品行障碍的儿童会降低。

一般情况下,单胺类的中枢递质去甲肾上腺素、5-羟色胺和多巴胺处于一种平衡状态,以保持适当的行为,当三者失去平衡时,即引起注意缺陷多动障碍的相关症状。Wender(1973年)提出,单胺类神经递质的代谢紊乱可能是活动过度的起源。

有学者认为,注意缺陷多动障碍儿童的中枢神经递质存在儿茶酚胺(CA)水平不足,以致脑的抑制功能不足,对进入的无关刺激起不到过滤作用。这样,患儿就对外来的各种刺激不加选择地做出反应,势必影响注意力的集中,并引起过多的活动。

\subsubsection{神经系统解剖及病理生理异常}

Yakoulev(1967年)提出ADHD是由于大脑额叶(前额叶)发育迟缓引起。Hynd等(1990年)应用共振成像检查发现ADHD儿童额叶异常。一切感觉刺激和运动功能都在前额叶进行分析综合和调节,前额叶功能异常,将引起活动过度、冲动攻击行为、情绪不稳、注意力不集中、挫折阈降低、做事有头无尾。

近年来,通过单光子发射电子计算机体层摄影(SPECT)发现,ADHD儿童的纹状体呈现灶性大脑灌输不足,而在感觉和感觉运动区则灌输过多(Lou等,1989年)。Zametkin等(1990年)应用于正电子发射计算机体层摄影(PET)研究,发现女性患者葡萄糖代谢在大脑皮质启动区和上部前额区与同龄儿童相比普遍低下。这些脑区与控制动作活动和注意有关。

功能性磁共振成像(fMRI)以非侵入性、无放射性、可重复扫描等诸多优点较适用于儿童,fMRI研究结果发现ADHD主要是扣带回背侧前部皮质(DACC)的功能失调,DACC主要被认为与很多复杂或需要努力的认知过程有关,且与反应的选择和抑制、错误的侦测及监测成就动机等有关。亦有研究结果显示,患儿在完成反应/不反应任务时出现纹状体异常(Rubia,
1999年;Durston等,2003)。

\subsubsection{心理社会因素}

尽管ADHD有较高的家族聚集倾向和遗传率,但同卵双生子罹患本症的表现差异和寄养儿的研究均提示环境因素存在的影响。父母性格不良或有精神障碍、养育方式差异、家庭气氛紧张、家庭结构不稳定、母孕期或围产期异常、经济困难、社会风气不良、环境污染等诸多心理社会因素的持续存在,对于诱发或加重本症有重要作用。

\subsection{临床表现}

注意缺陷多动障碍的核心症状为注意缺陷、多动和冲动。最初的症状始于7岁前,多在3岁左右发病。目前认为只有少部分患儿的症状不会影响生活功能,大部分患儿(50%~60%)的症状仍会持续到成人期。长期看来,核心症状会随年龄增长而变化:多动症状多会日益减轻,而注意缺陷和冲动则会长期存在,并随着课业负担加重和情绪波动加剧而凸现出来。

1.注意缺陷(attention
deficit) 患儿上课不能专心听讲,玩文具、画画、想动画片;易受外界的细微干扰而分心,如小鸟飞过、人走过都要看看;做作业粗心、漏题,做游戏不专心,一会玩这样一会玩那样,且玩具乱放;做事不能坚持,常半途而废等。

2.多动或活动过多(hyperactivity) 患儿在各种场合都会动不停,如在教室内喧哗吵闹、来回奔跑,在座位上来回扭动或离位,招惹别人,多嘴多舌,过度喧闹,喜欢玩危险游戏,不爱护玩具,丢三落四,常和同学争吵打架,难于安静。

3.冲动(impulsiveness) 任性冲动,情绪不稳。多动症儿童的情绪波动大,易激惹,易过度兴奋,易受外界影响、易受挫折,脾气暴躁,常对一些小事做出过分反应,常大哭吵闹,在冲动下做出一些危险举动及破坏伤人行为。但易冲动并非多动症的特异性症状。

4.学习困难 患儿智力水平大都正常或接近正常。但由于注意缺陷、活动过度等症状,不善于思考,导致学习困难。部分患儿可有认知功能障碍,如临摹图画时不能分清主体与背景的关系,不能将图形的各个部分很协调地组合在一起。有的患儿存在阅读、拼音及书写困难。有的患儿存有空间定位障碍,如分不清楚左右、将文字倒读、反写字等。有的患儿将“6”读成“9”,将“d”读成“b”。有的患儿还有拼音困难、口吃、言语表达能力差等。

5.神经系统体征 临床上约半数患儿具有轻微的神经系统异常体征。

(1)快速轮替动作不协调,如快速翻手动作笨拙等。

(2)共济运动失调,如指鼻或指指试验阳性、不能走直线等。

(3)连带运动时,对侧肢体出现类似动作。

(4)反射轻度亢进或不对称。

(5)肌张力轻度增高或不对称。

(6)动作笨拙,不能做精细动作。

(7)皮肤两点辨别觉差。

(8)语音缺陷、吐字不清、口吃等。

(9)眼球震颤或斜视等。

这些体征仅作为诊断参考,无特异性诊断意义。

\subsection{诊断}

\subsubsection{实验室检查}

1.心理测评

(1)智力测验:常用中国修订的韦氏学龄前儿童智力量表(WIPPS-CR)和韦氏学龄儿童智力量表(WISC-CR)。ADHD儿童大都无智力缺陷,或只有轻度低下。

(2)学习成就和语言功能测验:国外常使用广泛成就测验(WRAT),通过该测验发现ADHD儿童常有学习成绩低下(学校学习成就可作主要参考)。Peabody图画词汇测验及诵读测验,ADHD儿童成绩较低,但无特异性。

(3)注意测验:常用持续性操作测验(CPT)。注意缺陷多动障碍、智力低下、情绪和行为障碍儿童均可出现注意持续短暂,易分散,但无特异性。整合视听连续执行测试(IVACPT),主要是通过测定儿童的控制力商数和注意力商数等来反映儿童是否为ADHD。

(4)量表评定:常用儿童多动症简明症状问卷,问卷分大于15者,被认为有多动症可能。Achenbach儿童行为量表,可作为ADHD儿童伴随症状评定用。

2.脑电图检查 据国内外报道,有45%~90%的多动患儿有脑电图的异常改变,大多数呈轻度或中度异常。用脑超慢涨落分析仪检查,患儿脑电图主要显示慢波活动增多,α波活动减少,表现慢化,且α优势频率不稳定,脑有序度低。研究证明,脑有序度越低,脑自组织能力越差,行为、学习、多动、冲动等症状越严重。脑电图异常对诊断有参考意义,但无定性诊断价值。

\subsubsection{诊断标准}

目前多动症由于病因未明,尚缺乏有效实验室检查依据,因此,诊断主要依靠家长、教师提供的病史、体格检查、精神检查、量表评定、注意测定等进行综合判断。神经系统“软性”体征和脑电图异常有助于诊断,但阴性者也不能否定其诊断。

根据CCMD-3关于多动症的诊断标准如下:

1.注意障碍(注意力集中困难)至少有下列4项:

(1)学习时容易分心,听见任何外界动静都要去探望。

(2)上课很不专心听讲,常东张西望或发呆。

(3)做作业拖拉,边做边玩,作业又脏又乱,常少做或做错。

(4)不注意细节,在做作业时或其他活动中常常出现粗心大意的错误。

(5)丢失或特别不爱惜东西(如常把衣服、书本等弄得很脏很乱)。

(6)难以始终遵守指令完成家庭作业或家务劳动等。

(7)做事难以持久,常常一件事没做完,又去干别的事。

(8)与他人说话时常常心不在焉,似听非听。

(9)在日常活动中常常丢三落四。

2.多动 至少有下列4项:

(1)需要静坐的场合难以静坐或在座位上扭来扭去。

(2)上课时常有小动作或玩东西或,与同学讲悄悄话。

(3)话多,好插嘴,别人问话未完就抢着回答。

(4)十分喧闹,不能安静地玩耍。

(5)难以遵守集体活动的秩序和纪律,如游戏时抢着上场,不能等待。

(6)干扰他人的活动。

(7)好与小朋友打斗,易与同学发生纠纷,不受同伴欢迎。

(8)容易兴奋和冲动,有一些过火的行为。

(9)在不适当的场合奔跑,登高爬梯,好冒险,易出事故。

3.严重标准 对社会功能(如学习成绩、人际关系等)产生不良影响。

4.病程标准 起病于7岁前(多在3岁左右),符合症状标准和严重程度标准至少已6个月。

5.排除标准 排除精神发育迟滞、广泛发育障碍、情绪障碍。

\subsection{鉴别诊断}

1.儿童精神分裂症 在精神分裂症早期往往有注意涣散、学习困难、情绪不稳、兴奋不安和行为改变等,应和多动症鉴别。但精神分裂症患儿具有特殊的思维障碍、情感平淡、幻觉、妄想等特征性症状,故不难与多动症相鉴别。

2.精神发育迟滞 精神发育迟滞者存在语言、运动等方面的发育迟滞,学习成绩差与其智力水平相符合。而多动症患儿的学习成绩明显低于其智力所能达到的水平。

3.情绪障碍 情绪障碍的患儿虽可有明显的行为紊乱、多动、好攻击、易激惹、注意力不集中等症状,但是以情绪障碍如焦虑、抑郁、烦躁等为主导症状。呈间断性病程。而多动症患儿的病程为慢性持续性病程。

4.品行障碍 由于多动症患儿常常不服从管理,无组织纪律性,常出现不良行为,故需要和品行障碍鉴别。品行障碍者以反复而持久的反社会性、攻击性为特点,常频繁出现斗殴、外逃、偷窃、严重说谎、纵火等明显违反社会规范和道德准则的破坏或犯罪行为。这类儿童无注意障碍,用兴奋药治疗无效。多动症和品行障碍两组症状常同时存在,同病率高达30%~58%(Schachar,
1991年)。这类儿童应进行积极治疗,因为具有两组症状重叠的儿童预后不良。

5.抽动障碍 常伴有多动症,但主要表现为不自主、间歇性、多次重复的抽动,包括发音器官的抽动,症状奇特,不难鉴别。

6.正常儿童的活泼好动 一般发生在3~6岁,以男孩为多,也表现为好运动和注意力集中时间短暂。但这些小儿的多动常与外界无关刺激过多、疲劳、学习目的不明确、注意缺乏训练、生活习惯不好有关。

\subsection{治疗}

ADHD是由生物、心理、社会诸因素引起,因此需针对这三方面进行综合治疗。

\subsubsection{药物治疗}

1.中枢神经兴奋药 盐酸哌甲酯为ADHD的一线治疗药物。作用机制是增加突触前神经末梢结节释放多巴胺和去甲肾上腺素,同时也会抑制其再吸收,从而提高突触间隙多巴胺、去甲肾上腺素浓度。主要改善患儿的注意力,明显减少多动、冲动、违拗及与同伴关系不良等行为问题(Barkley,
1989年),使家庭关系有所改善(Schacher等,1987年),其学习成绩也会随之提高。服药后,70%~80%患儿的症状可以明显改善。由于本品可能出现失眠、眩晕、头晕、头痛、心悸、恶心、厌食、诱发抽动症等不良反应,故此类药物必须在专科医师指导下服用,不可自行滥用。6岁以下者原则上不用药。精神分裂症、甲状腺功能亢进、心律不齐、心绞痛、青光眼和对本药过敏者禁用。

目前国内常用剂型有立即释放型利他林(Ritalin)和控制释放型专注达(Cencerta)。利他林起效快,2小时可达最高血药浓度,药效维持3~4小时,常用量0.1~0.6mg/kg,从低剂量开始服药,一般每天服药2次,为减少胃肠道反应宜早餐和中餐后服用。最大剂量不超过每日30mg为宜。专注达疗效可持续12小时,只需每天早餐后服药一次。

2.非兴奋剂类药物

(1)盐酸托莫西汀(atomoxetine):为一种选择性去甲肾上腺素再摄取抑制剂,也可作为ADHD一线治疗药,可明显改善核心症状。优点在于每日1剂可维持稳定的血药浓度,包括晚上的症状均可持久改善,对于合并抽动或情绪障碍者疗效优于兴奋剂。初始治疗:①体重不足70千克的儿童和青少年用量,开始每日总剂量约0.5mg/kg,服用3天最低剂量后增加药量,至每日总目标剂量,约1.2mg/kg,可单次服用或早晚平均分为两次服用。对于儿童或青少年,每日最大剂量不应超过1.4mg/kg或100mg,选其中较小的一个剂量;②体重超过70千克的儿童、青少年或成人用量,开始每日总剂量40mg,服药3天后加量至目标剂量80mg。继续使用2~4周,若未达最佳疗效,每日剂量可增至最大推荐总剂量100mg。常见的不良反应为:消化不良、恶心、呕吐、疲劳、食欲不振、眩晕和情绪波动。治疗期间需定期复查肝功能。

(2)三环类抗抑郁药:常用丙咪嗪,对伴有焦虑和抑郁的ADHD比较适宜。剂量从早晚各12.5mg开始,根据疗效逐渐加量,每日总量不大于50mg。该类药最令人忧心的副作用是心血管变化,如心律失常,有猝死个案报道。其他不良反应有轻度激动、嗜睡、口干、头晕、便秘、震颤和肌肉抽动等。

(3)α受体拮抗药:一般选用可乐定(Clonidine),对抽动障碍和ADHD均有效,尤其适用于两者都存在的患儿。开始每日服0.05mg,以后缓慢加量至每日0.15~0.3mg,分3次服。需定时监测血压。

(4)其他药物:文拉法辛能有效地抑制5-羟色胺和去甲肾上腺素的再摄取,对多巴胺的再摄取也有一定的抑制作用,其适应证为抑郁症,也被发现可以改善注意力。

\subsubsection{脑电生物反馈治疗}

脑电研究发现12~15Hz的SMR波可抑制运动性活动,而4~8Hz的θ波与白日梦和困倦有关。而多动症患儿多有SMR波减少,θ波增多现象。脑电生物反馈治疗通过脑电生物反馈仪采集患儿脑电波,并以各种图像的方式进行实时反馈,以达到抑制θ波,强化SMR波的目的。脑电生物反馈法能改善多动症的核心症状,疗效持久,但起效慢,有人认为与药物配合治疗效果更好。

\subsubsection{心理治疗}

用行为矫正疗法、认知训练等方法结合药物对患儿进行治疗。如利用行为学习原理,通过奖惩逐步增强患儿的适应性行为;也可通过认知训练提高患儿自我控制、自我调节的能力,改善其冲动性,养成“三思而后行”的习惯。但对单纯由生物学因素引起的多动症患儿,如无明显社会心理因素者,采用心理治疗对患儿无明显效果。

\subsubsection{家庭治疗或教育咨询}

家庭治疗或教育咨询可帮助父母和教师正确认识注意缺陷多动障碍,帮助他们接纳孩子的特点,改变将患儿当作“坏孩子”的看法,适当降低对患儿的一些行为标准,并改变教育和教养方式,能够针对患儿特点,采用一些特殊教育方法。如:①明了患儿的疾病性质,寻觅及去除可能的致病诱因,不歧视,不粗暴对待;②疏散过多精力,如进行户外活动、打球、跑步等;③订立简单可行的规矩,如吃饭时不看图书,做作业时不玩玩具等;④对于打架、伤人以及毁物等不良行为,应像对待正常儿童一样严加制止,不可袒护;⑤有条件的学校对这类患儿可小班上课,加强个别辅导,对于良好行为,如静坐听课、注意力集中、作业不粗心等要及时表扬鼓励,以利于巩固。

\section{品 行 障 碍}

品行障碍(conduct
disorder)是指在儿童少年期反复、持续出现的攻击性和反社会性行为。这些行为违反了与年龄相适应的社会行为规范和道德准则,影响儿童少年本身的学习和社交功能,损害他人或公共的利益。儿童品行障碍是一个相对独立的诊断类别,包括偷窃、逃学、离家出走、说谎、纵火、虐待动物、性虐待、躯体虐待、违抗与不服从、破坏性行为等一系列异常行为。

\subsection{流行病学}

由于研究方法及诊断标准的差异,品行障碍的发生率各家报道不一。国外报道的患病率为3.0%~16%,国内报道男孩的患病率为2.48%,女孩为0.28%。品行障碍发病年龄最早可到5岁,但通常起病于儿童晚期或少年早期,男孩高于女孩,男女患病率之比(3~12):1。

\subsection{病因与发病机制}

\subsubsection{生物学因素}

1.遗传因素 研究资料显示,反社会性行为在同卵双生子之间的同病率明显高于异卵双生子,亲生父母有反社会性或犯罪行为的儿童犯罪危险性明显偏高。气质在很大程度上由遗传决定。困难气质儿童的行为问题发生率明显增高,因为困难气质儿童往往与其父母及环境呈现负性的相互作用方式,这使得原本存在的一些异常行为变得更加严重。

2.神经递质 近年来的研究揭示中枢神经5-HT功能降低与冲动攻击性行为有关。在动物实验和人类实验中为发现,攻击行为出现时,脑脊液中5-羟吲哚乙酸降低。

3.神经解剖、神经递质和激素相互作用 研究发现边缘系统不正常的电活动,如精神运动性发作与攻击性行为有关(Mark和Ervin,
1970年)。动物研究表明,一定的激素直接影响边缘系统,如睾丸酮及雌二醇集中在边缘系统特定的核中,雄性大鼠丘脑下部前部的损伤可减少攻击性行为和性行为,如将睾丸酮直接植入已去势的大鼠丘脑下部前部,这些因去势被抑制的性行为及攻击性行为可再恢复(Herbert,
1989年)。这些研究提示了品行障碍可能的神经解剖和激素水平的异常因素。

4.生理因素 Mednick等通过对照研究表明,反社会行为的青少年具有先天自主神经的低反应性,即受刺激后心率缓慢,反应水平低,从刺激效应恢复的速度快。这些生理缺陷可能妨碍儿童学会通过回避来避免受到惩罚的能力。

5.其他生物因素 主要是儿童早期遭受的各种有害生物因素,如母孕期情绪不佳以及患各种躯体疾病、早产、异常分娩等都可能会影响品行障碍的发生。因为这些因素可能妨碍胎儿脑的正常发育,或者提高对外界刺激的易感性,或者引起大脑皮质功能失调,都可成为品行障碍的生物学因素。

\subsubsection{家庭社会因素}

1.不良的家庭环境及教养方式 父母婚姻不和或离异,可造成小儿精神创伤,影响小儿心理社会发育。父母有犯罪史、酗酒、反社会行为等使得儿童从小在不良环境中成长,易引发品行问题。父母管教方法不当,如简单粗暴、放任不管、管束过严或忽视、教育态度不一致等都容易造成儿童出现品行问题。

2.社会文化因素 不同社会、民族或地域的文化传统或思想观念也影响少年违法犯罪率,各种文化媒体中过多出现暴力行为也可能会增加儿童和少年的违法行为。

\subsubsection{心理因素}

品行障碍儿童往往具有情绪不稳、冲动、自我中心、好攻击等心理特点。儿童早期对父母的依恋、关系不良、受同伴排斥、学业失败等会导致儿童缺乏安全感、自尊感,容易自卑且可能转向以不正当的途径获得情感的满足。

由此可见,品行障碍发病病因复杂,是生物、社会和心理三方面因素综合作用的结果。

\subsection{临床表现}

品行障碍主要表现为攻击性和反社会性行为,有以下几种形式:

1.一般攻击性和破坏行为 攻击行为主要表现为躯体攻击或言语攻击。幼儿表现为暴怒发作、吵闹,以后渐渐变为违抗成人命令,与人争吵,言语伤人,打架斗殴,恃强欺弱等行为。一些患儿甚至威胁或恐吓其他弱小儿童,索要他人钱物,或强迫他人为自己做事。这些儿童也常有残酷虐待动物的行为。破坏性行为主要表现为故意破坏家中或别人的东西,或破坏景物。

2.说谎 儿童最开始时说假话可能是为了获得奖励或逃避惩罚,后因能从说谎中获得益处,因此常用说谎来达到自己目的和愿望,渐渐变为经常有意说谎,甚至发展为说谎成性,进而成为行为模式,构成品行障碍。

3.偷窃 往往开始于学龄期,表现为未经同意拿走父母的钱,或把家里的东西拿到外面去,进而发展为占别人东西为己有,有意偷别人的东西。多数为个别性偷窃,常伴有说谎。少年期以后主要是表现为外出行窃,单独或团伙行窃。有的儿童通过行窃寻求刺激,或以偷窃为乐,把偷来的东西当作战利品保存起来。

4.逃学或离家出走 首先可能是做了错事或学习成绩不好怕惩罚不敢回家;或对学习无兴趣而旷课逃学;或因在家自尊心受到损伤,得不到父母关心,家庭气氛不良,父母经常争吵而出走;也有因迷恋网吧上网而不回家。常伴有说谎和偷窃等不良行为。

5.纵火 表现为单独或集体烧毁别人或公共的财物。纵火可能是为了报复或寻求刺激等,常伴有其他反社会性行为。

6.吸毒行为 多发生于青少年时期,表现为反复使用成瘾性物质。初次使用多出于好奇或受人利用,一旦成瘾后就长期反复使用,并不择手段地获取毒品,甚至发展为参与贩毒。常常伴有其他反社会性行为,形成少年违法。

7.性攻击 多发生于青春期以后的男性,表现为强奸、猥亵女性、集体淫乱性性行为。女性可因与异性发生性行为获得物质满足而出现卖淫和淫乱行为,构成社会违法行为。

部分患儿还伴有注意力不集中、活动过多等多动症表现,一些儿童还伴有抑郁、愤怒等情绪异常。

\subsection{诊断}

根据CCMD-3的诊断如下:

81 品行障碍[F91]

品行障碍的特征是反复而持久的反社会性、攻击性或对立性品行。当发展到极端时,这种行为可严重违反相应年龄的社会规范,较之儿童普通的调皮捣蛋或少年的逆反行为更严重。

如过分好斗或霸道;残忍地对待动物或他人;严重破坏财物;纵火;偷窃;反复说谎;逃学或离家出走;过分频繁地大发雷霆;对抗性挑衅行为;长期的严重违拗。明确存在上述任何一项表现,均可作出诊断,但单纯的反社会性或犯罪行为本身不能作为诊断依据,因为本诊断所指的是某种持久的行为模式。

81.1 反社会性品行障碍[F91.0局限于家庭内的品行障碍;F91.1反社会规范的品行障碍;F91.2对社会规范的局限性品行障碍]

【\textbf{症状标准} 】

(1)至少有下列3项:

①经常说谎(不是为了逃避惩罚);

②经常暴怒,好发脾气;

③常怨恨他人,怀恨在心,或心存报复;

④常拒绝或不理睬成人的要求或规定,长期严重的不服从;

⑤常因自己的过失或不当行为而责怪他人;

⑥常与成人争吵,常与父母或老师对抗;

⑦经常故意干扰别人。

(2)至少有下列2项:

①在小学时期即经常逃学(1学期达3次以上);

②擅自离家出走或逃跑至少2次(不包括为避免责打或性虐待而出走);

③不顾父母的禁令,常在外过夜(开始于13岁前);

④参与社会上的不良团伙,一起干坏事;

⑤故意损坏他人财产,或公共财物;

⑥常常虐待动物;

⑦常挑起或参与斗殴(不包括兄弟姐妹打架);

⑧反复欺负他人(包括采用打骂、折磨、骚扰及长期威胁等手段)。

(3)至少有下列1项:

①多次在家中或在外面偷窃贵重物品或大量钱财;

②勒索或抢劫他人钱财,或入室抢劫;

③强迫与他人发生性关系,或有猥亵行为;

④对他人进行躯体虐待(如捆绑、刀割、针刺、烧烫等);

⑤持凶器(如刀、棍棒、砖、碎瓶子等)故意伤害他人;

⑥故意纵火。

(4)必须同时符合以上第(1)、(2)、(3)项标准。

【\textbf{严重标准}
】 日常生活和社会功能(如社交、学习或职业功能)明显受损。

【\textbf{病程标准} 】 符合症状标准和严重标准至少已6个月。

【\textbf{排除标准}
】 排除反社会性人格障碍、躁狂发作、抑郁发作、广泛发育障碍或注意缺陷与多动障碍等。

81.2 对立违抗性障碍[F91.3]

多见于10岁以下儿童,主要为明显不服从、违抗,或挑衅行为,但没有更严重的违法或冒犯他人权利的社会性紊乱或攻击行为。必须符合品行障碍的描述性定义,即品行已超过一般儿童的行为变异范围,只有严重的调皮捣蛋或淘气不能诊断本症。有人认为这是一种较轻的反社会性品行障碍,而不是性质不同的另一类型。采用本诊断(特别对年长儿童)需特别慎重。

【\textbf{症状标准} 】

(1)至少有下列3项:

①经常说谎(不是为了逃避惩罚);

②经常暴怒,好发脾气;

③常怨恨他人,怀恨在心,或心存报复;

④常拒绝或不理睬成人的要求或规定,长期严重的不服从;

⑤常因自己的过失或不当行为而责怪他人;

⑥常与成人争吵,常与父母或老师对抗;

⑦经常故意干扰别人。

(2)肯定没有下列任何1项:

①多次在家中或在外面偷窃贵重物品或大量钱财;

②勒索或抢劫他人钱财,或入室抢劫;

③强迫与他人发生性关系,或有猥亵行为;

④对他人进行躯体虐待(如捆绑,刀割,针刺,烧烫等);

⑤持凶器(如刀、棍棒、砖、碎瓶子等)故意伤害他人;

⑥故意纵火。

【\textbf{严重标准} 】 上述症状已形成适应不良,并与发育水平明显不一致。

【\textbf{病程标准} 】 符合症状标准和严重标准至少已6个月。

【\textbf{排除标准}
】 排除反社会性品行障碍、反社会性人格障碍、躁狂发作、抑郁发作、广泛发育障碍,或注意缺陷与多动障碍等。

81.9 其他或待分类的品行障碍[F91.8;F91.9]

82 品行与情绪混合障碍[F92]

品行与情绪混合障碍是持久的攻击性、社会紊乱性或违抗行为与明显的焦虑,抑郁或其他情绪问题共同存在。

\subsection{鉴别诊断}

品行障碍诊断时需排除以下疾病:

1.注意缺陷多动障碍 主要表现为多动、注意力集中困难、冲动等,经过兴奋剂治疗后行为症状可以得到明显控制。一些注意缺陷多动障碍儿童的父母由于教育不当可能易使儿童发展成品行障碍,诊断时需要给予双重诊断。

2.情绪障碍 情绪障碍的病程为发作性的,其行为异常与情绪异常密切相关,经过抗焦虑或抗抑郁治疗后行为异常会逐渐恢复。

3.抽动-移语综合征 该综合征的患儿具有强迫性或冲动性骂人、秽语,也可以伴有攻击性行为;但主要表现为多发性的运动和发声抽动,使用氟哌啶醇等药物治疗,行为异常可以随着抽动症状的控制而消失。

4.儿童少年精神分裂症 精神分裂症患儿多伴有思维障碍、感知觉异常和言语异常等分裂症基本表现,用抗精神病药物治疗后行为异常可以改善。

5.癫痫
 发作时有意识障碍,既往有癫痫
发作史,可能有智力障碍以及脑电图上有癫痫
性放电等特征。

6.脑器质性精神障碍 可以根据有脑损害的病史和神经系统的阳性体征与品行障碍鉴别。

7.精神发育迟滞 根据智力低下和社会适应能力差的特点容易与品行障碍鉴别。

\subsection{治疗}

目前品行障碍的治疗多采用心理治疗、药物辅助治疗及教育咨询等方法。

\subsubsection{行为治疗}

利用操作性条件反射原理,改变儿童的行为模式,逐渐减少不良行为,包括阳性强化疗法和惩罚疗法等。当孩子出现亲社会行为时,及时给予奖励,及时发现并表扬其优点和进步的地方,帮助其树立自尊心和建立良好的行为模式。

\subsubsection{问题解决技巧训练}

其原理是认为品行障碍患儿存在认知能力缺陷。该训练包括四大步骤:①帮助患儿理解问题,将问题在头脑中以恰当的形式再现出来;②制订出获得结果的计划;③实施计划;④检验结果。这种训练在降低反社会性行为和增强亲社会行为方面的作用效果较好。

\subsubsection{家庭治疗}

家庭治疗是通过改变家庭的功能结构、互动模式等,继而改变患儿的行为。主要有:

1.父母管理训练 训练通过改变父母和儿童之间不良的相互方式,进而改变儿童不良的行为。包括:训练父母以适当的方法与儿童进行交流,采用阳性强化措施奖赏儿童的亲社会性行为,必要时采用一些轻微的惩罚消退不良行为等。

2.家庭功能治疗 主要从家庭功能的整体上来分析存在的问题,通过增加家庭成员之间的直接交流和相互支持,完善家庭的功能,帮助家庭找到解决问题的新方法,提高家庭的应对能力等,以达到改变患儿不良行为的目的。

\subsubsection{社区治疗}

主要是利用各社区内的优势进行干预,例如雇佣一些大学生或成人志愿者作为他们的伙伴,与他们建立朋友关系,树立行为榜样,引导他们改正不良行为。另外,可以实施一些学校干预计划,如社会技能训练计划和学习技能训练计划,以改善伙伴关系,提高学习成绩,增加患儿的自尊心,改善患儿的不良行为。

\subsubsection{药物辅助治疗}

药物治疗主要是用于治疗其他伴随症状。如用哌甲酯等中枢兴奋药治疗伴随的多动表现,用碳酸锂治疗情感症状,用抗抑郁药治疗抑郁症状。某些药物对抑制攻击性行为有一定的效果,如氟哌啶醇、碳酸锂和普萘洛尔等药物对控制部分患儿的攻击性行为和暴怒发作有效,可以作为严重攻击性行为的辅助治疗。

\section{抽 动 障 碍}

抽动障碍(tic
disorder)主要表现为一个部位或多部位肌肉不自主的、反复的、快速的运动抽动和发生抽动,并可伴有注意力不集中、多动、强迫性动作和思维或其他行为症状。本病多发生于儿童时期,少数可持续至成年,根据发病年龄、临床表现、病程长短和是否伴有发声抽动而分为抽动症、慢性运动或发声抽动障碍、Tourette综合征。

\subsection{短暂性抽动障碍}

短暂性抽动障碍(transient tic
disorder)又称抽动症(tics),是儿童时期最常见的一种抽动障碍类型。临床表现大多数为单纯的运动抽动,极少数病例为单纯发声抽动。运动抽动的部位一般多见于眼肌、面肌和颈部肌群;发声抽动如清嗓声、咳声、嘶嘶声等。通常病程持续数月至一年。

\subsubsection{流行病学调查}

国外报道,有10%~24%的儿童在其童年的某个时期会出现短暂的抽动(Shapiro,
1981年),国内报道为1%~7%。男孩较多见。

\subsubsection{病因与发病机制}

本病因尚未明确,致病因素较多,主要有以下几方面:

1.遗传因素 短暂性抽动障碍可有家族聚集性,患儿家族成员中患抽动障碍者较多见,故认为可能与遗传因素有关。

2.器质性因 素围生期损害,如产伤、窒息等因素可能与本病有关。

3.躯体因素 起始往往由于局部刺激而诱发,如眼结膜炎或倒睫刺激引起眨眼或因上呼吸道感染而出现吸鼻、面肌抽动。当局部疾病原因去除后,抽动症状仍继续存在,这可能与大脑皮质形成的惰性兴奋灶有关。

4.社会心理因素 家庭不和、亲人死亡、学习负担过重等各种负性生活事件可引起儿童紧张焦虑,导致抽动症状,或使已有的抽动症状加重。抽动成为心理应激的一种表现。

5.药源性因素 某些药物如中枢神经兴奋药、抗精神病药等,长期服用可能产生抽动的不良反应。

\subsubsection{临床表现}

本病首发症状大多数为简单性运动抽动,较局限。一般以眼、面肌抽动为多见,在数周或数月内症状波动或部位转移,可向颈部或上下肢发展。常见表现为眨眼、挤眉、翻眼、皱额、咬唇、露齿、张口、点头、摇头、伸脖、耸肩等动作。少数可出现简单发声抽动,如单纯反复咳嗽、哼气或清嗓等。抽动症状频率和症状严重程度不一,通常对患儿日常学习和适应环境无明显影响。大多数不伴有其他行为症状和强迫性障碍等。躯体检查包括神经系统检查,通常无异常发现。病程持续时间一般不超过1年。

\subsubsection{诊断}

根据CCMD-3的诊断标准如下:

1.有单个或多个运动抽动或发声抽动,常表现为眨眼、扮鬼脸或头部抽动等简单抽动。

2.抽动天天发生,1天多次,至少已持续2周,但不超过12个月。某些患儿的抽动只有单次发作,另一些可在数月内交替发作。

3.18岁前起病,4~7岁儿童最常见。

4.不是由于Tourette综合征、小舞蹈病、药物或神经系统其他疾病所致。

\subsection{慢性运动或发声抽动障碍}

慢性运动或发声抽动障碍是指临床表现符合抽动障碍的一般特征,可表现为简单的或复杂的运动抽动;或仅仅出现发声抽动,运动抽动和发声抽动不同时存在,而且症状相对不变,可以持续数年甚至终生。

\subsubsection{流行病学}

本类型可于儿童期和成年期起病。其患病率尚无确切的流行病学调查报告。有报道估计1%~2%成年人患有慢性运动抽动障碍或发声障碍。

\subsubsection{病因}

有关慢性运动或发声抽动障碍单独的病因研究未见报告,多数认为是由短暂性抽动障碍发展而来。

\subsubsection{临床表现}

此型抽动症状表现类似暂时性抽动障碍。不同之处在于慢性抽动障碍多累及面肌、颈肌和肩部肌群,很少有上下肢和躯干抽动,患者常表现一侧面部歪扭或眨眼。病程持续多年不等,至少超过一年。慢性发声抽动较少见,常见为清嗓声或轻微的吸吮声,胸、腹、膈肌收缩形成的声音。

\subsubsection{诊断}

根据CCMD-3的诊断标准如下:

1.不自主运动抽动或发声,可以不同时存在,常1天发生多次,可每天或间断出现。

2.在1年中没有持续2个月以上的缓解期。

3.18岁前起病,至少已持续1年。

4.不是由于Tourette综合征、小舞蹈病、药物或神经系统其他疾病所致。

\subsection{Tourette综合征}

Tourette综合征以多发性运动性抽动伴有不自主发声为主要特征,属于慢性神经精神障碍一类疾病。本综合征又称发声与多种运动联合抽动障碍,多种抽动症,多发性抽动症,冲动性抽动症等。该症最早由Itard于1825年首先描述。法国医师Tourette于1885年报道9例并做了详细叙述,故以Tourette's综合征命名。近年来,国外文献多称为Tourette's
syndrome(以下简称为TS)。

\subsubsection{流行病学}

TS流行病学调查资料较少,患病率为0.1%~0.5%。大多数起病于4~12岁,以7~8岁起病者占多数。男性比女性为多,男女比为(3~9):1。20世纪70年代以后,国内报道的TS病例数明显增多。

\subsubsection{病因与发病机制}

本症病因至今未明,近年来的研究报道提示TS可能是由于遗传因素、神经生理、生化代谢以及环境因素在发育过程相互作用的结果。

1.遗传因素 家族研究及双生儿调查研究结果发现,TS患儿家族成员中患抽动症和TS的较为多见,其发生率为10%~66%。TS双生儿同病一致性较高,单卵一致性为75%~95%;双卵一致性为8%~23%。TS遗传方式及机制未明。

2.神经生化因素 有学者研究认为TS是由于纹状体多巴胺活动过度或突触后多巴胺受体超敏感所致。氟哌啶醇、哌迷清和泰必利治疗本症有效,机制是选择性阻滞中枢多巴胺D\textsubscript{2}
受体,因此支持本症与脑内多巴胺功能亢进相关。小剂量盐酸可乐定具有刺激突触前α\textsubscript{2}
受体作用,从而反馈抑制中枢蓝斑区去甲肾上腺素的合成释放,使抽动症状减轻,故认为本症的病理机制与去甲肾上腺素能系统受累有关。而5-HT再摄取抑制药氯米帕明治疗本病诱发的强迫症有效,提示本症与5-HT功能失调有关。γ-氨基丁酸(GABA)具有抑制中枢神经递质的功能。有人认为本症是由于脑内GABA抑制功能降低,从而引起皮质谷氨酸能兴奋性增加,导致不适当行为发生。近年来内啡肽研究表明,中枢神经系统多巴胺、5-HT以及GABA等多种神经递质的失调可能继发于内源性阿片系统功能障碍,故认为内啡肽对Tourette综合征病理机制有重要影响。但神经生化方面的研究还有待深入。

3.器质性因素 TS患儿脑电图异常发生率较高,为50%~60%,主要为慢波或棘波增加,但无特异性改变。少数患儿CT异常。神经系统软体征较多见。有人认为,TS患儿行为运动的改变可能与杏仁核纹状体通路障碍有关,不自主发声抽动可能与扣带回基底核及脑干不规律放电有关。近年来的研究认为,基底核和边缘系统的特殊部位发育异常可能是TS的原因。这些均提示本症为器质性疾病。

4.社会心理因素 TS起因可能与应激因素有关,如由于精神创伤、生活事件或因日常精神过度紧张的影响。也有人认为母孕期应激事件、妊娠初期3个月反应严重是以后发生抽动障碍的危险因素。

5.药源性因素 长期不恰当或大剂量地应用抗精神病药物或中枢兴奋药,也可能引发该病的发生。

\subsubsection{临床表现}

1.运动

(1)简单性运动抽动:表现为迅速、突然、反复、无意义的运动抽动,如眨眼、眼球转动、咂嘴、翘鼻、伸舌、转头、点头、伸脖、张口、耸肩、挺腹、吸气等。

(2)复杂性运动抽动:发作缓慢,可表现为似有什么目的。复杂性抽动奇特多样、怪样丑态,如有的患儿走路走得好好的却要跳一下或突然蹲下跪地,有的写字时突然要往纸上戳一下,还有的表现为冲动性触摸人或物、走路旋转、转动腰臀、蹲下跪地或反复出现一系列连续无意义的动作。

2.发声

(1)简单发声抽动:表现为快而无意义的声音,反复发声,如清嗓、咳嗽、吸鼻声、吐痰声、哼声、吠叫声、啊叫声等。

(2)复杂的发声抽动:表现为重复言语或无意义的语音、无聊的语调,重复刻板同一的秽语,如出现类似“滚蛋”、“妈×”的声音。

3.伴发症状 TS除抽动症状之外,常会伴有注意力不集中多动障碍、强迫症状、攻击行为、自伤行为、学习困难以及情绪障碍等。

\subsubsection{诊断}

根据CCMD-3的诊断标准如下:

1.症状标准 表现为多种运动抽动和一种或多种发声抽动,多为复杂性抽动,两者常同时出现。抽动可在短时间内受意志控制,在应激下加剧,睡眠时消失。

2.严重标准 日常生活和社会功能明显受损,患儿感到十分痛苦和烦恼。

3.病程标准 18岁前起病,症状可延续至成年,抽动几乎天天发生,1天多次,至少已持续1年以上,或间断发生,且1年中症状缓解不超过2个月。

4.排除标准 不能用其他疾病来解释不自主抽动和发声。

\subsubsection{鉴别诊断}

抽动障碍诊断时需与下列疾病鉴别。

1.风湿性舞蹈症(小舞蹈症) 儿童较多见,是由于风湿性感染所致,具有相应的体征和阳性化验结果(如血沉、抗“O”及黏蛋白反应等),肢体大关节呈舞蹈样运动,不能随意克制,但非重复刻板的不自主动作,肌张力减低。舞蹈症一般可自行缓解或通过抗风湿治疗有效。舞蹈症很少有发声抽动或秽语、强迫障碍等,可以此鉴别。

2.亨丁顿(Huntington)舞蹈症 是一种神经系统家族遗传病,多于成年起病,但也有少年型。临床是以进行性不自主舞蹈样运动和智力障碍为特征,肌力和肌张力减低,各关节过度伸直,腱反射亢进或减低。

3.肝豆状核变性(Wilson病) 由先天性铜代谢障碍引起,临床有肝脏损害、精神障碍、神经系统损害(锥体外系体征),其不自主运动为锥体外系损害的表现,可为细微震颤伴肌张力增高,亦可为手足徐动症或舞蹈指划样动作。角膜有K-F色素环,血浆铜蓝蛋白减低等特征可资鉴别。

4.手足徐动症(athetosis) 本综合征有先天性和继发于出生后早期中枢神经系统感染、缺氧、中毒等引起纹状体损害,表现有手足徐动、肌强直、智力缺陷等征象。

5.肌阵挛 是癫痫
发作的一种类型,具有发作性特征,每次持续时间短暂,常伴有意识障碍、脑电图异常,解痉治疗有效。

6.急性运动性障碍 表现为突然发生不自主运动、肌张力不全、扭转痉挛或舞蹈样动作。常由于服用抗精神病药物、中枢兴奋药、左旋多巴胺以及胃复安等所引起,停药之后症状逐渐消失,可以此鉴别。

7.癔症与儿童精神分裂症 癔症的痉挛发作和儿童精神分裂症可表现类似抽动样动作,但有原发精神障碍的特征,可加以鉴别。

\subsubsection{治疗}

1.药物治疗

(1)针对抽动症状的药物

①氟哌啶醇(Haloperidol):通常从小剂量开始,每日1~2mg,分2~3次口服;而后逐渐增量,每日总量范围为1.5~8mg,同时可服用抗震颤麻痹药(如苯海索)以减少锥外系反应。常见不良反应有嗜睡、乏力、头昏、便秘、心动过速、排尿困难、锥体外系反应(如急性肌张力障碍、静坐不能、帕金森病样震颤等)。为减少不良反应可适当减量,急性反应严重者可肌内注射东莨菪碱每次0.3mg,每日1~2次。

②匹莫齐特(Orap、Pimozide):又称哌迷清,是一种选择性中枢多巴胺拮抗药,阻滞突触后多巴胺受体的钙离子通道。治疗TS的作用与氟哌啶醇相同,但镇静作用轻,对心脏不良反应较氟哌啶醇多见,可引起心电图改变(包括T波倒置、诱发U波出现、Q-T间期延长的心率缓慢),故在服药过程中需监测心电图的变化。每日服药1次,开始剂量为0.5~1mg,小量增加;儿童每日剂量范围为3~6mg。本品不良反应与传统抗精神病药物相似,但迟发性运动障碍较少见。

③硫必利(Tiapride,泰必利):属含甲砜基邻茴香酰衍生物,具有拮抗多巴胺的作用。本品主要作用于间脑和边缘系统,临床观察治疗TS有效,不良反应较氟哌啶醇轻,但见效稍慢,一般服后1~2周见效。剂量50~100mg,每日2~3次。本品的不良反应主要有头昏、无力、嗜睡。起始剂量过大,可产生恶心、呕吐反应。

④可乐定(Clonidine):是一种选择性中枢α\textsubscript{2}
受体激动剂,直接作用于中枢多巴胺神经元及去甲肾上腺系统,可缓解TS的运动抽动和发声抽动,改善伴发的注意力不集中和多动症状。疗效不及氟哌啶醇和哌迷清,但较安全。通常口服开始剂量为每日0.05mg(小年龄0.025mg),每隔一周酌情加量,分2~3次口服,一般学龄儿童每日剂量范围为0.15~0.25mg。主要不良反应为嗜睡、易激惹、口干、头昏、一过性低血压、失眠等。少数可产生心电图改变,服药期间应定期检查血压和心电图。

⑤利培酮(Risperidone):具有拮抗5-HT\textsubscript{2A}
受体、D\textsubscript{2} 作用,同时有拮抗α受体和H\textsubscript{1}
受体的作用。该药拮抗D\textsubscript{2}
作用较氟哌啶醇及哌迷清弱。近年来,有报道应用利培酮治疗TS获得疗效,可减轻抽动症状,其作用可能与阻滞基底核5-HTD\textsubscript{2}
受体有关。开始剂量为每日0.5~1mg,每5日增加0.5~1mg,平均日剂量为2.7mg。不良反应有头晕、镇静、静坐不能、肌张力障碍、头痛、软弱无力、失眠、抑郁心境、焦虑和激越行为。有学者认为利培酮治疗儿童少年抽动障碍出现锥体外系反应较精神分裂症为多见,而且常见有体重增加和疲劳的不良反应。因此儿童使用利培酮尚需慎重选择。

⑥肌苷(Inosine):控制抽动症状有效率为75%。有报道氟哌啶醇合用肌苷疗法,或以氟哌啶醇合用氯硝地泮(Clonazepam)治疗TS的效果较单一使用氟哌啶醇佳,并可减少氟哌啶醇的不良反应。

⑦其他:四氢小檗碱(tetrahydroberberine,
THB)具有阻滞多巴胺受体的作用,每次口服剂量为1.5~2mg,每日2次。有报道应用阿片受体拮抗剂纳洛酮和纳曲酮治疗TS有效。另外,氯硝基安定、喹硫平、阿立哌唑等治疗抽动障碍也获得明显效果。

(2)针对伴发行为症状的药物

①伴发注意缺陷多动障碍:可采用氟哌啶醇、可乐定、盐酸托莫西汀等治疗。呱法新(Guanfacine)为一种新型的α\textsubscript{2}
受体激动药,近年来有学者应用该药治疗TS伴发ADHD病例有效。该药对多动、注意力不集中及抽动症状均有改善。剂量范围为每日0.5~3.0mg。常见不良反应有:轻度疲劳和镇静作用,对心脏、血压无影响。

②伴发OCD:大多采用氟哌啶醇合并氯米帕明治疗,有明显效果。氯米帕明开始剂量为每日2~3mg,而后根据疗效及不良反应情况,适当调节剂量,一般用6.25~25mg,每日3次。服药期间需定期查血象及心电图。氟西汀也是治疗强迫症有效药物之一,儿童一般剂量范围为每日10~40mg,分2次服。作用与三环抗抑郁药相似。不良反应有消化不良,恶心,食欲减退,皮疹,轻度躁狂等。还有报道应用匹莫齐特、碳酸锂、氯硝西泮等治疗。少数报道以舍曲林(每日75mg)治疗可减轻抽动及强迫行为。舍曲林、匹莫齐特合用的疗效,较单一匹莫齐特效果佳。

③伴发自伤行为:应用氟西汀治疗可减少自伤行为,其机制尚未明确。也有报道应用阿片受体拮抗药纳曲酮(Naltrexone)、匹莫齐特、普萘洛尔等治疗自伤行为有效。

④伴发焦虑或抑郁:可用在治疗抽动症的药物基础上加用抗焦虑药或抗抑郁药。

(3)对于慢性运动或发声障碍,一般无需特别治疗,尤其对于症状已持久且固定不变、已形成了习惯者,如成年人清嗓或眨眼抽动,对日常生活、学习或工作并无影响者,一般无需用药治疗。

2.心理治疗 不同程度的抽动障碍可对患儿自身及其家庭的日常生活和学习带来不同程度的干扰和影响。患儿的症状易受精神创伤、情绪波动或学习负担过重等因素的影响而加重。因此除药物治疗之外,还需要进行心理治疗,包括行为疗法、支持性心理咨询、家庭治疗等。帮助患儿的家长和老师理解疾病的性质和特征,以取得他们的合作与支持,从而采取恰当的教养教育方式。同时改善家庭或学校的气氛,能针对患儿的特点安排患儿日常的作息时间和活动内容,避免过度和紧张疲劳等,有利于改善患儿的行为症状。近年来,有人提出了称为相反习惯训练(habit
reversal training,
HRT)的行为疗法,可减轻TS的抽动症状。如对于发声抽动患儿可进行闭口、有节奏缓慢地做腹式深呼吸,从而减少抽动症状。另外,还有用自我监视和松弛训练疗法,但相反习惯训练疗效最好。