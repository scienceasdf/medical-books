\chapter{会诊-联络精神病学}

随着医学模式向生物-心理-社会模式转变,综合医院内的精神卫生问题日益得到关注。心理社会因素在躯体疾患的预后、康复和生活质量方面具有重要的意义。但是,基层医疗保健、综合医院的临床医护人员由于缺乏本学科领域的相关专业知识,大多数躯体疾病患者的精神卫生问题没有得到早期诊断和及时有效的治疗。会诊-联络精神病学(consultation-liaison
psychiatry,
CLP)就是为解决躯体疾病患者的精神卫生问题以及综合医院精神障碍的临床诊断和处理而诞生并逐步发展的一门学科。其属于综合医院交叉学科、精神病学的亚专科,其学科架构与知识内涵来源于传统精神病学、其他医学学科、心理学和社会学等学科。随着综合医院中精神卫生服务的拓展,会诊-联络精神病学的概念范畴更为明确,服务范围不断扩大,专科发展日益受到重视。

\section{概  述}

\subsection{概念}

会诊-联络精神病学是以精神病学为基础,研究躯体疾病患者的社会心理因素、生物学因素与精神障碍之间相互关系的一门学科。会诊联络精神病学的主要工作是为临床各科医师提供联络和会诊服务,提高他们对各科患者所伴有的心理和精神科问题的识别和处理能力;为患者提供多维诊断和治疗;研究躯体疾病与心理反应的相互关系以及心理和行为治疗对躯体疾病的疗效。会诊精神病学是精神科医生应其他科医生的邀请,对该科患者提出精神病学诊断、治疗和处理建议,提供咨询服务;而联络精神病学则是精神病学与其他学科之间进行联合,共同协作研究和处理躯体疾病。

由于历史原因,我国精神卫生服务的重点长期偏重于精神病院的重性精神病患者。近年来,我国精神卫生事业迅速发展,对生活方式相关疾病,如厌食、病理性赌博、慢性酒精依赖以及心身疾病如哮喘、瘙痒症、肠易激综合征等心理因素的认识也逐渐增加。会诊-联络精神病学也得到了极大的发展。

\subsection{综合医院的心理问题}

根据世界卫生组织1997年报告,综合医院门诊未确诊的患者中有20%~30%有精神卫生问题;已在各科确诊的患者中15%~20%有精神科问题;在全科医生、家庭医生的工作中,40%~60%的患者需进行精神科处理。经校正年龄和性别因素后,慢性躯体疾病患者中精神障碍患病率为25%,无慢性疾病患者为17.5%;慢性躯体疾病患者精神障碍的终身患病率为42%(最常见的是物质依赖、心境障碍、焦虑障碍)。所以,非精神科的临床领域实际上是一个筛查、处理精神科问题,缓解患者心理痛苦的重要场所。

我国调查发现,在1010例综合医院门诊患者中,伴发情绪抑郁和焦虑者,内科为41.5%和34.78%,外科为34.32%和25.66%,妇科为23.96%和15.63%,其他科为25.33%和30%;门诊各科中神经症患者的比例:神经科为40%、外科为18%、妇科为25%、内科为20%、中医科为50%。在1002例住院患者中,伴发抑郁和焦虑者,内科为39.64%和35.11%,外科为20.23%和19.68%,妇科为21.57%和20.59%,其他科为27.66%和23.40%。综合医院精神疾病比例依次为:非精神病性精神障碍、器质性(脑和躯体疾病)精神障碍和重性精神病。研究发现,需要精神卫生干预的人群中约有85%初诊都在社区医疗站和综合医院;综合医院中至少有1/3的患者伴有精神障碍。

\subsection{会诊-联络精神病学的意义}

研究发现,医疗资源过度使用者一半有精神科问题。其中,重性抑郁或心境恶劣障碍占40%、焦虑障碍占21.8%、躯体化障碍占20%、惊恐障碍占12%、酒精或其他物质依赖占5%。此类患者大多在非精神科门诊或病房诊治。而与单纯躯体病患者相比,住在非精神科的抑郁症患者消耗医疗资源多2倍,费用多1倍,曾被送往急诊科的次数高出7倍。有惊恐发作症状的患者,在急诊科看病的次数是单纯躯体病患者的10倍;其中70%的患者在确诊前曾看过10个以上医生。伴焦虑症状的哮喘患者,短期住院的次数比无焦虑者高3倍。酒精滥用或依赖者的总体医疗费用比单纯躯体病患者高2倍,但1/4~1/2的患者收治在急诊内科或外科时未被识别。与躯体疾病共患的精神障碍是医疗费用增加的主要因素,原因之一是临床信息复杂化,提高了诊断难度,导致多余检查;二是治疗复杂化,常在实施躯体治疗方案时出现非预期的治疗后果,或由于合用精神药物而有副作用;三是容易出现医患间交流、沟通困难,影响治疗关系和依从性;四是医务人员对精神疾病的认识水平不高,忽视了精神疾病的诊断和治疗;五是患者对精神疾病的病耻感而不到精神科就诊。这些因素直接导致日均费用增加、住院时间延长。

综合医院门诊中约1/3患者有不同类型、不同程度的精神障碍,而其中仅有极少数患者得到专科治疗。建立和发展CLP的意义在于,更加强调心身整合的疾病概念,以适应人们对精神卫生服务的需求,同时使精神病学重返医学主流,克服“非医疗问题医疗化”的缺陷,促进“医学的社会-心理学化”。

综上所述,通过会诊-联络,可使患者缩短住院天数,提高病床周转率,这是综合医院医生最为关心的问题之一;预防、杜绝事故的纠纷隐患,如自杀、外逃、伤人毁物和医患冲突等;及时转诊和处理有关患者,为其解决求医无门、诊治无方的痛苦,减轻临床各科困难;通过医嘱和会诊行为,传播新医学模式的理论和操作方法,影响临床各科在服务态度、发展建设性人际关系方面的观念和做法。与此同时,也提高精神科医师对于躯体性问题的处理水平。

\section{会诊-联络精神病学的工作范畴}

\subsection{会诊-联络精神病学的任务}

CLP的任务主要涉及以下方面:①相关医护人员的精神科知识和技能的再培训,包括精神科和非精神科的医生、护理人员;②患者相关知识的教育;③躯体疾病患者的精神症状或精神障碍的识别及治疗;④心理、社会因素以及精神症状在躯体疾病发生、临床表现、疗效、依从性、预后等因素的影响的研究,以及在临床实践中提高对于相关疾病的生物、心理和社会层面的综合治疗水平。

\subsection{会诊-联络精神病学的服务模式}

我国目前精神科没有在综合医院普及,CLP服务不系统。国内外CLP服务模式大概有以下几种:

1.以非精神科医生为主的服务模式 目前在综合医院承担联络咨询临床工作的主要是非精神科医生,如神经内科、消化内科或通科医生等。随着精神卫生专业知识的教育培训以及临床工作经验的积累,现在从事这方面工作的有关人员的理论水平和临床技能得到明显的提高。这种模式的优点是临床医生在短时间内在边接受培训、边进行临床实践的基础上开展相应的工作,使更多的患者能够享受到精神卫生服务,同时对精神卫生知识在综合医院的迅速普及也起到了积极的作用。其不利的方面是临床医生兼职对这项工作的专门化和提高带来影响,同时亦影响该领域研究工作的开展。

2.综合医院精神科为主的服务模式 由于对精神卫生以及精神疾病的认识水平的提高,国内许多大型综合医院,甚至是中、小型综合医院都在积极建立或已经建立精神科或提供精神卫生服务的专业部门。这种模式的优点在于精神卫生工作者熟悉综合医院工作程序,并且受到综合医学理论与实践的良好培训,有比较牢固的临床医学知识基础。但是目前国内许多综合医院,甚至有的医学院校的教学医院也没有精神科,即使是建立了精神科,有关人员理论水平的提高以及临床实践经验的积累也需要较长的过程。因此短期内这种模式仅能在少数医院运行。

3.专科精神病院为主的服务模式 以精神科医院或相应的精神卫生专门机构为主体,综合医院可以请求会诊、专题讨论、共同坐诊等方式让精神卫生专业人员加入到识别和治疗躯体疾病患者的精神症状和心理问题中来。这种模式的关键在于建立良好的信息传递机制,特别是建立良好的城市内部或城市及地区之间的院际沟通,或建立专业学会之间的良好沟通。这种模式的优点在于能够充分利用现有的精神卫生人力资源,将精神卫生服务介入到综合医院的医疗工作中,既解决了人力资源的问题,又解决了精神病学融入大医学中的学科结合问题。这种模式运行的困难在于精神科专科医院的医生多数还需要熟悉综合医院的工作模式,同时多数还需要接受有关精神病学以外的其他医学知识的再培训,甚至需要重新接受有关心理学知识的培训。

4.会诊-联络中心的服务模式 由精神卫生专业人员以及其他相关医学领域的专业人员(如内科医生、神经内科医生、心理咨询及保健人员等)建立一个会诊-联络机构来执行这项任务。这种模式的优势在于各类人员之间可以直接交流,知识可以相互补充。此外由于专门机构的建立,容易在此基础上使这项工作逐步发展成新兴分支交叉学科。这种机构可以在卫生行政管理层面,也可以在专业学会的层面上建立。但这种模式运行的前提是人员的专门化程度要求较高,同时协调工作会面临一定的困难。

由于各地区发展不平衡和原有的条件不同,在今后相当长的时间内应该逐步形成多元的工作模式。

\subsection{会诊-联络精神病学的工作类型}

综合医院不同的会诊联络服务,要求的重点、方法、步骤和工作范围是不一样的。通常分为三种类型:

\subsubsection{以患者为中心的会诊}

患者的躯体状态、心理行为有问题(如患者企图自伤),在这种情况下精神科医生则成为处理责任承担者,这是最常见的一种会诊类型。会诊要求做到以下四点:①要求对患者的问题作出明确分析或诊断。②应该回答请求会诊者提出的问题,如患者是否有精神疾病,患者的人格特征是否影响其病情,是否情绪因素或应激发疾病,疾病对患者的意义如何,疾病给患者在人际关系及社会生活方面带来什么样的变化,与患者关系密切的人,如家属、同事等对患者患病的态度和反应如何,会不会有精神方面的残留症状,患者是否进行精神科的特殊治疗等等。③确定会诊者在治疗计划中担任什么角色,如果一些工作要交给邀诊者去做,应该明确告知如何去实施,明确责任。如果治疗涉及会诊者,则应按时随访,如果有些活动必须让患者参加,还应该明确告诉患者并向他说明原因。④治疗计划的实施,必须征得邀诊者或患者家属的同意和在患者能接受的情况下进行。

\subsubsection{以邀诊医生为中心的会诊}

一般涉及:①邀诊者与患者之间关系遭到破坏;②患者不同意其疾病性质和程度的判断;③患者的情感反应危及医生;④其他工作人员不同意对患者的处理;⑤患者拒绝实施治疗项目。

对于这种类型的会诊,会诊者处于中间人的角色,就必须了解双方的意见,分析矛盾发生和形成的原因。必要时还应邀请有关领导参加,因为他们同时了解医生和患者。在与双方交谈时应保持中间立场,尊重双方意见,表示同情和理解,不要马上给予肯定或否定的答复。在处理上也分为四个步骤:①了解双方,重点应该了解双方交往的过程和形式、患者的诊断;②会诊意见应包括整个情况,特别是患者的行为,导致关系更坏的第一次原因,最好对双方都提出意见;③治疗处理方案中会诊角色为中间人;④实施治疗计划,会诊者要与双方接触、交谈,包括说服、解释,要委婉而恳切地提出应该怎么办,使双方建立新的关系。

\subsubsection{以整个医疗小组为中心的会诊}

会诊医生作出建议时应考虑到参与护理患者有关的全组成员人与人之间的情况,这常在监护病房中应用。

\subsection{会诊-联络精神病学的主要内容}

从国内医院会诊服务的情况看,所有设置住院部的科室均有会诊申请,中医科及内科会诊率最高。申请会诊的主要问题具有“广谱”性,由轻到重可以是适应不良、人际关系危机(包括医患沟通障碍、医源性障碍、依从性低)、神经症症状、重症抑郁、焦虑恐惧、中毒、自杀、兴奋躁动、谵妄、厌食后恶病质。会诊后的诊断:器质性精神障碍为36.45%,神经症为32.9%,有精神病史者(包括现患)伴躯体疾病为10.64%,躯体疾病引起焦虑抑郁状态为6.77%,精神活性物质所致精神障碍为2.58%,其他(如手术恐怖、疼痛、医患关系问题等)为3.55%,无精神障碍为7.10%。主要为临床各科因诊治、转科、鉴定等缘故,需精神科提出排除精神疾病的诊断意见,或对手术、药物治疗和护理措施的心理社会及神经精神效应提供咨询意见。

1.临床各科需要与精神科共同处理的情况 综合医院各科室住院或门诊的患者中,往往既有躯体症状,又有精神症状,需要精神病学科与非精神病学科的共同处理。临床各科需要与精神科共同处理的四种情况:一是躯体疾病患者患病后出现的心理行为反应,如手术患者的术前焦虑和术后抑郁等;二是躯体疾病或治疗过程如药物导致的精神症状;三是患者的躯体功能障碍或不适是精神障碍的表现;四是躯体疾病与精神疾病的共病状态,即患者既有躯体疾病又有精神疾病,如抑郁症患者共患心肌梗死。

2.综合医院中常见精神卫生问题具体内容 在综合医院中,精神卫生问题涉及面非常广,遍布各科室的所有患者。具体内容包括:①一般心理问题,如轻度的烦恼、恐惧、焦虑、抑郁等。有些患者的心理反应严重,如各科危重患者、慢性病患者、创伤者、癌症患者、器官移植患者等。也可能是躯体疾病比较轻,但患者的心理承受能力差,同样会出现严童心理问题。②诊治过程中的心理问题。患者在医院中所接触的一切,如医院环境,医务人员的言语、行为举止、服务态度、各种仪器检查、各种收费、治疗(药物治疗、手术治疗、理疗、化疗、放疗、透析)等,都可引起患者的各种心理问题。③心身障碍和心身疾病。④神经症性障碍(如焦虑症、强迫症、恐惧症、躯体形式障碍、神经衰弱)、非精神病性精神障碍(如各种疾病引起的情绪障碍:焦虑综合征、抑郁综合征等)。⑤不良生活方式与行为所致的精神障碍。⑥心理因素相关的生理障碍。⑦人格特征突出与人格障碍及性心理障碍。⑧器质性精神障碍,包括脑和躯体疾病引起的精神障碍。⑨精神病性障碍,如原本患有精神病的患者又并发了躯体某系统疾病而到综合医院门诊就诊或住院治疗的患者,往往需要躯体疾病和精神病同时治疗。⑩其他精神障碍,如儿童少年期精神障碍、精神发育迟滞等。

\section{会诊-联络精神病学的临床应用}

\subsection{会诊-联络精神病学的服务职能}

以国内现有的医疗工作模式为基础,在CLP临床实践中,涉及综合医院各类会诊-联络服务人员的职能。

\subsubsection{精神科医生}

精神科医生的主要职能是对非精神病学专业的临床医生进行相关精神病学、医学心理学和心身医学知识的培训,特别是对常见精神病症状识别和治疗的培训,还包括对躯体疾病和中枢神经系统疾病患者精神症状的会诊,提出诊断、评估和治疗意见。

\subsubsection{非精神卫生专业的医生}

通科医生是诊治躯体疾病的主体,在CLP实践中的主要职能是:①对患者精神状态初步评估;②识别患者存在的精神症状或精神卫生问题;③对患者进行初步治疗;④就存在精神症状或精神卫生问题进行分诊(尤其应该成为综合医学急诊科和通科医生、社区保健医生的重要职能);⑤请精神科医生对较为复杂的精神症状或精神卫生问题申请会诊,或及时转诊。

\subsubsection{心理工作者}

目前国内许多医院的临床职能科室尚未设立专业心理工作者的岗位,甚至精神科也是如此。在CLP中,心理工作者的职能应该包括参与对患者精神状态的评估,参与对患者存在的心理社会因素与躯体疾病或中枢神经系统疾病关系的评估,对患者的心理卫生问题进行心理干预和心理保健,尤其是对患者所患躯体疾病的心理保健问题提出建议等。

\subsection{会诊-联络精神病学的服务场所}

\subsubsection{病房}

精神科医生应其他科医生的请求,对会诊患者进行检查,提供诊断和治疗意见。有时可能面对一组同类患者(如同时收治一组急性中毒伴有精神障碍的患者),此时精神科医生就不是以患者为中心的方式发挥作用,而是以联络精神科医生角色参加医疗小组,进行讨论,巡视病房,共同参与患者的诊断和治疗及其有关心理、社会问题的研究。

\subsubsection{急诊室与ICU}

精神科医生向其他科室急症患者的诊疗问题提供会诊和治疗服务,其中以各类急性中毒,兴奋状态及自杀、自伤急症为多见。

\subsubsection{门诊或联合门诊}

据粗略统计,神经内科、内科门诊患者中约有1/4是神经症患者,此时可在门诊或联合门诊进行检查、诊断和治疗,也可转往精神科门诊处置。

\subsection{会诊-联络精神病学基本技能}

CLP的基本技能包括:病例筛查、诊断、干预、治疗和沟通。

\subsubsection{病例筛查}

发现病例可通过以下方法完成:①在躯体检查或门诊时病史询问;②结构式访谈;③自评测查。结构式访谈和自评调查是筛查和确认内科患者中多数精神障碍的主要手段。指导非精神科医生进行结构访谈和使用简易的调查量表成为精神科医生在CLP领域的一项重要工作,当然这种指导也包含了和非精神科专业医务人员的沟通。结构式访谈一般由数个结构访谈问题组成。结构访谈对于非精神科医生在临床实践中了解和发现患者的精神问题是行之有效的。

\subsubsection{诊断与鉴别}

精神科医生的会诊-联络诊断中,面临两方面的困难:一是患者多是躯体疾病与精神障碍并存,治疗上较为复杂棘手;二是情况紧急,必须快速、正确而有效地作出决断。这就要求会诊医生既熟悉本科业务,又要了解各种躯体疾病可能发生精神症状及各种药物所致的精神障碍。因此,可在会诊前对患者所患疾病,各项检查结果,会诊的目的及要求解决的问题等进行了解,以便心中有数。精神科诊断所要求的病史资料常常收集困难,检查时间有限,加上疾病分类、诊断标准的不一致,都会给诊断带来一定困难。精神科的诊断,不论资料是否完全,精神科医生应根据一般状态、认知、情感及意志行为活动对患者的主要症状、检查所见及精神状态等予以综合分析,其中要特别注意患者的意识状态,如有无意识障碍,是哪种意识障碍等,因为这对诊断有无器质性精神障碍至关重要。患者的记忆力及智能情况如何,有无自知力,现实检验能力如何,可据此确定患者的精神状态属于精神病性抑或是神经症性障碍。对有些患者还要进行某些心理测验,尤真是智力测验,有助于鉴别诊断。此外,病前个性特征,各种实验室辅助检查,也可作为诊断及鉴别诊断的参考。

会诊医生在确定主要症候群症状后,要根据患者的实际情况分析精神症状与躯体疾病之间的关系。可根据二者起病时间关系设想可能有以下几种情况:精神障碍在前,躯体疾病在后;两者同时发生或者躯体疾病在前,精神障碍在后。若精神障碍发生在前,当然使我们想到患者原有某种精神疾病,现在又患躯体疾病,二者可能没有直接关系。如果二者同时出现,虽然二者发展过程、表现严重程度并不一致,其精神障碍是躯体疾病的一部分,属于器质性精神障碍。至于精神障碍发生在躯体疾病之后,这种情况就应该考虑两种诊断的可能性:一是精神症状是器质性疾病的心理反应或是由于严重的躯体疾病的应激结果;二是要考虑躯体疾病治疗过程中药物的影响,许多药物都可引起精神障碍,其症状也多种多样。

1.精神症状发现与判断 一般可分四种情况:第一是大脑结构的病变所致,如脑血管病变导致的多发梗死性痴呆症;第二是大脑功能障碍导致精神异常,如癫痫
发作,可以有明显的脑电波病变;第三是大脑代谢或生化病变所致的精神症状,如生化代谢病变(为某种酶缺乏)所致的精神发育不全;第四是病因或发病机制未明的所谓“功能性”精神病的症状,虽说目前对其病变机制不十分明了,但可以肯定有其病理基础,有待我们发现。

本书的第二章、第三章较为详细地描述了如何判断一个人的精神活动或行为表现是否为精神疾病的症状和精神症状检查手段与技巧,可以参考。

2.症状梯级概念与等级诊断制 精神症状是大脑的病理产物,不同的精神症状反映出大脑不同广度与不同严重程度的病理生理变化。大脑损害范围广、程度重时所产生的症状较之大脑损害程度轻、范围窄时所产生的症状等级要高,而越是等级高的症状越具特异性;相反,越是低等级的症状越具普遍性,其特异性就差。

精神症状在精神疾病诊断中的地位远远高于内科疾病症状在内科病诊断中的地位。如内科的发热症状常常无法使医师做出某一疾病的诊断,而需进一步检查以寻找某一疾病诊断的客观依据;而精神科的许多症状,如妄想、幻觉则往往是医师赖以作出精神病诊断的重要依据。比起内外科疾病的生化学或实验室诊断指标(如血糖值对糖尿病的诊断价值)来,精神科的症状诊断其特异性仍较低,任何一种精神病至今尚无独特的症状。一般来说,精神症状的特异性以脑器性症状群最高(如意识障碍、痴呆、遗忘等),因为它只见于脑器质性精神障碍;精神病性症状群次之(如幻觉、妄想),因为它可见于器质性精神障碍与“功能性”精神障碍;而神经症症状群特异性最差(如焦虑、头痛、失眠、疲劳等),因为它可见于任一精神疾病中。即如果一个患者有神经症症状和精神症性症状,肯定不考虑神经症的诊断;一个患者有器质性症状群和精神病性症状群,肯定考虑器质性精神障碍而不考虑“功能性”精神障碍。所以,临床上一旦发现患者有意识障碍、痴呆、遗忘、器质性人格改变等,就要考虑器质性精神障碍的可能性而做进一步检查。如果实验室检查能发现器质性病因(如感染、中毒等)以及这些病因与精神症状之间存在着互为消长的关系,诊断并不困难。然而,不少侵犯脑的疾病早期仅表现为精神病性症状(如幻觉、妄想)或神经症性症状,而无神经系统体征或意识、智能等方面的改变,就容易造成诊断的混淆甚至延误治疗。因此,临床医师要带着鉴别诊断的观念来检查、诊断每一位具有精神病性症状或神经症性症状的患者,以防漏诊“潜伏”在这些精神症状后面的器质性精神障碍。

3.躯体检查与特殊检查 精神症状可以由精神疾病所致,也可以是躯体疾病的伴发症状。精神病患者也可以伴有躯体疾病,因此进行体格检查、神经系统检查、实验室检查、脑影像学检查和神经电生理检查对精神障碍的诊断及鉴别诊断十分重要,也是拟定治疗方案的依据。面对任何一个具有精神症状的患者,第一假设应该是“他可能患有躯体疾病”,在这个假设的前提下进行排查。对住院患者均应按体格检查的要求系统地进行。对门诊或急诊患者也应根据病史,重点地进行体检。只重视精神症状而忽略体格检查往往会导致误诊。

神经科与精神科是两个关系密切的学科,不少神经科疾病可伴有精神症状,反之亦然。因此,对精神患者进行仔细的神经系统检查是特别重要的。

实验室检查对确定某些症状性精神病及脑器质性精神病的诊断,能提供可靠的依据。应根据病史结合临床所见,有针对性地进行某些辅助检查或特殊检验,如脑脊液及异常代谢产物的测定;对智能障碍、人格障碍等患者进行心理测验,如韦氏智力测验、人格测验和神经心理测验是必要的。

4.临床多维诊断 整体论的医学观要求临床医师具有一种在所观察到的现象间建立普遍联系,并对这些联系赋以意义的能力。例如,观察、诊断和治疗疾病时,常需考虑心-身、过去-现在-将来、个体-环境、社会人文-心理行为-自然科学等关系。我们需要由宏观到微观的不同观察视野。

(1)社会-文化背景:社会文化背景因素对某些疾病有重要的影响,如汽车文化与交通事故的增多、酒文化与肝硬化,均存在较直接的因果联系。所以,应设法了解患者的价值观、对待事物的态度、社会经济地位、性别角色、自我意识、受教育水平,还应熟悉相关的风俗习惯、国家政策、法律、传媒等等。

(2)人际系统:社会文化大背景的影响,常常要通过个体所在的密切交流网络才能实现。一个能从中获取情感支持或资源的“社会支持系统”(social
support
system)尤其重要,许多对于疾病过程有影响的积极因素或消极因素皆是起源于此。例如应激性处境或事件,它通常包括家庭、社区、工作单位、亲属网络、朋友-伙伴圈子,以及各种社会群众组织。

(3)个体心理特征、体验与行为:社会环境因素与躯体过程之间互动的结果,一定程度上取决于个体心理中介机制(mediate
mechanism)的加工过程。

(4)机体的生物学过程与病理生理方面:用生物医学的理论和方法探查、处理机体结构和功能方面的问题,是临床各专科会诊医生的基本工作内容。

综合多个层面的信息,临床医师可以较全面地对患者的处境作出评价,并提出有针对性的解决方案。

5.诊断思维步骤 在诊断思维过程中应予避免常见几种极端的形式:①过于自信,主观武断,强调直觉,即使证据不足,也轻易作出诊断;②过分的谨慎小心,遇事犹豫不决,反复权衡,这种疾病有可能,那种疾病亦不能排除,结果罗列多个可能的诊断,似乎是面面俱到,却解决不了实际问题,也不可能制定明确的治疗计划;③强调以往个人实际经验,沿引类推,针对患者某些个别症状与个别转归而提出自己独特的诊疗见解,走进了狭隘的死胡同而不能回头。以上几种情况不仅会造成误诊或延误治疗,同时束缚了自己的思维,限制了自己医疗水平的提高。为反映诊断思维的程序,现介绍临床实用的联络-会诊诊断步骤。

(1)收集资料:①临床病史:区别可靠与存疑的事实;②体格检查:包括躯体检查和神经系统检查;③精神状况检查:获取主要精神症状;④实验室检查:包括常规检查、EEG、CT、MRI、CSF检查等;⑤病程观察:疾病的演变情况。

(2)分析资料:①如实评价所收集的上述资料;②根据资料价值,排列所获重要发现的顺序;③选择至少一个,最好2~3个重要症状与体征;④列出主要症状存在于哪几种疾病,从器质性到重性精神疾病再到轻性精神疾病的等级逐一考虑;⑤在几种疾病中选择可能性最大的一种;⑥以最大可能性的一种疾病建立诊断,回顾全部诊断依据、正面指征与反面指征,最好能用一种疾病的诊断解释全部事实,否则考虑与其他疾病并存;⑦说明鉴别诊断与排除其他诊断的过程。

(3)随访患者、观察治疗反应,进一步确定诊断或否定诊断。

\subsubsection{治疗原则}

治疗原则应根据患者具体病情,做出恰当的处理建议,同时要考虑到邀请会诊科室的设施、管理条件,是否需要转至精神科治疗。如一个有强烈自杀企图的患者或极度兴奋的患者,在躯体疾病无严重后果的状况下,转精神科治疗为好。如果患者躯体情况不允许转诊,可由会诊-联络医生、护士协助处理。有时情况不甚明朗,会诊医生应权衡利弊妥善处理。治疗应该首先处理原发病,然后是对症处理精神症状。由于患者往往躯体情况不好,精神药物的剂量宜小,缓慢增量,密切观察药物反应及副作用,特别是对高龄患者。心理治疗的选择有很大余地,通常会诊医生只能提供时间有限的短期整合心理治疗及危机干预等方式。需要长期治疗的患者则由心理治疗师来进行。

\subsection{常见会诊-联络的临床问题}

综合医院,会诊-联络精神科医生常见到各种各样的患者,如器质性精神障碍(包括脑器质性精神障碍和躯体疾病所致精神障碍)、躯体疾病伴发精神障碍(伴发精神分裂、躁狂抑郁症及智能低下等)、心身疾病,其他特殊精神卫生问题(如监护室综合征,血液透析,濒死状态患者管理问题、疼痛、物质滥用和相关法律问题等)。现将较为常见者简介如下:

\subsubsection{脑器质性综合征}

脑器质性综合征包括以谵妄等意识障碍为主要症状的急性精神障碍和以痴呆为主的慢性病症。轻度的意识障碍和轻度智能低下常被忽略。如果中等度以上,出现行为异常致使无法进行有效躯体疾病治疗,病房管理困难。当躯体症状较重,又不能转往精神科,两科医生的密切协作就十分必要。随着人口老龄化,此种情况老年病房中常见。

\subsubsection{抑郁症、神经症性疾病和心身疾病}

抑郁症或抑郁状态、焦虑症及躯体形式障碍等患者,常首先到其他科就诊,而后转到精神料,也有的患者以躯体不适症状为主诉,但又无相应躯体体征时,往往请精神科医生会诊。值得注意的是难以理解的躯体形式障碍,分别占门诊患者的50%以上和住院患者的30%以上。这种异常情况可分为两类:一类是没有器质性病变,但由于有不适感,来寻求医疗帮助;另一类为对已有的躯体疾病,经过治疗或康复效果不佳,希望进行心理治疗者。还有一些既往有精神疾病史,而现时未发现精神障碍,从预防角度出发要求会诊者。

\subsubsection{外科手术前后的心理反应}

手术前主要是对麻醉剂的恐惧,手术的痛苦,手术失败的可能性,手术后遗症,社会康复及意外等的不安和担心,从而要求精神科医生干预。一般手术前的情绪压抑、不安和紧张焦虑较严重。术前的不安易对术中及术后过程产生影响。所以大手术前,有可能需要精神科医师参与。必要时应将手术的内容,手术与疾病的关系等在术前向患者解释说明,使其充分放松。

\subsubsection{监护室综合征}

监护室综合征指收治在监护室(ICU,
CCU)内的患者出现精神异常。当患者需要送入监护病房时,大部分患者的焦虑已构成极大的情绪压力。对监护病房的陌生感、其他危重病友的不良影响、医护人员忙碌、紧张的医疗措施、单调的器械声,加之装置各种导管,行动与饮食都受限制,难以与人交流等因素,都给患者带来极大的压力和不快感。对于一位意识清楚的患者,面对上述种种情景,可产生一系列强烈反应。另外,死亡威胁带来的恐惧,极度的焦虑、抑郁、急性梦样状态等,常常需要精神科会诊。此时患者对原有疾病肯定已使用过多种药物,可能形成药物之间相互作用,增加了精神障碍的复杂性。而对精神科医生的要求是,必须快速、正确而有效地作出诊断和治疗建议,这时需要全面而仔细地进行处理。

\subsubsection{人工透析及脏器移植}

肾病后期,由于肾功能不全而进行人工透析的患者,长期不能恢复工作,治疗时间长,依赖器械生存并时刻面临死亡的威胁,因此普遍存在心理问题,甚至发生精神障碍。器官移植问题也是当今的焦点之一。器官提供者的家属,接受器官移植的患者都有较大的心理负担,忧虑担心,判断障碍,有时有严重的心理障碍。精神科医生应该对器官提供者及接受者的心理危机有所评估,双方都有术后脆弱性,加上手术排异反应,都可导致精神障碍。

\subsubsection{癌症}

由于诊疗技术的进步,癌症患者的存活期有所延长,而患者的长期精神压力并未缓解,医务人员对患者的情感支持也就需要延长。精神科会诊的重点是患者与严重威胁生命的疾病争斗过程中所承受的痛苦、压力,这些磨难不仅影响患者,也波及家属。对大多数癌症患者而言,诊断之后的时间最关键。患者最初的反应往往是震惊,将信将疑,心情矛盾;接着是拒绝接受事实;随后可能是愤怒和忧郁。有时患者的不良反应可能会导致拒绝治疗,认为死亡不可避免,治不治都是一样。另一个常见的反应是寻找其他医生或其他方面的帮助,寄希望于“特效治疗”。如果患者能得到家人及医护人员强有力支持,往往能安然度过诊断、手术、放射治疗、化学治疗等阶段。对癌症患者最难忍受的是在病情显著好转之后又再度恶化,此时患者会有焦虑、忧郁、烦躁、食不甘味、失眠等,他们可能怀疑过去及未来的治疗是否有效,预感到死亡的来临,也可能变得多疑,不再信任医务人员。精神科医生往往因此被邀以评价患者身心状态,提出适当的建议,对患者施以心理治疗。