\chapter{精神疾病护理}

\section{概  述}

\subsection{精神科护理学}

精神科护理学是建立在护理学基础上,研究对精神病患者实施护理的一门专科护理学,它是精神医学的一个重要组成部分,又是护理学的一个分支。它的主要任务包括以下几个方面:

1.研究和实施护患沟通技巧,建立良好的护患关系,开展心理护理。

2.研究和实施对精神病患者科学管理的方法和制度,确保患者安全。

3.研究和实施对精神病患者进行观察的有效途径和护理记录的方法,防止意外事件的发生。

4.研究和实施对各种精神病患者开展以患者为中心的整体护理。

5.研究和实施对精神病患者执行各种治疗的护理,确保医疗任务完成。

6.研究和实施对患者、家属的健康教育,促进患者早日康复。

7.研究和实施对社区人群进行心理卫生教育,防止精神疾病的发生。

8.研究和实施对患者的康复护理,促进患者回归家庭、回归社会。

\subsection{精神科护理的发展趋势}

1.病房管理由封闭式管理向开放式管理转变,科学的管理模式有利于精神病患者的康复。

2.临床护理由功能制护理向整体护理转变,使患者得到全身心的护理。

3.重视健康教育,普及精神卫生知识,做到有效的防治结合。

4.加强对慢性精神病患者的康复护理,促进回归家庭和社会。

5.加强社区精神病患者的护理,充分利用护理资源。

\section{精神科护理的基本内容、基本要求与基本技能}

\subsection{精神科护理的基本内容}

\subsubsection{对异常精神活动的认识}

一般人对精神疾病都有一些错误的观念,认为精神病患者是可怕的、危险的、可耻的、不可治愈的,因此对精神病患者敬而远之。其实,精神病患者在日常生活中表现出来的异常精神活动,完全是病态反应,从患病的角度来看,精神病患者是受害者。所以,精神科护士对患者的精神活动异常应有正确认识。

首先,精神病患者行为异常的表现是在特定的致病因素影响下,只是一部分行为偏离正常,而不是全部。行为的正常与不正常只是程度上的不同,而非种类的差异。精神科护士应利用患者行为正常的部分,理解他们,尝试与其沟通,帮助他们走出疾病的困扰,逐步恢复正常行为。

其次,精神患者的行为是有目的、有意义的,是为满足其某种需要而表现出的行为。这种行为偏离正常人行为规范,但并非与正常人完全不同。因此,精神科护士应了解患者的真实需要、欲望的缘由,不制止,也不掩盖,密切观察患者的行为,采取适当的措施,帮助和保护患者,并指导其学习解决实际问题的有效方法。

第三,精神病患者也是人,他们也有各种情绪反应,只是表现形式不同。因此,精神科护士必须了解患者情绪反应的原因,尊重患者,无条件地接纳这个“人”。

第四,精神疾病在生理和心理上存在特殊性,疾病是生理、心理和社会因素互相作用的结果。因此,精神科护士应注重全面的护理,鼓励患者坚持长期而有效的治疗,改善疾病的治疗护理效果。

\subsubsection{精神科护理的特殊内容}

1.心理护理 心理护理对精神病患者来说十分重要。患者的各种异常活动往往不会引起别人的同情和理解,甚至还会遭到亲人或其他人的误解和指责,这些都会加重患者的心理负担。精神病患者在不同的疾病阶段会有不同的心理反应,采取针对性的心理护理,帮助患者解决心理问题,有利于疾病的康复。同时心理护理在不同的疾病阶段有着各自的侧重点。良好的护患关系是心理护理的基础,护士的专业知识、服务态度和工作技巧是心理护理成败的关键。

2.安全护理 精神病患者由于精神症状的影响,某些行为往往具有危险性,如自伤、自杀、攻击行为、出走行为等。因此,精神病患者的安全护理是精神科护理的重要工作。

3.饮食护理 精神病患者由于各种原因的影响会出现拒食、抢食、暴饮暴食、进食困难及吞食异物等表现,护理人员要按时按量,根据病情给予适宜的饮食,保证营养供应。

4.睡眠护理 睡眠障碍几乎见于各种精神病患者。睡眠的好坏与病情、服药的情况密切相关,因此做好睡眠护理,保证患者适当的睡眠,对巩固治疗效果、稳定情绪有重要作用。

5.个人卫生护理 意志减退、生活懒散和行为紊乱的患者不知道料理个人卫生;有的还整天沉湎于自己的世界,不主动自理生活。因此护士要协助或督促患者做好个人卫生,保持全身清洁。

6.保证医嘱的执行 与内、外科患者不同,大多数精神病患者缺乏对疾病的认识,不认为自己有病,因此,往往无主动求治的欲望,有的甚至强烈反对接受治疗。所以,保证准确地执行医嘱,让患者得到及时必要的治疗在精神科护理工作中显得尤为重要。

\subsection{精神科护理的基本要求}

\subsubsection{对精神科护理人员的要求}

精神科护理对象的复杂性、特殊性,给精神科的护理工作增加了难度,因而对精神科护理人员提出了较高的要求,精神科护理人员应具备一定的职业道德、心理素质和专业知识。具体表现在以下方面:

1.有良好的职业道德,树立全心全意为患者服务的思想。

2.有强烈的敬业精神,热爱自己的本职工作。

3.有同情心,维护患者的利益,尊重患者的人格和权利,替患者保守秘密。

4.有健康的心理和良好的情绪,适应工作任务和工作性质的需要。

5.有敏锐的观察能力和分析能力,善于发现问题和解决问题。

6.有慎独精神,严格执行各项操作规程和规章制度。

7.有丰富的生物医学、心理学和社会学知识,成为一个合格的护理人员。

8.有开展精神科护理教育与护理科研的能力,胜任精神科护理工作。

\subsubsection{精神科护士的角色作用}

社会角色是人们的社会地位决定的,为社会所期望的行为模式。每个人都承担着不同的社会角色,每个角色都要表现其角色的特征,使自己的行为与所承担的责任、义务一致。由于精神科护理的特殊性,护士应充当以下角色:护理的角色(这是最主要的角色)、管理的角色、教师的角色、协调者的角色、顾问的角色、安全员的角色、科研的角色。

\subsection{精神科护理的基本技能}

\subsubsection{精神疾病护患沟通}

护患沟通是了解病情的重要途径,也是精神科护理中最基本的工作之一。精神病患者由于精神障碍失去了与周围环境的正常联系,表现出许多异常的、难以理解的行为、思想和情感。护理人员要恰当地处理这些变化多端的情形,很好地与患者沟通,诱导患者保持正常的生活的,确是一件细致而复杂的工作。在接触患者时,不但要求护理人员具备一定的职业道德与业务技术水平,还要求对精神病患者有正确的认识,树立全心全意为患者的观点,并在实际工作中做到以下几点:

1.尊重病患者的人格,同情、关心和爱护患者精神病患者 心理状态比正常人更敏感,比健康人更渴望被尊重、被重视、被关怀,因此,在接触中,要特别注意尊重患者。不论患者的症状表现如何,都应像对待正常人一样,按其不同年龄、性别、习惯等给予恰当的称呼,不可轻视或戏弄患者,或任意给患者取绰号。对患者的态度应温和、亲切、耐心、严肃。对患者的不正常行为,不可嘲笑和愚弄。对患者提出的问题要注意倾听。对患者的合理要求,应尽量满足,不可哄骗或轻易答应一些办不到的事情。对不合理要求要耐心解释说明。关心患者的疾苦,处处体贴照顾患者,与患者建立良好的护患关系,以取得患者的信任和合作。

2.要熟悉病情 护理人员不但要认识每个患者,同时要阅读、熟悉每个患者的病历,了解患者的发病有关因素、发病过程、症状、诊断、治疗、特殊注意事项等,以便使自己更有把握地接触患者及恰当地处理患者的询问和要求。接触患者时,可以从患者的兴趣、爱好以及生活、工作等为话题,进行交谈,启发患者叙述要了解的内容。当患者叙述病情时,应耐心倾听,不要随便打断患者的谈话或贸然对其所谈的内容进行批评,以便掌握病情,做好护理工作。

3.与患者保持正常的护患关系 接触患者要普遍,避免只接触少数患者而忽视了大多数患者,除非是病情特别严重需要特别护理的患者。在接触异性患者时,要特别注意,一定要有第三者在场,接触时态度要自然、谨慎。有的患者由于病态的思想感情,可能会对医务人员产生不正常的情感,应加以注意。与患者接触时不应该谈及有关工作人员的私事,所有工作人员的名字、履历和住所及其他患者的病情等均应加以保密。

4.要提高自身素质,提升护士的影响力 注意自己的仪表,护士帽、工作服要穿戴整齐,工作时要精神饱满,给患者以愉快振作的印象。工作人员之间要团结、一致,互相配合,以提高患者对护理人员的信任感。避免在患者面前讨论其他护理人员的技术能力。

5.运用护患沟通的技巧 克服影响护患交流的不利因素,做到与患者有效地沟通。在沟通过程中要注意以下几点:眼神要正视对方,表情要自然,姿态要稳重,语态有修养,善于倾听患者诉说,善于引导患者话题,善于察言观色,适当运用沉默技巧,适时运用触摸法。尤其注意对于不同精神症状的患者需采取不同的接触技巧。

\subsubsection{精神疾病的观察与记录}

1.精神疾病的观察 严密观察病情,及时掌握病情变化,是精神疾病护理的重要环节,也是提高护理质量的重要标志之一,在精神科临床中有着特别重要的意义。

(1)观察的内容

①一般情况:个人卫生情况;生活自理程度;睡眠、进食、排泄、月经等情况;接触主动或被动,对人的态度热情或冷淡、粗暴或抗拒、合群或孤僻等;参加各种活动时的情况;对住院和治疗护理及检查的态度。

②精神症状:有无感知、思维、情感、意志、行为、注意、记忆、意识智能障碍;有无自杀、自伤、伤人、毁物及出走等企图;有无愚蠢、离奇、刻板、模仿等动作行为;精神状态有无周期性变化;有无自知力。

③心理状况:心理问题和心理需要;心理护理的效果。

④躯体情况:一般健康状况(体温、脉搏、呼吸、血压);有无躯体各系统疾病;全身有无外伤。

⑤治疗情况:患者对治疗的合作程度;治疗的效果及不良反应;其他明显的不适感。

⑥社会功能:包括学习、工作、社会交往的能力。

(2)观察的方法

①直接观察:是指护士与患者面对面进行交谈时,或患者独处、与其他人交往、参加集体活动时,护士直接观察患者的语言、表情及行为,从而获悉患者的精神症状、心理状态与躯体等方面的情况。

②间接观察:是指护士通过患者的家属、朋友、同事了解其情况,或从患者的书信、日记、绘画及手工作品等了解患者的情况。

(3)观察的要求

①针对患者的具体情况,分别掌握要点。如新入院患者及未确诊的患者,要全面观察。开始治疗的患者,要着重观察患者接受治疗的态度、治疗效果及不良反应。一般患者要观察病情动态变化以及病情好转、波动的先兆。疾病发展期患者要重点观察其精神症状和心理状态。缓解期患者要重点观察病情稳定程度及对疾病的认识程度。恢复期患者要重点观察症状消失的情况、自知力恢复的程度及对出院的态度。

②从患者异常的言语、表情、动作、行为中分析可能发生的问题。若发现患者一反常态,如抑郁症患者情绪突然豁然开朗,恢复期患者突然情绪低沉、闷闷不乐,都应严格注意患者的变化动向,认真地交班,预防意外事件的发生。

③要善于识别精神症状或躯体疾患的主诉。不可将患者的疑病症状误认为躯体疾病,也不可将患者的躯体主诉误认为精神症状,延误治疗。对患者的反映,应给予足够的重视,切不可看成“胡言乱语”而不予理睬。

④在患者不知不觉中进行观察。护士通过与患者交谈来观察时,要使患者感到是在轻松地谈心、聊天,此时患者所表达或表现的情况较为真实。

2.护理记录 护理记录是护士将观察到的结果及进行的护理过程用文字描述记录。书写好护理记录,可供医师参考,协助医师做出准确诊断,使患者得到恰当的治疗与处理。护理记录还是病案及法律的资料,可作为科研的资料或司法鉴定的材料,同时也是反映护理质量的重要标志之一。

(1)记录的要求

①保持客观性,尽可能将患者的原话记录下来,尽量少用医学术语。

②及时、准确、具体、简明扼要地记录所见所闻的事实。

③书写项目齐全,字迹端正、清晰,一目了然。

④记录不可涂改,如有错误,避免使用修正液、橡皮擦或剪贴,可用笔划掉,签上全名。

⑤记录完成后签全名及时间。

⑥新入院患者要日夜三班连续三天记录。重点患者(如精神症状严重或伴有躯体疾病严重者)日夜三班写护理记录,但躯体情况或特殊情况的记录不能代替责任护士的包干记录。病危护理患者每天日夜三班各记录1次。一级护理患者每周记录2次;二级护理及三级护理患者每周记录1次;病情波动的患者要随时记录。出院、请假离院、返院、转院、转科(病区)的患者要随时记录。

(2)记录的内容:护理记录内容必须丰富具体、有系统、有重点、有连续性,包括患者各种精神症状的变化(如知觉、思维、行为、情感等表现,特别是有无消极、冲动、逃跑等情况);躯体状况异常的变化及处理;一般生活情况(如饮食、睡眠、大小便、月经等);参加康复治疗活动的情况;接受治疗、服药后的反应情况。

对新入院患者,要记录入院时间、伴送者、住院次数、入室方式;入院前的主要异常表现;入院时的仪态、精神状态、躯体情况;患者对住院的态度;主要医嘱及注意事项;护理要点。

病情波动的患者,要记录精神症状的动态变化。

危重患者,要记录病情变化及抢救、护理过程。

死亡患者,要记录病情变化;抢救时间和整个抢救过程;呼吸、心跳停止的时间;尸体料理情况。

请假离院患者,要记录目前病情,请假离院时间及伴同者,带药情况和阐明的注意事项。

请假离院返院患者,要记录返院时间及伴送者,请假离院期间的表现,返院后的精神状态及其他情况。

出院患者,应记录出院时精神状况,有无自知力,治疗效果,出院时间以及接患者出院的家属。

转出患者,要记录转出的原因、去向及时间。转入患者记录同入院时记录。

特殊标本的留取情况及重点治疗,如各种穿刺、输血、输液、中药、针灸、电休克等治疗过程中出现的问题、治疗效果及不良反应。

\subsubsection{精神科患者的基础护理}

1.个人卫生护理

(1)重视卫生宣教,经常向患者宣传个人卫生,帮助患者养成卫生习惯。

(2)督促和协助患者养成早晚刷牙、漱口的卫生习惯,生活不能自理的患者,进行口腔护理。

(3)皮肤、毛发卫生

①新患者入院做好卫生处置,检查有无外伤、皮肤病、头虱、体虱等,及时对症处理。

②督促患者饭前便后洗手,每日按时洗脸、洗脚,女性患者清洗会阴。定期给患者洗澡、洗发、理发、剃须、修剪指甲。生活自理困难者,由护士帮助、代理。

(4)衣着卫生帮助患者保持衣着整洁,随季节变化关心帮助患者增减衣物。

(5)观察患者的排泄情况,及时处理便秘、排尿困难、尿潴留等情况。对大小便不能自理者,定时督促,保持衣裤、床单的干燥清洁。

2.饮食护理

(1)采用集体进餐,有助于患者消除对饭菜的疑虑,便于全面观察进食量、速度情况。餐室要光线明快、清洁整齐、宽敞舒适,有利于调动患者的进餐情绪。安排固定的座位,及时查对,不要遗漏。准备清洁消毒的餐具,餐前督促患者洗手。对需要特别管理的患者及特殊饮食的患者应事先安排好。如有家属探视,应在10~15分钟前停止会见,并请家属暂时离开病区。

(2)一般患者给予普食,特殊病情按医嘱给流质、高蛋白、少盐、低脂或无牙饮食等。对吞咽动作迟缓者,酌情为患者剔去鱼肉的骨刺,谨防呛食窒息。

(3)对抢食、暴食的患者应安置单独进餐,适当限制进食量,对症处置,谨防意外。

(4)对食异物的患者要重点观察,外出活动时需专人看护,严防吞服杂物、脏物等。

(5)对不愿进食、拒食的患者,针对不同原因,采取相应的措施,必要时鼻饲或静脉补液,并作进食记录。重点交班。

(6)会客时,向家属宣传饮食卫生知识,要关心家属所带食品是否卫生、适量,预防肠胃道疾病。

3.睡眠护理

(1)创造良好的睡眠环境:室内整洁,空气流通,光线柔和,温度适宜,环境安静,有利于安定患者情绪,使之易于入睡;床褥要干燥,清洁,平整;兴奋吵闹患者应安置于隔离室,并给予安眠处理,以免影响他人睡眠;工作人员做到“四轻”:说话轻、走路轻、操作轻、关门轻。

(2)安排合理的作息制度:白天除安排午睡外,要组织患者参加各种工、娱、体活动,以利夜间正常睡眠。

(3)做好睡眠时的生活护理:对生活自理能力差的患者应协助做好就寝时的一切生活护理。

(4)促进患者养成良好的睡眠习惯:向患者宣传睡眠与疾病的关系及睡眠的注意点;睡前忌服引起兴奋的药物或饮料;避免参加激情、兴奋的娱乐活动或谈心活动;不过量饮茶水,临睡前要解尿;睡前温水浸泡双脚;采取正确的睡眠姿势。

(5)加强巡视严防意外:要深入病室,勤查房,观察患者睡眠的姿势、呼吸声、是否入睡等。对有消极意念的患者要及时做好安睡处理,以防意外发生。

(6)未入眠患者的护理:分析失眠的原因,对症处理;体谅患者的痛苦与烦恼的心情;指导患者运用放松方法转移注意力帮助入眠,必要时遵医嘱给镇静催眠药。

4.安全护理

(1)和患者建立良好的护患关系,及时发现危险征兆:同情、关心、理解、尊重患者,及时满足患者的合理要求,使患者主动倾诉内心活动,做好心理护理,可避免意外事件发生。

(2)掌握病情,针对性做好防范:重视患者主诉,密切观察患者病情动态,对重症患者要安置在重病室内,24小时重点监护,以便及时发现不良预兆,严防意外发生,谨防意外。病情波动,及时记录与交班。

(3)严格执行护理常规与工作制度:护士应严格执行各项护理常规和工作制度、给药的护理、测体温护理、约束带护理、外出活动护理、交接班制度、岗位职责制度。

(4)加强巡视,严防意外:每10~15分钟巡视患者一次,仔细观察病情变化,定时查对患者人数,确保安全,在夜间、凌晨、午睡、开饭前、交接班等时段,病房工作人员较少的情况下,护士应特别加强巡视。厕所、走廊尽头、暗角、僻静处都应仔细察看。

(5)加强安全管理:病房设施要安全,门窗应随手关锁;病室内危险物品要严加管理,如药品、器械、玻璃制品、绳带,易燃物、锐利物品等,交接班时均要清点实物,一旦缺少及时追查。每日整理床铺时查看有无暗藏药品、绳带、锐利物品等;加强安全检查,凡患者入院、会客、请假离院返回,外出活动返回均需做好安全检查,严防危险品带进病室。每周1次对全病房的环境、床单位、患者个体做安全检查;凡是有患者活动的场所都应有护士看护,请假离院、出院时必须有家属陪伴。

(6)宣传和教育:重视对患者及其家属进行安全常识的宣传和教育。

\subsubsection{精神科患者的组织管理}

精神病患者的组织管理是精神科临床护理工作中的重要环节,是现代精神科病房科学管理的重要组成部分。做好患者的组织管理,能够调动患者的积极因素,改善护患关系,能维持病区良好秩序,有利于医疗护理工作的开展,促进患者康复。

1.坚持开放管理

(1)坚持开放管理,尽可能地让患者过正常化生活。精神病患者虽然有异常精神活动,但并不是完全丧失理智和难以管理,他们有正常的言行和心理需要,因此通过开放管理,发挥患者的特长和爱好,充分调动患者的主观能动性,组织患者学习和劳动,使患者摆脱疾病的困扰,促进恢复正常交往,有益于回归社会。

(2)坚持开放管理,尽可能使患者不脱离社会生活。鼓励患者关心国家大事,组织患者看书、读报、收听广播节目、收看电视新闻,组织患者参加适当的工娱疗活动,使患者与社会生活保持密切联系。

2.患者的组织管理

(1)患者的组织:在病区护士长领导下,由专职护理人员具体负责,帮助患者建立病室休养委员会和休养组织。通过委员会组织在患者中开展各项活动,充分调动患者的积极因素。

(2)患者的制度管理:制定患者作息制度、住院休养制度、探视制度、工休座谈会制度等,并宣传这些制度,使患者能够尽量自觉遵守。合理安排患者的作息制度,使其养成良好的生活习惯和行为;有计划地安排室内外工娱、体育活动与学习,丰富患者的住院生活。

(3)分级护理管理:住院患者实行三级护理制度。分级护理管理是根据患者病情的轻重缓急及其对自身、他人、病室安全的影响程度而采取不同的护理措施与管理方法。护理管理分为一级、二级、三级。

(4)患者的活动管理:根据患者病情及康复情况,实行三级开放制度。一级开放:患者活动范围局限于病区范围内;二级开放:患者可在医院范围内有组织地进行活动;三级开放:患者可自由出入病区。

\section{意外事件的防范与护理}

精神病患者由于精神症状的影响或严重的精神刺激等原因而出现各种意外事件,如暴力行为、自杀自伤行为、出走行为等。这些事件不仅对患者本身的健康和安全具有危害性,同时也会危及他人的安全和社会秩序,因此在精神科护理中占有十分重要的地位。

\subsection{暴力行为的护理}

暴力行为是精神科最为常见的意外事件,可能发生在家中、社区、医院等,会给患者、家人及社会带来危害及严重后果。暴力行为是基于愤怒、敌意、憎恨或不满等情绪,对他人、自身和其他目标所采取的破坏性攻击行为,可造成严重伤害或危及生命。表现为突然发生的冲动,可有自伤、伤人、毁物,以攻击性行为最突出。因此,需要对患者的暴力行为及时预测、严加预防和及时处理。

\subsubsection{暴力行为的预测因素}

1.人口学特征 年龄:年轻病人更容易发生暴力行为;性别:男性比女性更容易发生暴力行为;婚姻:单身病人发生暴力行为的可能性大;工作:失业使病人脾气不佳,容易产生暴力行为。

2.心理学特征 心理发展:早期的心理发展和生活经历,即长期经历过严重的情感剥夺、性格形成期暴露于暴力环境中的患者更易发生暴力行为;个性特征:多疑、固执、缺乏同情心和社会责任感;情绪不稳、易紧张、易产生挫折感;缺乏自尊与自信、应对现实及人际交往能力差。上述的性格特征可能与暴力行为有关,这类人的暴力行为发生率相当高。

3.精神疾病特征 精神分裂症病人,幻觉、被害妄想、敌意、不友善的态度,引起暴力行为;躁狂症病人,急性期时冲动、暴躁、缺乏耐性产生暴力行为;人格障碍病人,因人格处在极不稳定的状况,在病房中影响其他病患的情绪或想法,引起暴力行为产生;智障病人可能因智能不足引发情绪上的反应而出现暴力行为。

4.生物学特征 脑损伤因素、雄性激素水平升高、中枢5-HT功能低下等情况均能引起暴力行为的增加。

5.暴力行为史 个体受到挫折或受到精神症状控制时,是采取暴力行为还是退缩、压抑等方式来应对,与个体的应对方式有关。许多研究表明,既往暴力行为史是预测是否发生暴力行为的最重要预测因素。因此,习惯用暴力行为来应对挫折的个体最可能再次发生暴力行为。

\subsubsection{暴力行为的原因分析}

1.患者因素 患者受幻觉、妄想的支配,认为有人会对自身造成伤害而先发制人;不安心住院的患者强烈要求出院,不能满足时而出现暴力行为;有意识障碍患者出现无目的的暴力行为;患者之间因一些生活小节发生争吵,互不相让,易发生暴力行为;病房是个团体生活场所,若遇到病友的煽动或挑衅行为,容易互相影响情绪,或进一步触发暴力行为的产生。

2.患者家属因素 患者家属对患者的态度生硬,甚至指责谩骂患者;探视时将家里发生的不愉快的事情告诉患者,使患者情绪波动而出现暴力行为。

3.医护人员因素 医护人员在接触患者时由于语言不当、动作粗暴、嘲笑或虐待患者、对患者的正当要求不予满足,造成患者的反感,亦可诱发暴力行为;医护人员的个人特质,对暴力行为的期待或态度,及处理暴力行为的团队经验,均能影响工作人员面对暴力时的行为。

4.环境因素 患者生活在相对封闭的空间,认为医院铁门铁窗类似监狱,没有自由,因此心情烦躁而发生暴力行为;居住环境差、过分拥挤、缺乏隐私等易诱发暴力行为。

\subsubsection{暴力行为的预防}

1.评估患者细致 全面准确评估患者情况是防止暴力行为的基础。首先是入院评估内容,包括既往攻击行为史、精神症状、发病诱因、个性特征、自知力等。住院期间注重暴力行为的征兆评估。包括:先兆行为:踱步、不能静坐、握拳或用拳击物、下颚紧绷、呼吸增快、突然停止正在进行的动作;语言方面:威胁真实或想象的对象、强迫他人注意、大声喧哗、妄想性语言;情感方面:愤怒、敌意、异常焦虑、易激惹、异常欣快、情感不稳定。如出现上述情况,应高度警惕,严防暴力行为的发生。

2.安全制度落实 安全管理是防范暴力行为的保证。工作人员要充分认识到暴力行为的危害性,加强危险物品的管理,定期检查危险物品,严禁危险物品带入病房,消除安全隐患。值班时要坚守岗位,重点患者重点防范,加强巡视,注意巡视技巧。

3.掌握病情全面 全面掌握患者的病情,实施以实证护理为框架的护理。护士应全面了解患者病情变化的特点以及思想动态,对具有幻听、被害妄想、不协调性兴奋、易激惹、既往攻击行为史等预测暴力行为发生的高危因素的患者实施重点监护、重点观察、重点防范。

4.护患沟通有效 建立良好的护患关系,进行有效的护患沟通。要熟练掌握接触患者的技巧,尊重患者的人格,对患者要做到耐心、细心、温心,尽量满足患者的合理要求,把医源性暴力减少到最低程度。融洽的护患关系有利于处理各种矛盾,将暴力行为消灭在萌芽状态。

5.心理护理合适 心理护理是防止暴力行为发生的有效手段,尤其对敌对、猜疑、易激惹精神运动性兴奋症状突出的患者效果更为显著。运用启发、诱导、暗示等方式,耐心地解释、说服和安慰,创造良好的住院环境和氛围。

6.加强健康教育 对患者进行健康教育,让患者了解疾病的原因、症状、治疗、预后及预防,使患者认识疾病、安心住院、配合治疗;让患者学会控制情绪,分散注意力,转移暴力行为的方法,用正确的方式、方法来宣泄自己的情绪。

7.给予行为干预 要重视引导患者多参加集体活动、工娱疗活动,如下棋、打扑克、整理卫生等,既能丰富住院生活,又能分散注意力,消耗旺盛的精力,从而减少或避免暴力行为的发生。

\subsubsection{暴力行为的处理}

1.寻求帮助 当患者发生暴力行为时,首要且关键的一步要迅速呼叫其他工作人员的帮助,集体行动。

2.控制局面 一方面,转移被攻击的对象,疏散其他围观病友离开现场;另一方面,用简单、明确、直接的言语提醒患者暴力行为可能导致的后果,制止患者的行为,同时好言劝慰患者,答应患者的合理要求,尽可能说服患者停止暴力行为。

3.解除危险品 工作人员以坚定、冷静的语气告诉患者,将危险物品放下,并迅速将其移开。如果语言制止无效,一组人员转移患者的注意力,另一组人员乘其不备快速夺下危险物品。

4.隔离 在其他非限制性措施都无效时,需要将患者与其他病友分开,隔离在一个相对安全、安静的环境中,让其暂时脱离使其不安的人际关系,减轻其感官负荷,以防止其伤害自己和病友。

5.保护性约束 如果上述措施均无法控制患者的行为,则需要采取保护性约束。在接近患者前,要保证有足够的工作人员,每人应该负责患者身体的一部分,接触患者身体要果断迅速,多人行动要协调。约束时效率要高,注意不要伤害患者。

\subsection{自杀行为的护理}

自杀在精神科急诊常见,抑郁症、精神分裂症、脑器质性精神障碍及病态人格等都易出现自杀观念和行为。自杀的原因复杂,表现形式多种多样,绝大多数患者自杀前会暴露出一些自杀迹象,因此应严格观察病情,识别出有自杀企图、自杀意念的患者,采取适当措施防止患者自杀成功。

\subsubsection{精神科自杀的常见原因}

1.抑郁症病人自杀率最高 严重的抑郁情绪、顽固而持续的睡眠障碍、有自罪妄想和严重的自责、情绪紧张或激越、有抑郁和自杀家族史的患者,感到度日如年、生不如死,导致自杀以求解脱。

2.精神分裂症病人自杀常发生在疾病的早期 患者体验到自己的人格变化,引起内心的焦虑,受到幻觉、妄想的支配而出现自杀行为。由于思维内容障碍出现各种妄想,许多妄想可导致患者出现自杀企图和行为,如罪恶妄想、被害妄想、自责妄想等。幻觉,如命令性幻听,患者听到要他死的指令,他就执行这一命令而自杀。患者在意识模糊或错乱状态下,出现大量错、幻觉,而引起冲动性自杀行为。

3.精神分裂症病人大多发生在刚出院的缓解期 恢复期精神病患者感到病后行为能力有较大的破坏;知道自己是精神分裂症病人,或者害怕成为这种病人时,希望回避这种命中注定的结果;对疾病缺乏正确认识,认为病情给自己带来极大损失,看不到前途和生活的希望;出院后就突然断药,且得不到社会尤其是家庭的支持;不能正确面对自身的疾病,不能承受社会、家庭对他们的压力,不能承受得病后造成的学习、事业和经济上的重大损失,家庭离散、生活失去了目标等,因此产生自杀观念和行为。

4.神经症或主观失眠的患者感到十分痛苦而焦虑,坐卧不安,无法摆脱而自杀。

5.做态性自杀 多为神经官能症患者,以自杀手段减轻其心理上的压力,吸引他人注意,一旦失手便假戏真做。

6.严重药物反应和药源性抑郁 由于药物反应严重而难以忍受,或出现药源性抑郁状态,表现为焦虑、烦躁、消极悲观、自责自罪、自伤自杀等。

\subsubsection{精神科自杀的预防}

1.提供安全住院环境 患者生活的环境中应杜绝自杀工具,如刀、绳、玻璃、药物、有毒物品等。生活设施应安全,所有通向阳台或室外的门应随时关门上锁。办公室不得让患者随便出入,以防意外。室内电源、电路要设在墙壁内或较高处,并经常检查是否安全。教育探视者不要带给患者任何危险物品。

2.加强对患者的管理 对有自杀危险和自杀先兆表现的患者要置于护士易觉察的范围,加强巡视,且巡视病房的时间不能刻板固定,防止患者掌握规律,有机可乘。对高度自杀危险者进行一对一的守护。对新入院及请假离院返回的患者要认真检查,防止各种危险品带入病房,并严格交接班。注意观察患者的睡眠情况,对蒙头睡觉的患者应劝其将头露在被外以便于观察。对于睡眠差的患者要注意其行动。护士应密切观察患者情绪的变化,及时识破假象。

3.积极有效的心理干预 护士要耐心倾听患者的诉说,关心、同情、理解、尊重患者,了解其感受,给予支持性心理护理,并为其提供希望;医护密切配合,加强对患者的心理治疗,通过谈心做深入细致的思想工作,告知患者现在的痛苦是暂时的,通过治疗可获得好转,使之感到医务人员能够了解和分担他的痛苦,使之消除其悲观消极情绪;充分动员和利用社会支持系统,帮助患者战胜痛苦,增加对抗自杀的内在和外在动力;鼓励患者参加各种文娱活动,使其保持乐观愉快的情绪;鼓励病人正确对待各种矛盾,树立崇高的人生观,增强战胜疾病的信心。

4.了解患者病情变化 观察和记录患者心理活动、精神变化,予以适当的处理。严重自杀企图者应专人监护,形影不离,严禁单独活动,必要时予以保护性约束。

5.掌握自杀发生规律 自杀发生频率最高的时间是午夜之后,清晨起床后、中午休息时间和就餐时也是患者常自杀的时间。因为在这些时间段工作人员往往较少,或易麻痹松懈,所以应提高警惕,严加防范。

6.保证各种治疗及时 患者遵守医嘱服药,发药时要及时检查口腔,使之能够保证服下药物,严防藏药后一次服用;对可能导致意外的危险症状进行积极有效的治疗,如自杀意念严重,可以建议医生给予电休克治疗,以尽快消除自杀意念和自杀行为。

\subsubsection{常见自杀方式的紧急处理}

1.自缢 自缢是精神科常见的自杀手段,即使严加防范,有时患者仍会付诸行动。一旦发生自缢,不要离开现场,要抓紧时机;立即抱住患者身体向上抬举,解除颈部受压迫状态。如患者在低处勒缢,应立即剪断绳索,脱开缢套;将患者就地平放,松解衣扣和腰带,立即进行口对口人工呼吸和胸外心脏按压术,直至自主呼吸恢复后再搬移患者,做进一步的复苏治疗处理。

2.触电 一旦发现患者触电,要迅速切断电源,救护者不可直接用手接触带电人体。当找不到总电源时,可穿上胶鞋,用绝缘物体如被服类套住触电人体,牵拉患者脱离电源;意识清醒者,就地平卧休息,解松衣服,抬起下颌,保持呼吸道通畅;心跳呼吸停止者,立即进行口对口人工呼吸和胸外心脏按压,直至复苏有效指征出现,进一步治疗。

3.溺水 精神病患者在强烈的自杀欲望支配下,可将头或上半身没入洗手池,或寻觅机会跳入水池、浴池、水湾等处,以求自杀死亡。一旦发现患者溺水,应立即将患者搬离水面,解开领口腰带,摘除假牙,清除口鼻中的污物,保持呼吸道通畅,迅速清除呼吸道和上消化道的积水;如患者仍窒息,立即将其放平,同时进行口对口人工呼吸及胸外心脏按压,并酌情注射中枢兴奋药,给予吸氧等措施;注意保暖,去除患者身上的湿衣,裹以棉被等,促进血液循环和体温回升。

4.服毒 患者匿藏大量精神科药物或镇静安眠药,集中吞服,蓄意自杀。一经发现首先评估患者的意识、瞳孔、肤色、分泌物、呕吐物等;初步判断所服毒物的性质及种类;应迅速排除毒物,可采取催吐、洗胃导泻等方法,根据毒物性质采取不同的解毒措施;配合进一步的治疗抢救。

5.吞服异物 首先安慰吞服异物者,并检查有无口腔外伤、腹痛、内出血、柏油样便等。其次根据异物性质和大小,采取不同的措施。如系较小的异物多可从肠道排出;如为锐利物品,表面比较光滑,可让患者服用大量高纤维食物,使蔬菜纤维缠绕异物,迅速随粪便排出,不损伤胃肠黏膜;如系金属异物,应进行X线检查,确定异物所在位置,判断异物能否自行排出,如异物较大,不可能从肠道排出,应采用外科手术取出异物。最后,严密观察异物排出情况,患者大便应排在便盆内,仔细查找异物排出情况,直至异物全部排出,并详细记录交班。

\subsection{出走行为的护理}

出走行为是指患者私自突然离开家庭、单位或医院。这里所讨论的仅指患者在住院期间,未经医务人员批准,私自离开医院的行为。患者在住院期间可利用各种机会出走,如尾随工作人员或乘工作人员开门之机夺门而走,骗取工作人员信任乘外出活动之机出走。患者走出医院后,可能发生事故或影响社会治安,因此必须严格防止。

\subsubsection{出走的原因}

1.自知力缺乏 否认有病,对住院反感,千方百计想逃离医院。

2.受幻觉妄想支配 最常见的是有迫害性内容的幻觉和妄想,患者为了躲避迫害而离院出走。

3.对住院环境不适应 感到住院烦闷、不自由、受到限制,或住院时间较长而想念家庭和亲人,或对电休克等治疗方法感到恐惧等。

\subsubsection{出走的临床表现和方式}

出走患者在病史中可能有漫游、出走的情况。患者在出走前,多数会有异常表现,有的焦虑不安、徘徊不止、东张西望,经常站在大门口;有的表现为不眠或少眠;有的换穿自己的衣服在外出活动时乘机出走。

出走的方式多为隐蔽,常寻找不牢固的门窗而出走,或故意在病室门口附近活动,乘工作人员或患者家属出入时,从门口溜走;乘外出活动或检查时伺机出走;也有部分患者由于精神错乱明显,出走无计划、无目的,不讲方式,想走就走,这样的患者出走成功机会较少,但一旦成功,后果较严重,危害性大。

\subsubsection{出走的预防和护理}

1.详细了解病史资料,严密观察病情变化,对病史中有出走倾向的患者要重点观察和接触。

2.了解出走的想法和原因,开展心理疏导,帮助解决问题,如请家属来院探视。

3.做好病房安全管理工作,及时清除不安全因素,如及时修理损坏的门窗等。大门设专人监护,保管好病室的钥匙,发现丢失,应立即追查。

4.工作人员要加强巡视病房,对出走欲望强烈的患者,应安排在工作人员的视线范围内,避免患者在门口、窗口活动,同时做好严格的交接班工作。

5.安排患者外出活动和检查时,要加强观察,注意每个患者的动向,安排好护送管理人员,有组织地进行。

6.改善服务态度,加强心理护理,满足患者的合理要求,避免刺激性的言语,使患者能安心住院。

\subsubsection{出走后的护理}

1.一旦发生出走,要沉着、冷静,立即通知其他人员,并与家属联系,同时,分析与判断患者出走的时间、方式、去向,组织人力外出寻找。

2.当找到患者时,应婉言劝其返院,如患者拒绝可进行保护性约束送返回院。

3.出走患者返院后,应安排适当休息,加强心理护理,让患者讲述出走的原因和经过,以便进一步制定防范措施。

\section{精神异常状态的护理}

同样的精神异常状态可见于不同类型精神障碍的临床表现,如幻觉、妄想、意识障碍等。因此,精神异常状态护理有共同特征,可适用于相应的精神障碍护理,作为整体护理的组成部分。

\subsection{躁狂状态患者的护理}

躁狂状态的患者表现心境高涨,思维奔逸,动作增多。这种兴奋状态属于协调性兴奋,患者的举止言谈富有感染力,语量增多,滔滔不绝,随境转移,难以安静,在病房里易滋生事端。主要的护理措施如下:

1.提供安静的病室环境,室内陈设简单,光线柔和,避免噪音以减少刺激,减低患者的兴奋性。对急性期患者应限制活动范围,置于工作人员视线范围内,以保证安全。

2.护理人员要尊重患者,耐心倾听其叙述,建立良好的护患关系,稳定患者情绪。

3.密切观察患者的病情变化动态,注意突发的激情冲动和攻击性行为。对此,护理人员要沉着冷静地处理,用温和的语言进行劝阻,保证患者和其他人的安全。设法转移患者的注意力,缓和情绪。

4.对忙碌不休、难以安静的患者,可引导他们在室内进行简单可行的工娱活动,如手工叠纸等,分散其注意力,缓和其兴奋状态。

5.及时隔离兴奋患者,将其安置于重症室,加强巡视和看护,必要时给予保护性约束。

\subsection{抑郁状态患者的护理}

抑郁状态的患者主要表现为心境抑郁,在此基础上,可出现焦虑、易激动、激惹,对生活悲观失望,无信心,自卑感,能力下降。患者的精神运动性抑制,表现为思维迟滞,行动缓慢,言语少,声调低,严重者不语不动,卧床不起,拒食,常可发生躯体并发症,甚至出现强烈的消极观念和自杀行为。主要护理措施如下:

1.将患者安置在重症监护室,有严重自杀倾向者应安排专人看护。做好各项安全检查工作,排除一切危险物品。

2.建立良好的护患关系,接触患者时态度和蔼,要关心体贴患者,主动接触患者。交流时要注意技巧,言语恰当,加强理解患者的内心情感体验,帮助患者消除自卑和无能的心理状态,化解内心矛盾。鼓励患者树立对生活的信心和勇气。

3.严密观察患者的言语、动作和行为表现,以及非言语的情感反应,早期发现病情动态先兆。抑郁状态有昼重夜轻的变化规律,尤其在清晨或工作忙碌的时候应密切注意、加强护理,不给患者可乘之机,严防自杀行为。

4.洞察患者反常的情感变化,如果抑郁患者一反常态,情绪突然开朗,积极主动地与他人交往,在病室里表现活跃,这种突变可能是患者企图蒙骗他人的伎俩,实现其自杀的目的。

5.在病情缓解期要加强心理护理,使患者宣泄内心积郁,并指引积极的行为。询问患者自杀的目的、动机等,淡化患者的自杀意念,从而引导其建立正确的社会行为,化消极因素为积极因素。

\subsection{妄想状态患者的护理}

妄想状态患者的意识清晰,基本上能自理生活,但无自知力,对其妄想内容坚信不疑。妄想内容因人而异、种类多样,临床上多见于精神分裂症。主要护理措施如下:

1.入院时的护理接触 妄想状态患者入院时,尤其要注意服务态度和质量,因为此类患者在病态思维的支配下,常认为住院是“受迫害”,对医务人员怀有敌意。因此,护理人员的态度要和蔼、亲切,言语恰当,服务周到,关心照顾生活,以满足心身需求,缓和其情绪,使其安心住院。

2.与患者的交流技巧 患者对其妄想内容十分敏感,不愿暴露。护理人员与患者交流时,要掌握病情,注意技巧,不可贸然触及其妄想内容。如患者主动叙述,要注意倾听,不可与其争辩,也不能表示同意。如患者回避不谈,则不必追问,以免引起反感,要建立相互信赖的护患关系。

3.有被害妄想患者的护理 如患者有被害妄想而拒食,应鼓励患者集体进餐,与病友吃同样的饭菜,以减轻患者的疑虑。经常关心患者,站在患者身旁,使其有安全感。

4.有自罪妄想患者的护理 有自罪妄想的患者,常在病房里无休止地参加劳动,自称借以赎罪;或认为自己有罪不配吃饭,专拣食剩饭剩菜,或食脏物。对此,护理人员应主动监护、关心照顾其生活,耐心劝阻他们。保证患者正常进食,预防感染,并防止过度的体力消耗,影响健康,不利于治疗。

5.有关系妄想患者的护理 对有关系妄想的患者,切忌在他们面前低声与他人耳语,以免引起怀疑,影响护患关系和病友间的关系。

\subsection{幻觉状态患者的护理}

幻觉常出现于精神疾病的急性期,在幻觉症状的支配下,患者常可发生意外行为,这种情况是护理的重点。主要护理措施如下:

1.观察幻觉征兆的技巧 护理人员要掌握观察患者出现幻觉征兆的技巧,才能及时发现病情变化,如幻觉的内容、发生的频率和时间,采取护理干预。患者言语的和非言语的动作、姿势和情感反应,如某患者全神贯注,端坐侧耳倾听,面部表情时而欣快,时而愤怒、焦虑不安,时而自语,时而大声谩骂等行为表现,均提示幻觉的出现。

2.幻觉状态患者的护理 与患者建立相互信任的关系,鼓励其说出幻觉的内容。如某女患者对护士说:“我听见我孩子在窗外哭喊找妈妈啊!”护士平静地回答:“院内非常安静,我没有听见哭喊声音”。护士如此回答,目的是诱导患者理解幻听是病态声音,而实际并无人声。此时,护士可陪同患者去院内散步,查寻真相,以缓解患者的情绪。在适当时机,对其病态体验提出合理解释,并教会患者在出现幻觉时的应对方法或主动找医护人员帮助。

\section{抗精神病药物不良反应的护理}

抗精神病药物的不良反应常可发生,其轻重程度因人而异。为了保证治疗效果,护理人员要掌握抗精神病药物的基本知识和护理。其主要不良反应的护理干预如下:

1.吞咽困难的护理 精神药物引起咽喉肌群失调,发生吞咽困难,导致咳呛或噎食,是危及患者生命的原因之一。要加强饮食护理,小心缓慢喂食,给予半流质饮食,必要时进行鼻饲或输液。

2.便秘和尿潴留的护理 虽是一般的不良反应,但患者极为痛苦,老年患者尤应注意。有的患者缺乏主诉,常因躯体不适、烦躁不安,加重病情。护理人员要加强观察,及时发现问题,给予处理,保持大小便通畅,解除患者的痛苦。

3.直立性低血压的护理 这是服酚噻嗪类药物或三环抗抑郁药物常见的不良反应。患者行走或体位改变时,突然直立摔倒,血压下降,不省人事。服用此类药物时,应嘱患者服药后休息片刻再活动,改变体位或起床时动作要缓慢,夜间尤应注意。患者如有眩晕、心悸、乏力等不适感,要立即坐下或卧床,并告知医生或护理人员。患者发生直立性低血压时,突然直立摔倒,面色苍白,出冷汗,测血压低于80/65mmHg(10.6/8.7kPa),甚至测不到。应立即将患者就地平卧,不可挪动,取头低脚高位,立即进行护理抢救工作,检测生命体征变化,准备好抢救药品和器械。

4.皮炎的护理 药物性皮炎是精神药物引起的过敏反应所致,严重者可发展为剥脱性皮炎。服用酚噻嗪类药物的患者,如在阳光下暴晒可引起日光性皮炎。药物性皮炎多发生在治疗初期,多为点状红色斑丘疹。发生的部位最初以面部和背部为主,以后波及四肢和全身。在临床上注意早期发现异常情况,及时处理,以防病情发展。

5.恶性综合征的护理 使用高效价抗精神病药或多种药物联合使用时,可引起此种罕见的严重不良反应。护理人员应掌握病情特征,早期识别症状。善于观察症状是做好本病护理的关键,如严重的锥体外系症状、发热、心动过速、尿潴留等。严重时体温可骤升至40℃以上,高热持续不退,大汗淋漓,脱水,意识障碍,呼吸循环衰减,血压下降等,应按重症患者进行对症护理。

6.粒细胞缺乏症的护理 以服用氯氮平类药物为多见。应注意发现早期临床症状,如起病急骤,高热畏寒,咽痛乏力等,同时要密切关注白细胞化验结果。对严重粒细胞缺乏症患者,要实行保护性隔离措施,加强对症护理,严防继发感染。

7.锂盐中毒的护理 在患者服用锂盐初期,应注意早期发生的不良反应,如恶心、呕吐、腹泻、口渴、尿多、细颤等。要加强饮食护理,保证入量。如有严重的呕吐、腹泻、脱水现象,应予补充食盐量,每日摄入量不得少于3g。同时要关注血锂浓度的化验结果(正常值为1.6mmol/L)。锂盐的治疗量与中毒量极为相近,因此,如发现早期中毒症状,如细颤变为粗颤、眩晕、共济失调等,要及早处理,才能保证疗效。