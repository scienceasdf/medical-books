\chapter{参考文献}

1 李果珍主编.临床CT诊断学.北京:中国科学技术出版社,1994

2 陈星荣,沈天真,段承祥,主编.全身CT和MRI.上海:上海医科大学出版社,1994

3 李松年主编.现代全身CT诊断学.北京:中国医药科技出版社,2002

4 吴恩慧主编.头部CT诊断学.北京:人民卫生出版社,1995

5 沈天真,陈星荣,主编.中枢神经系统计算机体层摄影(CT)和磁共振成像(MRI).上海:上海医科大学出版社,1992

6 周康荣主编.胸部颈面部CT.上海:上海医科大学出版社,1996

7 周康荣主编.腹部CT.上海:上海医科大学出版社,1999

8 白人驹主编.医学影像诊断学.北京:人民卫生出版社,2001

9 刘树伟主编.断层解剖学.北京:高等教育出版社,2004

10 James C.Reed著,程家文译.胸部放射学.上海:同济大学出版社,1992

11 陈志刚主编.关节病影像诊断学.西安:陕西科学技术出版社,1999

12 李吉昌主编.泌尿男性生殖系统影像诊断学.济南:山东科学技术出版社,2000

13 郭宝华,毕万利,主编.眼科影像诊断学.济南:济南出版社,2001

14 徐爱德,徐文坚,刘吉华,主编.骨关节CT和MRI诊断学.山东科学技术出版社,2002

15 周康荣主编.螺旋CT.上海:上海医科大学出版社,1998

16 孟庆学主编.实用CT诊断问答.北京:中国医药科技出版社,2006

17 朱珍,帕米尔,朱杰明,等.Joubert综合征的CT和MRI诊断.中华放射学杂志,2005,39(12):1256

18 范晓颖,肖江喜,唐光健,等.神经元移行异常与癫痫.临床放射学杂志,2003,22(3):183

19 Hoeffner EG.Cerebral perfusion imaging.J
Neuroophthalmol,2005,25(4):313

20 胡洪斌.分隔型慢性硬膜下血肿的CT诊断.放射学实践,2001,16(5):313

21 张云亭.WHO中枢神经系统肿瘤分类对于影像学研究的指导意义.国际医学放射学杂志,2008,31(3):156

22 孔凡彬,熊维亚,张春宁.牵牛花综合征的CT诊断.中华放射学杂志,2000,34(11):774

23 Shields JA,Bakewell B,Augsburger JJ,et al.Space-occupying
orbital masses in children.A review of 250 consecutive
biopsies.Ophthaimology,1986,93:384

24 李玉花,巩若箴.颞骨内段面神经的影像学检查技术研究进展.医学影像学杂志,2006,16(11):1207

25 Fisher NA,Curti HD.Radiology of congenital hearing
loss.Otolaryngol Clin North Am,1994,27:511

26 朱俭,温志波,段承祥,等.鼻窦解剖变异的CT观察.临床放射学杂志,2005,24(5):395

27 杨智云,钟运其,张翎,等.嗅神经母细胞瘤的CT和MRI表现.中华放射学杂志,2005,39(3):244

28 Weber BP,Dillo W,Dietrich B,ea tl.Pediatric Cochlear
Implantation in Cochlear Malformations.Am J Otol,1998,19:747

29 杨本涛,王振常,于振坤,等.翼腭窝原发肿瘤的CT和MRI诊断.中华放射学杂志,2003,37(10):922

30 李威,张宇捷,姜英健,等.颌骨牙源性囊性病变的CT表现.临床放射学杂志,2001,20(7):496

31 刘连生,李恒国.腮腺病变的影像分析.临床放射学杂志,2006,25(10):913

32 全宏卫,曾鹏.多层螺旋CT对腮腺Warthin瘤的诊断价值.医学影像学杂志,2008,18(8):842

33 文利,戴书华,张冬,等.鳃裂囊肿的CT诊断.医学影像学杂志,2003,13(2):80

34 Ahuja AT,Ying M.Sonographic evaluation of cervical lymph
nodes.AJR,2005,184(5):1691

35 杨智云,孙木水,钟运其,等.头颈少见部位副神经节瘤.中华放射学杂志,2003,39(4):409

36 Na DG,Chung TS,Byun HS,et al.Kikuchi disease:CT and MR
findings.AJNR,1997,18:1729

37 程杰军,吴华伟,编译.气管-主支气管局灶性和弥漫性异常的CT诊断.国外医学临床放射学分册,2003,26(3):192

38 武宜,综述.小气道疾病的HRCT影像特点.国外医学临床放射学分册,2005,28(6):397

39 姜蕾编译.特发性间质性肺炎的国际新分类.国外医学临床放射学分册,2004,27(3):149

40 常恒.AIDS的胸部影像学表现.国外医学临床放射学分册,2002,25(6):351

41 李铁一.肺炎不同转归的影像学表现.中华放射学杂志,1998,32(8):571

42 程晓光,冯素臣,夏国光,等.SARS的胸部CT早期表现.中华放射学杂志,2003,37(9):790

43 Nakata M,Saeki H,TakataI,et al.Focal ground-glass opacity
detected by low-dose helical CT.Chest,2002,121:1464

44 王晓华,马大庆.孤立性肺结节的影像学研究进展.临床放射学杂志,2006,25(6):571

45 袁涛,于铁链.非典型肺转移瘤的放射学表现.国外医学临床放射学分册,2003,26(3):190

46 聂永康,马大庆,李铁一.弥漫性肺疾病支气管血管束高分辨率CT表现及其病理基础.中华放射学杂志,2000,34(7):464

47 薛敏娜,潘纪戍.嗜酸性肺病的影像学表现.国外医学临床放射学分册,2003,26(1):30

48 周建军,周康荣,曾蒙苏,等.孤立性纤维瘤的影像学诊断和鉴别.医学影像学杂志,2008,18(8):851

49 张优仪,余建群.冠状动脉粥样硬化斑块的影像学评价.医学影像学杂志,2008,18(2):187

50 Kimura F,Shen Y,Date S,et al.Thoracic aortic aneurysm and aortic
dissection:new endoscopic mode for three-dimensional CT dispiy of
aorta.Radiology,1996,198:573

51 李子川,黄连军,杨剑,等.主动脉壁内血肿的影像学诊断.临床放射学杂志,2004,23(1):49

52 程悦,沈文,祁吉.肝动静脉分流的多层螺旋CT检查.国外医学临床放射学分册,2007,30(4):261

53 熊燕,周翔平.小肝癌的影像学检测新技术.临床放射学杂志,1999,18(7):442

54 李绍林,张雪林,陈燕萍,等.肝内周围型胆管细胞癌CT和MRI诊断及病理基础研究.中华放射学杂志,2004,38(10):1072

55 Fong JA,Ruebner BH.Primary leiomyosarcoma of the liver.Hum
Pathol,1974,5:115

56 何志明.肝脏局灶性结节增生CT诊断.医学影像学杂志,2006,16(11):1162

57 孟晓春,单鸿,朱康顺,等.Budd-Chiari综合征多层CT动态增强扫描及CT血管成像分析.中华放射学杂志,2005,39(6):652

58 阳红艳,许乙凯.胆囊腺肌瘤病影像学诊断基础及现状.国外医学临床放射学分册,2005,28(3):157

59 郑晓琳,宁永见,王承缘.胆胰管十二指肠连接区小肿瘤CT诊断及鉴别.临床放射学杂志,2000,19(12):777

60 吴仁民,包宏伟,陈强,等.胆囊炎与厚壁型胆囊癌的CT鉴别诊断.医学影像学杂志,2005,15(2):125

61 王宗盛,周胜利,李树芸,等.成人原发性肝脏肉瘤的螺旋CT诊断.临床放射学杂志,2007,26(3):266

62 陈韵彬,Hoeffel,李铭山.假肿瘤性胰腺炎的CT表现.临床放射学杂志,2000,19(11):710

63 Wakabayashi T,Kawaura K,Satomure Y,et al.Clinical and imaging
features of autoimmune pancreatitis with focal pancreatic swelling or
mass formation:comparison with so-called tumor-forming pancreatitis and
pancreatic carcinoma.Am J Gastroenterol,2003,98:2679

64 苏平,向如意,焦锐,等.胰腺癌的CT诊断.中国医学影像技术,1999,15(12):970

65 钱银锋,余永强.胰腺肿瘤的影像学.国外医学临床放射学分册,2005,28(3):160

66 鲍润贤,孙鼎元.胰腺囊性病变的影像学诊断现状.国外医学临床放射学分册,2007,30(4):217

67 李邦国,穆贵勇,范其文.脾脏占位性病变的CT诊断.医学影像学杂志,2004,14(12):1007

68 韩长利,代景儒,杨庆彦,等.几种少见脾病变的CT诊断.中国医学影像技术,1999,15(1):60

69 Kalovidouris A,Pissiotis C,Pontifex G,et al.CT characterization
of multivesicular hydatid cysts.JCAT,1986,10:428

70 苑任,韩萍,史河水,等.黄色肉芽肿性肾盂肾炎的CT诊断和鉴别诊断.临床放射学杂志,2001,20(9):681

71 王爱辉,柳逢春,王胜林,等.三期增强扫描在肾结核CT诊断中的价值.临床放射学杂志,2007,26(12):1304

72 唐永华,谢吉,周建勤,等.腺性膀胱炎的影像学诊断.中华放射学杂志,2000,34(1):55

73 臧建,龚小龙,杨根东,等.髓质海绵肾的CT诊断.医学影像学杂志,2007,17(8):812

74 郁成,陈永强,罗泽斌.乏脂肪肾血管平滑肌脂肪瘤与肾细胞癌的CT鉴别诊断.临床放射学杂志,2007,26(11):1119

75 Pickhard PJ,Siegel CL,Mclarney JK.Collecting Duct Carcinoma of
the Kidney Are Imaging Findings Suggestive of the Diagnosis?
AJR,2001,176:627

76 李莹,彭东红,张海,等.综合影像学检查对肝肾间隙巨大占位性病灶的定位诊断.放射学实践,2001,16(5):295

77 文利,孙清荣,张冬,等.肾上腺肿瘤的CT诊断.临床放射学杂志,2003,22(4):307

78 杨磊,王东,陈军,等.肾上腺少见肿瘤的CT、MR评价.医学影像学杂志,2005,15(8):682

79 王夕富,白人驹,王蒿,等.肾上腺腺瘤和非腺瘤动态增强CT表现与血管生成相关性的初步研究.中华放射学杂志,2005,39(8):864

80 闫旭,沈文,李笑,等.多层螺旋CT在胃肿瘤诊断中的应用研究.临床放射学杂志,2002,21(8):609

81 Fleiter T,Brambs HJ.Possibilities of virtual endoscopy.Schweiz
Rundsch Med Prax,1999,88:65

82 靳勇,张华,吴达明,等.胃肠道淋巴瘤的多层螺旋CT影像学分析.临床放射学杂志,2006,25(10):928

83 龚静山,杨鹏,徐坚民,等.胃肠道间质瘤的CT和MRI诊断.临床放射学杂志,2008,27(1):62

84 易文中,李维金.小肠肿瘤螺旋CT诊断的进展.医学影像学杂志,2008,18(1):95

85 董旦君,章士正.嗜酸性肠炎三例.中华放射学杂志,2003,37(5):476

86 霍福涛,冷天罡,白人驹,等.急性肠缺血的CT诊断.国外医学临床放射学分册,2004,27(6):373

87 唐肇普,白人驹.穿孔性与非穿孔性阑尾炎的CT鉴别诊断价值.临床放射学杂志,2004,23(2):135

88 Russell N.Low,Sloane C.Chen,Robert Barone.Distinguishing benign
form malignant bowel obstruction in patients with malignancy:Finding at
MR imaging.Radiology,2003,228:157

89 齐滋华,徐惠,李传福.肠系膜脂膜炎的影像学表现.医学影像学杂志,2003,13(5):355

90 袁涛,于铁链.小肠系膜根:病变的解剖和CT表现.国外医学临床放射学分册,2002,25(4):224

91 秦将均,巫北海.肾周间隙周界的研究进展.国外医学临床放射学分册,2002,25(5):290

92 吴天,张翔,姜毅,等.原发性腹膜后肿瘤的CT表现.临床放射学杂志,2005,24(2):182

93 Tralce L,Antonelli A,Dotti P,et al.Epidemiology,clinical
features and treatment of Idiopathic retroperitoneal fibrosis:our
experiencec.Arch Ital Urol Androl,2004,76:135

94 阎建华,吴天,李传福.腹膜后纤维化的影像学诊断.医学影像学杂志,2006,16(5):529

95 袁明远,刘光华.腹部实质脏器创伤的CT诊断.国外医学临床放射学分册,2003,26(6):391

96 袁明远,刘光华.腹部空腔脏器创伤的CT检查技术和诊断.国外医学临床放射学分册,2004,27(6):380

97 徐洪恩,吴恩福.睾丸肿瘤的CT诊断.医学影像学杂志,2007,17(1):65

98 吕明权.宫颈癌术后复发的CT诊断.临床放射学杂志,2000,19(1):47

99 吕明权,蒋丽娜,李贤兴,等.侵袭性葡萄胎的CT诊断.中华放射学杂志,2000,34(12):850

100 徐万峰,徐希见,秦风香.黏液性与浆液性卵巢肿瘤的CT鉴别诊断.医学影像学杂志,2003,13(11):823

101 冯峰,陈午才,夏淦林,等.盆腔少见恶性肿瘤的CT诊断.临床放射学杂志,2006,25(2):150

102 王晓红,单鸿,姜在波,等.输卵管妊娠的CT表现和特点.中华放射学杂志,2004,38(6):640

103 孟令惠,刘怀军.盆腔静脉淤血综合征及其影像学表现.国外医学临床放射学分册,2005,28(6):424

104 朱振祥,吴利忠,姜永生.阴囊闭合性损伤的CT诊断价值.中华放射学杂志,2000,34(8):538

105 冯素臣,裴京哲,卢艳丽,等.儿童髋关节骨化过程的CT表现.临床放射学杂志,2003,22(7):575

106 刘旭林,周承涛,杜玉清,等.成人髋臼发育不良的CT测量评价.中华放射学杂志,2000,34(1):52

107 刘吉华,徐爱德,汪敬群,等.髂腰肌囊扩张的影像学诊断.中华放射学杂志,2003,37(2):140

108 高振华,刘吉华,孟悛非,等.股骨颈疝窝的影像学研究.中华放射学杂志,2005,39(5):531

109 Wittram C,Whitehhouse GH,Williams JW.A comparison of MR and CT
in suspected sacroiliitis.Journal of computer Assisted
Tomography,1996,20:68~72

110 柳祥庭,刘实,庄伯秦.附丽病变在骨关节疾患的诊断价值.中国医学影像技术,1991,7(4):34

111 程英升,钟烽为,李明华.腰椎间盘退行性病变的影像学评价.国外医学临床放射学分册,1996,19(4):194

112 孟增东,裴福兴.腰椎管狭窄的影像学检查.临床放射学杂志,2002,21(12):997

113 崔建岭,王溱.椎间盘炎的研究进展.国外医学临床放射学分册,2001,24(2):97

114 王振豫,李树新,阎守芳,等.腰椎椎间静脉压迫症的CT诊断.临床放射学杂志,1996,15(5):297

115 许骅,耿道颖,王蒿.脊髓肿瘤与非肿瘤性病变的CT和MR诊断.国外医学临床放射学分册,2003,26(3):146

