\PassOptionsToPackage{unicode=true}{hyperref} % options for packages loaded elsewhere
\PassOptionsToPackage{hyphens}{url}
%
\documentclass[
  12pt,
  twoside,
  banjiao]{ctexbook}
\usepackage{lmodern}
\usepackage{amssymb,amsmath}
\usepackage{ifxetex,ifluatex}

\usepackage{unicode-math}
\defaultfontfeatures{Scale=MatchLowercase}
\defaultfontfeatures[\rmfamily]{Ligatures=TeX,Scale=1}

% use upquote if available, for straight quotes in verbatim environments
\IfFileExists{upquote.sty}{\usepackage{upquote}}{}
\IfFileExists{microtype.sty}{% use microtype if available
  \usepackage[]{microtype}
  \UseMicrotypeSet[protrusion]{basicmath} % disable protrusion for tt fonts
}{}

\usepackage{xcolor}
\usepackage{subfig}
\usepackage{xurl} % add URL line breaks if available
\usepackage{bookmark}
\usepackage[version=4]{mhchem}
\usepackage{hyperref}
\usepackage{placeins}
\hypersetup{
  pdftitle={临床心电图详解与诊断},
  pdfauthor={github},
  pdfborder={0 0 0},
  breaklinks=true,
  bookmarksdepth=5}
\urlstyle{same}  % don't use monospace font for urls
\usepackage{longtable,booktabs}
% Allow footnotes in longtable head/foot
\IfFileExists{footnotehyper.sty}{\usepackage{footnotehyper}}{\usepackage{footnote}}
\makesavenoteenv{longtable}
\usepackage{graphicx,grffile}
\makeatletter
\def\maxwidth{\ifdim\Gin@nat@width>\linewidth\linewidth\else\Gin@nat@width\fi}
\def\maxheight{\ifdim\Gin@nat@height>\textheight\textheight\else\Gin@nat@height\fi}
\makeatother
% Scale images if necessary, so that they will not overflow the page
% margins by default, and it is still possible to overwrite the defaults
% using explicit options in \includegraphics[width, height, ...]{}
\setkeys{Gin}{width=\maxwidth,height=\maxheight,keepaspectratio}

\setlength{\emergencystretch}{3em}  % prevent overfull lines
% Redefines (sub)paragraphs to behave more like sections
\ifx\paragraph\undefined\else
  \let\oldparagraph\paragraph
  \renewcommand{\paragraph}[1]{\oldparagraph{#1}\mbox{}}
\fi
\ifx\subparagraph\undefined\else
  \let\oldsubparagraph\subparagraph
  \renewcommand{\subparagraph}[1]{\oldsubparagraph{#1}\mbox{}}
\fi

% set default figure placement to htbp
\makeatletter
\def\fps@figure{htbp}
\makeatother

\setcounter{secnumdepth}{5}
\usepackage{framed}
\usepackage{multirow}
\usepackage{ctex}
\usepackage{rotating}
\usepackage{tablefootnote}
\usepackage{caption}
\setCJKmainfont{思源宋体}
\setCJKfallbackfamilyfont{\CJKrmdefault}{宋体}
\setmainfont{思源宋体}
\usepackage[a4paper,top=1in, bottom=1in, left=0.8in, right=0.8in]{geometry}
\setlength{\parindent}{2em}
\setlength{\parskip}{0em}

%\newfontfamily\apostrophefont[Ligatures=TeX,Color=FF0000]{Liberation Serif}
\newfontfamily\apostrophefont[Ligatures=TeX]{Liberation Serif}
\XeTeXinterchartokenstate=1
\newXeTeXintercharclass \apostrophe

% Assign the new class to all Latin capital letters
\makeatletter
\@tempcnta=`'
\loop\unless\ifnum\@tempcnta>`'
  \XeTeXcharclass \@tempcnta \apostrophe
  \advance \@tempcnta by 1
\repeat
\makeatother

% Setup font change
\XeTeXinterchartoks 0 \apostrophe   = {\begingroup\apostrophefont}
\XeTeXinterchartoks \apostrophe 0   = {\endgroup}
\XeTeXinterchartoks 4095 \apostrophe = {\begingroup\apostrophefont}
\XeTeXinterchartoks \apostrophe 4095 = {\endgroup}

\renewcommand {\thetable} {\thechapter{}-\arabic{table}}
\renewcommand {\thefigure} {\thechapter{}-\arabic{figure}}
\newcommand\subsectiontitleformat[1]{\noindent 【#1】}

\ctexset {
  section = {
    name
    = {第,节},
    number = \chinese{section},
  },
  subsubsection = {
    name 
    = {(,)},
    number = \chinese{subsubsection}
  },
  subsection = {
    name 
    = {,、},
    number = \chinese{subsection}
  }
}


\title{临床心电图详解与诊断}
\author{github \\ 项目主页:\url{https://github.com/scienceasdf/medical-books}\\ 新书下载:\url{https://github.com/scienceasdf/medical-books/releases/latest}}


\begin{document}
\maketitle
\frontmatter
{
\setcounter{tocdepth}{1}
\tableofcontents
\addcontentsline{toc}{chapter}{目录}
}
\newpage

\mainmatter
\part{P、QRS、T、U各波段正常值及其异常改变}

本篇根据心电图波形产生的顺序着重讨论了各波段的心电图改变及其临床意义,包括窦性P波及其异常改变、异位P波、房性融合波、心房内差异性传导、PR段偏移及P-R间期异常与P-J间期、正常QRS波群及其异常改变、与心源性猝死相关的波(J波、Epsilon波、Brugada波及Lambda波)、正常ST段及其异常改变、正常T波及其异常改变、正常Q-T间期及其异常改变、正常U波及其异常改变,共8章。

\protect\hypertarget{text00007.html}{}{}

\protect\hypertarget{text00007.htmlux5cux23chapter7}{}{}

\chapter{窦性P波、异位P波及其异常改变}

\protect\hypertarget{text00007.htmlux5cux23subid1}{}{}

\section{窦性P波及其异常改变}

\protect\hypertarget{text00007.htmlux5cux23subid2}{}{}

\subsection{正常窦性P波}

1.心电图特征

(1)P波极性:在Ⅰ、Ⅱ导联直立,aVR导联倒置,V\textsubscript{1}导联呈正、负双相或直立,V\textsubscript{5} 、V\textsubscript{6}导联直立。

(2)P波时间:各导联P波时间<0.11s,两切迹的峰距<0.04s。

(3)P波振幅:Ⅱ、Ⅲ、aVF导联P波振幅<0.25mV或低电压时,其P波振幅小于同导联R波振幅的$\frac{1}{2}$
,V\textsubscript{1} 导联P波正相波的振幅<0.15mV,V\textsubscript{5}导联P波振幅<0.2mV。

(4)V\textsubscript{1} Ptf值:又称为V\textsubscript{1}导联P波负相终末电势,正常值<|-0.04mm·s|。

(5)P-P间期互差<0.16s。

2.形成机制

窦性冲动优先地通过前、中结间束等心房内传导组织使心房肌除极产生P波,其除极方向从右前上向左后下除极,所形成的P波前$\frac{1}{3}$
为右心房除极,后$\frac{1}{3}$
为左心房除极,中$\frac{1}{3}$
为左、右心房及房间隔同时除极。

\protect\hypertarget{text00007.htmlux5cux23subid3}{}{}

\subsection{电轴左偏型P波}

1.心电图特征

(1)Ⅰ、aVL导联P波直立,其振幅大于Ⅱ导联的P波。

(2)Ⅱ导联P波低平或正、负双相,Ⅲ、aVF导联P波呈正、负双相(Ⅲ导联P波可浅倒)。

(3)aVR导联P波浅倒或负、正双相。

(4)P波时间、振幅多正常。

(5)起卧活动后,Ⅱ、Ⅲ、aVF导联P波振幅明显增高。这反过来可证实电轴左偏型P波与起搏点起源于窦房结尾部有关(图\ref{fig1-1})。

\begin{figure}[!htbp]
 \centering
 \includegraphics[width=5.78125in,height=1.22917in]{./images/Image00006.jpg}
 \captionsetup{justification=centering}
 \caption{健康体检者出现P波电轴左偏(下行系活动后记录)}
 \label{fig1-1}
  \end{figure} 

2.发生机制

(1)与窦房结起搏点的位置改变有关:窦房结头部的自律性较尾部高,且头部的冲动优先地通过前结间束下传心房,所形成的P波电轴多在+15°~+75°,故Ⅱ、Ⅲ、aVF导联P波振幅最高。而尾部发放的冲动优先地通过中结间束下传心房,所形成的P波电轴多在+15°~-30°,此时Ⅰ、aVL导联P波振幅高于Ⅱ导联。

(2)与心脏在胸腔的位置改变有关:当心脏在胸腔内发生顺钟向或逆钟向转位时,会影响心房除极所产生的P向量环的位置(正常P向量环位于左后下),进而影响各个导联P波的振幅甚至极性。

3.鉴别诊断

主要与房性异位心律相鉴别。若P波在Ⅰ、aVL导联直立,Ⅱ、aVF导联呈负、正双相,Ⅲ导联浅倒,aVR导联呈正、负双相,则为房性异位心律。

\protect\hypertarget{text00007.htmlux5cux23subid4}{}{}

\subsection{二尖瓣型P波}

因该P波常见于风心病二尖瓣狭窄患者,故称为“二尖瓣型P波”。

1.心电图特征

(1)P波时间≥0.11s,呈双峰切迹,两峰距≥0.04s,多出现在Ⅱ、Ⅲ、aVF、V\textsubscript{3}~V\textsubscript{6}等导联;若伴有P波电轴左偏,则出现在Ⅰ、aVL、V\textsubscript{5} 等导联。

(2)V\textsubscript{1}Ptf值≥|-0.04mm·s|(多见于风心病二尖瓣狭窄患者)。

(3)P波振幅正常(图\ref{fig1-2})。

\begin{figure}[!htbp]
 \centering
 \includegraphics[width=5.19792in,height=2.5in]{./images/Image00007.jpg}
 \captionsetup{justification=centering}
 \caption{风心病、二尖瓣狭窄伴关闭不全患者,出现二尖瓣型P波及V\textsubscript{1}导联P波高尖(提示双心房肥大)、高侧壁及前侧壁异常Q波、左右胸前导联QRS波群高电压(提示双心室肥大)、前侧壁ST-T改变}
 \label{fig1-2}
  \end{figure} 


2.发生机制

当左心房扩大、肥大或房间束(Bachmann束)、左心房内传导功能减低时,将导致左心房除极时间延长,从而使整个心房的除极时间也相应地延长。

3.临床意义

(1)左心房负荷过重:主要见于早期风心病二尖瓣狭窄、左心房黏液瘤、急性左心衰竭等。

(2)左心房扩大或肥大:凡是能导致左心房负荷持续加重的病因,均可引起左心房扩大或肥大。主要见于风心病二尖瓣狭窄,也见于扩张型心肌病、高血压性心脏病等。

(3)不完全性左心房内传导阻滞或房间束(Bachmann束)传导阻滞:多见于冠心病、心肌梗死、心肌炎及低钾血症等。

(4)左心房扩大合并左心房内传导阻滞:左心房扩大易损伤心房内传导组织,引起心房内传导阻滞,导致P波时间明显增宽(>0.14s)。

(5)易发生各种房性心律失常:左心房负荷长期过重,导致左心房扩大或肥大,继而牵拉和损伤心房内传导组织,引起心房内异位起搏点自律性增高、折返现象或触发活动,诱发多源性房性早搏、短阵性房性心动过速、心房扑动或心房颤动等。

\protect\hypertarget{text00007.htmlux5cux23subid5}{}{}

\subsection{肺型P波}

因该P波常见于慢性肺心病患者,故称为“肺型P波”。

1.心电图特征

(1)P波形态高尖:在Ⅱ、Ⅲ、aVF导联P波振幅≥0.25mV,V\textsubscript{1}、V\textsubscript{2} 导联P波振幅≥0.15mV。

(2)低电压时,P波振幅≥同导联R波振幅的$\frac{1}{2}$
。

(3)P波时间多正常。

(4)部分患者V\textsubscript{1}Ptf值≥|-0.04mm·s|,但V\textsubscript{1}导联P波负相波表现为深而窄(图\ref{fig1-3})。

\begin{figure}[!htbp]
 \centering
 \includegraphics[width=4.94792in,height=3.17708in]{./images/Image00008.jpg}
 \captionsetup{justification=centering}
 \caption{慢性支气管炎、肺心病患者出现肺型P波、V\textsubscript{1}Ptf值增大及右心室肥大(V\textsubscript{4} ~V\textsubscript{6}导联定准电压0.5mV)}
 \label{fig1-3}
  \end{figure} 


2.发生机制

因右心房除极比左心房早,且较早结束除极,故右心房扩大、肥大或右心房内传导功能减低时,其除极时间虽然有所延长,但大多不至于延长到左心房除极结束之后。因此,整个心房除极时间并不延长,但因其除极时所产生的向右前向量增大,故出现P波高尖。

3.临床意义

(1)右心房负荷过重:见于急性右心衰竭、早期肺动脉高压、甲状腺功能亢进、急性支气管炎、肺部炎症及长期吸烟者等。

(2)右心房扩大或肥大:凡是能导致右心房负荷持续加重的病因,均可引起右心房扩大或肥大。主要见于肺心病、先心病(如法洛四联症、房间隔缺损等)等。

(3)不完全性右心房内传导阻滞:多见于冠心病、心肌梗死、心肌炎及低钾血症等。

(4)右心房扩大合并右心房内传导阻滞。

(5)颅内血肿、肿瘤亦可出现肺型P波。

(6)交感神经兴奋性增高。

(7)易发生各种房性心律失常,如多源性房性早搏、短阵性房性心动过速、心房扑动或心房颤动等。

\protect\hypertarget{text00007.htmlux5cux23subid6}{}{}

\subsection{先心型P波}

因该P波常见于先心病患者,故称为“先心P波”。

1.心电图特征

(1)P波形态高尖:在Ⅱ、Ⅲ、aVF导联P波振幅≥0.25mV,V\textsubscript{1}、V\textsubscript{2} 导联P波振幅≥0.15mV,V\textsubscript{5}导联P波振幅≥0.2mV(图\ref{fig1-4})。

\begin{figure}[!htbp]
 \centering
 \includegraphics[width=5.78125in,height=1.97917in]{./images/Image00009.jpg}
 \captionsetup{justification=centering}
 \caption{先心病、法洛四联症患者,出现右心房、右心室肥大}
 \label{fig1-4}
  \end{figure} 

(2)P波时间多正常,但右心房显著扩大者,其除极时间将会明显延长,甚至延长至左心房除极结束之后,此时P波时间可≥0.12s。

2.临床意义

(1)右心房负荷过重。

(2)右心房扩大或肥大。

(3)易发生各种房性心律失常。

(4)当P波时间≥0.12s时,易误诊为双心房肥大。

\protect\hypertarget{text00007.htmlux5cux23subid7}{}{}

\subsection{交感型P波}

交感神经兴奋时,如运动、紧张等因素引起心率显著增快,可使原正常的P波,其振幅明显增高,出现类似“肺型P波”特点,称为交感型P波。系交感神经兴奋引起心房肌除极速度加快,导致右、左心房除极同步化,两者同时除极的部分叠加后使P波振幅明显增高。心率减慢后,P波形态、振幅均恢复正常(图\ref{fig1-5})。

\begin{figure}[!htbp]
 \centering
 \includegraphics[width=5.42708in,height=2.61458in]{./images/Image00010.jpg}
 \captionsetup{justification=centering}
 \caption{平板运动试验患者出现窦性心动过速、交感型P波、假性电轴左偏(-90°)}
 \label{fig1-5}
  \end{figure} 

\protect\hypertarget{text00007.htmlux5cux23subid8}{}{}

\subsection{巨大型P波}

凡是P波时间≥0.11s、振幅≥0.25mV者,称为巨大型P波。

1.心电图特征

(1)P波振幅增高:在Ⅱ、Ⅲ、aVF导联P波振幅≥0.25mV,V\textsubscript{1}、V\textsubscript{2} 导联P波振幅≥0.15mV。

(2)P波时间增宽:P波时间≥0.11s,呈双峰切迹,两峰距≥0.04s,一般在Ⅰ、Ⅱ、aVF、V\textsubscript{3}~V\textsubscript{6} 导联增宽最明显。

(3)V\textsubscript{1} Ptf值多≥|-0.04mm·s|(图\ref{fig1-6})。

\begin{figure}[!htbp]
 \centering
 \includegraphics[width=5.58333in,height=2.19792in]{./images/Image00011.jpg}
 \captionsetup{justification=centering}
 \caption{先心病、房间隔缺损、二尖瓣狭窄患者出现巨大型P波(提示双心房肥大)、完全性右束支传导阻滞、前侧壁轻度ST段改变}
 \label{fig1-6}
  \end{figure} 

2.临床意义

(1)双心房负荷过重:严重的先心病患者,开始有左向右分流;当肺动脉压力大于左心室压力时,则出现右向左分流,引起左、右心房负荷过重。

(2)双心房扩大或肥大:左、右心房负荷持续过重,势必引起左、右心房扩大或肥大。见于风心病二尖瓣狭窄或伴关闭不全、扩张型心肌病及先心病(如室间隔缺损、动脉导管未闭)等。

(3)右心房扩大合并左心房内传导阻滞。

(4)左心房扩大合并右心房内传导阻滞。

(5)不完全性左、右心房内传导阻滞。

(6)右心房显著扩大:当右心房显著扩大时,右心房除极时间明显延长,甚至延长至左心房除极结束后,表现为P波振幅增高及时间增宽,酷似双心房扩大。见于严重的肺动脉高压、Ebstein畸形(先天性三尖瓣下移综合征)的部分患者。

(7)易发生各种房性心律失常。

\protect\hypertarget{text00007.htmlux5cux23subid9}{}{}

\subsection{间歇性P波改变}

窦性心律时,其P波形态、振幅呈间歇性改变,见于间歇性心房内传导阻滞、P波电交替、窦房结头部与尾部交替性或间歇性发放冲动等。

(1)间歇性心房内传导阻滞:在P-P间期基本规则时,间歇性出现正常P波、肺型P波或(和)二尖瓣型P波(图\ref{fig1-7}、图\ref{fig1-8})。

\begin{figure}[!htbp]
 \centering
 \includegraphics[width=5.82292in,height=0.72917in]{./images/Image00012.jpg}
 \captionsetup{justification=centering}
 \caption{风心病、二尖瓣狭窄患者出现二尖瓣型P波(左心房肥大所致)和肺型P波(间歇性右心房内传导阻滞所致)}
 \label{fig1-7}
  \end{figure} 

\begin{figure}[!htbp]
 \centering
 \includegraphics[width=5.78125in,height=1in]{./images/Image00013.jpg}
 \captionsetup{justification=centering}
 \caption{冠心病患者间歇性出现不完全性右心房内传导阻滞引起一过性肺型P波、ST-T改变}
 \label{fig1-8}
  \end{figure} 

(2)频率依赖性心房内传导阻滞:肺型P波、二尖瓣型P波的出现与窦性频率的快、慢有关。若心率增快时出现,则称为3相性或快频率依赖性右心房或左心房内传导阻滞;若心率减慢时出现,则称为4相性或慢频率依赖性右心房或左心房内传导阻滞(图\ref{fig1-9})。

\begin{figure}[!htbp]
 \centering
 \includegraphics[width=5.78125in,height=1.97917in]{./images/Image00014.jpg}
 \captionsetup{justification=centering}
 \caption{MV\textsubscript{5}导联连续记录,显示4相性右心房内传导阻滞引起一过性肺型P波、ST-T改变(与图\ref{fig1-8}系同一患者)}
 \label{fig1-9}
  \end{figure} 


(3)右心房内文氏现象:当P-P间期规则时,出现P波振幅由正常→较高尖→高尖→正常,周而复始有规律地改变。

(4)左心房内文氏现象:当P-P间期规则时,出现P波时间由正常→稍增宽→明显增宽→正常,周而复始有规律地改变。

\protect\hypertarget{text00007.htmlux5cux23subid10}{}{}

\subsection{右位心型P波}

心电图上具有特征性改变的是镜像右位心,有以下5个特征:

(1)Ⅰ导联上P-QRS-T波群均倒置,呈正常者的倒像。

(2)Ⅱ与Ⅲ导联、aVR与aVL导联图形互换,aVF导联图形不变。

(3)V\textsubscript{1} ~V\textsubscript{6}导联R波振幅逐渐降低而S波则相对变深(图\ref{fig1-10}A)。

\includegraphics[width=5.78125in,height=2.08333in]{./images/Image00015.jpg}

图\ref{fig1-10}A 右位心型P波、完全性右束支传导阻滞(正常连接的十二导联)

(4)加做V\textsubscript{3} R、V\textsubscript{4} R、V\textsubscript{5}R、V\textsubscript{6}R导联,其R波、T波振幅逐渐增高或者以V\textsubscript{4}R、V\textsubscript{5} R导联最高。

(5)左、右手导联线反接,胸前导联以V\textsubscript{2}、V\textsubscript{1} 、V\textsubscript{3} R、V\textsubscript{4}R、V\textsubscript{5} R、V\textsubscript{6}R导联方式检查,将会显露心电图的真面目(图\ref{fig1-10}B)。

\includegraphics[width=5.78125in,height=2.11458in]{./images/Image00016.jpg}

图\ref{fig1-10}B 左、右手导联线反接及加做右胸前导联记录,显示完全性右束支传导阻滞、电轴左偏(-35°)

\protect\hypertarget{text00007.htmlux5cux23subid11}{}{}

\subsection{房间隔阻滞型P波}

房间隔阻滞型P波是指Ⅱ、Ⅲ、aVF导联P波呈正、负双相伴时间≥0.12s。见于不完全性左心房内传导阻滞伴左心房逆行传导,是一种特殊类型的心房内传导阻滞。表现为窦性冲动在左心房内除极不仅延缓,还从左心房下部向上部除极,形成终末负相P波,系上房间束(Bachmann束)传导完全阻滞所致(图\ref{fig1-11})。

\begin{figure}[!htbp]
 \centering
 \includegraphics[width=5.72917in,height=2.89583in]{./images/Image00017.jpg}
 \captionsetup{justification=centering}
 \caption{扩张型心肌病患者出现正负双相型P波(上房间束传导阻滞所致)、右心房肥大、一度房室传导阻滞,提示房室结内双径路传导(P-R间期0.24、0.31s)、左前分支阻滞、伪室性融合波(心室起搏脉冲落在QRS波群中)、前壁r波振幅逆递增(r\textsubscript{V\textsubscript{2}}>r\textsubscript{V\textsubscript{3}}>r\textsubscript{V\textsubscript{4}})、高侧壁异常Q波、高侧壁及前侧壁ST-T改变}
 \label{fig1-11}
  \end{figure} 


1.心电图特征

(1)Ⅱ、Ⅲ、aVF导联P波呈正、负双相。

(2)P波时间≥0.12s。

(3)P波前半部分与后半部分的P环电轴夹角常>90°。

(4)心内电生理检查时,心房除极顺序为高位右心房→低位右心房→低位左心房→高位左心房。

2.鉴别诊断

需与窦性P波电轴左偏相鉴别。两者虽然均表现为Ⅱ、Ⅲ、aVF导联P波呈正、负双相,但后者P波时间正常,活动后P波转为直立,可资鉴别。

3.临床意义

(1)出现房间隔阻滞型P波是左心房扩大或肥大非常特异的征象,同时意味着上房间束传导完全阻滞。

(2)具有较高的快速性房性心律失常发生率,尤其是心房扑动。

\protect\hypertarget{text00007.htmlux5cux23subid12}{}{}

\subsection{游走性P波}

(1)窦房结内游走节律:起搏点在窦房结头、体、尾部游走不定引起P波形态、频率改变者。其心电图特征为:①P波极性一致,振幅由高→低或由低→高周期性改变,但不出现逆行P\textsuperscript{-}波,时间多正常;②P-P间期互差>0.16s,P波振幅较高时,其P-P间期较短,P波振幅逐渐减低时,其P-P间期又逐渐延长;③P-R间期多固定(图\ref{fig1-12})。

\begin{figure}[!htbp]
 \centering
 \includegraphics[width=5.78125in,height=0.8125in]{./images/Image00018.jpg}
 \captionsetup{justification=centering}
 \caption{窦房结内游走节律}
 \label{fig1-12}
  \end{figure} 

(2)窦房结至心房内游走节律:起搏点在窦房结头、体、尾部直至心房下部游走不定引起P波形态、极性和频率改变者。其心电图特征为:①P波极性有直立和倒置两种,振幅由高→低→浅倒→倒置或由倒置→浅倒→低→高周期性改变;②P-P间期不规则,P波振幅较高时,其P-P间期较短,P波振幅逐渐减低时,其P-P间期又逐渐延长;③P-R间期大多固定(图\ref{fig1-13})。

\begin{figure}[!htbp]
 \centering
 \includegraphics[width=5.78125in,height=0.48958in]{./images/Image00019.jpg}
 \captionsetup{justification=centering}
 \caption{窦房结至心房内游走节律(但不能排除非阵发性房性心动过速伴不同程度的房性融合波)}
 \label{fig1-13}
  \end{figure} 

\protect\hypertarget{text00007.htmlux5cux23subid13}{}{}

\subsection{P波低电压}

常规十二导联P波振幅均<0.1mV,称为P波低电压或振幅降低。

1.心电图特征

(1)所有导联P波振幅均<0.1mV。

(2)QRS波群低电压(肢体导联QRS波幅<0.5mV或胸前导联QRS波幅<1.0mV)。

(3)T波低平或倒置,Q-T间期延长。

(4)可出现过缓性心律失常及各种传导阻滞。

2.临床意义

(1)冲动起源于窦房结尾部。

(2)广泛而严重的心房肌纤维化。

(3)甲状腺功能减退。

(4)过度肥胖、大量心包积液或左侧气胸。

(5)高钾血症,随着血钾浓度逐渐增高,P波振幅逐渐减小直至消失。

(6)心房梗死。

\protect\hypertarget{text00007.htmlux5cux23subid14}{}{}

\subsection{P波电交替}

1.心电图特征

(1)P-P间期、P-R间期均必须固定,以确保是同一起搏点的激动,多见于窦性节律。

(2)交替出现两种形态的窦性P波,其振幅互差≥0.1mV,时间可有轻度互差。

(3)两种形态P波的极性一致,其额面P环电轴指向左下。

(4)这两种P波形态的改变与呼吸、伪差等心外因素无关(图\ref{fig1-14})。

(5)可伴有QRS波幅、ST段、T波、U波电交替现象。

\begin{figure}[!htbp]
 \centering
 \includegraphics[width=5.78125in,height=0.53125in]{./images/Image00020.jpg}
 \captionsetup{justification=centering}
 \caption{P波电交替现象}
 \label{fig1-14}
  \end{figure} 

2.发生机制

(1)心房内特殊传导组织或某部分心房肌传导障碍,导致交替性心房内传导阻滞。

(2)心房肌缺血致跨膜动作电位复极2相、3相发生交替性改变或心房肌不应期长、短交替性改变,导致交替性心房肌除极异常。

(3)窦房结头部与尾部交替性发放冲动,此时其P-P间期略有互差。

(4)窦房交接区双径路(双出口)交替传导,导致心房除极顺序发生改变。

(5)左、右心房起搏点等频性交替性发放冲动,形成双源性房性心律,或窦房结与心房异位起搏点交替性发放冲动。严格地说,这两种情况不属于P波电交替范畴内。

3.临床意义

(1)这是一种罕见的心电现象,多见于器质性心脏病,如心房梗死、心房负荷过重、心房扩大及心房肌严重缺血等,常提示心房病变严重而广泛,是一种预后不良的征象,死亡率较高。

(2)这是心房肌严重缺血、心电不稳定的表现,易发生各种房性心律失常。

\protect\hypertarget{text00007.htmlux5cux23subid15}{}{}

\subsection{P波缺失}

(一)一过性窦性P波缺失

1.窦性停搏

(1)心电图特征:①长P-P间期与短P-P间期之间无倍数关系;②长P-P间期>1.80~2.0s(白天1.80s,夜间2.0s)或长P-P间期大于短P-P间期的1.5倍(图\ref{fig1-15})。

\begin{figure}[!htbp]
 \centering
 \includegraphics[width=5.78125in,height=0.6875in]{./images/Image00021.jpg}
 \captionsetup{justification=centering}
 \caption{窦性心律不齐、窦性停搏}
 \label{fig1-15}
  \end{figure} 

(2)临床意义:①心源性窦房结功能障碍:又称为原发性病窦综合征,多由器质性心脏病所致;②外源性窦房结功能障碍:又称为继发性病窦综合征,多由心脏活性药物、迷走神经张力显著增高、低温、高钾血症、重度颅脑损伤等心外因素所致,以前两者影响最为重要;③特发性窦房结功能障碍:经多种检查无法明确病因,又无心脏病基础者。

2.窦房传导阻滞

(1)二度Ⅰ型窦房传导阻滞:P-P间期逐渐缩短直至出现1个长P-P间期,长P-P间期小于任何短P-P间期的2倍,P-P间期呈“渐短突长”规律,周而复始。

(2)二度Ⅱ型窦房传导阻滞:长P-P间期为短P-P间期的2~3倍(图\ref{fig1-16})。

(3)高度~几乎完全性窦房传导阻滞:长P-P间期≥4倍的短P-P间期。

\begin{figure}[!htbp]
 \centering
 \includegraphics[width=5.78125in,height=1.32292in]{./images/Image00022.jpg}
 \captionsetup{justification=centering}
 \caption{二度Ⅱ型窦房传导阻滞、ST段呈缺血型改变}
 \label{fig1-16}
  \end{figure} 

3.窦房结节律超速抑制现象。

见于短阵性房性心动过速、阵发性心房扑动、心房颤动或阵发性室上性心动过速等快速性异位心律终止后对窦房结节律的超速抑制。异位节律的频率愈快、持续时间愈长、窦房结功能愈差者,则窦性节律恢复所需的时间愈长(图\ref{fig1-17})。

\begin{figure}[!htbp]
 \centering
 \includegraphics[width=5.78125in,height=1.02083in]{./images/Image00023.jpg}
 \captionsetup{justification=centering}
 \caption{阵发性不纯性心房扑动终止后出现短暂性全心停搏、过缓的房室交接性逸搏}
 \label{fig1-17}
  \end{figure} 

4.窦性节律被重整或干扰性窦房分离。

下级起搏点自律性增高,如加速的房性逸搏心律、加速的房室交接性逸搏心律等异位冲动逆传窦房结使其节律连续重整或与窦性冲动在窦房交接区产生连续干扰形成不完全性干扰性窦房分离。

(二)较长时间窦性P波缺失

(1)窦性停搏:基本节律可为房性、房室交接性、室性逸搏心律或人工起搏心律等。

(2)三度窦房传导阻滞。

(3)窦-室传导:见于高钾血症,血钾恢复正常后,将出现窦性P波。

(4)阵发性室上性心动过速、阵发性心房扑动及心房颤动发作期间。

(5)加速的房性逸搏心律、加速的房室交接性逸搏心律等异位冲动持续重整窦性节律,形成假性窦性停搏或与窦性冲动在窦房交接区产生连续干扰形成干扰性窦房分离(假性三度窦房传导阻滞)。

(三)永久性窦性P波缺失

1.永久性三度窦房传导阻滞

2.永久性窦性停搏

3.永久性心房颤动

4.隐匿性窦性心律

(1)概念:指常规心电图中始终未见窦性P波,但经心房内或食管内心电图能记录到窦性P波一种少见的心电现象。

(2)心电图特征:①常规心电图始终未见心房电活动波,如P波、P\textsuperscript{-}波、F波或f波;②QRS波形正常或显示心室肥大、束支阻滞等图形,R-R间期符合窦性心律标准;③做心房内或食管内心电图时,可见QRS波群之前有A波或窦性P波,两者具有相关性。

5.心房静止

(1)概念:指心房肌因严重而广泛的不可逆损害,导致心房肌应激性丧失使其永久不能除极。

(2)心电图特征:①体表心电图、食管心电图、心房内心电图均未见P波或A波;②对心房肌进行电刺激时仍未见心房电活动;③X线胸透、超声心动图检查亦未见心房肌收缩活动;④QRS波形正常,基本节律起源于房室交接区。

(3)临床意义:①多见于器质性心脏病;②偶见于先天性心房肌应激功能异常。

\protect\hypertarget{text00007.htmlux5cux23subid16}{}{}

\subsection{窦性早搏}

1.心电图特征

(1)提早出现的P′波形态与同导联的窦性P波一致或略异。这取决于该早搏起搏点的位置、传出途径与窦性激动是否一致。

(2)呈等周期代偿,即P′-P间期等于窦性P-P间期。

(3)P′波下传的P′-R间期正常或呈干扰性P′-R间期延长,QRS波形正常或伴心室内差异性传导。

(4)偶联间期固定者,为折返型早搏;若偶联间期不等,两异位搏动之间相等或有一最大公约数,则为并行心律型早搏;否则为异位自律性增高型早搏(图\ref{fig1-18})。

\begin{figure}[!htbp]
 \centering
 \includegraphics[width=5.78125in,height=0.51042in]{./images/Image00024.jpg}
 \captionsetup{justification=centering}
 \caption{舒张晚期窦性早搏,呈二、三联律}
 \label{fig1-18}
  \end{figure} 

2.临床意义

(1)窦性早搏的出现,改变了早搏都是由异位起搏点发放冲动的观念,早搏也可发生在正常起搏点内。

(2)可以是功能性的,由心外因素所致;也可以是器质性的,由器质性心脏病所致。

\protect\hypertarget{text00007.htmlux5cux23subid17}{}{}

\subsection{窦性逸搏}

1.心电图特征

(1)在两阵快速或较快速异位性心动过速终止后间歇期内,略为延迟出现1~2次窦性搏动,其P波形态与正常窦性P波一致。

(2)若逸搏周期0.60~1.0s,频率60~100次/min,则称为窦性逸搏;若逸搏周期>1.0s,频率<60次/min,则称为过缓的窦性逸搏。

(3)若连续出现≥3次窦性逸搏,则称为窦性逸搏心律,亦称为正常的窦性心律。

2.临床意义

(1)窦性逸搏的出现,仍标志着窦房结有正常的起搏功能,出现窦性逸搏是由于下级起搏点发放快速或较快速激动后又突然终止所致。若异位节律点得到控制后,将自然恢复正常的窦性心律。

(2)过缓的窦性逸搏常见于窦房结自律性降低或病窦综合征患者,转复为窦性心律后,可出现窦性停搏。

\protect\hypertarget{text00007.htmlux5cux23subid18}{}{}

\subsection{窦性回波及窦房交接性早搏}

严格地说,窦性回波及窦房交接性早搏不属于窦性P波的范畴,但因其P波形态与窦性P波一致或相似,故放在此处讨论。

1.窦性回波

(1)发生机制:窦房结及其周围组织的电生理功能和房室结有一定的相似性,细胞间的不应期和传导性均存在明显差异,窦房结及窦房交接区可分离为两条传导功能不同的径路。适时的房性早搏由一条径路逆传侵入窦房结,再经另一条径路传出,再次激动心房,形成窦性回波。在房性早搏后,产生P′-P\textsubscript{3}-P\textsubscript{4} 序列,其中P\textsubscript{3}为窦性回波,P\textsubscript{4}为房性早搏重整窦性节律后窦性所发放的第1个激动,上述P′-P\textsubscript{3}-P\textsubscript{4} 序列中的P\textsubscript{3} -P\textsubscript{4}间期小于窦性P\textsubscript{1} -P\textsubscript{2} 间期(图\ref{fig1-19})。

(2)心电图特征:①适时的房性早搏(P′波)或有逆传心房的房室交接性早搏、室性早搏后,紧跟1个与窦性P波(P\textsubscript{1}、P\textsubscript{2} 、P\textsubscript{4})形态相同的P波(P\textsubscript{3});②窦房折返周期P′-P\textsubscript{3} 间期<P\textsubscript{1}-P\textsubscript{2}间期;③房性早搏的偶联间期多在0.25~0.46s之间,方能引起窦性回波P\textsubscript{3};④P\textsubscript{2} -P\textsubscript{3} 间期不等于P\textsubscript{1}-P\textsubscript{2}间期,可排除P′波为间位型房性早搏;⑤P\textsubscript{3}-P\textsubscript{4} 间期<P\textsubscript{1} -P\textsubscript{2}间期,可排除P′、P\textsubscript{3} 波为成对房性早搏(图\ref{fig1-19})。

\begin{figure}[!htbp]
 \centering
 \includegraphics[width=5.73958in,height=1.41667in]{./images/Image00025.jpg}
 \captionsetup{justification=centering}
 \caption{房性早搏诱发窦性回波}
 \label{fig1-19}
  \end{figure} 

2.窦房交接性早搏

(1)提早出现的P′波形态与窦性P波一致或略异。这取决于该早搏激动下传途径与窦性激动是否一致、心房除极顺序有无改变。

(2)可呈次等周期、等周期代偿或不完全性代偿间歇。这取决于窦房交接性早搏逆传重整窦性节律所需时间与前传激动心房所需时间差值的多少和窦房结节律的稳定性。在窦性节律规则时,当逆传重整窦性节律先于前传激动心房,且两者时间差刚好为窦房传导时间,则呈等周期代偿(图\ref{fig1-20}A),与窦性早搏难以鉴别;当逆传重整窦性节律明显先于前传激动心房,则呈次等周期代偿(图\ref{fig1-20}B);反之,当前传激动心房明显先于逆传重整窦性节律,则呈不完全性代偿间歇(图\ref{fig1-20}C),与房性早搏难以鉴别。故只有呈次等周期代偿时方能诊断。

\begin{figure}[!htbp]
 \centering
 \includegraphics[width=5.5625in,height=0.71875in]{./images/Image00026.jpg}
 \captionsetup{justification=centering}
 \caption{窦房交接性早搏出现3种代偿间歇的梯形图解模式(图A:等周期代偿;图B:次等周期代偿;图C:不完全性代偿间歇)}
 \label{fig1-20}
  \end{figure} 

(3)P′波下传的P′-R间期正常或伴干扰性P′-R间期延长,QRS波形正常或伴心室内差异性传导。

(4)偶联间期多固定(图\ref{fig1-21})。

\begin{figure}[!htbp]
 \centering
 \includegraphics[width=5.79167in,height=1.45833in]{./images/Image00027.jpg}
 \captionsetup{justification=centering}
 \caption{窦房交接区折返性早搏伴心室内差异性传导(呈左中隔支阻滞型及右束支阻滞型)}
 \label{fig1-21}
  \end{figure} 

\protect\hypertarget{text00007.htmlux5cux23subid19}{}{}

\section{异位P波}

\protect\hypertarget{text00007.htmlux5cux23subid20}{}{}

\subsection{逆行P\textsuperscript{-} 波}

1.基本概念

起源于心房下部、房室交接区或心室异位激动逆传心房时所产生的P波极性与窦性P波刚好相反,称为逆行P\textsuperscript{-}波,常用P\textsuperscript{-} 表示。

2.心电图特征

(1)P\textsuperscript{-} 波在Ⅱ、Ⅲ、aVF导联倒置,aVR导联直立。

(2)根据P\textsuperscript{-}-R间期的长短来判断异位起搏点的位置:若P\textsuperscript{-}-R间期≥0.12s,则起源于心房下部(图\ref{fig1-22});若P\textsuperscript{-}-R间期<0.12s,则起源于房室交接区(图\ref{fig1-23})。

\begin{figure}[!htbp]
 \centering
 \includegraphics[width=5.58333in,height=1.07292in]{./images/Image00028.jpg}
 \captionsetup{justification=centering}
 \caption{加速的房性逸搏心律(起源于心房下部,其P\textsuperscript{-}-R间期0.14s,频率92次/min)}
 \label{fig1-22}
  \end{figure} 


\begin{figure}[!htbp]
 \centering
 \includegraphics[width=5.30208in,height=0.88542in]{./images/Image00029.jpg}
 \captionsetup{justification=centering}
 \caption{男性患者发生心动过速半年余,提示持续性房室交接性心动过速(P\textsuperscript{-}-R间期0.10s,频率150次/min)}
 \label{fig1-23}
  \end{figure} 


(3)若逆行P\textsuperscript{-}波位于QRS波群之后,则根据R-P\textsuperscript{-}间期的长短、QRS波形的特征来判断异位起搏点的位置:若R-P\textsuperscript{-}间期≤0.16s,QRS波形正常,则起源于房室交接区;若R-P\textsuperscript{-}间期>0.16s,QRS波群宽大畸形,则起源于心室。

\protect\hypertarget{text00007.htmlux5cux23subid21}{}{}

\subsection{正相逆行P\textsuperscript{-} 波}

1.基本概念

起源于房室交接区或心室异位激动逆传心房时所产生的逆行P\textsuperscript{-}波,在Ⅱ、Ⅲ、aVF导联呈直立P波,称为正相逆行P\textsuperscript{-} 波。

2.心电图特征

(1)房室交接性心律伴正相逆行心房夺获。

(2)房室交接性或室性心律不齐时,在其QRS波群后面始终跟随与R波有固定关系的正相P波。

(3)与P-R间期延长到某临界值相关联的提早出现的正相P波或(和)反复性心动过速(图\ref{fig1-24})
。

(4)预激综合征时,出现正相P波的反复性心动过速。

\begin{figure}[!htbp]
 \centering
 \includegraphics[width=5.88542in,height=1.28125in]{./images/Image00030.jpg}
 \captionsetup{justification=centering}
 \caption{房室结双径路传导伴慢径路文氏现象及正相逆行P\textsuperscript{-}波(引自郑莹)}
 \label{fig1-24}
  \end{figure} 


3.发生机制

心房内的特殊传导纤维,如结间束、James束(大部分由后结间束组成,少部分由前、中结间束组成)及Kent束的存在为正相逆行P\textsuperscript{-}波的解释提供了解剖学基础。当起源于房室交接区或心室异位激动经房室正道逆传受阻时,可从James束或从出口处位于心房上部的Kent束逆传,使心房除极顺序与窦性激动相似而出现直立P波,或房室交接区激动优先通过前结间束快速逆行到房间束和窦房交接区先激动心房上部,使心房除极顺序与窦性激动相似,也可出现直立P波。

\protect\hypertarget{text00007.htmlux5cux23subid22}{}{}

\subsection{房性早搏}

1.心电图特征

(1)提早出现的P′波形态与窦性P波不一致,有时P′波重叠在T波上使T波变形。不论其后是否跟随QRS-T波群,均可作出房性早搏的诊断。

(2)多呈不完全性代偿间歇,舒张晚期的房性早搏可出现完全性代偿间歇。

(3)发生在收缩中、晚期及少数舒张早期的房性早搏,将出现各种房室干扰现象,如呈阻滞型、房室结内隐匿性传导、干扰性P′-R间期延长及心室内差异性传导,其中前两者称为房性早搏未下传(图\ref{fig1-25})。

(4)若P′波形态不一致而偶联间期相等,则为多形性房性早搏;若P′波形态及偶联间期均不相同,则为多源性房性早搏。

\begin{figure}[!htbp]
 \centering
 \includegraphics[width=5.73958in,height=1.32292in]{./images/Image00031.jpg}
 \captionsetup{justification=centering}
 \caption{房性早搏二联律伴房室干扰现象(P′波未下传、干扰性P′-R间期延长及心室内差异性传导),其中最后1个房性早搏下传的P′-R间期明显缩短,可能与房室结2相超常期传导有关}
 \label{fig1-25}
  \end{figure} 

2.临床意义

(1)功能性房性早搏:由心外因素所致。

(2)器质性房性早搏:由器质性心脏病所致。

(3)药物性房性早搏:由各种药物过量或毒副作用所致。

\protect\hypertarget{text00007.htmlux5cux23subid23}{}{}

\subsection{房性逸搏}

1.心电图特征

(1)在两阵窦性心律或两阵异位心律之间,延迟出现1~2次P′波或P′-QRS-T波群,P′波形态与窦性P波不同。若P′波形态一致,则为单源性房性逸搏;若P′波呈两种形态,则为双源性房性逸搏(图\ref{fig1-26});若P′波呈3种或3种以上形态,则称为多源性房性逸搏。

\begin{figure}[!htbp]
 \centering
 \includegraphics[width=5.78125in,height=0.48958in]{./images/Image00032.jpg}
 \captionsetup{justification=centering}
 \caption{折返型房性早搏后出现双源性房性逸搏(P\textsubscript{3}、P\textsubscript{7} ),但不能排除窦性P波伴非时相性心房内差异性传导}
 \label{fig1-26}
  \end{figure} 


(2)若逸搏周期1.0~1.20s,频率50~60次/min,则称为房性逸搏;若逸搏周期>1.20s,频率<50次/min,则称为过缓的房性逸搏;若逸搏周期0.60~1.0s,频率61~100次/min,则称为加速的房性逸搏。

(3)P′-R间期、QRS波形与窦性一致。有时房性逸搏P′波刚出现时,又发生了房室交接性逸搏或室性逸搏,则P′-R间期<0.12s,P′波被干扰而未能下传。

(4)有时可见延迟出现的P′波与窦性P波相融合,而成为房性融合波。

2.房性逸搏的定位诊断(见第十一章第四节房性早搏的定位诊断)

3.临床意义

房性逸搏及其逸搏心律的出现,表明心脏有潜在的起搏能力,它本身并无重要临床意义,主要取决于原发性心律失常。而加速的房性逸搏及其逸搏心律的出现,若不伴有窦性节律竞争,则说明窦房结自律性降低,部分见于器质性心脏病,如冠心病、病窦综合征等,部分则无器质性心脏病。若伴有窦房结-心房节律竞争,则见于心肌炎、急性心肌梗死、洋地黄中毒、心脏手术等。

\protect\hypertarget{text00007.htmlux5cux23subid24}{}{}

\section{房性融合波}

1.基本概念

发自心脏两个节律点的冲动从不同方向同时进入心房,且各自激动了心房的一部分,此时所形成的P波称为房性融合波。房性融合波是心房内绝对干扰所致。

2.心电图特征

(1)房性融合波的形态介于其他两种P波之间,视融合程度不同,其形态可以多变。

(2)房性融合波出现的时间必须是两个节律点冲动同时或几乎同时出现的时间。

3.房性融合波的类型

(1)窦性P波与房性P′波之间的融合:该P′波可以是舒张晚期房性早搏、房性逸搏或人工心房起搏搏动等。

(2)窦性P波与逆行P\textsuperscript{-}波之间的融合:该逆行P\textsuperscript{-}波可以是起源于心房下部、房室交接区或心室异位搏动逆传心房时产生(图\ref{fig1-27})。

\begin{figure}[!htbp]
 \centering
 \includegraphics[width=5.75in,height=1.47917in]{./images/Image00033.jpg}
 \captionsetup{justification=centering}
 \caption{窦性心动过缓、房室交接性逸搏心律及房性融合波}
 \label{fig1-27}
  \end{figure} 

(3)心房内两个异位P′波之间的融合:心房内有两个异位起搏点几乎同时发放冲动,各自激动了心房的一部分。见于双重性房性并行心律、多源性房性早搏、多源性房性心动过速等。

(4)房性异位P′波与逆行P\textsuperscript{-} 波之间的融合。

4.鉴别诊断

主要与心房分离时所出现的重叠波相鉴别,即心房分离时所产生的低小P′波重叠在主导节律所产生的P波不同部位上。两者是重叠而不是融合(详见第十九章第三节局限性完全性心房内传导阻滞)。

\protect\hypertarget{text00007.htmlux5cux23subid25}{}{}

\section{心房内差异性传导}

\protect\hypertarget{text00007.htmlux5cux23subid26}{}{}

\subsection{时相性心房内差异性传导}

1.心电图特征

(1)提早出现变形的P、P′或P\textsuperscript{-}波,不能用其他原因解释者。

(2)窦性早搏伴心房内差异性传导:窦性早搏的P波形态与其他窦性P波形态不同。

(3)房性早搏伴心房内差异性传导:房性早搏的P′波形态明显畸形,可呈肺型P波或二尖瓣型P波特点(图\ref{fig1-28})。

\begin{figure}[!htbp]
 \centering
 \includegraphics[width=5.64583in,height=0.72917in]{./images/Image00034.jpg}
 \captionsetup{justification=centering}
 \caption{房性早搏伴心房内差异性传导、间歇性不完全性左右心房内传导阻滞、ST段改变}
 \label{fig1-28}
  \end{figure} 

(4)房室交接性早搏或室性早搏逆传心房时伴心房内差异性传导:逆行P\textsuperscript{-}波形态明显畸形,通常是P\textsuperscript{-}波倒置加深或增宽呈反向“二尖瓣型P波” 。

2.发生机制

时相性心房内差异性传导(简称心房内差异性传导)系心房内相对干扰所致,主要与冲动过早出现遇及心房内传导组织(如结间束、Bachmann束)和心房肌相对不应期有关。

3.临床意义

因心房内传导组织、心房肌不应期均较短,一般情况下极少发生心房内差异性传导。一旦出现,则意味着心房有病变,多见于器质性心脏病患者。

\protect\hypertarget{text00007.htmlux5cux23subid27}{}{}

\subsection{非时相性心房内差异性传导}

1.心电图表现

(1)房性早搏、短阵性房性心动过速或能逆传心房的房室交接性早搏、室性早搏代偿间歇之后,出现1个或连续数个窦性P波形态发生改变。

(2)变形的P波又是窦性P波应出现的位置,且多次重复出现(图\ref{fig1-29})。

\begin{figure}[!htbp]
 \centering
 \includegraphics[width=5.78125in,height=0.73958in]{./images/Image00035.jpg}
 \captionsetup{justification=centering}
 \caption{折返型房性早搏后出现非时相性心房内差异性传导、轻度ST段改变}
 \label{fig1-29}
  \end{figure} 

2.发生机制

(1)早搏在心房传导组织内发生隐匿性传导:由于结间束、房间束的不应期不一致,早搏在逆传窦房结时,可在窦房交接区内产生隐匿性折返,并隐匿性地激动了结间束、房间束,使其产生新的不应期,影响下一个窦性激动的正常除极,导致P波形态改变。

(2)心房内4相性传导阻滞:早搏产生较长的代偿间歇,结间束或房间束发生4相性传导阻滞。

3.鉴别诊断

应注意与窦房结内游走节律、窦房结至心房内游走节律、房性逸搏及房性融合波相鉴别。

4.临床意义

见于器质性心脏病患者。常出现在心力衰竭患者,具有病理意义。

\protect\hypertarget{text00008.html}{}{}

\protect\hypertarget{text00008.htmlux5cux23chapter8}{}{}

\chapter{PR段偏移和P-R间期异常及P-J间期}

\protect\hypertarget{text00008.htmlux5cux23subid28}{}{}

\section{PR段偏移}

心房除极结束后至心室开始除极前,有一段无电位差的等电位线,称为PR段,通常以TP段的延长线作为基线。PR段常受Ta波的影响,可呈下斜型压低(≤0.08~0.1mV)或抬高(≤0.05mV)。

1.心电图特征

在P波明显的导联出现PR段压低(>0.08~0.1mV)或抬高(>0.05mV),特别是呈水平型压低时,则属异常改变,对心房梗死、急性心包炎、心房损伤等有较大的诊断价值(图\ref{fig2-1})。

\begin{figure}[!htbp]
 \centering
 \includegraphics[width=5.58333in,height=2.04167in]{./images/Image00036.jpg}
 \captionsetup{justification=centering}
 \caption{急性心包炎患者出现窦性心动过速、肢体导联QRS波群低电压倾向、PR段压低}
 \label{fig2-1}
  \end{figure} 

2.临床意义

(1)心房梗死:单纯性心房梗死罕见,多与心室梗死并存,常伴发快速性房性心律失常。

(2)急性心包炎:PR段偏移可能是急性心包炎最早出现的心电图异常,甚至是唯一可见的心电图改变,具有较高的特异性。

(3)心房肌劳损、心房外伤。

(4)心房肥大或扩大。

\protect\hypertarget{text00008.htmlux5cux23subid29}{}{}

\section{P-R间期异常改变}

\protect\hypertarget{text00008.htmlux5cux23subid30}{}{}

\subsection{P-R间期测量方法}

12导联同步记录时,以最早的P波起点至最早的QRS波群起点的间距作为P-R间期。若是3导联同步记录或单导联记录时,则选取2~3个P波最清晰、最宽大且有明显Q波(或q波)的“半正交”导联,如Ⅱ、Ⅲ、V\textsubscript{1}(V\textsubscript{5} )或Ⅰ(aVL)、aVF、V\textsubscript{2}(V\textsubscript{5})(P波电轴左偏时),按P-R间期最长者计算。其正常值为0.12~0.20s。

\protect\hypertarget{text00008.htmlux5cux23subid31}{}{}

\subsection{P-R间期缩短}

1.基本概念

P-R间期缩短仅指窦性P波、直立的房性异位P′波的P(P′)-R间期≤以下最低值:3岁以下0.08s,4~16岁0.10s,16岁以上0.11s。多表现为PR段缩短。

2.常见原因

(1)L-G-L综合征。

(2)W-P-W综合征。

(3)测量误差和年龄因素:少年儿童的房室结发育尚未完全,具有较快的传导功能。

(4)交感神经张力增高和心率增快:如发烧、运动、甲状腺功能亢进等因素可导致房室结的不应期缩短,传导速度加快。

(5)药物影响:如阿托品、肾上腺素、异丙基肾上腺素、洋地黄、糖皮质激素等均可加速房室传导。

(6)房室结加速传导现象:解剖学上较小的房室结、房室结发育不良、房室结内残存具有较快传导功能的特殊组织。

(7)正常的个体差异:成人短P-R间期并不一定反映传导异常。有报告显示,正常人群中P-R≤0.11s和≥0.21s者各占2\%,≤0.09s和≥0.24s者各占0.6\%、0.8\%。

(8)部分孕妇常出现短P-R间期(图\ref{fig2-2})。

\begin{figure}[!htbp]
 \centering
 \includegraphics[width=4.94792in,height=1.57292in]{./images/Image00037.jpg}
 \captionsetup{justification=centering}
 \caption{孕妇(妊娠34周)出现短P-R间期}
 \label{fig2-2}
  \end{figure} 

(9)等频性干扰性房室分离时出现钩拢现象,常引起假性的短P-R间期(图\ref{fig2-3})。

\begin{figure}[!htbp]
 \centering
 \includegraphics[width=5.78125in,height=1.07292in]{./images/Image00038.jpg}
 \captionsetup{justification=centering}
 \caption{上行显示窦性心动过缓、房室交接性逸搏心律、等频性干扰性房室分离引起房室钩拢现象及伪短P-R间期(下行系活动后记录)}
 \label{fig2-3}
  \end{figure} 

(10)部分室性融合波:室性异位搏动与窦性搏动在心室内融合出现1个或数个较短的P-R间期。

3.L-G-L综合征

窦性激动通过James束下传心室,又称为房-结旁道或房-束(希氏束)旁道或短P-R间期综合征,无房室交接区生理性0.05~0.10s延搁。其特征:①P-R间期≤0.11s;②无“δ”波,QRS波形、时间均正常或呈束支阻滞型;③有反复发作心动过速史。若仅有P-R间期缩短、QRS波形正常,临床上无反复发作心动过速史,则不宜诊断为L-G-L综合征,而应诊断为短P-R间期(图\ref{fig2-4})。

\begin{figure}[!htbp]
 \centering
 \includegraphics[width=5.14583in,height=0.91667in]{./images/Image00039.jpg}
 \captionsetup{justification=centering}
 \caption{特发性心房颤动患者出现房性早搏经James束下传(窦性P-R间期0.14s、房早P′-R间期0.08~0.10s)伴心室内差异性传导、James束内4相阻滞}
 \label{fig2-4}
  \end{figure} 

4.W-P-W综合征

窦性或房性激动通过Kent束下传心室,称为W-P-W综合征,又称为房-室旁道或典型预激综合征,属同源性室性融合波。其心电图特征:①P-R间期多为0.08~0.11s;②有“δ”波;③QRS波群时间≥0.11s;④P-J间期正常(≤0.27s);⑤有继发性ST-T改变。根据胸前导联QRS主波方向,W-P-W综合征分为A、B、C三型(图\ref{fig2-5}、图\ref{fig2-6})。

\begin{figure}[!htbp]
 \centering
 \includegraphics[width=5.78125in,height=1.72917in]{./images/Image00040.jpg}
 \captionsetup{justification=centering}
 \caption{交替性A型预激综合征、一度房室传导阻滞(P-R间期0.24s)}
 \label{fig2-5}
  \end{figure} 

\begin{figure}[!htbp]
 \centering
 \includegraphics[width=4.64583in,height=3.53125in]{./images/Image00041.jpg}
 \captionsetup{justification=centering}
 \caption{扩张型心肌病患者出现右心房肥大(Ⅱ、aVF导联P波电压0.25~0.3mV)、B型预激综合征}
 \label{fig2-6}
  \end{figure} 

\protect\hypertarget{text00008.htmlux5cux23subid32}{}{}

\subsection{P-R间期延长}

(一)引起P-R间期延长的常见原因

正常P-R间期0.12~0.20s,当P-R间期≥0.21s时,便称为P-R间期延长,见于下列6种情况。

1.房室传导延缓

在一般情况下,同一患者在心率相近时,前后两次心电图相互比较,出现P-R间期互差≥0.04~0.05s时,便认为已发生一度房室传导阻滞。既然P-R间期正常最高值为0.20s,似应将P-R间期≥0.24s诊断为一度房室传导阻滞,0.21~0.23s诊断为房室传导延缓更为恰当。

2.一度房室传导阻滞

(1)心电图特征:①P-R间期≥0.24s(儿童≥0.21s);②同一患者在心率基本相近时,其P-R间期有动态变化且≥0.04~0.05s时,即使延长后的P-R间期在正常范围内,也应诊断为一度房室传导阻滞。实际上动态变化的P-R间期延长,其临床意义更大。

(2)阻滞类型:①一度Ⅰ型房室传导阻滞:其P-R间期逐搏延长或逐搏缩短,但未出现心室漏搏(图\ref{fig2-7});②一度Ⅱ型房室传导阻滞:其P-R间期固定地延长,即通常所说的一度房室传导阻滞;③一度Ⅲ型房室传导阻滞:延长的P-R间期长短不一,与迷走神经张力波动有关;④3相性一度房室传导阻滞:心率增快时出现P-R间期延长,而心率减慢或长间歇后P-R间期恢复正常(图\ref{fig2-8});⑤4相性一度房室传导阻滞:心率减慢或长间歇后出现P-R间期延长,而心率增快时P-R间期恢复正常(图\ref{fig2-9});⑥间歇性一度房室传导阻滞:P-R间期延长与心率快、慢无关,可能存在房室结内双径路传导。

\begin{figure}[!htbp]
 \centering
 \includegraphics[width=5.58333in,height=0.78125in]{./images/Image00042.jpg}
 \captionsetup{justification=centering}
 \caption{反向一度Ⅰ型房室传导阻滞(P-R间期由0.38s缩短至0.30s)}
 \label{fig2-7}
  \end{figure} 

\begin{figure}[!htbp]
 \centering
 \includegraphics[width=5.58333in,height=0.45833in]{./images/Image00043.jpg}
 \captionsetup{justification=centering}
 \caption{室性早搏、3相性一度房室传导阻滞(代偿间歇后P-R间期由0.26s缩短至0.15s,但不能排除房室结内双径路传导)}
 \label{fig2-8}
  \end{figure} 

\begin{figure}[!htbp]
 \centering
 \includegraphics[width=5.58333in,height=0.71875in]{./images/Image00044.jpg}
 \captionsetup{justification=centering}
 \caption{窦性心动过缓伴不齐、4相性一度房室传导阻滞(P-R间期有0.20s、0.25s两种)}
 \label{fig2-9}
  \end{figure} 

(3)阻滞部位:一度房室传导阻滞部位可发生在心房内、房-结区、结区、结-希区、希氏束、双束支或三分支水平,其中以房室结内阻滞最为常见,约占90\%。希氏束电图能准确地判断房室传导阻滞的部位。体表心电图亦能初步诊断:①若P-R间期延长是由于P波时间明显增宽所致,且PR段时间正常,则阻滞部位多发生在心房或房-结区部位;②若仅PR段明显延长,则阻滞部位多发生在房室结内;③若P-R间期延长合并左束支阻滞,则阻滞部位绝大多数在双束支水平;④若P-R间期延长合并右束支阻滞,则阻滞部位在房室结内或双束支水平,各占50\%左右。

3.二度Ⅰ型、Ⅱ型房室传导阻滞

(1)典型的二度Ⅰ型房室传导阻滞:又称为房室文氏型阻滞或文氏现象,其P-R间期逐搏延长,直至P波受阻、QRS波群脱落,但每搏延长的增量逐渐缩短;R-R间期逐搏缩短,直至出现一个长R-R间期;长R-R间期小于任何短R-R间期的2倍(图\ref{fig2-10})。

(2)二度Ⅱ型房室传导阻滞:P-R间期固定(正常或延长),直至P波受阻、QRS波群脱落。

\begin{figure}[!htbp]
 \centering
 \includegraphics[width=5.58333in,height=0.83333in]{./images/Image00045.jpg}
 \captionsetup{justification=centering}
 \caption{二度Ⅰ型房室传导阻滞、房室呈4:3传导}
 \label{fig2-10}
  \end{figure} 

4.房室结内双径路传导

当快径路前向阻滞时,窦性激动沿着慢径路下传,出现P-R间期延长。诊断房室结内双径路传导,则要求P-P间期基本规则,出现长、短两种P-R间期且互差≥0.06s(图\ref{fig2-11})。

\begin{figure}[!htbp]
 \centering
 \includegraphics[width=5.58333in,height=1.875in]{./images/Image00046.jpg}
 \captionsetup{justification=centering}
 \caption{上、下两行系MV\textsubscript{5}导联同时不连续记录,显示一度房室传导阻滞、房室结内双径路传导(P-P间期规则时,其P-R间期有0.24s、0.38s两种)}
 \label{fig2-11}
  \end{figure} 


5.干扰性P-R间期延长

(1)窦性早搏、房性早搏、干扰性房室分离时的窦性夺获、间位型室性早搏后第1个窦性搏动落在T波上,下传的P(P′)-R间期较长,且其P(P′)-R间期与R-P(P′)间期呈反比关系,即P(P′)-R间期长,其R-P(P′)间期短;反之,P(P′)-R间期短,其R-P(P′)间期长。若P(P′)波分别落在ST段、T波及TP段上,其下传的P(P′)-R间期呈固定地延长,则应考虑房室结内慢径路下传。

(2)窦性搏动遇及房室交接区隐匿性早搏的相对不应期,将引起个别心搏的P-R间期突然延长,但需有显性房室交接性早搏出现作为佐证。多见于房室交接区并行心律。

6.高度~几乎完全性房室传导阻滞时,出现超常期传导可引起P-R间期显著延长。

(二)临床意义

(1)P-R间期延长可见于心外因素,如抗心律失常药物、电解质紊乱(低钾、高钾血症)、迷走神经张力过高、颅脑损伤等,但更多见于急性心肌炎、心肌缺血、扩张型心肌病等器质性心脏病或功能性改变,如房室结内双径路传导等。大部分预后良好,可终身不变。

(2)突然发生或新发生的一度房室传导阻滞,需进一步查明原因和随访,警惕发展为高度或三度房室传导阻滞。

(3)P-R间期过度延长时(>0.35s),可发生P-R间期过度延长综合征。此时心室舒张期及有效充盈期均显著缩短而引起二尖瓣返流及心功能不全,可置入双腔起搏器;若由房室结慢径路下传者,可行射频消融术。

\protect\hypertarget{text00008.htmlux5cux23subid33}{}{}

\subsection{P-R间期长、短交替}

窦性心律时,出现P-R间期长、短交替,见于下列5种情况:

(1)房室交接区快、慢径路交替性传导:当P-P间期基本规则时,P-R间期呈长、短交替出现,且两者互差≥0.06s。

(2)交替性预激:短P-R间期者,QRS波群起始部有“δ”波,QRS波群时间增宽,ST-T呈继发性改变,其P-J间期与正常P-R间期、正常QRS波群的P-J间期相等。系房室旁道呈2:1阻滞所致(图\ref{fig2-12})。

\begin{figure}[!htbp]
 \centering
 \includegraphics[width=5.78125in,height=0.59375in]{./images/Image00047.jpg}
 \captionsetup{justification=centering}
 \caption{一度房室传导阻滞、交替性A型预激综合征引起P-R间期长、短交替(与图\ref{fig2-5}系同一病例)}
 \label{fig2-12}
  \end{figure} 

(3)3:2房室文氏现象:文氏周期中,第2个搏动的P-R间期较第1个长,第3个搏动P波下传受阻,导致P-R间期长、短交替出现。

(4)房室结快、慢径路均呈3:1传导:第1个搏动由快径路下传,第2个搏动由慢径路下传,第3个搏动快、慢径路下传均受阻,导致P-R间期长、短交替出现(图\ref{fig2-13})。

\begin{figure}[!htbp]
 \centering
 \includegraphics[width=5.58333in,height=0.79167in]{./images/Image00048.jpg}
 \captionsetup{justification=centering}
 \caption{房室结快、慢径路均呈3:1传导,引起P-R间期长、短交替(0.17s、0.47~0.53s)}
 \label{fig2-13}
  \end{figure} 

(5)舒张晚期室性早搏二联律:短P-R间期者,QRS波群宽大畸形,ST-T呈继发性改变,其P-J间期与正常P-R间期、正常QRS波群的P-J间期不等。

\protect\hypertarget{text00008.htmlux5cux23subid34}{}{}

\section{P-J间期}

P-J间期是指P波开始到J点结束,代表心房开始除极到心室除极结束所需的时间,包括P-R间期和QRS波群时间,正常值≤0.27s。当发生一度房室传导阻滞、束支阻滞、不定型心室内传导阻滞等心室除极时间延长时,则P-J间期>0.27s。P-J间期在预激综合征诊断和鉴别诊断中具有重要意义。

(1)间歇性预激综合征(不完全性预激)与舒张晚期室性早搏的鉴别:两者P-R间期均缩短,QRS波群时间均增宽,但前者的P-J间期与正常QRS波群的P-J间期相等,而后者则不相等。

(2)预激综合征P-J间期>0.27s时,多数合并束支阻滞,少数可能合并房室旁道一度阻滞或同时伴有正道一度、三度阻滞。P-J间期包括P-R间期与QRS波群时间,绝大多数预激综合征的P-J间期正常(≤0.27s)。因束支阻滞的P-J间期等于房室结传导时间+希浦系传导时间+束支阻滞部位的心室终末除极时间,故一般均>0.27s;而预激综合征合并束支阻滞的P-J间期等于房室旁道传导时间+希浦系传导时间+束支阻滞部位的心室终末除极时间。两者相比,后者的P-J间期较前者略短。当房室旁道下传时间与心室间及心室内传导时间的总和≥房室结传导时间与心室间及心室内传导时间的总和时,束支阻滞图形不会被掩盖,此时P-J间期延长(>0.27s)。房室旁道一度阻滞或同时伴有正道一度、三度阻滞,也将导致P-J间期>0.27s。

(3)不同部位预激综合征对P-J间期的影响:大多数预激综合征的P-J间期正常(≤0.27s),少数预激综合征(右后、右后间隔旁道)的P-J间期明显短于正常QRS波群的P-J间期,因P-J间期包括冲动从正道下传的房室传导时间和心室除极时间,旁道传导不影响冲动从正道下传的房室传导时间,但可缩短心室除极时间,故可引起P-J间期缩短,尤其是房室旁道位置靠近心室最后除极部位(心室后基底部)及δ波较大者。

\protect\hypertarget{text00009.html}{}{}

\protect\hypertarget{text00009.htmlux5cux23chapter9}{}{}

\chapter{正常QRS波群及其异常改变}

\protect\hypertarget{text00009.htmlux5cux23subid35}{}{}

\section{正常QRS波群}

\protect\hypertarget{text00009.htmlux5cux23subid36}{}{}

\subsection{QRS波群的命名}

QRS波群是室间隔、右心室和左心室电激动过程中所产生的除极波。第1个向下的波称为Q(q)波,最初1个向上的波称为R(r)波,R(r)波之后向下的波称为S(s)波,有时S波之后又出现1个向上的波,则称为R′(r′)波,之后再出现一个向下的波,称为S′(s′)波;若只有向下的波,而没有向上的波,称为QS波。当波幅≥0.5mV时,用Q、R、S表示;当波幅<0.5mV时,用q、r、s表示。

\protect\hypertarget{text00009.htmlux5cux23subid37}{}{}

\subsection{各波的正常值}

1.Q(q)波

正常q波时间<0.04s,深度<$\frac{1}{4}$
R。若其时间≥0.04s或(和)深度≥$\frac{1}{4}$
R,则称为异常Q波。

2.R波振幅

(1)肢体导联:①心脏呈横位型时,R\textsubscript{I} +S\textsubscript{Ⅲ}<2.5mV,R\textsubscript{I} <1.5mV,R\textsubscript{aVL}
<1.2mV;②心脏呈悬位型时,R\textsubscript{Ⅱ、Ⅲ、aVF}
<2.0mV;③aVR导联Q/R>1,R<0.5mV;④所有肢体导联R+S>0.5mV。

(2)胸前导联:①R\textsubscript{V\textsubscript{1}}
<1.0mV,R\textsubscript{V\textsubscript{1}}
+S\textsubscript{V\textsubscript{5}} <1.2mV,V\textsubscript{1}导联R/S<1.②建议国人采用男性R\textsubscript{V\textsubscript{5}}
、\textsubscript{V\textsubscript{6}}
<3.0mV、R\textsubscript{V\textsubscript{5}}
+S\textsubscript{V\textsubscript{1}}
<4.5mV;女性R\textsubscript{V\textsubscript{5}}
、\textsubscript{V\textsubscript{6}}
<2.8mV、R\textsubscript{V\textsubscript{6}}
+S\textsubscript{V\textsubscript{1}} <4.0mV,V\textsubscript{5}、V\textsubscript{6} 导联R/S>1.③V\textsubscript{3}导联R+S<6.0mV。④所有胸前导联R+S>1.0mV。

3.QRS波群时间

<4岁的儿童,QRS波群时间<0.09s;4~16岁者,QRS波群时间<0.10s;成年男性,QRS波群时间≤0.11s。“心电图标准化与解析的建议------2009年国际指南”(以下简称“2009年国际指南”)推荐≥16岁者,QRS波群时间>0.11s为异常。

\protect\hypertarget{text00009.htmlux5cux23subid38}{}{}

\section{QRS波群振幅异常改变}

\protect\hypertarget{text00009.htmlux5cux23subid39}{}{}

\subsection{低电压}

1.心电图特征

所有肢体导联R+S<0.5mV或胸前导联R+S<1.0mV。

2.临床意义

(1)心外因素:见于肺气肿、胸腔积液或积气、心包积液、过度肥胖、甲状腺功能减退等。

(2)心内因素:①心肌梗死:大面积心肌梗死者,出现低电压,提示预后不良;②扩张型心肌病;③心力衰竭。

(3)正常人群:约有1\%的正常人可出现低电压。

\protect\hypertarget{text00009.htmlux5cux23subid40}{}{}

\subsection{高电压}

1.右胸导联高电压

(1)右心室肥大。

(2)右束支阻滞:①V\textsubscript{1}导联QRS波群呈rsR′型或M型;②其他导联终末S波或R波宽钝错折;③QRS波群时间>0.11s。

(3)Ⅱ型左中隔支阻滞:①V\textsubscript{1} 、V\textsubscript{2}导联QRS波群呈Rs型,R/s>1;②V\textsubscript{5} 、V\textsubscript{6}导联QRS波群呈Rs型或qRs型,其q波很小,时间<0.01s,深度<0.1mV;③R\textsubscript{V\textsubscript{2}}
>R\textsubscript{V\textsubscript{6}}
;④QRS波群时间正常(合并束支阻滞时除外);⑤多见于老年冠心病患者;⑥需排除右心室肥大、逆钟向转位、A型预激综合征、后壁心肌梗死等(图\ref{fig3-1})。

\begin{figure}[!htbp]
 \centering
 \includegraphics[width=2.98958in,height=2.91667in]{./images/Image00050.jpg}
 \captionsetup{justification=centering}
 \caption{V\textsubscript{1} ~V\textsubscript{6}导联定准电压为0.5mV,高血压病患者出现Ⅱ型左中隔支阻滞、左心室肥大伴劳损}
 \label{fig3-1}
  \end{figure} 


(4)A、C型预激综合征(图\ref{fig3-2})。

\begin{figure}[!htbp]
 \centering
 \includegraphics[width=4.96875in,height=1.77083in]{./images/Image00051.jpg}
 \captionsetup{justification=centering}
 \caption{心动过速患者体检时发现C型预激综合征}
 \label{fig3-2}
  \end{figure} 

(5)后壁心肌梗死:①V\textsubscript{3} R、V\textsubscript{1}、V\textsubscript{2}导联R波增高,呈Rs型伴ST段压低、T波高耸;②V\textsubscript{7}、V\textsubscript{8}导联出现异常Q波,呈QR、Qr、QS型伴ST段抬高、T波倒置。

(6)逆钟向转位:①V\textsubscript{1} ~V\textsubscript{3}导联呈Rs型或RS型,R/S>1;②V\textsubscript{5} 、V\textsubscript{6}呈qR、Rs型。

(7)右心室电压占优势:①见于婴幼儿、儿童;②心脏无病理性杂音;③电轴右偏;④V\textsubscript{1}、V\textsubscript{2} 导联呈Rs型,R/s>1。

2.左胸导联、肢体导联高电压

(1)左心室高电压:①R\textsubscript{Ⅰ} +S\textsubscript{Ⅲ}>2.5mV,R\textsubscript{aVL}
>1.2mV,见于横位型心脏、肥胖者;②R\textsubscript{Ⅱ、Ⅲ、aVF}
>2.0mV,见于悬位型心脏、瘦长型者;③男性R\textsubscript{V\textsubscript{5}}
+S\textsubscript{V\textsubscript{1}}
>4.5mV、女性R\textsubscript{V\textsubscript{5}}
+S\textsubscript{V\textsubscript{1}}
>4.0mV;④男性R\textsubscript{V\textsubscript{5}}
、\textsubscript{V\textsubscript{6}}
>3.0mV、女性R\textsubscript{V\textsubscript{5}}
、\textsubscript{V\textsubscript{6}} >2.8mV;⑤男性R\textsubscript{aVL}
+S\textsubscript{V\textsubscript{3}} >2.8mV、女性R\textsubscript{aVL}
+S\textsubscript{V\textsubscript{3}} >2.0mV。

(2)左心室肥大。

3.左、右胸导联均为高电压

(1)A型预激综合征。

(2)双心室肥大。

(3)左心室肥大伴逆钟向转位。

(4)左心室肥大合并Ⅱ型左中隔支阻滞(图\ref{fig3-1})。

\protect\hypertarget{text00009.htmlux5cux23subid41}{}{}

\subsection{右心室肥大}

在正常情况下,左心室壁较右心室壁约厚3倍。轻度右心室肥大所增加的向右前向量往往被左心室除极向量所抵消,其肥大的图形被掩盖。只有右心室显著肥大,且其心室壁厚度大于左心室时,才表现出右心室肥大的心电图特征。

1.心电图特征

(1)电轴右偏>+110°,R\textsubscript{Ⅲ} >R\textsubscript{aVF} 。

(2)aVR导联呈QR型,Q/R<1,R波幅>0.5mV。

(3)V\textsubscript{1} 导联呈qR、qRs、R、Rs、rsR′(R′波不宽钝)型。

(4)V\textsubscript{5} 、V\textsubscript{6} 导联呈RS型,R/S<1.

(5)出现肺型P波及V\textsubscript{1} ~V\textsubscript{6}导联均呈rS型,r/S<1,多见于肺心病。

(6)V\textsubscript{1} ~V\textsubscript{3}导联可有ST段压低,T波呈负正双向或倒置(图\ref{fig3-3})。

\begin{figure}[!htbp]
 \centering
 \includegraphics[width=5.78125in,height=1.40625in]{./images/Image00052.jpg}
 \captionsetup{justification=centering}
 \caption{法洛四联症患者出现右心房、右心室肥大、下壁及前侧壁轻度T波改变}
 \label{fig3-3}
  \end{figure} 

2.根据心电图特征分型

(1)轻度肥大:电轴轻度右偏、V\textsubscript{1}导联呈rsR′型,V\textsubscript{2} ~V\textsubscript{6}导联均呈rS型。多见于房间隔缺损、肺源性心脏病等。

(2)中度肥大:电轴中度右偏、V\textsubscript{1}导联呈Rs或RS型、V\textsubscript{5} 、V\textsubscript{6}呈RS或rS型。多见于室间隔缺损、风心病二尖瓣狭窄等。

(3)重度肥大:电轴重度右偏,V\textsubscript{1}导联呈qR、qRs、R型,V\textsubscript{5} 、V\textsubscript{6}导联呈rS型。多见于法洛四联症、肺动脉瓣狭窄等。

3.根据右心室负荷过重情况分型

(1)收缩期负荷过重型:系右心室射血时阻力增加,心肌发生代偿性肥厚所致。V\textsubscript{1}导联QRS波群呈qR、qRs、R、Rs型;V\textsubscript{1} ~V\textsubscript{3}导联ST段压低、T波倒置。多见于法洛四联症、肺动脉瓣狭窄。

(2)舒张期负荷过重型:右心室回心血量增多,使右心室舒张期负荷增加而扩张。V\textsubscript{1}导联QRS波群呈rsR′型。多见于房间隔缺损等。

4.右心室肥大合并右束支阻滞

右心室压力增高导致右心室肥大,易伤及右束支使其阻滞。表现为QRS波群时间≥0.12s,电轴右偏,aVR导联R波增宽、增高,Ⅰ、aVL、V\textsubscript{5}、V\textsubscript{6} 导联S波增宽。

\protect\hypertarget{text00009.htmlux5cux23subid42}{}{}

\subsection{左心室肥大}

1.心电图特征

(1)有左心室高电压的心电图表现。

(2)QRS波群时间轻度增宽(0.10~0.12s)。

(3)电轴轻、中度左偏(+30°~-30°)。

(4)以R波为主导联出现轻度ST段压低(<0.1mV)、T波低平,系继发性ST-T改变。

(5)有引起左心室肥大的临床依据。

2.根据心肌肥厚部位、心腔大小分型

(1)向心性肥厚:心室壁增厚,心腔不扩大。

(2)离心性肥厚:心腔扩大,心室腔与心室壁的比值不增加。

(3)扩张型心室肥厚:心腔不成比例增大,心室壁与心室腔比值缩小,心脏重量增加。

(4)非对称性流出道狭窄。

3.根据左心室负荷过重分型

(1)收缩期负荷过重型:系左心室射血时阻力增加,心肌发生代偿性向心性肥厚。左胸导联R波振幅增高伴ST段压低、T波低平或倒置。多见于高血压病、主动脉瓣狭窄及梗阻型肥厚性心肌病等。

(2)舒张期负荷过重型:系左心室回心血量增多,使左心室舒张期负荷增加而扩张,为离心性或扩张型心室肥厚。左胸导联R波振幅增高伴ST段抬高、T波高耸。多见于二尖瓣关闭不全、主动脉瓣关闭不全等。

4.左心室肥大伴劳损

有左心室肥大的心电图特征,同时伴有原发性ST-T改变,即ST段呈下斜型、水平型压低(≥0.1mV),T波倒置或负正双向以负为主,可伴有U波倒置(图\ref{fig3-4})。“2009年国际指南”指出不再应用“劳损”,而改称为“继发性ST-T改变”,本人持有异议。

\begin{figure}[!htbp]
 \centering
 \includegraphics[width=3.35417in,height=3.625in]{./images/Image00053.jpg}
 \captionsetup{justification=centering}
 \caption{主动脉瓣狭窄患者出现左心室肥大伴劳损、左前分支阻滞(定准电压均为0.5mV)}
 \label{fig3-4}
  \end{figure} 

5.左心室劳损

有左心室肥大的临床依据,但心电图QRS波群振幅正常,仅出现原发性ST-T改变者,可伴有U波倒置,称为左心室劳损。

\protect\hypertarget{text00009.htmlux5cux23subid43}{}{}

\subsection{双心室肥大}

1.双心室肥大的心电图表现形式

(1)心电图正常或大致正常。

(2)仅有QRS波群时间轻度增宽及轻度ST-T改变。

(3)仅显示左心室肥大的心电图改变,此时右心室呈轻、中度肥大。

(4)仅显示右心室肥大的心电图改变,此时右心室显著肥大。

(5)同时显示双心室肥大的心电图改变,左、右胸导联R波振幅均增高。

2.有明确的左心室肥大的心电图特征,同时伴有下列一项或几项改变者,应提示双心室肥大

(1)电轴右偏(>+110°,图\ref{fig3-5})。

\begin{figure}[!htbp]
 \centering
 \includegraphics[width=3.52083in,height=4.67708in]{./images/Image00054.jpg}
 \captionsetup{justification=centering}
 \caption{风心病、二尖瓣狭窄伴关闭不全,定准电压均为0.5mV。显示心房颤动、左心室肥大(R\textsubscript{V\textsubscript{5}}电压4.0mV)、电轴右偏(+120°)(提示双心室肥大)、不完全性右束支阻滞}
 \label{fig3-5}
  \end{figure} 


(2)aVR导联Q/R<1,R>0.5mV。

(3)V\textsubscript{1} 导联呈Rs型,R/s>1或呈R型。

(4)显著的顺钟向转位,V\textsubscript{5} 、V\textsubscript{6}导联有深的S波。

(5)出现肺型P波,系右心房肥大所致(图\ref{fig3-6})。

\begin{figure}[!htbp]
 \centering
 \includegraphics[width=3.55208in,height=3.69792in]{./images/Image00055.jpg}
 \captionsetup{justification=centering}
 \caption{风心病、双瓣膜病变患者心电图出现左心室、右心房肥大(定准电压均为0.5mV,心脏超声波、胸片均显示全心扩大)}
 \label{fig3-6}
  \end{figure} 

3.有明确的右心室肥大的心电图特征,同时伴有下列一项或几项改变者,应提示双心室肥大

(1)电轴左偏。

(2)V\textsubscript{5} 、V\textsubscript{6}导联R波振幅>2.5mV伴ST-T改变。

(3)V\textsubscript{3} 导联R+S>6.0mV,R/S≈1.

(4)男性R\textsubscript{aVL} +S\textsubscript{V\textsubscript{3}}
>2.8mV、女性R\textsubscript{aVL} +S\textsubscript{V\textsubscript{3}}
>2.0mV。

(5)R\textsubscript{Ⅰ} +S\textsubscript{Ⅲ}>2.5mV,R\textsubscript{aVL} >1.2mV。

\protect\hypertarget{text00009.htmlux5cux23subid44}{}{}

\subsection{心室肥厚、扩大、肥大的区别}

1.心室肥厚:主要指心肌细胞增粗、增长所致心室壁厚度增加、重量增加,但心室腔容积不增大。

2.心室扩大:主要指心室腔的容积增大,可伴有轻度心室壁增厚。

3.心室肥大:肥厚与扩大均兼有之。病程较久的病例,无论是收缩期负荷过重,还是舒张期负荷过重,往往存在着不同程度的心室壁增厚和心腔容积增大,故本人主张用“心室肥大”一词。

\protect\hypertarget{text00009.htmlux5cux23subid45}{}{}

\subsection{异常Q波}

1.异常Q波

(1)Q波时间≥0.04s。

(2)Q波深度≥$\frac{1}{4}$ R。

(3)呈QS型,起始部错折或呈QrS、Qrs或qrS型(该r波又称为胚胎型r波)。

2.等位性Q波(属异常Q波范畴)

(1)原无q波的导联上突然出现了q波伴ST段损伤型抬高。

(2)以R波为主导联,其R波振幅较原来显著降低伴ST段损伤型抬高。

(3)V\textsubscript{1} ~V\textsubscript{4}导联r(R)波振幅逐渐降低,呈逆递增现象。

(4)V\textsubscript{1} ~V\textsubscript{4}导联r(R)波振幅递增不良:相邻两个导联的r(R)波振幅递增量<0.1mV。

(5)不应该出现Q(q)波的导联出现了Q(q)波,如V\textsubscript{1}、V\textsubscript{2} 导联呈qrS型,或V\textsubscript{3}、V\textsubscript{4} 导联出现q波而V\textsubscript{5}、V\textsubscript{6} 导联无q波,或V\textsubscript{3}、V\textsubscript{4} 导联q波深度>V\textsubscript{5}、V\textsubscript{6} 导联q波深度。

(6)镜像改变,如V\textsubscript{1} 、V\textsubscript{2}导联R波振幅增高,而V\textsubscript{7} 、V\textsubscript{8}导联出现异常Q波。

(7)左束支阻滞时,Ⅰ、aVL、V\textsubscript{5} 、V\textsubscript{6}导联呈qR型。

3.引起异常Q波的常见原因

(1)急性心肌梗死引起心肌细胞组织学上坏死或电学上的电静止。

(2)陈旧性心肌梗死或心肌病患者出现心肌纤维化。

(3)显著右心室肥大导致心脏顺钟向转位,使Ⅰ、aVL导联或V\textsubscript{1}、V\textsubscript{2} 导联呈QS型、qR型。

(4)Ⅰ型左中隔支阻滞导致V\textsubscript{1} 、V\textsubscript{2}导联呈qR、QR、qrS或QS型。只有间歇性出现时,方能诊断(图\ref{fig21-9})。

(5)预激综合征时“δ”波呈负相时,可使不同导联出现异常Q波。

4.高侧壁(Ⅰ、aVL导联)出现异常Q波

(1)高侧壁或广泛前壁心肌梗死。

(2)预激向量指向右下方的预激综合征。

(3)显著的右心室肥大。

(4)右位心。

5.下壁(Ⅱ、Ⅲ、aVF导联)出现异常Q波

(1)下壁心肌梗死。

(2)左束支阻滞合并显著的电轴左偏:Ⅱ、Ⅲ、aVF导联可呈QS型,Ⅲ导联QS波深度大于Ⅱ导联QS波深度。

(3)预激向量指向左上方的预激综合征。

(4)二尖瓣脱垂:①Ⅱ、Ⅲ、aVF导联可呈QS型,且出现ST段压低、T波倒置;②听诊有喀喇音;③超声心动图显示二尖瓣脱垂的特征性改变。

6.右胸导联(V\textsubscript{1} 、V\textsubscript{2} 导联)出现异常Q波

(1)前间壁心肌梗死。

(2)室间隔肥厚性心肌病导致心肌纤维化。

(3)左束支阻滞。

(4)Ⅰ型左中隔支阻滞:V\textsubscript{1} 、V\textsubscript{2}导联呈qR、QR、qrS或QS型,需排除右心室肥大、前间壁心肌梗死、前间壁心肌纤维化及B型预激综合征。间歇性出现上述波形时,能明确诊断。

(5)肺气肿、肺心病。

(6)显著的右心室肥大。

(7)B型预激综合征。

(8)部分左心室肥大伴劳损。

(9)左前分支阻滞。

7.左胸导联(V\textsubscript{4} 、V\textsubscript{5} 、V\textsubscript{6}导联)出现异常Q波

(1)前壁或前侧壁心肌梗死。

(2)肥厚性梗阻型心肌病出现深而窄的Q波。

(3)左心室舒张期负荷过重导致左心室肥大时,可出现深而窄的Q波。

(4)C型预激综合征。

(5)右位心。

\protect\hypertarget{text00009.htmlux5cux23subid46}{}{}

\subsection{QRS波群电交替、电阶梯现象}

1.基本概念

(1)QRS波群电交替现象:系指源自同一起搏点的心搏(多为窦性节律),在排除2:1分支阻滞及心外因素影响下,其QRS波群时间不变,仅波形或(和)波幅每搏呈交替性改变。可同时伴有其他波、段的电交替。

(2)QRS波群电阶梯现象:系一种特殊的电交替现象,指源自同一起搏点的心搏(多为窦性节律),在排除心外因素影响及分支内文氏现象下,其QRS波群时间不变,仅波形或(和)波幅由浅→深→浅或由低→高→低,周而复始,有规律地演变。可同时伴有ST段、T波的电阶梯现象。

2.心电图特征

(1)QRS波群电交替现象:①心搏来源恒一,多为窦性节律;②QRS波群时间固定不变;③任何导联上QRS波幅相差≥0.1mV,以胸前导联为多见,尤以V\textsubscript{2}、V\textsubscript{3}导联最常见;④心率增快时(>100次/min),尤其是阵发性心动过速时更易出现,为快频率依赖性电交替;⑤与束支、分支阻滞无关,与心外因素无关,如呼吸、体位、胸腔积液等;⑥若同时伴有≥2个其他波、段的电交替,则称为完全性电交替现象(图\ref{fig3-7}、图\ref{fig3-8})。

\begin{figure}[!htbp]
 \centering
 \includegraphics[width=4.19792in,height=0.4375in]{./images/Image00056.jpg}
 \captionsetup{justification=centering}
 \caption{QRS波群、T波电交替现象}
 \label{fig3-7}
  \end{figure} 

\begin{figure}[!htbp]
 \centering
 \includegraphics[width=5.78125in,height=1.17708in]{./images/Image00057.jpg}
 \captionsetup{justification=centering}
 \caption{阵发性室上性心动过速出现QRS波群、T波电交替现象}
 \label{fig3-8}
  \end{figure} 

(2)QRS波群电阶梯现象:①心搏来源恒一,多为窦性节律;②QRS波群时间固定不变;③任何导联上QRS波形或(和)波幅由浅→深→浅或由低→高→低,振幅相差≥0.1mV,周而复始,有规律地演变;④与束支、分支阻滞无关,与心外因素无关,如呼吸、体位、胸腔积液等;⑤可同时伴有ST段、T波的电阶梯现象(图\ref{fig3-9})。

\begin{figure}[!htbp]
 \centering
 \includegraphics[width=5.78125in,height=0.51042in]{./images/Image00058.jpg}
 \captionsetup{justification=centering}
 \caption{加速的房性逸搏心律伴James束下传(P-R间期0.08s)、QRS波群电阶梯现象}
 \label{fig3-9}
  \end{figure} 

3.发生机制

(1)多与心肌、传导组织不同程度的缺血、缺氧引起不应期延长,导致心肌细胞除极、复极不完全有关,尤其是心室率过快导致心室舒张期明显缩短时。

(2)顺向型房室折返性心动过速时,其QRS波群电交替发生率较高,这与冲动在传导组织内发生交替性功能性传导延缓有关。

4.临床意义

(1)QRS波群电交替可见于大量心包积液或心包填塞、严重的心肌病变,如冠心病、心肌梗死、扩张型心肌病等。若发生在心率缓慢时,则提示心肌病变严重,预后较差;若发生在心动过速时,则无特别临床意义,随着心动过速的终止,QRS波群电交替自行消失。

(2)窄QRS心动过速伴有QRS波群电交替,对判断顺向型房室折返性心动过速具有高度的特异性(96\%)。

(3)宽QRS心动过速伴QRS波群电交替者,多有合并房室旁道传导。

(4)QRS波群电阶梯现象多见于严重的器质性心脏病、高钾血症等,提示心肌病变严重而广泛,预后较差,但与原发病有关。

\protect\hypertarget{text00009.htmlux5cux23subid47}{}{}

\section{QRS波群电轴偏移}

\protect\hypertarget{text00009.htmlux5cux23subid48}{}{}

\subsection{电轴测量方法及其分类标准}

1.电轴测量方法

通常测量额面心电轴。根据肢体六个导联QRS波群振幅的高低,可用目测法进行初步评估电轴是左偏、右偏还是不偏;而用振幅查表法,则能较准确地求出电轴偏移的度数。

(1)粗略的目测法:根据Ⅰ、Ⅲ(或aVF)导联QRS主波方向加以判断,因Ⅰ、aVF导联轴的夹角为90°,故以Ⅰ、aVF导联目测心电轴较为恰当:①Ⅰ、Ⅲ(或aVF)导联QRS主波均向上,电轴正常;②Ⅰ导联QRS主波向上,Ⅲ(或aVF)导联QRS主波向下(又称为背道而驰),电轴左偏;③Ⅰ导联QRS主波向下,Ⅲ(或aVF)导联QRS主波向上(又称为针锋相对),电轴右偏;④Ⅰ、Ⅲ(或aVF)导联QRS主波均向下,电轴极度右偏(又称为无人区电轴)。

(2)较为精确的目测法:要求熟悉六轴系统各个导联轴所在的位置与角度。找出R波振幅最高和次高的导联,则电轴就位于这两个导联轴之间,且偏向波幅最高的导联轴。若最高和次高导联QRS波幅相等时,则电轴就位于这两个导联轴的中间。这种方法一般误差在5°以内。例如,Ⅱ导联(+60°)R波振幅最高,aVF导联(+90°)次高,则电轴位于+60°~+90°之间偏向+60°,约在+65°~+70°;若Ⅱ、aVF导联R波振幅最高,且相等时,则电轴位于+60°~+90°中间,即+75°左右。

(3)查表法:根据Ⅰ和Ⅲ导联QRS波幅的代数和进行查表,有2种方法:①计算QRS波群正相波振幅最高与负相波最深的代数和,如R-S或R-Q,其方法简便,也更为精确;②计算QRS波群所有向上和向下各波幅的代数和,如(R+R′)-(Q+S)。

2.分类标准

(1)目前国内常用标准:①+30°~+90°,电轴正常;②+30°~0°,电轴轻度左偏;③0°~-30°,电轴中度左偏;④-30°~-90°,电轴重度左偏;⑤+90°~+120°,电轴轻度右偏;⑥+120°~+180°,电轴中度右偏;⑦+180°~-90°,电轴重度右偏。

(2)世界卫生组织推荐的标准:①-30°~+90°,电轴正常;②-30°~-90°,电轴左偏;③+90°~+180°,电轴右偏;④-90°~+180°,电轴不确定。

(3)“2009年国际指南”的标准:①-30°~+90°,电轴正常;②-30°~-45°,中度左偏;③-45°~-90°,显著左偏;④+90°~+120°,中度右偏;⑤+120°~+180°,显著右偏。

\protect\hypertarget{text00009.htmlux5cux23subid49}{}{}

\subsection{电轴偏离的临床意义}

(1)电轴轻度左偏、右偏:多属于正常变异。

(2)电轴中、重度左偏:见于左前分支阻滞、左心室肥大、原发孔型房间隔缺损、预激综合征、横位型心脏等。

(3)电轴中度右偏:见于左后分支阻滞、右心室肥大、高侧壁心肌梗死等。

(4)电轴重度右偏或不确定者:又称为假性电轴左偏或无人区电轴。窦性心律时见于重度右心室肥大、S\textsubscript{Ⅰ}S\textsubscript{Ⅱ} S\textsubscript{Ⅲ}综合征、右心室内传导延缓等;宽QRS波群时见于室性早搏、室性心动过速,具有很高的特异性。

\protect\hypertarget{text00009.htmlux5cux23subid50}{}{}

\subsection{真性、假性电轴左偏的鉴别}

假性电轴左偏与真性电轴左偏的鉴别,见表3-1所示。

\begin{table}[htbp]
\centering
\caption{假性电轴左偏与真性电轴左偏的鉴别}
\label{tab3-1}
\includegraphics[width=5.46875in,height=1.51042in]{./images/Image00059.jpg}
\end{table}

\protect\hypertarget{text00009.htmlux5cux23subid51}{}{}

\subsection{S\textsubscript{Ⅰ} S\textsubscript{Ⅱ} S\textsubscript{Ⅲ} 综合征}

1.基本概念

S\textsubscript{Ⅰ} S\textsubscript{Ⅱ} S\textsubscript{Ⅲ}综合征是指Ⅰ、Ⅱ、Ⅲ导联QRS波群同时存在明显的S波,其深度>0.3mV,且S\textsubscript{Ⅱ}>S\textsubscript{Ⅲ} ,又称为3S综合征。

2.心电图特征

(1)Ⅰ、Ⅱ、Ⅲ导联QRS波群中均有明显的S波。

(2)S波振幅>0.3mV。

(3)S\textsubscript{Ⅱ} >S\textsubscript{Ⅲ} 。

(4)aVR导联Q/R<1,R>0.5mV。

(5)V\textsubscript{5} 、V\textsubscript{6}导联呈RS型,R/S<1或S>$\frac{1}{2}$
R,呈高度顺钟向转位。

(6)上述心电图一旦出现,常持续存在(图\ref{fig3-10})。

\begin{figure}[!htbp]
 \centering
 \includegraphics[width=5.1875in,height=2.14583in]{./images/Image00060.jpg}
 \captionsetup{justification=centering}
 \caption{S\textsubscript{Ⅰ} S\textsubscript{Ⅱ} S\textsubscript{Ⅲ}综合征、顺钟向转位}
 \label{fig3-10}
  \end{figure} 


3.临床意义

(1)正常变异:可见于少数正常人,尤其是瘦长无力型的人群中,与右心室传导延缓有关。

(2)右心室肥大:各种病因引起的严重右心室肥大,特别是右心室漏斗部、右心室流出道肥厚出现右心室电势占优势时,容易出现典型的S\textsubscript{Ⅰ}S\textsubscript{Ⅱ} S\textsubscript{Ⅲ} 综合征。

(3)心肌梗死:各部位的心肌梗死,尤其是心尖部梗死更易出现典型的S\textsubscript{Ⅰ}S\textsubscript{Ⅱ} S\textsubscript{Ⅲ} 综合征。

(4)脊柱畸形:多见于直背综合征患者。

\protect\hypertarget{text00009.htmlux5cux23subid52}{}{}

\subsection{左前分支阻滞}

1.心电图特征

(1)Ⅰ、aVL导联QRS波群呈qR型,R\textsubscript{aVL}
>R\textsubscript{Ⅰ、aVR} ,Ⅱ、Ⅲ、aVF导联呈rS型,S\textsubscript{Ⅲ}>S\textsubscript{Ⅱ} >r\textsubscript{Ⅱ} 。

(2)心电轴左偏>-45°,有的学者认为心电轴左偏>-30°,即可诊断。

(3)V\textsubscript{1} ~V\textsubscript{6}导联R波振幅降低,V\textsubscript{3} ~V\textsubscript{6}导联S波加深呈RS型,有时V\textsubscript{1} 、V\textsubscript{2}导联出现q波,呈qrS型。

(4)QRS波群时间正常(图\ref{fig3-11})。

\begin{figure}[!htbp]
 \centering
 \includegraphics[width=3.84375in,height=1.8125in]{./images/Image00061.jpg}
 \captionsetup{justification=centering}
 \caption{左前分支阻滞及V\textsubscript{5} 、V\textsubscript{6}导联S波加深}
 \label{fig3-11}
  \end{figure} 


2.临床意义

左前分支阻滞约85\%由冠心病引起。此外,左心室肥大常合并左前分支阻滞。

\protect\hypertarget{text00009.htmlux5cux23subid53}{}{}

\subsection{左后分支阻滞}

1.心电图特征

(1)Ⅰ、aVL导联QRS波群呈rS型,S\textsubscript{aVL} >S\textsubscript{Ⅰ},Ⅱ、Ⅲ、aVF导联呈qR型,RⅢ>R\textsubscript{Ⅱ} 。

(2)心电轴右偏>+110°。若出现交替性或间歇性电轴右偏,又具有左后分支阻滞特征,即使未达到+110°,亦可诊断为左后分支阻滞。

(3)QRS波群时间正常。

(4)需排除右心室肥大、侧壁心肌梗死、悬位型心脏等。

2.临床意义

左后分支阻滞的发生率远低于左前分支阻滞。但一旦出现,则提示病变较广泛而严重。

\protect\hypertarget{text00009.htmlux5cux23subid54}{}{}

\section{QRS波群时间、形态异常改变}

\protect\hypertarget{text00009.htmlux5cux23subid55}{}{}

\subsection{左束支传导阻滞}

1.心电图特征

(1)V\textsubscript{1} 、V\textsubscript{2}导联QRS波群呈rS型或QS型,V\textsubscript{5} 、V\textsubscript{6}导联呈R型,R波平顶、挫折。

(2)Ⅰ、aVL导联QRS波群可呈R型或rS型,Ⅱ、Ⅲ、aVF导联可呈rS型或R型、qR型,心电轴可正常、左偏或右偏。

(3)QRS波群时间≥0.12s,多数达0.16s左右。

(4)ST-T方向多数与QRS主波方向相反,呈继发性改变。

(5)若QRS波群时间≥0.12s,则为完全性左束支传导阻滞;若QRS波群时间0.10~0.11s,则为不完全性左束支传导阻滞,但较罕见(图\ref{fig3-12})。

\begin{figure}[!htbp]
 \centering
 \includegraphics[width=5.58333in,height=2.29167in]{./images/Image00062.jpg}
 \captionsetup{justification=centering}
 \caption{不完全性左束支传导阻滞(QRS波群时间0.11~0.12s)}
 \label{fig3-12}
  \end{figure} 

2.临床意义

(1)绝大多数左束支传导阻滞见于器质性心脏病,如冠心病、心肌梗死、扩张型心肌病、高血压性心脏病等。

(2)常掩盖心肌梗死、心肌缺血、左心室肥大的心电图特征,易漏诊之。

(3)45岁以上发生左束支传导阻滞,其猝死的发生率为无束支传导阻滞者的10倍。

(4)若同时伴有一度、二度房室传导阻滞,则预后多严重,应及时安装人工起搏器。

\protect\hypertarget{text00009.htmlux5cux23subid56}{}{}

\subsection{右束支传导阻滞}

1.心电图特征

(1)V\textsubscript{1} 导联QRS波群多呈rS(s)R′型,ST段压低,T波倒置。

(2)其他导联QRS终末波宽钝、错折。

(3)QRS波群时间≥0.10s,电轴正常。若QRS波群时间≥0.12s,则为完全性右束支传导阻滞;若QRS波群时间0.10~0.11s,则为不完全性右束支传导阻滞。

(4)部分患者V\textsubscript{1}导联QRS波群可出现以下变异:①呈R型或M型;②呈qR型,多见于合并前间壁、广泛前壁心肌梗死及重度右心室肥大时;③呈rS型,而V\textsubscript{2}呈rS(s)R′型,见于右位心合并右束支传导阻滞;④呈rS或RS型,S波错折,加做V\textsubscript{3}R导联呈rS(s)R′或rS(s)r′型,见于逆钟向转位、隐匿性不完全性右束支传导阻滞。

2.V\textsubscript{1} 导联QRS波群呈rS(s)R′(r′)型的常见原因

V\textsubscript{1}导联QRS波群呈rS(s)R′(r′)型,时间≤0.11s,见于下列情况:

(1)正常变异:r′波系右心室流出道,尤其是室上嵴部位除极所致,r′<r波,当低一肋记录时,r′波可消失。Troudfit提出V\textsubscript{1}导联r波幅<0.8mV,r′波幅<0.6mV,r′/S<1,属正常变异。

(2)右心室肥大:常见于房间隔缺损、室间隔缺损等右心室舒张期负荷过重患者,与中度右心室肥大有关。R′(r′)波及其他导联终末波无明显宽钝。

(3)不完全性右束支传导阻滞:R′(r′)波及其他导联终末波宽钝、错折。

(4)右心室肥大合并不完全性右束支传导阻滞:R′波及其他导联终末波宽钝、错折,R′>1.0mV,伴电轴右偏,V\textsubscript{5}、V\textsubscript{6} 导联R/S<1或出现肺型P波。

(5)急性右心室扩张:急性肺栓塞时导致右心室扩张,心电图一过性出现V\textsubscript{1}导联呈rS(s)R′(r′)型,肢体导联呈S\textsubscript{I} Q\textsubscript{Ⅲ}、肺型P波和T波倒置。

(6)后壁心肌梗死:多数V\textsubscript{1}导联QRS波群呈R型伴T波高耸,偶尔亦呈rS(s)r′型伴T波高耸。

3.临床意义

(1)传统的观点认为右束支传导阻滞多无重要临床意义,现发现多数右束支传导阻滞是有病因可寻的,往往是心脏疾病的早期表现。

(2)引起右束支传导阻滞最常见的病因有冠心病、心肌炎、心肌病、先天性心脏病等。

\protect\hypertarget{text00009.htmlux5cux23subid57}{}{}

\subsection{不定型心室内传导阻滞}

1.心电图特征

(1)QRS波形不符合左、右束支传导阻滞图形的特征。

(2)QRS波群时间≥0.12s。若QRS波群时间≥0.16s,则称为特宽型QRS波群(图\ref{fig3-13})。

\begin{figure}[!htbp]
 \centering
 \includegraphics[width=2.71875in,height=4.53125in]{./images/Image00063.jpg}
 \captionsetup{justification=centering}
 \caption{扩张型心肌病、全心扩大患者出现左前分支传导阻滞、不定型心室内传导阻滞、前壁r波振幅逆递增}
 \label{fig3-13}
  \end{figure} 

2.临床意义

多见于冠心病、扩张型心肌病、高钾血症、心力衰竭、药物中毒等。阻滞部位在浦肯野纤维、心室肌内。QRS波群愈宽,则提示心肌病变愈广泛和严重,预后愈差。

\protect\hypertarget{text00009.htmlux5cux23subid58}{}{}

\subsection{预激综合征}

1.典型预激综合征(W-P-W综合征)

P-R间期缩短,有“δ”波,QRS波群时间增宽,P-J间期≤0.27s及继发性ST-T改变。

2.变异型预激综合征(传统的Mahaim纤维预激综合征)

传统的Mahaim纤维又称为结-室旁道、束-室旁道。多位于右心室,QRS波形呈左束支阻滞图形。心电图特征:①P-R间期正常或延长(合并一度房室传导阻滞时);②有“δ”波;③QRS波群呈左束支阻滞图形,时间增宽,但<0.15s;④Ⅰ导联QRS波群呈R型,Ⅲ导联呈rS型,电轴左偏(0~-75°);⑤胸前导联QRS主波由向下转为向上的过渡区在V\textsubscript{4}导联之后;⑥有继发性ST-T改变。

\protect\hypertarget{text00009.htmlux5cux23subid59}{}{}

\subsection{心室内差异性传导}

(一)时相性心室内差异性传导

它的发生与冲动提早出现有关,即通常所说的心室内差异性传导。

1.心电图特征

(1)提早出现室上性冲动,其下传QRS波群宽大畸形,时间<0.14s。其心电图特征:①室上性冲动指窦性早搏、窦房交接性早搏、房性早搏、房室交接性早搏、房室分离时窦性夺获、各类反复搏动、心房扑动、心房颤动、房性心动过速、房室交接性心动过速等;②宽大畸形QRS波群可呈右束支传导阻滞型(75\%~85\%)、左束支传导阻滞型(20\%~40\%)、右束支传导阻滞型加左前分支传导阻滞型(约18\%)、右束支传导阻滞型加左后分支传导阻滞型(约10\%)、单纯左前分支传导阻滞型(约33\%)、单纯左后分支传导阻滞型(约19\%)、左中隔支传导阻滞型及不定型心室内传导阻滞型。后两者少见。

(2)QRS波形易变性大,可呈完全性或不完全性束支、分支传导阻滞型。

(3)长-短周期后易出现心室内差异性传导,称为Ashman现象。

(4)偶见室性早搏伴心室内差异性传导,多发生在收缩中、晚期的同源性室性早搏,少数可发生在舒张早期。因遇及浦肯野纤维或心室肌的相对不应期而出现心室内差异性传导,其QRS波形较舒张期出现的室性早搏宽大畸形(图\ref{fig11-22})。

2.发生机制

(1)提早出现的室上性冲动下传心室时,束支、分支等传导组织生理性不应期尚未完全恢复正常,导致冲动在左、右束支内的传导时间互差>0.025~0.04s,左前分支与左后分支的传导时间互差>0.02s,便会出现功能性束支或(和)分支传导阻滞图形。

(2)束支、分支的不应期有病理性延长:出现在T波后面的舒张早期室上性冲动遇及束支、分支病理性延长的不应期,便会出现束支、分支阻滞图形,常称为3相性或快频率依赖性束支、分支阻滞。

(3)束支、分支间隐匿性传导引起的蝉联现象:室上性冲动通过室间隔隐匿激动对侧束支、分支,使其不应期后延,提早出现室上性冲动下传心室时,遇及该束支、分支不应期而出现功能性传导阻滞,引起宽大畸形QRS-T波群。心电图表现为:①房性早搏二联律时,出现交替性左、右束支传导阻滞图形(图\ref{fig3-14});②房性心动过速、心房扑动、心房颤动时,出现连续3次以上的心室内差异性传导(图\ref{fig3-15})。

\begin{figure}[!htbp]
 \centering
 \includegraphics[width=5.77083in,height=1.84375in]{./images/Image00064.jpg}
 \captionsetup{justification=centering}
 \caption{房性早搏二联律伴交替性左、右束支传导阻滞型的心室内差异性传导}
 \label{fig3-14}
  \end{figure} 

\begin{figure}[!htbp]
 \centering
 \includegraphics[width=5.58333in,height=1.46875in]{./images/Image00065.jpg}
 \captionsetup{justification=centering}
 \caption{短阵性房性心动过速伴连续的心室内差异性传导、房室呈1:1~3:2文氏现象}
 \label{fig3-15}
  \end{figure} 

(二)非时相性心室内差异性传导

它的发生与冲动出现的时相无明显关系,主要与异位起搏点的位置及下传途径有关。仅见于房室交接性逸搏、早搏。

1.心电图特征

(1)延迟或提早出现QRS波形与窦性略异,时间正常或略增宽,但<0.11s,为房室交接性逸搏或早搏伴非时相性心室内差异性传导(图\ref{fig3-16})。

\begin{figure}[!htbp]
 \centering
 \includegraphics[width=5.58333in,height=0.66667in]{./images/Image00066.jpg}
 \captionsetup{justification=centering}
 \caption{间位型房室交接性早搏伴非时相性心室内差异性传导、T波改变}
 \label{fig3-16}
  \end{figure} 

(2)可与窦性、室性激动形成交-窦、交-室的室性融合波。

2.发生机制

(1)异位起搏点位于房室束(希氏束)分叉部的近端。

(2)异位起搏点来源于心室分支部位。

(3)异位起搏点位于房室交接区的边缘区或下部,激动沿着房室交接区、希氏束内解剖上或功能上纵向分离的径路下传。激动首先通过希氏束的一部分传导纤维到达心室肌的特定部位使其提早除极,尔后再通过浦肯野纤维的快速传导径路到达心室的其他部分,导致逸搏QRS波形与窦性搏动不一致,但时间仍在正常范围。

(4)异位起搏点的激动通过异常传导径路下传心室,如Mahaim纤维传导。

\protect\hypertarget{text00009.htmlux5cux23subid60}{}{}

\subsection{室性异位搏动}

1.室性早搏

(1)提早出现宽大畸形QRS-T波群,时间≥0.12s;T波宽大,其方向与QRS主波方向相反。

(2)其前无相关的P波,其后偶有P\textsuperscript{-}波,R-P\textsuperscript{-} 间期<0.20s。

(3)大多数代偿间歇完全,若伴有P\textsuperscript{-}波,可出现不完全性代偿间歇。

2.室性逸搏

(1)延迟出现1~2次宽大畸形QRS-T波群,时间≥0.12s;T波与QRS主波方向相反。若其形态一致,则为单源性室性逸搏;若呈两种形态,则为双源性室性逸搏;若形态有3种或3种以上,则为多源性室性逸搏。

(2)其QRS波群前、中、后可有窦性P波,但P-R间期<0.12s,表明P波被干扰而不能下传;或QRS波群后面跟随逆行P\textsuperscript{-}波,其R′-P\textsuperscript{-} 间期<0.20s。

(3)若逸搏周期1.5~3.0s,频率20~40次/min,称为室性逸搏;若逸搏周期>3.0s,频率<20次/min,称为过缓的室性逸搏;若逸搏周期0.60~1.50s,频率41~100次/min,则称为加速的室性逸搏。其逸搏周期可稍有不规则。

\protect\hypertarget{text00009.htmlux5cux23subid61}{}{}

\subsection{室性融合波}

1.基本概念

两个起搏点的冲动从不同方向同时或几乎同时各自激动一部分心室肌所产生的融合搏动,称为室性融合波。室性融合波是心室内绝对干扰所致。不完全性预激系同源性室性融合波。

2.心电图特征

(1)室性融合波的形态介于其他两种QRS波群之间,视融合程度不同,其形态多变。

(2)室性融合波出现的时间必须是两个起搏点冲动应同时或几乎同时出现的时间。

3.室性融合波的类型

(1)窦-室室性融合波:最常见。其心电图表现为:①室性融合波的QRS波群之前必有窦性P波;②其P-R间期较窦性P-R间期短0~0.06s;③其QRS波形、时间介于窦性QRS波群与室性QRS波群之间,且易变性较大。常见于窦性心律合并室性并行心律、加速的室性逸搏心律、舒张晚期室性早搏或心室人工起搏心律(图\ref{fig3-17})。

\begin{figure}[!htbp]
 \centering
 \includegraphics[width=5.78125in,height=1.15625in]{./images/Image00067.jpg}
 \captionsetup{justification=centering}
 \caption{房室交接性早搏(R\textsubscript{3} 、R\textsubscript{6})及逸搏(R\textsubscript{7} )、加速的室性逸搏(R\textsubscript{1}、R\textsubscript{8} )、室性融合波(R\textsubscript{4})及不完全性右束支阻滞(定准电压0.5mV)}
 \label{fig3-17}
  \end{figure} 

(2)房-室室性融合波:房性异位心律,如房性逸搏心律、房性心动过速、心房扑动、心房颤动与室性异位搏动、心室人工起搏心律在心室内产生融合。

(3)交-室室性融合波:房室交接性逸搏心律、房室交接性心动过速与室性异位搏动、心室人工起搏心律在心室内产生融合(图\ref{fig3-18})。

\begin{figure}[!htbp]
 \centering
 \includegraphics[width=5.58333in,height=1.36458in]{./images/Image00068.jpg}
 \captionsetup{justification=centering}
 \caption{V\textsubscript{1}导联连续记录,定准电压0.5mV。显示窦性心动过缓伴不齐、完全性左束支阻滞、房室交接性逸搏(R\textsubscript{3}、R\textsubscript{4} )、室性逸搏(R\textsubscript{8})及由两者逸搏所形成形态正常化的室性融合波(R\textsubscript{5}~R\textsubscript{7} )}
 \label{fig3-18}
  \end{figure} 


(4)室-室室性融合波:心室内至少有两个异位起搏点同时发放冲动引起心室共同除极。其心电图表现为:①至少有两种固定形态的纯室性QRS波群;②有介于两者之间的室性融合波,其形态多变,且往往“正常化”;③融合波出现的时间,刚好是两个异位起搏点应发放冲动的时间。见于室性逸搏心律合并室性并行心律、双源性室性逸搏心律、多源性室性早搏及心室人工起搏心律合并室性早搏、室性逸搏等(图\ref{fig3-19})。

\begin{figure}[!htbp]
 \centering
 \includegraphics[width=5.78125in,height=1.51042in]{./images/Image00069.jpg}
 \captionsetup{justification=centering}
 \caption{V\textsubscript{1}导联连续记录,定准电压0.5mV。显示三度房室传导阻滞、多源性室性逸搏心律、室性早搏二联律及室性融合波正常化(下行R\textsubscript{2})}
 \label{fig3-19}
  \end{figure} 


(5)窦-交室性融合波:少见。窦性冲动与房室交接区异位冲动沿着不同径路下传共同激动心室。此时的房室交接区异位冲动必须伴有非时相性心室内差异性传导,才有可能产生和识别窦-交室性融合波。

(6)交-交室性融合波:罕见。起源于不同部位的房室交接区异位冲动沿着不同径路下传共同激动心室。此时有一源的房室交接区异位冲动必须伴有非时相性心室内差异性传导,才有可能产生和识别交-交室性融合波(图\ref{fig3-20})。

\begin{figure}[!htbp]
 \centering
 \includegraphics[width=5.58333in,height=0.90625in]{./images/Image00070.jpg}
 \captionsetup{justification=centering}
 \caption{V\textsubscript{1}导联连续记录,显示窦性心动过缓伴不齐、房性早搏(R\textsubscript{2})、双源性房室交接性逸搏心律(其中一源伴非时相性心室内差异性传导,如R\textsubscript{3}、R\textsubscript{4} )及由两者形成的室性融合波(如R\textsubscript{5}、R\textsubscript{6} )、不完全性干扰性房室分离}
 \label{fig3-20}
  \end{figure} 


(7)起源于房室旁道的异位冲动与窦性、房性、房室交接区或室性冲动同时激动心室,产生特殊类型的室性融合波。

(8)同源性室性融合波:见于W-P-W综合征的不完全性预激、Mahaim纤维预激。

4.临床意义

(1)表明心脏有两个起搏点在发放冲动或房室间有两条传导径路。

(2)在宽QRS心动过速中,如能见到室性融合波,则提示为室性心动过速。

\protect\hypertarget{text00009.htmlux5cux23subid62}{}{}

\subsection{QRS波群时间、形态呈交替性改变}

(1)交替性预激综合征:P-P间期基本规则时,P-R间期呈长、短交替出现,相应的QRS波群时间、形态呈交替性改变,两者的P-J间期相等。

(2)交替性束支阻滞:①P-P间期、P-R间期规则时,出现QRS波群时间、形态呈交替性改变;②室上性心动过速时,R-R间期规则,出现QRS波群时间、形态呈交替性改变(图\ref{fig3-21})。

\begin{figure}[!htbp]
 \centering
 \includegraphics[width=5.58333in,height=1.11458in]{./images/Image00071.jpg}
 \captionsetup{justification=centering}
 \caption{交替性左束支阻滞(定准电压均为0.5mV)}
 \label{fig3-21}
  \end{figure} 

(3)舒张晚期室性早搏或(和)室性融合波,呈二联律:畸形QRS波群的P-R间期较正常QRS波群的P-R间期短,两者的P-J间期不相等。

(4)室上性早搏二联律伴心室内差异性传导:房性早搏、房室交接性早搏二联律时,若伴有心室内差异性传导,则可出现QRS波群时间、形态呈交替改变(图\ref{fig3-14})。

(5)室性早搏二联律。

\protect\hypertarget{text00009.htmlux5cux23subid63}{}{}

\section{R-R间期长、短交替}

\protect\hypertarget{text00009.htmlux5cux23subid64}{}{}

\subsection{窦性节律时出现R-R间期长、短交替}

(1)房室交接区快、慢径路交替性传导:当快、慢径路呈交替性传导时,由快径路下传,其P-R间期较短导致R-R间期较短;经慢径路下传,其P-R间期较长导致R-R间期较长。

(2)文氏型3:2房室传导阻滞:长R-R间期小于短R-R间期的2倍。

(3)莫氏型3:2房室传导阻滞:长R-R间期为短R-R间期的2倍。

(4)交替性预激综合征。

(5)舒张晚期室性早搏或(和)室性融合波,呈二联律。

(6)阻滞型房性早搏三联律:每隔2个窦性搏动出现1次阻滞型房性早搏(图\ref{fig3-22})。

\begin{figure}[!htbp]
 \centering
 \includegraphics[width=5.58333in,height=0.66667in]{./images/Image00072.jpg}
 \captionsetup{justification=centering}
 \caption{窦性心动过缓、阻滞型房性早搏三联律、轻度T波改变}
 \label{fig3-22}
  \end{figure} 

(7)窦性早搏二联律。

(8)文氏型3:2窦房传导阻滞。

(9)莫氏型3:2窦房传导阻滞(图\ref{fig3-23})。

\begin{figure}[!htbp]
 \centering
 \includegraphics[width=5.77083in,height=1.20833in]{./images/Image00073.jpg}
 \captionsetup{justification=centering}
 \caption{二度Ⅱ型3:2窦房传导阻滞、轻度ST段改变}
 \label{fig3-23}
  \end{figure} 

(10)窦房交接区快、慢径路呈交替性传导。

(11)交替性窦性停搏,即每发放2次窦性冲动出现1次窦性停搏。

第7~11条的心电图表现,请见第十一章第二节窦性早搏。

\protect\hypertarget{text00009.htmlux5cux23subid65}{}{}

\subsection{房室交接区节律时出现R-R间期长、短交替}

(1)顺向型房室折返性心动过速。

(2)房室交接性逸搏心律或加速的逸搏心律伴3:2外出阻滞。

(3)房室交接性逸搏或加速的逸搏伴房室交接性早搏二联律。

(4)房室交接性逸搏或加速的逸搏伴房室交接性反复搏动,呈逸搏-反复二联律(图\ref{fig3-24})。

\begin{figure}[!htbp]
 \centering
 \includegraphics[width=5.77083in,height=0.84375in]{./images/Image00074.jpg}
 \captionsetup{justification=centering}
 \caption{病窦综合征患者出现窦性停搏、缓慢的房室交接性逸搏伴反复搏动二联律}
 \label{fig3-24}
  \end{figure} 

(5)房室交接性逸搏或加速的逸搏伴窦性夺获二联律,呈逸搏-夺获二联律。

(6)阵发性房室交接性心动过速伴3:2外出阻滞(图\ref{fig3-25})。

\begin{figure}[!htbp]
 \centering
 \includegraphics[width=5.79167in,height=1.45833in]{./images/Image00075.jpg}
 \captionsetup{justification=centering}
 \caption{窦性停搏、阵发性房室交接性心动过速伴3:2外出阻滞(上行梯形图)或加速的房室交接性逸搏伴房室交接性早搏或反复搏动二联律(下行梯形图)}
 \label{fig3-25}
  \end{figure} 

\protect\hypertarget{text00010.html}{}{}

\protect\hypertarget{text00010.htmlux5cux23chapter10}{}{}

\chapter{J点、J波、Epsilon波、Brugada波及Lambda波(λ波)}

\protect\hypertarget{text00010.htmlux5cux23subid66}{}{}

\section{J点与J波}

\protect\hypertarget{text00010.htmlux5cux23subid67}{}{}

\subsection{J点}

QRS波群终点与ST段起点的结合点称为J点。J点一般多在等电位线上,上下偏移<0.1mV,可随ST段偏移而移位。早复极综合征时,以R波为主导联J点抬高0.1~0.4mV,与迷走神经张力过高有关。

\protect\hypertarget{text00010.htmlux5cux23subid68}{}{}

\subsection{J波}

1.基本概念

当心电图J点从基线明显偏移后,形成一定的幅度(≥0.2mV)和持续一定的时间(≥20ms),并呈圆顶状或驼峰状特殊形态,称为J波(或Osborn波)。属心室提前发生的复极波,是由于心室肌除极和复极过程同时减慢,但以除极速度减慢明显,使更多心肌除极尚未结束就已复极,导致心室除极和复极的重叠区增宽,从而形成了J波。

2.J波基本特征

(1)J波常起始于QRS波群的R波降肢部分,其前面尖峰状R波与其特有的圆顶状或驼峰状波形构成了尖峰-圆顶状特殊波形。

(2)J波形态可呈多样化,以下壁和左胸导联最为明显。若J波在V\textsubscript{1}导联呈明显直立时,类似右束支阻滞的R′(r′)波,则易误诊为完全性右束支传导阻滞(图\ref{fig4-1});若J波在V\textsubscript{1}导联倒置,V\textsubscript{5} 、V\textsubscript{6}导联直立时,则易误诊为完全性左束支传导阻滞(图\ref{fig4-2})。

\begin{figure}[!htbp]
 \centering
 \includegraphics[width=5.48958in,height=1.72917in]{./images/Image00076.jpg}
 \captionsetup{justification=centering}
 \caption{脑外伤患者出现继发性J波、左心室高电压、ST-T改变及Q-T间期延长(V\textsubscript{5}导联定准电压0.5mV)}
 \label{fig4-1}
  \end{figure} 


\begin{figure}[!htbp]
 \centering
 \includegraphics[width=5.78125in,height=1.33333in]{./images/Image00077.jpg}
 \captionsetup{justification=centering}
 \caption{脑溢血患者出现继发性J波、ST-T改变及Q-T间期延长}
 \label{fig4-2}
  \end{figure} 

(3)J波形态和振幅呈频率依赖性改变,即心率减慢时J波明显,心率增快时J波可消失。

(4)J波尚受体温、pH值及电解质等因素的影响,如体温越低、pH值越低、血钙越高,则J波越明显;反之,则J波变低或消失。

(5)J波与恶性室性心律失常有密切关系。

3.J波类型及其临床意义

(1)特发性J波:无引起异常J波的其他原因存在,常伴有反复发作的原因不明的室性心动过速、心室颤动甚至猝死,平时多有迷走神经张力增高的表现,具有慢频率依赖性心室内传导阻滞等特征。一小部分早复极综合征患者,若出现明显J波,可能属于特发性J波的范畴,预示有发生恶性室性心律失常的倾向。

(2)继发性J波:出现异常J波有据可查,如全身性低温(≤34℃)、高钙血症、颅脑疾患、心肺复苏过程中、脑死亡等均可引起巨大的异常J波,多伴有Q-T间期延长及心动过缓,易诱发恶性室性心律失常。

\protect\hypertarget{text00010.htmlux5cux23subid69}{}{}

\subsection{缺血性J波}

1.基本概念

严重的急性心肌缺血(如急性心肌梗死、冠状动脉痉挛等)出现明显的J波或原有的J波振幅增高、时间延长,其出现的导联与心肌缺血的部位密切相关,称为缺血性J波,是心肌严重缺血时伴发的一种超急性期的心电图改变。

2.发生机制

心肌急性缺血引起心室外膜心肌细胞的I\textsubscript{to}
电流增加,并与心内膜心肌细胞出现1相和2相的复极电位差而形成缺血性J波。

3.临床意义

(1)见于严重的急性心肌缺血,如急性心肌梗死、变异型心绞痛及PCI术中等,有时是急性心肌梗死早期的唯一的心电图改变。

(2)缺血性J波提示心肌存在明显而严重的复极离散度,预示心电极不稳定,易发生恶性室性心律失常。

\protect\hypertarget{text00010.htmlux5cux23subid70}{}{}

\section{Epsilon波}

Epsilon波的心电图表现为类似右束支阻滞图形,右胸导联V\textsubscript{1}~V\textsubscript{3} 导联特别是V\textsubscript{2}导联QRS波群终末部、ST段起始部有小棘波,即为Epsilon波(图\ref{fig4-3})。Epsilon波是致心律失常性右室心肌病较为特异的指标之一。系右室壁部分心肌被脂肪组织包裹导致其最后除极所致,在Ⅰ、V\textsubscript{1}、V\textsubscript{2}导联最清楚。致心律失常性右室心肌病患者易出现呈左束支阻滞型的室性早搏或室性心动过速,心室晚电位阳性率高,具有家族性,是青年人猝死的原因之一。若伴有左心室受累及功能异常者,则更增加了其猝死的风险。

\begin{figure}[!htbp]
 \centering
 \includegraphics[width=3.1875in,height=2.30208in]{./images/Image00078.jpg}
 \captionsetup{justification=centering}
 \caption{致心律失常性右室心肌病患者,V\textsubscript{1}、V\textsubscript{2} 导联出现Epsilon波、前间壁T波倒置}
 \label{fig4-3}
  \end{figure} 


\protect\hypertarget{text00010.htmlux5cux23subid71}{}{}

\section{Brugada波}

Brugada波是指V\textsubscript{1} ~V\textsubscript{3}导联出现J波、ST段抬高、T波倒置酷似右束支阻滞图形,又称为右胸导联三联征。

1.心电图类型

(1)Ⅰ型:以突出的“穹隆型”ST段抬高为特征,表现为J波或抬高的ST段顶点>0.2mV,其ST段随即向下倾斜伴T波倒置(图\ref{fig4-4})。

\begin{figure}[!htbp]
 \centering
 \includegraphics[width=5.48958in,height=1.70833in]{./images/Image00079.jpg}
 \captionsetup{justification=centering}
 \caption{男性,32岁,健康体检发现“穹隆型”ST段抬高(Ⅰ型Brugada波)}
 \label{fig4-4}
  \end{figure} 

(2)Ⅱ型:呈“马鞍型”ST段抬高,表现为J波抬高(≥0.2mV),ST段呈下斜型抬高(在基线上方,仍然≥0.1mV),紧随正向或双向T波(图\ref{fig4-5})。

\begin{figure}[!htbp]
 \centering
 \includegraphics[width=3.8125in,height=3.3125in]{./images/Image00080.jpg}
 \captionsetup{justification=centering}
 \caption{男性,27岁,健康体检发现“马鞍型”ST段抬高(Ⅱ型Brugada波)}
 \label{fig4-5}
  \end{figure} 

(3)Ⅲ型:呈“马鞍型”或“穹隆型”,或两者兼有,ST段抬高(<0.1mV)。

2.心电图特征

(1)上述3种图形呈动态改变,具有多变性,可在同一患者观察到。

(2)Brugada波几乎仅见于男性。

(3)Brugada波可呈间歇性出现,能被药物(静脉注射缓脉灵)所激发,使其显露或更加明显、典型。

(4)交感神经张力增高、运动、心率增快可使Brugada波中抬高的ST段降低,甚至Brugada波消失;迷走神经张力增高、休息、心率减慢、抗心律失常药物(Ⅰa、Ⅰc、Ⅲ类)可使Brugada波、ST段抬高更明显。

(5)Brugada波易伴发恶性室性心律失常,如快速性、多形性室性心动过速或心室颤动,且易反复发作而猝死。

(6)移高V\textsubscript{1} ~V\textsubscript{3}导联心电图记录位置,可提高Brugada波的检出率。

3.发生机制

属原发性心电离子通道缺陷疾病,与SCN5A基因突变有关,可造成Na\textsuperscript{+}通道功能改变或功能丧失,导致心外膜心肌动作电位出现圆顶状波形,产生Brugada波;同时使右室心外膜与心内膜复极离散度明显增大,易产生2相折返引起室性早搏、室性心动过速或心室颤动。

\protect\hypertarget{text00010.htmlux5cux23subid72}{}{}

\section{Lambda波(λ波)}

Lambda波(λ波)是一个心室除极与复极均有异常,且与心源性猝死相关的一种心电图波。

1.心电图特征

(1)仅Ⅱ、Ⅲ、aVF导联QRS波群上升肢的终末部和降肢均出现切迹,且ST段呈下斜型抬高伴T波倒置。

(2)左胸导联呈镜像改变,表现为ST段压低。

(3)可合并恶性室性心律失常,如室性心动过速、心室颤动、心脏骤停等(图\ref{fig4-6})。

\begin{figure}[!htbp]
 \centering
 \includegraphics[width=3.40625in,height=4.63542in]{./images/Image00081.jpg}
 \captionsetup{justification=centering}
 \caption{Ⅱ、Ⅲ、aVF、V\textsubscript{6}导联出现Lambda波(λ波)及V\textsubscript{4} 、V\textsubscript{5}导联ST段压低(引自郭继鸿)}
 \label{fig4-6}
  \end{figure} 


2.临床特征

(1)常见于年轻的男性患者。

(2)有晕厥史。

(3)有晕厥或猝死的家族史。

(4)无器质性心脏病依据。

(5)有恶性室性心律失常的发生及心电图记录。

(6)常在夜间发生猝死。

3.发生机制

尚不清楚,属原发性心电离子通道缺陷疾病,可能与SCN5A基因突变有关。其猝死系原发性心脏停搏所致,即在短时间内突发心脏各级心电活动全部消失而成一条直线。

\protect\hypertarget{text00011.html}{}{}

\protect\hypertarget{text00011.htmlux5cux23chapter11}{}{}

\chapter{正常ST段及其异常改变}

\protect\hypertarget{text00011.htmlux5cux23subid73}{}{}

\section{ST段测量方法及其正常值}

\protect\hypertarget{text00011.htmlux5cux23subid74}{}{}

\subsection{ST段测量方法}

ST段代表心室除极结束后至复极开始这一短暂时间,多呈等电位线,可随J点的移位而移位。故测量ST段应从J点后0.04~0.08s处作一水平线,再根据TP段的延长线作为基线或采用两个相邻心搏的QRS波群起点的连线作为基线,借以确定有无ST段移位。ST段抬高时应自基线上缘测量至ST段上缘,压低时应从基线下缘测量至ST段下缘。欧盟心电图标准化工作小组推荐:QRS波群、J点、ST段、T波的振幅测量统一采用QRS波群起始部作为参考基线。“2009年国际指南”则推荐:以TP段和PR段作为基线。

\protect\hypertarget{text00011.htmlux5cux23subid75}{}{}

\subsection{ST段正常值}

(1)ST段压低:以R波为主导联ST段压低应≤0.05mV,但Ⅲ、aVL导联可压低0.1mV。

(2)ST段抬高:以R波为主导联ST段抬高应≤0.1mV,但V\textsubscript{1}~V\textsubscript{4}导联可抬高0.2~0.4mV,尤其是青壮年、运动员等身体素质较好者多见。

(3)ST段时间:为0.05~0.15s。

\protect\hypertarget{text00011.htmlux5cux23subid76}{}{}

\subsection{如何评价ST段偏移的临床意义}

1.根据ST段偏移的形态

(1)ST段呈上斜型(斜直型)、凹面向上型抬高:见于正常人、迷走神经张力过高者、急性心包炎、变异型心绞痛及超急性期心肌梗死等,需结合临床加以判断。

(2)ST段呈弓背向上型、单向曲线型、水平型、墓碑型抬高:多见于急性期心肌梗死、变异型心绞痛、电击伤及重症心肌炎等。

(3)ST段呈“穹隆型”或“马鞍型”抬高:多见于Brugada综合征患者。

(4)ST段呈上斜型压低:多无临床价值。

(5)ST段呈近水平型压低:需结合ST段压低的程度,若压低>0.1mV者,可能是异常表现。

(6)ST段呈水平型、下垂型(下斜型)压低:多见于心肌缺血、劳损及心肌炎等。

2.根据ST段偏移的程度及有无伴发QRS-T波群异常

(1)若QRS-T波群正常,ST段偏移在上述标准内,则为正常。

(2)若QRS波幅低电压,ST段偏移在上述标准内,则应视为异常表现。

(3)若以R波为主导联T波低平或倒置,ST段偏移在上述标准内,也应视为异常表现。

3.确认ST段偏移是原发性改变、继发性改变还是电张调整性改变

(1)原发性ST段改变:指心室除极正常而出现复极异常者,表现为QRS波形、时间正常而出现ST段改变。多有临床价值,见于心肌缺血、劳损、低钾血症及β受体功能亢进等。

(2)继发性ST段改变:指心室除极异常而出现复极异常者,表现为QRS波群宽大畸形而出现ST段改变。多无临床价值,见于室性异位搏动、束支传导阻滞、预激综合征等。

(3)电张调整性ST段改变:指心室异常除极消除后恢复正常除极一段时间内仍存在明显的ST段改变者。多无临床价值,属功能性改变,见于室性心动过速后、间歇性束支传导阻滞、间歇性预激综合征、右心室起搏患者等。

4.需结合临床病史、心肌酶谱、心脏超声心动图等相关资料

\protect\hypertarget{text00011.htmlux5cux23subid77}{}{}

\section{ST段异常改变及其临床意义}

ST段异常改变包括抬高、压低、延长或缩短。心外膜下心肌损伤表现为ST段抬高,心内膜下心肌损伤则表现为ST段压低、水平型延长。

\protect\hypertarget{text00011.htmlux5cux23subid78}{}{}

\subsection{ST段抬高}

ST段抬高可表现为短暂性、较久性或持续性,其形态有上斜型(斜直型)、凹面向上型、弓背向上型、单向曲线型、水平型、墓碑型、“穹隆型”或“马鞍型”、“巨R型”等抬高。分析时应注意动态观察ST段的形态、幅度、持续时间及与症状的关系,并结合T波改变情况综合分析。

1.ST段呈上斜型(斜直型)抬高

正常凹面向上的ST段变直、烫平,与T波正常连接角消失,导致两者不易区分且间接地使T波变宽,继之,ST段直线向上升高并倾斜地与高耸宽大的T波相连,ST段形状呈不对称性。见于超急性期心肌梗死、变异型心绞痛及迷走神经张力过高者等(图\ref{fig5-1})。

\begin{figure}[!htbp]
 \centering
 \includegraphics[width=2.82292in,height=3.72917in]{./images/Image00082.jpg}
 \captionsetup{justification=centering}
 \caption{下壁、侧壁超急期心肌梗死患者出现ST段呈上斜型抬高伴T波高耸及高侧壁、前间壁ST段呈缺血型压低、房室传导延缓(P-R间期0.23s)}
 \label{fig5-1}
  \end{figure} 

2.ST段呈凹面向上型抬高

ST段呈凹面向上型抬高者多伴有T波直立,见于急性心肌梗死早期、急性心包炎、早复极综合征、电击复律后、颅内出血、高钾血症及左心室舒张期负荷过重等(图\ref{fig5-2})。

\begin{figure}[!htbp]
 \centering
 \includegraphics[width=5.78125in,height=2.88542in]{./images/Image00083.jpg}
 \captionsetup{justification=centering}
 \caption{患者男性,19岁,发热、胸痛2天,拟诊急性心包炎。图A系初诊时记录,表现为下壁、前侧壁ST段呈凹面向上型抬高伴T波高耸;图B系入院2天后记录,表现为ST段呈上斜型抬高伴T波正负双向}
 \label{fig5-2}
  \end{figure} 

3.ST段呈弓背向上型、单向曲线型抬高

抬高的ST段其凸面向上形似弓背状,并与缺血性T波平滑地连接,两者无明确界限,构成一条凸起在基线以上的弓状曲线,称为单向曲线。若此时T波直立高耸,ST段凸面光滑而对称,则形成抛物线样改变。见于急性心肌梗死早期、变异型心绞痛、心室壁运动异常或室壁瘤形成等(图\ref{fig5-3})。

\begin{figure}[!htbp]
 \centering
 \includegraphics[width=5.07292in,height=1.55208in]{./images/Image00084.jpg}
 \captionsetup{justification=centering}
 \caption{前间壁、前壁陈旧性心肌梗死3年余,患者仍出现弓背向上型、单向曲线型ST段抬高,心脏超声心动图显示心尖部室壁瘤形成}
 \label{fig5-3}
  \end{figure} 

4.ST段呈水平型抬高

此型少见,见于急性心肌梗死早期、变异型心绞痛等(图\ref{fig5-4})。

\begin{figure}[!htbp]
 \centering
 \includegraphics[width=5.58333in,height=1.51042in]{./images/Image00085.jpg}
 \captionsetup{justification=centering}
 \caption{变异型心绞痛患者出现一度房室传导阻滞、完全性右束支传导阻滞、下壁及前侧壁ST段呈水平型抬高}
 \label{fig5-4}
  \end{figure} 

5.ST段呈墓碑型抬高

其ST段向上凸起并快速上升高达0.8~1.6mV,凸起的ST段顶峰高于其前的r波,r波矮小且持续时间短暂,通常<0.04s,抬高的ST段与其后T波上升肢相融合,难以单独辨认T波,且T波常直立高耸。见于急性心肌梗死超急期、早期,以老年人多发,均发生于穿壁性心肌梗死。易并发急性左心衰竭、严重室性心律失常、完全性房室传导阻滞等,死亡率显著增高。可作为判断急性心肌梗死预后的一项独立指标(图\ref{fig5-5})。

\begin{figure}[!htbp]
 \centering
 \includegraphics[width=4.79167in,height=1.95833in]{./images/Image00086.jpg}
 \captionsetup{justification=centering}
 \caption{广泛前壁急性心肌梗死患者出现墓碑型ST段抬高(V\textsubscript{2}、V\textsubscript{3} 、V\textsubscript{4} 导联)}
 \label{fig5-5}
  \end{figure} 


6.ST段呈“穹隆型”或“马鞍型”抬高

以V\textsubscript{1} ~V\textsubscript{3}导联ST段呈“穹隆型”或“马鞍型”抬高(≥0.1mV),酷似右束支传导阻滞图形,心脏结构无明显异常,易反复发作多形性室性心动过速及心室颤动而导致晕厥或猝死为特征,该室性心动过速发作常以极短联律间期的室性早搏起始,QRS波形多变,频率很快,常>260次/min,有家族性遗传特征。多见于Brugada综合征患者(图\ref{fig4-4}、图\ref{fig4-5})。

7.ST段呈“巨R型”抬高

(1)心电图特征:①QRS波群与ST-T融合在一起,J点消失,R波下降肢与ST-T融合,浑然成一斜线下降,致使QRS波群、ST段与T波形成单个三角形,呈峰尖、边直、底宽的宽波,难以辨认各波段的交界,酷似“巨R型”波形;②“巨R型”ST段常出现在ST段抬高最明显的导联;③ST段抬高程度与S波减少成正比,凡ST段抬高最明显的导联,其S波减少也最明显甚至消失,但QRS波群起始向量不变;④QRS波群时间可稍增宽,Q-T间期可轻度延长;⑤“巨R型”ST段常呈一过性改变,仅持续数分钟,心肌缺血一旦改善或恶化即可消失(图\ref{fig5-6})。

\begin{figure}[!htbp]
 \centering
 \includegraphics{./images/Image00087.jpg}
 \captionsetup{justification=centering}
 \caption{广泛前壁急性心肌梗死患者出现“巨R型”ST段抬高、肢体导联QRS波群低电压}
 \label{fig5-6}
  \end{figure} 

(2)临床意义:①超急性期心肌梗死,尤其是前壁心肌梗死,偶见于下壁心肌梗死;②急性而严重的心肌缺血,如不稳定型心绞痛、变异型心绞痛、经皮腔内冠状动脉成形术中等;③急性心肌损伤,如电击伤、心脏除颤等;④偶见于颅脑损伤患者。

\protect\hypertarget{text00011.htmlux5cux23subid79}{}{}

\subsection{ST段抬高的诊断与鉴别诊断}

1.早复极综合征

(1)以R波为主导联J点或J波抬高,以V\textsubscript{2}~V\textsubscript{5} 导联最为明显。

(2)J点或J波抬高导联其ST段呈上斜型、凹面向上型抬高(0.1~0.6mV),有时可达1.0mV。

(3)ST段抬高导联出现T波高耸。

(4)基本节律多为窦性心动过缓,频率多<50次/min。

(5)上述心电图改变持续多年不变,多无临床症状,见于运动员、身强力壮的体力劳动者等。

(6)运动或心率加快后,J点、ST段抬高程度减轻或恢复正常,这一点有助于与其他疾病导致ST段抬高者相鉴别。

2.变异型心绞痛

(1)胸痛发作时,ST段立即呈损伤型抬高(≥0.3mV),多在0.5mV左右,常伴T波高耸,随着症状的缓解,ST-T逐渐恢复正常(图\ref{fig5-7})。

\begin{figure}[!htbp]
 \centering
 \includegraphics[width=5.58333in,height=1.38542in]{./images/Image00088.jpg}
 \captionsetup{justification=centering}
 \caption{冠心病患者模拟Ⅱ导联在05:17(Ⅱa)、05:19(Ⅱb)、05:22(Ⅱc)、05:23(Ⅱd)胸痛发作和缓解时ST-T改变}
 \label{fig5-7}
  \end{figure} 

(2)相关导联出现T波高耸,两肢不对称,基底部增宽,伴Q-T间期延长。

(3)可出现一过性QRS波群时间增宽(0.10~0.12s)及QRS波群、ST段、T波电交替现象。

(4)常伴发一过性心律失常,如室性早搏、房室传导阻滞等。

(5)心肌酶谱正常范围。

3.超急性期心肌梗死(图\ref{fig5-1}、5-6)

(1)胸痛发作时,ST段立即呈损伤型急剧抬高(≥0.3~0.5mV),伴T波高耸。

(2)出现急性损伤阻滞图形,即QRS波群时间增宽至0.10~0.12s,心室壁激动时间延长。

(3)常伴发各种心律失常,如室性早搏、室性心动过速、房室传导阻滞及束支传导阻滞等。

(4)心肌酶谱增高。

(5)随着有效治疗,上述心电图改变可恢复正常;若恶化,则出现异常Q波、ST-T呈心肌梗死特有的动态演变。

4.急性心包炎(图\ref{fig5-2})

(1)窦性心动过速。

(2)广泛导联ST段呈凹面向上型抬高,但程度较轻,多在0.2mV左右;多伴有T波低平或浅倒置,其振幅<0.5mV。

(3)广泛导联PR段呈水平型压低(0.1~0.15mV),aVR导联PR段抬高。

(4)QRS波幅低电压,可出现P-QRS-T波群电交替现象。

(5)无异常Q波,心肌酶谱正常范围或轻度增高。

5.早复极综合征合并心绞痛

(1)合并典型心绞痛发作:①原有J点或J波及ST段抬高的程度反而减轻或恢复正常,呈伪善性改变;②原有T波高耸转为低平;③可出现U波倒置。

(2)合并变异型心绞痛发作:①除符合两者心电图改变外,其ST段抬高、T波高耸更为明显;②可出现U波倒置。

6.室壁瘤(图\ref{fig5-3})

(1)多发生在广泛前壁、前壁急性穿壁性心肌梗死后。

(2)相应导联有异常Q波、ST段呈弓背向上型、单向曲线型持续抬高≥0.1mV达1个月以上或≥0.2mV持续15天。

(3)心脏超声波、胸透可明确诊断。

\protect\hypertarget{text00011.htmlux5cux23subid80}{}{}

\subsection{ST段压低}

ST段压低可表现为短暂性、较久性或持续性,其形态有水平型、下垂型(下斜型)、近水平型压低、上斜型及鱼勾样压低。这些形态的区分主要依据从R波顶峰作一条垂直线与J点后0.04~0.08s处沿着ST段走向作一直线,两者所形成夹角的大小。若两者所形成的夹角为90°,则ST段呈水平型压低;若两者所形成的夹角>90°,则呈下垂型(下斜型);若两者所形成的夹角在81°~90°,则为近水平型压低;若两者所形成的夹角<81°,则为上斜型。ST段压低往往是非特异性的,需密切结合临床加以判断。

1.ST段呈水平型、下垂型(下斜型)压低

两者统称为缺血型改变。多见于心肌缺血、心肌劳损、心肌炎、低钾血症及β受体功能亢进等。若ST段呈缺血型显著压低(≥0.3mV),则应警惕急性心内膜下心肌梗死的可能(图\ref{fig5-8})。

\begin{figure}[!htbp]
 \centering
 \includegraphics[width=5.78125in,height=2.65625in]{./images/Image00089.jpg}
 \captionsetup{justification=centering}
 \caption{胸痛患者出现前间壁、前壁ST段呈缺血型显著压低伴心肌酶谱增高,符合急性心内膜下心肌梗死}
 \label{fig5-8}
  \end{figure} 

2.ST段呈近水平型压低

其价值较缺血型改变为低,需结合压低的程度及与临床症状的关系加以判断。

3.ST段呈上斜型压低及鱼钩样改变

(1)ST段呈上斜型压低:多无临床价值。

(2)ST段呈鱼钩样改变:多见于洋地黄药物影响。

\protect\hypertarget{text00011.htmlux5cux23subid81}{}{}

\subsection{ST段延长}

ST段正常时间为0.05~0.15s。当其时间≥0.16s时,便称为ST段延长。见于低钙血症、心内膜下心肌缺血、Q-T间期延长综合征、三度房室传导阻滞伴缓慢心室率、阿-斯综合征发作后等。

\protect\hypertarget{text00011.htmlux5cux23subid82}{}{}

\subsection{ST段缩短}

当ST段时间<0.05s时,便称为ST段缩短。ST段代表心室肌动作电位2相平台期,具有心率依赖性,受儿茶酚胺、细胞外钙离子浓度、心肌病变及药物等因素的影响。凡是使用儿茶酚胺类及洋地黄类药物、高钙血症、心肌急性缺血、缺氧、损伤时致细胞膜受损引起钙离子持续内流,均可使ST段缩短或消失。此外,早复极综合征、心电-机械分离、特发性短Q-T间期综合征等也可导致ST段缩短或消失(图\ref{fig5-9})。

\begin{figure}[!htbp]
 \centering
 \includegraphics[width=5.64583in,height=0.73958in]{./images/Image00090.jpg}
 \captionsetup{justification=centering}
 \caption{心肺复苏后出现窦性心动过缓、二度房室传导阻滞、ST段消失及继发性Q-T间期缩短(0.26s)}
 \label{fig5-9}
  \end{figure} 

\protect\hypertarget{text00011.htmlux5cux23subid83}{}{}

\section{ST段电交替现象}

1.心电图特征

(1)连续出现ST段或ST-T形态交替,可表现为ST段抬高与压低、抬高与正常或正常与压低等交替性改变(图\ref{fig5-10})。

\begin{figure}[!htbp]
 \centering
 \includegraphics[width=5.07292in,height=0.64583in]{./images/Image00091.jpg}
 \captionsetup{justification=centering}
 \caption{可疑冠心病患者,平板运动试验后出现ST段电交替现象}
 \label{fig5-10}
  \end{figure} 

(2)ST段电交替现象持续时间较短,一般仅持续数秒至数分钟,呈一过性改变。

(3)ST段电交替随着ST段抬高而加剧,抬高越高,其电交替越明显。有时早搏后可出现ST段电交替或表现得更为明显。

(4)ST段电交替常伴发各种室性心律失常,如室性早搏、室性心动过速等。

(5)可同时伴有QRS波群、T波、U波的电交替。

(6)多见于胸前导联,与左冠状动脉前降支严重病变有关。

(7)与心外因素无关,如呼吸、体位、心包积液、胸腔积液等。

2.发生机制

与心肌缺血导致心肌有效不应期明显延长且呈交替性改变有关,即当ST段较低时,其有效不应期较长;当ST段较高时,其有效不应期较短。有效不应期长短交替的程度与ST段电交替呈正相关关系,提示两者存在因果关系。

3.临床意义

(1)心率正常时出现ST段电交替,常是严重心肌缺血的佐证,与室性心律失常发生有密切关系,是出现室性心律失常的前兆。

(2)心率正常时出现ST段电交替,常提示冠状动脉痉挛性病变,尤以左冠状动脉前降支多见;对变异型心绞痛的诊断具有高度特异性,是变异型心绞痛电不稳定的表现。

(3)心动过速时(>150次/min)出现ST段电交替,多无临床价值,随着心率恢复正常而消失。

\protect\hypertarget{text00012.html}{}{}

\protect\hypertarget{text00012.htmlux5cux23chapter12}{}{}

\chapter{正常T波及其异常改变}

\protect\hypertarget{text00012.htmlux5cux23subid84}{}{}

\section{正常T波及T波改变的类型}

\protect\hypertarget{text00012.htmlux5cux23subid85}{}{}

\subsection{正常T波的特征}

(1)正常T波的形态:前肢上升缓慢,后肢下降较快,波顶呈圆钝状。

(2)方向与振幅:多与QRS主波方向一致,如以R波为主导联T波直立(Ⅲ、aVL导联可浅倒),且其振幅≥$\frac{1}{10}$
R;V\textsubscript{1} ~V\textsubscript{4}导联T波振幅逐渐增高或倒置者应逐渐变浅,一般以V\textsubscript{4}、V\textsubscript{5} 导联T波振幅最高,可达1.2~1.5mV;V\textsubscript{1}导联T波振幅应<0.4mV,若其振幅≥0.4mV,应警惕后壁心肌梗死可能;年轻者出现T\textsubscript{V\textsubscript{1}}
、\textsubscript{V\textsubscript{2}}
>T\textsubscript{V\textsubscript{5}}
、\textsubscript{V\textsubscript{6}}
,多为正常变异;若>40岁者出现T\textsubscript{V\textsubscript{1}}
、\textsubscript{V\textsubscript{2}}
>T\textsubscript{V\textsubscript{5}}
、\textsubscript{V\textsubscript{6}}
,则见于左心室收缩期负荷过重、早期冠心病等,具有一定的临床价值。

(3)T波时间:一般<0.25s。

\protect\hypertarget{text00012.htmlux5cux23subid86}{}{}

\subsection{T波改变的类型}

T波代表心室肌复极过程中未被抵消的心室复极电位差。根据T波的极性、形态可分为倒置、直立、双相和低平;根据与心室除极的关系可分为原发性T波改变(心室除极正常而复极异常者)、继发性T波改变(心室除极异常而导致复极异常者)和电张调整性T波改变(心室异常除极消除后恢复正常除极一段时间内仍存在明显的T波改变者);根据病变性质可分为器质性T波改变(病理性)和良性T波改变(功能性)。

\protect\hypertarget{text00012.htmlux5cux23subid87}{}{}

\section{T波倒置}

本节所述的T波倒置均为原发性T波改变。一般T波倒置的深度多在0.25~0.6mV。若倒置T波的深度在0.5~1.0mV,则称为T波深倒置;若常规心电图中有3个以上导联倒置T波的深度>1.0mV,则称为巨倒T波,见于冠心病、肥厚型心肌病、脑血管意外及嗜铬细胞瘤等疾病。

\protect\hypertarget{text00012.htmlux5cux23subid88}{}{}

\subsection{冠状T波}

冠状T波又称为缺血性T波倒置、箭头状T波。其倒置的T波两肢对称、基底部狭窄、波谷尖锐,常伴Q-T间期延长,可伴有异常Q波出现(图\ref{fig6-1}),真正反映了透壁性心肌缺血。见于慢性或亚急性期心肌梗死、慢性冠状动脉供血不足、肥厚型心肌病等。若心电图无左心室肥大表现,则持续性冠状T波对冠心病尤其是冠心病合并心肌病变有独特的预测价值。

\begin{figure}[!htbp]
 \centering
 \includegraphics[width=5.78125in,height=2.07292in]{./images/Image00093.jpg}
 \captionsetup{justification=centering}
 \caption{前间壁陈旧性心肌梗死后出现前间壁及前壁冠状T波、Q-T间期延长}
 \label{fig6-1}
  \end{figure} 

\protect\hypertarget{text00012.htmlux5cux23subid89}{}{}

\subsection{Niagara(尼加拉)瀑布样T波}

脑血管意外、阿-斯综合征发作后及有交感神经兴奋性异常增高的急腹症等患者出现一种特殊形态的巨倒T波,酷似美国与加拿大交界的Niagara瀑布,故被命名为Niagara瀑布样T波,亦称为交感神经介导性巨倒T波(图\ref{fig6-2})。

\begin{figure}[!htbp]
 \centering
 \includegraphics[width=4.95833in,height=4.01042in]{./images/Image00094.jpg}
 \captionsetup{justification=centering}
 \caption{脑溢血患者出现三度房室传导阻滞(P-R间期长短不一)、房室交接性逸搏心律、完全性右束支阻滞、下壁及前侧壁ST段呈缺血型压低、Niagara瀑布样T波改变、Q-T间期延长}
 \label{fig6-2}
  \end{figure} 

1.心电图特征

(1)巨倒T波基底部宽阔、两肢明显不对称、前肢或后肢向外膨出或向内凹陷使T波不光滑,有切迹及顶部圆钝。

(2)巨倒T波的深度多≥1.0mV,偶可>2.0mV。常出现在V\textsubscript{3}~V\textsubscript{6}导联,也可出现在肢体导联,而在Ⅲ、aVR、V\textsubscript{1}等导联,则可出现宽而直立的T波。

(3)巨倒T波演变迅速,可持续数日后自行消失。

(4)Q-T间期或Q-T\textsubscript{C} 显著延长,常延长20\%以上。

(5)U波增高,其振幅常≥0.15mV。

(6)大多不伴有ST段偏移及异常Q波。

(7)常伴有快速性室性心律失常。

2.发生机制

系交感神经过度兴奋释放大量儿茶酚胺刺激下丘脑星状交感神经节及冠状动脉痉挛造成急性心肌缺血,使心室肌复极过程明显受到影响而出现巨倒T波和Q-T间期显著延长。

3.临床意义

常见于脑血管意外、颅脑损伤、脑肿瘤、阿-斯综合征发作后、伴发交感神经过度兴奋的一些疾病,如各种急腹症、神经外科手术后、肺动脉栓塞等。出现巨倒T波,死亡率增加22\%。巨倒T波若发生在脑血管意外患者中,则提示颅内、蛛网膜下腔出血量大或脑梗死面积广泛,预后不良。

\protect\hypertarget{text00012.htmlux5cux23subid90}{}{}

\subsection{心内膜下梗死性巨倒T波}

心内膜下心肌梗死后出现的巨倒T波,两肢基本对称、基底部可宽可窄、波谷较尖锐或较圆钝,常伴有Q-T间期延长(图\ref{fig6-3}),以R波为主导联ST段可显著压低,但不出现异常Q波,诊断心内膜下心肌梗死需结合心肌酶谱或(和)肌钙蛋白检测。

\begin{figure}[!htbp]
 \centering
 \includegraphics[width=5.78125in,height=2.03125in]{./images/Image00095.jpg}
 \captionsetup{justification=centering}
 \caption{陈旧性下壁心肌梗死患者(下壁异常Q波),脑溢血后出现巨倒T波、Q-T间期显著延长、提示心内膜下心肌梗死(心肌酶谱增高、肌钙蛋白阳性)}
 \label{fig6-3}
  \end{figure} 

\protect\hypertarget{text00012.htmlux5cux23subid91}{}{}

\subsection{劳损型T波倒置}

以R波为主导联T波倒置,两肢不对称,前肢下降较缓慢、后肢上升较快,基底部较窄,且伴有ST段呈下垂型、水平型、弓背向上型压低及R波电压明显增高,为左心室肥大伴劳损或心尖肥厚性心肌病的特征性心电图改变。见于左心室收缩期负荷过重的疾病,如高血压性心脏病、梗阻型肥厚性心肌病及心尖肥厚性心肌病等(图\ref{fig6-4})。

\begin{figure}[!htbp]
 \centering
 \includegraphics[width=3.22917in,height=4.29167in]{./images/Image00096.jpg}
 \captionsetup{justification=centering}
 \caption{梗阻型肥厚性心肌病患者出现左心室肥大伴劳损、V\textsubscript{2}导联r波逆递增、巨倒T波、浅倒U波}
 \label{fig6-4}
  \end{figure} 


\protect\hypertarget{text00012.htmlux5cux23subid92}{}{}

\subsection{功能性T波倒置}

1.孤立性负T综合征

孤立性负T综合征又称为心尖现象。倒置的T波多发生在V\textsubscript{4}导联,偶见于V\textsubscript{4} 、V\textsubscript{5}导联;右侧卧位时,可使倒置的T波恢复直立。可能系心尖与胸壁之间的接触干扰了心肌的复极程序所致。多见于瘦长型的健康青年,属正常变异,但易误诊为心肌炎、心尖肥厚型心肌病。

2.持续性童稚型T波

童稚型T波又称为幼年型T波(图\ref{fig6-5}),常见于婴幼儿。其心电图特征是:①倒置的T波仅见于V\textsubscript{1}~V\textsubscript{4} 导联,且以V\textsubscript{2} 、V\textsubscript{3}导联倒置最深;②倒置的深度多<0.5mV,肢体导联及V\textsubscript{5}、V\textsubscript{6} 导联T波正常。少数人V\textsubscript{1}~V\textsubscript{4}导联T波倒置可一直持续到成人,故称为持续性童稚型T波,可能与无肺组织覆盖“心切迹”区有关,属正常变异。但年轻者易误诊为心肌炎、心尖肥厚型心肌病;年长者易误诊为前间壁心肌缺血。深吸气或口服钾盐可使倒置的T波转为直立,可资鉴别。

\begin{figure}[!htbp]
 \centering
 \includegraphics[width=2.40625in,height=3.86458in]{./images/Image00097.jpg}
 \captionsetup{justification=centering}
 \caption{7岁女孩出现童稚型T波改变(V\textsubscript{2} 、V\textsubscript{3}导联T波倒置深于V\textsubscript{1} 导联)}
 \label{fig6-5}
  \end{figure} 


3.“两点半”综合征

当额面QRS电轴指向+90°(相当于钟表长针指向6字),而T电轴指向-30°(相当于钟表短针指向2字),T-QRS电轴类似于钟表的两点半。心电图特征为:①Ⅰ导联QRS波幅的代数和为零;②Ⅱ、Ⅲ、aVF导联QRS主波向上,而T波倒置,其中Ⅲ导联倒置最深、aVF导联次之;③口服钾盐或运动可使T波恢复正常;④多见于瘦长型健康人。发生在年轻者易误诊为心肌炎,发生在年长者易误诊为心肌缺血。

4.站立性T波改变

T波极性和形态随着体位的改变而改变,多发生在Ⅱ、Ⅲ、aVF导联。站立位或深吸气时,可使Ⅱ、Ⅲ、aVF导联T波倒置加深,或者由平卧位转为站立位时T波由直立转为倒置;反之,可使倒置T波转为直立。多见于心脏神经官能症、瘦长型女性患者。口服普萘洛尔可使此T波异常转为正常。

5.过度通气性T波改变

过度通气在健康人可引起一过性T波倒置。以胸前导联多见,多伴有Q-T间期延长,口服普萘洛尔可防止这种改变。可能与交感神经兴奋早期引起心室肌复极非同步性缩短有关。

6.饱餐后T波改变

饱餐后30min内,即可出现T波倒置,以Ⅰ、Ⅱ、V\textsubscript{2}~V\textsubscript{4}导联明显,空腹时T波恢复正常。如餐中加服钾盐,可防止这种异常T波的产生。可能与餐后糖类吸收使血钾暂时性降低有关。

\protect\hypertarget{text00012.htmlux5cux23subid93}{}{}

\section{T波高耸}

若常规心电图中有3个以上导联出现高耸T波,其振幅≥1.0mV或以R波为主导联T波振幅高于同导联QRS波群的振幅,均称为T波高耸。见于下列情况。

1.超急性期心肌梗死

T波高耸是急性心肌梗死最早的心电图征象,往往出现在ST段抬高之前。胸痛发生后数分钟至数小时内,梗死相关导联即可出现T波高耸。该T波两肢不对称,基底部增宽,伴Q-T间期延长,ST段呈上斜型或斜直型抬高(图\ref{fig6-6})。之后出现异常Q波及ST-T动态演变。其T波高耸的发生机制可能与急性心肌缺血引起早期复极及舒张期除极有关。

\begin{figure}[!htbp]
 \centering
 \includegraphics[width=2.97917in,height=3.47917in]{./images/Image00098.jpg}
 \captionsetup{justification=centering}
 \caption{前间壁、前壁超急性期心肌梗死患者出现r波振幅逆递增、ST段上斜型抬高及T波高耸}
 \label{fig6-6}
  \end{figure} 

2.变异型心绞痛

变异型心绞痛发作时,相关导联出现T波高耸呈箭头状,两肢不对称,基底部增宽,伴Q-T间期延长(图\ref{fig6-7}),ST段呈上斜型或弓背向上型抬高。若原有ST段压低、T波倒置,可使ST段恢复正常或略抬高、T波直立,出现伪善性改变,但无异常Q波出现,无ST-T动态演变,心肌酶谱正常,肌钙蛋白阴性。

\begin{figure}[!htbp]
 \centering
 \includegraphics[width=4.94792in,height=2.15625in]{./images/Image00099.jpg}
 \captionsetup{justification=centering}
 \caption{变异型心绞痛发作时V\textsubscript{2} ~V\textsubscript{4}导联出现ST段呈上斜型抬高伴T波高耸}
 \label{fig6-7}
  \end{figure} 


3.高钾血症

T波高耸呈箭头状,两肢对称,基底部狭窄,称为帐篷状T波,与冠状T波的极性刚好相反,以胸前导联最为显著,常伴Q-T间期缩短。T波高、尖、窄、对称是高钾血症最早的心电图征象(图\ref{fig6-8})。

\begin{figure}[!htbp]
 \centering
 \includegraphics[width=5.78125in,height=2.22917in]{./images/Image00100.jpg}
 \captionsetup{justification=centering}
 \caption{慢性肾炎、高钾血症(6.15mol/L)患者出现前间壁异常Q波(提示Ⅰ型左中隔支阻滞)、帐篷状T波改变}
 \label{fig6-8}
  \end{figure} 

4.早复极综合征

左胸导联出现T波高耸,常呈拱形,两肢基本对称,基底部较宽,多伴有J点抬高(0.1~0.4mV),ST段呈凹面向上型抬高,R波降肢粗钝(图\ref{fig6-9});运动后抬高的J点、ST段恢复正常或减轻。系迷走神经张力过高引起心室肌不同步提前复极所致。多见于运动员、年轻体力劳动者等健壮人,多属正常变异,极少数可能与心源性猝死有关。

\begin{figure}[!htbp]
 \centering
 \includegraphics[width=1.79167in,height=3.60417in]{./images/Image00101.jpg}
 \captionsetup{justification=centering}
 \caption{健康体检者出现左心室高电压,符合早复极综合征波形特点}
 \label{fig6-9}
  \end{figure} 

5.左心室舒张期负荷过重

左心室舒张期负荷过重的主要病理变化是左心室扩张,其心电图特征为左胸导联R波电压增高,ST段轻度抬高,T波高耸,两肢不对称,基底部较宽。见于二尖瓣关闭不全、主动脉瓣关闭不全等(图\ref{fig6-10})。

\begin{figure}[!htbp]
 \centering
 \includegraphics[width=1.70833in,height=4.95833in]{./images/Image00102.jpg}
 \captionsetup{justification=centering}
 \caption{风心病、二尖瓣狭窄伴关闭不全患者出现心房颤动、左心室肥大、T波高耸}
 \label{fig6-10}
  \end{figure} 

6.部分脑血管意外

部分脑血管意外患者可出现高而宽的T波,多伴有Q-T间期延长,以V\textsubscript{3}~V\textsubscript{6}导联最显著(图\ref{fig6-11})。动物实验证明,若刺激左侧星状神经节,则出现T波明显高耸;若刺激右侧星状神经节,则T波明显倒置。

\begin{figure}[!htbp]
 \centering
 \includegraphics[width=3.60417in,height=1.47917in]{./images/Image00103.jpg}
 \captionsetup{justification=centering}
 \caption{脑溢血患者V\textsubscript{2} ~V\textsubscript{6}导联出现ST段上斜型抬高、T波高耸}
 \label{fig6-11}
  \end{figure} 


7.左束支传导阻滞

左束支传导阻滞患者V\textsubscript{1} ~V\textsubscript{4}导联出现ST段抬高,高耸T波,两肢不对称,基底部较宽。

\protect\hypertarget{text00012.htmlux5cux23subid94}{}{}

\section{双峰T波}

1.典型的“圆顶-尖角状”T波

V\textsubscript{2} ~V\textsubscript{4} 导联尤其是V\textsubscript{3}导联出现典型的“圆顶-尖角状”T波,其特征是双峰T波的第2峰呈尖角状,并高于第1峰,第2峰上升肢始于第1峰下降肢早期(图\ref{fig6-12})。多见于室间隔缺损。

\begin{figure}[!htbp]
 \centering
 \includegraphics[width=2.82292in,height=4.26042in]{./images/Image00104.jpg}
 \captionsetup{justification=centering}
 \caption{先心病、室间隔缺损患者,V\textsubscript{2} 、V\textsubscript{3}导联出现“圆顶-尖角状”T波,V\textsubscript{5} 、V\textsubscript{6}导联出现T波高耸、左心室肥大}
 \label{fig6-12}
  \end{figure} 


2.顶部微凹型双峰T波

多数导联T波基底部增宽,振幅降低,顶部呈微凹型双峰波,伴有Q-T间期延长。多见于药物影响(如胺碘酮等)、电解质紊乱(低钾、低镁血症)、脑血管意外及正常变异等。系左、右心室复极在时间上的差异所致,前峰为左心室复极,后峰为右心室复极。

\protect\hypertarget{text00012.htmlux5cux23subid95}{}{}

\section{电张调整性T波改变}

在间歇性束支传导阻滞、间歇性预激综合征、右心室起搏或宽QRS心动过速患者中,心室异常除极消除后恢复正常除极一段时间内,仍存在明显的T波改变酷似心肌缺血,其T波方向与异常除极时QRS主波方向一致,Rosenbaum称之为电张调整性T波改变。电张调整性T波改变是介于原发性与继发性T波改变之间的第3种T波改变,不具有病理性意义,是一种正常的电生理现象。不论是自发还是诱发引起的心室除极异常,均可出现继发性T波改变和电张调整性T波改变,但后者往往被前者所掩盖,只有心室异常除极消失恢复正常除极时,继发性T波改变消失后,电张调整性T波改变才得以显现出来,即正常除极后的T波方向与原异常除极时QRS主波方向一致。电张调整性T波改变往往持续一段时间,这与心脏记忆现象和积累作用有关。

1.心动过速后T波倒置

(1)室性心动过速后T波倒置:①该室性心动过速QRS波形多呈左束支阻滞型伴电轴左偏,时间≥0.12s;②心动过速后Ⅱ、Ⅲ、aVF、V\textsubscript{4}~V\textsubscript{6}导联T波仍倒置,与室性心动过速时QRS主波的方向一致;③倒置T波的深度、宽度与室性心动过速持续时间成正比;④倒置T波恢复正常极性、形态需要一定的时间,数小时至数周不等,与心室异常除极的时间成正比。

(2)室上性心动过速伴心室内差异性传导后T波倒置:该心室内差异性传导QRS波形呈左束支阻滞型伴电轴左偏,心动过速后T波特征与室性心动过速后T波倒置相似。

2.间歇性左束支阻滞消失后T波倒置

其心电图特征:①异常除极的QRS波群呈左束支阻滞图形;②左束支阻滞消失后,V\textsubscript{4}~V\textsubscript{6}导联或同时伴Ⅱ、Ⅲ、aVF导联T波倒置;③倒置T波的深度、宽度和持续时间与左束支阻滞发作的时间成正比。

3.预激综合征消失后T波倒置

其心电图特征:①异常除极的QRS波群呈预激波形特征;②预激消失或射频治疗后,V\textsubscript{4}~V\textsubscript{6}导联T波倒置;③倒置T波的深度、宽度和持续时间与预激程度、持续时间成正比。

4.右心室起搏后T波倒置

其心电图特征:①倒置T波多发生在Ⅱ、Ⅲ、aVF、V\textsubscript{5}、V\textsubscript{6}导联;②倒置T波的深度、宽度和持续时间与起搏的强度和持续的时间有关(图\ref{fig6-13})。

\begin{figure}[!htbp]
 \centering
 \includegraphics[width=5.78125in,height=1.75in]{./images/Image00105.jpg}
 \captionsetup{justification=centering}
 \caption{心房扑动、VVI起搏器术后,V\textsubscript{5}导联出现电张调整性T波改变}
 \label{fig6-13}
  \end{figure} 


\protect\hypertarget{text00012.htmlux5cux23subid96}{}{}

\section{T波电交替现象}

1.基本概念

T波电交替现象是指心脏自身复极过程中所出现的T波形态、振幅甚至极性发生交替性改变,不伴有QRS波群交替变化,通常每隔1次心搏出现1次,并排除呼吸、体位、胸腔或心包积液等心外因素。

2.心电图特征

(1)主导节律恒定,多为窦性,其QRS波形、振幅一致。

(2)T波交替性改变的幅度较明显,发生在以R波为主导联价值大。

(3)心动过缓时出现比心动过速时出现T波电交替价值大。

(4)常伴有Q-T间期延长或同时伴Q-T间期长、短交替。

(5)与心外因素无关,如呼吸、体位、心包积液、胸腔积液等。

(6)可伴有ST段、U波甚至P波、QRS波群电交替(图\ref{fig6-14}、图\ref{fig6-15})。

\begin{figure}[!htbp]
 \centering
 \includegraphics[width=5.78125in,height=0.64583in]{./images/Image00106.jpg}
 \captionsetup{justification=centering}
 \caption{窦性心动过缓、左心室高电压、T波电交替现象(定准电压0.5mV)}
 \label{fig6-14}
  \end{figure} 

\begin{figure}[!htbp]
 \centering
 \includegraphics[width=5.1875in,height=1.40625in]{./images/Image00107.jpg}
 \captionsetup{justification=centering}
 \caption{3:1传导的心房扑动患者,V\textsubscript{5} 导联出现T波电交替现象}
 \label{fig6-15}
  \end{figure} 

3.发生机制

T波电交替可能与电解质紊乱(低钙、低镁、低钾血症)、心肌缺血缺氧、支配心脏的植物神经失衡等因素有关。

4.临床意义

显著的T波、Q-T间期电交替,是心室复极不一致、心电活动不稳定的表现,易诱发严重的室性心律失常而猝死。多见于长Q-T间期综合征、心肌缺血、心功能不全及电解质紊乱等患者。有T波电交替者,发生致命性室性心律失常的危险性增加14倍。T波电交替目前已成为识别心源性猝死高危患者的一个重要而非常直观的指征。

\protect\hypertarget{text00012.htmlux5cux23subid97}{}{}

\section{与心动周期长短有关的倒置T波}

与心动周期长短有关的倒置T波又称为与慢心率相关的频率依赖性T波改变,仅出现在心率缓慢时(图\ref{fig6-16})。随着运动或给予阿托品、异丙基肾上腺素使心率增快,则倒置的T波恢复正常。可能系迷走神经反射所致。有学者认为长间歇可使心室充盈期延长,其舒张容积增加,导致心室复极改变或与长间歇后心肌收缩性的改变有关,或长间歇使心室内压力升高,影响冠状动脉血流量导致心内膜下心肌缺血,或心室内血流动力学改变引起心肌纤维的伸展等,这些因素均可造成T波改变。

\begin{figure}[!htbp]
 \centering
 \includegraphics[width=5.39583in,height=1.03125in]{./images/Image00108.jpg}
 \captionsetup{justification=centering}
 \caption{冠心病、心房颤动患者出现与心动周期长短有关的巨倒T波}
 \label{fig6-16}
  \end{figure} 

\protect\hypertarget{text00012.htmlux5cux23subid98}{}{}

\section{早搏后T波改变}

室性早搏、房性早搏伴心室内差异性传导后都会引起心室除极异常,其后的第1个或数个正常除极的窦性搏动的T波出现改变,如增高、降低、平坦、切迹或倒置,甚至高低交替出现或振幅逐渐改变(图\ref{fig6-17})。既往认为这一现象属病理性的原发性T波改变,提示心脏有器质性病变。但Leachman等认为这类T波改变与冠状动脉疾病、左心室功能不全的存在与否均无关,而仅与早搏后较长的代偿间歇有关。现倾向于心室电张调整所致,是一种功能性改变。

\begin{figure}[!htbp]
 \centering
 \includegraphics[width=5.59375in,height=0.91667in]{./images/Image00109.jpg}
 \captionsetup{justification=centering}
 \caption{显著窦性心动过缓、房室交接性逸搏及逸搏心律(R\textsubscript{1}、R\textsubscript{3} ~R\textsubscript{5} ,其中R\textsubscript{3}逆传心房)、室性早搏伴逆传心房及早搏后T波高耸、轻度ST段改变}
 \label{fig6-17}
  \end{figure} 


\protect\hypertarget{text00012.htmlux5cux23subid99}{}{}

\section{间歇性T波改变}

间歇性T波改变是指心脏自身复极过程中所出现的T波形态、振幅甚至极性发生间歇性改变,可伴有QRS波群间歇性变化,通常每隔数次心搏出现1次间歇性改变,并排除呼吸、体位、胸腔或心包积液等心外因素。可能与电解质紊乱(低钙、低镁、低钾血症)、心肌缺血缺氧、心肌炎症及支配心脏的植物神经失衡等因素有关(图\ref{fig6-18}、图\ref{fig6-19})。

\begin{figure}[!htbp]
 \centering
 \includegraphics[width=5.58333in,height=1.30208in]{./images/Image00110.jpg}
 \captionsetup{justification=centering}
 \caption{冠心病患者,MV\textsubscript{5}导联在频率相等、QRS波形一致时出现T波负相波深浅不一及ST段压低程度不一}
 \label{fig6-18}
  \end{figure} 


\begin{figure}[!htbp]
 \centering
 \includegraphics[width=5.58333in,height=0.75in]{./images/Image00111.jpg}
 \captionsetup{justification=centering}
 \caption{冠心病患者出现QRS波群、ST段及T波间歇性改变}
 \label{fig6-19}
  \end{figure} 

\protect\hypertarget{text00013.html}{}{}

\protect\hypertarget{text00013.htmlux5cux23chapter13}{}{}

\chapter{正常Q-T间期及其异常改变}

\protect\hypertarget{text00013.htmlux5cux23subid100}{}{}

\section{Q-T间期及Q-T\textsubscript{C}}

Q-T间期是QRS波群、ST段及T波时间的总和,代表心室肌除极和复极所需的时间。其长短主要取决于内向钠、钙电流和外向钾、氯电流的表达、特性及其之间的平衡,同时也受心率、年龄、性别及心外因素(如酸碱平衡失调、电解质紊乱、儿茶酚胺、乙酰胆碱)等影响。

在正常情况下,Q-T间期等于K$\sqrt{\text{R-R}}$
,K为常数,即0.39±0.04,通常以0.40计算之,R-R为相邻两个心搏的心动周期。心率校正后Q-T间期称为Q-T\textsubscript{c},Q-T\textsubscript{c} =Q-T$\sqrt{\text{R-R}}$
,正常值为男性0.40±0.04s,女性0.42±0.04s。

Q-T间期异常主要表现为Q-T间期延长和Q-T间期缩短。前者易诱发严重的室性心律失常而猝死,而后者近年来亦认为是致心律失常性猝死和临终前的心电图改变之一,也是一种严重的心电现象。“2009年国际指南”提出Q-T间期延长的标准为:女性≥0.46s,男性≥0.45s。Q-T间期缩短的标准为:男性、女性均≤0.39s(本人持有异议)。

\protect\hypertarget{text00013.htmlux5cux23subid101}{}{}

\section{Q-T间期延长}

\protect\hypertarget{text00013.htmlux5cux23subid102}{}{}

\subsection{特发性Q-T间期延长}

特发性Q-T间期延长又称为先天性长Q-T间期综合征。具有家族性遗传特征,猝死危险性高,主要由尖端扭转型室性心动过速和心室颤动所致。心电图特征有:①Q-T间期延长或Q-Tc男性≥0.47s、女性≥0.48s;②T波改变,表现为T波宽大、双峰切迹或低平、ST段平直或上斜型延长伴T波高尖;③U波增高;④有时可见Q-T间期长、短交替及T波、U波电交替,具有诊断意义;⑤常于运动、激动、惊恐等交感神经张力增高时发作尖端扭转型室性心动过速,具有肾上腺素能依赖性的临床特征。尖端扭转型室性心动过速若短期内自行终止,仅表现为晕厥;若蜕变为心室颤动,则极易导致猝死(图\ref{fig7-1}、图\ref{fig7-2})。

\begin{figure}[!htbp]
 \centering
 \includegraphics[width=5.78125in,height=1.27083in]{./images/Image00114.jpg}
 \captionsetup{justification=centering}
 \caption{先天性长Q-T间期综合征患者出现窦性心动过缓、Q-T间期延长达0.62s(正常最高值0.48s)}
 \label{fig7-1}
  \end{figure} 

\begin{figure}[!htbp]
 \centering
 \includegraphics[width=5.78125in,height=2.39583in]{./images/Image00115.jpg}
 \captionsetup{justification=centering}
 \caption{与图\ref{fig7-1}系同一患者,室性早搏落在T波降肢上诱发尖端扭转型室性心动过速}
 \label{fig7-2}
  \end{figure} 

\protect\hypertarget{text00013.htmlux5cux23subid103}{}{}

\subsection{继发性Q-T间期延长}

继发性Q-T间期延长又称为后天获得性长Q-T间期综合征(图\ref{fig7-3})。由药物(多由Ⅰ类和Ⅲ类抗心律失常药)、电解质紊乱(低钾、低钙、低镁血症)、甲状腺功能减退、脑血管意外、冠心病、心肌病、心肌梗死12~24h后伴随T波倒置时、缓慢性心律失常等所致,除Q-T间期或Q-Tc间期延长外,尖端扭转型室性心动过速常以长-短周期顺序和间歇依赖性的形式发作。

\begin{figure}[!htbp]
 \centering
 \includegraphics[width=5.78125in,height=1.86458in]{./images/Image00116.jpg}
 \captionsetup{justification=centering}
 \caption{脑血管意外患者,显示窦性心动过缓、Q-T间期延长达0.57s(正常最高值0.44s)、T波改变}
 \label{fig7-3}
  \end{figure} 

\protect\hypertarget{text00013.htmlux5cux23subid104}{}{}

\section{Q-T间期缩短}

Rautahariju等提出Q-T间期预测值(单位ms)为656÷(1+心率/100),正常Q-T间期的下限值为Q-T间期预测值的88\%。当所测的Q-T间期小于预测值的88\%或Q-Tc<0.40s或Q-T间期<0.33s时,便可认为Q-T间期缩短或短Q-T间期。Q-T间期缩短分为特发性和继发性两种。短Q-T间期与长Q-T间期一样,也是发生猝死的危险因素,应值得关注。

\protect\hypertarget{text00013.htmlux5cux23subid105}{}{}

\subsection{特发性Q-T间期缩短}

1.基本概念

特发性Q-T间期缩短又称为特发性短Q-T间期综合征,是近年来发现的又一种可诱发严重心律失常而猝死的原发性心电异常疾病,与心脏离子通道功能异常有关,是一种单基因突变引起的遗传性疾病。

2.分类及机制

分为3种类型。这3种类型均可引起动作电位时程和不应期不均一性缩短,导致Q-T间期缩短、心室易损期增加及M细胞与其他心肌细胞的复极离散度增加,促使致命性心律失常的发生。

(1)Ⅰ型:由于HERG基因的N588K突变导致I\textsubscript{kr}
(快速激活的延迟整流钾离子流)功能获得显著增加,使心肌细胞的动作电位3相钾离子流迅速外流,导致动作电位2、3相时程缩短。

(2)Ⅱ型:由于KCNQ1基因的V307L突变导致I\textsubscript{ks}
(缓慢激活的延迟整流钾离子流)功能获得显著增加,使动作电位2相时程缩短。

(3)Ⅲ型:由于KCNJ2基因的D172N突变导致I\textsubscript{kl}
(内向整流钾离子流)功能获得显著增加,使动作电位3相时程缩短。

3.临床及心电图特征

(1)具有家族遗传性,多数病例有心悸、头晕等症状,且有晕厥、心脏骤停、猝死或猝死家族史。

(2)无心脏结构异常和其他器质性心脏病。

(3)持续出现短Q-T间期,大多为216~290ms,为Q-T间期预测值的52\%~78\%,Q-Tc为248~302ms。

(4)多数病例表现为非频率依赖性持续性短Q-T间期;少数病例表现为慢频率依赖性短Q-T间期矛盾性缩短,即心室率较慢时,其Q-T间期缩短,而心室率较快时,其Q-T间期反而恢复正常或延长。

(5)ST段明显缩短(<50ms)或消失,T波高尖,近似于对称,尤以胸前导联为明显。

(6)在症状明显时,多数病例可出现心房颤动或心室颤动,个别病例可出现一过性心动过缓或二度~三度房室传导阻滞。

(7)电生理检查时,其心房、心室有效不应期缩短(<170ms),易诱发心房颤动、室性心动过速、心室颤动(图\ref{fig7-4})。

\begin{figure}[!htbp]
 \centering
 \includegraphics[width=5.78125in,height=1.48958in]{./images/Image00117.jpg}
 \captionsetup{justification=centering}
 \caption{上行系43岁男性患者晕厥时记录,显示心室颤动;下行系电击复律后记录,显示窦性心动过缓、Q-T间期缩短(0.28s),提示特发性短Q-T间期综合征}
 \label{fig7-4}
  \end{figure} 

4.心电图表现类型

(1)A型:ST段、T波时间均缩短,同时伴有T波高尖,易发生房性和室性心律失常。

(2)B型:以T波高尖和时间缩短为主,ST段改变不明显,以房性心律失常为主。

(3)C型:以ST段缩短为主,T波时间缩短不明显,以室性心律失常为主。

\protect\hypertarget{text00013.htmlux5cux23subid106}{}{}

\subsection{继发性Q-T间期缩短}

1.基本概念

继发性Q-T间期缩短又称为继发性短Q-T间期综合征,是继发于电解质异常(高钙血症、高钾血症)、儿茶酚胺类药物(肾上腺素、异丙肾上腺素、多巴胺等)影响、洋地黄效应或中毒、超急性期心肌梗死、甲状腺功能亢进、迷走神经张力过高引起的早复极综合征及心肺复苏后的危重病例等。

2.发病机制

(1)ST段代表心室肌动作电位2相平台期,具有心率依赖性,受儿茶酚胺、细胞外Ca\textsuperscript{2+}
浓度、心肌病变及药物等因素的影响,如使用儿茶酚胺类及洋地黄类药物、高钙血症、心肌急性缺血、缺氧、损伤等导致细胞膜受损,出现Ca\textsuperscript{2+}
持续内流,均可引起ST段缩短或消失。

(2)T波代表心室肌动作电位3相,凡是能引起心肌细胞膜对K\textsuperscript{+}通透性增加使3相复极加速,均可导致T波变窄,时间缩短。

(3)一过性矛盾性Q-T间期缩短常由心外因素所致,受自主神经调节。当心脏迷走神经张力异常增高时,释放过量的乙酰胆碱将抑制I\textsubscript{Ca}
电流和激活I\textsubscript{K、Ach} 电流,导致心室复极时间缩短。

3.临床及心电图特征

(1)继发于其他疾病或药物影响。

(2)短Q-T间期多小于预测值的80\%。

(3)ST段明显缩短或消失,QRS波群结束后立即出现T波上升肢,T波高尖,近似于对称,尤以胸前导联为明显。

(4)心肺复苏后发生的短Q-T间期,多伴随心动过缓、二度~三度房室传导阻滞、不定型心室内传导阻滞及心室停搏等(图\ref{fig7-5})。

\begin{figure}[!htbp]
 \centering
 \includegraphics[width=5.89583in,height=0.66667in]{./images/Image00118.jpg}
 \captionsetup{justification=centering}
 \caption{冠心病、多发性损伤患者,心肺复苏后出现窦性心动过缓、不完全性左心房内传导阻滞、二度房室传导阻滞、不定型心室内传导阻滞、继发性Q-T间期缩短(0.31s)}
 \label{fig7-5}
  \end{figure} 

\protect\hypertarget{text00013.htmlux5cux23subid107}{}{}

\subsection{临床意义}

(1)Q-Tc延长和缩短者与平均Q-Tc正常者(400~440ms)相比,猝死的危险性均增加2倍,表明短Q-T间期与长Q-T间期、Brugada综合征一样,也是发生猝死的危险因素。

(2)心肺复苏过程中、复苏后出现的短Q-T间期,是一种严重的心电现象,预示着很快会出现二度、三度房室传导阻滞及心室停搏,是临终前的心电图表现之一。

(3)短Q-T间期尚见于服用雄性激素患者。有学者认为,Q-Tc间期<0.38s是一项预测滥用雄性激素强有力的指标(敏感性83\%、特异性88\%),故提出检测Q-T间期,可作为运动员服用兴奋剂筛选的指标。

\protect\hypertarget{text00013.htmlux5cux23subid108}{}{}

\section{Q-T间期交替性改变}

由于Q-T间期包括QRS波群、ST段及T波的时间,这3个波段中任何一个波段时间的交替性改变,均会导致Q-T间期长、短交替性改变。有以下两种分类方法:

1.根据各个波段时间发生长、短交替性改变分类

(1)QRS波群时间发生长、短交替性改变:见于交替性预激综合征、交替性束支阻滞等。

(2)ST段时间发生长、短交替性改变。

(3)T波时间发生长、短交替性改变。

2.根据Q-T间期正常与否分类

(1)短Q-T间期时长、短交替:Q-T间期均缩短,但缩短的程度呈长、短交替。

(2)长Q-T间期时长、短交替:Q-T间期均延长,但延长的程度呈长、短交替。

(3)正常Q-T间期时长、短交替:Q-T间期正常范围,但出现长、短交替性改变。

\protect\hypertarget{text00014.html}{}{}

\protect\hypertarget{text00014.htmlux5cux23chapter14}{}{}

\chapter{正常U波及其异常改变}

\protect\hypertarget{text00014.htmlux5cux23subid109}{}{}

\subsection{正常U波}

在正常情况下,U波与T波方向一致,振幅<0.2mV,不超过同导联$\frac{1}{2}$
T波,时间0.08~0.26s,在V\textsubscript{2} ~V\textsubscript{4}导联最为明显。U波具有频率依赖性,当心率>95次/min时,则很少出现;而心率减慢时,则出现U波或U波振幅增高。至于U波是怎样形成的,尚不清楚,但机械-电耦联所引起的后电位形成U波的学说得到关注,即心室肌的伸展能够激活心肌细胞机械敏感的离子通道而形成后电位。

\protect\hypertarget{text00014.htmlux5cux23subid110}{}{}

\subsection{U波增高}

当U波振幅增高大于同导联T波或其振幅≥0.20mV时,便称为U波增高;若U波振幅>0.5mV,则为明显增高。当U波增高与T波融合后,则测量Q-U间期。若Q-U间期大于Q-T间期正常最高值加U波时间均值0.20s,则称为Q-U间期延长,意义如同Q-T间期延长,甚至更为严重。U波增高见于下列情况:

(1)电解质紊乱:低钾血症、高钙血症等(图\ref{fig8-1})。

\begin{figure}[!htbp]
 \centering
 \includegraphics[width=4.92708in,height=1.83333in]{./images/Image00120.jpg}
 \captionsetup{justification=centering}
 \caption{低钾血症(3.1mol/L)出现完全性右束支阻滞、T波宽钝切迹、U波增高(V\textsubscript{2}、V\textsubscript{3} 导联)、T波与U波融合及Q-U间期延长}
 \label{fig8-1}
  \end{figure} 


(2)药物影响:抗心律失常药物影响(如胺碘酮等)、洋地黄、肾上腺素、钙剂、抗精神病药物等。

(3)三度房室传导阻滞、缓慢性心律失常长R-R间歇后、早搏代偿间歇后等。

(4)急性脑血管意外:出血性比缺血性脑血管疾病更常见,尤以蛛网膜下腔出血为著。

(5)迷走神经张力过高。

(6)心绞痛发作或运动时出现胸前导联U波增高,见于左冠状动脉回旋支或(和)右冠状动脉狭窄75\%以上(敏感性70\%,特异性98\%)。

(7)急性后壁、下壁心肌梗死:约60\%~72\%患者左胸导联出现U波增高。

(8)右心室肥大或负荷过重:右胸导联出现U波倒置和左胸导联U波增高。

(9)若服用可引起Q-T间期延长及尖端扭转型室性心动过速药物后,U波增高的病理意义超过Q-T间期延长,在高大的U波之后常出现室性早搏,甚至是尖端扭转型室性心动过速。

\protect\hypertarget{text00014.htmlux5cux23subid111}{}{}

\subsection{U波倒置}

在正常情况下,以R波为主的导联,U波不应该倒置。若出现U波倒置,则见于下列情况:

(1)急性心肌梗死:前壁梗死发生率约10\%~60\%,下壁梗死约30\%~33\%,多见于ST-T改变和异常Q波出现之前,而在冠状动脉介入治疗或急性期后数小时~24h可消失。

(2)心肌缺血:尤其是左冠状动脉前降支病变所引起的心肌缺血。若运动试验后出现U波倒置,则是心肌缺血的佐证,为运动试验阳性标准之一,常提示左前降支近端或左主干病变。

(3)高血压病:其倒置程度随着血压升高而加深,随着血压降低和恢复正常而变浅或直立,可作为判断病情和疗效的参考指标之一。

(4)左心室劳损:左心室肥大、负荷过重时,除U波倒置外,常合并ST-T改变。

(5)右心室肥大或负荷过重:右胸导联出现U波倒置和左胸导联U波增高。

(6)急性肺栓塞:可出现一过性右胸导联U波倒置。

\protect\hypertarget{text00014.htmlux5cux23subid112}{}{}

\subsection{双相型U波改变}

(1)初始型:即U波呈负、正双相。见于高血压病、左心室肥大及老年患者等,提示左心室早期劳损或舒张功能不全。

(2)终末型:即U波呈正、负双相。见于心肌缺血、冠心病等。

(3)不稳定型心绞痛发作时,左胸导联出现双相型U波是发生急性心肌梗死的独立预测指标之一。

\protect\hypertarget{text00014.htmlux5cux23subid113}{}{}

\subsection{U波电交替现象}

1.心电图特征

(1)同一导联上U波均直立,其振幅呈高、低交替或均倒置,其深、浅程度交替或直立与倒置呈交替发生(图\ref{fig8-2}、图\ref{fig8-3})。

\begin{figure}[!htbp]
 \centering
 \includegraphics[width=5.76042in,height=1.51042in]{./images/Image00121.jpg}
 \captionsetup{justification=centering}
 \caption{冠心病患者出现2:1二度房室传导阻滞及ST段、T波、U波电交替现象}
 \label{fig8-2}
  \end{figure} 

\begin{figure}[!htbp]
 \centering
 \includegraphics[width=5.57292in,height=0.66667in]{./images/Image00122.jpg}
 \captionsetup{justification=centering}
 \caption{反复晕厥患者,出现Q-T间期延长、U波电交替现象(直立与倒置交替)及尖端扭转型室性心动过速(图片未刊出)(引自陆菊芬)}
 \label{fig8-3}
  \end{figure} 

(2)常伴Q-T间期或Q-T\textsubscript{C} 延长,标志着心室复极延迟。

(3)心率缓慢或长间歇之后U波增高,易发生电交替现象。

(4)早搏之后或室性心动过速之前,U波往往增高伴电交替,有人称为舒张期振荡波,U波越高,越易诱发室性心律失常。

2.发生机制

U波振幅电交替与心输出量交替性改变有关,并非心电活动异常所致。心室容量越大、心室收缩越强,其U波振幅越高大。而U波极性电交替,则可能与心肌损害有关。

3.临床意义

(1)U波电交替常合并交替脉,是提示左心功能不全有意义的征象。

(2)高大U波伴电交替是心肌兴奋性增高的表现,常是严重室性心律失常的前兆。

(3)U波电交替见于低钾、低钙、低镁血症及胺碘酮中毒、脑外伤等。

(4)U波电交替和长间歇后胸前导联倒置的U波转为直立,与儿茶酚胺敏感性室性心动过速发生有关。

\protect\hypertarget{text00014.htmlux5cux23subid114}{}{}

\subsection{早搏后U波改变}

室性早搏、房性早搏后的第1个或数个窦性搏动的U波出现改变,如增高、倒置或振幅逐渐改变。具体机制及临床意义不详,可能与早搏后较长的代偿间歇有关,部分高大U波易触发室性早搏的发生(图\ref{fig8-4}、图\ref{fig8-5})。

\begin{figure}[!htbp]
 \centering
 \includegraphics[width=5.78125in,height=1.6875in]{./images/Image00123.jpg}
 \captionsetup{justification=centering}
 \caption{MV\textsubscript{5}导联连续记录,显示室性早搏后出现U波振幅电阶梯现象}
 \label{fig8-4}
  \end{figure} 


\begin{figure}[!htbp]
 \centering
 \includegraphics[width=5.78125in,height=1.125in]{./images/Image00124.jpg}
 \captionsetup{justification=centering}
 \caption{窦性心动过缓、短阵性房性心动过速后U波振幅明显增高、房性逸搏(P\textsubscript{6})、房性融合波(P\textsubscript{7} )、ST-T-U改变、Q-T间期延长}
 \label{fig8-5}
  \end{figure} 


\protect\hypertarget{text00014.htmlux5cux23subid115}{}{}

\subsection{U波消失}

有文献报道,成年人心电图始终未出现U波者,可能是易发生心肌梗死的危险因素。

\protect\hypertarget{text00015.html}{}{}

\protect\hypertarget{text00015.htmlux5cux23chapter15}{}{}


\part{休  克}

\chapter{休 克 概 论}

休克(shock)是指机体由于受到外来的或内在的强烈致病因素打击或两者共同作用而出现的以机体代谢异常和循环功能紊乱为主的一组临床综合征,这些致病因素包括大出血、创伤、中毒、烧伤、窒息、感染、过敏及心脏泵功能衰竭等。1743年法国医生Henri
Francois Le
Dran首次报告了严重外伤后的“choc”现象,认为休克是由于中枢神经功能紊乱导致的循环以及其他器官功能不全的危重状态。此后,Warren在19世纪后期又将创伤性休克描述为面色苍白或发绀、四肢湿冷、脉搏细速、尿少和神经淡漠的“濒死状态”,Crile后来又补充了“低血压”这一重要特征,这些特征性描述已经基本上反映出了休克的主要临床特点。20世纪60年代后,随着Lillehei提出的“微循环障碍学说”的创立以及基础研究的深入发展,对休克现象的认识水平逐步提高。目前认为休克是各种病因所致的急性血液循环障碍,是一个以低血压和微循环灌注锐减为特点、导致重要器官灌注不足、组织氧供(O\textsubscript{2}
supply)和氧需(O\textsubscript{2}
demand)失衡以及细胞功能紊乱和代谢障碍的危重病理过程。

\subsection{病因与发病机制}

休克的发生的病理生理过程与其诱因密切相关,其分类也以病因学分类为主,目前对于休克分类尚无统一认识,临床上可根据休克发生的始动特点将其简明地分为三大类:①低血容量性(hypovolemic)休克,包括失血性(hemorrhagic)、烧伤性(burn)和创伤性(traumatic)休克三种,其共同环节都有血容量的降低;②血管扩张性(vasodilatory)或者分布性(distributive)休克,包括感染性或者脓毒症(septic)、过敏性(anaphylactic)和神经源性(neurogenic)休克三种;③心源性(cardiogenic)休克,包括心脏本身病变、心脏压迫或梗阻引起的休克。美国外科医师学会把由于心脏外因素包括大面积肺栓塞、心包填塞及缩窄性心包炎等所致的心脏泵功能障碍单独归类为梗阻性(obstructive)休克,但其血流动力学特点与心脏本身疾患所致休克的特点类似。

另外,按休克时血液的动力学的特点也可分为两类,即低排高阻型休克或称低动力型休克(hypodynamic
shock)和高排低阻型休克或称高动力型休克(hyperdynamic
shock)。低动力型休克临床上常见,包括了低血容量性、心源性和大多数感染性休克,特点是心脏排血量低,而总外周血管阻力高,由于皮肤血管收缩,血流量减少,使皮肤温度降低,故又称为“冷性休克(cold
shock)”;而高动力型休克常见于部分感染性休克,其总外周血管阻力低,心脏排血量高,由于皮肤血管扩张,血流量增多,使皮肤温度升高,故亦称“温性休克(warm
shock)”。

\subsubsection{病因}

\paragraph{低血容量性休克}

患者发生低血容量性休克的内在原因在于血管内容量(绝对或相对)不足,并由此引起心室充盈不足和心搏量减少,如果增加心率仍不能代偿,可导致心排量降低。血流动力学特点表现为心室前负荷下降并导致心室舒张末压力及容积、每搏量、心输出量降低,最终导致血压下降。休克严重程度与液体损失程度及速度密切相关。一般15分钟内失血少于全身循环血量的10\%时,机体可通过加快心率及增加体循环阻力(SVR)进行代偿,使血压和组织灌流量保持稳定。若快速失血量超过循环血量的20\%左右,机体将开始处于失代偿状态,SVR显著上升,组织微循环血流下降,无氧酵解增强,乳酸水平开始上升。当失血量超过循环血量40\%时,患者表现出明显的休克体征。

低血容量性休克的常见的原因为急性出血,但亦可由于体液丧失增加造成。后者发展到低血容量常需数小时以上,过程相对较隐匿,由于血液浓缩,可伴血红蛋白(Hb)或血细胞比容(Hct)增加。①失血性休克:是指因大量失血,迅速导致有效循环血量锐减而引起外周循环衰竭的一种综合征。常见于消化性溃疡、食管静脉曲张、主动脉夹层破裂以及妊娠或生产等原因引起的大出血,出血可为显性(如呕血或黑粪)或隐性(如异位妊娠破裂)。②烧伤性休克:大面积烧伤,伴有血浆大量丢失,可引起烧伤性休克。休克早期与疼痛及低血容量有关,晚期可继发感染,发展为感染性休克。③创伤性休克:严重创伤可导致创伤性休克,这种休克的发生与疼痛刺激和失血有关。

\paragraph{血管扩张性休克}

这类休克通常是由于血管扩张所致的血管内容量相对不足,其循环血容量正常或增加,但心脏充盈和组织灌注不足,很多情况下存在着广泛的静脉或小动脉扩张。血管阻力减低而心排量不能相应增加使得体循环血压下降、器官灌注不良,当冠脉灌注不足时可继发出现心肌功能不全或其他病理机制(如心肌抑制因子或其他毒性物质的释放),使血管扩张所致的休克复杂化。其最主要特点是外周血管扩张,毛细血管通透性增加,液体渗漏,有效循环血量下降,前负荷降低。常见于以下三种:①感染性休克:是临床上最常见的休克类型之一,各种病原微生物感染严重时均可引起感染性休克,临床上以G\textsuperscript{−}
杆菌感染最常见。细菌内毒素作用是导致感染性休克微循环障碍和炎性器官功能损害的重要因素。根据血流动力学的特点又可将其分为低动力型的冷休克和高动力型的暖休克两型。②过敏性休克:由IgE介导的Ⅰ型超敏反应引起。已致敏的机体再次接触到抗原物质时,可发生强烈的变态反应。IgE激活肥大细胞或嗜碱性粒细胞,导致组胺、白介素、缓激肽、前列腺素等炎症因子大量释放入血,使患者全身容量血管扩张,毛细血管通透性增加和支气管痉挛,继之出现的弥散性非纤维蛋白血栓、血压下降、组织灌注不良可使多脏器受累。最近的文献提示,人群中有1\%~2\%的人可能发生过敏性休克。这类休克常见的变应原有药物(如青霉素、神经肌肉阻断剂)、花粉、海鲜或其他特殊蛋白制品、某些蚊虫叮咬所传播的寄生虫等。治疗上主要是通过去除变应原,补液、注射肾上腺素改善血流动力学状态。③神经源性休克:交感神经系统急性损伤或被药物阻滞可引起受累神经所支配小动脉扩张,血容量增加,出现相对血容量不足和血压下降。这类休克预后好,常可自愈。常见原因有剧痛、脊髓麻醉或损伤等。

\paragraph{心源性休克}

心源性休克常是指患者心脏泵功能受损或心脏血流排出道受阻,心输出量下降,代偿性血管收缩不足,导致有效循环血量不足、低灌注和低血压。心脏因素包括大面积急性心肌梗死、心肌病变、心脏手术、缺血再灌注损伤和严重的心律失常等,以及心外阻力因素心包填塞、张力性气胸等,常是引起心源性休克的原因。

美国外科医师学会把此类以心脏泵功能衰竭为主要特点的休克分为狭义的(心脏性)心源性休克和梗阻性休克,分别指由于心脏本身原因包括心肌或者瓣膜结构异常所致的休克和由于心脏外因素包括大面积肺栓塞、心包填塞及缩窄性心包炎所致的休克。心脏本身原因引起休克是由于心脏本身泵动力不足所致,表现为持续的低血压(收缩压<
90mmHg,或比基础血压下降超过30mmHg)、心脏指数严重下降{[}未治疗情况下<
1.8L/(min•m\textsuperscript{2}
){]}及心室充盈压力增高(左室舒张末压力> 18mmHg或右室舒张末压>
10~15mmHg),外周血管、神经内分泌系统及炎症因子在此类休克的发生及发展过程中起一定作用。而梗阻性休克出现于心包填塞造成的心脏舒张受限或者大面积肺栓塞的情况下,此时患者血液回流受阻,右室后负荷增加,心输出量下降,但心脏本身的收缩和舒张功能正常。梗阻性休克的临床表现与疾病的发病速度有关,心肌梗死后的心肌破裂可导致心包腔在数分钟内积聚150ml左右的血液,这足以压迫心脏而出现休克,需要立即引流和手术;肿瘤或炎症所致的心包腔内积液,液体积聚速度较慢,当液体体积到达1~2L时才可能出现休克。

不同类型休克血流动力学改变基本特点见表\ref{tab19-1}。为便于与美国外科协会的四类休克的分类法比较,以下把心脏外的外源性因素所致的梗阻性休克与心脏本身因素所致的休克分别列出。从其特点可见,两者休克的血流动力学改变的基本特点类似,这也是笔者把两者合并阐述的基本依据。

\subsubsection{发病机制}

机体承受的内在或外在打击足够剧烈时,均可导致休克现象。休克是一个有着复杂病理生理过程的临床综合征。虽然休克的病因各异,类型不一,临床表现也不尽相同,但其本质相同,即休克发生后机体重要器官微循环处于低灌流状态,导致细胞缺血缺氧,细胞代谢异常,继续发展可导致细胞损害、代谢紊乱,组织结构损伤,重要器官功能失常,最终可出现MODS。

在临床方面,及时发现并解除休克成因、纠正低血压状态有助于休克治疗,但这些并不意味着休克引起的内环境紊乱或并发症会随之改善,有时休克时出现的组织器官功能损害反而会继续发展并造成病情反复加重,此即所谓的重症难治性休克状态(irreversible
shock
state),这些特点提示我们在处理休克时要重视其发病机制,对其过程和特点有全面、深入的认识。

\hypertarget{text00055.htmlux5cux23CHP2-1-1-2-1}{}
(一) 休克时微循环变化及机制

1964年
Lillehei提出的休克微循环障碍学说目前已得到大多数学者的认可,许多新研究使微循环学说的内容更加丰富。虽然,休克成因不同,休克不同阶段组织灌流量减少的机制各异,但体内重要器官微循环处于低灌流状态的特点是相近的,下面以典型的失血性休克为例从时相变化和血流变化两方面分析其微循环障碍的特点。

\paragraph{时相变化}

\hypertarget{text00055.htmlux5cux23CHP2-1-1-2-1-1-1}{}
(1) 缺血性缺氧期(休克代偿期):

休克早期,微血管系统持续痉挛,口径明显缩小,毛细血管前阻力显著增加,血管自律运动增强,同时大量真毛细血管网关闭,毛细血管血流限于直捷通路,动静脉吻合支开放,组织灌流减少,出现少灌少流,灌少于流的情况。这一现象在皮肤、肌肉、肾脏等脏器尤为显著,其结果是保证了心、脑等重要器官的供血,对维持有效循环血量、回心血量及血压有一定代偿意义。机体出现微循环血管持续痉挛的始动因素是交感-肾上腺髓质系统兴奋。休克时大量儿茶酚胺释放入血,血中儿茶酚胺含量比正常高几十倍甚至几百倍。儿茶酚胺大量释放,既刺激α受体,造成皮肤、内脏血管明显痉挛,又刺激β受体,引起大量动静脉短路开放,构成了微循环非营养性血液通路,使器官微循环血液灌流锐减。此外,休克时体内产生的其他体液因子,如血管紧张素Ⅱ、加压素、内皮素、心肌抑制因子(MDF)、血栓素(TXA\textsubscript{2}
)和白三烯等物质等也都有收缩血管的作用。

\begin{table}[htbp]
\centering
\caption{不同类型休克血流动力学改变基本特点}
\label{tab19-1}
\includegraphics[width=6.58333in,height=1.16667in]{./images/Image00083.jpg}
\end{table}

N:正常

\hypertarget{text00055.htmlux5cux23CHP2-1-1-2-1-1-2}{}
(2) 淤血性缺氧期(可逆性失代偿期):

随着休克持续,微循环中血管自律运动首先消失。血管床对儿茶酚胺的反应进行性降低,微动脉和毛细血管前括约肌收缩逐渐减退,血液大量涌入真毛细血管网,而毛细血管的流出道的阻力增加,血液淤积在毛细血管中,微循环灌注量进一步下降。此时,内脏微循环出现灌流减少和血液淤滞现象,失代偿期的出现与长时间血管收缩、缺血缺氧及多种体液因子形成有关。首先,随休克病程发展,逐渐出现血管收缩因子和舒张因子间平衡失调,这种平衡失调的发生与休克时持续缺血缺氧使组织氧分压下降、CO\textsubscript{2}
和乳酸堆积、酸中毒有关:①酸中毒导致平滑肌对儿茶酚胺的反应性降低;②皮肤和腹腔内脏长期缺血缺氧在局部产生各种扩血管因子,如三磷酸腺苷(ATP)的大量分解,其产物腺苷在局部积聚;③细胞分解代谢增强使K\textsuperscript{+}
释放增多,导致Ca\textsuperscript{2+}
内流减少;④肥大细胞释放组胺;⑤激肽-缓激肽系统激活产生激肽酶;⑥内皮细胞产生和释放NO、PGI\textsubscript{2}
;⑦应激激素如β-内啡肽大量释放等。这种血管收缩因子和舒张因子间的平衡失调是造成血管容量显著增大、微循环障碍加剧的主要原因。其次,休克期血流变慢,白细胞贴壁、滚动并黏附于内皮细胞上,加大了毛细血管后阻力。同时,血液浓缩,血浆黏度增大,血细胞比容增大,红细胞聚集。这些血液流变学改变是造成微循环血流变慢,血液泥化、淤滞,甚至血流停止的重要原因。

还应当重视的是细菌和内毒素的肠源性转位(translocation)和吸收在休克发展过程中的作用,这点与MODS发生的“肠源”学说类似:随着休克病程的发展,常出现肠源性细菌转位和脂多糖(LPS)入血现象,从而通过激活激肽系统和补体系统、激活免疫细胞、损伤内皮细胞、影响心功能等多种途径,引起血管扩张、血流动力学性质的改变,并引起持续性低血压。

另外,休克时缺血、酸中毒和炎症反应紊乱均可刺激和损伤血管内皮细胞,引起血管舒缩活性失调和微循环的内皮细胞发生形态改变,表达各种黏附分子,促进与白细胞间的黏附,影响血液回流。此时,机体处于失代偿阶段,微循环血管床大量开放,有效循环血量锐减,回心血量减少,心输出量和血压进行性下降,进一步兴奋交感-肾上腺髓质,使组织血液灌流进行性下降,组织缺氧日趋严重,形成恶性循环。

\hypertarget{text00055.htmlux5cux23CHP2-1-1-2-1-1-3}{}
(3) 难治性休克期(不可逆期):

休克在失代偿期未能被逆转,病情继续发展,持续较长时间以后,就进入难治期,表现为微循环的“无复流”现象和脏器功能严重损害,而微血管麻痹和弥漫性血管内凝血(DIC)是造成微循环“无复流”现象主要原因。此时,微血管发生麻痹性扩张,反应性显著下降,去甲肾上腺素浓度越来越高,而收缩反应性却越来越不明显,发生微循环衰竭;同时,由于毛细血管内血细胞黏着和微血管嵌塞,加之各种组织因子释放,启动凝血系统,导致血管内皮细胞损伤和微血栓堵塞管腔,诱发DIC出现。在重要脏器的功能方面,休克时机体出现持续性重度低血压,血流动力学恶化,细胞损伤越来越严重,同时多种体液因子如溶酶体酶、氧自由基及各种细胞因子过度释放也加重器官损伤,结果使得包括肾、肝、肺、心、脑等器官在内的重要脏器的代谢和功能损害不断加重,甚至衰竭。

目前认为,在休克难治期,肠道严重缺血缺氧,屏障和免疫功能降低,内毒素吸收增加及肠道致病菌移位入血,激活炎性细胞(单核巨噬细胞和中性粒细胞等),造成机体全身炎症反应综合征(systemic
inflammatory response
syndrome,SIRS),SIRS与机体发生的高消耗状态“恶性炎症(malignant
inflammation)”和多器官功能损害(MODS)有着密切的关系。休克发生时,一方面炎性细胞被激活,大量炎性介质包括肿瘤坏死因子(TNF)、白细胞介素(IL-1、IL-6、IL-8)等释放入血,引起炎症反应失控,即SIRS;另一方面,包括IL-4、IL-10、IL-13在内的抗炎介质过度表达,引起代偿性抗炎反应综合征(compensatory
antiinflammatory response
syndrome,CARS)。当循环中出现大量失控的炎性因子时,各种因子间存在广泛的“交叉对话(cross
talk)”,亦即炎症因子之间构成了一个具有交叉作用、相互影响的复杂网络体系。当SIRS和CARS共存、其作用互相加强时,会导致更严重的炎症紊乱,此即所谓的“混合性拮抗反应综合征(mixed
antagonists response
syndrome,MARS)”。无论SIRS、CARS或MARS,均是休克不可逆期器官功能损害发生发展的基础。

\paragraph{血液细胞流变学变化}

细胞流变学方面的研究发现,休克时白细胞附壁黏着、红细胞和血小板聚集以及微血栓形成是导致微循环阻力增加的重要原因。

休克发生时,微循环中发生白细胞扣押和嵌塞毛细血管(leukocyte
sequestration and plugging in
capillary)现象。随着休克发展:①白细胞变形能力下降,硬度增加,体积变大变圆;②内皮细胞受损,可发生肿胀,造成毛细血管管腔狭窄;③血压下降使驱动白细胞流动的灌流压又逐渐降低;④白细胞的附壁黏着(adhesion),这种黏附作用主要是通过经典的黏附蛋白(cell
adhesion
molecules,CAMs)途径实现的:在多种体液介质的作用下,血管内皮表面CAMs表达增多,使得白细胞和内皮细胞之间的黏着力增加,由选择素(selectin)-碳水化合物介导的白细胞黏附参与了早期接触和滚动的发生,由整合素(integrin)-多肽介导的黏附作用参与了白细胞的黏着和游出的发生。白细胞扣押和毛细血管嵌塞现象使得微循环障碍逐步加重。

再者,白细胞在附壁黏着同时释放出的大量毒性介质对于细胞流变学改变和休克的发展也有着相当重要的作用。缺血缺氧导致的白细胞黏附不仅加重微循环障碍,还通过释放多种炎性物质直接导致细胞损害:①白细胞在激活过程中出现呼吸暴发(respiratory
burst),产生大量自由基,使细胞膜的流动性下降和通透性增加;②蛋白交联变化又影响酶活性,从而带来一系列细胞代谢功能的损害;③蛋白酶的释放促进细胞自溶和器官衰竭;④休克时胞浆内Ca\textsuperscript{2+}
增加。炎性物质损害细胞的这种作用,可加重休克时的循环紊乱并影响休克的预后。

除了上述白细胞的特点外,休克过程中红细胞的变形能力也明显下降,随着红细胞变形能力的降低,血液黏度增加,血流阻力增加,引起血液淤积。同时,休克的原发和继发因素可造成血管内皮的损伤,血流减慢,血小板聚集激活剂增多,血小板伪足样突起和聚集型血小板数目增多,结果导致血小板聚集和微血栓形成。红细胞变形能力下降引起血液黏度增加,血小板的聚集引起微血栓的形成都加重了循环障碍,而组织灌流绝对和相对不足影响休克的发展和预后。

\hypertarget{text00055.htmlux5cux23CHP2-1-1-2-2}{}
(二) 休克时迷走神经活动亢进

近年来研究表明
,休克时迷走神经亢进,乙酰胆碱(acetylcholine,Ach)从突触内大量释放,而红细胞乙酰胆碱酯酶(acetyl
cholinesterase,AchE)活性降低,结果Ach大量积聚于突触间隙并持续作用于效应器官的M受体和N受体,使得休克加重、难以恢复,因为Ach一方面对心血管系统有抑制作用,可直接收缩内脏、皮肤、肾脏和肺循环的静脉,另一方面却对骨骼肌的动、静脉均有扩张作用。相关动物实验也证实了这一观点,休克时使用东莨菪碱或维拉帕米(异搏定)可阻止Ach囊泡释放并明显提高AchE活力,由此有效地阻断了休克的迷走效应。

\hypertarget{text00055.htmlux5cux23CHP2-1-1-2-3}{}
(三) 体液因子在休克中的作用

各种有害因素侵袭机体时
,立即引起神经体液反应,产生多种体液因子,介导各种休克病因对机体的作用。体液因子的释放可激起级联反应(cascade
reaction)或称瀑布样反应(downpool
reaction),这种强烈的多系统参与的机体反应并不受休克最初原发病因的影响,反应失控导致内环境紊乱进一步加重。以感染性休克发病过程为例:当局部感染灶细菌入血后,细菌本身或其内毒素、外毒素等成分刺激细胞产生各种体液因子,包括细胞因子(如TNF、IL-1等)、激素(如儿茶酚胺、加压素、血管紧张素等)、黏附分子(如selectin、integrin、细胞间黏附分子ICAMs等)、脂质因子(如血小板活化因子PAF、前列腺素系统PGs、血栓素TXA\textsubscript{2}
、白三烯等)以及内源性阿片肽、心肌抑制因子(MDF)、一氧化氮(NO)等,这些因子可使血管张力失常,内皮损伤,血流动力学发生改变,导致心肌抑制、心室扩张,从而影响体、肺循环以及心脏功能,导致心血管功能障碍,引发感染性休克。

\hypertarget{text00055.htmlux5cux23CHP2-1-1-2-4}{}
(四) 休克时细胞代谢障碍和细胞损伤及机制

随着认识的深化
,人们对休克关注的目光也逐步从微循环学说向细胞代谢障碍及分子水平的异常等方向转移,休克发生发展过程中的细胞机制渐受重视。休克时细胞损伤可以继发于微循环障碍,但也可以原发于休克原始动因直接损伤,因此,有学者提出了休克细胞(shock
cell)的概念并认为细胞损伤是器官功能障碍的基础。

\paragraph{休克时细胞代谢障碍}

①糖酵解和酸中毒:休克时微循环严重障碍造成组织低灌注和细胞缺氧,糖的有氧氧化受阻、无氧酵解增强,结果ATP生成明显减少而乳酸生成显著增多,所有这些因素都导致了细胞功能障碍。首先,细胞能量不足导致胞膜上的钠泵失灵,钠、水内流而胞内钾外流,导致细胞水肿和高钾血症;再者,糖酵解增加引起的高乳酸血症是造成局部酸中毒的原因,而灌流障碍和二氧化碳不能及时清除也加重了局部酸中毒。②细胞内Ca\textsuperscript{2+}
超载:休克时的应急和应激反应导致儿茶酚胺的大量释放,激活胞膜上的Ca\textsuperscript{2+}
通道,使Ca\textsuperscript{2+}
内流增加;同时,由于组织细胞缺氧缺血,胞膜通透性增加使Ca\textsuperscript{2+}
内流增加,Ca\textsuperscript{2+} -ATP酶减少导致Ca\textsuperscript{2+}
清除障碍,而伴随线粒体ATP的释放和利用,Ca\textsuperscript{2+}
又大量溢出到胞质。上述因素导致胞内Ca\textsuperscript{2+}
超载,同时也使神经突触中的Ca\textsuperscript{2+}
增加并进一步促进递质的释放。这样,一方面交感递质(儿茶酚胺)和迷走递质(Ach)均可使血管异常收缩,加重微循环障碍;另一方面,胞浆内Ca\textsuperscript{2+}
增加激活磷脂酶A\textsubscript{2}
,使细胞磷脂膜分解,释放出花生四烯酸,花生四烯酸通过脂氧化酶生成白三烯类物质,白三烯类物质促进循环紊乱和器官衰竭的发生。

\paragraph{休克时细胞的损伤}

①细胞膜的变化:细胞膜是休克时细胞最早发生损伤的部位,造成膜损伤的因素包括缺氧、ATP减少、高钾、酸中毒、溶酶体酶、自由基的释放引起膜脂质过氧化以及其他炎性介质和细胞因子等。休克时,细胞膜离子泵功能的障碍使得细胞丧失了调节自身容量的能力,而膜磷脂微环境的变化则降低了胞膜的流动性;此外,膜上的蛋白变性、交联以及受体蛋白磷酸化过程紊乱损伤了膜相应受体的功能,造成代谢障碍和功能障碍。②线粒体的变化:休克时,线粒体首先出现功能损害,继之发生形态改变。线粒体功能变化涉及电子传递链功能损害,氧化磷酸化障碍,ATP酶活性下降,钙转运功能降低等各方面;而形态变化则表现为线粒体肿胀,致密结构和嵴消失,钙盐沉积,甚至线粒体崩解。线粒体的破坏预示细胞体的死亡。③溶酶体的变化:休克时,溶酶体膜通透性增加使其中的水解酶释出,不仅可引发线粒体功能障碍和细胞自溶,还可因水解酶入血使循环紊乱加重,促进MDF的形成。“休克发生的溶酶体学说”认为,休克时溶酶体的改变及水解酶的释放对休克的发生和发展有重要影响,因其加重了休克时的循环的障碍,造成细胞和器官功能紊乱。④细胞凋亡:休克时,活化的炎性细胞可产生包括细胞因子和自由基在内的多种炎症介质攻击网状-内皮细胞系统和各脏器实质细胞,细胞的炎性损伤可导致细胞变性坏死(necrosis)或凋亡(apoptosis)。休克时的细胞凋亡是细胞损伤的一种表现,也是重要脏器功能衰竭的基础。实验表明,用非致死量的细胞因子和氧自由基攻击可导致细胞凋亡,而致死量会导致细胞坏死。

\hypertarget{text00055.htmlux5cux23CHP2-1-1-2-5}{}
(五) DO\textsubscript{2} -VO\textsubscript{2} 病理性依赖

近年来临床发现,在机体处于ARDS、严重创伤、严重感染、脓毒症、休克和MODS等病理状态下,存在一种被称作“病理性氧供依赖标志”(pathological
supply dependence)的现象,其突出特征是氧输送(oxygen
delivery,DO\textsubscript{2}
)的阈值似乎非常高,有的可以测出(Ⅰ型病理性依赖),有的根本测不出(Ⅱ型病理性依赖)。病理性依赖均伴有氧提取率的严重损害(斜线斜率变化)。休克在其后期特别是伴发MODS时,全身血液重新分布使局部氧消耗(oxygen
consumption,VO\textsubscript{2} )和DO\textsubscript{2}
之间失去平衡,而微循环障碍包括微血栓形成、血管内皮细胞损伤、微循环动静脉短路开放等以及红细胞变形性下降、细胞线粒体损伤又导致组织氧摄取障碍,表现出VO\textsubscript{2}
对DO\textsubscript{2}
的依赖现象,这种病理性氧供依赖标志着全身组织氧合不足,是细胞缺氧的表现,常伴有动脉血乳酸水平的升高。

\hypertarget{text00055.htmlux5cux23CHP2-1-1-2-6}{}
(六) 缺血再灌注损伤(I/R)

休克时,组织器官灌注不良和细胞的缺氧导致细胞能量储备极度下降以及酶活性、胞膜通透性、渗透浓度和pH的异常改变,当缺血组织再灌注(ischemia-reperfusion,I/R)时,细胞不能耐受原本“正常的”再灌注,出现细胞的“过激”反应,导致细胞损伤。I/R表现为再灌注一开始,Ca\textsuperscript{2+}
即大量快速内流并在胞内积蓄(“钙反常”现象),缺血组织重新获得氧供后反而发生细胞损伤(“氧反常”现象),甚至出现氧自由基产生的“呼吸暴发(respiratory
burst)”现象,结果使再灌注的组织细胞急剧肿胀、超微结构改变,对氧、基质利用下降,ATP、糖原减少,最后可导致细胞死亡。I/R的病理机制尚未完全明了,有人通过对心肌I/R模型研究发现Ach、Ca\textsuperscript{2+}
和氧自由基之间存在着一定内在联系,认为Ach的释放可能是心肌I/R的始动因素。

\subsection{诊断}

\subsubsection{临床表现特点}

休克患者临床表现取决于休克的病因、组织灌流损害程度及代偿反应,既可以表现为轻微意识障碍、心动过速,也可表现为显著血压下降,少尿甚至多器官功能损害。

\paragraph{休克早期}

在原发症状体征为主的情况下出现轻度兴奋征象,如意识尚清,但烦躁焦虑,精神紧张,面色、皮肤苍白,口唇甲床轻度发绀,心率加快,呼吸频率增加,出冷汗,脉搏细速,血压可骤降(如大失血),也可略降,甚至正常或稍高(代偿性),脉压缩小,尿量减少。部分患者表现肢暖、出汗等暖休克特点。眼底可见动脉痉挛。

\paragraph{休克中期}

患者烦躁,意识不清,呼吸表浅,四肢温度下降,心音低钝,脉细数而弱,血压进行性降低,可低于50mmHg或测不到,脉压小于20mmHg,皮肤湿冷并出现花斑,尿少或无尿。若原来伴高热的患者体温骤降,大汗,血压骤降,意识由清醒转为模糊,亦提示休克直接进入中期。

\paragraph{休克晚期}

表现为DIC和多器官功能衰竭。①DIC表现:顽固性低血压,皮肤发绀或广泛出血,甲床微循环淤血,血管活性药物疗效不显,常与器官衰竭并存。②急性呼吸功能衰竭表现:吸氧难以纠正的进行性呼吸困难,进行性低氧血症,呼吸促,发绀,肺水肿和肺顺应性降低等表现。③急性心功能衰竭表现:呼吸急促,发绀,心率加快,心音低钝,可有奔马律、心律不齐。如出现心率缓慢,面色灰暗,肢端发凉,亦属心功能衰竭征象,中心静脉压升高,肺动脉楔压升高,严重者可有肺水肿表现。④急性肾功能衰竭表现:少尿或无尿、氮质血症、高血钾等水电解质和酸碱平衡紊乱。⑤其他表现:意识障碍程度常反映脑供血情况,如脑水肿时呕吐、颈项强直、瞳孔及眼底改变。肝衰竭者可出现黄疸,血胆红素增加,由于肝脏具有强大的代偿功能,肝性脑病发生率并不高。胃肠道功能紊乱常表现为腹痛,消化不良,呕血和黑便等。

\subsubsection{实验室辅助检查}

\paragraph{化验}

休克的实验室检查应当尽快进行,为全面了解内环境紊乱状况和各器官功能并帮助判断休克原因和休克程度,还应当注意检查内容的广泛性。一般应注意的项目包括:①血常规;②血生化(包括电解质、肝功能等)检查和血气分析;③肾功能检查以及尿常规包括尿比重的测定;④出、凝血指标检查包括与DIC有关的项目的检查;⑤包括CK-MB在内的血清酶学和肌钙蛋白(cTnT或cTnI)、肌红蛋白(Myo)、D-二聚体(D-dimer)等心肌损伤相关指标的检查;⑥各种体液、排泌物等的培养、病原体检查和药敏测定等等。

\paragraph{感染和炎症因子的血清学检查}

通过血清免疫学检测手段,检查血中降钙素原(PCT)、C-反应蛋白(CRP)、念珠菌或曲霉菌特殊抗原标志物或抗体以及LPS、TNF、PAF、IL-1、IL-6等因子,有助于快速判断休克是否存在感染因素、可能的感染类型以及体内炎症反应紊乱状况。

\subsubsection{临床观察重点}

迄今为止,作为休克的临床监测与复苏的评估指标,皮温与色泽、心率、血压、尿量和精神状态等依然是最常用的临床指标。然而、必须认识到,这些指标在休克各阶段评估作用的局限性。为避免休克发展成为难治性休克或出现MODS等并发症,早诊早治显得尤为重要,但早诊有时会存在困难,应从以下几个方面进行细致的观察:

\paragraph{脉搏和血压}

由于机体的代偿反应,休克早期脉搏变化先于血压波动,因此注意脉搏变化更有益于休克的早期判断,但脉搏改变不足以反映休克的严重程度。一般来讲,早期脉搏加速,脉搏增速>
20次/分提示血容量低,休克时脉搏一般>
120次/分。若休克继续发展,脉搏可变快变弱直至触摸不清,此即中医所谓气绝的脉搏“细数”和“细弱”。此外,当患者脉搏快且原因不明时,可通过短时间快速补液的负荷试验(loading
test)判断是否由容量不足所致。若将负荷试验时的脉搏或心率改变与中心静脉压(central
venous pressure,CVP)和肺动脉楔压(pulmonary artery wedge
pressure,PAWP)变化结合起来考虑,则对容量不足的判断意义更大。

休克初期血压方面的改变可仅表现为收缩压微降、舒张压略升,脉压减小。当收缩压<
80mmHg、脉压< 30mmHg或高血压患者血压下降> 20\%或自基础水平下降>
40mmHg,即可诊断休克。另外,也可通过双腿抬高试验了解休克时的微循环状况:患者平卧并快速抬高双下肢呈90°角,若30秒内血压上升100mmHg则为阳性结果,表明微循环淤血。

\paragraph{皮肤和周围灌注}

皮肤、黏膜温暖且色泽红润表明毛细血管舒缩功能正常、周围阻力无大变化;感染性休克早期和神经源性休克小动脉阻力下降,可见皮肤比正常温暖且红润;而毛细血管痉挛伴小动脉阻力增高时皮肤变湿冷、苍白,甚至出现发绀和花斑。利用皮肤毛细血管充盈试验可帮助了解休克发展情况:正常情况下指压额前部、耳缘或胸骨柄部皮肤2~3秒,放手后皮肤由苍白回复恢复红润时间<
5秒;休克时若指压皮肤变白不明显则提示皮层下小血管收缩,若苍白恢复时间延长表明休克发展,若静脉充血,指压处苍白明显、周围发绀且历时数分钟不褪,则说明休克恶化。

除对皮肤黏膜的直接观察外,还可通过低倍镜下观察甲皱下毛细血管袢数、管径长度、血色、流速、红细胞聚集程度判断休克时的微循环状况。

\paragraph{意识状态和眼底检查}

休克时意识由烦躁转为抑郁、淡漠甚至昏迷,表明患者脑组织血流灌注不足,脑功能受损。眼底检查可以从一个方面反映不同休克状态时的脑组织灌注情况:眼底动静脉比例正常时为2∶3,灌注不良时变为1∶3~1∶4;例如在休克初期可见眼底血管痉挛,后期静脉则扩张,休克严重时可见视网膜水肿和视乳头水肿。

\paragraph{尿量}

尿量是反映生命重要器官血流灌注状态最敏感指标之一。休克时肾血流量改变最为显著,尿量也随之改变。休克早期尿量多在20~30ml/h,随着肾血流量进一步下降,尿量可少于400ml/d,肾损害加剧可致尿闭。由于临床上出现的非少尿型急性肾衰有增多趋势,因此少尿并不是肾衰的关键表现。

\paragraph{中心静脉压(CVP)和肺毛细血管楔压(PCWP)}

CVP其正常值范围一般被认为是4~10mmHg。CVP可用于指导扩容治疗,其反映血容量、回心血量及右心功能,但不反映左心功能。CVP升高(CVP
> 12mmHg)常提示右心功能不全或输液超负荷、肺循环阻力增加,降低(CVP <
4mmHg)常表示心脏充盈欠佳或血容量不足,即使动脉压正常,仍需输入液体。PCWP多以PAWP代替,其正常值范围6~12mmHg,可间接反映左室功能状态及其前负荷。由于左心房和肺静脉之间不存在瓣膜,左心房压可逆向经肺静脉传至肺毛细血管,如无肺血管病变,PAWP可反映左房压。如无二尖瓣病变,PAWP可以间接反映左心室舒张末期压力(LVEDP)。近年来,对这两种监测指标的应用价值有了重新的评估,请参阅后面的“静态血流动力学监测”部分。

\paragraph{内环境和氧合指标}

心排出量和氧输运、氧消耗指标反映了休克状况下机体输送氧和利用氧的能力,将这些指标与pHi监测、动脉血气值等结合起来分析对于全面了解休克时脏器供血和功能情况并判断病情发展趋势很有帮助。例如从组织氧耗Vo\textsubscript{2}
= 1.38 × CO × Hb ×(SaO\textsubscript{2} − ScvO\textsubscript{2} )×
10这样一个公式我们可以看到,要提高组织对于氧的消耗(VO\textsubscript{2}
),就必须纠正贫血(输血),提高心输出量(强心),提高血氧饱和度SaO\textsubscript{2}
(高压氧或辅助通气支持)并降低混和静脉血氧含量ScvO\textsubscript{2}
(恰当使用血管活性药、改善组织器官灌注),其中任何环节处理不好,都会组织用氧并损及器官功能。当然,机体内环境紊乱也可从另一方面严重影响细胞的代谢功能。通过血气、电解质、血乳酸和血糖测定,我们可以及时发现患者的内在问题并及时纠正,从而提高外源性支持治疗的效率。

\subsubsection{主要监测手段及进展}

早期发现在休克患者的治疗当中尤为重要,由于机体代偿机制的存在,休克早期,患者血压和心率无明显改变,但组织缺氧已经存在。能够早期发现休克的存在并早期治疗,就能逆转休克的进一步恶化,提高生存率。

休克的微循环和血流动力学监测对于了解其组织器官灌注现状及液体复苏效果有重要参考价值。但是,任何一种监测技术都不是完美的,任何一种监测方法都不是绝对的。各种血流动力学指标经常受到许多因素的影响,因此单一指标的数值并不能正确反映血流动力学状态,应该结合患者症状、体征综合判断,监测分析参数的动态变化,并采用多项指标综合评估某一种功能状态。

\hypertarget{text00055.htmlux5cux23CHP2-1-2-4-1}{}
(一) 血流动力学监测

血流动力学监测帮助了解患者循环功能状态
,临床上用于休克高危患者早期鉴别、预防发生并优化治疗,对已有休克帮助区分类型,指导制订治疗方案并反馈其实施效果。血流动力学常可分为静态血流动力学和动态血流动力学。前者包括压力指标,如中心静脉压(CVP)、肺动脉楔压(PAWP)和容量指标,如心室舒张末容积(VEDV)、胸内血容量(ITBV)。后者则主要是对一些指标的变化率的监测。

\paragraph{静态血流动力学监测}

长久以来,CVP和PAWP的临床价值就饱受争议,这些指标的测定虽然比较好地反映了心脏的前负荷的水平,但由于受到心室顺应性的影响,其所反映的前负荷水平可能并不很准确。除去医务人员的技术原因,CVP和PAWP的测定还会受到后负荷、机械通气、心率等的因素影响,在正常志愿者中测得的这些指标与心室舒张末容积的相关性也很差。

虽然CVP可以用于帮助评估液体治疗以及血管活性药物治疗的效果,但是,不应仅以CVP的单次测定值来决定体内的容量状态,更不应强求以输液来维持所谓CVP的值正常。在判断循环血容量和心血管功能间的关系时,若结合每搏量指数(SVI)评估结果更为可靠,如果SVI低,CVP小于4mmHg,可能反映低血容量;SVI低,CVP大于12mmHg,可能反映右心衰竭。在评价心脏对容量反应方面,CVP的动态变化更有意义,但对于正压通气的患者,CVP的动态变化有时亦不能准确预测心脏对容量的反应,此时应用每搏量变异率(SVV)与脉压变异率(PPV)则可能具有更好的评价作用。

PAWP可以估计肺循环状态和左心室功能,鉴别心源性或肺源性肺水肿,判定血管活性药物的治疗效果,诊断低血容量以及判断液体治疗效果等。如果SVI降低,PAWP
<6mmHg,可能存在低血容量;如果SVI低,PAWP >
18mmHg,反映左心功能衰竭,PAWP大于25mmHg反映存在急性肺水肿。同样,PAWP在反映LVEDP时,如存在主动脉反流、肺切除或肺栓塞时,肺分支血管血流明显减少,左室顺应性降低,PAWP低于LVEDP;相反如存在气道压增加、肺静脉异常、心率>
130次/分、二尖瓣狭窄等病变时,PAWP高于LVEDP。研究结果表明,若正压通气且PEEP
< 10mmHg时不会影响PAWP,但PEEP >
10mmHg会使PAWP明显升高。动物实验表明腹腔压升高可提高CVP、PAWP水平,腹内压达到20mmHg时尤为显著。

因此,CVP和PAWP的单个测量值价值不大,但在参考基线水平的基础上观察其动态变化则有一定意义。由于数据获得的复杂性,其应用受到了很大限制。

\paragraph{功能性血流动力学监测}

由于静态指标的局限性,因而功能性血流动力学的应用也受到了更广泛地关注。近些年国外有学者提出了功能性血流动力学监测(functional
hemodynamic
monitoring,FHM)的概念。FHM是全新的血流动力学监测方式,它是以心肺交互作用为基本原理,将循环系统受呼吸运动影响的程度作为衡量指标,以此预测循环系统对液体负荷的反应结果,进而对循环容量状态进行判断的血流动力学监测方式。FHM的指标是功能性的、动态的参数,不同于目前临床常用的静态指标。FHM的手段相对微创、并发症少、安全性高。

功能性血流动力学参数(functional hemodynamic
parameter,FHP)是某一时间段内容量、压力、血流速或腔静脉直径的变化率,代表了一种变化程度,故均以百分数的形式表示。多种传统和新近的监测手段包括经胸超声心动图(TTE)、经食管超声心动图(TEE)、脉搏波形分析心排出量监测(PiCCO),甚至脉搏血氧饱和度的波形均可以得到FHP。下腔或上腔静脉直径呼吸变异率(分别简称下腔或上腔变异率)、主动脉峰值血流速变异率(△Peak)、收缩压变异率(SPV)、每搏量变异率(SVV)、脉压变异率(PPV)等都是通过以上手段获得的FHP。

FHP的共同的特点在于其均以心肺交互作用为基本原理,综合考虑了循环系统本身和呼吸运动对血流动力学的影响作用,因而对患者循环状态的评价更全面、更准确。与某一时间点测得的静态参数不同,FHP是动态的指标,FHP反映的是某一时间段内容量、压力或血流速度等静态参数的变化率,所以,可以说FHP是预测循环系统对液体治疗反应性的参数,体现了心脏对液体治疗的敏感性,直接反映循环前负荷状态。绝大部分FHP只可应用于控制性机械通气的患者。

\hypertarget{text00055.htmlux5cux23CHP2-1-2-4-2}{}
(二) 组织灌注与微循环监测

\paragraph{SvO\textsubscript{2} 和ScvO\textsubscript{2} 监测}

混合静脉血氧饱和度(saturation of mixed venous blood
oxygen,SvO\textsubscript{2}
)是感染性休克复苏的重要监测指标之一,反映组织器官摄取氧的状态。当全身氧输送降低或全身氧需求超过氧输送时,SvO\textsubscript{2}
降低,提示机体无氧代谢增加。当组织器官氧利用障碍或微血管分流增加时,可导致SvO\textsubscript{2}
升高,尽管此时组织的氧需求量仍可能增加。在严重感染和感染性休克早期,全身组织的灌注已经发生改变,即使常规血流动力学指标仍处于正常范围,此时可能已经出现SvO\textsubscript{2}
降低,提示SvO\textsubscript{2}
能较早地发现病情的变化。中心静脉血氧饱和度(saturation of central venous
blood oxygen,ScvO\textsubscript{2} )与SvO\textsubscript{2}
有一定的相关性,在临床上更具可操作性,虽然测量的ScvO\textsubscript{2}
值要比SvO\textsubscript{2}
值高5\%~7\%,但它们所代表的趋势是相同的,可以反映组织灌注状态。一般情况下,SvO\textsubscript{2}
的范围约60\%~80\%,SvO\textsubscript{2} <
60\%提示氧供不足,但SvO\textsubscript{2} >
70\%并不代表微循环灌注充足。研究显示,心脏停搏患者动脉血氧分压增高及SvO\textsubscript{2}
> 80\%提示组织氧利用障碍。

\paragraph{血乳酸监测}

严重感染与感染性休克时组织缺氧,乳酸生成增加。在常规血流动力学指标改变之前,已经存在组织低灌注、缺氧以及乳酸水平升高。有研究表明,乳酸持续升高与APACHE
Ⅱ评分密切相关,当感染性休克的血乳酸>
4mmol/L时,患者的病死率达80\%,因此乳酸可作为评价疾病严重程度及预后的指标之一。但仅以血乳酸浓度尚不能充分反映组织氧合状态,因为血乳酸浓度并不是组织缺氧的特异性指标。有研究报告,在感染性休克患者早期目标指导治疗中,以血乳酸清除率≥10\%与以SvO\textsubscript{2}
>
70\%为目标的复苏治疗短期生存率无显著差异。因此,动态监测乳酸浓度变化或计算乳酸清除率可能是更好的监测目标。

\paragraph{胃黏膜内}

pH和黏膜PCO\textsubscript{2} 、黏膜-动脉PCO\textsubscript{2} 差值监测
胃黏膜内pH(pHi)、黏膜PCO\textsubscript{2}
以及黏膜-动脉PCO\textsubscript{2} 差值(mucosal-arterial
PCO\textsubscript{2} gap,Pr-aCO\textsubscript{2}
)均反映局部黏膜组织的灌注状态。休克发生时,胃肠道血流灌注降低,导致黏膜细胞缺血缺氧,H\textsuperscript{+}
释放增加与CO\textsubscript{2}
积聚。研究表明,若连续24小时监测严重创伤患者的pHi,可以发现pHi≥7.30组的存活率明显高于pHi
< 7.30组;当pHi <
7.30的状态持续24小时,病死率可高达50\%。Poeze的研究证实,感染性休克死亡组黏膜PCO\textsubscript{2}
及Pr-aCO\textsubscript{2}
明显高于存活组,说明局部氧代谢状态与预后密切相关。但最近的一项大样本前瞻性研究却发现,即便维持了胃黏膜pHi≥7.30,也未能显著降低感染性休克的病死率。

\paragraph{偏正光谱成像和侧流暗视野视频显微镜技术}

偏正光谱成像(orthogonal polarization
spectral,OPS)和侧流暗视野视频显微镜技术(sidestream dark
field,SDF)是近年来发展的新技术,采用床边直视设备观察感染性休克患者微循环变化,包括血管密度下降和未充盈、间断充盈毛细血管比例升高等指标,可以更为直观、量化地为临床复苏提供可靠依据。

\paragraph{组织氧饱和度}

组织氧饱和度(tissue oxygen saturation,StO\textsubscript{2}
)是一种利用红外线光谱持续、无创监测肌肉组织氧代谢状况的技术手段。创伤性休克患者的StO\textsubscript{2}
评估有助于了解休克造成的脏器功能损害;在感染性休克的研究中也发现StO\textsubscript{2}
与血乳酸相关度良好,复苏后存活组患者的StO\textsubscript{2}
明显高于死亡组,StO\textsubscript{2}
≤78\%者28天死亡率明显增高。目前,StO\textsubscript{2}
虽然可以直接量化组织氧代谢,但尚缺乏大宗临床研究资料的循证医学证据支持,将来可以作为常规监测项目。

\subsection{治疗}

根据休克的发病机制和病理生理,治疗应在去除病因前提下采取综合性措施,以支持生命器官的微循环灌注和改善细胞代谢为目的。

\subsubsection{病因学的治疗}

一旦休克出现,应首先采取止血、抗感染、输液、镇痛等措施,去除休克发展的原始动因,同时积极处理引起休克的原发病。对于严重威胁生命又必须外科处理的原发疾患如体腔内脏器大出血、肠坏死、消化道穿孔或腹腔脓肿等,不应仅仅为了等待休克“纠正”而贻误手术机会,应在积极抗休克同时,积极进行术前准备,包括插管、呼吸支持、配血、备皮等,争分夺秒挽救生命。当然,患者家属对于手术、麻醉风险和其他可能危险性的理解也应该是所有急救医生应当重视的事情。

\subsubsection{综合治疗}

\paragraph{一般处理}

患者应平卧(下肢可抬高15°~20°)、吸氧、保温、必要时适度镇静。

\paragraph{液体复苏}

各种休克都存在有效循环血量的绝对或相对不足,除心源性休克外,进行液体复苏是纠正有效循环血量下降、改善器官微循环灌注的首要措施。

\hypertarget{text00055.htmlux5cux23CHP2-1-3-2-2-1}{}
(1) 复苏的目标:

对于休克患者,保持循环稳定的最好治疗是早期复苏,但是,临床研究已反复证实只有50\%的休克患者真正对液体复苏有反应(心输出量CO上升≥15\%);液体复苏的初期目标是保证足够的组织灌注,补液的目的是增加每搏量和心输出量,但前提是心脏处于Frank-Starling曲线的上升部分,心脏仍有代偿能力。一旦心脏代偿能力耗尽,处于曲线的平台部分,增加前负荷就不能相应地增加心输出量,反而会导致组织水肿的缺氧,因此,预测患者是否对液体复苏有反应就显得尤为重要。

目前,常用于指导复苏的血流动力学指标并不能敏感、及时地反映局部组织的灌注情况。前面已经提到,大量研究证实CVP、PAWP并不能预测休克患者对于液体复苏的反应,或者说,若仅根据这些监测指标进行的液体复苏不能真正达到改善微循环的目的。对观察复苏是否达到目标有真正指导意义的监测手段应是能够反映微循环的一些指标。Trzeciak采用SDF量化复苏并监测24小时器官功能改变,结果发现以SDF为标准的微循环改善能降低器官功能损伤。也有荟萃分析显示,由动脉压力波形获得的功能性血流动力学指标能够很好的预测哪些机械通气的休克患者应进行液体复苏以及这些患者的SV和CO可能改变的程度。

\hypertarget{text00055.htmlux5cux23CHP2-1-3-2-2-2}{}
(2) 液体种类:

补液种类有晶体和胶体两种。晶体液以平衡液为主,可提高功能性细胞外液容量,并可部分纠正酸中毒。在输液的最初阶段不应大量补充葡萄糖液,因为休克早期儿茶酚胺分泌增加、肝糖原分解产生高血糖,但机体糖利用率低下,输注的葡萄糖不能被有效利用,高血糖会加重应激反应和代谢紊乱,并在血压回升时引起糖尿及渗透性利尿,不利于休克的彻底纠正。再者,由于晶体液维持血容量的时间有限,故必须适当补充胶体溶液。常用的胶体溶液有低分子右旋糖酐、白蛋白、血浆及其代用品,胶体溶液通过提高胶体渗透压达到扩容目的。

关于用晶体液还是胶体液的争论一直没有停歇过。胶体液在血管内驻留时间长,就此而言,胶体液有更好的复苏效果。但是胶体液相对于晶体液价格昂贵,尤其是人血白蛋白。并且人工合成代血浆会不同程度地影响凝血功能。最近一项名为SAFE的研究显示,以胶体液或晶体液复苏的休克患者生存率相当,此研究同时证明,在脓毒症低蛋白血症组患者中,输注白蛋白能够改善生存率。

一般认为,高张盐溶液通过使细胞内水进入循环而扩充容量。近期研究表明高渗盐溶液还具有抗炎作用。荟萃分析显示,休克复苏时,7.5\%氯化钠高张盐溶液或其与6\%~12\%右旋糖酐-70混和成的高张高渗液(hypertonic
saline
dextran,HSD)扩容效率优于平衡盐溶液和生理盐水,但是对死亡率没有影响。HSD的其他作用特点见后续的治疗进展部分。

目前,没有任何一种液体是最理想的,同时也没有证据证明哪种液体比另一种液体更优越。因此,选择所要输注的液体,最好是在综合基础疾病、损失体液成分、休克程度、血浆白蛋白水平及是否出血等因素后作出判断。

\hypertarget{text00055.htmlux5cux23CHP2-1-3-2-2-3}{}
(3) 液体量和速度:

液体复苏时的扩容原则是“按需供给”,需要多少就补充多少,充分扩容。补液总量应视患者具体情况及心肾功能状态而定,可监测CVP或PCWP,两者同时监测对防治肺水肿有重要意义,但并不能真正能指示抗休克治疗是否已经达到了改善微循环的目标。最初1小时补液速度按10~20ml/kg。在最初的补液阶段,因补液量大、速度快,应注意使用强心药以避免心力衰竭。扩容时要注意纠正血液流变学异常,根据血细胞比容的变化决定输血和输液的比例,使血细胞比容控制在35\%~40\%范围。2008年国际脓毒症治疗指南(Surviving
Sepsis
Campaign)建议30分钟内输注晶体液500~1000ml或胶体液300~500ml。但对于失血性休克患者,快速大量补液等可导致并不牢固的血栓松动,有再出血的危险。

临床上对患者进行液体管理有“湿派”和“干派”两方面观点,前者认为患者多补液能降低并发症出现的风险,而“干派”认为补液少能降低风险。但关于ARDS的NETWORK研究的结果表明,对于发生急性肺损伤的患者,宽松的液体管理和严格的液体管理对两组患者死亡率无影响。这样看来,双方的看法都有失偏颇,在休克的治疗中,应该通过血流动力学监测来优化每个患者的液体管理,使患者处于最佳的容量状态。

\paragraph{纠正代谢性酸中毒}

除了引起高血钾外,酸中毒还可通过H\textsuperscript{+}
和Ca\textsuperscript{2+}
的竞争作用直接影响血管活性药物的疗效,也影响心肌收缩力。另外,酸中毒还使肝素灭活加速,肝血管阻力增加,影响内脏血灌注并促进DIC发生。因此,休克时纠正酸中毒十分重要,可根据血气分析及二氧化碳结合力补充碱性液体,这方面常用药物有5\%碳酸氢钠(首选)、乳酸钠(肝功能损害者不宜采用)和THAM液(适用于需限钠患者)。

\paragraph{合理应用血管活性药物}

血管活性药物通过调节血管张力来达到改善循环的目的。应用血管活性药物旨在降低血管阻力,调节血管功能,故扩血管药物较缩血管药物更具优点。但缩血管药在休克的治疗上有其适应证,故针对不同情况合理使用缩血管和扩血管药物,可起到相互配合的作用。低血容量休克的患者一般不常规使用血管活性药物,研究证实这些药物有进一步加重器官灌注不足和缺氧的风险。在积极进行容量复苏情况下,对于存在持续性低血压的低血容量休克患者,可选择使用血管活性药物。但对于感染性休克患者,即便是在进行容量复苏,也可考虑同时应用血管活性药物。

扩血管药物在休克时的应用前提是充分扩容,在低排高阻型休克或缩血管药物致血管严重痉挛休克患者以及体内儿茶酚胺浓度过高的中晚期休患者可使用血管扩张剂,这类药物包括:①抗胆碱能药物,主要有山莨菪碱、阿托品等,可通过阻断M受体和α受体而起血管解痉作用,同时还能够兴奋呼吸中枢,解除支气管痉挛,调节迷走神经,降低心脏前后负荷,改善微循环,抑制血小板和中性粒细胞聚集。山莨菪碱有明显的保护细胞膜的功效且副作用较阿托品轻微,临床首选,每10~30分钟给药1次,剂量为50~100mg,根据末梢微循环改善情况逐渐减量或延长给药时间间隔。②α受体阻滞剂:如酚妥拉明或酚苄明,可解除去甲肾上腺素致微血管痉挛、微循环淤滞,降低血管阻力。低浓度时可增加α肾上腺素能作用,促进脏器血液灌注,有利于保护重要脏器,而且心肌毒性小,不易诱发心律失常。

缩血管药物是治疗过敏性休克和神经源性休克的最佳选择。早期轻型的休克或高排低阻型休克,在综合治疗的基础上,也可采用缩血管药物。血压低至心脑血管临界关闭压(50mmHg)以下,扩容又不能迅速进行时,应使用缩血管药物升压以确保心脑灌注。对于血管活性药物的选择上,首选多巴胺和去甲肾上腺素,但2008年国际脓毒症休克治疗指南并未对此两种药物进行区分。多巴胺是常用的缩血管药物之一,该药在5~10μg/kg•min的静脉用量时可同时兴奋β\textsubscript{1}
和α受体,增加心肌收缩力和心排出量,提升血压,同时收缩外周血管,但增加肾、肠系膜等内脏血管供血。近期的实证医学研究表明,去甲肾上腺素在升压治疗方面出现的副作用特别是在肾损害方面的副作用并不大于多巴胺。多中心随机对照试验也证实,接受多巴胺的休克患者组与接受去甲肾上腺素的休克患者组28天生存率无显著差异,但多巴胺组患者心律失常不良事件增多,并且多巴胺在心源性休克的应用增加了患者的死亡率,所以,应该采取更审慎的态度对待多巴胺的抗休克应用。对于突发的过敏性休克,临床上常用肾上腺素进行紧急治疗,此外,对于多巴胺和去甲肾上腺素升压效果不佳的脓毒症休克,也可首选使用肾上腺素治疗。

\paragraph{肾上腺皮质激素的应用}

糖皮质激素有减轻毒血症和稳定细胞膜和溶酶体膜的作用,大剂量时还能:①增加心搏量,降低外周阻力,扩张微血管,改善组织灌流;②维护血管壁、细胞壁和溶酶体膜的完整性,降低脑血管通透性,抑制炎症渗出反应;③稳定补体系统从而抑制过敏毒素、白细胞趋化聚集、黏附和溶酶体释放;④抑制花生四烯酸代谢,控制脂氧化酶和环氧化酶产物的形成;⑤抑制垂体β-内啡肽的分泌;⑥维持肝线粒体正常氧化磷酸化过程。严重感染和感染性休克患者往往存在肾上腺皮质功能不全,机体对促肾上腺皮质激素释放激素(ACTH)反应改变,并失去对血管活性药物的敏感性,因此需要应用糖皮质激素。虽然大剂量、短疗程糖皮质激素能够阻止感染性休克时炎症反应的瀑布样释放,但不能提高患者的生存率,且副作用明显,已被摒弃。2009年的一项脓毒症激素治疗的荟萃分析结果提示,应用糖皮质激素总体上不能改善28天生存率,但对长期(≥5天)应用小剂量糖皮质激素(氢化可的松≤300mg/d)患者进行亚组分析,却发现其生存率得以改善,同时低量激素也没有增加胃肠道出血及院内双重感染的风险。

目前,对于成人对补液复苏和血管升压药治疗反应欠佳或依赖的感染性休克患者静脉给予氢化可的松,是多数ICU采用的治疗方法之一,但用药的方式、用药的时间和停药方式仍未统一,一般每天给予氢化可的松200~300mg,用药5天以上。也有人建议短期内(3~5天)应用地塞米松10~20mg/d或甲泼尼龙20~80mg/d静滴。

\paragraph{肠道保护}

休克严重时可引起腹胀、肠麻痹、应激性溃疡、肠道菌群紊乱和细菌、内毒素转位,使病情进一步恶化,故应注意休克时的肠道保护问题。应激因素重时应适当使用黏膜保护剂、制酸剂或生长抑素避免消化道应激出血;情况允许时应尽早启动肠内营养;肠道菌群紊乱严重时还可采用“扶正祛邪”措施予以纠正:一方面给予抗LPS血清、抗体或丙种球蛋白,口服肠道不吸收的抗生素进行选择性肠道去污染,另一方面给予益生菌和益生素,尽快恢复肠道正常生态。

\paragraph{其他综合治疗手段}

休克可引起内环境紊乱和多器官功能不全,故治疗中应注意纠正体内水、电解质、代谢紊乱和酸中毒,同时应注意评估其余各脏器的功能,并根据特点进行保护和支持治疗,防止MODS出现。例如,急性心功能不全时,除强心利尿外还应减少补液量,适当降低前、后负荷;出现肾功能损害时,要注意利尿,必要时行血液净化治疗;出现休克肺时,要正压给氧,改善呼吸功能。

\subsubsection{抗休克治疗的进展}

\paragraph{高张高渗液}

对于低血容量休克,近年研究表明小剂量(4ml/kg)应用7.5\%氯化钠高张盐溶液或其与6\%~12\%右旋糖酐-70混和成的高张高渗液(hypertonic
saline
dextran,HSD)有良好复苏效果。其作用机制可能是由于其高渗扩容作用,减轻了细胞水肿,刺激心肌和神经反射机制、改善血液的流态、重建小动脉自主活动和周围动脉扩张等。HSD中高张盐液和胶体液的作用相加,HSD以4ml/kg输注可扩容8~12ml/kg,这对于低血容量休克院前急救很有意义,抢救时可先静滴HSD
250ml,再常规抗休克扩容。有人将这种疗法称之为“小剂量复苏”。

\paragraph{促炎介质拮抗剂的应用}

休克时机体释放多种内源性介质参与机体全身性炎症反应调控,这些介质对休克病程和预后有重要影响。通过不同途径干涉这些介质的水平或作用,促进休克的康复,是许多学者正努力奋斗的目标。

\hypertarget{text00055.htmlux5cux23CHP2-1-3-3-2-1}{}
(1) 炎症因子拮抗剂:

这类物质能通过干涉或阻断炎症信号转导通路的某个因子而达到抗炎效果,例如:①在实验研究中,抗LPS、抗TNF-α和抗IL-1受体的单克隆抗体表现出了对内毒素休克良好的防治作用,掌握好这些单抗的输入时机是治疗感染性休克成功的前提;②己酮可可碱可抑制TNF-α、IL-1、IL-6、IL-8的释放;③PAF拮抗剂及具有阻断PAF的致炎作用,部分甚至可以完全逆转低血压,纠正低有效循环和心排出量状态。

\hypertarget{text00055.htmlux5cux23CHP2-1-3-3-2-2}{}
(2) 自由基清除剂:

氧自由基能直接损害细胞膜结构和DNA,感染性休克时线粒体的呼吸暴发、核苷酸降解代谢增加以及I/R都使体内自由基产量猛增,结果导致线粒体功能障碍、细胞发生凋亡或坏死,使用SOD、还原性谷胱甘肽、维生素C、辅酶Q\textsubscript{12}
、别嘌呤醇等自由基清除剂和抗氧化剂用于休克的治疗可减轻自由基引起的破坏。另外,氮自由基包括一氧化氮(NO)均与感染性休克时的炎症紊乱和微循环障碍的形成机制有关,在一些研究中,使用NO合成酶抑制剂(iNOS)可明显改善动物内毒素休克模型的预后。

\hypertarget{text00055.htmlux5cux23CHP2-1-3-3-2-3}{}
(3) 环氧化酶(COX)抑制剂:

TXA\textsubscript{2} 可收缩小血管,促血小板聚集,PGI\textsubscript{2}
则与之作用相反,休克时TXA\textsubscript{2} /PGI\textsubscript{2}
增加,导致组织灌注不良和DIC。非甾体类抗炎药物(NSAIDs)阿司匹林、吲哚美辛等能抑制COX,减少前列腺素的生成,还能抑制NFκB的跨膜转运。NSAIDs的作用是通过抑制COX-2而实现的,具有同样COX-2抑制作用的还有糖皮质激素类药物以及COX-2的特异性抑制剂如塞来昔布(商品名:西乐葆,Celecoxib)、罗非昔布(rofecoxib)等。

\hypertarget{text00055.htmlux5cux23CHP2-1-3-3-2-4}{}
(4) 细胞核因子(NF)抑制剂或配体:

NF类的NFκB
和PPARγ能通过不同的途径调控炎症反应,在休克时使用NFκB的抑制剂或PPARγ配体可以减轻抗炎的剧烈程度,从而保护器官功能。

\hypertarget{text00055.htmlux5cux23CHP2-1-3-3-2-5}{}
(5) 酶抑制剂:

这类物质包括乌司他丁、抑肽酶和NOS抑制剂等。乌司他丁是广谱酶抑制剂,可抑制炎症细胞释放的多种蛋白、糖和脂水解酶,保护溶酶体膜的稳定性,减少MDF和细胞因子的生成;抑肽酶抑制细胞释放的胰蛋白酶、纤维蛋白酶等多种酶类,从而对休克时的毛细血管通透性增加、血压下降和心功能降低以及DIC等有抑制作用。

\hypertarget{text00055.htmlux5cux23CHP2-1-3-3-2-6}{}
(6) 其他抗炎药物:

苯海拉明拮抗组胺的生成,色苷酸钠可稳定溶酶体酶、防止组胺释放,这些药物在抗休克的实验性治疗中均体现出一定的作用。

\paragraph{热休克蛋白}

(HSP)诱导剂
HSP在机体的应激反应中起重要作用,可从分子水平调节细胞内平衡,启动内源性保护机制,提高抗氧化应激能力,抑制细胞凋亡,修复细胞损伤。现在人们已研制出HSP诱导剂如bimoclomol等,希望通过诱导HSP的表达,在器官、组织和细胞水平上抵抗休克所造成的损伤。

\paragraph{休克的亚低温治疗}

实验研究发现,轻度低温(34~36℃)与正常体温(38℃)相比,可延长出血性休克鼠存活时间约1倍,这将为休克的临床救治开辟一条新的思路。

\paragraph{阿片样物质拮抗剂}

内源性阿片样物质(OLS)中以β内啡肽与休克关系密切。β内啡肽广泛存在于中枢神经系统,休克时血中含量增加5~6倍,通过中枢阿片受体抑制血管功能,使血压下降。纳洛酮为内源性OLS的特异性拮抗剂,其结构与吗啡相似,能阻断OLS与阿片受体结合,包括拮抗β内啡肽效应,可提高血压,使左室收缩力加倍,外周血管阻力降低,改善组织灌注。

\paragraph{镁剂和 Ca\textsuperscript{2+} 拮抗剂}

镁制剂为Ca\textsuperscript{2+}
拮抗剂,休克时使用镁剂有助于改善由于细胞内钙超载引起的损害,其他Ca\textsuperscript{2+}
拮抗剂如异博定能阻断小动脉平滑肌的Ca\textsuperscript{2+}
跨膜内流而使血管扩张,减轻I/R损伤,试验中使用ATP-MgCl\textsubscript{2}
和Ca\textsuperscript{2+} 拮抗剂均可观察到其线粒体保护作用。

\paragraph{血管紧张素转化酶抑制剂(ACEI)}

血管紧张素Ⅱ能强烈收缩血管,刺激醛固酮分泌,强化交感神经的缩血管效应,导致休克恶化,因而使用ACEI有益于休克救治。

\paragraph{中西医结合治疗}

祖国医学对休克治疗有着悠久的历史,随着休克的病理生理机制的进一步阐明,中药对于休克的疗效也日益受到重视,一部分具有抗炎、强心作用的中药如大黄、黄连和人参、丹参等已被做成单方或复方针剂广泛用于临床急救,对于中西医结合抗休克治疗的深入研究不仅可以丰富祖国医学的理论宝库,也有可能使我们在休克的基础和临床工作方面取得有特色的突破。

简而言之,尽管纠正休克症状的救治过程大致相同,但由于各种病因的差异,具体细节和治疗中的侧重点仍有差别,因此,在休克症状得以控制后,临床工作的主要任务应从对症治疗转变为对因治疗,同时重视患者的脏器功能恢复和内环境包括微循环状况的稳定和改善。

\protect\hypertarget{text00056.html}{}{}

\hypertarget{text00056.htmlux5cux23CHP2-1-4}{}
参 考 文 献

1. Millham FH. A brief history of shock. Surgery,2010,148
(5):1026-1037.

2. Fouche Y,Sikorski R,Dutton RP. Changing paradigms in surgical
resuscitation. Crit Care Med,2010,38(9 Suppl):S411-S420.

3. De Backer D,Biston P,Devriendt J,et al. Comparison of dopamine and
norepinephrine in the treatment of shock. N Engl J
Med,2010,362(9):779-789.

4. Jones AE,Shapiro NI,Trzeciak S,et al. Lactate clearance vs central
venous oxygen saturation as goals of early sepsis therapy:a randomized
clinical trial. JAMA,2010,303(8):739-746.

5. Annane D,Bellissant E,Bollaert PE,et al. Corticosteroids in the
treatment of severe sepsis and septic shock in adults:a systematic
review. JAMA,2009,301(22):2362-2375.

6. Marik PE,Cavallazzi R,Vasu T,et al. Dynamic changes in arterial
waveform derived variables and fluid responsiveness in mechanically
ventilated patients:a systematic review of the literature. Crit Care
Med,2009,37(9):2642-2647.

7. Trzeciak S,McCoy JV,Phillip DR,et al. Early increases in
microcirculatory perfusion during protocol-directed resuscitation are
associated with reduced multi-organ failure at 24 h in patients with
sepsis. Intensive Care Med,2008,34 (12):2210-2217.

8. Klijn E,Den Uil CA,Bakker J,et al. The heterogeneity of the
microcirculation in critical illness. Clin Chest
Med,2008,29(4):643-654.

9. 刘大为.实用重症医学.北京:人民卫生出版社,2010.

10. Levy MM,Dellinger RP,Townsend SR,et al. The Surviving Sepsis
Campaign:results of an international guideline-based performance
improvement program targeting severe sepsis. Crit Care
Med,2010,38(2):367-374.

\protect\hypertarget{text00057.html}{}{}

\chapter{感染性休克}

感染性休克(septic
shock)指由各种病原微生物及其毒素或通过抗原抗体复合物激活机体潜在反应系统,其中包括交感-肾上腺髓质系统、补体系统、激肽系统、凝血与纤溶系统等,使单核-吞噬细胞系统功能损害,神经-内分泌系统反应强烈,分泌过量儿茶酚胺类物质,导致微血管痉挛、微循环障碍、代谢紊乱、重要脏器灌注不足和再灌流损伤等征象。因此感染性休克是微生物因子和机体防御机制相互作用的结果,微生物的毒力数量以及机体的内环境与应答是决定感染性休克发生发展的重要因素。

脓毒性休克(septic
shock)是指严重脓毒症患者在给予足量液体复苏后仍存在组织低灌注(无法纠正的持续性低血压状态或血乳酸浓度≥4mmol/L),是对感染性休克认识的深化,感染性休克这一传统概念正越来越广泛地被脓毒性休克取代。

在了解感染性休克和脓毒性休克之间关联之前先介绍几个概念:感染(infection):指微生物入侵机体组织,在其中生长繁殖并引起从局部到全身不同范围和程度的炎症反应。这一概念强调了疾病是由病原微生物的入侵所引起的。菌血症(bacteriemia):指循环血液中存在活体细菌,其诊断依据主要为阳性血培养。败血症(septicemia):泛指血液循环中存在微生物或其毒素引起明显的临床症状。由于此含义规定血中细菌不断繁殖,但往往有明显感染症状者血培养不全是阳性,因此造成歧义太多,容易导致概念混乱,现已基本废止。全身炎症反应综合征(systemic
inflammatory response
syndrome,SIRS):指任何致病因素作用于机体所引起的全身性炎症反应,且具备以下2项或2项以上体征:体温>
38℃或< 36℃;心率> 90次/分;呼吸频率> 20次/分或PaCO\textsubscript{2}
< 32mmHg;外周血白细胞计数> 12.0 × 10\textsuperscript{9} /L或< 4.0 ×
10\textsuperscript{9} /L,或未成熟粒细胞>
0.10。脓毒症(sepsis):指由感染引起的全身炎症反应,证实有细菌存在或有高度可疑感染灶,其诊断标准同SIRS。严重脓毒症(severe
sepsis):是指脓毒症伴有器官功能不全、组织灌注不良或低血压。脓毒性休克可以被认为是严重脓毒症的一种特殊类型。

感染性休克和脓毒性休克概念貌似不同,但实际上两者叙述同一临床问题,而脓毒性休克更值得推广使用。两者均都强调了组织的低灌注,但感染性休克更强调的是感染对脓毒症和休克的发生和发展的重要性,但相当一部分脓毒性休克患者自始至终未找到明确的感染病灶或细菌学证据,因此脓毒症和脓毒性休克是否发生、发展,并不完全依赖于细菌和毒素的发生和发展,而是依赖于感染或非感染因素所导致的SIRS,感染很多时候是作为触发启动因素存在,脓毒性休克更强调机体的反应性和细菌代谢产物对机体的病理生理作用。

\subsection{病因与发病机制}

\subsubsection{病因}

\paragraph{病原菌}

各种感染性疾病如肺炎、败血症、腹膜炎、急性重型胰腺炎和各类脓肿等均可导致脓毒性休克。其病原体以革兰阴性细菌为最常见,如不动杆菌、大肠埃希菌、铜绿假单胞菌、肠杆菌、嗜麦芽假单胞菌、克雷白杆菌、痢疾杆菌和脑膜炎球菌等;亦可见于革兰阳性菌,如金葡菌、粪链球菌、肺炎链球菌、产气荚膜杆菌等。此外病毒(如流行性出血热、巨细胞病毒性肺炎等)、支原体等亦可引起脓毒性休克。

\paragraph{宿主因素}

原有慢性基础疾病,如肝硬化、糖尿病、恶性肿瘤、白血病、烧伤、器官移植以及长期接受肾上腺皮质激素等免疫抑制剂、抗代谢药物、细菌毒类药物和放射治疗,或应用留置导尿管或静脉导管者可诱发脓毒性休克。因此本病较多见于医院内感染患者,老年人、婴幼儿、分娩妇女、大手术后体力恢复较差者尤易发生。

\subsubsection{脓毒症的病理生理机制}

脓毒症的病理生理机制尚未完全阐明,可能与下列过程相关:

\paragraph{炎症失衡及免疫功能紊乱}

正常情况下,机体合成和释放促炎介质的同时产生抗炎介质,以遏制促炎介质作用过度而导致组织细胞损伤。机体受到微生物侵袭后,炎症反应和抗炎反应达到平衡状态,机体内环境才能稳定,炎症反应才不致失控。

脓毒症时机体表现为一种复杂的免疫功能紊乱状态。一方面,表现为促炎介质过度释放增加过度的炎症反应;另一方面,具有免疫抑制作用的抗炎介质大量释放,出现免疫功能抑制或麻痹,表现为免疫防御反应低下,吞噬杀菌能力减弱,抗感染防御能力下降等。

\paragraph{神经}

-内分泌-免疫网络
脓毒症早期,神经系统将炎症信息传递到中枢神经,通过调节内分泌系统、免疫系统或通过神经递质直接影响脓毒症的病理过程。

\paragraph{低血压与氧弥散及氧利用障碍}

过度炎症反应状态下出现:①内源性扩血管物质一氧化氮、前列环素、组胺、缓激肽等增加,造成血管对缩血管物质失去反应性而致功能障碍,循环阻力下降,出现低血压甚至休克;②机体同时释放内皮素-1、血栓素、血管紧张素和5-羟色胺等缩血管物质,舒、缩血管物质分泌紊乱和血管反应性低下,一部分组织器官过度灌注而出现“窃血”现象,导致氧供障碍;③氧自由基等损伤造成红细胞变形性下降和内皮细胞水肿,使红细胞难以通过更小的血管,从而影响氧弥散;④组织水肿造成氧弥散距离增加,导致氧利用障碍。

\paragraph{心肌抑制}

炎症介质如TNF-a、PFA、白三烯等具有抑制心肌收缩力的负性肌力作用,减少冠状动脉血流量,使心脏射血分数和心排出量明显下降。

\paragraph{内皮细胞受损及血管通透性增加}

大多数炎症介质均可导致血管内皮细胞损伤并使血管通透性增加,形成组织和器官水肿。

\paragraph{凝血功能障碍及微血栓形成}

脓毒症时凝血系统活化,并促进炎症的发展;炎症反应也可引起凝血系统活化,两者相互影响,共同促进脓毒症的恶化。

\paragraph{高代谢和营养不良}

过度炎症反应导致机体蛋白分解,抑制糖和脂类利用的高代谢反应,大量细胞因子分泌和消耗,机体可在短期内陷入重度营养不良,加重组织器官损伤。

\paragraph{肠道细菌}

/内毒素移位及金黄色葡萄球菌外毒素肠道是机体最大的细菌库及不明原因感染的“策源地”,其所致的肠源性感染与脓毒症和MODS密切相关。有学者认为,内毒素血症并不是感染引起,而是肠道细菌/内毒素移位所致,它参与了脓毒症及其并发症的病理过程。

\paragraph{受体与信号转导}

外界刺激对免疫、炎症等细胞功能的调节与受体及细胞内多条信号转导通路的活化密切相关,引起细胞应激、生长、增值、分化、凋亡、坏死等生物学效应。

\paragraph{基因多态性}

基因多态性是决定人体对应激打击易感性与耐受性、临床表现多样性及药物治疗反应差异性的重要因素。严重创伤或感染后全身炎症反应失控及器官损害受体内众多基因调控,表现出高度的个体差异,有些人群易发生脓毒症,有些人群则不容易发生。

\subsubsection{休克的病理生理机制}

\hypertarget{text00057.htmlux5cux23CHP2-2-1-3-1}{}
(一) 微循环变化

\paragraph{微循环舒缩功能异常}

典型脓毒性休克的发展过程有微血管痉挛、微血管扩张和微血管麻痹三个阶段。由于休克早期存在交感神经节后纤维释放去甲肾上腺素和肾上腺髓质释放肾上腺素和去甲肾上腺素,其血浆儿茶酚胺水平高于正常200~500倍,同时血管紧张素Ⅱ等大量分泌,使微血管平滑肌强烈痉挛。以上是由于细菌内毒素通过以下作用机制所致:①内毒素本身有拟交感神经作用;②内毒素作用于白细胞和血小板而释放组胺、缓激肽、5-羟色胺,使肺小静脉收缩,回心血量减少,心排量下降,有效血循环量不足,造成血压下降;③内毒素与补体相结合产生血管活性多肽和心血管毒性因子,使微血管收缩;④内毒素提高微血管对儿茶酚胺反应敏感性,尤其是微静脉和小静脉。在脓毒性休克的中晚期,微血管常发生舒张,其机制是:①缺氧、酸中毒;②β受体发生兴奋使微血管舒张;③组胺释放后,外周血管扩张;④内毒素休克晚期,血管平滑肌摄Ca\textsuperscript{2+}
能力低,ATP酶活性降低,胞浆内Ca\textsuperscript{2+}
储存少,血管平滑肌张力降低,对血管活性药缺乏反应;⑤由于微动脉痉挛,微静脉扩张,旁路开放,组织缺血缺氧进一步加重。

\paragraph{微血管壁通透性增高}

\hypertarget{text00057.htmlux5cux23CHP2-2-1-3-1-2-1}{}
(1) 微血管壁渗漏:

毛细血管是微循环中主要血管,其壁由单层内皮细胞组成,真毛细血管与组织细胞间非常靠近,有利物质与气体交换,而毛细血管壁相近的两个内皮细胞间是紧密连接(tight
junction),仅存狭窄细缝,宽约3~20nm,脓毒性休克时体内酸性物质、组胺、5-羟色胺、缓激肽等剧增,使内皮细胞中微丝发生收缩,纤维连接蛋白破坏,从而使毛细血管内皮细胞间裂缝加大,其通透性增高,严重时可发生渗漏现象,临床上称“渗漏综合征”,此为脓毒性休克发生发展的重要机制。

\hypertarget{text00057.htmlux5cux23CHP2-2-1-3-1-2-2}{}
(2) 自身体液调节障碍:

在休克早期,由于微动脉强烈收缩,微循环旁路开放,毛细血管血流减少,流速减慢甚至停滞,流体静水压下降,组织间液通过毛细血管壁进入微血管(即功能性细胞外液)起“自身输液”作用,此对微循环的灌流具有维持有效循环血量、起着一定代偿作用,但对组织细胞起着不利影响,故应注意及时补充功能性细胞外液。随着休克发展,微循环淤血缺氧和组织酸中毒加重,组胺等积蓄,使微动脉血管平滑肌对儿茶酚胺类反应性降低,造成血液不仅通过直接动静脉短路,而且大量进入毛细血管网,使毛细血管流体静水压上升,而功能性细胞外液通过毛细血管壁进入毛细血管的“自身输液”作用停止。相反随着毛细血管壁通透性增加,使血管内液体成分大量外渗,其速度可高达600ml/h,结果反而造成一个“自身失液”,严重时毛细血管壁破损而发生渗漏,甚至将大分子血浆蛋白渗漏至组织间隙。临床上,胸、腹腔、面颈、四肢等出现水肿,严重者气管内血浆样物质外渗,从而造成有效循环血量大减,血液浓缩和黏度增高,局部血小板聚集在损伤血管内皮上,产生血小板栓子,进而形成DIC。

\paragraph{微血管流态紊乱}

休克发生微血管流态紊乱不仅在晚期,而且还可出现在早期,其变化过程有以下三个阶段:

\hypertarget{text00057.htmlux5cux23CHP2-2-1-3-1-3-1}{}
(1) 血细胞聚集:

在内毒素和血小板释放促凝物质,使血细胞与微血管间,血细胞相互间黏附力增加,微血流不畅,出现血小板聚集和红细胞聚集等现象。

\hypertarget{text00057.htmlux5cux23CHP2-2-1-3-1-3-2}{}
(2) 血池及微血流淤泥形成:

当微血管痉挛、微血流紊乱时,微血管可以发生扩张,甚至形成微血管瘤,此时微循环中有血液蓄积,从而造成血池。由于血池中淤滞一定量血液,进而促使血流淤泥,形成液体在上、有形成分在中、凝聚团块在下的血池现象,此常是DIC前奏。

\hypertarget{text00057.htmlux5cux23CHP2-2-1-3-1-3-3}{}
(3) 弥散性血管内凝血(DIC):

DIC与脓毒性休克紧密相连,其机制有以下几点:①严重感染所致应激反应(stress)使血液凝固性升高;②休克时出现微循环障碍,血流滞缓,黏度增高,微血管淤泥等现象,同时又有细菌、病毒、内毒素等所致血管内皮和组织损伤促使微血栓形成;③内毒素引起全身性Shwartzman反应即连续2次静脉注射小剂量内毒素产生DIC而死亡。因首次接触内毒素后单核巨噬细胞功能抑制,机体处于高凝低纤溶状态,当第二次接触小剂量内毒素后即可发生促凝,产生DIC;④内毒素使血小板聚集并释放大量血小板因子3(PF\textsubscript{3}
)、因子4 (PF\textsubscript{4}
)及β-血栓球蛋白等促凝物质,PF\textsubscript{3}
加速凝血酶激活;PF\textsubscript{4}
能中和肝素并使可溶性纤维蛋白复合物沉淀,而内毒素能增加血小板激活凝血因子X的活性作用;⑤内毒素作用于粒细胞,B淋巴细胞,特别是单核细胞,当内毒素中类脂A与此类细胞膜接触而发生细胞破坏,释放组织因子,促发外凝系统;⑥内毒素可激活纤溶、激肽和补体系统,促进凝血。

\hypertarget{text00057.htmlux5cux23CHP2-2-1-3-2}{}
(二) 代谢障碍

\paragraph{氧化磷酸化障碍}

在缺氧、糖酵解加强、高能磷酸化合物生成减少情况下产生以下结果:

\hypertarget{text00057.htmlux5cux23CHP2-2-1-3-2-1-1}{}
(1) 乳酸增多:

乳酸(L)和丙酮酸(P)的比即L/P增高,因前者是糖无氧酵解产物,后者是糖有氧氧化产物,为此L/P可表示细胞氧化还原状态(正常10∶1),此外剩余乳酸(excess
lactate)或称超乳(简称XL),指与丙酮酸不成比例增高的乳酸(正常XL为0),如L增加而P不变或减低,提示细胞缺氧。当动脉血乳酸盐>
4.5mmol/L时,死亡率明显增加,据Peretz,统计< 1.4mmol/L其死亡率为0;<
4.4mmol/L为22\%;< 8.9mmol/L为73\%;> 13mmol/L为100\%。Buoulec提出XL
> 3时很少存活。以上提示乳酸的增高程度和预后密切相关。

\hypertarget{text00057.htmlux5cux23CHP2-2-1-3-2-1-2}{}
(2) 酸中毒:

当局部或全身pH下降,血浆中溶酶体酶的增多,细胞膜通透性升高,细胞内Na\textsuperscript{+}
升高,K\textsuperscript{+}
降低,使细胞功能发生障碍。临床上患者意识障碍的变化程度与Na\textsuperscript{+}
升高量呈正相关,检查红细胞内Na\textsuperscript{+}
含量,有利于分析判断中枢神经系统功能状态。

\paragraph{线粒体功能变化}

在内毒素性休克时,线粒体功能早期变化是膜的肿胀和转运钙能力障碍,影响细胞呼吸和ATP合成以及细胞的收缩功能。

\paragraph{溶酶体变化}

当内毒素休克时细胞溶酶体膜通透性升高,完整性破坏,溶酶体酶释出并可产生如下作用:①对心肌抑制和内脏血管收缩;②体内水解酶增多,造成细胞内线粒体功能的崩解和细胞的自溶;③促使血小板聚集。故有人认为休克时器官功能衰竭与溶酶体大量裂解有关。

\paragraph{氧自由基对细胞损伤}

主要通过破坏细胞膜和DNA以及使蛋白质变性。细胞经氧自由基作用5分钟,细胞膜上的泵蛋白和载体蛋白即受影响,细胞膜迅速去极化,作用35分钟细胞开始溶解。氧自由基首先损害血管内皮细胞,使其肿胀、通透性增加,体液外流,继而造成MODS。

总之,休克发病中细胞代谢障碍,具有重要意义,细胞膜功能障碍,进而细胞代谢异常,最后导致细胞死亡。Abboud认为休克发病机制常从代偿性低血压使组织灌注减少,微循环衰竭逐步发展到细胞膜损伤和细胞死亡。并提出休克分以下三期,即Ⅰ期:代偿性低血压期;Ⅱ期:组织灌流减少期;Ⅲ期:微循环衰竭和细胞膜损伤期。在脓毒性休克中细胞代谢障碍是由内毒素直接作用造成,乃属原发性。最近有人提出血管对活性药无反应是与体内一氧化氮(NO)过多有关。

\subsection{诊断}

脓毒性休克的诊断需满足:符合SIRS的标准;有感染的证据;在给予足量液体复苏后仍存在组织低灌注(无法纠正的持续性低血压状态或血乳酸浓度≥4mmol/L)。

全身炎症反应综合征(SIRS)的诊断标准:指任何致病因素作用于机体所引起的全身性炎症反应,且具备以下2项或2项以上体征:体温>
38℃或< 36℃;心率> 90 次/分;呼吸频率>
20次/分或动脉二氧化碳分压(PaCO\textsubscript{2} )< 32mmHg(1mmHg =
0.133kPa);外周血白细胞计数> 12.0 × 10\textsuperscript{9} /L或< 4.0 ×
10\textsuperscript{9} /L,或未成熟粒细胞> 0.10。

脓毒性休克根据血流动力学改变属于血流分布异常性休克(表\ref{tab20-1})。血流分布异常性休克有低前负荷型和正常前负荷型两类,前者属低排高阻型(低动力型),后者为高排低阻型(高动力型)。当合并有心肌梗死或严重心肌缺血心功能不全则不能代偿性增加心排量,特别注意冠脉血流量并不减少,但流经心肌动静脉血氧含量差明显减少,存在供需失衡。

\begin{table}[htbp]
\centering
\caption{脓毒性休克的血流动力学分型}
\label{tab20-1}
\includegraphics[width=3.25in,height=2.16667in]{./images/Image00084.jpg}
\end{table}

脓毒性休克病例诊断需重视以下几个方面:

\subsubsection{临床表现特点}

\paragraph{感染史}

脓毒性休克常有严重感染基础,尤其注意急性感染、近期手术、创伤、器械检查以及传染病流行病史。当有广泛非损伤性组织破坏和体内毒性产物的吸收亦易发生脓毒性休克。临床表现有寒战、高热、多汗、出血、栓塞、各器官功能减退等。

\paragraph{脑}

脑组织耗氧量很高,对缺氧特别敏感,轻者烦躁不安,重者昏迷抽搐,但脑血管舒缩范围较小,其血流灌注主要取决于供应脑的动静脉血压差。休克早期脑血管代偿性舒张,故脑灌注尚能维持,当休克加重血压明显下降,脑灌注不良,即可产生脑水肿,进一步加重脑灌注不足。患者意识可反映中枢神经系统微循环血流灌注量减少情况,但酸碱、水电解质失衡和代谢产物积蓄对意识有一定影响。临床上休克早期表现为烦躁不安,以后转为抑郁淡漠,晚期嗜睡昏迷。

\paragraph{皮肤}

能反映外周微循环血流灌注情况,所以注意检查皮肤色泽、温度、湿度,有条件可监测血液温度、肛门直肠温度和皮肤腋下温度之差,正常情况各差0.5~1℃,如大于2~3℃则提示外周微血管收缩,皮肤循环血流灌注不足。临床上根据四肢皮肤暖冷差异又可分为“暖休克”和“冷休克”,两者之比较见表\ref{tab20-2}。

\begin{table}[htbp]
\centering
\caption{暖休克与冷休克的比较}
\label{tab20-2}
\includegraphics[width=3.32292in,height=1.95833in]{./images/Image00085.jpg}
\end{table}

\paragraph{肾}

肾脏血流量很大,正常达1000~1500ml/min,占全身血流量的25\%,休克时血流产生重新分配,出现肾小动脉收缩,肾灌注量减少,造成少尿或无尿,肾缺血又引起肾小管坏死,影响尿液的浓缩和稀释及酸化功能,出现低比重尿(正常1.010~1.020)、尿pH
>
5.5,提示肾曲小管缺损,存在碳酸氢钠渗漏或远曲小管分泌H\textsuperscript{+}
障碍。

\paragraph{肺}

氧分压(PaO\textsubscript{2} )、氧饱和度(SaO\textsubscript{2}
)和呼吸改变是脓毒性休克时肺功能减退的可靠指标,主要表现在呼吸急促、PaO\textsubscript{2}
和SaO\textsubscript{2}
下降,皮肤和口唇发绀等缺氧表现,其原因有三:①肺泡微循环灌注存在而有通气障碍如肺泡萎陷、肺间质和肺泡水肿、肺炎症等;②肺泡通气良好而有灌注障碍,如回心血量少,心排量降低,肺动脉痉挛,肺微循环栓塞等造成肺血流灌注减少;③肺泡微循环和通气均有障碍。临床常表现为ALI和ARDS。

\paragraph{心脏}

由于细菌毒素作用,常发生中毒性心肌炎;由于细胞线粒体、溶酶体和代谢障碍酸中毒对心肌产生抑制作用,心肌收缩力减退,心排血量减少,血压下降、脉压小、冠状动脉灌注不足,心肌缺血、缺氧等造成心功能损害,急性心力衰竭和心律失常发生,进一步加重休克。

\paragraph{胃肠和肝}

在脓毒性休克时胃肠和肝可发生充血、水肿、出血和微血栓形成,消化道常发生应激性溃疡、糜烂、出血。肝细胞因内毒素和缺血缺氧而发生坏死,使肝功能各项酶、胆红素和血糖升高。

\paragraph{造血系统}

由于内毒素作用,常发生造血抑制和微血栓形成,结果造成血小板和各项凝血指标下降,临床出现DIC。

\paragraph{甲皱循环与眼底改变}

脓毒性休克时常因微血管痉挛造成甲皱毛细血管袢数目减少,周围渗出明显,血流呈断线、虚线或淤泥状,血色变紫,眼底检查小动脉痉挛,小静脉淤血扩张,动静脉比例由正常2∶3变为1∶2或1∶3,严重时有视网膜水肿,颅内压增高者可出现视乳头水肿。

\subsubsection{血流动力学变化特点}

常采用Swan-Ganz导管热稀释法或冷稀释法及非创伤性阻抗法测定血流动力学改变。

\paragraph{动脉血压与脉压}

在脓毒性休克情况下,上臂袖带式听诊法常出现听不清,无法了解血压真实数值,故主张桡动脉或股动脉插管直接测压法,当收缩压下降到80mmHg以下或原有高血压者下降20\%即患者的基础血压值降低30mmHg以下者,应认为血压已降低,组织微循环血液出现灌流减少,临床上可诊断为休克。脉压大小与组织血液灌注紧密相关,加大脉压有利改善组织供血供氧。一般要求维持收缩压在80mmHg、脉压>
30mmHg以上。

\paragraph{中心静脉压(CVP)}

主要反映回心血量与右心室搏血能力,有助于鉴别心功能不全还是血容量不足引起的休克,对决定输液的量和质,对选用强心、利尿或血管扩张剂均有较大指导意义。正常CVP为6~12cmH\textsubscript{2}
O,它与右心室充盈压成正比,在无肺循环或右室病变情况下,亦能间接反映右心室舒张末压和心脏对输液的负荷能力。

\paragraph{肺动脉楔压(PAWP)与左心房平均压}

与左心室舒张末压密切相关,在无肺血管和二尖瓣病变时测定PAWP,能反映左心室功能,对估计血容量,掌握输液速度和防止肺水肿等是一个很好指标。其正常值5~16mmHg。

\paragraph{心排血量(CO)}

反映心脏泵功能的一项综合指标,受心率、前负荷、后负荷及心肌收缩性等因素的影响。其正常值4~8L/min。

\paragraph{脉搏和静脉充盈情况}

脓毒性休克早期脉呈细速(120~140次/分),在休克好转过程中脉搏强度的恢复较血压早。休克时需观察静脉充盈程度,当静脉萎陷,且补液穿刺有困难,常提示血容量不足,而静脉充盈过度则反映心功能不全或输液过多。

临床上根据血流动力学变化,将脓毒性休克分为三个类型:①低排高阻型:常由严重G\textsuperscript{−}
杆菌感染而释放类脂多糖蛋白复合体,刺激机体防御功能分泌大量儿茶酚胺,引起微动脉及微静脉痉挛,动静脉短路开放,回心血量减少,同时毒素和代谢产物对心肌抑制,造成心排血量降低;②高排低阻型:常由G\textsuperscript{+}
球菌严重感染引起机体组胺释放β受体兴奋使心排量增加,外周血管扩张阻力减低,血压下降所形成。③低排低阻型:在休克晚期进入微循环衰竭期,血管平滑肌麻痹而扩张,心功能衰竭,排血量又减少,各重要脏器损伤,并发症相继出现,常属濒死阶段。

\subsubsection{实验室检查}

1.血象
脓毒性休克其白细胞总数多升高,中性粒细胞增加,核左移。但如感染严重,机体免疫抵抗力明显下降时,其白细胞总数可降低。血细胞比容和血红蛋白增高,提示血液浓缩。并发DIC时,血小板进行性下降。

2.尿和肾功能
当有肾衰竭时尿比重由初期偏高转为低而固定,血肌酐和尿素氮升高,尿与血的肌酐浓度之比<
1∶5,尿渗透压降低,尿/血浆渗透压的比值< 1.5,尿钠排出量>
40mmol/L,尤其警惕尿量多比重低,尿素氮、肌酐增高的“非少尿性肾衰”。

3.血气分析 常有低氧血症、代谢性酸中毒、而PaCO\textsubscript{2}
早期由于呼吸代偿而可轻度下降呈呼吸性碱中毒,晚期出现呼吸性酸中毒。

4.血清电解质 血钠和氯多偏低,血钾高低不一。

5.出凝血各项指标 多有异常改变,动态监测提高DIC诊断警惕性。

6.动脉血乳酸浓度
是反映休克程度和组织灌注障碍重要指标,需2~4小时监测一次。

7.寻找病原体,有利于去除病因。

\subsubsection{诊断注意事项}

笔者提出如下注意点供临床参考:

\paragraph{意识状态}

意识变化随血压变化出现烦躁转入昏迷,但需因人而异,老年患者有动脉硬化,即使血压下降不明显,亦可出现明显意识障碍,反之体质好,脑对缺氧耐受性高,虽然血压测不到,其神志仍可清醒。

\paragraph{血压}

是诊断休克的一项重要指标,但在休克早期,由于交感神经兴奋,儿茶酚胺释放过多,可以造成血压“假性”升高,此时如使用降压药,将会引起严重后果。

\paragraph{尿量}

既反映肾微循环血流灌注量,亦可间接反映重要脏器血流灌注情况,当血压维持在80mmHg,尿量>
30ml/h,表示肾灌注良好。当冷休克时,袖带法血压听不清,而尿量尚可时,表示此血压尚能维持肾灌注,反之使用血管收缩剂,血压虽在90mmHg以上,但四肢皮肤湿冷,无尿或少尿,同样提示肾和其他脏器灌注不良,预后差(表\ref{tab20-3}和表\ref{tab20-4})。

\paragraph{肾功能}

肾功能判断不仅注意尿量,而且尿比重和pH以及血肌酐和尿素氮水平进行综合分析,不要被单纯尿量所迷惑,注意对非少尿性急性肾功能衰竭的鉴别,此时尿量可超过1000ml/d,但尿比重低且固定,尿pH上升,提示肾小管浓缩和酸化功能差,血清肌酐和尿素氮上升,表示肾脏功能不佳,应予注意识别。

\paragraph{低氧}

对于低氧血症和ARDS诊断,应有足够认识。由于低氧血症原因未能很好寻找,救治措施不力,产生一系列代谢紊乱,结果出现不可逆休克。作者体会尽早行机械辅助通气,纠正低血氧。但要注意排除非ARDS(血气胸、连枷胸反常呼吸、气道堵塞和严重充血性心力衰竭等)引起的严重低氧血症。

\paragraph{血糖}

休克时血糖水平经常较高,此因感染性休克时交感神经兴奋,生糖激素释放,肝功能受损,胰岛功能减退,外源性葡萄糖补充等影响,造成继发性高血糖,对细菌、真菌生长创造了很好条件,同时高血糖又带来血液高渗,尤其对中枢神经损害和血管反应性进一步下降,休克加剧。

\paragraph{心率判断}

正常心率60~100次/分,但脓毒性休克时机体处于高代谢状态,同时细菌毒素和代谢产物对心脏作用,故心率代偿性增快常在100次/分以上,一旦下降至60~70次/分钟,常预示心脏失代偿而即将停止跳动,不要误认为心功能改善。

\paragraph{血清电解质变化}

需要准确判断与分析。由于脓毒性休克代谢性酸中毒,细胞释放K\textsuperscript{+}
,故血清钾有时很高且难以下降,但大剂量利尿剂、脱水剂和胃肠减压等影响,血清钾均可下降;又由于体液丧失,血液浓缩,使血清钾相对升高,而此时细胞内可以存在严重低钾,故应结合血生化、心电图和临床综合分析判断。脓毒性休克时,常存在镁、锌、铜、磷等降低,尤其镁的补充对MODS防治可获裨益。

\paragraph{酸碱失衡鉴别}

脓毒性休克组织缺血、缺氧、代谢性酸中毒是酸碱失衡基础,但由于呼吸深快的代偿作用,出现代谢性酸中毒+呼吸性碱中毒,血pH可以在正常范围,一旦呼吸抑制出现代谢性酸中毒+呼吸性碱中毒,病情加剧。当同时合并低氯、低钾又产生代谢性碱中毒时,血气分析判断更为复杂,为三重性酸碱失衡,应结合临床进行鉴别。

\begin{table}[htbp]
\centering
\caption{组织灌流检查}
\label{tab20-3}
\includegraphics[width=6.69792in,height=4.01042in]{./images/Image00086.jpg}
\end{table}

\begin{table}[htbp]
\centering
\caption{休克程度的鉴定}
\label{tab20-4}
\includegraphics[width=6.66667in,height=5.19792in]{./images/Image00087.jpg}
\end{table}

\paragraph{注意二重感染}

鉴于抗生素使用广泛,且剂量大,常可掩盖局部严重感染征象,又由于抗休克时采用大剂量激素,容易并发真菌感染,注意血、尿、粪、痰和口腔检查真菌病原体,争取早发现,早处理。

\subsection{治疗}

\subsubsection{早期目标治疗(EGDT)}

患者由于脓毒症导致的休克定义为组织的低灌注(表现为经过最初的液体复苏后持续低血压或者血乳酸浓度≥4mmol/L),此时应当进行早期复苏,并且应当在确定存在组织低灌注第一时间进行而不是延迟到患者入住ICU以后实施。在进行早期复苏的最初6小时内,由脓毒症导致的休克所存在的组织低灌注复苏目标包括以下方面:①中心静脉压(CVP):8~12mmHg;②平均动脉压(MAP):≥65mmHg;③尿量:≥0.5ml/(kg•h);④中心静脉(上腔静脉)氧饱和度(ScvO\textsubscript{2}
)或者混合静脉氧饱和度(SvO\textsubscript{2} )分别≥70\%或者≥65\%。

脓毒性休克在最初6小时复苏过程中,虽然经过液体复苏CVP已经达到了目标,但是对应的ScvO\textsubscript{2}
与SvO\textsubscript{2}
没有达到70\%或者65\%,可以为患者输入浓缩红细胞达到血细胞比容≥30\%同时(或者)输入多巴胺(最大剂量为20μg/kg•min)来达到目标。

\subsubsection{控制感染}

抗感染治疗是救治脓毒性休克主要环节。在无明确病原菌前,应经验性选择抗感染治疗方案,根据原发病灶和临床表现,推测最可能的致病菌,选用一种或者更多的强力、广谱抗生素以对抗所有可能的病原微生物{[}细菌和(或)真菌{]},并且要有足够的药物浓度可以渗透到可能导致脓毒症的感染病灶中去。要尽可能快的寻找病因并诊断或者排除诊断,所有表现为脓毒性休克的患者,要对其感染灶的病原学控制情况做出评估,尤其是当患者有脓肿引流或者有局部感染灶,感染后坏死组织清创,摘除可引起感染的医疗工具等。

对于抗生素应用有主张从一代头孢菌素开始逐步升级至三、四代。但脓毒性休克的发生常来势凶猛,病情危急,且细菌的病原菌不明,常带来治疗困难,故按“降阶梯治疗”实行“猛拳出击全面覆盖”原则,可选用碳青霉烯类(美平、泰能等),疗效较高,应尽早应用。

确认脓毒性休克后,应在1小时之内尽早静脉使用抗生素进行治疗。在进行抗生素应用之前留取合适的标本,但是不能为留取标本而延误抗生素的使用。如果患者现有的临床症状被确定为非感染性因素引起,应停止抗生素治疗以减少患者可能被抗生素耐药细菌引起感染和与药物相关的副作用风险。

\subsubsection{液体复苏}

脓毒性休克时均有血容量不足,推荐用天然/人工的胶体或晶体液进行液体复苏。目前没有证据支持某种液体优于其他种类液体。目前实验表明使用白蛋白等同于晶体液。一些关于ICU患者的小规模研究的Meta分析表明晶体和胶体复苏效果没有差异。要达到同样的治疗目标时,晶体液量要明显多于胶体液量,但晶体液更便宜。

液体复苏的初始治疗目标是使CVP至少达到8mmHg(机械通气患者需达到12mmHg)。对怀疑有血容量不足的患者进行液体冲击时,在开始的30分钟内要至少用1000ml的晶体液或300~500ml的胶体液。当只有心脏充盈压(CVP或者肺动脉楔压)增加而没有血流动力学改善时,才应该降低补液速度。

\subsubsection{血管活性药的应用}

使用血管活性药物使平均动脉压保持在≥65mmHg。即使在低血容量还没有得到纠正时,就该使用血管加压类药物,使MAP达到65mmHg以上,以保证低血压时能维持组织灌注。另外,在达到MAP治疗目标时应该考虑到患者既往基础病。

脓毒性休克患者推荐将去甲肾上腺素或多巴胺作为纠正脓毒性休克时低血压的首选血管加压药物。不推荐将肾上腺素、去氧肾上腺素或抗利尿激素作为脓毒性休克的首选血管加压药物。如果去甲肾上腺素或多巴胺效果不明显时可以首选肾上腺素。不推荐将低剂量的多巴胺作为肾脏保护药物。随机临床试验和Meta分析在比较低剂量多巴胺和安慰剂的作用时,没有发现差异。因此,目前尚无可用数据支持低剂量多巴胺可以维持肾功能。在出现心脏充盈压升高心输出量降低,出现心肌功能障碍时,应该静脉滴注多巴酚丁胺,但是反对使用它来增加心指数达超常水平,研究发现使用多巴酚丁胺将严重脓毒症患者的氧输送提高到超常水平并没有益处。

笔者认为冷休克低排高阻情况联合使用血管活性药与血管扩张剂常可获裨益。由于脓毒性休克晚期是血管痉挛收缩,故加用血管扩张剂是合理的,它不仅解除微动脉痉挛,而且有降低心脏前后负荷,解除支气管痉挛,有利通气改善,有利于恢复有效循环血量及组织灌注,使组织代谢酸性产物进入血液循环,从而得到及时纠正,达到消除休克之目的。使用血管扩张剂注意点:①在扩容基础上,其有效血容量得到充分补充情况下可加用血管扩张剂;②剂量应逐步升与降,防止机体不适应和反跳现象;③注意首剂综合征发生,有的患者对某种血管扩张剂(如哌唑嗪等)特别敏感,首次用后产生严重低血压反应,故药物种类和剂量需因人而异;④血管扩张剂单一长期应用可产生“受体脱敏”现象,对药物产生不敏感性,故应予更换。莨菪类药物在脓毒性休克救治上为我国首创。纳洛酮治疗脓毒性休克已获得成功,该药可阻断β内啡肽等物质的降压作用,因而使血压回升,同时有稳定溶酶体膜,降低心肌抑制因子的作用,使心排量增加。中药丹参、川芎等具有使微血管淤滞或缓慢流动的血细胞加快流速,降低血液黏度,开放毛细血管网,扩张微血管,疏通微循环,此外尚有抗凝、调整纤溶和清除氧自由基等作用,达到活血化淤改善微循环防治DIC的作用。

\subsubsection{纠正酸中毒}

纠正酸中毒可增强心肌收缩力,恢复血管对血管活性药物的反应性,防止DIC的发生。根本措施在于改善组织的低灌注状态。缓冲碱主要起治标作用,且血容量不足时,缓冲碱的效能亦难以充分发挥。首选的缓冲碱为5\%碳酸氢钠,次为11.2\%乳酸钠(肝功能损害者不宜用)。三羟甲基氨基甲烷(THAM)适用于需限钠患者,因其易透入细胞内,有利于细菌内酸中毒的纠正;其缺点为滴注溢出静脉外时可致局部组织坏死,静滴速度过快可抑制呼吸、甚至呼吸停止。此外,尚可引起高钾血症、低血糖、恶心呕吐等。当pH≥7.15时,不推荐使用碳酸氢盐改善血流动力学状态或减少升压药应用。

\subsubsection{肾上腺皮质激素}

糖皮质激素具有抗过敏、抗炎、抗毒素、抗休克等作用,经临床大量观察证明其可降低脓毒血症、脓毒性休克病死率。但只建议在血压对于液体复苏和血管加压药治疗不敏感时应用静脉肾上腺皮质激素。且如果可以使用氢化可的松不推荐使用地塞米松,而氢化可的松不应大于300mg/d。当不再需要血管升压类药物时,即应停用皮质激素治疗。

\subsubsection{重组人类活化蛋白C(rhAPC)}

rhAPC具有促进纤维蛋白溶解、抑制血栓形成及抑制白细胞活化的特性。脓毒性休克患者体内内源性活化蛋白C水平显著下降,并有研究表明rhAPC的补充可降低脓毒性休克的病死率,故国外指南建议脓毒症诱导的器官功能不全伴有高死亡危险(大多数APACHEⅡ≥25或多器官功能不全)的成年患者,如果没有禁忌证,应接受rhAPC治疗。低死亡危险严重脓毒症成年患者(APACHEⅡ<
20或一个器官衰竭),不予rhAPC治疗。

\subsubsection{血液制品}

一旦发现成人组织低灌注难以减轻,如心肌缺血、严重低氧血症、急性出血、发绀型心脏病或乳酸酸中毒,推荐当血红蛋白下降低于7.0g/dl(70g/L)时输注红细胞,使血红蛋白维持在7.0~9.0g/dl(70~90g/L)。在没有出血或有计划的侵入性操作时,如果凝血实验正常,不建议用新鲜冷冻血浆。新鲜冷冻血浆对于危重患者预后的影响,尽管没有临床研究评价,但当证实有凝血因子缺乏(凝血酶原时间、国际标准化比率或部分凝血活酶延长)、活动性出血或外科手术或侵入性操作前,推荐输注新鲜冷冻血浆。当血小板计数<
5 × 10\textsuperscript{9}
/L,无论是否有出血,都推荐输注血小板。当血小板计数5 ×
10\textsuperscript{9} ~30 × 10\textsuperscript{9}
/L并且有明显出血危险时,可以考虑输注血小板。外科手术或侵入性操作特别需要血小板计数≥50
× 10\textsuperscript{9} /L。

\subsubsection{血糖控制}

对进入重症监护病房后已经初步稳定的重症脓毒症合并高血糖患者,应使用强化静脉胰岛素治疗来控制血糖,使血糖控制在8.33mmol/L(150mg/dl)以下。所有接受静脉胰岛素治疗的患者都可以用葡萄糖作为热量来源,每1~2小时监测一次血糖,直到血糖和胰岛素用量稳定后可每4小时监测一次。多项研究显示,强化静脉胰岛素治疗,血糖控制可减少ICU死亡率、减少器官功能障碍及住ICU时间。但使用Leuven方案强化胰岛素治疗发生低血糖的风险是传统经验治疗的3倍左右。因此,虽然主张积极控制高血糖,但也需警惕低血糖的发生。

\subsubsection{脓毒性休克的支持治疗}

\paragraph{心功能不全的防治}

重症休克和休克后期病例常并发心功能不全,乃因细菌毒素、心肌缺氧、酸中毒、电解质紊乱、心肌抑制因子、肺血管痉挛、肺动脉高压和肺水肿加重心脏负担,及输液不当等因素引起。老年人和幼儿尤易发生,可预防应用毒毛旋花苷或毛花苷丙。出现心功能不全征象时,应严重控制静脉输液量和滴速。除给予快速强心药外,可予血管舒张剂,但必须与去甲肾上腺素或多巴胺合用以防血压骤降。大剂量肾上腺皮质激素有增加心搏血管和降低外周血管阻力、提高冠状动脉血流量的作用,可早期短程应用。同时给氧、纠正酸中毒和电解质紊乱,并给能量合剂以纠正细胞代谢失衡状态。

\paragraph{维持呼吸功能、防治ARDS}

肺为休克的主要靶器官之一,顽固性休克常并发呼吸功能衰竭。此外脑缺氧、脑水肿等亦可导致呼吸衰竭。休克患者均应给氧,经鼻导管(4~6L/min)或面罩间歇加压输入。吸入氧浓度以40\%左右为宜。必须保持呼吸道通畅。在容量补足后,如患者神志欠清、痰液不易清除、气道有阻塞现象时,应及早考虑作气管插管或切开并行辅助呼吸,并清除呼吸道分泌物,注意防治继发感染。对吸氧而不能使PO\textsubscript{2}
达满意水平(>
9.33~10.7kPa)、间歇正压呼吸亦无效的A-V短路开放病例,应及早给予呼气末正压呼吸(PEEP),可通过持续扩张气道和肺泡、增加功能性残气量,减少肺内分流,提高动脉血氧分压、改善肺的顺应性、增高肺活量。除纠正低氧血症外,应及早给予血管解痉剂以降低肺循环阻力,并应正确掌握输液、控制入液量。对于ALI/ARDS患者进行机械通气,应设定6ml/kg的潮气量;设定初始平台压上限≤30cmH\textsubscript{2}
O,评估气道压力时应考虑胸壁顺应性因素;允许PaCO\textsubscript{2}
高于正常水平,如需要,可减少平台压和潮气量;设置PEEP防止呼气末肺泡塌陷;对需要保持可能对机体造成潜在损伤水平较高FiO\textsubscript{2}
或气道高压的ARDS患者,只要不存在体位改变的风险,应考虑使用侧卧体位;除非禁忌,机械通气患者应取半卧位,床头抬高30°~45°。

\paragraph{肾功能的维护}

休克患者出现少尿、无尿、氮质血症等时,应注意鉴别其为肾前性或急性肾功能不全所致。在有效心搏血量和血压回复之后,如患者仍持续少尿,可行液体负荷与利尿试验:快速静滴甘露醇100~300ml,或静注速尿40mg,如排尿无明显增加,而心脏功能良好,则可重复一次,若仍无尿,提示可能已发生急性肾功能不全,应给予肾脏替代治疗。若循环稳定可予血液透析治疗,对血流动力学不稳定患者,CVVH更适宜、方便、安全。

\paragraph{脑水肿的防治}

脑缺氧时,易并发脑水肿,出现神志不清、一过性抽搐和颅内压增高征,甚至发生脑疝,应及早给予血管解痉剂、抗胆碱类药物、渗透性脱水性(如甘露醇)、速尿、头部降温与大剂量肾上腺皮质激素(地塞米松10~20mg)静注以及高能合剂等。

\paragraph{DIC的治疗}

DIC的诊断一经确立后,采用肝素或低分子肝素,并适当输注新鲜血浆、全血及血小板。在DIC后期、继发性纤溶成为出血的主要原因时,可加用抗纤溶药物。

\paragraph{应激性溃疡的防治}

脓毒性休克患者可以使用H\textsubscript{2}
受体阻滞剂或质子泵抑制剂PPI来预防应激性溃疡导致的上消化道出血,但也要考虑到胃内pH升高可能增加呼吸机相关性肺炎的风险。在ICU患者预防应激性溃疡致上消化道出血的收益的研究中,有20\%~25\%的合并脓毒症。这种获益应适用于重症脓毒症及脓毒性休克患者。此外,受益于应激性溃疡预防性治疗的几种情况的患者(凝血功能障碍,机械通气,低血压)经常合并严重脓毒症和脓毒性休克。预防上消化道出血的同时也必须权衡胃内pH升高的潜在的影响,这会增加发生呼吸机相关肺炎的风险。那些上消化道出血风险最大的严重脓毒症患者最能从预防应激性溃疡中获益。

\protect\hypertarget{text00058.html}{}{}

\hypertarget{text00058.htmlux5cux23CHP2-2-4}{}
参 考 文 献

1. 中华医学会重症医学分会
.成人严重感染与感染性休克血流动力学监测及支持指南(草案).中国实用外科杂志,2007,27:7-13.

2. Delling RP,LeveMM,Carlet JM,et al. Surviving Sepsis
Campaign:International guidelines for management of severe sepsis and
septic shock:2008. Crit Care Med,2008,36:296-327.

3.
周荣斌,王新华.从脓毒症基础研究热点展望临床治疗前景.中国急救医学,2009,29(8):674-688

\protect\hypertarget{text00059.html}{}{}

\chapter{心源性休克}

心源性休克(cardiogenic
shock,CS)是指各种原因所致的以心脏泵血功能障碍为特征的急性组织灌注量不足而引起的临床综合征。与其他休克一样,其共同特征是有效循环血量不足,组织和细胞的血液灌注虽经代偿仍受到严重限制,从而引起全身组织和脏器的血液灌注不良。其主要临床表现除有原发性心脏病的表现外,尚伴有血压下降,面色苍白,四肢湿冷和肢端发绀,浅表静脉萎陷,脉搏细弱,全身乏力,尿量减少,烦躁不安,反应迟钝,神志淡漠,甚至昏迷等。

CS最常见的病因为急性心肌梗死(AMI)。狭义上的心源性休克指的是发生于AMI泵衰竭的严重阶段。广义上心源性休克还包括其他原因,如充血性心力衰竭、急性心肌炎、心肌病,乳头肌或腱索断裂、瓣叶穿孔,严重心脏瓣膜病变和严重的心律失常等心脏因素以及心包填塞、张力性气胸等外在阻力因素等。有学者曾将心源性休克分为两大类,即冠状动脉性休克和非冠状动脉性休克。也有主张按病理和临床特点进行分类为心肌坏死后心源性休克和非心肌坏死性心源性休克。GUSTO-1的研究表明AMI并发CS的发生率为7\%,几乎所有的休克都是在发病48小时内发生,并且30天的死亡率高达57\%。总的看来,AMI并发CS的发生率为7\%~10\%,病死率高达50\%~80\%。

\subsection{病因与发病机制}

\subsubsection{病因}

根据致病因素的特点,可以把心源性休克的主要病因归为以下几类:

\paragraph{急性心肌梗死}

CS更常见于ST段抬高性心肌梗死(ST-segment elevation myocardial
infarction,STEMI)患者。目前约5\%~8\%的STEMI患者合并CS,而非ST段抬高性心肌梗死(non-ST-segment
elevation myocardial
infarction,NSTEMI)患者为2.5\%。AMI伴CS患者通常冠状动脉病变严重,约2/3为3支血管病变,20\%为左主干病变,常见于左心室梗死面积大于40\%的患者。约40\%的患者有既往梗死史,对于既往大面积梗死者,即使是小面积再次梗死也可诱发CS。梗死面积扩展或梗死相关动脉再闭塞,非梗死区域心肌代偿失调等,均是迟发性CS的因素。右心室梗死同样是CS的重要原因,分别占GUSTO-1试验和SHOCK登记研究CS患者的20\%和5\%,梗死相关动脉96\%为右冠状动脉,其年龄相对较轻,陈旧性梗死病史相对较少,前壁梗死和多支血管病变相对少见,但病死率与左心室CS相仿,早期血运重建对两者病死率的影响相似。左心室泵衰竭同时伴有右心功能不全,并不进一步增加病死率。包括心室间隔穿孔、乳头肌功能不全或断裂所致的急性二尖瓣反流、心室游离壁破裂等在内的机械并发症,仍是导致CS的重要病因,病死率极高。下后壁梗死,尤其是首次梗死伴CS者,应高度怀疑合并机械并发症。

AMI伴CS的危险因素包括:高龄、前壁梗死、高血压、糖尿病、多支血管病变、既往梗死史和心绞痛史、既往心力衰竭史、STEMI和左束支传导阻滞等。

\paragraph{其他原因}

①终末期心肌病。②急性弥漫性心肌炎。③心肌挫伤或心脏手术后。④感染性休克所致严重心肌抑制。⑤严重心律失常。⑥左室流出道梗阻:主动脉狭窄、肥厚梗阻性心肌病。⑦左室充盈受阻:严重二尖瓣狭窄、左房黏液瘤。⑧瓣膜破裂所致急性二尖瓣反流。⑨急性心肌梗死时部分心源性休克的发生与不恰当的治疗或过度治疗有关,如过量的肾上腺素β受体阻滞药、ACEI、ARB、吗啡、利尿药均可导致低血压和组织低灌注。

\subsubsection{发病机制}

\paragraph{病理生理学}

心源性休克患者的心功能不全往往是由急性心肌梗死或缺血所导致,而心功能不全又会加重心肌缺血的情况,导致恶性循环。当大块心肌缺血坏死并产生泵血功能下降时,每搏输出量及心输出量也将下降。冠状动脉供血主要取决于心脏舒张时冠脉循环与左室内压力的压力梯度,而低血压及心动过速将影响这一压力梯度并导致冠脉供血下降而加重心肌缺血。心室泵功能的下降将进一步降低冠脉灌注压,心室充盈又令心肌需氧量增加,因而加重心肌缺血。心输出量的下降令外周循环灌注下降,进而导致乳酸性酸中毒,进一步降低心肌收缩力。

随着心肌功能的下降,数种代偿机制将被激活,包括交感神经兴奋增加心率及心肌收缩力,肾素-血管紧张素-醛固酮系统(RAAS)的激活致水钠潴留增加前负荷、血管收缩而增加后负荷。这些代偿机制的激活将进一步诱发心源性休克的进展,心率增加及心肌收缩力增加将增加心肌氧需求量及加重心肌缺血,液体潴留、因心动过速及心肌缺血所导致的舒张期充盈过度将导致肺淤血及低氧血症。为维持有效血压水平机体血管收缩,又加重了心肌的后负荷水平并令心肌功能进一步受损及增加心肌氧需求,增加的心肌氧需求量和灌注下降,令心肌缺血加重,以上因素如不能及时进行干预将形成恶性循环而导致死亡(图\ref{fig21-1})。

\begin{figure}[!htbp]
 \centering
 \includegraphics[width=3.11458in,height=2.96875in]{./images/Image00088.jpg}
 \captionsetup{justification=centering}
 \caption{心源性休克的发病机制}
 \label{fig21-1}
  \end{figure} 

另外
,在心肌梗死时大量的非梗死细胞出现可逆性功能丧失是患者在心肌梗死后出现心源性休克的重要因素,而这些细胞的可逆性功能丧失可以归为两类:心肌顿抑及心肌冬眠。心肌顿抑,是指心肌缺血后再灌注但心肌功能未能及时恢复,但最终这部分心肌功能可以完全恢复。心肌缺血后氧自由基损伤、钙离子失衡、肌纤维对钙离子反应性下降等可能参与了心肌顿抑的发生。此外循环中心肌顿抑因子也参与了心肌顿抑时的心肌收缩力下降。心肌顿抑的程度与再灌注前心肌缺血的程度有关。

心肌冬眠则是指在冠脉血流严重下降时心肌细胞功能静息,改善冠脉供血将令冬眠心肌细胞的功能恢复正常。心肌冬眠可以被认为是低灌注的心肌为了恢复血流灌注与心肌功能之间的平衡而降低心肌收缩力,是心肌的一种适应性反应,从而避免心肌缺血及坏死。心肌顿抑及心肌冬眠尽管在概念上及病理生理上截然不同,然而在临床上两者难以完全分开,往往合并存在。通过重建冠脉的血流,恢复心肌供血有助于改善心肌冬眠,而尽早恢复心肌供血及恢复心肌细胞离子平衡也有助于改善心肌顿抑。对于心源性休克患者,血流动力学支持以及尽可能降低心肌细胞坏死是极为重要的治疗措施。

\paragraph{心肌病理}

心源性休克时同时存在收缩与舒张功能不全,对心源性休克患者的临床与病理研究均发现有心肌细胞的进行性坏死。部分患者在入院后出现休克,可能是由于特定梗死血管的再梗阻所导致的梗死扩展、冠脉内栓子的扩散或心肌细胞需氧量增加及冠脉灌注下降等所导致。在梗死灶边缘的心肌细胞对缺血因素更为敏感,有进一步发生坏死的危险,而远离梗死灶的心肌细胞也由于缺血而导致收缩功能下降。

在心源性休克患者,往往观察到多支血管病变,冠脉的储备及自我调节功能下降,低血压及代谢产物堆积将影响非梗死的心肌细胞的收缩功能。心肌缺血时心肌顺应性下降,舒张末期容量负荷增加导致左室舒张末期压增加,而为了维持每搏输出量,左室内血容量代偿性增加,进一步令左室舒张末期压增加,继而导致肺淤血及低氧血症的发生。

瓣膜功能异常也参与了肺淤血的发生过程。由于心肌缺血所致的乳头肌功能不全导致二尖瓣关闭不全,继而导致左房压力负荷增加及肺淤血发生,后负荷降低可以减轻二尖瓣反流水平。乳头肌的完全性破裂将在短时间内迅速导致肺淤血及心源性休克的发生。

\paragraph{细胞病理}

组织低灌注及继发的细胞缺氧将导致葡萄糖无氧酵解,ATP生成减少,细胞内能量储备下降,而糖的无氧酵解又会导致乳酸堆积及细胞性酸中毒,能量不足导致细胞各种能量依赖性的离子通道功能丧失,继而导致跨膜电位的改变,细胞内钠、钙离子堆积,细胞水肿,细胞缺血及钙离子堆积将激活细胞内蛋白酶。严重而持续的心肌缺血将令心肌细胞损伤不可逆转,继而发生一系列改变,如线粒体的水肿、破裂,蛋白变性,染色体破坏,溶酶体破裂等,表现为典型的心肌细胞坏死。

细胞凋亡目前被认为参与了心肌梗死时的心肌细胞数量减少过程。在梗死灶以细胞坏死为主,而在梗死灶边缘区甚至远离梗死灶的心肌,由于缺血及低灌注等因素,可发现心肌细胞凋亡的表现,炎症因子暴发性释放、氧自由基损伤及心肌细胞的机械性拉伸等被认为是心肌梗死时凋亡通路启动的主要启动因素。就细胞凋亡对心肌梗死所起的作用,目前尚未完全明确,但在心肌缺血后再灌注的动物模型中进行抗凋亡治疗能减轻心肌细胞损伤。

除外细胞凋亡过程,目前发现心肌细胞胀亡(oncosis)也是心肌梗死及心肌缺血后再灌注损伤过程中导致梗死灶及邻近细胞迅速坏死的重要机制之一,细胞胀亡往往有外部刺激因素所触发,不同于以细胞缩小为主的凋亡过程。细胞胀亡时,胀亡的细胞胞体肿胀,胞膜通透性增加、完整性破坏,胞浆空泡化,内质网储钙能力下降,线粒体肿胀、嵴结构破坏、呼吸链受阻、ATP缺乏或胞核内染色质聚集成块,然而目前针对细胞胀亡的分子机制尚不明确,线粒体的损伤可能触发细胞发生胀亡,因为细胞凋亡过程需要线粒体产生相应所需的ATP方可完成,而在线粒体损伤的情况下ATP生成不足则相应地触发细胞胀亡;而且由于ATP不足会导致内质网过度释放钙离子,这也是导致细胞胀亡的重要机制之一;ATP的耗竭又会进一步导致细胞膜各种离子泵如Na\textsuperscript{+}
-K\textsuperscript{+}
-ATP酶等的功能丧失,同时细胞膜磷脂发生水解破坏,也进一步加剧细胞损伤及死亡。研究发现在发生缺血坏死的区域,供血供氧相对足够的部分常以凋亡为主,而在缺血缺氧较严重的部分则以胀亡居多;研究指出在ATP耗竭超过80\%~85\%时细胞胀亡就可能发生。

\paragraph{炎症反应机制}

患者发生心源性休克时伴随着更严重的炎症反应。THEROUX等研究观察发现白介素-6
(IL-6)和肿瘤坏死因子-α(TNF-α)异常增高的急性心肌梗死患者,入院时Killip分级为Ⅰ级,数日后发展为心源性休克。IL-6和TNF-α均有抑制心肌收缩力的作用。TNF-α还有损伤血管内皮功能的作用,使冠状动脉血流更趋减少。

\subsection{诊断}

\subsubsection{临床表现特点}

\paragraph{原发病的症状和体征}

如胸闷、胸痛、气促,心脏扩大、心前区抬举感,心律失常、心音遥远、出现第三和(或)第四心音、心脏杂音,颈静脉充盈或怒张,肺部细湿啰音,急性心肌梗死患者有典型的心电图及心肌酶学改变。

\paragraph{血压}

动脉收缩压≤80mmHg,舒张压<
60mmHg,原为高血压患者的收缩压≤90mmHg,或由原水平降低30\%以上。

\paragraph{循环不良体征}

皮肤苍白、发绀或出现花斑,皮肤湿冷,手、足背静脉塌陷,脉搏细速,胸骨部位皮肤指压恢复时间大于2秒等。

\paragraph{意识精神状态改变}

烦躁不安、焦虑、反应迟钝,昏睡甚至昏迷。

\paragraph{其他}

呼吸深快、心动过速(并发缓慢型心律失常者除外)、尿量减少。

\subsubsection{临床评估及辅助检查}

心源性休克是临床急症,需要在休克状态导致不可逆的重要器官损伤前迅速进行评估并尽早开始治疗,准确而迅速的病史采集和体格检查有助于了解、诊断原发病。注意排除低血容量、出血、脓毒血症、肺动脉栓塞、主动脉夹层等。对患者的神志情况、尿量、皮肤状况、肺部啰音等的监测有助于监测病情。

\paragraph{心电图}

心电图检查应该及时进行,可以确立心肌梗死的部位、范围,同时应进行心电监护,评估心率、心律,及时发现各种心律失常。

\paragraph{连续性血压检测}

包括床边无创连续性血压监测及动脉内插管测压,血压监测有助于对病情严重性、预后及治疗效果进行评估。由于休克状态下血管收缩以及各种血管活性药物的使用,无创测压测值往往低于实际值,因此采用动脉内插管测压较准确,多作桡动脉内插管测压。

\paragraph{超声心动图}

超声心电图检查有助于确诊心源性休克并排除其他原因所致的休克,能够反映总体及局部心肌的收缩功能,可以发现乳头肌断裂、急性二尖瓣反流、室间隔破裂或室壁瘤的破裂、心包填塞等。

\paragraph{侵入性血流动力学监测}

侵入性血流动力学监测能够排除血容量不足等情况,对治疗方法的选择、疗效及预后的判断有重要作用。常用指标有:

(1) CVP:CVP < 2cmH\textsubscript{2} O时提示存在血容量不足,CVP
>15cmH\textsubscript{2}
O提示右心功能不全,多数心源性休克患者CVP升高,二尖瓣反流、肺动脉栓塞、慢性阻塞性肺部疾病(COPD)、血容量过高时CVP也会升高。

(2)
LVEDP:LVEDP升高提示左室射血功能障碍及心室顺应性下降,LVEDP越高,提示心源性休克越严重,测定LVEDP可能诱发严重的室性心律失常,常采用测定PCWP来间接反映LVEDP。

(3)
PCWP:心源性休克时PCWP常高于15mmHg,PCWP在18~20mmHg时提示存在轻度肺淤血,21~25mmHg时提示中度肺淤血,26~30mmHg时提示重度肺淤血,31~35mmHg提示出现肺水肿,超过36mmHg提示重度肺水肿。对个别患者,由于左室舒张功能下降,为维持最佳的心室充盈压,PCWP可高于15mmHg。

(4)
RAP:心源性休克患者PCWP升高,但RAP一般仅稍升高或正常,合并右心功能不全或心包填塞时RAP明显升高。

(5) CI、CO、SVR:CI < 2.2L(/min•m\textsuperscript{2}
)提示存在心源性休克,CI < 1.8L(/min•m\textsuperscript{2}
)提示严重的心源性休克,动态观察CI、CO、SVR可以了解心脏收缩功能及体循环血管阻力。

(6)
其他指标:通过右心内导管监测血氧饱和度,在室间隔破裂时由于动静脉血混合右心内血氧饱和度升高;出现巨大的V波提示存在严重的二尖瓣反流;右室心肌梗死时右室充盈压明显升高但PCWP正常甚至下降。

\paragraph{冠脉造影}

进行急诊冠脉造影可以发现致梗死的罪犯血管,有助于判断预后,左前降支或多支病变患者发生心源性休克可能性更大,预后更差,在造影同时进行PTCA或支架植入的重建冠脉血流对治疗有重要作用。

\paragraph{其他指标}

血常规、电解质、心肌酶学、凝血功能等应即时进行检测及动态监测,有助于监测病情,了解心肌梗死程度并指导治疗方案。动脉血气分析可以提示是否存在呼吸功能衰竭。血乳酸水平检测可以反映休克持续的时间及循环障碍的程度。胸部X线拍片可以发现肺水肿等情况。

\subsubsection{诊断标准与病因判断}

\paragraph{休克诊断标准}

1982年2月全国急性“三衰”会议制订的休克诊断试行标准为:①有诱发休克的病因;②意识异常;③脉细速,超过100次/分或不能触及;④四肢湿冷,胸骨部位皮肤指压阳性(指压后再充盈时间>
2秒),皮肤花纹、黏膜苍白或发绀,尿量< 30ml/h或无尿;⑤收缩压<
80mmHg;⑥脉压< 20mmHg;⑦原有高血压者收缩压较原水平下降30\%以上。

凡符合以上①,以及②、③、④中的两项,和⑤、⑥、⑦中的一项者,可诊断为休克。

\paragraph{心肌损伤所致心源性休克的诊断}

CS为心室泵衰竭导致心输出量锐减,出现靶器官的低灌注状态,其核心为心室泵衰竭诱发的血流动力学紊乱伴有组织灌注不足。表现为持续性(超过30分钟)低血压(如收缩压<
80~90mmHg或者平均动脉压低于基线水平30mmHg,或者需要药物或机械支持使血压维持在90mmHg左右)伴有心脏指数(cardiac
index,CI)严重降低{[}无器械支持时< 1.8L/ (min•m\textsuperscript{2}
),或器械支持时< 2.0~2.2L/(min•m\textsuperscript{2}
){]}、心室充盈压升高(左心室舒张末压> 18mmHg或右心室舒张末压>
10~15mmHg)。临床上出现心率增快、肢端湿冷、尿少、呼吸困难和神志的改变,短期预后直接与血流动力学紊乱程度相关。肺动脉漂浮导管和(或)多普勒超声心动图检查有助于CS诊断的确立。

\hypertarget{text00059.htmlux5cux23CHP2-3-2-3-2-1}{}
(1) 急性心肌梗死并发心源性休克:

同时具备心肌梗死及休克的临床表现,血流动力学监测提示PCWP≥15mmHg,CI <
2.2L/(min•m\textsuperscript{2}
),右室心肌梗死并发心源性休克的血流动力学指标:SBP < 80mmHg,MAP <
70mmHg,RAP≥6.5mmHg,RAP >
PADP,PCWP≤15mmHg,CI≤1.8L/(min•m\textsuperscript{2} )。

\hypertarget{text00059.htmlux5cux23CHP2-3-2-3-2-2}{}
(2) 急性弥漫性心肌炎并发心源性休克:

好发于儿童及青壮年,常有病毒感染史,常伴有心律失常、晕厥等,体格检查发现有心动过速、心律失常、心脏扩大、心音遥远等,心电图有ST-T改变,心肌酶升高,肌钙蛋白T升高,但无急性心肌梗死的动态改变,病毒学检测阳性,心肌活检发现心肌炎性改变及检测出病毒RNA/DNA片段。

\hypertarget{text00059.htmlux5cux23CHP2-3-2-3-2-3}{}
(3) 心脏直视手术后低心排综合征:

心脏直视手术后出现CI下降、SVR升高。补充血容量、应用正性肌力药物及行IABP有效。

\paragraph{其他原因所致心源性休克的诊断}

\hypertarget{text00059.htmlux5cux23CHP2-3-2-3-3-1}{}
(1) 重度二尖瓣狭窄:

有风湿性心脏病、心房内黏液瘤、巨大血栓堵塞二尖瓣开口等病史,体格检查发现有二尖瓣狭窄的体征,无心肌梗死的心肌酶学改变及心电图改变,超声心动图发现二尖瓣狭窄表现。

\hypertarget{text00059.htmlux5cux23CHP2-3-2-3-3-2}{}
(2) 严重心律失常:

多见于持续快速性室性心律失常,有相应的临床表现及心电图表现,无心肌梗死的心肌酶学改变及心电图改变,复律后休克随之纠正。

\hypertarget{text00059.htmlux5cux23CHP2-3-2-3-3-3}{}
(3) 心包填塞:

突然发生,可由于主动脉夹层破入心包,Marfan综合征主动脉瘤破入心包、心脏介入手术损伤心包等造成,表现为急性心包填塞,超声心动图、X线检查有助于诊断,心包穿刺具有诊断及治疗作用。

\hypertarget{text00059.htmlux5cux23CHP2-3-2-3-3-4}{}
(4) 大面积肺梗死:

突发胸痛、气促、发绀、咯血、右心功能不全,有长期卧床、手术创伤等病史及外周血管内血栓形成的证据如下肢深静脉血栓形成,胸部X线拍片检查、高分辨率CT、核素扫描及肺动脉造影有助于确诊。

\subsubsection{鉴别诊断}

\paragraph{与其他类型休克的鉴别}

\hypertarget{text00059.htmlux5cux23CHP2-3-2-4-1-1}{}
(1) 脓毒性休克:

有畏寒、发热等感染征象,常合并其他器官损伤的表现,心脏损害可出现心功能不全、心肌酶学及心电图改变,无心肌梗死的心肌酶学改变及心电图改变,血常规白细胞总数及中性粒细胞水平增加,血培养提示血道性病原体感染。

\hypertarget{text00059.htmlux5cux23CHP2-3-2-4-1-2}{}
(2) 低血容量性休克:

有大量失血或体液丢失病史,血常规发现血细胞比容增加或血红蛋白水平显著下降,血流动力学检测提示CVP、CI、PCWP等都降低,SVR升高,补充血容量治疗有效。

\hypertarget{text00059.htmlux5cux23CHP2-3-2-4-1-3}{}
(3) 过敏性休克:

有过敏史或致敏原接触史,起病急,迅速出现喉头水肿、心肺受损等表现,大剂量激素、肾上腺素能受体激动剂、抗过敏治疗有效。

\hypertarget{text00059.htmlux5cux23CHP2-3-2-4-1-4}{}
(4) 神经源性休克:

有脑、脊髓受损史或腰麻平面过高史,查体发现有神经系统定位体征。

\paragraph{其他疾病}

\hypertarget{text00059.htmlux5cux23CHP2-3-2-4-2-1}{}
(1) 急性重症胰腺炎:

可于病初数小时内发生休克,既往有胰腺炎或胆道疾病史,发作时有明显的胃肠道症状及腹膜刺激征,心电图可发现一过性Q波和ST-T改变,但无典型的急性心肌梗死的心电图动态改变,心肌酶变化不大而淀粉酶显著升高。

\hypertarget{text00059.htmlux5cux23CHP2-3-2-4-2-2}{}
(2) 肾上腺危象:

严重乏力、低血压甚至休克,常伴有恶心、呕吐、腹痛、腹泻等消化道症状,实验室检查提示低血糖及电解质紊乱,常规抗休克治疗效果欠佳,予大剂量激素治疗有效。

\hypertarget{text00059.htmlux5cux23CHP2-3-2-4-2-3}{}
(3) 糖尿病酮症酸中毒:

糖尿病病史,伴有感染、脱水、停用胰岛素等诱因,除血压降低外伴有呼吸深快,带酮味,血糖显著升高,血及尿酮体阳性,血气分析提示酸中毒,大量补液及小剂量胰岛素治疗有效。

\subsection{治疗}

心源性休克的治疗包括对病因的治疗以及对休克的纠正。有可导致心源性休克可能的原发病应及时对因治疗。如针对心肌梗死及时进行溶栓治疗或其他冠脉血流重建治疗;心律失常者及时进行抗心律失常治疗,争取迅速复律;心包填塞时及时进行心包穿刺或其他手术治疗等。

\subsubsection{基本治疗}

\paragraph{补充血容量}

在心源性休克患者,除非合并肺水肿,否则应进行液体复苏,但由于心脏泵功能衰竭,应在血流动力学监测各种指标(表\ref{tab21-1})的指导下严格控制补液。尽快建立静脉通道包括中心静脉置管、漂浮导管置入等,监测CVP、PCWP,CVP及PCWP较低时提示血容量不足,可予适当补充晶体液或胶体液,CVP及PCWP在正常范围时补液应谨慎,必要时采用补液试验(10分钟内试验性静脉给予100ml液体观察血流动力学指标、循环状况、尿量等)指导补液,如CVP≥18cmH\textsubscript{2}
O、PCWP≥18mmHg时则提示血容量过高或肺淤血,应停止补液并使用血管活性药、利尿剂等。右室、下壁心肌梗死时出现低血压,应增加补液恢复血压,PCWP稍高于18mmHg可以接受,不作为停止补液的指征。

\paragraph{纠正电解质紊乱及酸碱失衡}

低钾低镁会增加发生室性心律失常的危险,酸中毒会影响心肌收缩力,需要及时纠正。

\paragraph{维持气道通畅及氧合}

常规予鼻导管或面罩吸氧,必要时进行气管插管及呼吸机辅助呼吸。

\paragraph{镇痛镇静}

常用吗啡,如收缩压较低可选用芬太尼,可减轻交感神经兴奋、降低氧需求量、降低前后负荷等。

\paragraph{心律失常}

心律失常所致心源性休克通过抗心律失常治疗可纠正休克状态,其他病因所致心源性休克如出现心律失常时应及时纠正,包括抗心律失常药的应用、电复律或安装临时起搏器。

\begin{table}[htbp]
\centering
\caption{Killip泵功能分级的处理}
\label{tab21-1}
\includegraphics[width=6.8125in,height=1.65625in]{./images/Image00089.jpg}
\end{table}

\paragraph{药物}

硝酸酯类、β受体阻断剂、ACEI等药物有助于改善心肌梗死预后,但在心源性休克时可加重低血压,故以上药物在患者病情稳定前应暂停使用。为控制静脉补液量,应尽量进行微泵静脉给药。

\subsubsection{改善心脏功能及外周循环状况}

心源性休克患者存在泵衰竭及外周循环衰竭,除一般抗休克治疗外,应针对以上情况进行治疗。如患者血容量足够仍出现组织低灌注,则应予正性肌力药物加强心肌收缩力治疗及血管活性药物支持治疗。

\paragraph{正性肌力药物}

原则上应选用增加心肌收缩力而不会大幅增加心肌耗氧、维持血压而不加快心率甚至导致心律失常的药物。

\hypertarget{text00059.htmlux5cux23CHP2-3-3-2-1-1}{}
(1) 多巴酚丁胺:

为选择性β\textsubscript{1}
-肾上腺素能受体激动剂,可以在不显著增加心率及外周血管阻力的情况下增加心肌收缩力及心输出量,较少增加心肌耗氧,同时降低LVEDP,在急性心肌梗死、肺梗死等所致心源性休克患者可作为首选正性肌力药,常用剂量5~15μg/(kg•min),逐渐调整给药速度至血流动力学指标改善,但连用72小时以上会出现受体耗竭导致药效下降。

\hypertarget{text00059.htmlux5cux23CHP2-3-3-2-1-2}{}
(2) 强心苷:

有可靠的正性肌力作用,但由于心源性休克时缺血和正常心肌在交感神经兴奋及儿茶酚胺释放等影响下心电活动不稳定性增加,合并氧合不足及低钾低镁时则更不稳定,可诱发严重心律失常,而且损伤心肌对药物反应下降,对洋地黄类毒性增加,故在心源性休克时强心苷应用有所限制,仅在其他药物效果欠佳及合并快速性室上性心律失常时使用,应用时剂量减少,并应选用短效制剂如毛花苷丙等。

\hypertarget{text00059.htmlux5cux23CHP2-3-3-2-1-3}{}
(3) 磷酸二酯酶抑制剂:

通过抑制磷酸二酯酶Ⅲ的活性从而减少cAMP降解,cAMP增加活化胞膜通道令钙离子动员增加,心肌细胞内钙离子浓度增加而令其收缩功能增强,对血管特别是肺循环血管有一定的扩张作用,半衰期长,其正性时相作用及致心律失常作用较小。常用药物有氨力农及米力农,后者效果更强,应用时首先予以负荷量随后继续静脉维持,氨力农负荷量0.5~0.75mg/kg静脉注射(大于10分钟),继以5~10μg/(kg•min)静脉滴注,每日总剂量不超过10mg/kg;米力农负荷量25~50μg/kg静脉注射(大于10分钟),继以0.25~0.50μg/(kg•min)静脉滴注。此类药物不宜长期维持。常见不良反应有低血压和心律失常。

\hypertarget{text00059.htmlux5cux23CHP2-3-3-2-1-4}{}
(4) 钙离子通道增敏剂:

左西孟旦(levosimendan)是一种钙增敏剂,通过结合于心肌细胞上的肌钙蛋白C促进心肌收缩,还通过介导ATP敏感的钾通道而发挥血管舒张作用和轻度抑制磷酸二酯酶的效应。其正性肌力作用独立于β肾上腺素能刺激,可用于正接受β受体阻滞剂治疗的患者。临床研究表明,急性心衰患者应用本药静脉滴注可明显增加CO和每搏量,降低PCWP、全身血管阻力和肺血管阻力;冠心病患者不会增加病死率。用法:首剂12~24μg/kg静脉注射(大于10分钟),继以0.1μg/(kg•min)静脉滴注,可酌情减半或加倍。对于收缩压<
100mmHg的患者,不需要负荷剂量,可直接用维持剂量,以防止发生低血压。

\hypertarget{text00059.htmlux5cux23CHP2-3-3-2-1-5}{}
(5) 重组人B型利钠肽(recombinate human B-type natriuretic
peptide,rhBNP):

急性失代偿性心力衰竭患者BNP血药浓度增高,心衰越重,BNP含量越高。机体内源性分泌BNP增多是一种代偿性的自救机制,当房内压高时分泌,以拮抗醛固酮。不仅可拮抗醛固酮的水钠潴留和心脏的重构,还能均衡地扩张动、静脉,拮抗肾素-血管紧张素系统、交感神经系统、内皮素系统和血管加压素系统在心力衰竭时过度代偿带来的循环系统效率降低。代偿不足,且患者存在对BNP的抵抗。故需要外源性补充BNP。

rhBNP近几年刚应用于临床,属内源性激素物质,与人体内产生的BNP完全相同。国内制剂商品名为新活素,国外同类药名为奈西立肽(nesiritide)。其主要药理作用是扩张静脉和动脉(包括冠状动脉),从而降低前、后负荷,在无直接正性肌力作用情况下增加CO,故将其归类为血管扩张剂。实际该药并非单纯的血管扩张剂,而是一种兼具多重作用的治疗药物;可以促进钠的排泄,有一定的利尿作用;还可抑制RAAS和较高神经系统,阻滞急性心衰演变中的恶性循环。该药临床试验的结果尚不一致。晚近的两项研究(VMAC和PROACTION)表明,该药的应用可以带来临床和血流动力学的改善,推荐应用于急性失代偿心衰。国内一项Ⅱ期临床研究提示,rhBNP较硝酸甘油静脉制剂能够显著降低PCWP,缓解患者的呼吸困难。应用方法:先给予负荷剂量1.5μg/kg,静脉缓慢推注,继以0.0075~0.0150μg/(kg•min)静脉滴注;也可不用负荷剂量而直接静脉滴注。疗程一般3天,不超过7天。

\paragraph{血管活性药物}

包括拟交感神经药、血管扩张药等。

\hypertarget{text00059.htmlux5cux23CHP2-3-3-2-2-1}{}
(1) 拟交感神经药:

以多巴胺、多巴酚丁胺为主的升压药和正性肌力药物是药物支持的中心环节,但以增加心肌氧耗和能量消耗为代价而改善血流动力学,临床上应尽可能小剂量使用以维持冠状动脉和重要脏器的灌注直至IABP置入或休克缓解。大剂量的升压药已被证实降低存活率,可能与其潜在的血流动力学恶化和直接的心脏毒性作用有关。ACC/AHA推荐去甲肾上腺素用于更加严重的低血压患者(≤70mmHg)。NOS抑制剂可竞争性抑制NO的合成,目前临床上试用的甲基-L-精氨酸(L-NMMA)使CS患者平均动脉压明显升高、尿量增加,静脉注射10分钟后即可起效;使持续性CS患者30天病死率由67\%降至27\%。

多巴胺曾是心源性休克时首选的血管活性药,同时兼有正性肌力作用。小剂量{[}≤2.5μg/(kg•min){]}兴奋DA\textsubscript{1}
受体,改善肾、脑、冠脉血流,同时兴奋突触前膜上的DA\textsubscript{2}
受体,减少内源性去甲肾上腺素释放;中剂量(2.5~10μg/
kg•min)兴奋β\textsubscript{1}
受体,令肾血流增加同时又令心肌收缩力增加,心率加快心输出量增加,外周血管阻力变化不一;大剂量{[}>
10μg/(kg•min){]}兴奋外周多数血管α受体,致血管收缩,血压升高。在心源性休克时多采用中剂量,达到大剂量时仍然不能使血压升高,则可加入间羟胺一同使用,多巴胺由于会增加心率及外周血管阻力,可能会加重心肌缺血。间羟胺与去甲肾上腺素作用类似,但较之弱而持久,对α、β受体都有作用,可用于协同多巴胺升高血压。无效时和严重低血压时可用去甲肾上腺素0.5~1.0mg加入5\%葡萄糖液100ml内以2~8μg/min静脉滴注。

\hypertarget{text00059.htmlux5cux23CHP2-3-3-2-2-2}{}
(2) 血管扩张药:

单独使用血管扩张剂可使心输出量增加和左室充盈压下降,但由于冠状动脉灌注压也明显降低,血管扩张剂会导致心肌灌注进一步恶化,加重循环恶化。因此只有在各种升压措施处理后血压仍不升,而PCWP增高(PCWP
> 18mmHg),心排血量低{[}CI < 2.2L/(min•m\textsuperscript{2}
){]}或周围血管显著收缩致四肢厥冷并有发绀时使用。血管扩张剂应与正性肌力药物联合应用。硝普钠从15μg/min开始,每5分钟逐渐增加至PCWP降至15~18mmHg;硝酸甘油从10~20μg/min开始,每隔5~10分钟增加5~10μg/min,直至左室充盈压下降。对有心动过缓或房室传导阻滞的CS,可用胆碱能受体阻滞剂如山莨菪碱静滴。一般情况下血管扩张剂与正性肌力药和主动脉内气囊反搏术联合应用,能增加心输出量,维持或增加冠状动脉灌注压。

\paragraph{利尿剂}

主要用于控制肺淤血、肺水肿,同时有助于改善氧合,但可能对血压产生影响。

\subsubsection{机械循环支持}

\paragraph{主动脉内球囊反搏(intra-aortic balloon pump,IABP)}

是对CS患者机械支持治疗的主要手段,是维持血流动力学稳定的有效措施,主要通过舒张期球囊充气以改善冠状动脉和外周血流灌注,收缩期球囊放气使后负荷明显减轻从而提高左心室功能。IABP适应证:①血流动力学不稳定,患者需要循环支持以做心导管检查,冠状动脉造影以发现可能存在的外科手术可纠正的病变,或是为冠状动脉旁路移植术(GABG)或经皮冠状动脉介入治疗(PCI);②对内科治疗无效的;③患者有持续性心肌缺血性疼痛,对100\%氧吸入、β受体阻滞剂和硝酸酯治疗无效的患者。SHOCK登记研究显示接受IABP的患者病死率明显下降,尤其是对于接受溶栓治疗的患者。因此,只要条件允许,应尽快在血运重建前置入IABP。ACC/AHA和ESC指南均将置入IABP列为药物治疗无效CS的Ⅰ类适应证。但并非所有患者均对IABP有血流动力学反应,也并非所有患者均能从IABP中获益,如高度狭窄的冠状动脉并未显示血流灌注增加。但有反应者提示预后良好。既往IABP并发症高达10\%~30\%,现已明显降低,尤其是在IABP置入量高的中心。大样本人群研究显示,总并发症和严重并发症的发生率为7.2\%和2.8\%。主要包括肢端缺血、主动脉夹层、股动脉破裂、感染、溶血、血栓形成及栓塞等。出现该并发症的主要危险因素为女性、身材小和外周血管疾病;禁忌证包括主动脉瓣反流、主动脉夹层和外周血管疾病。

\paragraph{左心室辅助设备}

(left ventricular assist device,LVAD)
LVAD借助外置的机械设备,暂时的、部分的代替心脏的功能,有助于组织的灌注,等待心功能的恢复,并打断心源性休克时的恶性循环,是心源性休克的重要治疗措施。左心室辅助设备以外科手术方法或导管方法从机体取血。常用的取血部位为左心室和左心房,将血以一定的压力回到升主动脉。左心室辅助设备常可作为心脏移植的过渡,对何种患者需果断使用左心室辅助设备尚需进一步研究。传统的左室辅助装置的安置需体外循环下手术经胸植入,近年来经皮左室辅助装置开始(percutaneous
left ventricular assist
device,PLVAD)逐渐应用于临床。目前临床运用比较成熟的PLVAD有两种,一种经股静脉、股动脉途径建立左房股动脉引流途径,另一种经股动脉植入微型轴流泵,直接建立左室-升主动脉引流途径。研究随机比较了PLVAD和IABP在急性心梗并发心源性休克中的疗效,PLVAD治疗组初级终点心脏指数及肾功能明显改善,血清乳酸水平降低,改善左室功能,但30天死亡率无明显差别。基础研究结果显示LVAD应用后可逆转心肌重构过程,并可改善心肌细胞β肾上腺素能受体信号功能,这些基础研究的发现均提示LVAD可能改善心衰及心源性休克患者的预后。

\subsubsection{血流重建治疗}

心源性休克最主要的病因是急性心肌梗死,重建冠脉血流对于恢复心肌供血及心肌功能有关键性的意义。

\paragraph{溶栓治疗}

对于急性心肌梗死,溶栓治疗已经被确认有助于降低急性心肌梗死的病死率,然而溶栓治疗在心源性休克中的地位尚未完全明确,早期溶栓治疗有助于降低心源性休克的发生率,但对于已经发生心源性休克的患者,多个临床试验未能证明溶栓治疗可以降低病死率。这个结果与患者的冠脉再灌注率有关,多数心源性休克患者冠脉再灌注失败,而在少数再灌注成功的患者,可以观察到病死率的下降。目前认为,血流动力学、机械因素以及代谢因素影响了溶栓治疗在心源性休克患者治疗中的作用,当心源性休克时由于动脉压力下降,溶栓药物较少到达栓子,梗死灶内血管在低血压时发生塌陷也影响了溶栓药物的作用,酸中毒则影响了纤维蛋白酶原向纤维蛋白原的转化。部分临床试验支持同时采用血管活性药物可以改善溶栓效果。

\paragraph{血管重建}

包括直接血管重建以及冠状动脉旁路手术。直接血管重建,包括经皮穿刺冠状动脉成形术(PTCA)及支架植入等,PTCA可以令80\%~90\%急性心肌梗死患者的狭窄血管达到TIMI
3级的血管通畅度,高于溶栓治疗的成功率(50\%~60\%),部分临床试验更指出,在高危患者(年龄>
70岁,大面积心肌梗死,心率>
100次/分)PTCA较溶栓治疗能更多地减低病死率。因此,对心源性休克患者进行急诊直接血管重建可能有益,除了改善梗死灶处心肌活动,梗死灶远端的心肌收缩力也有改善。而在进行PTCA同时植入支架,又能改善不能成功完成PTCA患者的存活率。血管重建后的抗血小板治疗对维持冠脉通畅有意义,血小板糖蛋白Ⅱb/Ⅲa受体拮抗剂应用可以改善近期内血管重建术的预后。冠状动脉旁路手术(CABG)同样被多项临床试验支持对心源性休克患者有益,然而手术进行需时而且手术有较高的机会出现各种并发症,令CABG应用受到限制。

\subsubsection{特殊情况的心源性休克治疗}

\paragraph{右室心肌梗死}

约30\%的下壁心肌梗死患者合并右室心肌梗死,患者表现为低血压、颈静脉充盈怒张、肺野清晰,右胸导联的心电图检查可发现典型的ST-T改变,右心内导管检查可发现RAP及RVEDP升高及PCWP正常或下降,CO下降,超声心动图提示右室心肌收缩力下降。右室心肌梗死患者发生心源性休克其预后稍好。通过补液维持右室前负荷,然而补液可能在增高PCWP后而不能令CO增加,右室过度充盈会导致心肌耗氧增加,降低右冠状动脉的灌注压,加重右心心肌缺血,也会影响左心室的充盈及CO。多巴酚丁胺治疗可有助于改善CO。血流动力学持续不稳定的患者采用IABP可能有益。恢复右心冠脉血流对改善预后有重要作用。

\paragraph{急性二尖瓣反流}

下壁心肌梗死、乳头肌缺血或梗死都可能导致急性二尖瓣反流,多发生于心肌梗死后2~7天,发生急性二尖瓣反流时会在短时间内出现肺水肿、低血压及心源性休克,当乳头肌断裂时心脏杂音仅在收缩早期可闻而且往往柔和甚至听不到,主要是因为左房左室内的压力差迅速消失。超声心动图可以迅速诊断,血流动力学监测有一定帮助,治疗上可用硝普钠降低后负荷,IABP有助于暂时控制病情,正性肌力药及血管活性药有助于维持CO及血压,手术修补或更换受损瓣膜是唯一彻底的治疗方法并且应尽早进行。

\paragraph{室间隔破裂}

患者表现为严重心功能衰竭及心源性休克,可闻及全收缩期杂音及触及心前区震颤,血流动力学监测发现左向右分流表现(右室内血氧饱和度升高),超声心动图有助于确诊。早期行IABP及药物支持,争取尽快进行手术治疗,一般建议是在发生破裂后的48小时内。

\paragraph{室壁破裂}

多发生于急性心肌梗死后第1周内,常见于老年女性伴高血压患者,早期溶栓治疗可以降低发生率。患者表现为急性心包填塞,应尽早认识到病情并行紧急心包穿刺抽液,减轻心包填塞,尽早行手术治疗。

\protect\hypertarget{text00060.html}{}{}

\hypertarget{text00060.htmlux5cux23CHP2-3-4}{}
参 考 文 献

1. Reynolds HR,Hochman JS. Cardiogenic shock:current concepts and
improving outcomes. Circulation,2008,117:86-697

2. 刘品明
.急性心肌梗死伴心源性休克.中华心血管病杂志,2009,37(10):956-960

3. Trost JC,Hillis LD. Intra-aortic balloon counterpulsation. Am J
Cardiol,2006,97(9):1391-1398

4. Buja LM. Myocardial ischemia and reperfusion injury. Cardiovasc
Pathol,2005,14(4):170-175

5. Bedi MS,Alvarez RJ,Jr.,Kubota T,et al. Myocardial Fas and
cytokine expression in end-stage heart failure:impact of LVAD support.
Clin Transl Sci,2008,1(3):245-248

\protect\hypertarget{text00061.html}{}{}

\chapter{失血性休克}

失血性休克(hemorrhagic
shock)是各种创伤和疾病引起的急性失血所导致循环血容量短期内丢失超过机体应急代偿能力而出现的有效循环血量与心排血量减少,继而引起组织灌注不足、细胞代谢紊乱和功能受损的一系列病理生理过程。是休克最常见的一种类型,系最具有代表性的低血容量性休克(hypovolemic
shock)。失血性休克的主要病理生理改变是有效循环血容量急剧减少,导致组织低灌注、无氧代谢增加、乳酸性酸中毒、再灌注损伤以及内毒素易位,最终导致的多器官功能障碍综合征(MODS)。其主要死因是组织低灌注以及大出血、感染和再灌注损伤等原因导致的MODS。

\subsection{病因与发病机制}

失血性休克的常见病因有:严重创伤、骨折、挤压伤等所致的外出血和内脏(如肝脾)破裂引起内出血;各种原因如消化性溃疡、急性胃黏膜病变、食管胃底静脉曲张破裂等所致的消化道出血;呼吸道出血引起的咯血;泌尿道出血引起的血尿;女性生殖道出血引起的阴道流血;腹腔、腹膜后、纵隔等出血、动脉瘤破裂出血等内出血。其中,创伤是急性失血性休克的最常见原因。

上述各种原因所致的内出血或外出血,都造成有效循环血容量丧失。有效循环血容量丢失触发机体各系统器官产生一系列病理生理反应,以保存体液,维持灌注压,保证心、脑等重要器官的血液灌流。

低血容量导致交感神经-肾上腺轴兴奋,儿茶酚胺类激素释放增加并选择性地收缩皮肤、肌肉及内脏血管。其中动脉系统收缩使外周血管总阻力升高以提升血压;毛细血管前括约肌收缩导致毛细血管内静水压降低,从而促进组织间液回流;静脉系统收缩使血液驱向中心循环,增加回心血量。儿茶酚胺类激素使心肌收缩力加强,心率增快,心排血量增加。

低血容量兴奋肾素-血管紧张素Ⅱ-醛固酮系统,使醛固酮分泌增加,同时刺激压力感受器促使垂体后叶分泌抗利尿激素,从而加强肾小管对钠和水的重吸收,减少尿液,保存体液。

上述代偿反应在维持循环系统功能相对稳定,保证心、脑等重要生命器官的血液灌注的同时,也具有潜在的风险。这些潜在的风险是指代偿机制使血压下降在休克病程中表现相对迟钝和不敏感,导致若以血压下降作为判定休克的标准,必然贻误对休克时组织灌注状态不良的早期认识和救治;同时,代偿机制对心、脑血供的保护是以牺牲其他脏器血供为代价的,持续的肾脏缺血可以导致急性肾功能损害,胃肠道黏膜缺血可以诱发细菌、毒素易位。内毒素血症与缺血-再灌注损伤可以诱发大量炎性介质释放入血,促使休克向不可逆发展。

机体对低血容量休克的反应还涉及代谢、免疫、凝血等系统,同样也存在对后续病程的不利影响。肾上腺皮质激素和前列腺素分泌增加与泌乳素分泌减少可以造成免疫功能抑制,患者易于受到感染侵袭。缺血缺氧、再灌注损伤等病理过程导致凝血功能紊乱并有可能发展为弥漫性血管内凝血。

组织细胞缺氧是休克的本质。休克时微循环严重障碍,组织低灌注和细胞缺氧,糖的有氧氧化受阻,无氧酵解增强,三磷酸腺苷(ATP)生成显著减少,乳酸生成显著增多并组织蓄积,导致乳酸性酸中毒,进而造成组织细胞和重要生命器官发生不可逆性损伤,直至发生MODS。

\subsection{诊断}

\subsubsection{临床表现特点}

对于因胃肠道、呼吸道、泌尿道、生殖道等发生大量出血所致的休克,由于出血一般均排出体外,诊断较易;但消化道出血有时在出现呕血和便血之前即有休克,如未注意,可延误诊断。

腹腔内出血常见者为脾破裂、异位妊娠破裂出血。脾破裂患者常有腹部外伤史,但有时可自发破裂,此时脾脏常因有充血或感染而肿大,故易于破裂,患者常先有左上腹疼痛,以后转为全腹部,且伴有腹部压痛和移动性浊音,诊断性腹腔穿刺可以确诊。异位妊娠破裂出血的患者常有短期(一般6~7周)闭经史,继而有阴道流血和腹痛。腹痛剧烈,首先位于下腹部,且有腹部移动性浊音。妇科检查有少量阴道出血,宫口闭,宫体可扩大,但不及应有孕期的大小。宫体的一侧可有清楚块质,且有压痛,阴道后穹隆膨出,经后穹隆穿刺可得血液而确诊。

胸腔出血可由外伤、肿瘤、胸膜粘连带的撕裂以及主动脉夹层等引起。患者常先有一侧胸痛,随呼吸而加剧,叩诊变浊,呼吸音降低。出血如在500ml以上,胸部X线可发现胸腔积液。确诊在于胸腔穿刺抽到血液。

因骨折而出血有时亦可引起休克,尤其是骨盆、股骨等骨折,出血迅速而量大,可在1000ml以上而外观可不明显,休克较易发生。例如:脊柱、盆腔骨折可因腰部静脉撕破而致腹膜后大量出血,血液能积存于组织间隙高达2000ml以上;股骨骨折时血液可储存于大腿的软组织中,虽积血达1000~1500ml也常不发生令人注目的急性肿胀。诊断时应特别重视。

此外,流行性出血热、肝破裂、肾破裂、腹主动脉瘤破裂、肿瘤(如肝癌结节破裂)、各种出血性疾病、应用抗凝剂等也可引起腹腔内或腹膜后大出血而发生休克,应予以足够的警惕。

失血量的临床估计方法可参考表\ref{tab22-1}。

\begin{table}[htbp]
\centering
\caption{失血量的临床估计方法}
\label{tab22-1}
\includegraphics[width=3.33333in,height=2.17708in]{./images/Image00090.jpg}
\end{table}

不论失血的病因如何,失血性休克多表现为冷型休克(低排高阻型休克),突出的表现特点是“5P”:皮肤苍白(pallor)、冷汗(prespiration)、虚脱(prostration)、脉搏细弱(pulselessness)、呼吸困难(pulmonary
dificiency)。最初反应为交感神经兴奋,表现为精神紧张、烦躁、皮肤苍白、出冷汗、四肢末端发凉、脉细速,血压可正常但脉压小;若出血量大或在较晚期血压常下降,并有呼吸困难。

\subsubsection{诊断注意事项}

失血性休克的早期诊断对预后至关重要。传统的诊断主要依据为病史、症状、体征,包括精神状态改变、皮肤湿冷、收缩压下降(<
90mmHg或较基础血压下降> 40mmHg)或脉压减少(< 20mmHg)、尿量<
0.5ml/(kg•h)、心率>100次/分、中心静脉压(CVP)<
5mmHg或肺动脉楔压(PAWP)<
8mmHg等指标。对于多发创伤和以躯干损伤为主的失血性休克患者,床边超声可以早期明确出血部位从而早期提示手术的指征;CT检查比床边超声有更好的特异性和敏感性。氧代谢与组织灌注指标对失血性休克早期诊断有更重要参考价值。

\subsubsection{失血性休克的分级}

临床上,根据失血量等指标可将失血性休克分成四级(以体重70kg的成年男性为例):

Ⅰ级(早期):失血量< 750ml,占血容量比例<
15\%,心率(HR)≤100次/分,血压正常或稍增高;

Ⅱ级(代偿期):失血量750~1500ml,占血容量比例15\%~30\%,HR >
100次/分,呼吸增快(20~30次/分),血压下降,皮肤苍白、发凉,毛细血管充盈延迟,轻~中度焦虑,尿量减少(20~30ml/h);

Ⅲ级(进展期):失血量1500~2000ml,占血容量比例30\%~40\%,HR >
120次/分,呼吸急促(30~40次/分),血压明显下降,神志改变如萎靡或躁动不安,尿量明显减少(5~20ml/h);

Ⅳ级(难治期):失血量> 2000ml,占血容量比例> 40\%,HR >
140次/分,脉搏细弱,呼吸窘迫(>
40次/分),血压显著下降,皮肤发绀、湿冷,意识障碍,无尿。

大量失血定义为24小时内失血超过患者的估计血容量或3小时内失血量超过估计血容量的一半。

\subsubsection{失血性休克的监测}

\paragraph{一般临床监测}

包括皮温与色泽、心率、血压、尿量和精神状态等监测指标。尿量是反映肾灌注较好的指标,可以间接反映循环状态。当尿量<
0.5ml/(kg•h)时,应继续进行液体复苏。需注意临床上患者出现休克而无少尿的情况,如高血糖和造影剂等有渗透活性的物质造成的渗透性利尿。血压的变化需要严密地动态监测。休克初期由于代偿性血管收缩,血压可能保持或接近正常。对未控制出血的失血性休克维持“允许性低血压”(permissive
hypotention),即维持平均动脉压(MAP)在60~80mmHg。

\paragraph{有创血流动力学监测}

包括:①MAP监测:有创动脉血压(IBP)较无创动脉血压(NIBP)高5~20mmHg。持续低血压状态时,NIBP测压难以准确反映实际大动脉压力,而IBP测压较为可靠,可保证连续观察血压和即时变化。②CVP和PAWP监测:CVP和PAWP监测有助于对已知或怀疑存在心功能不全的休克患者的液体治疗,防止输液过多导致的前负荷过度。③心排量(CO)和每搏量(SV)监测:连续监测CO与SV,有助于动态判断容量复苏的临床效果与心功能状态。

\paragraph{氧代谢监测}

传统临床监测指标往往不能对组织氧合的改变具有敏感反应,此外,经过治疗干预后的心率、血压等临床指标的变化也可在组织灌注与氧合未改善前趋于稳定。因此,同时监测和评估一些全身灌注指标如氧输送(oxygen
delivery,DO\textsubscript{2} )、氧消耗(oxygen
consumption,VO\textsubscript{2}
)、血乳酸、混合静脉血氧饱和度(saturation of mixed venous blood
oxygen,SvO\textsubscript{2} )或中心静脉血氧饱和度(saturation of
central venous blood oxygen,ScvO\textsubscript{2}
)等以及局部组织灌注指标如胃黏膜内pH(pHi)与胃黏膜CO\textsubscript{2}
张力(PaCO\textsubscript{2} of gastric mucosa,PgCO\textsubscript{2}
)等具有较大的临床意义。其中,动脉血乳酸浓度是反映组织缺氧的高度敏感的指标之一,动脉血乳酸增高常较其他休克征象先出现。持续动态的动脉血乳酸以及乳酸清除率监测对休克的早期诊断、判定组织缺氧情况、指导液体复苏及预后评估具有重要意义。碱缺失(base
deficit,BD)可间接反映血乳酸的水平。当休克导致组织供血不足时碱缺失下降,提示乳酸血症的存在。碱缺失可分为:轻度(−2~−5mmol/L),中度(<
−5~≥−15mmol/L),重度(<
−15mmol/L)。碱缺失与血乳酸结合是判断休克组织灌注较好的方法。

\paragraph{实验室监测}

包括:①血常规监测:动态观察红细胞计数、血红蛋白(Hb)及血细胞比容(Hct)的数值变化,对失血性休克的诊断和判断是否存在继续失血有参考价值。②电解质监测与肾功能监测:对了解病情变化和指导治疗十分重要。③凝血功能监测:常规凝血功能监测包括血小板计数、凝血酶原时间(PT)、活化部分凝血活酶时间(APTT)、国际标准化比值(international
normalized
ratio,INR)和D-二聚体等。若PT和(或)APTT延长至正常值的1.5倍,即应考虑凝血功能障碍。

\subsection{治疗}

包括原发病治疗(止血)和纠正休克(补充血容量)两个方面。原发病的有效治疗是失血性休克抢救成功的基础。创伤后引起大出血,尤其是难以控制的大出血,多在伤后1~2小时内死亡,因此伤后的“黄金1小时”内应以挽救生命为主。而“黄金1小时”的前10分钟尤为重要,多因血容量急剧减少而诱发心脏骤停,被称为“白金10分钟”。因此,对于急性失血性休克的最初阶段,应以适当扩充血容量以避免心脏骤停为主要目标;同时处理失血原因及相关并发症是赢取救治时间的关键。

\subsubsection{原发病治疗(止血)}

原发病的有效治疗是失血性休克抢救成功的基础。对于出血部位明确、存在活动性出血的休克患者,应尽快进行手术或介入止血。不去设法制止出血,只顾用输血来补充血量以纠正休克状态,是无效和错误的,治疗出血的首要任务是止血。在补充血容量的同时,应尽快进行止血,否则,在不断出血的情况下,尽管积极补液、输血,血容量仍不会恢复,休克也不会得到纠正。原则上是先采用暂时止血措施,待休克初步纠正后,再进行根本的止血措施;但是在难以用暂时止血的措施止血时,即应一面补充血容量,一面施行根本的止血措施。

采用何种止血方法,应根据出血来源而定:①四肢、头颅或身体表浅部位的较大出血,可先采用填塞、加压包扎暂时止血,待休克基本纠正后,再作手术处理。②内脏脏器如肝、脾破裂、宫外孕破裂等出血,则应尽早进行手术。③各种原因的上消化道出血、咯血,一般宜行内科保守治疗,必要时可考虑手术。

\subsubsection{补充血容量(液体复苏)}

液体复苏的目的是维持机体血流动力学的稳定,纠正代谢紊乱,恢复组织器官的正常灌注。

\hypertarget{text00061.htmlux5cux23CHP2-4-3-2-1}{}
(一) 液体复苏的发展

\paragraph{立即液体复苏}

经典的失血性休克液体复苏方法始于20世纪60年代,并被美国外科医师学院规范在创伤生命支持高级训练课程(ATLS)中,其主要内容是:一旦确认发生失血性休克,便立即和迅速地给予大容量输液,要求维持血压在正常范围内,直至出血被制止,这个过程被描述为“stay
and treat”(停下来抢救)。

\paragraph{延迟液体复苏}

近十多年的研究结果对上述经典的失血性休克液体复苏方法提出强力挑战。Bickell与Turner等的研究结果显示“早期不进行液体复苏比进行液体复苏的预后要好”。这种与经典的失血性休克液体复苏方法预料相悖的结果,其原因除了早期复苏可能延误决定性治疗(如外科手术)外,最主要原因是,在出血未被有效控制的情况下,大容量液体复苏和提升血压可以导致持续出血、血液稀释和体温下降,进而造成氧输送不足、凝血功能障碍和低体温,构成所谓“死亡三角”。对此,一些学者提出,在出血未被有效制止前,应该尽快将伤员转送到有手术条件的医院,复苏只在即将手术前才开始进行,这个策略被称作“scoop
and run”(卷起就跑)。

\paragraph{限制性液体复苏}

多数学者研究认为,对失血性休克是否需要早期复苏取决于失血的情况和伤员状态,为避免伤员在短期内死亡,对大出血和严重休克患者给予液体复苏是必要的,但同时也应该避免因快速和大量液体复苏所引发的问题。这样,提出了一个主张低度干预的复苏策略------“treat
and
run”(边治边走):要求用尽可能少的液体将血压维持在能够勉强保持组织灌注的较低水平,即“可允许性低血压”(permissive
hypotension)(收缩压80~90mmHg或MAP
60~80mmHg)。这种输液量因患者而宜,有活动性出血的休克患者,出血未控制之前不主张早期快速给予大量的液体进行复苏,在到达手术室彻底止血前,给予一定量的液体维持机体的基本需要,在相应的手术处理后再进行常规液体复苏,此即限制性液体复苏。研究证实:限制性液体复苏策略可降低病死率、减少再出血率及并发症。

限制性液体复苏策略已经开始向临床推荐。2003年美国和加拿大创伤学会提出了战场条件下休克的液体复苏方案:①评估是否需要复苏主要依据伤员的意识和脉搏状态;②如伤员意识清楚、桡动脉有力,不需给予任何输液;③对脉搏微弱和意识水平降低者应给予输液;④复苏应使收缩压维持在80~85mmHg;⑤复苏应给予小剂量的高渗晶体或人工胶体液。

但早期限制性液体复苏是否适合各类失血性休克,需维持多高的血压,可持续多长时间尚未有明确的结论。对于颅脑损伤患者,合适的灌注压是保证中枢神经组织氧供的关键。颅脑损伤后颅内压增高,此时若机体血压降低,则会因脑血流灌注不足而继发脑组织缺血性损害,进一步加重颅脑损伤。因此,一般认为对于合并颅脑损伤的严重失血性休克患者,宜早期输液以维持血压,必要时合用血管活性药物,将收缩压维持在正常水平,以保证脑灌注压,而不宜延迟复苏。允许性低血压在老年患者应谨慎使用,在有高血压病史的患者也应视为禁忌。

\hypertarget{text00061.htmlux5cux23CHP2-4-3-2-2}{}
(二) 液体复苏常用的液体种类

\paragraph{晶体溶液}

常用有生理盐水、复方氯化钠注射液(林格液)、乳酸钠林格注射液(平衡盐溶液,简称平衡液)、高渗盐溶液等。生理盐水含钠、氯浓度均为154mmol/L,电解质浓度308mmol/L;复方氯化钠注射液含钠、钾、钙、氯浓度分别为146mmol/L、4mmol/L、2.5mmol/L、155mmol/L,电解质浓度307.5mmol/L;乳酸钠林格注射液含钠、钾、钙、氯浓度分别为130mmol/L、4mmol/L、1.5mmol/L、109mmol/L,电解质浓度272.5mmol/L。乳酸钠林格注射液是液体治疗或复苏时最常用的晶体液。高渗盐溶液有7.5\%氯化钠注射液和3\%氯化钠注射液。高渗晶胶溶液有7.5\%氯化钠注射液+
6\%右旋糖酐溶液(HSD)和4.2\%氯化钠注射液+羟乙基淀粉(霍姆)等制剂。

失血性休克时有明显的细胞外液减少和细胞内液增加。生理盐水、复方氯化钠注射液、乳酸钠林格注射液主要分布于细胞外液,输注的晶体液约有25\%存留在血管内,而其余75\%则分布于血管外间隙。临床上输注1L等张晶体液后,血管内容量可增加100~200ml。复方氯化钠注射液中所含的乳酸在复苏过程中可以迅速代谢,不会影响动脉血乳酸测定。等渗电解质液无携氧功能,改善血流动力学效果差和时间短,用其单独纠正严重休克时其用量需为失液量的3~4倍才能维持循环,而且往往在输液结束后即有70\%~80\%漏到了血管外,大量输入等渗液有可能会进一步增加细胞与组织的水肿。

高渗盐溶液治疗机制是高渗和提高循环渗透压作用使细胞与组织脱水,细胞内和组织中的水分至血管中起到自体输液的作用,达到扩张有效循环血量的作用,使组织灌注好转及尿量增加,使血流动力学及全身情况获得明显改善。7.5\%氯化钠注射液4ml/kg输入休克机体后,扩充血浆容量约8ml/kg,等量的高渗晶胶溶液增加的血浆容量约14ml/kg,增加的血浆容量维持时间可达2小时。与等渗电解质液相比,其好处是用量小,能产生明显的血流动力学效果和改善组织水肿。对存在颅脑损伤的患者,由于可以很快升高MAP而不加剧脑水肿,因此高渗盐溶液可能有很好的前景。

休克患者应不给含糖液体,尤其是伴有中枢神经系统损伤的患者应禁止补充含糖液体,尽管补充含糖液体也可提升血压,但输注含糖液体后可引起和加重再灌注损伤。

\paragraph{胶体溶液}

临床上可供使用的有:

\hypertarget{text00061.htmlux5cux23CHP2-4-3-2-2-2-1}{}
(1) 右旋糖酐:

包括右旋糖酐40(低分子右旋糖酐)和右旋糖酐70(中分子右旋糖酐)。前者扩容效果较差,且持续时间短暂,有渗透性利尿作用,静脉滴注每次250~500ml,每日不超过20ml/kg;后者扩容效果与血浆相似,每克可结合水25ml左右,扩容效果可维持12小时,静脉滴注每次500ml,每日最大量不超过1000~1500ml。

\hypertarget{text00061.htmlux5cux23CHP2-4-3-2-2-2-2}{}
(2) 琥珀酰明胶(血定安,佳乐施):

为胶体性代血浆,扩容效能类似于4\%白蛋白。可增加血浆容量,使静脉回流及心输出量增加,改善微循环,增加血液的运氧能力;也能减轻组织水肿,有利于组织对氧的利用。其渗透性利尿作用有助于维持休克患者的肾功能。静脉输入的剂量和速度取决于患者的实际情况,严重急性失血时可在5~10分钟内输入500ml,直至低血容量症状缓解。大量输入时应确保维持Hct不低于0.25。

\hypertarget{text00061.htmlux5cux23CHP2-4-3-2-2-2-3}{}
(3) 羟乙基淀粉(淀粉代血浆,706代血浆):

输注1L能使循环容量增加700~800ml,至24小时后仍可维持40\%的最大扩容效果。静脉滴注500~1000ml。

\hypertarget{text00061.htmlux5cux23CHP2-4-3-2-2-2-4}{}
(4) 中分子羟乙基淀粉200/0.5(贺斯,haesteril):

为血容量扩充药。本品较低分子羟乙基淀粉有较高的分子量、独特的取代程度(克分子取代级MS
= 0.5)和取代方式(以C\textsubscript{2} 位置为主,C\textsubscript{2}
/C\textsubscript{6} =
5∶1),故有较强的容量扩充效应和较长的维持时间。用于预防和治疗各种原因引起的血容量不足和休克,如手术、创伤、感染、烧伤等。还有防止和堵塞毛细血管漏的作用,在毛细血管通透性增加的情况下使用本品,可减少白蛋白渗漏,减轻组织水肿,减少炎症介质产生,对危重患者更有利。有两种制剂,6\%中分子羟乙基淀粉200/0.5最大日剂量为33ml/kg,每小时最大滴速为20ml/kg;10\%中分子羟乙基淀粉200/0.5最大日剂量为20ml/kg,每小时最大滴速为20ml/kg。

\hypertarget{text00061.htmlux5cux23CHP2-4-3-2-2-2-5}{}
(5) 中分子羟乙基淀粉130/0.4(万汶,Voluven):

作用与中分子羟乙基淀粉200/0.5相似,但本品在此基础上作了进一步改良处理:适当减少分子量;降低取代级,下降约20\%(MS
= 0.4);改变了取代方式(C\textsubscript{2} /C\textsubscript{6} =
9∶1);分子量分布更加集中(减少了对血液流变学和凝血有不利影响的大分子比例,也减少了分子量低于肾阈值而快速排出小分子的比例)。这些改进使其安全性、耐受性、提高胶体渗透压的作用均有所增加。最大日剂量可用至33~50ml/kg。据患者需要可持续使用数日。中分子羟乙基淀粉130/0.4(万汶)是人血白蛋白最好的替代物,也是目前所有人工胶体溶液中最安全的药物。

\hypertarget{text00061.htmlux5cux23CHP2-4-3-2-2-2-6}{}
(6) 聚明胶肽(血代,海脉素):

扩容效能类似于琥珀酰明胶。一般500ml约在1小时内输入,急救时可在5~15分钟内输入500ml。一日最大剂量为2000ml。因钙离子浓度高达6.2mmol/L,对高钙血症、正在使用洋地黄治疗的患者禁用。

\hypertarget{text00061.htmlux5cux23CHP2-4-3-2-2-2-7}{}
(7) 白蛋白:

白蛋白是一种天然的血浆蛋白质,在正常人构成了血浆胶体渗透压的75\%~80\%。正常血浓度为35~50g/L。规格有5\%、10\%、15\%、20\%的注射液和5g、10g的冻干粉。5\%人血白蛋白溶液250ml的胶体渗透压为18~20mmHg/L,而25\%人血白蛋白溶液50ml的胶体渗透压为100mmHg/L。在复苏治疗初期,输注5\%人血白蛋白溶液1L血浆溶液增加500~1000ml。输注25\%人血白蛋白溶液100ml,如果体液能够从组织间隙进入血管内,1小时后可使血管内容量增加400~500ml。

白蛋白、新鲜或冻干血浆是液体复苏治疗时常用的胶体溶液,但其价格较昂贵,具有传播多种血道传染病的潜在危险,输入白蛋白后会发生毛细血管渗漏,将弥散至组织间质中,且不能自由地回至血管内,导致组织间质渗透压升高,组织水肿。

\hypertarget{text00061.htmlux5cux23CHP2-4-3-2-2-2-8}{}
(8) 高渗羟乙基淀粉200/0.5氯化钠注射液(贺苏):

本品为高渗胶体溶液,含7.2\%氯化钠与6\%羟乙基淀粉。由于本品的高渗作用(2464mOsm/L),液体可快速由组织间隙向血管内转移,血流动力学指标如血压、心输出量快速上升,并与输注量及输注速度有关。规格:250ml/袋。本品仅用于单次快速静脉输注或加压输注(推荐在2~5分钟内输完),用量约为4ml/kg体重(相当于体重60~70kg的患者使用250ml)。输注本品后,扩容维持时间较短,必须立即给予足够的液体治疗,以稳定血流动力学。

\hypertarget{text00061.htmlux5cux23CHP2-4-3-2-3}{}
(三) 复苏液体的选择

目前尚无足够的证据表明晶体液与胶体溶液用于低血容量休克液体复苏的疗效与安全性方面有明显差异。合理的液体选择方式是:晶体液为开始复苏的首选及主要选择(Ⅱ类证据);胶体溶液可在对晶体液复苏反应满意时加用(Ⅲ类证据);从经济方面考虑,应优先使用非蛋白类胶体溶液(Ⅱ类证据)。液体复苏时晶体液与胶体溶液比例通常为3∶1。无论用晶体液或胶体溶液,也无论用量多少,必须维持Hct在0.25以上。

\subsubsection{输血与防治凝血功能障碍}

失血性休克时,在输注晶体液和代血浆的同时,常应用新鲜冰冻血浆(FFP)、冷沉淀、浓缩血小板悬液,以改善凝血功能。

\paragraph{浓缩红细胞}

当Hb降至70g/L时应考虑输血。对于有活动性出血的患者,老年人以及有心肌梗死风险者,Hb保持在较高水平更为合理。无活动性出血的患者每输注1单位(200ml全血)的红细胞其Hb升高约10g/L,Hct升高约3\%。24小时内输血>
10单位为大量输血。但大量输注红细胞时易导致凝血紊乱,应及时补充血小板和凝血因子等特殊成分。Holcomb等报道以1∶1∶1的比例输注血浆、血小板、红细胞对预后有利。与成分输血相比,新鲜全血含有更多的凝血因子、血小板、红细胞,能更有效地纠正贫血和改善凝血功能。

\paragraph{血小板}

血小板输注主要适用于血小板数量减少或功能异常伴有出血倾向的患者。急性失血患者的血小板应维持在50
× 10\textsuperscript{9}
/L以上;严重创伤和中枢神经系统损伤的患者,应在100 ×
10\textsuperscript{9}
/L以上。对大量输血后并发凝血异常的患者联合输注血小板和冷沉淀可显著改善止血效果。

\paragraph{新鲜冰冻血浆(FFP)}

输注FFP的目的是为了补充凝血因子的不足。FFP不仅可以迅速改善凝血功能,还可起到扩容,改善微循环的作用。PT或APTT大于正常值的1.5倍时,应输入FFP纠正凝血紊乱,FFP输入量10~15ml/kg。

\paragraph{冷沉淀}

内含凝血因子Ⅴ、Ⅷ、Ⅻ、纤维蛋白原等,适用于特定凝血因子缺乏所引起的疾病、肝移植围术期以及肝硬化食管静脉曲张等出血。还可用于预防大量输血后的出血倾向。要求凝血因子活性至少在正常值的20\%~30\%,<
20\%易发生出血。

\subsubsection{血管活性药与正性肌力药}

临床通常仅对于足够的液体复苏后仍存在低血压或者输液还未开始的严重低血压患者.才考虑应用血管活性药与正性肌力药。可选用多巴胺、间羟胺、多巴酚丁胺等。

\subsubsection{纠正代谢性酸中毒}

代谢性酸中毒的处理应着眼于病因处理、容量复苏等干预治疗,在组织灌注恢复过程中酸中毒状态可逐步纠正,过度的血液碱化使氧解离曲线左移,不利于组织供氧。因此,在失血性休克的治疗中,碳酸氢盐的治疗只用于紧急情况或pH
< 7.20时,不主张常规使用。

\subsubsection{早期恰当使用止血药物}

严重出血无疑会导致凝血功能的异常,这就为抗纤溶药物的使用提供了可能。最近一项高质量的临床试验揭示创伤后使用抗纤溶药物氨甲环酸可有效减少出血并改善预后,而血管栓塞事件并不高于安慰剂组。所以,对于各种原因所致严重出血,建议早期使用氨甲环酸负荷量1g静脉注射,10分钟后再予氨甲环酸1g持续静滴8小时。

重组Ⅶ因子(rFⅦa)是一个很有前景的药物。当创伤患者有难以控制的出血,其纤维蛋白原≤0.5g/L,血小板≤50
× 10\textsuperscript{9} /L,pH≤7.2时,可以考虑使用rFⅦa。

\subsubsection{注意体温监测、防治低体温}

低体温(<
35℃)可影响血小板的功能、降低凝血因子的活性、影响纤维蛋白的形成。因此,要从现场复苏开始就给予重视,其中控制和减少出血是关键。要去除患者身上潮湿的衣物,减少非损伤部位的暴露,使用毛毯、加热毯或睡袋包裹伤员,转送与救治途中(急诊室、手术室与ICU)保温,液体或血液制品使用前进行加热等,以维持患者体温正常。但是,对入院时GCS评分4~7分的失血性休克合并颅脑损伤患者能从控制性降温中获益,应在外伤后尽早开始实施,并予以维持。

\subsubsection{复苏终点与预后评估指标}

对于低血容量休克的复苏治疗,人们常把神志改善、心率减慢、血压升高和尿量增加等传统临床指标作为复苏目标。然而,在机体应激反应和药物作用下,这些指标往往不能真实地反映休克时组织灌注的有效改善。有报道高达50\%~85\%的低血容量休克患者达到上述指标后,仍然存在组织低灌注,而这种状态的持续存在最终可能导致病死率增高;因此,目前不主张把这些传统指标的正常化作为复苏的终点。

血乳酸的水平、持续时间与低血容量休克患者的预后密切相关,持续高水平的血乳酸(>
4mmol/L)预示患者的预后不佳。血乳酸清除率比单纯的血乳酸值能更好地反映患者的预后。以乳酸清除率正常化作为复苏终点优于MAP和尿量,也优于以DO\textsubscript{2}
、VO\textsubscript{2}
和CI。以达到血乳酸浓度正常(≤2mmol/L)为标准,复苏的第一个24小时血乳酸浓度恢复正常(≤2mmol/L)极为关键,在此时间内血乳酸降至正常的患者,在病因消除的情况下,患者的存活率明显增加。因此,目前认为:动脉血乳酸恢复正常的时间和血乳酸清除率与低血容量休克患者的预后密切相关,复苏效果的评估应参考这两项指标。

碱缺失可反映全身组织酸中毒的程度。碱缺失可分为:轻度(−2~−5mmol/L),中度(<
−5~≥−15mmol/L),重度(<
−15mmol/L)。碱缺失水平与创伤后第一个24小时晶体液和血液补充量相关,碱缺失加重与进行性出血大多有关。对于碱缺失增加而似乎病情平稳的患者须细心检查有否进行性出血。多项研究表明,碱缺失水平与患者的预后密切相关,复苏时应动态监测碱缺失水平。

\protect\hypertarget{text00062.html}{}{}

\hypertarget{text00062.htmlux5cux23CHP2-4-4}{}
参 考 文 献

1. Anjaria DJ,Mohr AM,Deitch EA. Haemorrhagic shock therapy. Expert
Opin. Pharmacother,2008,9(6):901-911

2. Angele MK,Schneider CP,Chaudry IH. Bench-to-bedside review:Latest
results in hemorrhagic shock. Critical Care,2008,12(4):218-231

3.
中华医学会重症医学分会.低血容量休克复苏指南(2007).中国实用外科杂志,2007,27(5):581-587

4. Spaniol JR,Knight AR,Zebley JL,et al. Fluid resuscitation therapy
for hemorrhagic shock. Journal of Trauma Nursing,2007,14(3):152-160

5. 马俊勋
,赵茂,赵晓东.失血创伤性休克限制性液体复苏的最新进展.中华急诊医学杂志,2009,18(4):445-446

6. Lienhart HG,Lindner KH,Wenzel V. Developing alternative strategies
for the treatment of traumatic haemorrhagic shock. Curr Opin Crit
Care,2008,14:247-253

7. 张文武
,黄子通.失血性休克的处理策略.中华实用诊断与治疗杂志,2010,24(1):6

\protect\hypertarget{text00063.html}{}{}

\chapter{过敏性休克}

过敏性休克(anaphylactic
shock,anaphylaxis)是由于一般对人体无害的特异性变应原作用于过敏患者,导致以急性周围循环灌注不足为主的全身性速发变态反应。除引起休克的表现外,常伴有喉头水肿、气管痉挛、肺水肿等征象。低血压和喉头水肿是致死的主要原因。如不紧急处理,常导致死亡。

\subsection{病因与发病机制}

\paragraph{病因}

引起过敏性休克的病因或诱因变化多端,以药物与生物制品常见。

\hypertarget{text00063.htmlux5cux23CHP2-5-1-1-1}{}
(1) 异种(性)蛋白:

内分泌激素(胰岛素、加压素)、酶(糜蛋白酶、青霉素酶)、花粉浸液(豚草、树)、食物(蛋清、牛奶、坚果、海产品、巧克力)、抗血清、职业性接触的蛋白质(橡胶产品)、蜂类毒素等。

\hypertarget{text00063.htmlux5cux23CHP2-5-1-1-2}{}
(2) 常用药物:

如抗生素(青霉素、头孢菌素、两性霉素B)、局部麻醉药(普鲁卡因、利多卡因)、诊断性制剂(碘化X线造影剂)、职业性接触的化学制剂(乙烯氧化物)等。其中最常见者为青霉素过敏。青霉素不论肌肉注射、皮下注射、皮内注射、划痕试验、滴眼(耳、鼻)、阴道子宫颈上药、牙龈黏膜注射以及婴幼儿注射青霉素后的眼泪或尿液污染母体皮肤等均可发生过敏性休克。

\hypertarget{text00063.htmlux5cux23CHP2-5-1-1-3}{}
(3) 其他:

昆虫螫伤(蚂蚁、蜜蜂、大胡蜂、黄蜂等)、吸入物及接触物等。个别患者由某些非常特殊的因素造成,如对蟑螂的粪便、飞蛾的鳞毛、动物的皮屑、喷涂油漆等。

\paragraph{发病机制}

绝大多数过敏性休克是典型的Ⅰ型变态反应在全身多器官、尤其是循环系统的表现。

上述变应原进入机体,刺激机体淋巴细胞或浆细胞产生对变应原具有特异性的IgE抗体,吸附于组织的肥大细胞和血液中的嗜碱性粒细胞上,此时机体即已对变应原处于致敏状态。当患者再次接触变应原时,变应原的抗原决定簇迅速与相应抗体结合,使肥大细胞和嗜碱性粒细胞脱颗粒,释放大量的过敏性物质如组胺、5-羟色胺、慢反应物质(SRS-A)、缓激肽、血小板活化因子(PAF)、嗜酸性粒细胞趋化因子(ECFA)、乙酰胆碱等,使血管舒缩功能发生紊乱,毛细血管扩张通透性增加,血浆外渗,循环血量减少,致多系统脏器的循环灌注不足而引起休克;平滑肌收缩与腺体分泌增加,导致呼吸道、消化道症状,加重休克。有些药物之间有交叉反应可能,例如对青霉素过敏的患者,对链霉素也可发生过敏。少数患者初次应用抗生素或其他药物也会发生过敏性休克,此可能与真菌感染、空气或食物中含有过敏物质有关。

在输血、血浆或免疫球蛋白的过程中,偶然也可见到速发型的过敏性休克,它们的病因有三:①供血者的特异性IgE与受者正在接受治疗的药物(如青霉素G)起反应。②选择性IgA缺乏者多次输注含IgA血制品后,可产生抗IgA的IgG类抗体。当再次注射含IgA的制品时,有可能发生IgA-抗IgA抗体免疫复合物,发生Ⅲ型变态反应引起的过敏性休克。③用于静脉滴注的丙种球蛋白(丙球)制剂中含有高分子量的丙球聚合物,可激活补体,产生C3a、C4a、C5a等过敏毒素;继而活化肥大细胞,产生过敏性休克。少数患者在应用药物如鸦片酊、右旋糖酐、电离度高的X线造影剂或抗生素(如多黏菌素B)后,主要通过致肥大细胞脱颗粒作用,也会发生过敏性休克的临床表现。人们将不存在变应原与抗体反应的,仅通过非免疫机制而发生的过敏性休克称之为过敏样反应(anaphylactoid
reaction)。但其治疗是相似的。

\subsection{诊断}

\hypertarget{text00063.htmlux5cux23CHP2-5-2-1}{}
(一) 临床表现特点

患者接触变应原后迅速发病
。按症状出现距变应原进入的时间不同,可分为两型:①急发型过敏性休克:休克出现于变应原接触后0.5小时之内,约占80\%~90\%,多见于药物注射、昆虫螫伤或抗原吸入等途径。此型往往病情紧急,来势凶猛,预后较差。如青霉素过敏性休克常呈闪电样发作,出现在给药后即刻或5分钟内。②缓发型过敏性休克:休克出现于变应原接触后0.5小时以上,长者可达24小时以上,约占10\%~20\%。多见于服药过敏、食物或接触物过敏。此型病情相对较轻,预后亦较好。

过敏性休克有两大特点,一是有休克表现即血压急剧下降到80/50mmHg以下,患者出现意识障碍;二是在休克出现之前或同时,常有一些与过敏相关的症状。主要表现有:①由喉头或支气管水肿与痉挛引起的呼吸道阻塞症状:是本症最多见的表现,也是最重要的死因。患者出现喉头堵塞感、胸闷、气急、呼吸困难、窒息感、发绀等;②循环衰竭症状:如心悸、苍白、出汗、脉速而弱、四肢厥冷、血压下降与休克等。有冠心病背景者在发生本症时由于血浆的浓缩和血压的下降,常易伴发心肌梗死;③神经系统症状:如头晕、乏力、眼花、神志淡漠或烦躁不安、大小便失禁、抽搐、昏迷等;④消化道症状:如恶心、呕吐、食管梗阻感、腹胀、肠鸣、腹绞痛或腹泻等;⑤皮肤黏膜症状:往往是过敏性休克最早且最常出现的征兆,包括一过性的皮肤潮红、周围皮痒,口唇、舌部及四肢末梢麻木感,继之出现各种皮疹,重者可发生血管神经性水肿。还可出现喷嚏、水样鼻涕、刺激性咳嗽、声音嘶哑等。

\hypertarget{text00063.htmlux5cux23CHP2-5-2-2}{}
(二) 辅助检查

过敏性休克的诊断与治疗一般不需影像学检查等辅助检查。除常规心电图检查外,辅助检查主要用于评估反应的严重程度或在诊断不详时用于支持诊断或鉴别诊断。

1.血常规检查。

2.血液生化指标
测定血电解质(电解质异常可导致休克或由休克引起)、肝肾功能、淀粉酶、心肌酶谱、凝血功能、血乳酸等。

3.氧合情况
动脉血气或混合静脉血气分析(测量氧合、通气、酸碱状态),血氧饱和度监测等。

4.尿液分析与监测。

5.其他检查 床边X线检查、床边B超和超声心动图等检查。

\hypertarget{text00063.htmlux5cux23CHP2-5-2-3}{}
(三) 诊断注意事项

1.本病发生很快
,必须及时做出诊断。凡在接受(尤其是注射)抗原性物质或某种药物,或蜂类叮咬后立即发生全身反应,而又难以药品本身的药理作用解释时,就应马上考虑到本病的可能。

2.过敏性休克的诊断不依赖于实验室检查和特殊检查
,根据病情有明确用药史或接触变应原史,迅速发生上述的特征性临床表现,即可作出过敏性休克的诊断。但在诊断时应注意除外以下情况:

(1) 迷走血管性昏厥(或称迷走血管性虚脱,vasovagal
collapse):多发生在注射后,尤其患者有发热、失水或低血糖倾向时。患者常呈面色苍白、恶心、出冷汗,继而可昏厥,很易被误诊为过敏性休克。但此症无瘙痒或皮疹,昏厥经平卧后立即好转,血压虽低但脉搏缓慢,这些与过敏性休克不同。迷走血管性昏厥可用阿托品类药物治疗。

(2) 遗传性血管性水肿(hereditary
angioedema):这是一种由常染色体遗传的缺乏补体C\textsubscript{1}
酯酶抑制物的疾病。患者可在一些非特异性因素(例如感染、创伤等)刺激下突然发病,表现为皮肤和呼吸道黏膜的血管性水肿。由于气道的堵塞,患者也常有喘鸣、气急和极度呼吸困难等,与过敏性休克颇为相似。但本症起病较慢,不少患者有家族史或自幼发作史,发病时通常无血压下降,也无荨麻疹等,据此可与过敏性休克相鉴别。如果有药,血管性水肿可用C\textsubscript{1}
酯酶抑制因子替代治疗,否则,可用新鲜冰冻血浆治疗。

\subsection{治疗}

一旦出现过敏性休克,应立即就地抢救。

\hypertarget{text00063.htmlux5cux23CHP2-5-3-1}{}
(一) 一般处理

1.立即脱离或停止进入可疑的过敏物质
。如过敏性休克发生于药物注射之中,应立即停止注射,并可在药物注射部位之近心端扎止血带,视病情需要每15~20分钟放松止血带一次防止组织缺血性坏死。如属其他变应原所致,应将患者撤离致敏环境或移去可疑变应原。

2.即刻使患者取平卧位,松解领裤等扣带。如患者有呼吸困难,上半身可适当抬高;如意识丧失,应将头部置于侧位,抬起下颌,以防舌根后坠堵塞气道;清除口、鼻、咽、气管分泌物,畅通气道,面罩或鼻导管吸氧(高流量)。严重喉头水肿有时需行气管切开术;严重而又未能缓解的气管痉挛,有时需气管插管和辅助呼吸。对进行性声音嘶哑、舌水肿、喘鸣、口咽肿胀的患者推荐早期选择性插管。

3.对神志 、血压、呼吸、心率和经皮血氧饱和度等生命体征进行密切监测。

\hypertarget{text00063.htmlux5cux23CHP2-5-3-2}{}
(二) 药物治疗

1.肾上腺素
立即肌内注射0.1\%肾上腺素0.3~0.5ml,小儿每次0.02~0.025ml/kg。由药物引起者最好在原来注射药物的部位注射,以减缓药物吸收。如需要,可每隔15~20分钟重复1次。皮下注射的吸收和达到最大血浆浓度的时间均很长,并且因休克的存在而明显延缓,故抢救过敏性休克时,主张肌肉注射肾上腺素。如第一次注射后即时未见好转,或严重病例,可用肌注量的1/2~2/3稀释于50\%葡萄糖液40ml中静脉注射。肾上腺素能通过α受体效应使外周小血管收缩,恢复血管的张力和有效血容量;同时还能通过β受体效应缓解支气管痉挛,阻断肥大细胞和嗜碱性粒细胞炎性介质释放,是救治本症的首选药物。如呼吸、心跳停止,立即行心肺复苏术。一般经过1~2次肾上腺素注射,多数患者休克症状在0.5小时内均可逐渐恢复。

对链霉素引起的过敏性休克,有学者认为应首选钙剂,可用10\%葡萄糖酸钙或5\%溴化钙10~20ml稀释于25\%~50\%葡萄糖液20~40ml中缓慢静注;0.5小时后如症状未完全缓解,可再给药1次。

2.立即为患者建立静脉通道(最好两条),用地塞米松10~20mg或氢化可的松300~500mg或甲泼尼龙120~240mg加入5\%~10\%葡萄糖液500ml中静滴,或先用地塞米松5~10mg静注后,继以静滴。糖皮质激素对速发相反应无明显的治疗效果,但可以阻止迟发相过敏反应的发生。因严重支气管痉挛致呼吸困难者,可用氨茶碱0.25g稀释入25\%葡萄糖液20~40ml中缓慢静注。

3.补充血容量
过敏性休克中的低血压常是血管扩张和毛细血管液体渗漏所致。对此,除使用肾上腺素等缩血管药物外,必需补充血容量以维持组织灌注。宜选用平衡盐液,一般先输入500~1000ml,以后酌情补液。注意输液速度不宜过快、过多,以免诱发肺水肿。

4.应用升压药
经上述处理后,血压仍低者,应给予升压药。常用多巴胺20~40mg静注或肌注,或用较大剂量加入液体中静滴;或用去甲肾上腺素1~2mg加入生理盐水250ml中静脉滴注。

5.加用抗组胺药物
如异丙嗪25~50mg肌注或静滴,或苯海拉明20~40mg肌注,或H\textsubscript{2}
受体阻滞剂(如西咪替丁300mg口服、肌注或静滴)等。

6.吸入 β肾上腺素能药
如有明显支气管痉挛,可以喷雾吸入0.5\%沙丁胺醇溶液0.5ml,以缓解喘息症状。吸入沙丁胺醇对由于使用β受体阻滞剂所致的支气管痉挛特别有效。注意:一些发生濒死哮喘的过敏反应患者,应该接受重复剂量的支气管扩张剂而不是肾上腺素。

7.胰高血糖素的使用
胰高血糖素有不依赖于β受体的变力性、变时性和血管效应。胰高血糖素也可引起内源性儿茶酚胺的释放。用β受体阻断剂的患者在治疗过敏性休克心血管效应时肾上腺素和其他肾上腺素能药物的效果可能较差,这些患者胰高血糖素可能有效。此时,除使用较大剂量肾上腺素外,还应使用胰高血糖素,1~10mg静脉或肌肉注射(代表性用法是1~2mg,每5分钟一次)。患者过量使用β受体阻断剂时建议使用较大剂量。

\hypertarget{text00063.htmlux5cux23CHP2-5-3-3}{}
(三) 防治并发症

过敏性休克可并发肺水肿
、脑水肿、心跳骤停或代谢性酸中毒等,应予以积极治疗。参见有关章节。

休克改善后,如血压仍有波动者,可口服麻黄碱25mg,每日3次;如患者有血管神经性水肿、风团或其他皮肤损害者,可口服泼尼松20~30mg/d,抗组胺类药物如氯苯那敏(扑尔敏,4mg,每天3次)、阿司咪唑(息斯敏,10mg,每天1次)等。同时对患者应密切观察24小时,以防过敏性休克再次发生。

\hypertarget{text00063.htmlux5cux23CHP2-5-3-4}{}
(四) 病因治疗

过敏性休克往往可以预防
,最好的病因治疗是周密的预防,杜绝过敏性休克的发生。因此,过敏性休克的特异性病因诊断对本症的防治具有重要意义,进行变应原测验应该:①在休克解除后;②在停用抗休克及抗过敏药物后;③如作皮肤试验,最好先由斑贴、挑刺等试验开始,严格控制剂量,并准备好必要的抗休克药物。应注意:少数皮试阴性患者仍有发生本症的可能。曾对叮咬、刺螫、食物或其他不可避免的因素产生严重过敏反应的患者有使用肾上腺素自动注射器的指征,它可以做成包括口服抗组胺药的抗过敏急救盒。

\protect\hypertarget{text00064.html}{}{}

\hypertarget{text00064.htmlux5cux23CHP2-5-4}{}
参 考 文 献

1. 陈灏珠 ,林果为.实用内科学.第13版.北京:人民卫生出版社,2009

2. 徐腾达,于学忠.现代急症诊断治疗学.北京:中国协和医科大学出版社,2007

3. Simon GA,Brown MBBS. The pathophysiology of shock in anaphylaxis.
Immunol Allergy Clin N Am,2007,27:165-175

\protect\hypertarget{text00065.html}{}{}

\chapter{神经源性休克}

神经源性休克(neurogenic
shock)是指由于强烈的神经刺激,如创伤、剧烈疼痛等引起某些血管活性物质如缓激肽、5-羟色胺等释放增加,导致周围血管扩张,大量血液淤滞于扩张的血管中,有效循环血量突然减少而引起的休克。

\subsection{病因与发病机制}

\paragraph{病因}

①严重创伤、剧烈疼痛刺激:如胸腹腔或心包穿刺时,周围血管扩张,大量血液淤积于扩张的微循环血管内,反射性的血管舒缩中枢被抑制,导致有效血容量突然减少而引起休克。②药物:许多药物可破坏循环反射功能而引起低血压休克如氯丙嗪、降血压药物(神经节阻滞剂、肾上腺素能神经元阻滞剂和肾上腺受体拮抗剂)以及麻醉药物(包括全麻、腰麻、硬膜外麻醉),均可阻断自主神经,使周围血管扩张,血液淤积,发生低血压休克。尤其当患者已有循环功能不足因素存在时,应用上述药物更易出现低血压。

\paragraph{发病机制}

强烈的神经刺激,如创伤、剧烈疼痛等引起某些血管活性物质如缓激肽、5-羟色胺等释放增加,导致周围血管扩张,大量血液淤滞于扩张的血管中,有效循环血量突然减少而引起的休克。此类休克也常发生在脑损伤或缺血、深度麻醉、脊髓高位麻醉或脊髓损伤交感神经传出通路被阻断时。在正常情况下,血管运动中枢不断发出冲动,传出的交感缩血管纤维到达全身小血管,维持血管一定的张力。当血管运动中枢发生抑制或传出的缩血管纤维被阻断时,小血管张力丧失,血管扩张,外周阻力降低,大量血液聚集在血管床,回心血量减少,血压下降,出现休克。这种休克发生常极为迅速,具有很快逆转的倾向,大多数情况下不发生危及生命的、持续严重的组织灌流不足。

\subsection{诊断}

\paragraph{临床表现特点}

在正常状态下,周围血管接受神经系统血管舒缩中枢的调节,维持一定的紧张度,而保证全身的血液供应。在强烈的神经刺激,如创伤、剧烈疼痛等时,可引起反射性血管舒缩中枢抑制,导致周围血管扩张,血液大量淤积于扩张的微循环血管内,有效循环血容量突然减少而引起休克。临床主要表现有:①循环衰竭症状:如心悸、面色苍白、出汗、脉速而弱、四肢厥冷、血压下降与休克等。②神经系统症状:如头晕、乏力、眼花、神志淡漠或烦躁不安、大小便失禁、抽搐、昏迷等。其他症状如恶心、呕吐、四肢湿冷、黏膜苍白或发绀等。

\paragraph{辅助检查}

同过敏性休克一样,神经源性休克的诊断一般不需影像学检查等辅助检查。除常规心电图检查外,辅助检查主要用于评估反应的严重程度或在诊断不详时用于支持诊断或鉴别诊断。

\paragraph{诊断注意事项}

(1)
正如上述,神经源性休克常发生于强烈的神经刺激时。因此,在临床上存在强烈的神经刺激如剧痛、各种穿刺操作时,出现上述的临床表现,又难以用原发病解释时,就应马上考虑到本病的可能。

(2)
神经源性休克的诊断主要依赖于两点:①病史:有引起神经源性休克的病因,如剧烈疼痛与精神创伤、药物(麻醉药、安眠药)、麻醉(脊髓、腰麻、硬膜外麻)、穿刺(脑室、胸腔、心包、腹腔)等。②有休克的临床表现。

(3)
神经源性休克在诊断时应注意与两种情况相鉴别:①迷走血管性昏厥:多发生在注射后,尤其患者有发热、失水或低血糖倾向时。患者常呈面色苍白、恶心、出冷汗,继而可昏厥,有时被误诊为神经源性休克。迷走血管性昏厥经平卧后立即好转,血压虽低但脉搏缓慢。迷走血管性昏厥可用阿托品类药物治疗。②过敏性休克:与神经源性休克的区别主要有两点:①有接触或使用变应原病史;②存在与过敏相关的伴发表现:全身或局部荨麻疹或其他皮疹,伴喉头水肿并出现吸气性呼吸困难。

\subsection{治疗}

\paragraph{一般处理}

(1)
体位:患者应保持安静,取平卧位,除去枕头,下肢抬高15°~30°,使其处于头低脚高的休克体位,以增加回心血量,增加脑部血供。如有意识丧失,应将头部置于侧位,抬起下颏,以防舌根后坠堵塞气道。

(2)
吸氧:畅通呼吸道,充分供氧。应用鼻塞或面罩吸氧,保证患者各脏器充分的氧供。

(3) 对神志、血压、呼吸、心率和经皮血氧饱和度等生命体征进行密切监测。

\paragraph{药物治疗}

\hypertarget{text00065.htmlux5cux23CHP2-6-3-2-1}{}
(1) 肾上腺素:

是首选药物。立即肌内注射0.1\%肾上腺素0.3~0.5ml,小儿每次0.02~0.025ml/kg。严重病例可以将肾上腺素稀释于50\%葡萄糖液40ml中静注,也可用1~2mg加入5\%葡萄糖液100~200ml中静滴。

\hypertarget{text00065.htmlux5cux23CHP2-6-3-2-2}{}
(2) 补充血容量:

迅速建立静脉通道,补充血容量,常用的晶体液为生理盐水、平衡盐液、5\%葡萄糖氯化钠溶液等。一般先快速静滴500
~1000ml,以后根据血压情况再给。

\hypertarget{text00065.htmlux5cux23CHP2-6-3-2-3}{}
(3) 应用镇痛、镇静药物:

由于剧烈疼痛引起的休克需要应用镇痛药物,可用吗啡5~10mg静脉入壶或肌注,哌替啶(度冷丁)50~100mg肌注;情绪紧张患者应给予镇静药物如地西泮(安定)10mg肌注,或苯巴比妥钠0.1~0.2g肌注。

\hypertarget{text00065.htmlux5cux23CHP2-6-3-2-4}{}
(4) 糖皮质激素:

该药能改善微循环,提高机体的应激能力。可给予地塞米松5~10mg静脉入壶或氢化可的松200~300mg溶于5\%葡萄糖液500ml中静滴。因严重支气管痉挛致呼吸困难者,可用氨茶碱0.25g稀释入25\%葡萄糖液20~40ml中缓慢静注。

\hypertarget{text00065.htmlux5cux23CHP2-6-3-2-5}{}
(5) 应用升压药:

经上述处理后血压仍低者,应给予缩血管药。一般常用多巴胺或间羟胺20~60mg加入100~
200ml溶液中静滴,待休克好转后,逐渐减量以至停用。

\paragraph{对因治疗}

根据导致患者神经源性休克的不同病因进行相应处理。例如,由胸腔、腹腔或心包穿刺引起者立即停止穿刺。

\paragraph{防治并发症}

神经源性休克可并发脑水肿、心跳骤停或代谢性酸中毒等,应予以积极治疗。参见有关章节。

\protect\hypertarget{text00066.html}{}{}

\hypertarget{text00066.htmlux5cux23CHP2-6-4}{}
参 考 文 献

1. 陈灏珠 ,林果为.实用内科学.第13版.北京:人民卫生出版社,2009

2.
徐腾达,于学忠.现代急症诊断治疗学.北京:中国协和医科大学出版社,2007:175

3. 孟庆义.急诊内科诊疗精要.北京:军事医学科学出版社,

2006:139

\protect\hypertarget{text00067.html}{}{}


\part{常见的心脏病、电解质紊乱及抗心律失常药物所致的心电图改变}

本篇根据先天性心脏病、后天性心脏病、各类心肌病、电解质紊乱、药物影响顺序进行编写。主要讲述先天性、后天性心脏病的病理生理改变与心电图表现的相关性及其特征,以及电解质紊乱、药物对心脏的影响所产生的心电图改变,并将心脏血管支配部位、心脏电生理特性等基础知识有机地融进各个章节之中,共6章。

\protect\hypertarget{text00049.html}{}{}

\protect\hypertarget{text00049.htmlux5cux23chapter49}{}{}

\chapter{常见的先天性心脏病的心电图改变}

\protect\hypertarget{text00049.htmlux5cux23subid580}{}{}

\section{法洛四联症}

1.病理生理改变

法洛四联症包括肺动脉狭窄、室间隔缺损、主动脉骑跨及右心室肥大。其血流动力学改变主要是由肺动脉狭窄引起右心室收缩期负荷过重,导致右心室肥厚(主要表现为右心室肥厚、肺动脉圆锥显著膨隆、室上嵴增厚及右心室内乳头肌和肉柱显著增粗)和右心房肥大,晚期可伴有右心室腔扩张。肥厚的右心室可将左心室推向左后方,右心暴露面增多,占据心尖部。通常以肺动脉瓣下2cm处右心室前壁肌层厚度>0.5cm(正常约0.3~0.4cm)作为右心室肥厚的诊断标准。

2.心电图特征

(1)先心型P波:Ⅱ、Ⅲ、aVF、V\textsubscript{1} 、V\textsubscript{2}
等导联P波高尖,肢体导联电压≥0.25mV,V\textsubscript{1}
、V\textsubscript{2} 导联电压≥0.15mV,V\textsubscript{5}
导联电压≥0.2mV,P波时间大多正常。

(2)电轴右偏:常>+110°。

(3)aVR导联QRS波群呈qR或QR型,q(Q)/R<1,R波电压>0.5mV。

(4)V\textsubscript{1}
导联QRS波形依据肥厚程度可呈qR型、R型、Rs型,R/s>1及rsR′型,R波电压显著增高等。

(5)V\textsubscript{5} 、V\textsubscript{6}
导联QRS波群可呈RS型,R/S<1或rS型等。

(6)V\textsubscript{1} 、V\textsubscript{2}
导联可伴有ST段压低、T波倒置(图\ref{fig41-1})。

\begin{figure}[!htbp]
 \centering
 \includegraphics[width=5.5625in,height=2.09375in]{./images/Image00683.jpg}
 \captionsetup{justification=centering}
 \caption{女性,16岁,法洛四联症患者,出现右心房及右心室肥大、下壁异常Q波}
 \label{fig41-1}
  \end{figure} 

\protect\hypertarget{text00049.htmlux5cux23subid581}{}{}

\section{房间隔缺损}

房间隔缺损为最常见的先天性心脏病,包括继发孔缺损型、原发孔缺损型及高位缺损型等。

1.病理生理改变

由于右心室同时接受上、下腔静脉和左心房流入右心房的血液,导致右心室舒张期负荷过重,出现右心房、右心室肥大及扩张。原发孔缺损型房间隔缺损,大多形成部分或完全性房室通道,左束支明显向后下移位,导致左前分支相对发育不良,易出现一度房室传导阻滞及电轴左偏;若伴有二尖瓣关闭不全,则可出现左心室肥大。

2.继发孔缺损型的心电图特征(图\ref{fig41-2})

\begin{figure}[!htbp]
 \centering
 \includegraphics[width=4.79167in,height=1.10417in]{./images/Image00684.jpg}
 \captionsetup{justification=centering}
 \caption{女性,44岁,先心病、继发孔型房间隔缺损。心电图显示右心房肥大、完全性右束支阻滞、提示合并右心室肥大、钩型R波(引自陈琪)}
 \label{fig41-2}
  \end{figure} 

(1)右心房肥大:Ⅱ、Ⅲ、aVF导联P波高尖,电压≥0.25mV;V\textsubscript{1}
、V\textsubscript{2} 导联正相波高尖,电压≥0.15mV。

(2)电轴右偏:一般为轻、中度右偏。右偏越严重,表明右心室肥大越明显。

(3)不完全性或完全性右束支阻滞图形:V\textsubscript{1}
导联QRS波群呈rsR′型或rsr′型,时间多<0.12s,与右心室流出道、室上嵴及圆锥部肥厚有关,有一定的诊断意义。

(4)右心室肥大。

(5)出现钩形R波:Ⅱ、Ⅲ、aVF导联QRS波群起始后80ms内,即R波升肢或顶峰部位出现切迹呈钩形。

(6)可出现一度房室传导阻滞及各种的房性心律失常。

3.原发孔缺损型的心电图特征(图\ref{fig41-3})

\begin{figure}[!htbp]
 \centering
 \includegraphics[width=5.78125in,height=0.63542in]{./images/Image00685.jpg}
 \captionsetup{justification=centering}
 \caption{男性,17岁,先心病、原发孔型房间隔缺损伴二尖瓣瓣裂。心电图显示右心房及右心室肥大、电轴-30°、V\textsubscript{1}Ptf值明显增大(提示左心房负荷过重)、一度房室传导阻滞(P-R间期0.24s)、提示不完全性右束支阻滞、侧壁轻度T波改变(Ⅲ、V\textsubscript{3}~V\textsubscript{5} 导联定准电压均为0.5mV)}
 \label{fig41-3}
  \end{figure} 


(1)右心房肥大。

(2)电轴左偏:约-30°,类似左前分支阻滞图形,而有别于继发孔缺损型的心电图改变。

(3)一度房室传导阻滞:P-R间期延长(≥0.24s)。

(4)不完全性或完全性右束支阻滞图形。

(5)右心室肥大或合并左心室肥大。

(6)房性心律失常。

\protect\hypertarget{text00049.htmlux5cux23subid582}{}{}

\section{室间隔缺损}

1.病理生理改变

室间隔缺损包括膜部缺损和肌部缺损两种。当缺损较小,左向右分流量少时,血流动力学变化不明显,心电图可正常;当缺损较大,左向右分流量较大时,导致左、右心室舒张期负荷过重,出现左、右心室肥大或以左心室肥大为主;当左向右分流量很大时,出现轻、中度肺动脉高压,产生双心室肥大;出现重度肺动脉高压时,右心室收缩期负荷过重,导致右心室显著肥大和右心房负荷过重及肥大,此时分流量反而减少,甚至出现逆向分流。

2.心电图特征(图\ref{fig41-4})

\begin{figure}[!htbp]
 \centering
 \includegraphics[width=5.38542in,height=4.65625in]{./images/Image00686.jpg}
 \captionsetup{justification=centering}
 \caption{男性,28岁,先心病、室间隔缺损。显示右心房肥大、双心室肥大、下壁异常Q波、完全性右束支阻滞、ST-T改变}
 \label{fig41-4}
  \end{figure} 

(1)心电图正常。

(2)单纯左心室肥大:呈左心室舒张期负荷过重图形,表现为V\textsubscript{5}
、V\textsubscript{6} 导联R波电压增高,ST段轻度抬高,T波高耸。

(3)双心室肥大。

(4)右心室肥大。

(5)P波高大。

(6)可出现各种心律失常

\protect\hypertarget{text00049.htmlux5cux23subid583}{}{}

\section{动脉导管未闭}

1.病理生理改变

未闭的动脉导管位于主动脉峡部和左肺动脉根部,血流从主动脉分流入肺动脉,使肺循环血流量增多,回流至左心房和左心室血流增加,导致左心室舒张期负荷过重,出现左心房、左心室肥大;当发生肺动脉压力增高时,分流量反而减少,出现右心室肥大。

2.心电图特征

(1)心电图正常:见于细小的动脉导管未闭,分流量不大,肺动脉压力不高。

(2)左心房、左心室肥大:P波增宽呈双峰切迹,V\textsubscript{1}
Ptf负值增大;QRS波群呈左心室舒张期负荷过重图形。见于中等大小的动脉导管未闭,肺动脉压力轻、中度增高者(>60mmHg)。

(3)双心房、双心室肥大:见于粗大的动脉导管未闭,肺动脉压力重度增高者(>90mmHg)。

(4)右心室肥大掩盖左心室肥大:当肺动脉压力长期重度增高时,右心室肥大更为明显,可掩盖左心室肥大的心电图改变。

\protect\hypertarget{text00049.htmlux5cux23subid584}{}{}

\section{三尖瓣下移畸形}

1.病理生理改变

三尖瓣下移畸形又称为Ebstein畸形,指部分或整个有效的三尖瓣环向下移位,使右心房增大,右心室缩小,出现右心室一部分心房化,存在三尖瓣返流,常伴有房间隔缺损。

2.心电图特征

(1)右心房扩大:Ⅱ、Ⅲ、aVF导联和V\textsubscript{1} 、V\textsubscript{2}
导联P波高尖,有学者称之为喜马拉雅P波(图\ref{fig41-5})。

\begin{figure}[!htbp]
 \centering
 \includegraphics[width=5.80208in,height=1.36458in]{./images/Image00687.jpg}
 \captionsetup{justification=centering}
 \caption{Ebstein畸形患者,出现右心房扩大、一度房室传导阻滞(P-R间期0.24s)、完全性右束支阻滞、前侧壁ST段改变、下壁T波改变(引自临床心电学杂志)}
 \label{fig41-5}
  \end{figure} 

(2)右束支阻滞和V\textsubscript{1} 、V\textsubscript{2}
导联r(R)波和s波低小,为本病特征性的心电图改变。

(3)约25\%患者存在B型预激综合征,出现B型预激与右束支阻滞图形并存的现象(图\ref{fig41-6})。

(4)常出现房室折返性心动过速等心律失常

\begin{figure}[!htbp]
 \centering
 \includegraphics[width=5.08333in,height=1.97917in]{./images/Image00688.jpg}
 \captionsetup{justification=centering}
 \caption{女性,51岁,Ebstein畸形。出现V\textsubscript{1}、V\textsubscript{2} 导联P波略高尖、B型预激综合征合并完全性右束支阻滞}
 \label{fig41-6}
  \end{figure} 


\protect\hypertarget{text00049.htmlux5cux23subid585}{}{}

\section{右位心}

广义的右位心包括镜像右位心、右旋心和心脏右移(图\ref{fig41-7})。通常所说的右位心仅指镜像右位心。

\begin{figure}[!htbp]
 \centering
 \includegraphics[width=5.58333in,height=1.54167in]{./images/Image00689.jpg}
 \captionsetup{justification=centering}
 \caption{心脏右位的解剖示意图(图A正常、图B镜像右位心、图C右旋心、图D心脏右移)}
 \label{fig41-7}
  \end{figure} 

1.镜像右位心

心脏位于右侧胸腔内,左右心房、心室的关系发生反位,宛如正常心脏的镜中像,可伴有其他内脏的转位。心电图检查对镜像右位心的诊断具有确诊价值,但需排除左、右手的导联线反接。心电图具有以下5个特征:

(1)Ⅰ导联P、QRS、T波均倒置,为正常Ⅰ导联图形的倒镜像改变。

(2)Ⅱ导联和Ⅲ导联、aVR导联与aVL导联的图形互换,而aVF导联图形不变。

(3)胸前导联V\textsubscript{1} ~V\textsubscript{6}
的R波振幅逐渐减低,而S波逐渐加深。

(4)加做右胸导联V\textsubscript{3} R、V\textsubscript{4}
R、V\textsubscript{5} R、V\textsubscript{6}
R,其R波振幅逐渐增高或者以V\textsubscript{4} R、V\textsubscript{5}
R导联R波振幅最高。

(5)将左、右手的导联线反接,用V\textsubscript{2} 、V\textsubscript{1}
、V\textsubscript{3} R、V\textsubscript{4} R、V\textsubscript{5}
R、V\textsubscript{6} R导联分别代表V\textsubscript{1}
、V\textsubscript{2} 、V\textsubscript{3} 、V\textsubscript{4}
、V\textsubscript{5} 、V\textsubscript{6}
导联,即可得到一幅和正常人完全相同的心电图波形(图\ref{fig41-8}、图\ref{fig41-9})。

\begin{figure}[!htbp]
 \centering
 \includegraphics[width=5.1875in,height=2.58333in]{./images/Image00690.jpg}
 \captionsetup{justification=centering}
 \caption{女性,36岁,右位心、风心病、二尖瓣狭窄伴关闭不全。肢体导联心电图显示右位心特点,而胸前导联则显示正常情况下QRS-T波群特点,除此以外,尚显示P波高大、左心室高电压}
 \label{fig41-8}
  \end{figure} 

\begin{figure}[!htbp]
 \centering
 \includegraphics[width=5.1875in,height=2.47917in]{./images/Image00691.jpg}
 \captionsetup{justification=centering}
 \caption{与图\ref{fig41-8}系同一患者,对左、右手导联线反接后及右胸导联进行记录。显示肢体导联P波极性正常,但形态与上图有所不同,电轴轻度右偏,R波高电压;右胸导联显示右心室肥大,结合临床病史,提示该患者存在双心房、双心室肥大}
 \label{fig41-9}
  \end{figure} 

2.右旋心

右旋心指心脏在发育过程中下降和左旋不良,甚至右旋,使心脏不同程度地移至右侧胸腔,心尖指向右前方,但左右心房、心室的解剖关系正常,常伴有心脏其他畸形,如房间隔缺损、室间隔缺损等,不伴有内脏的转位。心电图改变有以下3个特征:

(1)各肢体导联P波极性正常。

(2)Ⅰ导联QRS、T波均倒置,而Ⅱ、Ⅲ导联QRS、T波均为正向。

(3)V\textsubscript{1} ~V\textsubscript{3}
导联QRS波群振幅增高,且呈Rs型或qR型,V\textsubscript{5}
、V\textsubscript{6} 导联R波振幅降低,且常伴有T波倒置。

\protect\hypertarget{text00050.html}{}{}

\protect\hypertarget{text00050.htmlux5cux23chapter50}{}{}

\chapter{后天性心脏病的心电图改变}

\protect\hypertarget{text00050.htmlux5cux23subid586}{}{}

\section{冠心病}

冠心病可分为隐匿型、心绞痛型、心肌梗死型、心力衰竭和心律失常型及猝死型5种类型,但这5种类型可以同时出现。

\protect\hypertarget{text00050.htmlux5cux23subid587}{}{}

\subsection{隐匿型冠心病}

隐匿型冠心病亦称为无症状型冠心病。患者虽无临床症状,但静息时或运动试验后ST段呈缺血型压低、T波低平或倒置。

\protect\hypertarget{text00050.htmlux5cux23subid588}{}{}

\subsection{心绞痛型冠心病}

有发作性胸骨后疼痛,常为一过性心肌供血不足所致。由体力劳动、运动等其他增加心肌耗氧量情况下所诱发的短暂性胸痛发作,经休息或含服硝酸甘油后,疼痛迅速缓解者,称为劳累性心绞痛;若胸痛发作与心肌耗氧量增加无明显关系,则称为自发性心绞痛,这种胸痛一般持续时间较长,程度较重,不易被硝酸甘油所缓解,但心肌酶谱正常。一般分为稳定型心绞痛、不稳定型心绞痛、变异型心绞痛及混合型心绞痛4种类型。

1.稳定型心绞痛

(1)基本概念:稳定型心绞痛又称为典型心绞痛,在3个月内,心绞痛发作的诱因、次数、疼痛性质和程度及持续时间均无明显变化者。

(2)心电图特征:心绞痛发作时,立即出现下列一项或数项改变,症状缓解后,马上恢复原状:①缺血型ST段改变:缺血部位所对应的导联ST段呈水平型、下斜型压低≥0.1mV(图\ref{fig42-1});若原有ST段压低,则在原有基础上再下降≥0.1mV;若原有ST段抬高,则ST段可回复到正常或程度减轻,出现“伪善性”改变而易被误诊;有时ST段可呈水平型延长>0.16s。②T波改变:有ST段压低的导联会出现一过性T波低平、双相或倒置,甚至出现“冠状T波”。③一过性Q-T间期延长。④U波改变:左胸导联U波倒置,偶见U波振幅增高。⑤一过性心律失常:以室性早搏多见。⑥V\textsubscript{1}
Ptf负值增大。

\begin{figure}[!htbp]
 \centering
 \includegraphics[width=4.75in,height=2.90625in]{./images/Image00692.jpg}
 \captionsetup{justification=centering}
 \caption{男性,69岁,胸痛发作数分钟。显示窦性心动过速、左前分支阻滞、下壁及前侧壁ST段呈缺血型改变(压低0.1~0.4mV);冠状动脉造影显示右冠状动脉近乎全部阻塞、左前降肢95\%狭窄}
 \label{fig42-1}
  \end{figure} 

2.不稳定型心绞痛

(1)基本概念:不稳定型心绞痛是指近3个月内心绞痛发作的诱因有明显变化(活动耐量减少)、发作次数增加、疼痛性质改变和持续时间延长,是介于稳定型心绞痛与急性心肌梗死之间的过渡类型。

(2)类型:①进行性心绞痛:指同等程度劳累所诱发心绞痛发作次数、程度及持续时间进行性加重,又称为恶化型劳累性心绞痛;②新发的心绞痛:指近3个月内出现的心绞痛;③中间型心绞痛:指近1个月内病情恶化,疼痛剧烈,反复发作,硝酸甘油不能缓解,但心肌酶谱正常;④心肌梗死后心绞痛:指急性心肌梗死后1个月内发生的心绞痛。

(3)心电图特征:①R波振幅可突然降低或增高,与以前图形不相符合。②ST-T改变:可出现缺血性ST-T改变,表现为ST段呈缺血型压低、T波倒置;亦可出现损伤型ST-T改变,表现为ST段呈损伤型抬高、T波高耸。③一过性心律失常:以室性早搏多见。④左胸导联U波倒置。⑤如病情进一步发展而发生急性心肌梗死,其梗死部位与原不稳定型心绞痛发作时ST-T改变的导联所反映的部位相一致。

3.变异型心绞痛

(1)基本概念:变异型心绞痛是指心绞痛发作与心肌耗氧量增加无明显关系,主要由冠状动脉一过性痉挛引起急性心肌缺血、透壁性损伤,出现损伤性ST段抬高和T波高耸。该心绞痛发作往往无明确诱因,有定时发作倾向,以夜间、凌晨多见,发作时疼痛程度较重、持续时间较长,含服硝酸甘油不能缓解,而用钙离子拮抗剂防治效果好。属自发性心绞痛范畴。

(2)心电图特征:①ST段呈损伤型抬高:面对缺血区导联ST段抬高≥0.2mV,而对应导联ST段压低;若原有ST段压低,则可出现“伪善性”改变而易误诊。②T波高耸:ST段抬高导联T波直立高耸(图\ref{fig42-2}、图\ref{fig42-3});若原有T波倒置,则可出现T波直立或倒置程度减轻而呈“伪善性”改变。③出现急性损伤阻滞图形:其特征是QRS波群时间增宽、室壁激动时间延长及R波振幅增高和S波变浅。④一过性心律失常:若急性心肌缺血、透壁性损伤发生在前壁,则以室性心律失常多见;若发生在下壁,则以房室传导阻滞多见(图\ref{fig42-4})。⑤左胸导联U波倒置,偶见U波振幅增高。⑥一部分患者可出现QRS、ST、T等波段电交替现象。⑦疼痛缓解后,上述图形改变可恢复原状,若进一步发展为心肌梗死,则梗死部位与ST段抬高、T波高耸的导联相吻合。

\begin{figure}[!htbp]
 \centering
 \includegraphics[width=5.5625in,height=1.67708in]{./images/Image00693.jpg}
 \captionsetup{justification=centering}
 \caption{冠心病患者,模拟12导联动态心电图显示变异型心绞痛发作时出现窦性心动过缓、下壁及前间壁ST段损伤型抬高及T波高耸、QRS波群时间增宽(0.11s)}
 \label{fig42-2}
  \end{figure} 

\begin{figure}[!htbp]
 \centering
 \includegraphics[width=5.5625in,height=2.63542in]{./images/Image00694.jpg}
 \captionsetup{justification=centering}
 \caption{冠心病患者,变异型心绞痛发作时出现前间壁及前侧壁ST段损伤型抬高、前间壁及前壁T波高耸、左心室高电压}
 \label{fig42-3}
  \end{figure} 

\begin{figure}[!htbp]
 \centering
 \includegraphics[width=5.58333in,height=0.80208in]{./images/Image00695.jpg}
 \captionsetup{justification=centering}
 \caption{与图\ref{fig42-2}系同一患者,除了上述改变外,尚出现窦性停搏、房室交接性逸搏、室性早搏}
 \label{fig42-4}
  \end{figure} 

4.混合型心绞痛

指患者同时存在劳累型和自发型或变异型心绞痛,即心绞痛发作时,同时存在心肌耗氧量增加和冠状动脉供血减少这两种因素参与者。

(1)劳累型合并变异型心绞痛:早已确诊为劳累型心绞痛,但近来胸痛多在夜间、凌晨发作,含服硝酸甘油不能缓解。

(2)劳累型合并自发型心绞痛:劳累后或休息时均有心绞痛发作,心电图显示缺血部位相同,白天以劳累型为主,夜间为自发型发作。

\protect\hypertarget{text00050.htmlux5cux23subid589}{}{}

\subsection{心肌梗死型冠心病}

症状多严重,由冠状动脉闭塞引起心肌急性缺血性坏死所致,心电图大多数表现为异常Q波、ST段呈损伤型抬高、T波呈缺血型倒置,其中ST-T呈动态演变是急性心肌梗死特征性改变,尤其是对非穿透性心肌梗死具有诊断价值,详见第四十四章经典的心肌梗死及其进展。

\protect\hypertarget{text00050.htmlux5cux23subid590}{}{}

\subsection{心力衰竭和心律失常型冠心病}

表现为心脏扩大、心力衰竭和心律失常,为长期心肌缺血导致心肌纤维化所致,即缺血性心肌病。详见第四十三章各类心肌病的心电图特征。

\protect\hypertarget{text00050.htmlux5cux23subid591}{}{}

\subsection{猝死型冠心病}

因原发性心脏骤停而猝死,多为缺血心肌局部发生心电紊乱,引起严重的室性心律失常所致。

\protect\hypertarget{text00050.htmlux5cux23subid592}{}{}

\subsection{平板运动试验}

1.试验方法

(1)根据年龄计算最大心率的Bruce方案,对年龄较大者采用Bruce修正方案。

(2)分极量运动试验和次极量运动试验(为最大心率的85\%~90\%)两种运动量。

2.禁忌证

(1)怀疑有急性心肌梗死。

(2)不稳定心绞痛或休息期心绞痛。

(3)已服用洋地黄类药物或低钾血症者。

(4)心电图已诊断为左心室肥大、预激综合征、左束支阻滞。

(5)严重肺部疾病、高血压者(BP>160/100mmHg)。

(6)年老体衰、行动不便者。

(7)未控制的伴有临床症状或血流动力学障碍的心律失常。

(8)临床未控制的心力衰竭。

(9)急性心肌炎、心包炎。

(10)肥厚性心肌病或其他流出道梗阻性心脏病。

(11)高度房室传导阻滞。

3.终止运动试验目标

(1)心率达到预计标准。

(2)出现典型心绞痛。

(3)ST段呈缺血型压低≥0.2mV或抬高≥0.1mV。

(4)出现严重心律失常:频发多源性及多形性室性早搏、短阵性室性心动过速、高度~三度房室传导阻滞。

(5)心率在1min内减少20次。

(6)收缩压下降10~20mmHg或收缩压上升至≥210mmHg。

(7)极度疲劳不能坚持者。

(8)出现头晕、面色苍白、步态不稳者。

4.运动过程中注意事项

(1)严格掌握好适应证、禁忌证及终止运动试验目标。

(2)必须有一定临床经验的心内科医生参与监护。

(3)配备除颤器、氧气袋(或氧气钢瓶)、注射器及相关抢救药物(肾上腺素、异丙肾上腺素、阿托品、利多卡因及硝酸甘油等。)

(4)运动过程中严密观察心电图,定期测量血压,并观察患者神态,询问患者有无不适。

(5)若出现严重反应,立即终止运动,并进行诊治或抢救。

5.运动试验阳性标准的评定

(1)运动中出现典型的心绞痛,含服硝酸甘油有效。

(2)运动中或运动后出现ST段呈缺血型压低≥0.1mV,或在原有压低基础上再下降≥0.1mV。

(3)运动中或运动后出现ST段呈损伤型抬高≥0.1mV,或在原有抬高基础上再抬高≥0.1mV。

(4)运动中或运动后出现严重心律失常:多源性室性早搏、室性心动过速、高度~三度房室传导阻滞、心房颤动、高度窦房传导阻滞。

(5)运动中或运动后出现U波倒置。

6.运动试验阳性价值的评定

(1)运动试验阳性者,约有30\%~40\%患者,经冠状动脉造影证实有病变。

(2)ST段呈缺血型压低时,若出现时间越早,则冠状动脉病变程度越严重。

(3)ST段呈缺血型压低≥0.3mV者,往往属于三支病变或左冠状动脉主干病变。

(4)ST段呈缺血型压低持续时间:运动停止后,ST段恢复时间长短亦可提示冠状动脉病变程度。若运动后立即出现ST段压低并持续时间≥8min者,多为二支、三支冠状动脉病变。

(5)运动中、后出现心绞痛、低血压,均有重要临床意义。

(6)出现U波倒置,是左冠状动脉前降支狭窄严重的标志,具有高度特异性。

7.运动试验应用价值:①冠心病的诊断;②已知或可疑冠心病患者的严重程度、危险性和预防的评价;③急性心肌梗死患者出院前早期危险性评估;④评估心脏功能。

\protect\hypertarget{text00050.htmlux5cux23subid593}{}{}

\section{高血压性心脏病}

\protect\hypertarget{text00050.htmlux5cux23subid594}{}{}

\subsection{高血压性心脏病}

1.概述

长期高血压将引起左心室收缩期负荷过重,导致左心室代偿性增厚,可显著地增加心源性猝死、心肌缺血、心力衰竭和室性心律失常等心血管意外事件的发生率和死亡率。

2.病理生理改变

长期左心室收缩期负荷过重及神经内分泌等体液因素影响,导致左心室心肌细胞增大、肌纤维增粗增长、心肌间质重构引起心肌重量和硬度增加,形成左心室代偿性肥厚,包括向心性对称性左心室肥厚、不对称性左心室肥厚及离心性左心室肥大。左心室肥厚将引起心肌缺血、室性心律失常及各种传导阻滞。

3.心电图特征

(1)左心室高电压:左胸导联或(和)肢体导联R波电压明显增高。

(2)左心室肥厚伴劳损:左胸导联、肢体导联R波电压均明显增高,QRS波群时间轻度增宽,同时伴有ST段呈缺血型压低、T波负正双相或倒置、U波负正双相或倒置(图\ref{fig42-5})。

\begin{figure}[!htbp]
 \centering
 \includegraphics[width=5.58333in,height=3.02083in]{./images/Image00696.jpg}
 \captionsetup{justification=centering}
 \caption{男性,56岁,患高血压病20余年,心脏超声波显示不对称性左心室肥厚、不能排除心尖肥厚性心肌病。心电图显示左心室肥大伴劳损(V\textsubscript{1}~V\textsubscript{6} 导联定准电压均为0.5mV)}
 \label{fig42-5}
  \end{figure} 


(3)室性心律失常:以多源性室性早搏、短阵性室性心动过速多见。与下列因素有关:①左心室肥厚多伴有心内膜下心肌缺血,可强烈地刺激室性异位灶发放冲动;②心室肌不规则肥厚和局部纤维化妨碍冲动均一地传至整个心肌而引起折返活动;③肥大心肌细胞的电生理与正常细胞的电生理不同,更易引发心律失常;④交感神经系统和神经体液内分泌因子活性加强,促使心律失常的发生。

(4)传导阻滞:以左束支阻滞、左前分支阻滞及右束支阻滞多见。与左心室肥厚扩张牵拉左束支、左前分支使之受损有关,而右束支阻滞则与心室收缩时,室间隔向右侧膨出牵拉右束支使之受损有关。

\protect\hypertarget{text00050.htmlux5cux23subid595}{}{}

\subsection{高血压性心肌病}

1.基本概念

继发于高血压引起的左心室肥厚、舒张充盈受损、左心室收缩功能减退、心室电活动不稳定、冠状动脉储备下降及心肌缺血,出现以扩张型或限制型心肌病、心力衰竭为特征的特异性心肌病。

2.病理生理改变

左心室肥厚时,左心室舒张期硬度增加,心室顺应性减退,首先出现舒张功能减退,继而影响收缩功能,使左心室射血分数下降,最终导致心力衰竭,加重心肌缺血、心电紊乱和传导阻滞。

3.心电图特征

与高血压性心脏病心电图特征类似,但室性心律失常及心室内传导阻滞的发生率更高、更严重。

\protect\hypertarget{text00050.htmlux5cux23subid596}{}{}

\section{肺源性心脏病}

\protect\hypertarget{text00050.htmlux5cux23subid597}{}{}

\subsection{慢性肺源性心脏病}

1.基本概念

由支气管、肺部慢性疾病引起肺循环阻力增加、肺动脉压力升高,导致右心室肥厚、右心房扩大,最后发生右心充血性心力衰竭的一组疾病。

2.病理生理改变

长期肺动脉高压引起右心室收缩期负荷过重,首先出现右心室流出道肥厚,继之出现右心室游离壁肥厚,随着病情的发展,出现右心室和右心房扩张、心脏顺钟向转位;当病变累及传导组织时,可出现各种心律失常及传导阻滞。

3.心电图特征

(1)肺型P波:①Ⅱ、Ⅲ、aVF导联P波高耸,电压≥0.25mV;②Ⅱ、Ⅲ、aVF导联P波呈尖峰状,电压在0.20~0.24mV,P电轴>+80°;③低电压时,P波呈尖峰状,其振幅>$\frac{1}{2}$
R,P电轴>+80°。符合以上P波改变之一者,均可诊断为肺型P波(图\ref{fig42-6})。

\begin{figure}[!htbp]
 \centering
 \includegraphics[width=5.78125in,height=1.88542in]{./images/Image00697.jpg}
 \captionsetup{justification=centering}
 \caption{慢性肺源性心脏病患者出现肺型P波、V\textsubscript{1}Ptf值增大、顺钟向转位}
 \label{fig42-6}
  \end{figure} 


(2)V\textsubscript{1}
导联P波呈正负双相型或直立尖角型:呈正负双相型时,正相波振幅乘以时间,代表右心房除极向量面积,主要反映右心房结构和功能。当该面积≥+0.03mm·s时,肺心病诊断的敏感性约57\%,特异性90\%;部分患者负相波表现为深而窄,V\textsubscript{1}
Ptf值增大≥|-0.04mm·s|。

(3)心电轴右偏≥+90°或出现S\textsubscript{Ⅰ} S\textsubscript{Ⅱ}
S\textsubscript{Ⅲ} 综合征,呈假性电轴左偏。

(4)aVR导联QRS波群R/S或R/Q>1。

(5)胸前导联出现重度顺钟向转位:V\textsubscript{1} ~V\textsubscript{6}
导联QRS波群均呈rS型,r/S<1,或V\textsubscript{5} 、V\textsubscript{6}
导联呈RS型,R/S<1或S>$\frac{1}{2}$ R。

(6)右胸导联出现异常Q波:有时V\textsubscript{1} ~V\textsubscript{3}
导联QRS波群呈QS、Qr、qr或qR型,约占12\%。

(7)V\textsubscript{1} 导联R/S>1,R波电压>1.0mV或呈qR型。

(8)R\textsubscript{V\textsubscript{1}}
+S\textsubscript{V\textsubscript{5}}
≥1.2mV:R\textsubscript{V\textsubscript{1}}
电压增高反映了右心室游离壁肥厚引起的向前向量增大,而S\textsubscript{V\textsubscript{5}}
增深,则反映了右心室流出道肥厚引起的向右后向量增大。这一诊断指标,敏感性约为27\%,而特异性高达100\%。

(9)QRS波群低电压:每个肢体导联QRS波群R+S电压<0.5mV或每个胸前导联R+S电压<1.0mV。

(10)心律失常:以窦性心动过速、多源性房性早搏、短阵性房性心动过速及室性早搏多见。

(11)右束支阻滞。

(12)非特异性ST-T改变:多见于下壁导联与右胸导联。

慢性肺源性心脏病除了右心室肥厚外,还兼有右心房扩大、顺钟向转位及肺气肿。故除了一般右心室肥厚心电图特征外,肺型P波、重度顺钟向转位、低电压是其特征性改变,可根据这3个特征性改变加上右心室肥厚图形,便可确立其诊断。

4.预后判断的心电图指标

(1)心率、Ⅱ导联及aVF导联P波振幅:这三项指标的数值越大,预后越差。

(2)V\textsubscript{1}
导联QRS波群呈qR型,是严重肺心病的特征性改变,提示预后不良。

\protect\hypertarget{text00050.htmlux5cux23subid598}{}{}

\subsection{肺栓塞(急性肺源性心脏病)}

1.基本概念

急性肺栓塞是指肺动脉某支血管内突然发生血源性阻塞,引起肺动脉反射性痉挛,导致右心室急剧扩张和急性右心衰竭,严重者可发生休克或猝死,又称为急性肺源性心脏病。临床上约50\%患者出现具有诊断意义的心电图特征,但应密切结合临床。

2.病理生理改变

肺动脉突然栓塞及同时出现的神经体液异常,导致肺动脉压力骤然升高,引起急性右心室收缩期负荷增加,右心室和右心房扩张,心脏顺钟向转位,出现右心室劳损和右心房扩大图形。此外,右心室壁张力增高可引起局部心肌缺血,出现ST-T改变。

3.心电图特征(图\ref{fig42-7})

\begin{figure}[!htbp]
 \centering
 \includegraphics[width=5.78125in,height=1.67708in]{./images/Image00698.jpg}
 \captionsetup{justification=centering}
 \caption{多发性骨折、急性肺栓塞患者出现加速的房性逸搏心律、新发的完全性右束支阻滞、S\textsubscript{Ⅰ}Q\textsubscript{Ⅲ} 型、非特异性ST-T改变(引自刘子荣)}
 \label{fig42-7}
  \end{figure} 


(1)窦性心动过速:为最常见的心律失常,心率多在100~125次/min,临床上若心率>90次/min,即对诊断有帮助。

(2)肺型P波、PR段压低:Ⅱ导联P波电压≥0.25mV,可能与右心房负荷过重、右心房扩大及心动过速有关。约1/3患者出现PR段压低。

(3)S\textsubscript{Ⅰ} Q\textsubscript{Ⅲ} T\textsubscript{Ⅲ}
型或S\textsubscript{Ⅰ} Q\textsubscript{Ⅲ}
型:即Ⅰ导联出现较明显的S波(>0.15mV),Ⅲ导联出现较明显的Q波(多呈QR型、qR型,其时间多<0.04s,深度<
$\frac{1}{4}$
R)伴T波倒置,为常见而重要的心电图特征。反映了急性右心室扩张和(或)一过性左后分支阻滞。

(4)电轴偏移:可发生右偏(+90°~+100°或较发病前右偏20°以上)或左偏。

(5)重度顺钟向转位:移行区左移至V\textsubscript{5}
导联或V\textsubscript{1} ~V\textsubscript{6} 导联均呈rS型。

(6)新出现的右束支阻滞:呈不完全性或完全性右束支阻滞图形。

(7)非特异性ST-T改变:Ⅰ、Ⅱ、V\textsubscript{5} 、V\textsubscript{6}
导联ST段轻度压低,Ⅲ、aVR、V\textsubscript{1} ~V\textsubscript{3}
导联ST段呈弓背向上型轻度抬高;V\textsubscript{1} ~V\textsubscript{3}
导联T波倒置,若呈对称性倒置,则见于大块肺栓塞的早期(24h内)。

(8)可出现各种房性心律失常:以心房颤动、扑动多见,常为一过性。

4.鉴别诊断

因急性肺栓塞临床上可出现胸痛、呼吸困难,心电图出现S\textsubscript{Ⅰ}
Q\textsubscript{Ⅲ} T\textsubscript{Ⅲ} 型及V\textsubscript{1}
~V\textsubscript{3}
导联ST段抬高、T波倒置,应与急性下壁、前间壁心肌梗死相鉴别。

\protect\hypertarget{text00050.htmlux5cux23subid599}{}{}

\section{风湿性心脏病}

\protect\hypertarget{text00050.htmlux5cux23subid600}{}{}

\subsection{二尖瓣狭窄}

1.病理生理改变

正常成人二尖瓣口直径约为3~3.5cm,面积约为4~6cm\textsuperscript{2}
。当二尖瓣口狭窄到一定程度时(约1/2),可引起左心房压力增高、代偿性扩大,继之出现肺静脉、肺毛细血管压力升高导致肺淤血和肺动脉压力增高,从而引起右心室肥大、扩张。

2.心电图表现

(1)二尖瓣型P波及V\textsubscript{1}
Ptf负值增大:与左心房扩大及心房内传导延缓有关。

(2)右心室肥大的心电图特征。

(3)可出现肺型P波:与右心房负荷过重、扩大有关。

(4)心律失常:①房性心律失常:早期以多源性房性早搏、短阵性房性心动过速多见,晚期几乎都有心房扑动、颤动发作,且以后者多见;②室性心律失常:以多源性、多形性室性早搏多见,可见短阵性室性心动过速,多与洋地黄毒性作用、低钾血症等因素有关。

(5)传导阻滞:以房室传导阻滞、束支阻滞多见。

\protect\hypertarget{text00050.htmlux5cux23subid601}{}{}

\subsection{二尖瓣关闭不全}

1.病理生理改变

二尖瓣关闭不全时,部分血液在左心室收缩时返流到左心房,引起左心房和左心室的容量增大,从而导致左心房、左心室肥大或扩张。

2.心电图表现

(1)二尖瓣型P波及V\textsubscript{1} Ptf负值增大。

(2)左心室肥大:左胸导联R波电压增高,T波高耸,表现为舒张期负荷过重的心电图特征。

(3)可出现各种心律失常及传导阻滞。

\protect\hypertarget{text00050.htmlux5cux23subid602}{}{}

\subsection{二尖瓣狭窄伴关闭不全}

二尖瓣狭窄伴关闭不全,表现为左心房和左心室的舒张期容量负荷增大及右心室的收缩期负荷过重,严重者可出现双心房、双心室肥大的心电图特征(图\ref{fig42-8})。

\begin{figure}[!htbp]
 \centering
 \includegraphics[width=5.58333in,height=3.78125in]{./images/Image00699.jpg}
 \captionsetup{justification=centering}
 \caption{风心病、二尖瓣狭窄伴关闭不全患者,出现双心房肥大、局限性前间壁异常Q波、电轴左偏、提示双心室肥大伴劳损、高侧壁及前侧壁ST-T改变}
 \label{fig42-8}
  \end{figure} 

\protect\hypertarget{text00050.htmlux5cux23subid603}{}{}

\subsection{主动脉瓣狭窄}

正常主动脉瓣口面积为3cm\textsuperscript{2}
。当瓣口面积<1cm\textsuperscript{2}
时,左心室射血受阻,收缩期负荷过重,心搏出量减少,收缩期末左心室残余血量增加,舒张期血液充盈量增加,出现代偿性肥厚,最后发生左心衰竭。心电图上可出现左心室肥厚伴劳损改变及室性心律失常、左前分支阻滞或左束支阻滞(可掩盖左心室肥厚图形)。

\protect\hypertarget{text00050.htmlux5cux23subid604}{}{}

\subsection{主动脉瓣关闭不全}

主动脉瓣关闭不全时,左心室舒张期同时接受来自左心房流入的血液和从主动脉返流回来的血液,故左心室舒张期容量负荷明显增加,导致左心室肥大、扩张。心电图表现为左心室肥大、电轴左偏、T波高耸,呈现舒张期负荷过重的特征。

\protect\hypertarget{text00050.htmlux5cux23subid605}{}{}

\section{甲状腺功能紊乱性心脏病}

\protect\hypertarget{text00050.htmlux5cux23subid606}{}{}

\subsection{概述}

甲状腺功能紊乱性心脏病包括功能亢进引起的心脏病和功能减退引起的心脏病。

\protect\hypertarget{text00050.htmlux5cux23subid607}{}{}

\subsection{甲状腺功能亢进性心脏病}

1.病理生理改变

甲状腺功能增强,分泌甲状腺素过多,一方面引起代谢亢进,增加心脏负荷,另一方面甲状腺素直接作用于心肌和周围血管,并加强儿茶酚胺作用,导致心率加快、心脏肥大、心肌耗氧量增加及心房肌兴奋性增高、不应期缩短。

2.心电图表现

(1)窦性心动过速:心率多在120次/min左右,系甲状腺素毒性作用和交感神经兴奋性增高所致。

(2)心脏肥大:可表现为右心房、右心室、左心室或全心肥大的心电图特征。

(3)非特异性ST-T改变。

(4)心律失常:以房性早搏、短阵性房性心动过速及心房颤动多见,与甲状腺素使心房肌兴奋性增高,不应期缩短有关。尚可出现房室传导阻滞、束支阻滞,与低钾血症、传导组织局灶性坏死和纤维化有关。

3.甲状腺功能亢进性心肌病

(1)基本概念:当甲状腺功能亢进患者出现以心脏扩大、心力衰竭和心房颤动为主要特征时,便可称为甲状腺功能亢进性心肌病。与甲状腺素毒性作用、长期心动过速或心房颤动有关。经治疗后,上述心脏异常可消失或明显好转。

(2)诊断标准:①甲状腺功能亢进诊断明确;②有心脏扩大或心力衰竭或心绞痛或明显心律失常或二尖瓣脱垂伴杂音之一者;③经治疗后,心脏异常改变消失或明显好转;④除外其他器质性心脏病。

4.可能由甲状腺功能亢进引起的心血管异常改变

(1)原因不明的阵发性或持续性心房颤动,心室率快而不易被洋地黄类药物所控制。

(2)原因不明的右心衰竭,但患者无贫血、发热或脚气病等,洋地黄类药物疗效不佳。

(3)无法解释的心动过速。

(4)血压波动而脉压差增大者。

(5)器质性心脏病患者发生心力衰竭时,常规治疗疗效不佳或心率增快难以控制者。

\protect\hypertarget{text00050.htmlux5cux23subid608}{}{}

\subsection{甲状腺功能减退性心脏病}

1.病理生理改变

甲状腺素分泌不足,机体基础代谢率低下,心肌能量代谢及心肌对儿茶酚胺敏感性均降低,心肌可发生非特异性病理改变,导致心率减慢,心搏出量减少及心脏扩大;毛细血管通透性增加、嗜水性粘多糖和粘蛋白堆积,出现心包积液;血中胆固醇增高,易发生动脉粥样硬化。

2.心电图表现

(1)窦性心动过缓。

(2)出现各种传导阻滞:房室传导阻滞、束支阻滞及不定型心室内传导阻滞。

(3)QRS波幅低电压。

(4)非特异性ST-T改变及Q-T间期延长。

\protect\hypertarget{text00050.htmlux5cux23subid609}{}{}

\section{心肌炎}

1.概述

心肌炎可由感染性(病毒、细菌、支原体等微生物感染)、过敏或变态反应、化学、物理或药物等因素引起心肌内局部性或弥漫性炎症性病变。以病毒性心肌炎多见,临床上诊断比较困难,需要心肌活检才能确诊。心电图检查对心肌炎的诊断具有一定的价值,并能指导制订治疗方案和判断预后。

2.分型

临床上最常见的病毒性心肌炎可分为3型。

(1)急性心肌炎:以心肌炎症、损伤为主,无或仅有轻微纤维化;临床上短时间内发生心力衰竭和各种心律失常、传导阻滞,多在6个月内死亡或痊愈。

(2)亚急性心肌炎:有少量心肌损害灶,出现广泛的心肌纤维化和愈合性心肌损害灶;临床上可交替出现心功能代偿和心力衰竭,多伴心律失常及传导阻滞,病程6个月至数年。

(3)慢性心肌炎:病程缓慢,达3~5年以上;临床上表现为心脏肥大、扩张,可遗留程度不等的心力衰竭症状及各种心律失常、传导阻滞。

3.心电图表现

(1)窦性心律失常:以窦性心动过速多见,若炎症累及窦房结,则可出现显著的窦性心动过缓、窦房传导阻滞、窦性停搏,表现为病窦综合征。

(2)传导阻滞:以一度、二度房室传导阻滞、心室内传导阻滞多见,大多数是可逆性的,约有30\%患者迅速发展为三度房室传导阻滞。

(3)心律失常:以房性、室性早搏及短阵性房性、室性心动过速多见。

(4)QRS波幅低电压:约占12\%。

(5)Q-T间期延长:约占30\%。

(6)非特异性ST-T改变。

(7)少数重症心肌炎患者可出现异常Q波、ST段呈损伤型抬高酷似急性心肌梗死图形,预示心肌损害较严重。

4.急性病毒性心肌炎心电图诊断标准

急性上呼吸道、消化道感染后1~3周内新出现下列心电图改变:

(1)房室或窦房传导阻滞、束支阻滞。

(2)两个以上导联ST段呈缺血性压低>0.05~0.1mV或多个导联ST段异常抬高或有异常Q波。

(3)频发多形性、多源性、成对早搏或并行性早搏,短阵性房性、室性心动过速等。

(4)两个以上以R波为主的导联T波低平或倒置。

(5)频发房性早搏或室性早搏。

具有(1)~(3)任何一项,即可考虑诊断急性病毒性心肌炎;具有(4)或(5)项,无明显病毒感染史者,要补充左心室收缩功能减弱、病程早期有心肌酶谱增高这两个条件。

\protect\hypertarget{text00050.htmlux5cux23subid610}{}{}

\section{心包炎}

\protect\hypertarget{text00050.htmlux5cux23subid611}{}{}

\subsection{急性心包炎}

急性心包炎除了心包脏层和壁层间的渗出性炎症外,心包下的心外膜心肌也受到波及而发生弥漫性炎症性反应,出现损伤性改变和缺血性改变;若心包内有积液,则心肌产生的电流会发生短路现象。

1.心电图特征

(1)窦性心动过速。

(2)广泛导联ST段呈凹面向上型抬高:发病早期,即胸痛发生后数小时,Ⅰ、Ⅱ、aVF、V\textsubscript{2}
~V\textsubscript{6}
导联ST段呈凹面向上型抬高,一般<0.5mV,以Ⅱ、V\textsubscript{5}
、V\textsubscript{6} 导联为明显,aVR、V\textsubscript{1}
导联ST段压低,持续数小时至数天,便回到等电位线。与炎症累及心外膜下浅层心肌产生损伤性电流有关(图\ref{fig42-9})。

\begin{figure}[!htbp]
 \centering
 \includegraphics[width=5.58333in,height=1.83333in]{./images/Image00700.jpg}
 \captionsetup{justification=centering}
 \caption{男性,17岁,发热、胸痛,临床诊断为急性心包炎、心肌炎。显示窦性心动过缓伴P电轴左偏、P波低电压、高侧壁及下壁和前侧壁ST段抬高伴T波高耸酷似变异型心绞痛的心电图改变}
 \label{fig42-9}
  \end{figure} 

(3)T波改变:以R波为主的导联T波低平或倒置(<0.5mV),多发生在ST段回到等电位线后。与心外膜下心肌缺血有关。

(4)PR段偏移:PR段偏移方向与ST段偏移方向相反,即ST段抬高导联,其PR段多呈水平型压低0.05~0.15mV。PR段偏移发生在急性心包炎早期,可早于ST段抬高,甚至是唯一表现,具有早期特异性诊断价值。与心房肌较薄,较易损伤引起心房复极异常有关。

(5)QRS波幅低电压:与心包积液有关。

(6)偶尔可见QRS、ST、T等波段电交替现象。

2.分期

急性心包炎典型的心电图改变,可分为4期:①Ⅰ期:主要是PR段压低和ST段抬高,这两者具有特征性改变,具有诊断价值;②Ⅱ期:ST段回到基线;③Ⅲ期:T波倒置;④Ⅳ期:T波回到基线(表42-1)。

\begin{table}[htbp]
\centering
\caption{急性心包炎心电图改变}
\label{tab42-1}
\includegraphics[width=6.20833in,height=1.47917in]{./images/Image00701.jpg}
\end{table}

3.鉴别诊断

急性心包炎患者早期有胸痛、ST段呈凹面向上型抬高,需与急性心肌梗死、早复极综合征相鉴别(表42-2)。

\begin{table}[htbp]
\centering
\caption{急性心包炎与急性心肌梗死、早复极综合征的心电图鉴别}
\label{tab42-2}
\includegraphics[width=6.20833in,height=2.5625in]{./images/Image00702.jpg}
\end{table}

\protect\hypertarget{text00050.htmlux5cux23subid612}{}{}

\subsection{慢性心包炎}

1.病理生理改变

心包膜纤维组织广泛增生、增厚及粘连,妨碍心脏收缩和舒张功能,尤其是左心室舒张受限,导致左心房、肺静脉压力增高,引起肺动脉高压和右心室肥大;心包腔内积液,造成心脏电流传导短路现象及受压的心肌细胞萎缩,可出现QRS波幅低电压;心包炎可引起心肌病变及缺血,出现ST-T改变。

2.心电图表现

(1)窦性心动过速:心率100~160次/min多见,与心脏收缩和舒张功能受限,心排出量减少后一种代偿性反应有关。

(2)QRS波幅低电压。

(3)广泛导联ST-T改变:以R波为主的导联ST段压低、T波低平或倒置。

(4)P波改变:可出现P波增高、增宽及双峰切迹,与心房肌受累、心房扩大或心房内传导阻滞有关。

(5)房性心律失常:可出现房性早搏、短阵性房性心动过速、心房颤动等,与心房肌受累及心房内压力长期增高有关。

(6)病程较长者,可出现右心室肥大、右束支阻滞(图\ref{fig42-10})。

\begin{figure}[!htbp]
 \centering
 \includegraphics[width=5.58333in,height=0.98958in]{./images/Image00703.jpg}
 \captionsetup{justification=centering}
 \caption{慢性心包炎患者出现窦性心动过速、肺型P波、低电压、完全性右束支阻滞、前间壁异常Q波、顺钟向转位、右心室肥大}
 \label{fig42-10}
  \end{figure} 

\protect\hypertarget{text00050.htmlux5cux23subid613}{}{}

\subsection{心包积液}

大量心包积液时,心电图上可出现窦性心动过速、低电压、广泛导联T波改变及P、QRS、T各波段电交替现象。在肯定有心包积液情况下,电交替现象提示有大量心包积液或心包填塞。

\protect\hypertarget{text00051.html}{}{}

\protect\hypertarget{text00051.htmlux5cux23chapter51}{}{}

\chapter{各类心肌病的心电图改变}

\protect\hypertarget{text00051.htmlux5cux23subid614}{}{}

\subsection{基本概念和分类}

1996年世界卫生组织(WHO)和国际心脏病学会(FSH)将心肌病定义为心肌病变伴心功能障碍的疾病,并将其分为原发性心肌病和特异性(继发性)心肌病两种,前者包括扩张型、肥厚型、限制型、致心律失常性右室心肌病和未分类心肌病,后者指继发于已明确病因的心肌疾病,如缺血性心肌病、瓣膜性心肌病、高血压性心肌病、炎症性心肌病、代谢性心肌病、酒精性心肌病、围生期心肌病、糖尿病性心肌病、克山病、家族遗传性心肌病、心动过速性心肌病等。这类患者临床特征为进行性心脏扩大、心功能减退及各种心律失常和传导障碍,病理特征为弥漫性心肌退行性变及纤维化,心室肥厚扩张。临床上以扩张型心肌病最为常见,其次为肥厚型心肌病;而心电图改变以肥厚型心肌病最具特征性,其次为致心律失常性右室心肌病。

2006年美国心脏病协会对心肌病重新进行了定义和分类,将离子通道疾病如长Q-T间期综合征、短Q-T间期综合征、Brugada综合征、异常J波、Lenegre综合征及儿茶酚胺介导的心动过速等原发性心电活动异常疾病归入心肌病范畴;把由其他心血管疾病所致的心肌病理改变不包括在心肌病范畴,如心脏瓣膜病、高血压性心脏病、先天性心脏病、冠心病等所致心肌病,建议不再使用“缺血性心肌病”这一命名。仍将心肌病分为原发性心肌病和继发性心肌病两大类。原发性心肌病是指病变仅局限在心肌,根据发病机制可分为遗传性、混合性及获得性3种。遗传性心肌病包括肥厚型心肌病、致心律失常性右室心肌病、左心室致密化不全、线粒体肌病和离子通道病等;混合性心肌病包括扩张型和限制型心肌病;获得性心肌病包括炎症性心肌病、应激性心肌病、围生期心肌病、心动过速性心肌病、酒精性心肌病等。继发性心肌病是指心肌的病变为全身多器官病变的一部分,心脏受累的程度变化很大,包括淀粉样变性心肌病、糖尿病性心肌病、糖原蓄积所致的心肌病、脚气病性心肌病等。

\protect\hypertarget{text00051.htmlux5cux23subid615}{}{}

\subsection{扩张型心肌病}

1.基本概念

扩张型心肌病是指由原发性或混合性心肌疾病导致一侧或双侧心腔扩大,继以心室收缩功能减退的原因不明的心肌病,约30\%~50\%患者具有家族遗传特点,常伴有骨骼肌和神经肌肉病变。

2.病理生理改变

心肌细胞肥大、纤维组织增生,并出现非特异性退行性改变及间质纤维化;病变弥散,波及全心,但以左心室扩张为主,心室壁肥厚相对不明显甚至变薄;心脏收缩功能减退,心排血量减少引起心力衰竭。病变累及传导组织可引起各种心律失常和传导障碍。附壁血栓脱落可引起心、脑、肾等重要器官栓塞。

3.心电图改变

几乎所有病例都有心电图异常改变,以异位搏动和异位心律最为常见,其次为传导阻滞和STT改变(图\ref{fig43-1}、图\ref{fig43-2})。

\begin{figure}[!htbp]
 \centering
 \includegraphics[width=3.41667in,height=4.51042in]{./images/Image00704.jpg}
 \captionsetup{justification=centering}
 \caption{男性,38岁,扩张型心肌病。心电图显示右心房肥大、V\textsubscript{1}Ptf增大(提示左心房负荷过重)、左心室肥大伴劳损、不定型心室内传导阻滞(QRS时间0.20s)、下壁异常Q波}
 \label{fig43-1}
  \end{figure} 


\begin{figure}[!htbp]
 \centering
 \includegraphics[width=5.19792in,height=1.38542in]{./images/Image00705.jpg}
 \captionsetup{justification=centering}
 \caption{男性,68岁,扩张型心肌病。显示非阵发性房性心动过速(70~85次/min)、右心房肥大、完全性左束支阻滞、异常Q波(MV\textsubscript{1}导联定准电压0.5mV)}
 \label{fig43-2}
  \end{figure} 


(1)异位搏动和异位心律:90\%的患者有复杂性室性心律失常,如多源性和(或)多形性室性早搏、成对室性早搏、短阵性室性心动过速等,10\%~20\%的患者出现房性心律失常,如房性早搏、短阵性房性心动过速及心房颤动等。有时,一些顽固性、难治性心律失常可能是扩张性心肌病早期诊断的重要线索。

(2)传导阻滞:最常见的是房室传导阻滞,以二度、三度阻滞多见,阻滞部位多在希氏束分叉以下,其次为不定型心室内传导阻滞、束支阻滞、双分支或三分支阻滞。传导阻滞的出现与病变累及传导系统及继发于心脏扩大,导致希-浦系统广泛受损有关。

(3)左心室高电压:约10\%的患者出现左心室高电压,其发生率低与心室以扩张为主而心室壁增厚不明显有关。

(4)QRS波幅低电压:约占15\%,与心肌细胞退行性变、坏死、纤维化导致心室除极时所产生的电位减少有关。

(5)异常Q波:约占11\%~20\%,常见于左胸导联及肢体导联,与心肌细胞片状坏死、疤痕形成(纤维化)有关。出现异常Q波,意味着心肌有较严重的病理学改变。

(6)非特异性ST-T改变:约占40\%~50\%,以R波为主导联ST段呈水平型或下斜型压低,T波低平、负正双相或倒置。

(7)Q-T间期延长:约占20\%,与心室除极、复极时间延长有关。

(8)P波时间增宽:约占20\%,与左心房负荷过重、扩大及左心房传导延缓有关。

4.易引发猝死的心电图表现

(1)多源性成对室性早搏、短阵性或持续性室性心动过速伴心室晚电位阳性者。

(2)不定型心室内传导阻滞、双分支阻滞及三分支阻滞。

\protect\hypertarget{text00051.htmlux5cux23subid616}{}{}

\subsection{肥厚型心肌病}

1.基本概念

肥厚型心肌病是指原因不明的左心室心肌不对称、不均匀性肥厚,心室腔变小,以左心室血液充盈受阻及舒张期顺应性降低为基本病变的心肌病。约50\%的患者具有家族遗传特点,由基因突变导致肌节功能异常所致,为常染色体显性遗传的家族遗传性疾病。

2.病理生理改变

心室肌纤维肥大,排列紊乱,病变主要累及室间隔和左心室,导致室间隔呈显著不对称性肥厚、左心室游离壁部分或全部非对称性或弥漫性肥厚,前者出现左心室流出道狭窄而成为梗阻型心肌病。心肌细胞间质纤维化、结缔组织增生,心室僵硬度增高,左心室舒张功能受损导致舒张期顺应性明显降低。由于心室腔变小,舒张期顺应性降低,左心室充盈受阻,心搏出量下降,将引发心肌缺血或心绞痛。若病变累及传导组织可引起各种心律失常和传导障碍,严重者可导致猝死。

3.分型

根据病理解剖所见,可分为4型:室间隔肥厚型、心尖部肥厚型、室间隔后部肥厚型及左心室侧壁肥厚型。

4.心电图特征(图\ref{fig43-3})

\begin{figure}[!htbp]
 \centering
 \includegraphics[width=4.15625in,height=4.54167in]{./images/Image00706.jpg}
 \captionsetup{justification=centering}
 \caption{男性,46岁,心尖部肥厚型心肌病(心尖厚度2.3cm),定准电压均为0.5mV。显示窦性心动过缓、左心室肥大伴劳损、前壁巨倒T波}
 \label{fig43-3}
  \end{figure} 

(1)持续性ST-T改变:最常见且最具特征性。ST段呈水平型或下斜型压低0.1~0.3mV,T波常呈对称性倒置,深度≥0.5~1.0mV,酷似“冠状T波”,以胸前导联尤其是V\textsubscript{3}
、V\textsubscript{4} 导联最为明显,多见于心尖部肥厚型心肌病。

(2)左心室高电压或左心室肥厚:R\textsubscript{V\textsubscript{5}}
及R\textsubscript{V\textsubscript{5}}
+S\textsubscript{V\textsubscript{1}}
电压均明显增高,有时V\textsubscript{1}
导联QRS波群呈Rs型,R波电压>1.0mV,这不是右心室肥大的表现,而是异常增厚的室间隔左侧面除极时所产生的向右前向量增大所致。

(3)窄而深的异常Q波:具有特征性改变,常见于Ⅱ、Ⅲ、aVF导联或V\textsubscript{5}
、V\textsubscript{6}
导联,同时这些导联R波电压增高,T波常直立,而有别于心肌梗死的异常Q波,多见于室间隔肥厚型心肌病。

(4)心电轴左偏。

(5)P波时间增宽:P波时间增宽与左心房肥大、左心房内传导障碍有关,因左心室顺应性降低,左心室舒张期末压增高,导致左心房负荷过重,久之将引起左心房肥大和左心房内传导障碍。

(6)心律失常:可见房性心律失常(房性早搏、房性心动过速、心房颤动)、传导阻滞(房室传导阻滞、束支阻滞)及室性心律失常(多源、多形性室性早搏、短阵性室性心动过速),以室性心律失常多见且易引发恶性心律失常而猝死。

(7)部分患者可出现预激综合征的图形。

5.诊断线索

(1)年轻男性患者,无高血压病史,出现左胸导联R波电压增高伴ST段压低、胸前导联T波倒置,应高度怀疑心尖部肥厚型心肌病。

(2)年轻男性患者,无高血压病史,出现左胸导联窄而深的异常Q波伴R波电压增高,T波直立,应高度怀疑室间隔肥厚型心肌病。

\protect\hypertarget{text00051.htmlux5cux23subid617}{}{}

\subsection{致心律失常性右室心肌病}

1.基本概念

致心律失常性右室心肌病是指右室心肌被脂肪浸润及纤维组织所替代,导致右心室弥漫性扩张、心室壁变薄变形、心肌萎缩、收缩运动进行性减弱,出现右室心力衰竭、右室源性心律失常及发作性晕厥为特征的原因不明的心肌病。主要见于青少年,约30\%有家族史,为常染色体显性遗传,是年轻人猝死的常见原因之一。

2.病理生理改变

右室心肌被脂肪浸润及纤维组织所替代,导致右心室扩张、收缩性减弱及右室心力衰竭,出现右心房负荷过重、扩大;病变累及传导组织,出现右心室内传导障碍及室性心律失常。

3.心电图特征

(1)P波高尖:系右心房负荷过重、肥大或扩张所致。

(2)局限性QRS波群时间增宽:右心室部分心肌除极延迟,导致局限性V\textsubscript{1}
~V\textsubscript{3}
导联QRS波群时间≥0.11s,其特异性为100\%,敏感性为55\%;如(V\textsubscript{1}
+V\textsubscript{2} +V\textsubscript{3}
)QRS波群时限/(V\textsubscript{4} +V\textsubscript{5}
+V\textsubscript{6}
)QRS波群时间≥1.2,则特异性为100\%,敏感性为93\%,反映了右心室部分心肌除极延迟,同时V\textsubscript{1}
~V\textsubscript{3} 导联的Q-T间期相应延长。

(3)右束支阻滞图形:约33\%的患者出现不同程度右束支阻滞图形,但阻滞部位并非真正地发生在右束支主干,而是发生在右心室壁内的传导障碍。如在右束支传导阻滞基础上,V\textsubscript{1}
~V\textsubscript{3} 导联QRS波群时间比V\textsubscript{6}
导联延长0.05s以上,则具有非常诊断意义。

(4)Epsilon波:V\textsubscript{1} 、V\textsubscript{2}
导联QRS波群终末部或ST段起始处出现向上小棘波,偶呈凹缺状,约持续0.02s,有时出现在右胸V\textsubscript{3}
R、V\textsubscript{4}
R导联。放大定准电压(20mm/mV),加快纸速(50mm/s),可提高检出率,或者用双极胸前导联(将右上肢导联用吸球吸在胸骨柄处作为阴极,左上肢导联用吸球吸在剑突处作为阳极,左下肢导联用吸球吸在V\textsubscript{4}
导联位置作为阳极,选择在Ⅰ、Ⅱ、Ⅲ导联进行记录),可提高检出率2~3倍。Epsilon波是致心律失常性右室心肌病一个特异性较强的心电图指标,具有诊断价值,是右心室被脂肪组织包绕的岛样有活性心肌细胞延迟除极所致(图\ref{fig43-4}、图\ref{fig43-5})。

\begin{figure}[!htbp]
 \centering
 \includegraphics[width=5.20833in,height=1.65625in]{./images/Image00707.jpg}
 \captionsetup{justification=centering}
 \caption{女性,53岁,家族性致心律失常性右室心肌病。右胸导联出现Epsilon波及T波倒置(引自蔡海鹏)}
 \label{fig43-4}
  \end{figure} 

\begin{figure}[!htbp]
 \centering
 \includegraphics[width=5.58333in,height=1.59375in]{./images/Image00708.jpg}
 \captionsetup{justification=centering}
 \caption{男性,28岁,致心律失常性右室心肌病。显示V\textsubscript{1}导联有Epsilon波及间歇性T波倒置、QRS波群及部分T波电交替现象、假性电轴左偏-90°、顺钟向转位、右心室肥大?}
 \label{fig43-5}
  \end{figure} 
、V\textsubscript{2}


(5)心律失常:主要表现为起源于右心室的室性早搏和室性心动过速,其QRS波群呈类似左束支阻滞图形,其次为房性心律失常,如房性早搏、房性心动过速、心房扑动及颤动等。

(6)胸前导联T波倒置:为该心肌病的特征性表现之一,绝大多数发生在V\textsubscript{1}
~V\textsubscript{3} 导联,偶尔发生在V\textsubscript{1}
~V\textsubscript{6} 导联。

(7)心室晚电位阳性。

4.心电图诊断标准

Fisher提出致心律失常性右室心肌病的心电图诊断标准为:①V\textsubscript{1}
~V\textsubscript{3} 导联T波倒置;②出现V\textsubscript{1}
~V\textsubscript{3}
导联局限性QRS波群时限≥0.11s;③Epsilon波;④频发类似左束支阻滞型的室性早搏(>1000次/24h);⑤反复出现类似左束支阻滞型的室性心动过速;⑥心室晚电位阳性。

\protect\hypertarget{text00051.htmlux5cux23subid618}{}{}

\subsection{缺血性心肌病}

1.基本概念

缺血性心肌病是指由冠心病或冠状动脉末梢弥漫性病变引起心肌长期缺血缺氧,导致心肌纤维化,出现充血性心力衰竭为主的综合征,而不能用冠状动脉病变或缺血损伤程度来解释的收缩功能障碍。2006年美国心脏病协会建议不再使用“缺血性心肌病”这一命名。

2.病理生理改变

心脏肥大呈球形结构,心室壁厚薄交错不均匀,被大片疤痕组织代替,可累及右心室;心肌收缩力减退、心室顺应性降低,出现心力衰竭,并且反复发作。若累及传导组织,可出现各种心律失常及传导阻滞。

3.心电图改变

(1)异常Q波:反映心肌坏死、纤维化。

(2)缺血性ST-T改变。

(3)左心室高电压:V\textsubscript{5} 、V\textsubscript{6}
导联R波电压增高。

(4)Q-T间期延长。

(5)心律失常:以窦性心动过速、房性心律失常、室性心律失常等多见。

(6)传导阻滞:房室传导阻滞、束支阻滞、不定型心室内传导阻滞等。

4.诊断

诊断缺血性心肌病,必须具备3个肯定条件和2个否定条件。

(1)3个肯定条件为:①有明确的冠心病史,至少有≥1次心肌梗死或冠状动脉造影阳性;②心脏明显扩大;③顽固性心力衰竭。

(2)2个否定条件为:①排除冠心病并发症引起的室壁瘤、室间隔穿孔、乳头肌功能不全;②排除其他心脏病和其他原因引起的心脏扩大和心力衰竭。

\protect\hypertarget{text00051.htmlux5cux23subid619}{}{}

\subsection{围生期心肌病}

1.基本概念

围生期心肌病是指在妊娠过程中,特别是在妊娠末3个月至产后6个月内首次发生的以累及心肌为主的一种与妊娠有密切关系的心肌病。多发生在产后,以急性心力衰竭起病。

2.病理生理改变

与扩张型心肌病病理生理改变类似,4个心腔均有不同程度的扩张,但以左心室扩张最为显著。若病变累及传导系统,可出现各种心律失常和传导阻滞。

3.心电图改变

与扩张型心肌病心电图改变类似,主要表现为室性和房性心律失常、传导阻滞、非特异性STT改变及心房、心室扩大的心电图改变。

4.诊断

在确定围生期心肌病诊断之前,必须明确心力衰竭的原因,排除其他心脏病。Silber提出3条诊断标准:①既往无任何心脏病证据;②妊娠末3个月至产后6个月内出现心脏病及心力衰竭;③心脏病和心力衰竭不能用其他病因来解释。

\protect\hypertarget{text00051.htmlux5cux23subid620}{}{}

\subsection{心动过速性心肌病}

1.基本概念

各种长期反复发作的心动过速引起心脏进行性扩大、心功能减退,经积极治疗控制心动过速后,扩大的心脏会逐渐缩小,心功能部分或完全恢复正常,这种继发于心动过速性心肌疾病,称为心动过速性心肌病。

2.病因

心动过速是引起心肌病的直接原因。心动过速可分为阵发性室上性心动过速、心房扑动、心房颤动、室性心动过速、起搏器介导性心动过速及不适当性窦性心动过速等。心动过速持续时间越长,频率越快,则心肌受损越严重,病变越广泛。心动过速性心肌病的形成需要数年或更长时间。

3.分型

(1)单纯型:心动过速是导致心脏扩大、心功能异常的唯一因素,心脏无其他异常改变。

(2)混合型:除了心动过速外,尚合并其他导致心功能异常的病因。

4.病理生理改变

持续性心动过速或心动过速每天发作总时间超过10\%~15\%,将会导致心脏扩大,尤其是心室腔扩张,心室壁变薄,心脏收缩功能、舒张功能均减退,出现心力衰竭;若病变累及传导系统,还可出现各种心律失常和传导阻滞。

5.心电图改变

在原有心动过速基础上,可出现其他心律失常,如早搏、传导阻滞及非特异性ST-T改变等。

6.诊断

病史和临床表现是目前诊断心动过速性心肌病的唯一可靠手段,有心脏扩大或心力衰竭和持续性心动过速或反复发作心动过速的患者应高度怀疑此病。其诊断要点为:①心动过速发作前心功能正常;②在频繁发作或持续性心动过速后出现心功能进行性损害,并能排除其他因素影响;③心动过速治愈或控制后,扩大的心脏改善或恢复正常。

\protect\hypertarget{text00051.htmlux5cux23subid621}{}{}

\subsection{应激性心肌病}

1.基本概念

应激性心肌病由精神刺激所引发的左心室功能不全、影像学与心电图呈一过性改变的一组症候群。表现为:①发病初期患者胸痛,左心室造影及心脏超声心动图均有左心室心尖和前壁下段运动减弱或消失,基底部心肌运动代偿性增强;②左心室平均射血分数降低;③冠状动脉造影正常。

2.发病机制

与体内过高的儿茶酚胺对心肌细胞的直接毒性作用引起的心肌顿抑有关。

3.心电图改变

(1)类似急性心肌梗死,一般出现在发病后4~24h。

(2)发病急性期,绝大多数患者胸前导联出现ST段抬高(0.2~0.6mV)。

(3)半数患者在急性期和亚急性期(2~18d)T波逐渐转为倒置,T波出现深倒置是患者处于恢复期的心电图特征性表现。

(4)约1/3患者出现病理性Q波,常见于V\textsubscript{1}
~V\textsubscript{4} 导联。

(5)Q-T间期延长出现在发病后48h内,但很快恢复正常。

(6)可出现各种心律失常。

4.诊断依据

(1)发病年龄与性别:多发生于老年绝经期后的女性,女性发病率是男性的7倍。

(2)病史:发病前有强烈的心理或躯体应激状态。

(3)症状:绝大多数患者出现类似急性心肌梗死胸痛和呼吸困难。

(4)辅助检查:①心电图异常;②左心室造影及心脏超声心动图均提示一过性心室壁运动异常,左心室心尖和前壁下段运动减弱或消失,基底部心肌运动代偿性增强;③左心室平均射血分数降低;④冠状动脉造影正常;⑤心肌酶谱正常或轻度增高。

(5)转归:心功能常在短时间内恢复正常,预后一般良好。

\protect\hypertarget{text00052.html}{}{}

\protect\hypertarget{text00052.htmlux5cux23chapter52}{}{}

\chapter{经典的心肌梗死及其进展}

\protect\hypertarget{text00052.htmlux5cux23subid622}{}{}

\subsection{心脏的血液供应}

1.心肌的血液供应

左冠状动脉(左心室80\%的血液由其供应):起源于主动脉根部的左后主动脉窦,长约0.5~1.0cm,很快分为前降支、回旋支(左旋支),少数人还有对角支。

(1)前降支(左心室50\%的血液由其供应):①左心室前支,供应左心室前壁、前乳头肌;②右心室前支,供应右心室前壁;③前室间隔支(又称为前穿支),供应室间隔前上2/3部分、希氏束、左右束支及左前分支、左中隔支等。

(2)回旋支(左旋支):①左心室前支,供应左心室前壁;②左心室后支,供应左心室侧壁、后壁、后乳头肌;③左心房支,供应左心房;④左缘支,供应左心室的最外侧缘。

(3)对角支:供应左心室前壁的上部。

右冠状动脉(左心室20\%的血液由其供应):起源于右主动脉窦,在房室交接处作“U”字形弯曲,称为U袢,并延伸为后降支,分为右心房支、右心室支、后降支等。

(1)右心房支:供应右心房。

(2)右心室支:供应右心室前壁、侧壁。

(3)后降支:发出后室间隔支、右缘支,供应室间隔后下1/3部分、下壁、后壁、左后分支、窦房结、房室结等。

2.传导系统的血液供应

(1)窦房结:绝大多数由单支血管供应,即由窦房结动脉供血,约2/3发自右冠状动脉的近端,1/3发自左冠状动脉回旋支近端。窦房结动脉亦供应心房肌、房间隔的大部分及心房内传导组织。

(2)房室结:由多支血管供血,血源丰富,主要由房室结动脉供血,绝大部分起源于右冠状动脉远端的U袢,少部分发自左冠状动脉的回旋支;此外,房室结尚接受回旋支等动脉的血供。

(3)希氏束、束支:由房室结动脉、前室间隔支双重血管供应。急性心肌梗死时如发生左束支阻滞,提示左、右冠状动脉均有病变。

(4)分支:左前分支、左中隔支由前室间隔支供应,若前室间隔支发生阻塞,则可引起左前分支、左中隔支阻滞;左后分支由前室间隔支、后降支双重供血,故单纯性左后分支阻滞或右束支合并左后分支阻滞少见。

\protect\hypertarget{text00052.htmlux5cux23subid623}{}{}

\subsection{定位诊断与相关动脉病变部位的判断}

根据心电图相关导联出现的异常Q波、ST段改变、T波改变及传导阻滞类型,来推测可能是哪一支相关动脉发生病变。

(1)高侧壁心肌梗死及其相关病变的动脉:Ⅰ、aVL导联面对左心室高侧壁,该部位由回旋支的左缘支、左心室后支供血。若高侧壁发生心肌梗死,则其病变动脉为回旋支的左缘支、左心室后支部位。

(2)下壁心肌梗死及其病变的动脉:Ⅱ、Ⅲ、aVF导联面对左心室下壁,该部位多数患者由右冠状动脉的后降支和左心室后支供血(右冠状动脉优势)。若出现下壁、右心室心肌同时梗死,则其病变动脉为右冠状动脉近端、锐缘支发出前的部位(图\ref{fig44-1}A的A点);若仅为单纯的下壁心肌梗死,则为右冠状动脉锐缘支的远端(图\ref{fig44-1}A的B点)。而对左冠状动脉优势型患者,该部位由回旋支的左心室后支供血,若同时并发侧壁、正后壁、下壁心肌梗死,则其病变动脉为回旋支近端部位(图\ref{fig44-1}B的A点);若为单纯的下壁心肌梗死,则其病变动脉为左心室后支(图\ref{fig44-1}B的B点)。下壁心肌梗死易并发房室传导阻滞,且提示阻塞部位在U袢之前。

\begin{figure}[!htbp]
 \centering
 \includegraphics[width=4.17708in,height=2.42708in]{./images/Image00709.jpg}
 \captionsetup{justification=centering}
 \caption{下壁心肌梗死相关的病变动脉}
 \label{fig44-1}
  \end{figure} 

(3)前间壁心肌梗死及其病变的动脉:V\textsubscript{1}
、V\textsubscript{2} 、(V\textsubscript{3}
)导联面对左心室间隔的前部,该部位还含有希氏束、束支,由前室间隔支供血。若梗死部位局限在前间壁,则为间隔支近端发生阻塞(图\ref{fig44-2}的A点),且常合并右束支阻滞;若同时累及左心室前壁,则病变的动脉是前降支的间隔支发出之前(图\ref{fig44-2}的B点)。

\begin{figure}[!htbp]
 \centering
 \includegraphics[width=2.15625in,height=2.19792in]{./images/Image00710.jpg}
 \captionsetup{justification=centering}
 \caption{前间壁心肌梗死相关的病变动脉}
 \label{fig44-2}
  \end{figure} 

(4)前壁心肌梗死及其病变的动脉:V\textsubscript{3} 、V\textsubscript{4}
、(V\textsubscript{5}
)导联面对左心室前壁,该部位由左前降支供血。若心肌梗死仅局限在V\textsubscript{3}
、V\textsubscript{4} 、(V\textsubscript{5}
)导联部位,则其病变动脉为左前降支中段(图\ref{fig44-3}的A点);若扩展到V\textsubscript{1}
、V\textsubscript{2} 导联,即V\textsubscript{1} ~V\textsubscript{5}
导联,则其病变动脉为左前降支近端(图\ref{fig44-3}的C点),易发生双束支阻滞或完全性房室传导阻滞。

\begin{figure}[!htbp]
 \centering
 \includegraphics[width=2.16667in,height=2.15625in]{./images/Image00711.jpg}
 \captionsetup{justification=centering}
 \caption{前壁心肌梗死相关的病变动脉}
 \label{fig44-3}
  \end{figure} 

(5)前侧壁心肌梗死及其病变的动脉:V\textsubscript{4}
、V\textsubscript{5} 、V\textsubscript{6}
导联面对左心室前侧壁,该部位由左前降支供血。若前侧壁发生心肌梗死,则其病变动脉为左前降支发出对角支之前的B点部位(图\ref{fig44-3}的B点)。

(6)侧壁心肌梗死及其相关病变的动脉:Ⅰ、aVL、(V\textsubscript{5}
)和V\textsubscript{6}
导联面对左心室侧壁,该部位由回旋支、前降支和右冠状动脉右心室支供血。若侧壁发生心肌梗死,则其病变动脉为回旋支近端或对角支及前降支的近端部位(图\ref{fig44-4}的A、B点)。

\begin{figure}[!htbp]
 \centering
 \includegraphics[width=2.16667in,height=2.23958in]{./images/Image00712.jpg}
 \captionsetup{justification=centering}
 \caption{侧壁心肌梗死相关的病变动脉}
 \label{fig44-4}
  \end{figure} 

(7)正后壁心肌梗死及其病变的动脉:V\textsubscript{7}
、V\textsubscript{8} 、(V\textsubscript{9}
)导联面对左心室正后壁,该部位多数患者由回旋支的左心室后支供血,少数由右冠状动脉后降支供血或由这两者共同供血。右冠状动脉优势患者,若同时并发下壁、正后壁心肌梗死,则其病变动脉为右冠状动脉近端;左冠状动脉优势型患者,若仅出现正后壁心肌梗死,则其病变动脉为钝缘支(图\ref{fig44-5})。

\begin{figure}[!htbp]
 \centering
 \includegraphics[width=2.21875in,height=2.39583in]{./images/Image00713.jpg}
 \captionsetup{justification=centering}
 \caption{正后壁心肌梗死相关的病变动脉}
 \label{fig44-5}
  \end{figure} 

(8)广泛前壁心肌梗死及其病变的动脉:Ⅰ、aVL、V\textsubscript{1}
~V\textsubscript{6}
导联面对左心室高侧壁、前间壁、前侧壁,称为广泛前壁。该部位由左冠状动脉前降支、回旋支及右冠状动脉后降支供血。若发生广泛前壁心肌梗死,表明上述冠状动脉发生病变,并且易出现双束支阻滞或完全性房室传导阻滞。

(9)右心室心肌梗死及其病变的动脉:需要加做右胸V\textsubscript{3}
R、V\textsubscript{4} R、V\textsubscript{5} R、V\textsubscript{6}
R导联,该部位由右冠状动脉的右心室支、左冠状动脉的前降支的右心室前支供血。单纯性右心室梗死少见,由右冠状动脉的锐缘支起始部阻塞所致(图\ref{fig44-6}的B点);右心室梗死往往伴发下壁梗死,由右冠状动脉近端、锐缘支发出前的部位发生阻塞所致(图\ref{fig44-6}的A点)。故下壁心肌梗死时,一定要注意是否同时伴发了右心室梗死,应当加做右胸导联。

\begin{figure}[!htbp]
 \centering
 \includegraphics[width=2.02083in,height=2.19792in]{./images/Image00714.jpg}
 \captionsetup{justification=centering}
 \caption{右心室心肌梗死相关的病变动脉}
 \label{fig44-6}
  \end{figure} 

(10)心房梗死及其病变的动脉:心房的血液由回旋支的左心房支和右冠状动脉的右心房支供应。单纯性心房梗死少见,并且易并发房性心律失常。

\protect\hypertarget{text00052.htmlux5cux23subid624}{}{}

\subsection{心肌梗死的基本心电图改变}

冠状动脉粥样硬化引起管腔狭窄、斑块脱落、血栓形成而导致某一支冠状动脉突然阻塞或冠状动脉发生严重而持久痉挛性闭塞引起心肌急性缺血、损伤直至坏死,产生一系列特征性的心电图改变:中心区表现为坏死型异常Q波,中间区表现为损伤型ST段抬高,外侧区表现为缺血型T波倒置。

(一)缺血型T波改变

缺血型T波改变是冠状动脉急性闭塞后最早出现的改变,首先表现为T波直立高耸,两肢基本对称,呈帐篷状,振幅可高达2.0mV,对早期诊断急性心肌梗死具有重要的临床意义。数分钟至数小时后,T波很快由高耸转为倒置。

(1)心内膜下心肌缺血:通常缺血最早发生在心内膜下心肌层,面对缺血区导联出现T波直立高耸,两肢基本对称,基底部变窄,呈帐篷状,类似高钾血症的T波改变。

(2)心外膜下心肌缺血:随着缺血的进一步加重,出现心外膜下心肌缺血,面对缺血区导联的T波两肢呈对称性倒置,表现为冠状T波。

(3)穿壁性心肌缺血:倒置的T波进一步加深,伴Q-T间期延长。

(二)损伤型ST段改变

随着心肌缺血进一步加重,将出现损伤型ST段改变,表现为面对损伤区导联出现ST段抬高或压低,为急性心肌梗死早期的另一种心电图表现。ST段抬高的形态、程度及其动态演变对诊断急性心肌梗死和预后判断都具有重要的临床意义。

(1)心内膜下心肌损伤:面对损伤区导联出现ST段呈水平型、下斜型压低≥0.1mV。

(2)心外膜下心肌损伤:面对损伤区导联ST段呈斜直型、弓背向上型、单相曲线型、墓碑型、巨R型抬高≥0.1mV。

(3)穿壁性心肌损伤:ST段抬高更加明显,多>0.5mV。

(三)坏死型QRS波群改变

持续而严重的心肌缺血、损伤,将导致心肌坏死,出现异常Q波,包括组织学上的坏死和电学上的“电静止”,后者是由于心肌细胞膜电位负值降至阈电位以下,暂时丧失了电活动能力,即出现心肌顿抑现象,供血改善后,异常Q波可消失。多数患者在急性心肌梗死发生后6~14h出现异常Q波。

1.异常Q波(病理性Q波)的诊断标准

(1)旧标准:①Q波时间≥0.04s;②Q波深度≥$\frac{1}{4}$
R;③呈QS型,起始部错折;④出现胚胎型r波,即呈qrS型或QrS型。

(2)新标准:相邻两个导联的Q波时间≥0.03s、深度≥0.1mV,但不包括Ⅲ、aVR导联。

2.异常Q波形成的条件

(1)心肌梗死范围:梗死区直径>2.0~3.0cm时,将产生异常Q波。若梗死区直径<2.0cm,累及左心室≤10\%,则不会出现异常Q波,仅出现q波或等位性Q波。

(2)心肌梗死深度:梗死厚度>0.5~0.7cm,累及左心室厚度的50\%以上时,将产生异常Q波。人的心内膜厚度约占心室壁的50\%,若梗死厚度<50\%,则不会出现异常Q波,仅引起QRS波形改变,如顿挫、切迹、R波电压降低等。

(3)心肌梗死部位:出现异常Q波,除了心肌梗死范围足够大、深度足够深外,梗死区还必须是在QRS起始0.04s部位。否则,不会出现异常Q波,如基底部梗死,仅引起QRS终末0.04s处切迹、顿挫或S波加深。

3.等位性Q波(或相当性Q波)

等位性Q波是指因梗死面积较少或局限于基底部、心尖部或梗死极早期尚未充分发展等原因,未形成典型的异常Q波,仅产生各种特征性QRS波形改变,这些伴随临床症状出现的特征性QRS波形改变与异常Q波有等同的诊断价值,称为等位性Q波或相当性Q波,但必须结合临床及同导联ST-T改变情况。

(1)部分q波:当梗死面积较小,虽位于QRS起始0.04s除极部位,但不能形成典型的异常Q波,仅出现q波:①V\textsubscript{1}
~V\textsubscript{6} 导联均出现q波;②V\textsubscript{3}
~V\textsubscript{6}
导联q波宽于和深于下一个胸前导联q波,即qV\textsubscript{3}
>qV\textsubscript{4} >qV\textsubscript{5} >qV\textsubscript{6}
;③右胸导联出现q波,而左胸导联q波消失,能排除右心室肥大、左前分支阻滞,即V\textsubscript{1}
、V\textsubscript{2} 导联出现q波而V\textsubscript{5}
、V\textsubscript{6}
导联未见q波;④下壁导联Ⅱ导联有q波,Ⅲ导联呈Q波,aVF导联q波时间0.03s左右,深度接近$\frac{1}{4}$
R。

(2)QRS波群起始部切迹、顿挫:V\textsubscript{4} ~V\textsubscript{6}
导联QRS起始部出现≥0.05mV负相波,即呈rsR′型,与心尖部心肌梗死或前壁小面积心肌梗死有关。

(3)进展性Q波:同一患者在相同体位、部位进行动态观察,原有q波进行性增宽、加深,或原无q波的导联出现新的q波。

(4)存在病理性Q波区:某个胸前导联q波虽未达到病理性Q波的诊断标准,但在其导联周围(上、下或左、右)均可记录到Q波,表明存在病理性Q波区域,为诊断心肌梗死有力佐证。

(5)R波电压变化:①动态观察,同一导联R波电压进行性降低,又称为R波丢失;②胸前导联R波振幅逆递增,如rV\textsubscript{2}
>rV\textsubscript{3} >rV\textsubscript{4}
;③胸前导联R波振幅递进不良,如V\textsubscript{1} ~V\textsubscript{4}
导联,r波振幅递进量<0.1mV;④右胸导联V\textsubscript{3}
R、V\textsubscript{1} 、V\textsubscript{2}
导联R波电压增高伴T波高耸,呈镜像改变,表明存在正后壁心肌梗死,应加做V\textsubscript{7}
、V\textsubscript{8} 、V\textsubscript{9}
导联;⑤Ⅱ导联有Q波,Ⅲ、aVF导联未见q波或Q波,但其QRS波群电压≤0.25mV,或Ⅱ导联R波电压≤0.25mV伴Ⅲ、aVF导联有Q波;⑥相邻的两个胸前导联的R波振幅相差≥50\%,如R\textsubscript{V\textsubscript{3}}
>$\frac{1}{2}$
R\textsubscript{V\textsubscript{4}}
;⑦新消失的室间隔q波,即Ⅰ、aVL、V\textsubscript{5} 、V\textsubscript{6}
导联q波消失或减小。

\protect\hypertarget{text00052.htmlux5cux23subid625}{}{}

\subsection{急性心肌梗死的诊断标准及心电图分类}

1.急性心肌梗死的诊断标准

临床对急性心肌梗死的诊断一直沿用WHO的诊断标准:①有缺血性胸痛症状;②有心电图特征性ST-T动态演变或伴异常Q波;③有血清心肌酶谱升高与回落。满足其中2条者,即可诊断。2000年,欧洲/美国心脏病学会重新修订了急性心肌梗死的诊断标准,即有典型心肌坏死生化标志物(肌钙蛋白或CK-MB)的升高与回落,同时伴有下列1项者,即可诊断为急性心肌梗死:①有心肌缺血症状;②出现病理性Q波;③有ST段抬高或压低;④冠状动脉介入治疗(PTCA)术后。

2.急性心肌梗死的心电图分类

急性心肌梗死的心电图分类经历了三个阶段:

(1)透壁性和非透壁性心肌梗死:20世纪80年代前沿用此分类方法,现已废弃不用。

(2)有Q波和非Q波心肌梗死:20世纪80年代后,根据有无出现Q波对心肌梗死进行分类。但出现Q波,意味着心肌细胞已坏死,不能满足临床早诊断、早干预、挽救濒死心肌的需求。但由于该分类方法简单明确,且两者在临床和预后上均有很大差异,故这一分类方法对临床仍有一定参考价值。无Q波心肌梗死者,其冠状动脉新形成的血栓较少、侧支循环较丰富、心肌损伤标志物水平较低,心肌灌注缺损不均匀较轻,心室壁运动异常程度较轻,心力衰竭发生率及近期死亡率均较低,但再梗死发生率高;而有Q波心肌梗死者,则刚好相反。

(3)ST段抬高型和非ST段抬高型梗死:近年来,国内外均采用ST段抬高型和非ST段抬高型进行分类,使心肌梗死诊断的时间大大提前,为早干预、早治疗、挽救濒死心肌赢得了宝贵时间,极大地改善了患者的预后,突出了早期干预的重要性和“时间就是心肌”的诊治理念。

ST段抬高型:相关导联先表现为T波高耸呈帐篷状,继之ST段呈斜直型、弓背向上型、单相曲线型,甚至墓碑型、巨R型抬高。ST段抬高是冠状动脉闭塞早期的心电图表现,是早期干预的标志。若心绞痛患者经治疗后不能缓解,持续时间达20min以上,相邻两个或两个以上导联ST段抬高(胸前导联抬高≥0.2mV、肢体导联抬高≥0.1mV),高度提示发生了急性心肌梗死,需尽早干预,挽救濒死心肌;否则,大部分患者将发展为Q波性心肌梗死。经治疗后,抬高的ST段快速回落(2h内回落>50\%)是冠状动脉再通的无创指标,少数患者可出现再灌注性损伤,表现为ST段快速回落前呈现一过性再抬高。

非ST段抬高型:大多数非Q波性心肌梗死,相关导联ST段呈水平型、下斜型压低≥0.2mV,持续时间>24h,或(和)伴有T波对称性倒置及其动态演变,诊断时,需结合临床症状及心肌损伤标志物升高(肌钙蛋白、CK-MB)。

\protect\hypertarget{text00052.htmlux5cux23subid626}{}{}

\subsection{心肌梗死的心电图演变规律及分期}

随着急性心肌梗死诊断新标准的实施,早诊断、早干预,实施早期再灌注治疗,挽救了许多濒死的心肌细胞,缩小了梗死范围,缩短了心肌梗死的病程,降低了异常Q波的发生率,部分地改变了心肌梗死心电图的演变规律。传统的心肌梗死心电图可分为超急性期、急性期、亚急性期(演变期)及陈旧期(慢性期)4期。亦有学者将其分为急性期、亚急性期及慢性期3期,其中急性期又分为3个亚期:超急性期(T波改变期)、进展期或急性早期(ST段改变期)、心肌梗死确定期(Q波及非Q波期)。

1.超急性期

超急性期又称为超急性损伤期,是急性心肌梗死最早期阶段,在冠状动脉闭塞后立即出现,持续时间极为短暂,约数分钟至数小时。心电图表现为相关导联T波高耸、ST段上斜型或斜直型抬高、急性损伤性阻滞及心律失常(图\ref{fig44-7})。

\begin{figure}[!htbp]
 \centering
 \includegraphics[width=5.79167in,height=1.63542in]{./images/Image00715.jpg}
 \captionsetup{justification=centering}
 \caption{男性,68岁,胸痛发作0.5h。显示下壁、后侧壁及右心室超急性期心肌梗死、前间壁ST段改变}
 \label{fig44-7}
  \end{figure} 

(1)T波高耸呈“帐篷状”改变:与细胞内K\textsubscript{+}
大量逸出而呈短暂性细胞外高\textsubscript{K}
+状态有关,如不及时干预治疗,异常Q波将出现在T波高耸的导联上。

(2)ST段呈上斜型或斜直型抬高:出现在T波高耸的导联上。

(3)急性损伤性阻滞:与损伤区域心肌组织传导延缓有关,表现为QRS波群轻度增宽(为0.11~0.12s)、振幅略增高。该现象持续时间较短,发生在异常Q波和T波倒置之前。

(4)U波倒置。

(5)出现各种室性心律失常:与损伤区域心肌处于严重的电生理紊乱状态有关。

2.急性期

从超急性期过度到急性期,在异常Q波尚未出现前,心电图可出现一过性假性正常化波形。急性期发生在梗死后数小时至数天内。

(1)出现异常Q波:梗死区相关导联出现异常Q波,与心肌细胞组织学上坏死或电学上坏死即“电静止”有关,后者经积极治疗后,异常Q波可消失。

(2)损伤型ST段抬高:可呈弓背向上型、单向曲线型、墓碑型、巨R型抬高(图\ref{fig44-8})。

\begin{figure}[!htbp]
 \centering
 \includegraphics[width=5.58333in,height=0.97917in]{./images/Image00716.jpg}
 \captionsetup{justification=centering}
 \caption{男性,71岁,冠心病。显示心房颤动、前侧壁急性期心肌梗死}
 \label{fig44-8}
  \end{figure} 

(3)冠状T波:高耸的T波逐渐下降并呈对称性倒置

3.亚急性期(演变期)

持续时间约数周至数月。

(1)相对稳定的异常Q波或R波振幅降低。

(2)抬高的ST段逐渐回至基线或呈稳定性抬高(与室壁瘤形成有关)。

(3)T波动态演变:T波逐渐加深,又逐渐变浅转为低平或直立,也可呈恒定性T波倒置(图\ref{fig44-9})。

\begin{figure}[!htbp]
 \centering
 \includegraphics[width=3.54167in,height=4.47917in]{./images/Image00717.jpg}
 \captionsetup{justification=centering}
 \caption{男性,78岁,急性心肌梗死后1月余。显示肢体导联QRS波群低电压、前间壁异常Q波及前壁R波振幅降低伴T波改变(符合亚急性期心肌梗死)、高侧壁T波改变、Q-T间期延长}
 \label{fig44-9}
  \end{figure} 

4.陈旧性期(慢性稳定期)

临床上规定急性心肌梗死发病1个月后,即称为陈旧性期。一般情况以>3个月为陈旧性期。

(1)异常Q波很少有变化或转为QR、Qr型或转为q波或Q波消失。

(2)ST段恢复正常或呈缺血型压低或呈恒定性抬高。

(3)T波恢复正常或低平、倒置或呈恒定性冠状T波(图\ref{fig44-10}、图\ref{fig44-11})。

\begin{figure}[!htbp]
 \centering
 \includegraphics[width=5.58333in,height=2in]{./images/Image00718.jpg}
 \captionsetup{justification=centering}
 \caption{男性,64岁,陈旧性心肌梗死3年余。显示前间壁异常Q波(符合陈旧性期心肌梗死)、完全性右束支阻滞、左后分支阻滞、下壁及前侧壁T波改变}
 \label{fig44-10}
  \end{figure} 

\begin{figure}[!htbp]
 \centering
 \includegraphics[width=3.23958in,height=4.58333in]{./images/Image00719.jpg}
 \captionsetup{justification=centering}
 \caption{男性,75岁,心肌梗死8个月。显示高侧壁、前间壁及局限性前壁异常Q波伴T波改变(符合陈旧性期心肌梗死)、前侧壁T波改变}
 \label{fig44-11}
  \end{figure} 

\protect\hypertarget{text00052.htmlux5cux23subid627}{}{}

\subsection{远离梗死区ST段改变的临床意义}

面对梗死区的导联出现异常Q波、损伤性ST段抬高、缺血性T波倒置,称为指示性改变。而与上述对应的导联出现R波振幅增高、ST段压低、T波高耸,则称为对应性改变,如下壁急性心肌梗死时,Ⅱ、Ⅲ、aVF导联ST段抬高,而Ⅰ、aVL导联ST段压低;后壁急性心肌梗死时,V\textsubscript{7}
、V\textsubscript{8} 、V\textsubscript{9}
导联出现异常Q波、ST段抬高、T波倒置,而V\textsubscript{3}
R、V\textsubscript{1} 、V\textsubscript{2}
导联出现R波增高、ST段压低、T波高耸。还有一部分远离梗死区的导联出现ST段压低者,则称为远离性ST段改变,其住院病死率和心肌梗死复发率均高于无ST段压低者,常合并多支血管病变,具有重要临床意义。

1.下壁急性心肌梗死伴其他导联ST段压低

(1)伴V\textsubscript{1} ~V\textsubscript{3}
导联ST段压低≥0.1mV(图\ref{fig44-7}):提示下壁梗死面积较大,多累及后侧壁、室间隔后下1/3部,由右冠状动脉近端阻塞所致,或者除了右冠状动脉阻塞(远端阻塞多见)外,尚合并回旋支阻塞(约占27\%)。若V\textsubscript{2}
导联ST段压低幅度与aVF导联ST段抬高幅度的比值≤0.5,则往往提示合并右心室急性心肌梗死;若V\textsubscript{3}
导联ST段压低幅度与Ⅲ导联ST段抬高幅度比值<0.5,则提示右冠状动脉近端阻塞;若比值在0.5~1.2之间,则提示右冠状动脉远端阻塞;若比值>1.2,则提示回旋支阻塞。

(2)伴V\textsubscript{4} ~V\textsubscript{5}
导联ST段压低≥0.1mV:比V\textsubscript{1} ~V\textsubscript{3}
导联ST段压低更具严重意义,其左心室功能更差、并发症更多,早期及远期(1~5年)死亡率明显高于无ST段压低者。多伴有前降支病变,且右冠状动脉近端阻塞及合并三支冠状动脉病变者明显高于无ST段压低者及V\textsubscript{1}
~V\textsubscript{3} 导联ST段压低者,侧支循环更差。

(3)伴侧壁(Ⅰ、aVL、V\textsubscript{6}
)导联ST段压低≥0.1mV:除了对应性改变外,还可能合并回旋支的左缘支病变。

(4)伴aVR导联ST段压低≥0.1mV:提示下壁梗死面积较大,心肌酶谱峰值较高,并发症发生率增高。

2.下壁急性心肌梗死伴其他导联ST段抬高

(1)伴V\textsubscript{1} ~V\textsubscript{3}
导联ST段抬高:为下壁梗死合并右心室梗死的象征。

(2)伴侧壁(Ⅰ、aVL、V\textsubscript{6}
)一个或数个导联ST段抬高:90\%由回旋支近端(钝缘支开口前)阻塞所致。

3.前壁急性心肌梗死伴其他导联ST段改变

前壁心肌梗死时,只要有任一导联ST段压低,其远期病死率高,此后发生心脏事件多。

(1)伴下壁导联ST段压低:前壁急性心肌梗死者伴下壁导联ST段压低,除了对应性改变外,还提示前壁心肌严重缺血及左前降支近端闭塞或远端闭塞合并第1对角支病变,为心肌梗死面积大、心功能差、并发症多、预后差的表现。

(2)伴aVR导联ST段抬高:提示左前降支阻塞位于左心室前支近端。

4.前侧壁急性心肌梗死伴aVR导联ST段压低

提示心肌梗死面积较大,心肌酶谱峰值较高,住院过程充血性心力衰竭发生率高,心功能较差,LVEF≤35\%。

\protect\hypertarget{text00052.htmlux5cux23subid628}{}{}

\subsection{特殊类型的心肌梗死}

1.右心室心肌梗死

单纯性右心室心肌梗死是罕见的,往往是下壁或下后壁急性心肌梗死波及右心室心肌而出现右心室心肌梗死。下壁、下后壁约40\%~50\%合并右心室梗死,均由右冠状动脉近端或右心室缘支近端阻塞所致。因此,对前间壁、下壁及后壁急性心肌梗死患者,必须加做V\textsubscript{3}
R、V\textsubscript{4} R、V\textsubscript{5} R、V\textsubscript{6}
R及V\textsubscript{7} 、V\textsubscript{8} 、V\textsubscript{9}
导联,以免漏诊。

(1)右心室心肌梗死的心电图改变:①V\textsubscript{3}
R~V\textsubscript{6}
R导联ST段抬高≥0.1mV,出现较早,且发病后24h内大多降至基线,以V\textsubscript{4}
R导联ST段抬高敏感性和特异性最高;②QRS波群在V\textsubscript{1}
导联呈rS型,在V\textsubscript{3} R~V\textsubscript{6}
R导联呈QS型;③V\textsubscript{1} ~V\textsubscript{3}
导联ST段呈损伤型抬高,但其抬高程度逐渐减轻且无异常Q波出现或V\textsubscript{1}
导联ST段抬高,V\textsubscript{2} 导联ST段压低。

(2)下壁急性心肌梗死时,出现下列改变者,强烈提示合并右心室梗死:①Ⅲ导联ST段抬高>Ⅱ导联ST段抬高,且ST\textsubscript{Ⅱ、Ⅲ}
≥0.1mV,诊断价值仅次于V\textsubscript{3} R~V\textsubscript{6}
R导联ST段抬高,诊断符合率达72\%~100\%(图\ref{fig44-7});②V\textsubscript{1}
~V\textsubscript{3}
导联ST段抬高,且抬高程度逐渐减轻或V\textsubscript{1}
导联ST段抬高≥0.1mV,而V\textsubscript{2}
导联ST段压低;③V\textsubscript{2}
导联ST段压低幅度与aVF导联ST段抬高幅度的比值≤0.5者,其敏感性为80\%左右,特异性90\%以上;④Ⅰ、aVL导联ST段压低>0.2mV者。

(3)正后壁急性梗死时,当V\textsubscript{1} ~V\textsubscript{3}
导联ST段压低不明显时,应高度怀疑合并右心室急性心肌梗死。

(4)下壁、正后壁急性心肌梗死合并电轴右偏、Ⅰ、aVL、V\textsubscript{5}
、V\textsubscript{6}
导联Q波消失,应高度怀疑合并右心室急性心肌梗死,因室间隔Q波消失与右冠状动脉病变引起右心室缺血具有高度相关性(图\ref{fig44-7})。

(5)在临床上,若遇及下壁、下后壁急性心肌梗死患者出现急性右心功能衰竭或窦性心动过缓、窦性停搏、房性心律失常(可能合并心房梗死)、房室传导阻滞、右束支阻滞等改变时,亦应高度怀疑合并右心室急性心肌梗死。

2.心内膜下心肌梗死

急性心内膜下心肌梗死有超急性期和急性期两个时相,心电图主要表现为ST-T的动态演变。

(1)面对梗死区导联ST段呈缺血型显著而持久地压低(图\ref{fig44-12}):以V\textsubscript{2}
~V\textsubscript{5}
导联最多见,ST段压低≥0.1mV,发病后第2~4天ST段压低达最大值,以后逐渐恢复正常。

\begin{figure}[!htbp]
 \centering
 \includegraphics[width=3.79167in,height=3.39583in]{./images/Image00720.jpg}
 \captionsetup{justification=centering}
 \caption{前间壁、局限性前壁急性心内膜下心肌梗死}
 \label{fig44-12}
  \end{figure} 

(2)面对梗死区导联T波倒置呈“冠状T波”(图\ref{fig6-3}):ST段压低的导联,其T波倒置逐渐加深,呈“冠状T波”,约持续1周后,T波又逐渐变浅或转为正常。

(3)不出现异常Q波,面对梗死区导联的R波振幅降低或进行性降低。

(4)Q-T间期延长。

在临床上,若遇胸痛患者经治疗后,胸痛持续时间超过20min而不能缓解,出现上述心电图改变伴心肌坏死生化标志物增高,可诊断为急性心内膜下心肌梗死。

3.再发性心肌梗死(复发性心肌梗死)

(1)基本概念:在原有心肌梗死基础上再次发生新的心肌梗死,称为再发性心肌梗死。包括原心肌梗死灶延伸、毗邻原梗死区或远离原梗死区的部位发生新的梗死灶这3种类型。

(2)类型及其心电图特征

原梗死灶延伸:是指急性心肌梗死后4周内(以2周内多见),同一支血管供血区域的心肌再次发生梗死,使原梗死灶范围或深度扩大,导致原为非穿壁性或心内膜下梗死延伸为穿壁性心肌梗死,出现Q波型心肌梗死,或者延伸至梗死区毗邻部位使其发生急性心肌梗死。心电图特征:①坏死性Q波增深增宽,或由q波转为Q波或QS波,或QRS波幅降低;②ST段再度抬高,且ST-T呈动态演变(图\ref{fig44-13});③毗邻原梗死区的导联亦出现ST段抬高、ST-T动态演变,可伴有异常Q波出现。

\begin{figure}[!htbp]
 \centering
 \includegraphics[width=5.58333in,height=0.91667in]{./images/Image00721.jpg}
 \captionsetup{justification=centering}
 \caption{男性,72岁,陈旧性心肌梗死1年余、再发胸痛0.5h。显示一度房室传导阻滞、前间壁及局限性前壁异常Q波、前间壁及前壁ST-T改变(提示又发生超急性期心肌梗死)、Q-T间期延长}
 \label{fig44-13}
  \end{figure} 

远离梗死区部位再梗死:是指首次心肌梗死后,再过若干时间,其他部位又发生新的急性心肌梗死。心电图特征:①原陈旧性梗死图形与新发生的急性心肌梗死图形并存。②当新发的梗死部位与原陈旧性梗死部位相对应时,若梗死范围大致相等,则表现为原有异常Q波消失,仅出现新发梗死区域导联ST段抬高及T波改变;若新发的梗死面积较大,则显示新发的梗死图形,而原有的梗死图形可部分或全部被掩盖。

(3)提高对再发性心肌梗死的警惕性:原发生过心肌梗死患者,若又出现不能缓解的胸痛或不明原因的心力衰竭、心源性休克,应高度警惕再发性心肌梗死的可能,特别注意以下心电图改变:①新出现q波或Q波伴ST段抬高;②QRS波群电压降低、切迹较多、时间增宽;③原有Q波增深、增宽或由q波转为Q波、QS波;④原有ST-T改变突然发生改变,甚至出现伪善性“正常”图形;⑤心电轴改变;⑥新出现房室传导阻滞、束支阻滞、室性心律失常或V\textsubscript{1}
Ptf增大。

4.心房梗死

单纯性心房梗死极其罕见,绝大部分是伴随着左心室梗死,右心房梗死比左心房梗死多见。心电图上可表现为PR段抬高或压低、P波呈M型或W型、房性心律失常、房室传导阻滞及窦性心动过速或过缓、停搏等。

5.从室性异位搏动图形中诊断心肌梗死

极少数急性心肌梗死患者,基本QRS-T波形正常,无异常Q波、ST段损伤型抬高和T波倒置,但在室性早搏QRS波群中却呈QR、QRs、qR型,ST段呈损伤型抬高伴T波高尖或倒置,显现急性心肌梗死的图形特征。可能由于基本节律引起室间隔前下1/3左心室面除极与左心室游离壁除极时,其向量指向了左前方,使心肌梗死的波形特征被掩盖。当出现室性早搏引起心室非同步除极时,梗死图形才在室性早搏中充分显示出来。从室性早搏QRS波形中诊断心肌梗死必须符合以下先决条件:①室性早搏QRS主波必须向上;②必须是反映心室电势的左胸导联(图\ref{fig44-14})。

\begin{figure}[!htbp]
 \centering
 \includegraphics[width=4.58333in,height=4.46875in]{./images/Image00722.jpg}
 \captionsetup{justification=centering}
 \caption{前间壁、前壁异常Q波伴ST-T改变,符合急性心肌梗死;室性早搏的QRS-T波群亦显示急性心肌梗死波形特征(V\textsubscript{1}~V\textsubscript{3} 导联第3个搏动、V\textsubscript{4}~V\textsubscript{6} 导联第1个搏动)}
 \label{fig44-14}
  \end{figure} 


\protect\hypertarget{text00052.htmlux5cux23subid629}{}{}

\subsection{心肌梗死合并心室除极异常时的诊断}

凡是能影响QRS初始向量的心室除极异常,均能掩盖心肌梗死的典型图形,给诊断带来困难,如左束支阻滞、左前分支阻滞、预激综合征、室性异位心律、心室人工起搏心律等。

1.心肌梗死合并左束支阻滞

发生率约为8\%,心室初始除极向量发生了改变(室间隔除极从右下向左上进行),左心室延迟缓慢除极,心肌梗死典型图形将被掩盖。此时,ST-T改变和动态演变、左束支阻滞QRS波形不典型的改变为心肌梗死的诊断提供线索和佐证,ST段抬高导联即是梗死灶的部位。有以下心电图表现者,可提示左束支阻滞合并心肌梗死:

(1)与QRS主波同向的导联,其ST段抬高≥0.1mV,即以R波为主导联ST段抬高≥0.1mV,如V\textsubscript{5}
、V\textsubscript{6} 导联ST段抬高≥0.1mV。

(2)与QRS主波异向的导联,其ST段抬高≥0.5mV,即以S波为主导联ST段抬高≥0.5mV,如V\textsubscript{1}
、V\textsubscript{2} 导联ST段抬高≥0.5mV。

(3)与QRS主波异向的导联,其ST段压低≥0.1mV,即以S波为主导联ST段压低≥0.1mV,如V\textsubscript{1}
~V\textsubscript{3} 导联ST段压低≥0.1mV。

(4)若V\textsubscript{1}
导联QRS波群初始r波振幅增高,Ⅰ、aVL、V\textsubscript{5}
、V\textsubscript{6}
导联出现q波,则提示合并右下室间隔梗死;如伴rV\textsubscript{1}
>rV\textsubscript{2} >rV\textsubscript{3} 及V\textsubscript{5}
、V\textsubscript{6}
导联R波第1峰电压降低、变形,则提示合并穿壁性室间隔梗死。

(5)若V\textsubscript{2} ~V\textsubscript{4}
导联呈rS或QS型,S波升支出现持续0.05s的切迹(Cabrera征)或Ⅰ、aVL、V\textsubscript{5}
、V\textsubscript{6}
导联R波升支出现切迹(Chapman征),则提示合并前壁心肌梗死。

(6)若V\textsubscript{5} 、V\textsubscript{6}
导联R波振幅降低,呈短小的M型或W型,或出现S波呈RS、rS型,在除外右心室肥大、肺气肿、顺钟向转位情况下,则提示合并前侧壁梗死(图\ref{fig44-15})。

\begin{figure}[!htbp]
 \centering
 \includegraphics[width=3.10417in,height=4.77083in]{./images/Image00723.jpg}
 \captionsetup{justification=centering}
 \caption{曾有心肌梗死病史患者,出现完全性左束支阻滞伴电轴左偏、前侧壁r波振幅逆递增或递增不良并出现S、s波,提示前侧壁陈旧性心肌梗死(V\textsubscript{1}~V\textsubscript{6} 导联定准电压均为0.5mV)}
 \label{fig44-15}
  \end{figure} 


(7)若V\textsubscript{2} ~V\textsubscript{6}
导联尤其是V\textsubscript{4} ~V\textsubscript{6}
导联呈现明显切迹的QS型或(和)QRS电压明显降低(低于肢体导联),则提示合并广泛前壁梗死。

(8)若Ⅱ、Ⅲ、aVF导联QRS电压显著降低,出现q波及终末S波,则提示合并下壁梗死。

上述QRS波群改变,如同时伴有特征性ST-T改变及演变规律,则诊断意义更大。

2.下壁心肌梗死合并左前分支阻滞

左前分支阻滞时,QRS初始0.02s向量位于左下;而下壁心肌梗死时,QRS初始0.02s向量则位于左上;两者并存时,可互相影响QRS波群的典型表现,下列两点有助于两者并存的诊断:

(1)QRS波群改变:①左前分支阻滞掩盖下壁梗死(仅累及下壁前部),具有左前分支阻滞的特点,Ⅱ、Ⅲ、aVF导联呈rS型,若r\textsubscript{Ⅲ}
>r\textsubscript{aVF} >r\textsubscript{Ⅱ}
,Ⅱ导联r波若呈双峰或其前有q波,则提示合并下壁梗死;②大面积下壁梗死(累及全下壁)掩盖左前分支阻滞,Ⅱ、Ⅲ、aVF导联呈QS型,若Ⅱ、Ⅲ、aVF导联S波不降低(即S波仍较深),且无终末R波,则提示合并左前分支阻滞。

(2)特征性ST-T改变和演变:Ⅱ、Ⅲ、aVF导联出现ST段损伤型抬高,是诊断合并下壁急性梗死的有力证据。

此外,少数左前分支阻滞在V\textsubscript{1} 、V\textsubscript{2}
导联可出现q波,呈qrS型,酷似前间壁陈旧性心肌梗死,但低一肋间描记,q波即消失,应注意鉴别。

3.心肌梗死合并预激综合征

预激综合征影响QRS初始向量,正向δ波将掩盖心肌梗死的Q波,而负向δ波则酷似心肌梗死的Q波。以下三点可提示或疑有预激综合征合并急性心肌梗死:

(1)以R波为主导联出现ST段抬高。

(2)以S波为主导联出现倒置或深尖的T波。

(3)ST-T有动态演变。

急性损伤性ST-T动态演变(具有定位意义),结合临床症状、心肌酶谱、肌钙蛋白阳性是确诊预激综合征合并急性心肌梗死的主要依据(图\ref{fig36-16})。对合并陈旧性心肌梗死的定位诊断只有消除δ波或诱发顺向型折返性心动过速时,方能明确诊断。

4.心肌梗死合并右束支阻滞

两者图形能同时显现,但前间壁急性梗死时,右束支阻滞的继发性ST段压低将会影响ST段抬高程度,使其抬高程度减轻或回到基线形成伪善性改变。

5.心肌梗死合并左后分支阻滞

(1)下壁梗死合并左后分支阻滞:①电轴右偏(>+110°);②Ⅱ、Ⅲ、aVF导联QRS波群呈QR型,R\textsubscript{Ⅲ}
>R\textsubscript{aVF} >R\textsubscript{Ⅱ} ,Ⅰ、aVL导联呈rS型。

(2)前侧壁梗死合并左后分支阻滞:①Ⅰ、aVL导联QRS波群呈QS型;②Ⅱ、Ⅲ、aVF导联初始q波消失,出现R波,且R\textsubscript{Ⅲ}
>R\textsubscript{aVF} >R\textsubscript{Ⅱ} 。

6.急性心肌梗死合并室性异位心律时的诊断

(1)室性异位QRS波群的特殊形态:在左心室外膜面导联(V\textsubscript{1}
、aVR导联除外),若室性异位QRS波群呈qR、QR、qRs、QRs型,则提示存在心肌梗死;若呈QS型QRS波群时间增宽(≥0.18s),QS波中有>0.05s的挫折,特别是其后伴有ST段呈弓背向上型抬高时,也应考虑心肌梗死。

(2)出现原发性ST-T改变:①出现与QRS主波同向的ST段抬高或以负向波为主时其ST段呈弓背向上型抬高,均提示急性心肌梗死;②T波顶峰变尖,两肢对称呈帐篷状或冠状T波时,可能是心肌梗死最早期的征象。

7.急性心肌梗死合并心室人工起搏心律时的诊断

(1)以R波为主导联出现ST段抬高≥0.1mV或伴T波高耸。

(2)以S波为主导联出现ST段抬高≥0.5mV伴T波高耸,敏感性53\%,特异性88\%。

(3)以S波为主导联ST段压低≥0.1mV或伴T波倒置,敏感性29\%,特异性82\%。

(4)以上ST-T改变呈动态改变时,其诊断价值更大。

\protect\hypertarget{text00052.htmlux5cux23subid630}{}{}

\subsection{心肌梗死并发症的心电图改变}

急性心肌梗死后所出现的并发症主要包括急性心力衰竭、心源性休克、心律失常、梗死后综合征、心脏破裂及室壁瘤形成等。本文着重讨论后4种并发症的心电图改变。

1.心律失常

(1)缺血性心律失常(冠状动脉闭塞性心律失常):可分为梗死后早期心律失常(冠状动脉闭塞后数分钟至0.5h内发生)和后期心律失常(冠状动脉闭塞后4~48h内发生),而冠状动脉闭塞后0.5~4h内很少有心律失常发生,则称为寂静期。早期心律失常以折返机制为主,晚期以自律性增高为主,多表现为室性早搏、短阵性室性心动过速,并易恶化为心室颤动,部分患者可表现为房性早搏、短阵性房性心动过速、心房颤动等房性心律失常和缓慢性心律失常(如窦性心动过缓、窦性停搏)及传导阻滞(如房室传导阻滞、心室内传导阻滞等)。

(2)再灌注心律失常(请见本章下面的内容)。

2.梗死后综合征

(1)基本概念:急性心肌梗死后坏死的心肌和心包的抗原与抗体免疫系统产生自身免疫反应,引起心包腔内无菌性炎症伴液体渗出。通常发生在梗死后2~3周内。

(2)心电图改变:①多数导联又出现ST段突然抬高,但程度较轻;②原倒置T波可转为直立或双向;③可出现PR段抬高;④心包大量积液时,出现QRS波幅低电压,可伴有QRS、ST、T各波段电交替现象。

3.心脏破裂

心脏破裂是急性心肌梗死最严重的并发症之一,常发生在透壁性心肌梗死的第1周内,尤其是第1天内最为常见,严重者可引起猝死。

(1)左心室游离壁破裂:常发生急性心包填塞而猝死。

(2)室间隔穿孔:见于室间隔透壁性梗死,如穿孔较小,仅表现为前间壁急性梗死图形;如穿孔较大,可出现右心室容量负荷增加的心电图改变。

(3)乳头肌断裂:可造成二尖瓣关闭不全,出现左心室容量负荷增加的心电图改变。

4.室壁瘤形成

(1)基本概念:指梗死面积较大的急性透壁性心肌梗死灶愈合过程中被结缔组织所取代,受心室腔压力的作用,梗死区心室壁向外呈袋状、囊状或不规则状膨出。发生率约10\%~30\%,能引起心功能不全、恶性室性心律失常、血栓形成等多种并发症,严重威胁患者的生命。

(2)形成原因:①梗死面积大;②透壁性梗死;③梗死区血管完全闭塞而无侧支循环形成。

(3)分类:按病理解剖分类:①真性室壁瘤:梗死灶被结缔组织所取代形成薄弱的瘢痕区,心脏收缩呈反向运动(矛盾运动);②假性室壁瘤:心肌梗死急性期心室壁已破裂,破口周围被血栓堵塞或粘连,瘤壁由心包膜组成。按病程分类:①急性室壁瘤:指心肌梗死发病后24h内形成的室壁瘤(实为坏死区坏死组织在心脏收缩向外膨出),易发生心脏破裂;②慢性室壁瘤:指心肌梗死发生15天后由结缔组织所取代而形成的室壁瘤。

(4)心电图改变:急性室壁瘤体表心电图难以诊断,而慢性室壁瘤体表心电图具有重要的预测和诊断价值,符合下列条件越多,诊断准确性越高:①ST段抬高至少出现在4个导联;②V\textsubscript{1}
~V\textsubscript{3} 导联ST段抬高≥0.2mV,V\textsubscript{4}
~V\textsubscript{6}
导联及以R波为主的肢体导联ST段抬高≥0.1mV持续1个月,或者≥0.2mV持续15天;③ST段抬高的导联有异常Q波;④运动试验时,在原有异常Q波导联上出现ST段呈弓背向上抬高≥0.1mV;⑤前壁梗死后V\textsubscript{3}
~V\textsubscript{5} 导联出现持续性ST段抬高伴V\textsubscript{1}
导联T波直立或低平,对诊断心尖部室壁瘤有较高特异性和准确性(图\ref{fig44-16})。

\begin{figure}[!htbp]
 \centering
 \includegraphics[width=5.08333in,height=1.59375in]{./images/Image00724.jpg}
 \captionsetup{justification=centering}
 \caption{陈旧性前间壁、前壁心肌梗死3年,V\textsubscript{2}~V\textsubscript{5} 导联出现持续性ST段抬高伴V\textsubscript{1}导联T波直立,心脏超声波显示心尖部室壁瘤形成}
 \label{fig44-16}
  \end{figure} 


\protect\hypertarget{text00052.htmlux5cux23subid631}{}{}

\subsection{再灌注治疗对急性心肌梗死转归的影响}

随着对急性心肌梗死早期诊断、早期实施溶栓、PTCA及放置支架等再灌注治疗,大多数患者将会缩小梗死面积、减少异常Q波发生率、缩短病程、改善预后,但少数患者反而出现再灌注损伤和心律失常,使病情加重,甚至危及生命。

1.再灌注治疗心肌血供改善有效性的心电图改变

再灌注治疗后,早期(3h内)主要观察ST段是否快速回落,随后的12~24h内,则主要观察T波变化。

(1)抬高ST段快速回落:再灌注治疗开始后2h内或相隔0.5h,抬高ST段快速回落≥50\%,或者ST段回落>0.2mV,或者再灌注3h内ST段回落>25\%,以上改变均属ST段早期快速回落。3~24h之间仍有一部分ST段缓慢回落,72h达较稳定水平。ST段早期快速回落是心肌再灌注成功的指标,若ST段完全回落(早期回落≥70\%或ST段抬高<0.1mV)所需时间愈短、幅度愈大,则预后愈好。

(2)不出现异常Q波或Q波消失或变小:成功的再灌注治疗,约有1/3患者ST段抬高导联不出现异常Q波,部分Q波消失或变小。

(3)加速T波演变:成功的再灌注治疗将加速T波演变,可使T波出现两次加深的演变。①直立高耸的T波振幅明显降低;②24h内ST段抬高导联出现早期T波倒置,为心肌再灌注成功的表现,是梗死相关动脉再通的独立指标;③两次T波倒置加深演变:第1次最深出现在再灌注后48~72h,提示有较多心肌细胞获救,变浅几天后再加深,第2次最深出现在梗死后2~4周;④以后T波倒置深度又逐渐变浅直至恢复正常,预示梗死区“冬眠”心肌功能恢复,T波转为直立时间越早,左心室功能恢复越好。

(4)原有的心律失常减轻或消失。

2.再灌注性损伤

(1)基本概念:指心肌严重缺血持续一段时间再恢复血液灌注后,反而出现缺血性损伤进一步加重的病理现象,表现为心肌结构破坏和心功能损害更为明显,是一种严重的治疗矛盾,将影响治疗效果,甚至危及生命。

(2)发生机制:缺血性损伤是再灌注损伤发生、发展的基础,再灌注恢复供血后产生大量氧自由基、细胞内Ca\textsuperscript{2+}
超载、白细胞炎性反应作用及高能磷酸化合物缺乏等原因直接引起心肌细胞损伤、死亡及微循环出现无复流现象加重心肌缺血性损伤。

(3)影响因素:①缺血时间:再灌注损伤易发生在缺血性心肌可逆性损伤期内(一般在缺血15~45min后发生的再灌注);②侧支循环:急性心肌梗死后,如易于建立侧支循环,则不易发生再灌注损伤;③再灌注条件:如低压、低温(25℃)、低pH值、低钠、低钙液灌流,可使再灌注损伤减轻、心功能迅速恢复,反之,则可诱发或加重再灌注损伤;④缺血范围:当缺血面积>20\%时,再灌注损伤发生率高;⑤再灌注的血流速度:当血流速度快,冲洗作用强时,其发生率就高;⑥再灌注区可逆性心肌细胞数量多时,其发生率高。

(4)临床及心电图特征:①临床症状(如胸痛等)持续加重或缓解后出现反弹和加重、心肌酶谱持续增高;②ST段持续性抬高、进行性抬高或回落后再次抬高(>0.1mV);③出现再灌注性心律失常;④心肌坏死面积增加导致异常Q波出现的导联数增多,再灌注治疗后6h或第1天的死亡率增加。

3.再灌注性心律失常

再灌注性心律失常的发生率高达80\%,以心肌血供中断15~45min后的再灌注,特别是再灌注后的5min内,心律失常发生率最高,以非阵发性室性心动过速(或加速的室性逸搏心律)、成对室性早搏、短阵性室性心动过速多见,严重者可发生心室颤动而死亡,也可出现缓慢性心律失常,如窦性心动过缓、窦性停搏及房室传导阻滞等。

再灌注性心律失常的发生机制有:①可逆性心肌细胞与正常心肌细胞之间电生理异常引起折返;②再灌注时的冲洗现象,使堆积的乳酸、儿茶酚胺入血,引起自律性增高;③细胞内Ca\textsuperscript{2+}
超载,引起触发活动。

4.持续性ST段抬高的临床意义

(1)再灌注治疗早期ST段未能快速回落,持续在较高水平,是心肌水平未得到再灌注的表现。

(2)ST段进行性抬高伴临床症状加重,提示病情进展,可能存在梗死灶延伸、毗邻梗死区再梗死或再灌注性损伤。

(3)24h后ST段再抬高,应警惕再梗死的发生。

(4)ST段持续抬高2周以上,应警惕室壁瘤形成的可能(图\ref{fig44-17})。

\begin{figure}[!htbp]
 \centering
 \includegraphics[width=4.4375in,height=1.25in]{./images/Image00725.jpg}
 \captionsetup{justification=centering}
 \caption{女性,84岁,前间壁、前壁心肌梗死1年余。显示完全性右束支阻滞、前间壁及前壁异常Q波伴ST段抬高,提示室壁瘤形成(被心脏超声波证实)}
 \label{fig44-17}
  \end{figure} 

\protect\hypertarget{text00052.htmlux5cux23subid632}{}{}

\subsection{心肌梗死面积的心电图评估}

1.QRS计分法(Wanger法)

(1)原理:急性心肌梗死后异常Q波和R波改变的导联数量越多,则梗死面积越大。QRS计分法就是利用常规12导联心电图加权计分系统评价心肌梗死面积总百分比的方法。

(2)基本要求:①必须是室上性节律,心室率<120次/min;②基线稳定,足以准确测量QRS波群各波的时间和电压;③必须是12导联同步记录,至少是6个导联同步;④Q波时间准确测量极为重要。

(3)基本方法:先除去Ⅲ、aVR导联,计算剩余10个导联的Q波和R/Q振幅比的变化而计分(表44-1),将计分结果代入以下公式计算:前壁梗死面积(\%)=3.6×计分值+3.2,下壁梗死面积(\%)=2.5×计分值+2.9。

\begin{table}[htbp]
\centering
\caption{简化的心肌梗死面积QRS计分法}
\label{tab44-1}
\includegraphics[width=6.19792in,height=4.42708in]{./images/Image00726.jpg}
\end{table}

(4)评价:QRS计分法需在梗死后1周、梗死范围相对稳定时应用,对前壁急性心肌梗死面积的评估最佳,其次为下壁、后侧壁。QRS计分与急性心肌梗死患者预后显著相关,当QRS计分≥10分时(约30\%左心室梗死),患者1月、1年的病死率增加。QRS计分对前壁陈旧性梗死面积也有较好的评估价值。

2.ST段总抬高计分法(Aldrich计分法)

(1)方法:计算ST段抬高的导联数及抬高值的总和,对梗死面积进行评估。前壁梗死面积(\%)=3×(1.5×ST段抬高导联-0.4),下壁梗死面积(\%)=3×(0.6×下壁导联ST段抬高值的总和+2)

(2)评价:Aldrich计分法简便实用,对梗死面积有半定量的价值,可预测左心室功能、心肌再灌注治疗的疗效。

3.综合ST段抬高、QRS波和T波改变(Wikins法)

(1)方法:计算ST段抬高的导联数、抬高值的总和、异常Q波的导联数、异常Q波时间总和及T波高耸的导联数。其中异常Q波标准为Ⅰ、Ⅱ、aVL、aVF、V\textsubscript{5}
、V\textsubscript{6} ≥0.03s,V\textsubscript{4}
≥0.02s,V\textsubscript{1} ~V\textsubscript{3}
导联出现Q波;T波高耸标准为Ⅱ≥0.6mV,Ⅲ、aVL≥0.3mV,Ⅰ、aVF、V\textsubscript{1}
≥0.45mV,V\textsubscript{2} ≥1.1mV,V\textsubscript{3}
≥1.5mV,V\textsubscript{4} ≥1.2mV,V\textsubscript{5}
≥0.9mV,V\textsubscript{6}
≥0.65mV。前壁梗死范围=1.88×ST段抬高导联数+2.19×异常Q波导联数+1.56×T波高耸导联数+5.8,下壁梗死范围=0.94×ST段抬高总和+1.13×Q波时间总和+8.73。

(2)评价:对前壁梗死面积的评估价值较大,对下壁梗死面积的评估需进一步完善。

4.根据梗死灶大小、临床及病理所见,可将心肌梗死分为局灶性梗死(显微镜下梗死)、小面积梗死(<左心室的10\%)、中面积梗死(左心室的10\%~30\%)、大面积梗死(>左心室的30\%)。

\protect\hypertarget{text00052.htmlux5cux23subid633}{}{}

\subsection{心电图及其相关检查判断急性心肌梗死病情及预后的价值}

急性心肌梗死患者的病情及预后,主要取决于梗死的范围和部位,这又受以下5个因素的影响:①冠状动脉阻塞部位及其持续时间;②开始时心肌缺血的范围;③缺血区侧支循环建立情况;④缺血区心肌代谢情况;⑤早期再灌注治疗的有效性和有无再灌注损伤出现。常规心电图检查具有简便、快捷、经济、无创及可反复检查等优点,不仅可以确定急性心肌梗死的诊断,还能判断梗死部位,并进行分期,根据心电图演变情况可对心肌梗死患者的病情及预后进行评估,指导临床治疗。

1.急性心肌梗死患者病情重、预后差的心电图表现

有下列心电图改变者,提示急性心肌梗死患者病情重、预后差:

(1)出现墓碑型ST段抬高:该心肌梗死以老年人多发。均发生在穿壁性心肌梗死中,入院1周内并发症多,如循环衰竭、严重心律失常、三度房室传导阻滞/束支阻滞、心肌梗死后心绞痛及扩展明显增多,死亡率显著增高,是急性心肌梗死近期预后险恶的独立指标。

(2)急性心肌梗死出现新发的左束支阻滞、不定型心室内传导阻滞、房室传导阻滞:并不是新发生的各种传导阻滞使患者的病情迅速加重,而是新发生的各种传导阻滞代表着病情在进展、梗死面积在扩大。因此,新发生的各种传导阻滞都提示左心室前壁和前降支受累、梗死还在发展和面积在扩大、患者的预后差,死亡率可增加40\%~60\%,心源性休克的发生率高达70\%以上。

(3)急性前壁心肌梗死出现新发的右束支阻滞:为大面积心肌梗死的表现,常伴有心力衰竭、三度房室传导阻滞、心室颤动和高死亡率。

(4)急性前壁心肌梗死伴任一导联ST段压低:梗死后发生再梗死、心力衰竭、室性心律失常等心脏事件增多,其远期病死率高。

(5)急性前壁心肌梗死伴ST段持续抬高、T波直立;前壁急性梗死后2~5天内ST段仍持续抬高伴T波直立,高度提示左心室内有血栓形成(敏感性96\%,特异性93\%)。

(6)急性前侧壁心肌梗死伴aVR导联ST段压低:提示心肌梗死面积较大,CK-MB峰值较高,心功能较差,LVEF≤35\%,心力衰竭发生率高。

(7)急性下壁心肌梗死伴左胸导联(V\textsubscript{4} ~V\textsubscript{6}
)ST段压低:多伴有前降支病变,且右冠状动脉近端阻塞及合并三支冠状动脉病变发生率高,为冠状动脉病变严重而广泛且侧支循环差的表现;同时,其左心室功能也差,并发症亦多,预后较差,住院期间的病死率可达41\%。

(8)急性下壁心肌梗死患者在下壁导联出现高耸T波而ST段抬高不明显或抬高<0.1mV,同时出现左胸导联ST段压低,表明既往有过梗死或同时伴有前壁心肌缺血,属极高危型患者,住院期间的病死率可高达69\%。由此可见,下壁心肌梗死患者预后主要取决于前壁心肌缺血情况。

(9)下壁急性心肌梗死伴aVR导联ST段压低:表明心肌梗死面积较大,CK-MB峰值较高,住院率和1年并发症发生率增高。

(10)广泛导联出现既宽又深的异常Q波,表明梗死范围广、厚度深呈透壁性梗死,易形成室壁瘤或心脏破裂而猝死,如广泛前壁心肌梗死。

(11)出现持续性或进行性ST段抬高:早期见于梗死灶延伸、毗邻梗死区再梗死或再灌注性损伤,提示病情进展或进行性加重;若持续抬高2周以上,提示室壁瘤形成,容易导致心功能不全、恶性室性心律失常、血栓形成等多种并发症,严重威胁患者的生命。

(12)急性心肌梗死半年后T波仍持续倒置:预示透壁性坏死,左心室功能恢复差,远期预后差。

(13)再灌注治疗后出现持续性ST段再抬高:是心肌再次损伤的标志,见于冠状动脉再闭塞、梗死面积扩大、再灌注损伤及侧支循环较差等情况,再灌注治疗后6h或第1天的死亡率增加,左心室功能恢复较差,远期病死率增加。

(14)再发性心肌梗死:患者左心室功能恢复较差,近期与远期病死率均增加。

(15)急性心肌梗死伴T波电交替:多见于心肌缺血、心功能不全、电解质紊乱等患者。有T波电交替者,发生致命性室性心律失常的危险性增加14倍。T波电交替已成为识别高危患者的一个重要而非常直观的指征。

(16)出现严重的心室内传导阻滞(QRS波群时间>0.16s):当窦性QRS波群呈左、右束支阻滞型或不定型心室内传导阻滞时或室性异位搏动的QRS波群时间>0.16s,称为特宽型QRS波群。QRS波群宽度与心室负荷程度及心肌病变严重程度相关,具有诊断及预后的意义,多见于严重的器质性心脏病患者,尤其是老年冠心病患者。现已证明,完全性左束支或右束支阻滞,均为独立的危险因素。

(17)出现严重的快速性心律失常:各种类型的室性心动过速、阵发性室上性心动过速、心房颤动或扑动伴极快的心室率等,最终易导致心室扑动、颤动而猝死。

(18)出现严重的缓慢性心律失常:病窦综合征、持久性或阵发性三度房室传导阻滞伴心室停搏,尤其是较长时间的心室停搏(>5.0s)或短时间内出现高频度的心室停搏等,易发生阿-斯综合征而猝死。

(19)出现复杂性室性心律失常:频发成对的、多源性、多形性、特宽型(时间>0.16s)、特矮型(振幅<1.0mV)及Ron-T、Ron-P的室性早搏,易诱发室性心动过速或心室颤动而危及生命。

(20)出现严重的慢-快型综合征:在各种缓慢性心律失常的基础上,出现阵发性心房颤动、扑动、室上性心动过速、室性心动过速等快速性心律失常,易导致心力衰竭或加重心力衰竭。

(21)出现严重的快-慢型综合征:阵发性心房颤动、扑动、室上性心动过速、室性心动过速等快速性心律失常发作终止时,在恢复窦性心律之前,出现长R-R间歇,易发生晕厥、阿-斯综合征而猝死。

(22)出现心室电分离现象:多见于垂危心脏病患者的临终期或严重器质性心脏病患者,是一种不可逆的病理现象,它使血流动力学及冠状动脉灌注严重恶化,进而导致心肌缺血,在心肌的不同层次发生碎裂波,表现心电离散。故心室分离提示心肌病变严重而广泛,预后极差。

(23)心室晚电位阳性:表明心室内存在潜在的折返环,是产生折返性室性心动过速的电生理基础,易引发致命性室性心律失常,对心肌梗死患者的预后预测、冠心病、心力衰竭患者猝死危险性预测有重要意义。

(24)心率变异性(HRV)异常:自主神经系统与心源性猝死密切相关,心电稳定性有赖于交感、副交感神经和体液调节之间的平衡。若交感神经张力过度增高,则有利于致命性心律失常的发生;而副交感神经激活,则具有保护心脏和抗心室颤动作用。其中SDNN反映交感与副交感神经总的张力大小,SDANN、SDNN\textsubscript{index}
值降低,表明交感神经张力增高;而r-MSSD、PNN\textsubscript{50}
值降低,则表明副交感神经张力降低,如SDNN<50ms者的死亡相对危险性高出SDNN>100ms者5倍;心肌梗死后6~12个月HRV仍不能恢复正常者,则提示预后不佳。

(25)窦性心律震荡现象不明显或消失:见于心肌梗死后猝死的高危患者。震荡初始(TO)、震荡斜率(TS)指标对猝死高危患者预测作用稳定而可靠,两者均异常时,是猝死最敏感的预测指标,其阳性预测精确度达32\%,同时阴性预测精确度达90\%。

(26)心脏变时性功能不全:运动试验中无ST段压低而有变时性功能不全者,经冠状动脉造影,72\%患者有明显的冠状动脉病变;运动试验中有ST段压低伴变时性功能不全者,冠状动脉三支病变的发生率高于仅有ST段压低者。提示运动试验中变时性功能不全是诊断冠心病的一个独立而敏感的阳性指标,也是冠心病事件(如心绞痛、心肌梗死、猝死)发生风险及预后判断指标之一。

2.急性心肌梗死预后较好的心电图表现

(1)急性前壁心肌梗死伴V\textsubscript{4} ~V\textsubscript{6}
导联U波倒置:约30\%急性前壁梗死患者出现V\textsubscript{4}
~V\textsubscript{6}
导联U波倒置,与无U波倒置患者比较,前者心肌坏死面积较少,左心室功能较好,故急性前壁梗死时出现U波倒置是预后较好的一个心电图指标,与侧支循环较丰富有关。

(2)急性前壁心肌梗死伴V\textsubscript{4} ~V\textsubscript{6}
导联巨倒T波:急性前壁梗死后5天内,V\textsubscript{4}
~V\textsubscript{6}
导联如出现巨倒T波(深度≥1.0mV),则预示有R波重现可能和较好的左心室功能,预后较佳。

(3)急性心肌梗死早期再灌注治疗后2h内抬高ST段快速回落≥50\%或完全回落(回落≥70\%或ST段抬高<0.1mV),是心肌组织水平再灌注的客观指标,ST段快速回落或完全回落所需时间愈短,回落幅度愈大,则心肌组织水平再灌注愈好,左心室收缩功能恢复愈佳,近、远期死亡率愈低。

(4)再灌注治疗后ST段抬高导联未出现异常Q波或q波较小较浅。

(5)再灌注治疗后24h内ST段抬高导联出现早期T波倒置:是心肌组织再灌注成功的心电图表现,是梗死相关动脉再通的独立指标,并与住院期间存活率相关;T波倒置愈深,提示有较多的心肌获救,心功能恢复较好,是慢性期左心室壁运动异常恢复的预测指标。

(6)心肌梗死后,倒置T波转为直立的时间越早,则左心室功能恢复越好,预后越佳。

\protect\hypertarget{text00052.htmlux5cux23subid634}{}{}

\subsection{急性心肌梗死鉴别诊断}

(1)急性肺栓塞:急性肺栓塞临床上可出现胸痛、呼吸困难,心电图出现S\textsubscript{Ⅰ}
Q\textsubscript{Ⅲ} T\textsubscript{Ⅲ} 型及V\textsubscript{1}
~V\textsubscript{3}
导联ST段抬高、T波倒置,应与急性下壁、前间壁心肌梗死相鉴别。但前者常出现窦性心动过速、肺型P波、电轴右偏、显著的顺钟向转位及一过性右束支阻滞,且ST段抬高程度较轻,心肌酶谱正常或轻度增高,而后者ST段明显抬高,心肌酶谱明显升高。

(2)急性心包炎:患者有胸痛、ST段抬高,需与急性心肌梗死相鉴别。具体鉴别请见第四十二章第七节急性心包炎。

(3)变异型心绞痛:患者有胸痛、硝酸甘油不能缓解,ST段抬高伴T波高耸,酷似急性心肌梗死,但变异型心绞痛用Ca\textsuperscript{2+}
拮抗剂治疗有效,随着症状的缓解,ST-T逐渐恢复正常,心肌酶谱正常范围。若病情进一步发展,且不能缓解,则很可能发展为急性心肌梗死。

(4)早复极综合征合并变异型心绞痛:患者有胸痛,硝酸甘油不能缓解,ST段显著抬高伴T波高耸,酷似急性心肌梗死,但前者心肌酶谱正常范围,Ca\textsuperscript{2+}
拮抗剂治疗有效。

(5)早复极综合征:患者有ST段抬高伴T波高耸,若伴有其他原因引起的胸痛,有时易误诊为变异型心绞痛或急性心肌梗死,但前者多见于年轻身体素质良好者,平时心率较慢,活动后或心率加快后ST段抬高程度减轻或恢复正常,心肌酶谱正常。

(6)急性重症心肌炎(暴发型心肌炎):少数重症心肌炎患者起病急骤,病情凶险,出现异常Q波、ST段呈损伤型抬高、心肌酶谱增高酷似急性心肌梗死。一般地说,年轻患者,发病前有感染史,既往无心脏病史,以暴发型心肌炎可能性为大,冠状动脉造影有助两者的鉴别。

(7)肥厚型心肌病:部分患者出现异常Q波、显著ST段压低及T波倒置,类似冠状T波,酷似急性心内膜下心肌梗死,但前者异常Q波多表现为深而窄的Q波,时间不增宽,心肌酶谱正常,心脏超声波检查可资鉴别。

(8)心脏肿瘤:较大的心脏肿瘤其所对应的导联可出现异常Q波、ST段抬高酷似急性心肌梗死,心肌酶谱、肌钙蛋白、心脏超声波及核磁共振检查可资鉴别(图\ref{fig44-18})。

\begin{figure}[!htbp]
 \centering
 \includegraphics[width=5.04167in,height=2.61458in]{./images/Image00727.jpg}
 \captionsetup{justification=centering}
 \caption{女性,70岁,心脏超声波诊断右心室肿瘤。心电图显示房室传导延缓(P-R间期0.22s)、不完全性右束支阻滞、V\textsubscript{3}R~V\textsubscript{6} R、V\textsubscript{1} ~V\textsubscript{4}导联持续性ST段抬高、前侧壁轻度ST段改变、下壁轻度T波改变}
 \label{fig44-18}
  \end{figure} 


(9)高钾血症:血钾过高可导致部分心肌细胞膜的静息电位低于阈电位而出现“电静止”现象,产生可逆性异常Q波;此外,血钾过高可产生损伤电流样改变出现ST段抬高,酷似急性心肌梗死的心电图改变(图\ref{fig44-19})。但当血钾恢复正常后,异常改变的心电图即恢复正常而有别于急性心肌梗死特有的ST-T动态演变规律。

\begin{figure}[!htbp]
 \centering
 \includegraphics[width=5.78125in,height=2.27083in]{./images/Image00728.jpg}
 \captionsetup{justification=centering}
 \caption{男性,68岁,尿毒症、高钾血症(8.6mmol/L)患者。心电图显示显著的窦性心动过缓伴窦-室传导、左后分支阻滞(电轴由原来的+58°增至+107°)、不定型心室内传导阻滞、前间壁异常Q波伴ST段损伤型抬高、T波高耸,符合高钾血症的心电图改变。治疗后血钾恢复正常,心电图亦恢复正常(引自张茜)}
 \label{fig44-19}
  \end{figure} 

(10)Brugada综合征或Brugada波:V\textsubscript{1} ~V\textsubscript{3}
导联ST段呈“穹隆型”或“马鞍型”抬高伴T波倒置或正负双向酷似急性前间壁心肌梗死,但Brugada综合征或Brugada波有家族性遗传特点,多见于年轻人,一般情况尚好,心肌酶谱正常可资鉴别。

\protect\hypertarget{text00053.html}{}{}

\protect\hypertarget{text00053.htmlux5cux23chapter53}{}{}

\chapter{电解质异常的心电图改变}

\protect\hypertarget{text00053.htmlux5cux23subid635}{}{}

\section{电解质与心肌细胞特性的关系}

心肌细胞具有自律性、兴奋性、传导性、收缩性和舒张性5种生理特性,其中前三者属于电生理特性,是以心肌细胞膜的生物电活动为基础,由细胞内外各种离子不均匀分布及其跨膜运动所决定,与心电产生及心律失常发生有密切关系;而后两者则属于机械特性,与心脏泵血功能有关。

1.自律性

自律性是指心脏起搏细胞(主要为窦房结细胞和浦肯野细胞)自动发生节律性兴奋的特性。自律性的高低主要取决于4相舒张期自动除极化速率、最大舒张期电位及阈电位水平,用每分钟发放冲动的次数来衡量。凡是能加快4相自动除极化速率、缩小最大舒张期电位与阈电位水平之间的距离,均能提高自律性,如加快起搏细胞的Ca\textsuperscript{2+}
、Na\textsuperscript{+} 内流或延缓K\textsuperscript{+}
外流的因素;反之,均能降低自律性。

2.兴奋性

兴奋性是指心肌细胞受到刺激时产生兴奋的能力。用刺激阈值来衡量兴奋性的高低。刺激阈值高,表示兴奋性低;反之,则表示兴奋性高。它主要取决于细胞膜的静息电位或最大舒张电位的水平及引起0相除极化的离子通道性状。凡是能缩小细胞膜静息电位或最大舒张电位与阈电位水平之间的距离及增加静息状态的Na\textsuperscript{+}
通道数量(快反应细胞)、L型Ca\textsuperscript{2+}
通道数量(慢反应细胞),均能提高心肌细胞的兴奋性;反之,则降低心肌细胞的兴奋性。

心肌细胞兴奋性具有下列周期性改变:有效不应期、易颤期(仅指心房肌、心室肌)、相对不应期、超常期及应激期。

(1)绝对不应期(ARP)与有效不应期(ERP):前者是指从动作电位的0相开始到复极3相膜电位降至-55mV这一段时期,后者是指从动作电位的0相开始到复极3相膜电位降至-60mV这一段时期,对任何刺激均不发生反应。相当于QRS波群、ST段及T波顶峰之前的时间。

(2)相对不应期(RRP):是指有效不应期之后,膜电位从-60mV继续复极化到-80mV这一段时期,此时需要阈上刺激才能发生反应。相当于T波顶峰至T波结束的时间。

(3)易颤期:指在绝对不应期与相对不应期之间,各心肌细胞兴奋性的恢复不一致或不同步,此时若受到强刺激,极易发生纤维性颤动,称为易颤期。心房易颤期相当于R波降肢和S波时间内,心室易颤期相当于T波上升肢到顶峰前20~40ms,或在T波顶峰前30ms,约持续20~60ms。

(4)超常期:指相对不应期之后,膜电位从-80mV继续复极化到-90mV这一段时期,此时阈下刺激即能引起反应,称为超常期。相当于心电图中T波顶峰至U波结束的这段时间。

决定不应期长短的因素有:①膜电位水平;②心率因素:心率增快使心房肌、心室肌、房室旁道的不应期缩短,而房室结的不应期则在心率增快到一定程度时却反而延长;③解剖部位:房室结的不应期>心室肌>心房肌,右束支的不应期>左前分支>左束支>左后分支>左中隔支,90\%房室旁道的不应期>房室结(易出现顺向型房室折返性心动过速),90\%房室结快径路的不应期>慢径路(易出现慢-快型房室结内折返性心动过速);④年龄、性别:女性、年长者不应期长;⑤神经因素:迷走神经张力增高使房室结不应期延长,心房肌不应期缩短,而心室肌影响不大,交感神经张力增高,则使房室结不应期缩短;⑥药物因素。

3.传导性

传导性是指心肌细胞具有传导兴奋的能力。用传播速度来衡量传导性的高低。它主要取决于心肌细胞结构特点和电生理特性,前者与心肌细胞直径、细胞内的电阻大小及细胞间缝隙连接数量和功能状态有关,凡是细胞直径大、电阻小、缝隙连接数量多及处于开放状态,都能加快兴奋的传导;而后者则与0相除极化的速度和幅度及邻近未兴奋部位细胞膜的反应性有关,凡是能增加细胞膜的反应性、加大静息电位或最大舒张期电位水平(指负值加大)、降低阈电位水平(指负值加大,阈电位下移)及减少膜电阻和膜电容,均可提高兴奋传导的速度。

传导性根据动作电位时相可分为两类:

(1)0相传导:指邻近的心肌组织凭着0相除极所产生的电位差和电流依次除极的过程。

(2)2相传导:当部分心肌组织2相平台期消失,出现2相复极时的电位差和电流,引起邻近细胞依次除极的过程,可出现2相早搏、2相折返等心律失常,如Brugada综合征、特发性心室颤动患者发生的致命性心律失常都与2相传导、2相折返有关。

传导性根据电生理特性也可分为两类:

(1)快反应纤维(快反应细胞):含有快Na\textsuperscript{+}
通道,能快速传导冲动的心肌细胞,包括心房肌、心室肌、特殊的传导组织如结间束、希氏束、浦肯野纤维。其电生理特征:①静息膜电位负值大,约-80~-90mV;②0相上升速率快、幅度大,传导速度快;③含有快Na\textsuperscript{+}
通道,0相除极时离子流为Na\textsuperscript{+} ;④除极时阈电位为-65mV。

(2)慢反应纤维(慢反应细胞):含有慢Ca\textsuperscript{2+}
通道,缓慢传导冲动的心肌细胞,包括窦房结、房室结、冠状窦口邻近的心肌细胞。其电生理特征:①静息膜电位负值小,约-60~-70mV;②0相上升速率慢、幅度小,传导速度慢;③含有慢Ca\textsuperscript{2+}
通道,0相除极时离子流为Ca\textsuperscript{2+}
;④除极时阈电位为-30~-40mV。

4.收缩性和舒张性

心肌细胞兴奋时,通过兴奋-收缩耦联机制,触发心肌细胞收缩和随后的舒张,并与瓣膜的启闭相配合,造成心房和心室压力和容积的变化,从而推动血液在心血管系统内流动。心肌兴奋-收缩耦联主要与细胞内外Ca\textsuperscript{2+}
有关,但当血K\textsuperscript{+}
明显增高时,心房肌、心室肌将出现停止收缩而处于舒张状态。

5.动作电位

当心肌细胞受到刺激而兴奋时,细胞膜对离子的通透性发生了一系列的变化,出现一系列的离子跨膜运动,使膜内外的电位差发生迅速变化,称为动作电位。它包括除极化与复极化两个过程。每1次动作电位可分为5个时相:①0相除极化:与快Na\textsuperscript{+}
通道开放有关;②1相复极化:为快速复极化初期,与快Na\textsuperscript{+}
通道失活后,K\textsuperscript{+}
外流有关;③2相复极化:为缓慢复极化期,又称为平台期,与K\textsuperscript{+}
外流、Ca\textsuperscript{2+} 内流及少量Na\textsuperscript{+}
内流有关;④3相复极化:为快速复极化末期,与L型Ca\textsuperscript{2+}
通道失活后细胞膜对K\textsuperscript{+}
通透性急剧升高引起K\textsuperscript{+}
外流明显加快有关;⑤4相:又称为电舒张期、静息期或极化期,通过细胞膜上Na\textsuperscript{+}
-K\textsuperscript{+} 泵、Ca\textsuperscript{2+}
泵及Na\textsuperscript{+} -Ca\textsuperscript{2+}
交换体的活动,将细胞内的Na\textsuperscript{+} 、Ca\textsuperscript{2+}
排出,并将细胞外的K\textsuperscript{+}
摄入细胞内,以恢复细胞内外各种离子的正常浓度梯度,维持心肌细胞的正常兴奋性。

无论是心肌细胞的动作电位,还是自律性、兴奋性、传导性及收缩性都与Na\textsuperscript{+}
、K\textsuperscript{+} 、Ca\textsuperscript{2+}
等各种离子有关。一旦发生电解质紊乱,势必会影响心肌细胞的生理特性,引发各种心律失常、传导阻滞及心肌收缩性降低。心电图检查可为临床诊断、治疗提供重要价值。

\protect\hypertarget{text00053.htmlux5cux23subid636}{}{}

\section{血钾异常的心电图改变}

\protect\hypertarget{text00053.htmlux5cux23subid637}{}{}

\subsection{低钾血症}

1.基本概念

低钾血症是指血清钾浓度<3.5mmol/L的一种病理状态。可因钾摄入不足、排出过多或因稀释及转移到细胞内而导致血清钾浓度降低。

2.低钾血症对心肌细胞电生理的影响

正常人体内的K\textsuperscript{+}
主要分布在细胞内,为细胞外K\textsuperscript{+}
浓度的30倍。当血K\textsuperscript{+}
稍有减少(低钾血症早期),即可使细胞内、外K\textsuperscript{+}
浓度差更加显著,使细胞膜静息电位负值增大,细胞处于超极化状态,静息电位与阈电位之间的距离增大,导致心肌细胞的自律性和兴奋性降低;但随着血K\textsuperscript{+}
的进一步降低,细胞膜对K\textsuperscript{+}
的通透性降低,最终结果是静息电位负值轻度减少,一方面使0相除极化速度和幅度下降,传导性略降低,出现轻度的传导阻滞(如心房内传导阻滞、房室传导阻滞、束支阻滞等),另一方面缩短了与阈电位水平之间的距离,使心肌细胞自律性与兴奋性均增高;此外,由于3相阶段K\textsuperscript{+}
逸出减慢,复极时间延长,导致动作电位时间延长,超常期延长,尤其是浦肯野纤维动作电位时间延长的程度超过心室肌,使Q-T间期延长、T波低平及U波增高,也导致浦肯野纤维与心室肌的复极离散度增大,有利于产生早期后除极及折返而引起室性心律失常。

综上所述,低钾血症早期,心肌细胞自律性和兴奋性降低,而传导速度影响不明显;当缺钾进一步加重时,则使心肌细胞自律性和兴奋性增高,传导速度减慢,出现传导阻滞、心室内折返现象及早期后除极而引发室性心律失常。

3.心电图特征

(1)U波增高,T-U波融合,U波振幅>0.2mV或U波振幅>T波振幅,血K\textsuperscript{+}
越低,U波改变越明显,甚至出现巨大U波。

(2)T波增宽伴切迹,振幅降低。

(3)ST段多呈下斜型压低。

(4)Q-T间期或Q-U间期延长。

(5)心律失常:以多源性、多形性室性早搏、短阵性室性心动过速多见,有时出现尖端扭转型室性心动过速等恶性室性心律失常。

(6)传导阻滞:可出现不完全性心房内传导阻滞、房室传导阻滞、束支阻滞等(图\ref{fig45-1}、图\ref{fig45-2}、图\ref{fig45-3})。

\begin{figure}[!htbp]
 \centering
 \includegraphics[width=5.58333in,height=1.72917in]{./images/Image00729.jpg}
 \captionsetup{justification=centering}
 \caption{周期性麻痹、低钾血症患者(血K\textsuperscript{+}3.1mmol/L),出现窦性心律伴P电轴左偏、T波及U波改变、Q-T间期延长、符合低钾血症的心电图改变}
 \label{fig45-1}
  \end{figure} 


\begin{figure}[!htbp]
 \centering
 \includegraphics[width=5.58333in,height=1.09375in]{./images/Image00730.jpg}
 \captionsetup{justification=centering}
 \caption{低钾血症患者(血K\textsuperscript{+}2.9mmol/L),V\textsubscript{5}导联出现二度Ⅰ型房室传导阻滞呈3:2~5:4传导、T波与U波融合、Q-U间期延长}
 \label{fig45-2}
  \end{figure} 


\begin{figure}[!htbp]
 \centering
 \includegraphics[width=5.79167in,height=2.28125in]{./images/Image00731.jpg}
 \captionsetup{justification=centering}
 \caption{周期性麻痹、低钾血症患者(血K\textsuperscript{+}2.6mmol/L),出现三支阻滞(完全性右束支阻滞、左前分支阻滞、左后分支二度阻滞呈3:1传导)、肢体导联左心室高电压、T波及U波改变、Q-T间期延长、符合低钾血症的心电图改变}
 \label{fig45-3}
  \end{figure} 


\protect\hypertarget{text00053.htmlux5cux23subid638}{}{}

\subsection{高钾血症}

1.基本概念

高钾血症是指血清钾浓度>5.5mmol/L的一种病理状态。多见于急、慢性肾功能衰竭、溶血性疾病、挤压综合征、大面积烧伤、输血过多等。一旦出现高钾血症,预后严重,如不及时处理,常危及生命。

2.高钾血症对心肌细胞的影响

(1)对心肌细胞电生理的影响:血K\textsuperscript{+}
增高,使细胞内外K\textsuperscript{+} 浓度差减小,K\textsuperscript{+}
平衡电位减小,导致细胞膜静息电位负值减小及增加K\textsuperscript{+}
电导,使细胞膜对K\textsuperscript{+}
通透性增加,由此产生以下影响:①心肌细胞兴奋性先增高后降低:即血K\textsuperscript{+}
轻度增高,对阈电位水平影响不大时,膜电位与阈电位距离缩短,心肌细胞兴奋性增高;但随着血K\textsuperscript{+}
的进一步增高,膜电位负值减小到一定程度时,Na\textsuperscript{+}
通道失活,阈电位水平上移,兴奋阈值升高,导致心肌细胞兴奋性降低。②传导速度减慢:静息电位负值减小,Na\textsuperscript{+}
通道失活增多,0相除极化上升速度和幅度均下降,使传导性降低,出现各种传导阻滞。③快反应细胞自律性降低:因细胞膜对K\textsuperscript{+}
通透性增加,使K\textsuperscript{+}
外流速度加快,导致4相自动除极化速率减慢。④动作电位时程缩短:因细胞膜对K\textsuperscript{+}
通透性增加,使3相复极化速度加快,时间缩短,导致动作电位时程缩短,出现T波高耸、Q-T间期缩短。

(2)对心肌细胞收缩性的影响:血K\textsuperscript{+}
增高,抑制心肌的收缩性。当血K\textsuperscript{+}
>8mmol/L时,心房肌处于麻痹状态,出现窦-室传导;当血K\textsuperscript{+}
>10mmol/L时,心脏将出现停搏。

3.心电图特征

(1)“帐篷状”T波及Q-T间期缩短:当血K\textsuperscript{+}
>5.5mmol/L时,以R波为主的导联便出现T波高尖、两肢对称、基底部狭窄呈“帐篷状”,同时伴Q-T间期缩短,为高钾血症最早期的特征性改变(图\ref{fig45-4})。

\begin{figure}[!htbp]
 \centering
 \includegraphics[width=3.38542in,height=2.84375in]{./images/Image00732.jpg}
 \captionsetup{justification=centering}
 \caption{急性肾功能衰竭、高钾血症患者(血钾6.2mmol/L)出现P波振幅降低、帐篷状T波}
 \label{fig45-4}
  \end{figure} 

(2)各种传导阻滞:当血K\textsuperscript{+}
>6.5mmol/L时,可出现窦房传导阻滞、心房内传导阻滞、房室传导阻滞、束支阻滞及不定型心室内传导阻滞等。

(3)P波振幅渐低、时间渐宽,直至消失,出现窦-室传导:当血K\textsuperscript{+}
>8.0mmol/L时,冲动在心房内的传导、除极均受到抑制,直至心房麻痹,但窦性冲动仍可通过结间束、房间束传至房室交接区直至心室,形成窦-室传导。

(5)QRS-T波群融合形成正弦波:当血K\textsuperscript{+}
>10mmol/L时,QRS波群振幅明显降低、时间更宽,T波振幅反趋降低而圆钝,两者融合形成正弦波;频率缓慢而不规则,Q-T间期延长,直至出现心脏停搏或心室扑动、颤动而死亡(图\ref{fig45-5})。

\begin{figure}[!htbp]
 \centering
 \includegraphics[width=5.1875in,height=1.36458in]{./images/Image00733.jpg}
 \captionsetup{justification=centering}
 \caption{糖尿病、酮症酸中毒、高钾血症患者(血钾8.6mmol/L)出现QRS-T波群融合形成正弦波、心室颤动(引自朱同新)}
 \label{fig45-5}
  \end{figure} 

(6)偶尔可使心房颤动暂时转为窦性节律及出现异常Q波、ST段抬高和J波酷似急性心肌梗死(图\ref{fig44-19})或Brugada波等。

4.血K\textsuperscript{+} 浓度异常与心电图改变的关系

血K\textsuperscript{+}
浓度高低并不一定与心电图改变平行一致。因心电图改变取决于心肌细胞内K\textsuperscript{+}
含量,血清钾测定并不能及时真实地反映心肌细胞内K\textsuperscript{+}
含量,如急性失钾时,血钾已降低,但心电图检查无异常改变;又如慢性失钾时,由于细胞内K\textsuperscript{+}
释放到细胞外,血钾测定可在正常范围内,但心电图检查已显示低钾血症改变。此外,Na\textsuperscript{+}
、Ca\textsuperscript{2+}
等电解质及酸碱平衡失调亦可改变钾对心肌的影响,如低钠血症、低钙血症、酸中毒可加重高钾血症异常的心电图改变。

5.鉴别诊断

高钾血症早期出现的T波高耸、Q-T间期缩短,需与早复极综合征、短Q-T间期综合征、超急性期心肌梗死、左心室舒张期负荷过重、脑血管意外、变异型心绞痛等引起T波高耸相鉴别。

\protect\hypertarget{text00053.htmlux5cux23subid639}{}{}

\section{血钙异常的心电图改变}

\protect\hypertarget{text00053.htmlux5cux23subid640}{}{}

\subsection{低钙血症}

当血清钙<1.75mmol/L时,便称为低钙血症。常见于慢性肾功能衰竭、甲状旁腺功能减退等。低钙血症主要引起动作电位平台期Ca\textsuperscript{2+}
内流减慢使2相时间延长。心电图表现为ST段呈水平型延长(>0.16s)、Q-T间期延长(图\ref{fig45-6})。需注意与心内膜下心肌缺血相鉴别。

\begin{figure}[!htbp]
 \centering
 \includegraphics[width=5.8125in,height=2.27083in]{./images/Image00734.jpg}
 \captionsetup{justification=centering}
 \caption{女性,65岁,慢性肾功能不全、低钙(1.4mmol/L)及低钾血症(2.4mmol/L)。显示2:1传导的二度房室传导阻滞、ST段呈水平型延长、U波增高、Q-T间期延长、符合低钙及低钾血症的心电图改变}
 \label{fig45-6}
  \end{figure} 

\protect\hypertarget{text00053.htmlux5cux23subid641}{}{}

\subsection{高钙血症}

当血清钙>3.0mmol/L时,便称为高钙血症。常见于甲状旁腺功能亢进、多发性骨髓瘤、骨转移癌等。高钙血症主要引起动作电位2相时间缩短。心电图表现为ST段缩短或消失,Q-T间期缩短及U波增高;严重高钙血症时,可出现各种传导阻滞、室性心律失常等。需与短Q-T间期综合征相鉴别。

\protect\hypertarget{text00053.htmlux5cux23subid642}{}{}

\section{血镁异常的心电图改变}

正常血清镁浓度为0.75~1.25mmol/L。约99\%的镁分布在细胞内,而细胞外液中的镁仅占1\%,其中约60\%为游离的Mg\textsuperscript{2+}
。游离的Mg\textsuperscript{2+}
在体内具有多种生理功能:酶的激活剂、调节神经肌肉及心血管的兴奋性、降低细胞膜的通透性及组成骨盐的成分等。

\protect\hypertarget{text00053.htmlux5cux23subid643}{}{}

\subsection{低镁血症}

1.基本概念

当血清镁浓度<0.75mmol/L时,便称为低镁血症。较常见,大医院约占10\%,在急救中心,其发生率可达65\%。常见于长期禁食、厌食、严重腹泻、急性胰腺炎、长期使用利尿剂、糖尿病酮症酸中毒及甲状腺功能亢进等疾病。

2.低镁血症对心肌细胞的影响

低镁血症时,心肌细胞膜上的Na\textsuperscript{+} -K\textsuperscript{+}
-ATP酶活性降低,引起细胞内K\textsuperscript{+}
外流减小,静息电位负值减小,兴奋性增高;因Mg\textsuperscript{2+}
有阻断浦肯野纤维等快反应细胞的Na\textsuperscript{+}
内流作用,低镁血症时,这种阻断作用减弱,以致Na\textsuperscript{+}
内流增快、增多,4相除极化加速,自律性增高。随着心肌细胞兴奋性和自律性的增高,易发生心律失常。低镁血症时,因血管平滑肌细胞内Ca\textsuperscript{2+}
含量增高,血管平滑肌对缩血管物质反应性增强,可引起冠状动脉痉挛,导致心肌缺血,甚至心肌梗死。低镁血症时,常合并低钾血症和低钙血症,故在低钾血症或低钙血症时,如经补钾、补钙后仍不能纠正,则应考虑有缺镁的存在,并且也只有同时补镁后方能见效。

3.心电图改变

(1)类似低钾血症时的ST-T改变,有时出现T波电交替现象(图\ref{fig45-7})。

\begin{figure}[!htbp]
 \centering
 \includegraphics[width=5.1875in,height=0.67708in]{./images/Image00735.jpg}
 \captionsetup{justification=centering}
 \caption{低镁血症(0.61mmol/L)患者出现T波电交替现象(引自郭继鸿)}
 \label{fig45-7}
  \end{figure} 

(2)可出现各种心律失常及传导阻滞。

\protect\hypertarget{text00053.htmlux5cux23subid644}{}{}

\subsection{高镁血症}

当血清镁浓度>1.25mmol/L时,便称为高镁血症。但血镁不超过2.0mmol/L时,对机体影响很小,只有当血镁高达3.0mmol/L时,才会出现症状。常见于肾功能衰竭、甲状腺功能减退、肾上腺皮质功能减退、镁摄入过多等。高镁血症时,心肌兴奋性降低,传导抑制。心电图改变类似高钾血症的改变,表现为心动过缓、完全性房室传导阻滞,QRS波群增宽,甚至心脏停搏。

\protect\hypertarget{text00054.html}{}{}

\protect\hypertarget{text00054.htmlux5cux23chapter54}{}{}

\chapter{药物影响及其诱发的心律失常}

\protect\hypertarget{text00054.htmlux5cux23subid645}{}{}

\section{洋地黄对心脏的作用及心电图改变}

\protect\hypertarget{text00054.htmlux5cux23subid646}{}{}

\subsection{洋地黄对心脏的作用}

(1)增强心肌收缩力、降低交感神经张力、间接兴奋迷走神经:洋地黄有抑制心肌细胞膜Na\textsuperscript{+}
-K\textsuperscript{+}
泵ATP酶系统作用,促使肌浆网释放Ca\textsuperscript{2+}
,增强心肌收缩力,改善心功能。

(2)降低窦房结自律性:与洋地黄降低窦房结4相去极化速度和间接兴奋迷走神经作用有关,可出现窦性心动过缓、窦性停搏。

(3)延长窦房交接区、房室交接区的有效不应期并降低其传导速度:可出现窦房传导阻滞、房室传导阻滞,降低心房扑动、颤动时的心室率。

(4)缩短房室旁道的有效不应期并加快其传导速度:预激综合征合并心房颤动、房室逆向型折返性心动过速时,严禁使用洋地黄类药物。

(5)缩短心房肌的有效不应期并加快其传导速度:与洋地黄直接对心房肌作用和间接兴奋迷走神经作用有关。低浓度时,间接兴奋迷走神经作用占优势,降低心房内异位灶的自律性;高浓度时,洋地黄直接作用占优势,心房内异位灶的自律性增高。

(6)增强心室内异位灶的自律性及折返性心律失常:洋地黄能使浦肯野纤维4相去极化加速、膜电位负值减小更接近阈电位,导致其自律性增高,出现室性心律失常。膜电位负值减小后,膜反应性和传导速度减慢,易形成折返性心律失常。

(7)缩短心室肌的2相动作电位,使Q-T间期缩短。

(8)增强触发活动而引发心律失常:洋地黄过量时,细胞膜上Na\textsuperscript{+}
-K\textsuperscript{+} 泵受到抑制,使细胞内Na\textsuperscript{+}
增加,通过Na\textsuperscript{+} -Ca\textsuperscript{2+}
交换,大量Ca\textsuperscript{2+} 内流,细胞内Ca\textsuperscript{2+}
超负荷,引起延迟后除极而诱发心律失常。

\protect\hypertarget{text00054.htmlux5cux23subid647}{}{}

\subsection{洋地黄治疗量时心电图表现}

(1)鱼钩样ST-T改变:以R波为主导联ST段呈下斜型压低、T波负正双相或倒置,其前肢与ST段融合,呈鱼钩样改变。

(2)Q-T间期缩短。

(3)U波增高。

\protect\hypertarget{text00054.htmlux5cux23subid648}{}{}

\subsection{洋地黄中毒时的心电图特征}

洋地黄中毒时,主要是由兴奋性增高引起的各种心律失常及由抑制作用引起的缓慢性心律失常和传导阻滞,或两者联合作用引起的心律失常。最能预示洋地黄中毒的心电图表现有频发多形性或多源性室性早搏二联律、室性心动过速、双向性心动过速、高度或三度房室传导阻滞、非阵发性房室交接性或室性心动过速、心房扑动或颤动等。

(1)室性心律失常:①频发单源性、多源性或多形性室性早搏,多呈二、三联律,是洋地黄中毒最常见、最早出现的心律失常,尤其是在心房颤动基础上出现(图\ref{fig13-19});②室性心动过速,可呈短阵性或持续性,常为洋地黄中毒的晚期表现(图\ref{fig46-1}),死亡率高达68\%~100\%;③非阵发性室性心动过速或加速的室性逸搏心律;④双向性室性心动过速,为重度中毒表现,常在心房颤动基础上发生(图\ref{fig46-2}),死亡率很高;⑤心室颤动。

\begin{figure}[!htbp]
 \centering
 \includegraphics[width=5.71875in,height=2.15625in]{./images/Image00736.jpg}
 \captionsetup{justification=centering}
 \caption{冠心病、心房颤动患者。Ⅱ导联系服用洋地黄后连续记录,显示心房颤动、频发加速的室性逸搏(77次/min)、加速的房室交接性逸搏或室性融合波(上行R\textsubscript{13})、频发短阵性室性心动过速伴心室折返径路内不典型4:3文氏现象、完全性干扰性房室分离,提示洋地黄中毒}
 \label{fig46-1}
  \end{figure} 


\begin{figure}[!htbp]
 \centering
 \includegraphics[width=5.6875in,height=1.26042in]{./images/Image00737.jpg}
 \captionsetup{justification=centering}
 \caption{心房颤动服用洋地黄患者。aVF、V\textsubscript{1}导联同步记录,显示双向性室性心动过速,提示洋地黄中毒}
 \label{fig46-2}
  \end{figure} 


(2)房室交接性心律失常:①非阵发性房室交接性心动过速或加速的房室交接性逸搏心律;②过缓的房室交接性逸搏及其逸搏心律。

(3)房性心律失常:①阵发性房性心动过速伴房室传导阻滞;②心房扑动或颤动。

(4)房室分离或房室传导阻滞:①原有心房颤动,经洋地黄治疗后,出现加速的房室交接性逸搏心律合并房室分离,诊断洋地黄中毒具有很高的特异性,且发生率较高(图\ref{fig13-20}、图\ref{fig13-21});②出现二度、高度、三度房室传导阻滞合并房室交接性或室性逸搏及其逸搏心律(图\ref{fig13-17}、图\ref{fig13-18});③出现一度、二度Ⅰ型房室传导阻滞,前者为洋地黄中毒早期表现,后者为最常见表现之一,其阻滞部位多发生在房室结内,很少发生在房室结以下。

(5)双重性心动过速:出现非阵发性或阵发性房性心动过速、非阵发性或阵发性房室交接性心动过速、非阵发性或阵发性室性心动过速或上述6种心律失常的不同组合。

(6)窦性心律失常:出现显著的窦性心动过缓、窦性停搏或窦房传导阻滞。

(7)原有心房颤动,经洋地黄治疗后心室率反而更快者,多数是洋地黄中毒的表现。

\protect\hypertarget{text00054.htmlux5cux23subid649}{}{}

\subsection{非洋地黄中毒性心律失常}

有些心律失常尽管在洋地黄化或使用洋地黄病人中出现,但它们与洋地黄中毒并无关系,有学者称为“非洋地黄中毒性心律失常”,其心电图表现有以下6点:①并行心律型室性早搏、室性心动过速;②阵发性房室交接性心动过速;③由房室结以下部位阻滞引起的二度Ⅱ型房室传导阻滞;④由房室结以下部位阻滞引起的三度房室传导阻滞伴加速的室性逸搏心律或室性逸搏心律;⑤各种的心室内传导阻滞,如束支阻滞、分支阻滞及不定型心室内传导阻滞;⑥窦性心动过速。

\protect\hypertarget{text00054.htmlux5cux23subid650}{}{}

\subsection{识别洋地黄中毒心电图特征的临床意义}

洋地黄类药物是治疗充血性心力衰竭、快速型心房颤动的常用药物之一,其治疗量约为中毒剂量的60\%,故临床上约有20\%的患者发生中毒现象,表现为心律失常或(和)房室传导阻滞。在中毒病例中约3\%~21\%因心脏毒性反应而死亡。因此,早期识别并及时处理洋地黄中毒引起的心律失常,具有极为重要的临床意义。

\protect\hypertarget{text00054.htmlux5cux23subid651}{}{}

\subsection{诊断洋地黄中毒应注意的问题}

(1)洋地黄中毒和用量大小无绝对比例关系,小剂量洋地黄中毒,多与肾功能减退、心肌严重受损、电解质紊乱或应用利尿剂等因素有关。

(2)洋地黄中毒可毫无自觉症状,须观察对比用药前后症状及心电图改变,如心率、传导情况等。

(3)在洋地黄治疗过程中,临床上遇及以下4种特征性改变,应疑及中毒,及时做心电图检查:①正常心率或快速心率转为心动过缓;②正常心率时突然出现心动过速;③不规则的心律变为规则的心律;④呈现有规律的不规则心律。

(4)洋地黄中毒的有些表现可能不为人们所注意,如窦性心动过速及(或)心力衰竭恶化,易误认为洋地黄用量不足而进一步加大剂量,加重中毒程度。凡是在洋地黄加量后心率反而加快及(或)心力衰竭恶化者,应考虑中毒可能。

(5)原有心力衰竭在使用洋地黄后曾一度好转而又突然或进行性加重,并发展为难治性心力衰竭者,应警惕洋地黄中毒。

(6)快速型心房颤动伴心力衰竭时经洋地黄治疗后,心室率仍较快且伴有室性早搏出现,该早搏出现不一定是洋地黄中毒的表现,可能是洋地黄用量不足、心力衰竭尚未纠正所致,可采用“西地兰耐量试验”观察判断。

(7)正确对待血清地高辛浓度的测定,需密切结合临床加以评估与判断。

\protect\hypertarget{text00054.htmlux5cux23subid652}{}{}

\section{抗心律失常药物的致心律失常作用}

\protect\hypertarget{text00054.htmlux5cux23subid653}{}{}

\subsection{抗心律失常药物的分类}

根据药物对心肌细胞的电生理作用不同而分为4类。

(1)Ⅰ类药物:又称为膜稳定剂,抑制Na\textsuperscript{+}
内流及起搏细胞4相除极化速度,增加K\textsuperscript{+}
外流作用。可分为:①Ⅰa类药物,如奎尼丁、普鲁卡因酰胺等;②Ⅰb类药物,如利多卡因、美律西平(慢心律)等;③Ⅰc类药物,如普罗帕酮(心律平)、乙吗噻嗪等。

(2)Ⅱ类药物:为β受体阻滞剂,如倍他乐克等。阻断β肾上腺素能受体和限制Ca\textsuperscript{2+}
内流作用,降低窦房结和异位起搏点的自律性;也有轻度抑制Na\textsuperscript{+}
内流及K\textsuperscript{+} 外流作用,缩短不应期,减慢传导速度。

(3)Ⅲ类药物:为复极抑制剂,如胺碘酮、索他洛尔等,主要抑制2、3相的K\textsuperscript{+}
外流,使动作电位时间和有效不应期延长。

(4)Ⅳ类药物:为Ca\textsuperscript{2+}
拮抗剂,如维拉帕米(异搏定)、硫氮酮等。阻止慢反应细胞Ca\textsuperscript{2+}
内流,降低窦房结、房室结细胞的自律性,延长房室结的不应期及传导时间。

\protect\hypertarget{text00054.htmlux5cux23subid654}{}{}

\subsection{致心律失常作用的概念、机制及诊断标准}

1.基本概念

由抗心律失常药物引起新的心律失常或使原有的心律失常加重现象,称为致心律失常作用。绝大多数抗心律失常药物均有致心律失常作用,尤其是有心肌损害时。胺碘酮致心律失常作用最小。由抗心律失常药物引起的传导异常,则不属于致心律失常作用的范畴。

2.致心律失常作用的机制

(1)机体的特异质反应。

(2)药物本身的毒性作用。

(3)与心肌复极、不应期不一致有关:正常和异常心肌组织的传导性、不应期及复极过程等电生理特性均有明显差异,局部心肌血流差异还可影响药物在组织中的分布和结合,从而影响电生理参数;若合并电解质紊乱、酸碱平衡失调,则可增强心脏对药物的敏感性;延长Q-T间期药物,在心动过缓或长短间歇后,易诱发尖端扭转型室性心动过速。

3.致心律失常作用的诊断标准

(1)抗心律失常药物治疗过程中,出现新的快速性室性心律失常而无其他诱因。

(2)室性早搏加重:对照期为1~50次/h者,次数增加10倍;对照期为51~100次/h者,增加5倍;对照期为101~300次/h者,增加4倍;对照期为≥301次/h者,则增加3倍。

(3)室性心动过速发作时频率显著增快者。

(4)室性心律失常发生变异:由短阵性室性心动过速发展为持续性室性心动过速、由单一室性心动过速发展为扭转性、多形性、多源性室性心动过速或心室颤动。

(5)终止快速型室性心律失常的难度增大。

\protect\hypertarget{text00054.htmlux5cux23subid655}{}{}

\subsection{常用抗心律失常药物的心电图改变及致心律失常作用的特征}

(一)胺碘酮(可达龙)

1.作用机制及适应证

为Ⅲ类药物,主要作用于动作电位2、3相,抑制K\textsuperscript{+}
外流,使动作电位和有效不应期延长,尚有抑制Ca\textsuperscript{2+}
内流及Na\textsuperscript{+}
内流,并兼有抗心绞痛、β受体阻滞剂作用。适用于各种早搏、心动过速、阵发性心房颤动及扑动。

2.心电图表现

(1)减慢心率:可使基础心率降低10\%~15\%,当心率较快时,减慢心率作用更为明显。

(2)T波时间增宽,呈双峰切迹、振幅降低。

(3)Q-T间期延长:以T波时间延长为主,若延长超过正常最高值25\%,应减量或停药。

(4)可出现U波振幅增高。

(5)剂量过大时,可引起扭转型室性心动过速、心室颤动、窦性停搏及高度房室传导阻滞等。

(二)普罗帕酮(心律平)

1.作用机制及适应证

为Ⅰc类药物,抑制动作电位0相的快Na\textsuperscript{+}
通道开放、延长有效不应期及阻断β受体的效能。对异位刺激或折返机制所致的心律失常有显著的效果。

2.心电图表现

(1)减慢心率引起窦性心动过缓。

(2)可出现P-R间期延长及QRS波群时间增宽。

(3)可出现Q-T间期延长。

(4)剂量过大或毒性作用时,可出现窦性停搏、高度窦房或房室传导阻滞、多形性或尖端扭转型室性心动过速及心室颤动等。

(三)美律西平(慢心律)

1.作用机制及适应证

为Ⅰb类药物,除抑制Na\textsuperscript{+}
内流外,突出的作用是加速复极期K\textsuperscript{+}
外流,缩短不应期。改善心室内传导,尚能抑制浦肯野纤维4相除极化,降低心室内异位起搏点的自律性。适用于室性心律失常的治疗。

2.心电图表现

(1)对窦房结功能正常者无明显影响,对其功能不全者,可引起窦性心动过缓、窦性停搏等。

(2)剂量过大或静脉注射时,可出现房室传导阻滞、心室颤动、心室停搏等。

(四)利多卡因

为Ⅰb类药物,作用机制与美西律相似,对室性心律失常是安全有效的。常于给药的开始两天内出现窦性心动过缓、窦性停搏、窦房传导阻滞、房室传导阻滞或心室内传导阻滞等。

(五)苯妥英钠

为Ⅰb类药物,作用机制与美律西平相似,仅用于洋地黄中毒引起的室性心律失常。剂量过大或给药过快时,可出现窦性心动过缓、房室传导阻滞或心脏骤停等。

(六)倍他乐克(美托洛尔)

为β受体阻滞剂,兼有弱的细胞膜抑制作用,用于窦性心动过速、早搏及心绞痛、高血压的治疗。可引起窦性心动过缓、窦房或房室传导阻滞等。

(七)维拉帕米(异搏定)

为Ca\textsuperscript{2+}
拮抗剂,能抑制心脏及房室传导,减慢心率。对阵发性室上性心动过速、分支型室性心动过速及短偶联间期尖端扭转型室性心动过速综合征有效。当剂量过大或静脉注射过快时,可出现窦性心动过缓、窦性停搏、房室传导阻滞或室性心律失常,甚至出现心脏、呼吸骤停等;能加速或改变房室旁道为顺向性传导,增加心室率,可使预激综合征合并室上性心动过速、心房颤动者发生心室颤动而死亡。

(八)阿托品

为胆碱能M受体拮抗剂,对窦房结具有双重作用。小剂量(<0.4mg)引起迷走神经张力增高,降低窦性频率,诱发房室交接性逸搏及逸搏心律;大剂量(>0.5mg)可解除迷走神经对心脏的抑制作用,使窦性频率加快,可诱发窦性心动过速、多源性室性早搏、室性心动过速等。

(九)肾上腺素(副肾素)

为肾上腺素能受体兴奋剂,直接兴奋α、β受体,使周围血管收缩,心率加快,血压升高。用于支气管哮喘、过敏性休克及心脏骤停复苏者。剂量过大或静脉注射过快,可引起室性心律失常,如室性早搏、室性心动过速甚至心室颤动。

\protect\hypertarget{text00054.htmlux5cux23subid656}{}{}

\subsection{如何预防及减少药物致心律失常作用}

(1)使用抗心律失常药物时,应对患者作出全面、正确的评估,去除诱因,治疗病因,注意药物个体化及关注药物致心律失常作用,是减少致心律失常作用的关键。

(2)严格掌握药物的应用指征。

(3)尽量先用一种药物,并从小剂量开始,治疗前、后应予24h动态心电图监测。

(4)多种抗心律失常药物联合治疗仅适用于单一药物最大耐受量治疗无效或致命性心律失常的患者需要额外保护时,需注意协同和拮抗作用。

(5)使用抗心律失常药物前、后应测定并及时纠正电解质紊乱、酸碱平衡失调。

(6)注意抗心律失常药物和其他药物相互不良作用及配伍禁忌。

(7)静脉给药时,应进行心电监护。

(8)长期服药者,最好能做血液药物浓度监测。

\protect\hypertarget{text00055.html}{}{}

\protect\hypertarget{text00055.htmlux5cux23chapter55}{}{}


\part{临床常见脑病与危象}

\chapter{高血压脑病}

高血压脑病(hypertensive
encephalopathy,HE)是指血压因某种诱因突然显著的增高(原发性或继发性高血压),突破了脑血管的自动调节机制,导致脑血流灌注过多,液体经血脑屏障漏出到血管周围脑组织,导致脑水肿、颅内压增高,而发生的一种急性一过性以神经障碍为主的高血压危象。临床上主要表现有剧烈头痛、烦躁、恶心呕吐、视力障碍、抽搐、意识障碍,甚至昏迷等症状,若不及时救治,常可导致死亡。由于有效防治急进型高血压、急性肾炎和妊娠期高血压病等,本病发生率已有显著下降。

\subsection{病因与发病机制}

高血压是最基本的病因。在此基础上受到某些诱因(如过度劳累、情绪激动、神经紧张、气候变化及内分泌失调等)的激发,或无明显诱因而突然发生血压的急剧升高(舒张压常超过120mmHg),即可导致高血压脑病。临床上以急进型高血压(又称恶性高血压)引起者最常见,其次为急慢性肾炎、肾盂肾炎、子痫、原发性高血压、嗜铬细胞瘤等患者。急性或慢性脊髓损伤患者,因膀胱充盈或胃肠潴留等过度刺激自主神经,可诱发高血压脑病。血压升高的速率对本病的发生也起决定性作用,如急性或新近发生的高血压,可以在慢性高血压患者能够耐受的血压水平上,发生高血压脑病。从血压升高到出现高血压脑病一般需要12~48小时,但也可短至几分钟。

HE是血压急骤升高,发生脑水肿的结果。传统观点认为血压急剧上升时,全身小动脉普遍痉挛收缩,脑小动脉收缩,血管阻力明显增高,脑血流量减少,毛细血管壁由于缺血变性,渗透性增加,使体液和血浆蛋白向血管外渗透加速,从而发生急性脑水肿(脑小动脉痉挛学说)。目前则认为是脑血管“自身调节崩溃”所致。在正常情况下,脑动脉血管的舒缩维持相对的恒定,脑血流量是以脑血管自动调节,主要由血压的高低对血管平滑肌作出相应的反应。当血压低时脑动脉扩张,血压高时脑动脉收缩,以保障大脑组织的血流量供应相对恒定。在正常人,平均动脉压(mean
arterial pressure,MAP,MAP =舒张压+
1/3脉压)在60~120mmHg范围内脑血流量(CBF)保持恒定状态。当正常血压者短时间内突然产生高血压,可在相对较低水平高血压下发生HE,如儿童急性肾小球肾炎等。在慢性高血压患者,由于血压长期缓慢升高,使小动脉壁发生适应性结构改变,即血管壁增厚,管腔狭窄,整个自动调节曲线右移,MAP在120~160mmHg范围内CBF恒定。当MAP
>
160~180mmHg时,便超越了自身调节能力,收缩的血管不能承受这样高的压力,脑小动脉则不能继续收缩,脑动脉自身调节功能降低,继而出现崩溃引起脑动脉被动性或强制性扩张,进入脑的血流量突然增大,灌注量过多而发生脑水肿。毛细血管壁本身变性坏死,继发性点状出血和小灶性梗死,导致脑功能障碍,出现脑病症状。

\subsection{诊断}

\subsubsection{临床表现特点}

HE的病程长短不一,短则几分钟,长则可达数天之久。起病急骤,常因过度劳累、紧张和情绪激动所诱发。病情发展快,进行性加重,发病前常见有血压显著增高,剧烈头痛、恶心、呕吐、精神紊乱等先兆。发病后以脑水肿症状为主,大多数患者具有头痛、抽搐和意识障碍三大特征,称之为HE三联征。头痛常是HE的早期症状,多数为全头痛或额枕部疼痛明显,咳嗽、活动用力时头痛加重,伴有恶心、呕吐,当血压下降后头痛可得以缓解。随着脑水肿进行性加重,于头痛数小时至1~2天后多出现程度不同的意识障碍,如嗜睡、昏睡、木僵、躁动不安、谵妄、定向力障碍、精神错乱,甚至昏迷。若视网膜动脉痉挛时,可有视力模糊、偏盲或黑矇。有时还可出现一过性偏瘫、半身感觉障碍、脑神经瘫痪、甚至失语;亦可见全身性或局限性抽搐等神经系统症状。严重者可出现呼吸中枢衰竭症状。血压多显著升高,舒张压常>
130mmHg,患者多有心动过缓、呼吸困难。长期高血压者可有左心室肥大,心前区可闻及舒张期奔马律,第三心音、第四心音,心电图示有左室劳损。少数病例于脑病后可出现肾功能不全、尿毒症表现。眼底检查有视网膜动脉痉挛,还可有视神经乳头水肿和出血、渗出。脑脊液压力升高(一般不作此项检查,除非必要时,宜选用细针穿刺),化验检查除可有蛋白含量增多和偶有少量红细胞外,余无异常。上述表现常于血压急剧升高12~48小时内明显,若抢救不及时,可于短时间内死亡。

\subsubsection{辅助检查}

颅脑CT扫描可见脑水肿的弥漫性脑白质密度降低,脑室变小;MRI显示脑水肿敏感,呈T\textsubscript{1}
低信号T\textsubscript{2}
高信号,顶枕叶水肿对HE具有特征性,偶见小灶性缺血或出血灶。脑电图常见双侧同步的慢波活动。

\subsubsection{诊断注意事项}

根据患者血压急剧升高后出现上述(头痛、抽搐和意识障碍)神经症状和体征,本病一般不难诊断。但HE为排除性诊断,在确立诊断前,须与脑出血、蛛网膜下腔出血(SAH)、急(慢)性硬膜下血肿、脑栓塞、脑梗死及脑瘤等鉴别。可从以下几点作出判断:

\paragraph{发病情况}

对鉴别诊断很有价值。本病的意识障碍和其他病症多在剧烈头痛发生后数小时才出现,而脑出血、SAH时则多在急剧头痛发生后数分钟至1小时内出现。急、慢性硬膜下血肿患者也有严重头痛,但常有颅脑损伤史,且神经症状体征多在数小时、数日甚至数周逐渐出现。脑梗死尽管起病急,但头痛不明显。脑瘤患者在就诊前常有数周至数月的进行性头痛加重史,其血压升高也不如本病明显。

\paragraph{对降压治疗的反应}

此为重要鉴别点。若予以有效的降压后病情迅速恢复,则支持本病诊断;反之,其他疾病的可能性大。但若本病治疗不及时,使脑组织发生持久性损害,或本病合并尿毒症时,则血压下降后病情恢复较慢或不完全。

\paragraph{眼底检查}

本病有严重的弥漫性或部分性视网膜动脉痉挛,可伴视神经乳头水肿或出血、渗出。脑出血时也可有类似表现。若发生视神经乳头水肿时不伴视网膜动脉痉挛,则提示脑瘤、慢性硬膜下血肿或SAH。视网膜动脉栓塞多提示脑栓塞。

\paragraph{脑脊液检查}

本病的CSF可无或偶有少量红细胞,而脑出血时CSF常为血性,SAH则为明显血性。

\paragraph{颅脑}

CT和(或)MRI检查 可确立诊断。

\subsection{治疗}

当病史和一般检查支持本病诊断时,应立即予以降压治疗,控制血压至安全水平。此时一般不宜花时间去作特殊检查(如CT或MRI检查),以免延误抢救。治疗原则包括紧急降压治疗,制止抽搐和治疗脑水肿,以防发生不可逆的脑损害,注意保护肾功能等。在脑病缓解之后,要积极治疗高血压及引起高血压的原发病,防止HE的复发。

\subsubsection{迅速降低血压}

迅速有效地降低血压是治疗的关键。对HE患者必须在2~4小时之内将血压降至治疗目标值。业已发现,无论正常血压者或高血压患者,脑的自动调节机制下限均约比休息时的MAP低25\%。因此,HE降压治疗的目标值是使MAP降低20\%~25\%,使血压维持在避免高血压危害并保证器官适当灌注的范围。一般要使舒张压迅速降至110mmHg(高血压患者)或80mmHg(血压正常者)以下。在降压过程中要严密监测血压、心率、精神状态,随时调整给药的滴速。另外,要注意因血压降的过快过低,而出现低灌注危象。大多数HE患者的症状随血压的降低而改善,若治疗过程中精神症状没有改善或反而恶化,应重新考虑诊断是否正确并适当升高血压,然后再缓慢降压。常用药物有:

\paragraph{尼卡地平(nicardipine)}

是二氢吡啶类钙拮抗剂。静脉滴注5~10分钟起效,作用持续1~4小时(长时间使用后持续时间可超过12小时),起始剂量为5.0mg/h(可用剂量是5~15mg/h),然后渐增加至达到预期治疗效果;也可直接用2mg静脉注射,快速控制血压后改为静脉滴注。一旦血压稳定于预期水平,一般不需要进一步调整药物剂量。副作用有头痛,恶心、呕吐,面红,反射性心动过速等。尼卡地平能够减轻心脏和脑缺血,对有缺血症状的患者更为有利。尼卡地平治疗HE的特点是:降压作用起效迅速、效果显著、血压控制过程平稳、血压波动小;能有效保护靶器官;用量调节简便;副作用少且症状轻微,停药后不易出现反跳,长期用药也不会产生耐药性,安全性好。与硝普钠相比降压效果近似,而其安全性及对靶器官的保护作用明显优于硝普钠,已成为HE首选药物之一。因其可能诱发反射性心动过速,在治疗合并冠心病的HE时宜加用β受体阻滞剂。

\paragraph{乌拉地尔(urapidil)}

又名压宁定。主要通过阻断突触后膜α\textsubscript{1}
受体而扩张血管,还可以通过激活中枢5-羟色胺-1A受体,降低延髓心血管调节中枢交感神经冲动发放。乌拉地尔扩张静脉的作用大于动脉,并能降低肾血管阻力,对心率无明显影响。其降压平稳,效果显著,有减轻心脏负荷、降低心肌耗氧量、改善心搏出量和心输出量、降低肺动脉压和增加肾血流量等优点,且安全性好,无直立性低血压、反射性心动过速等不良反应,不增加颅内压,不干扰糖、脂肪代谢。肾功能不全可以使用。孕妇、哺乳期禁用。用法:12.5~25mg稀释于20ml生理盐水中静脉注射,监测血压变化,降压效果通常在5分钟内显示;若在10分钟内效果不够满意,可重复静脉注射,最大剂量不超过75mg;继以100~400μg/min持续静脉滴注,或者2~8μg/(kg•min)持续泵入,用药时间一般不超过7天。

\paragraph{拉贝洛尔(labetalol)}

是联合的α和β肾上腺素能受体拮抗剂,静脉用药α和β阻滞的比例为1∶7,多数在肝脏代谢,代谢产物无活性。与纯粹的β阻滞剂不同的是,拉贝洛尔不降低心排血量,心率多保持不变或轻微下降,可降低外周血管阻力,脑、肾和冠状动脉血流保持不变。脂溶性差,很少通过胎盘。静脉注射2~5分钟起效,5~15分钟达高峰,作用持续2~6小时。用法:首次静脉注射20mg,接着20~80mg/10min静脉注射,或者从2mg/min开始静脉滴注,最大累积剂量24小时内300mg,达到血压目标值后改口服。副作用有恶心、乏力,支气管痉挛,心动过缓,直立性低血压等。

\paragraph{其他静脉用降压药物}

酚妥拉明治疗儿茶酚胺过多的高血压急症有良效,如嗜铬细胞瘤、可乐定撤药、可卡因过量等。但因其可增加心肌做功和耗氧量,故禁用于心肌梗死的患者。硝普钠、硝酸甘油能直接增加脑血流量,因此一般不用于HE的患者。

\subsubsection{制止抽搐}

有抽搐者,可用地西泮(安定)10~20mg直接静脉注射,同时肌注苯巴比妥0.2g。如发生癫痫持续状态,其治疗详见本书第85章“癫痫与癫痫持续状态”部分。

\subsubsection{降低颅内压、减轻脑水肿}

可选用20\%甘露醇液250ml静注或快速静滴,依病情每4~8小时1次,可辅以应用呋塞米(速尿)、地塞米松等。详见本书第41章“颅高压危象”部分。

\subsubsection{对症支持疗法}

包括吸氧,卧床休息,保持环境安静,严密观察病情变化,维持水电解质平衡,防治心肾等并发症等。参见有关章节。
\protect\hypertarget{text00098.html}{}{}

\hypertarget{text00098.htmlux5cux23CHP4-1-4}{}
参 考 文 献

1.
郭树彬.高血压急症的处理与靶器官保护策略.中国急救医学,2009,29(1):76

2. 田国祥
,孟庆义.高血压急症治疗-尼卡地平的兴起.中国循证心血管医学杂志,2010,2(1):6

3. Rodriguez MA,Kumar SK,De Caro M. Hypertensive crisis. Curr Drug
Targets,2009,10(8):788

\protect\hypertarget{text00099.html}{}{}

\chapter{肺 性 脑 病}

肺性脑病(pulmonaryencephlopathy)是由慢性胸肺疾患伴有呼吸衰竭,出现缺氧与二氧化碳(CO\textsubscript{2}
)潴留而引起以精神及神经系统症状为主要表现的一种综合征。突出表现为严重呼吸性酸中毒、自主呼吸减弱及中枢神经系统功能障碍的精神神经症状。

肺性脑病是我国独特应用的疾病诊断名词,相当于国际文献所称的“二氧化碳麻醉(carbon
dioxide narcosis)”,主要病因是由于严重的CO\textsubscript{2}
潴留。广义的肺性脑病是指由于肺功能障碍所引起的脑部症状,包括高碳酸血症和低氧性及过度通气所致的脑部症状等。狭义的肺性脑病是指由通气功能不全所致的动脉血CO\textsubscript{2}
急性潴留或慢性潴留加重时所产生的脑部神经系统症状,可伴有不同程度的缺氧,属于低氧血症合并高碳酸血症的Ⅱ型呼吸衰竭。

本症是慢性胸肺疾病的一个严重并发症,常伴有程度不同的多脏器损害,病死率高达30\%以上。

\subsection{病因与发病机制}

慢性阻塞性肺疾病(COPD)为肺性脑病的基础病因,以慢性肺心病所致者多见。导致肺性脑病的常见诱因有:①急性呼吸道与肺部感染,严重支气管痉挛,气道内痰液阻塞,使原已受损的肺通气功能进一步下降致体内CO\textsubscript{2}
潴留。②医源性因素,如镇静剂应用不当,高浓度吸氧,导致呼吸抑制而加重CO\textsubscript{2}
麻醉状态;不适当应用脱水剂及利尿剂,致痰液黏稠而加重气道阻塞。③慢性阻塞性肺疾病伴有右心衰竭时,由于脑血流量减少,加重脑缺氧及脑代谢功能紊乱。

肺性脑病的发病机制尚未完全阐明,但目前认为低氧血症、CO\textsubscript{2}
潴留和酸中毒三个因素共同损伤脑血管和脑细胞是最根本的发病机制。

脑组织耗氧量大,约占全身耗氧量的1/5~1/4。中枢皮质神经元细胞对缺氧最为敏感。通常完全停止供氧4~5分钟即可引起不可逆的脑损害。对中枢神经影响的程度与缺氧的程度和发生速度有关。当PaO\textsubscript{2}
降至60mmHg时,可以出现注意力不集中、智力和视力轻度减退;当PaO\textsubscript{2}
迅速降至40~50mmHg以下时,会引起一系列神经精神症状,如头痛、不安、定向与记忆力障碍、精神错乱、嗜睡;低于30mmHg时,神志丧失乃至昏迷;PaO\textsubscript{2}
< 20mmHg时,只需数分钟即可造成神经细胞不可逆性损伤。

慢性胸肺疾患严重呼吸衰竭时,肺泡通气功能极度下降,呼吸道中气流阻力增加,致使肺内CO\textsubscript{2}
排出障碍而在肺泡内潴留;血中CO\textsubscript{2}
增加和潴留,体内缺氧,降低了主要缓冲系统BHCO\textsubscript{3}
/H\textsubscript{2} CO\textsubscript{3}
的比值(正常为20∶1)而使血中pH下降。当pH下降到7.35以下,PaCO\textsubscript{2}
升高到70mmHg时,临床就出现呼吸性酸中毒;若PaCO\textsubscript{2} >
80~90mmHg,pH < 7.35,加上CO\textsubscript{2}
潴留使脑血管扩张、脑血流量增加,以及脑血管壁通透性增强,引起颅内压增高及脑水肿,临床病例早期常表现为头痛(晚上加重)、白天嗜睡、晚上失眠、多汗以及皮质中枢兴奋表现,如易激动、烦躁等精神症状;PaCO\textsubscript{2}
> 120~130mmHg,pH <
7.15时,则出现皮质中枢抑制状态,表情淡漠、神志恍惚、精神错乱,出现昏迷,即所谓CO\textsubscript{2}
麻醉状态。然而,临床观察表明,肺性脑病的发生与CO\textsubscript{2}
潴留的急缓有密切关系:CO\textsubscript{2}
在短期内急剧潴留,易诱发肺性脑病;CO\textsubscript{2}
缓慢潴留不易发生肺性脑病;且与脑脊液(CSF)pH直接相关。有时观察到血pH很低时,CSF
pH并不低,患者清醒;若患者血pH不低而CSF
pH很低,则患者意识障碍;表明患者意识障碍与CSF
pH明显降低呈正相关,而和血pH不相关。可能的原因是,{}
和H\textsuperscript{+} 缓慢地通过血脑屏障,而CO\textsubscript{2}
能较迅速地通过血脑屏障和细胞膜,在脑组织内、毛细血管和CSF中很快平衡,使CSF的PaCO\textsubscript{2}
在短时间急剧升高,CSF的pH迅速下降,从而造成动脉血与CSF中的pH出现不一致的CSF酸中毒,导致患者意识障碍,神志不清。而在慢性CO\textsubscript{2}
潴留时,PaCO\textsubscript{2}
虽然增高,但由于CSF中的HCO\textsubscript{3} \textsuperscript{−}
能逐渐代偿,促使CSF
pH维持在正常范围,则不宜发生肺性脑病。因此,肺性脑病的发生与CSF
PaCO\textsubscript{2} 急剧上升和CSF的pH迅速下降呈正相关。

缺氧及CO\textsubscript{2}
潴留均会使脑血管扩张,血流阻力降低,血流量增加以代偿脑缺氧。缺氧和酸中毒还能损伤血管内皮细胞使其通透性增高,导致脑间质水肿;缺氧使细胞ATP生成减少,造成Na\textsuperscript{+}
-K\textsuperscript{+}
泵功能障碍,离子经细胞膜的正常转运功能遭到破坏,因钠泵不能运转,以致钾离子从细胞内移出而进入组织间隙和血液,而Na\textsuperscript{+}
和H\textsuperscript{+} 则进入细胞内取代K\textsuperscript{+}
,结果导致细胞内H\textsuperscript{+}
浓度增加,加重细胞内酸中毒,由于细胞内Na\textsuperscript{+}
增加与进入细胞内的Cl\textsuperscript{−}
结合成NaCl,则引起细胞内渗透压升高,细胞外水分进入细胞内,结果致细胞内Na\textsuperscript{+}
及水增多,形成脑细胞水肿。以上情况均可引起脑组织充血、水肿和颅内压增高,压迫脑血管,进一步加重脑缺血、缺氧,形成恶性循环,严重时出现脑疝。另外,神经细胞内的酸中毒可引起抑制性神经递质γ-氨基丁酸生成增多,加重中枢神经系统的功能和代谢障碍。

肺性脑病还与严重缺氧时的肝、肾功能障碍和体内氨基酸代谢失衡有关。所以当芳香族氨基酸增多、支链氨基酸降低时,因脑组织的芳香族氨基酸增多而导致假神经递质的合成,影响脑的正常功能。

\subsection{诊断}

\subsubsection{临床表现特点}

\paragraph{基础疾病的表现}

有慢性胸肺疾患伴呼吸衰竭的表现。

\paragraph{CO\textsubscript{2} 潴留的神经系统表现}

症状与PaCO\textsubscript{2}
上升的速度及pH下降程度密切相关。早期轻症患者有头痛、头胀、烦躁、恶心呕吐,视力、记忆力和判断力减退;睡眠规律改变(白天嗜睡不醒,夜间失眠、惊醒);继之有神志恍惚、谵语、幻觉、精神错乱、抓空摸床、无意识动作和抽搐、扑翼样震颤;逐渐出现昏迷,眼底视神经乳头水肿,眼球突出,球结合膜充血水肿,出现锥体束征,对各种刺激无反应,脑疝形成等。

\paragraph{缺氧的神经系统表现}

可引起注意力不集中、定向力减退、头痛、兴奋,继而烦躁、谵妄、肌肉抽搐,神经肌腱反射亢进;中枢神经系统受抑制,伴有神志恍惚、昏迷。

\paragraph{血气分析}

示PCO\textsubscript{2} > 70mmHg,pH常小于7.25。

\subsubsection{临床分型与分级}

\paragraph{临床分型}

根据其神经精神症状,可将肺性脑病分为三型:①抑制型:以神志淡漠、嗜睡、昏迷等中枢神经抑制状态为主;②兴奋型:以烦躁不安、谵妄、多语等神经兴奋症状为主;③不定型:抑制与兴奋症状交替出现。

\paragraph{临床分级}

①轻型:神志恍惚、淡漠、嗜睡、精神异常或兴奋、多语而无神经系统异常体征者。②中型:浅昏迷、谵妄、躁动,肌肉轻度抽动或语无伦次,对各种刺激反应迟钝、瞳孔对光反应迟钝而无上消化道出血或DIC等并发症。③重型:昏迷或出现癫痫样抽搐,对各种刺激无反应;反射消失或出现病理性神经体征、瞳孔扩大或缩小;可合并上消化道出血、DIC或休克。

\subsubsection{诊断注意事项}

对慢性胸肺疾病,临床病程中出现神经精神症状时,首先应考虑肺性脑病。但出现精神障碍的神经症状者并不全是肺性脑病,临床极易混淆,故应注意与感染中毒性脑病、严重电解质紊乱、脑出血、DIC、脑动脉硬化、单纯型碱中毒等相鉴别。因此,必须详细询问病史、重视体检、眼底检查,特别是结合血气分析、电解质和有关化验测定,谨慎地加以区别。否则,一律或盲目按肺性脑病处理,必然会造成严重后果。

\subsection{治疗}

肺性脑病的病情复杂,并发症多,治疗上应根据致病原因及病情变化的不同阶段,分别对待,关键在于改善通气功能。为此,必须采取以下综合措施。

\subsubsection{正确氧疗}

通过增加吸入氧浓度来纠正患者缺氧状态的治疗方法即为氧疗。氧疗目标是使SaO\textsubscript{2}
上升至90\%以上或PaO\textsubscript{2}
>60mmHg,同时不使PaCO\textsubscript{2} 上升超过10mmHg或pH <
7.25。氧疗无疑能纠正肺性脑病患者的低氧血症,改善高碳酸血症和酸碱平衡。但若氧疗方法和给氧浓度掌握不当,会导致病情加重,甚至危及生命。肺性脑病因呼吸性酸中毒,有严重高碳酸血症,呼吸中枢对CO\textsubscript{2}
刺激不敏感,此时靠低氧刺激颈动脉窦及主动脉弓的化学感受器以兴奋呼吸。若突然吸入高浓度氧,则可使上述化学感受器不敏感,反而致使呼吸抑制,通气量减少,CO\textsubscript{2}
潴留更多,加重呼吸衰竭和肺性脑病病情。因此,对未行机械通气的患者给氧原则仍以持续性、低浓度、低流量为准。一般吸氧浓度为28\%~30\%,氧流量为1~2L/min。鼻导管或鼻塞法是长时间连续低流量给氧常用的方法,吸氧浓度(\%)=
21 + 4 ×氧流量(L/min)。亦可用面罩给氧。

此外,过量吸氧使换气冲动传入到呼吸肌的作用减弱,导致肺泡通气功能障碍,易致所谓的吸氧性呼吸停止。因此,在吸氧的同时配伍呼吸兴奋剂,以利有效通气,便可防止上述情况的发生。

\subsubsection{保持呼吸道通畅、增加通气量、改善CO\textsubscript{2} 潴留}

积极改善通气,纠正缺O\textsubscript{2} 和CO\textsubscript{2}
潴留是抢救肺性脑病的关键性措施。

\paragraph{清除痰液}

肺性脑病由于呼吸道感染,痰多黏稠,影响痰液不易引流或咳出;并因支气管痉挛、黏膜及黏膜下水肿和纤毛破坏,稠痰更不易咳出。可采用以下措施:①痰液黏稠者:可用祛痰剂如溴己新(必嗽平)8mg,每日3次;氨溴索30mg,每日3次;鲜竹沥液10~20ml,每日3次;10\%氯化铵10ml,每日3次;棕色合剂10ml,每日3次,及其他中药止咳祛痰。氨溴索静脉、肌内及皮下注射,成人每次15mg,每日2次;亦可加入液体中静滴。②无效或积痰干结者:可用药物雾化吸入或超声热蒸气雾化吸入治疗。③咳痰无力者:可采用翻身、拍背、体位引流等措施帮助排痰。必要时可在给氧情况下,通过纤维支气管镜吸引气管、支气管内的分泌物。

\paragraph{解除支气管痉挛}

可用各种支气管扩张剂,以茶碱类、皮质激素和β\textsubscript{2}
受体兴奋剂最常用。①氨茶碱:能解除支气管痉挛,兴奋呼吸中枢,增加心排血量和冠脉流量,利尿,增强呼吸肌与膈肌收缩力量,使通气量增加,PaO\textsubscript{2}
上升,PaCO\textsubscript{2}
降低,肺动脉压下降。用法:0.1~0.2g每日3次口服;或用0.125~0.25g加入25\%葡萄糖液20ml中缓慢静注。注射速度≤0.25mg/(kg•min),静脉滴注维持量为0.6~0.8mg/(kg•h),日注射量一般≤1.0g。应注意:本药剂量过大会引起恶心、呕吐等消化道症状,继而可有心悸、兴奋、心律失常、抽搐等;若已有心肌损害、心律失常、癫痫与活动性溃疡病者,不宜用。合用西咪替丁、喹诺酮类、大环内酯类药物等可影响茶碱代谢而使其排泄减慢,应减少用药量。②皮质激素可用甲泼尼龙80~160mg或氢化可的松300~500mg加入液体中静滴。③β\textsubscript{2}
受体兴奋剂:常用的有沙丁胺醇(salbutamol,舒喘灵)、特布他林(terbutaline,博利康尼,喘康速)、福莫特罗(formoterol)、丙卡特罗(procaterol,美喘清)、克仑特罗(clenbuterol,克喘素)等,可酌情选用。

\paragraph{呼吸兴奋剂的应用}

呼吸兴奋剂可刺激呼吸中枢或主动脉弓、颈动脉窦化学感受器,在气道通畅的前提下提高通气量,从而纠正缺氧和促进CO\textsubscript{2}
的排出,减轻CO\textsubscript{2}
潴留,尚能使患者暂时清醒,有利于咳痰、排痰。它须与氧疗、抗感染、解痉和排痰等措施配合使用,方能更好发挥作用。应用时注意以下几点:①气道必须通畅,否则会增加耗氧量;②脑缺氧或脑水肿未纠正而出现频繁抽搐者慎用;伴有高血压、动脉硬化、癫痫的患者慎用;③若停用呼吸兴奋剂最好逐渐减量或延长给药间隔,使患者呼吸中枢兴奋性逐步恢复,不宜突然终止;④应严格掌握呼吸兴奋剂的适应证:它常用于慢性阻塞性肺病伴有呼吸中枢敏感性降低,或应用镇静催眠药、氧疗使低氧刺激消失后引起的呼吸抑制,或肺性脑病氧疗过程中以及机械呼吸撤离前后配合应用;对以肺换气功能障碍为主所导致的呼衰患者不宜使用。既往常用尼可刹米、洛贝林,用量过大可引起不良反应,现已基本不用。取而代之的有多沙普仑(doxapram),常用20~50mg加入液体中静滴,该药对镇静催眠药过量引起的呼吸抑制和COPD并发急性呼衰有显著的呼吸兴奋效果。

纳洛酮是阿片受体阻断剂,有兴奋呼吸中枢作用,可行肌肉或静脉注射,每次0.4~0.8mg,静脉滴注1.2~2.8mg加入5\%葡萄糖液250ml中静脉滴注。

\paragraph{机械通气}

呼吸衰竭时应用机械通气能维持必要的肺泡通气量,降低PaCO\textsubscript{2}
;改善肺的气体交换效能;使呼吸肌得以休息,有利于恢复呼吸肌功能。急性呼吸衰竭患者昏迷逐渐加深,呼吸不规则或出现暂停,呼吸道分泌物增多,咳嗽和吞咽反射明显减弱或消失时,应行气管插管使用机械通气。机械通气过程中应根据血气分析和临床资料调整呼吸机参数。

若患者具备以下基本条件,可行无创正压通气(noninvasive positive pressure
ventilation,NIPPV):①清醒能够合作;②血流动力学稳定;③不需要气管插管保护(即患者无误吸、严重消化道出血、气道分泌物过多且排痰不畅等情况);④无影响使用鼻/面罩的面部创伤;⑤能够耐受鼻/面罩。在COPD急性加重早期给予无创机械通气可以防止呼吸功能不全加重,缓解呼吸肌疲劳,减少后期气管插管率,改善预后。

\subsubsection{控制感染}

呼吸道感染是发生呼吸衰竭进而引起肺性脑病之主要原因。因此,控制呼吸道感染、解除痰液壅塞、改善通气功能,是缓解肺性脑病病情发展和降低病死率的重要环节。肺性脑病患者往往多为老年患者,体弱、反应能力差,临床虽有感染,但表现多不典型。因此,咳嗽、咳黄痰应视为感染征象;肺内呼吸音低、啰音的存在亦应作为判断感染的征象。常见致病菌多为肺炎链球菌、流感嗜血杆菌、卡他莫拉菌、肠杆菌科细菌(肺炎克雷伯菌、大肠埃希菌、变形杆菌等)、铜绿假单胞菌等,或先为病毒感染后继发细菌感染。亦有部分患者为口腔不洁混以厌氧菌感染。应根据痰或呼吸道分泌物细菌培养与药敏,选用有效的抗生素。可供选用的抗菌药物常见的有β-内酰胺/β-内酰胺酶抑制剂(复方阿莫西林、阿莫西林/舒巴坦等)、第二、三代头孢菌素和氟喹诺酮类药物(左氧氟沙星、莫西沙星、环丙沙星)等,或选择具有抗铜绿假单胞菌活性的β-内酰胺类抗生素±氨基糖苷类抗生素。对广谱抗生素使用剂量大、时间较长的病例,尤其是同时使用糖皮质激素者,较易继发真菌感染,应加强口腔护理,每日可用生理盐水清洗口腔2~3次。一旦证实为真菌感染,应给予相应的抑制真菌药物。

\subsubsection{其他治疗}

\paragraph{脑水肿的治疗}

对重症者可以采取轻度或中度脱水,并以缓慢的或中等速度利尿为宜,再辅以冰帽、降温等物理措施。常用制剂为20\%甘露醇1~2g/kg,快速静滴,每日1~2次。也可使用β-七叶皂苷钠5~10mg静注,每日1~2次,或20mg/d加入液体中静滴。肾上腺皮质激素对缺氧所致的脑水肿也有良好的作用。

\paragraph{纠正酸碱平衡失调与电解质紊乱}

①呼吸性酸中毒:在慢性呼吸衰竭中最常见。保持呼吸道通畅,增加肺泡通气量是纠正此型失衡的关键。仅当pH
<
7.20时,可少量补充5\%碳酸氢钠(40~50ml)。②呼吸性酸中毒合并代谢性碱中毒:常见于呼吸性酸中毒的治疗过程中,多为医源性因素所致。处理原则为纠正呼吸性酸中毒的同时,只要每日尿量在500ml以上,可常规补充氯化钾3~5g。若pH过高,可静脉滴注盐酸精氨酸10~20g(加入5\%葡萄糖液500ml中)等。③呼吸性酸中毒合并代谢性酸中毒:常提示病情危重、预后差。处理包括增加肺泡通气量、纠正CO\textsubscript{2}
潴留;治疗引起代谢性酸中毒的病因;适当使用碱剂,补碱的原则同单纯型呼吸性酸中毒,一次可补充5\%碳酸氢钠(80~100ml),以后根据血气,再酌情处理。④呼吸性碱中毒:多因CO\textsubscript{2}
排出过多所致。一般不需特殊处理,以治疗原发病为主。电解质紊乱的处理见本书第6篇“水、电解质和酸碱平衡失调”部分。

\paragraph{防治心力衰竭}

慢性肺心病出现心功能不全以右心衰竭为主。一般经过氧疗、控制呼吸道感染、改善肺功能后,症状可减轻或消失,不需常规使用利尿剂和强心剂。较重者或经上述治疗无效者可选用小剂量缓效利尿剂。氢氯噻嗪25mg,每日1~3次;螺内酯40mg,每日1~2次。对肺性脑病出现脑水肿或重度水肿者可选用呋塞米(速尿)20mg缓慢静脉注射。应注意利尿剂可引起低血钾、低血氯,诱发或加重代谢性碱中毒;利尿过多可致血液浓缩、痰液黏稠加重气道阻塞。当慢性呼吸衰竭伴有左心功能不全时,可考虑适当使用洋地黄类药物。用药原则是选用小剂量(常规用量的1/2~1/3)、作用快、排泄快的强心剂。常以毛花苷丙(西地兰)0.2~0.4mg或毒毛花苷K
0.125~0.25mg加入葡萄糖液20ml内缓慢静脉注射(20分钟)。应注意纠正缺氧、防治低血钾,不宜依据心率的快慢观察疗效。如患者血压稳定,可考虑使用血管紧张素转化酶抑制剂治疗。

\paragraph{镇静剂的应用}

对肺性脑病患者的谵妄、狂躁不安和精神症状,在排除代谢性碱中毒后,应着重改善肺泡通气,避免用能加重呼吸抑制的镇静剂,如吗啡、哌替啶、巴比妥类药物、氯丙嗪等。必要时可用东莨菪碱0.3~0.6mg肌注,或地西泮10mg肌注。亦可用中成药醒脑静注射液(安宫牛黄注射液)2~4ml肌注。

\paragraph{脑细胞代谢与保护剂的应用}

如细胞色素C、辅酶A、ATP、胞磷胆碱、脑活素、纳洛酮等。

\paragraph{支持疗法}

可酌情输给血浆、复方氨基酸及人体白蛋白,以增强机体抵抗力。补给维生素B、C等。

\paragraph{防治并发症}

包括心力衰竭、休克、上消化道出血、DIC等,详见有关章节。
\protect\hypertarget{text00100.html}{}{}

\hypertarget{text00100.htmlux5cux23CHP4-2-4}{}
参 考 文 献

1. Hodgkin JE. Chronic obstructive pulmonary disease. Clin Chest
Med,1990,11:363

2. 陆再英,钟南山.内科学.第7版,北京:人民卫生出版社,2008:141

3. 秦桂玺 ,阎明.急危重症病与急救.北京:人民卫生出版社,2005:340

4. 俞森洋
,蔡柏蔷.呼吸内科主治医生660问.第2版.北京:中国协和医科大学出版社,2009:453

\protect\hypertarget{text00101.html}{}{}

\chapter{肝 性 脑 病}

肝性脑病(hepatic
encephalopathy,HE)是由严重肝病引起的以代谢紊乱为基础的、意识障碍、行为失常和昏迷为主要表现的中枢神经系统功能失调综合征。既往曾称肝昏迷(hepatic
coma),目前认为肝昏迷是HE程度相当严重的第四期,并不代表HE的全部。其发生是多种因素综合作用的结果,发病机制涉及氨中毒、假性神经递质、血浆氨基酸失衡、γ-氨基丁酸(GABA)、硫醇增多、短链脂肪酸代谢紊乱和星形细胞功能异常等学说,但主要原因是因肝细胞功能衰竭(肝细胞弥漫病变)和来自胃肠道未被肝细胞代谢去毒的物质经体循环(肝内外分流)至脑部而引起。

既往认为,重症肝炎或药物中毒所致者,起病急剧并进行性加重,称为急性肝性脑病;其中呈暴发性过程,短期内出现意识障碍者,又称为暴发性肝功能衰竭(fulminant
hepatic
failure,FHF)。它系由于肝脏大块或广泛坏死,残存肝细胞不能代偿生物代谢作用,代谢失衡或代谢毒物不能有效的被清除,导致中枢神经系统的功能紊乱,故亦称为内源性HE,或非氨性HE。此型HE,由于肝细胞广泛坏死所致,故预后极差。慢性肝性脑病是指严重慢性肝病(如肝硬化、原发性肝癌)及(或)门-体分流术后,从肠道中吸收入门脉系统的毒性物质,通过分流未经肝脏的首次通过作用(first
pass
effect)进入体循环,引起中枢神经系统的功能紊乱,因而亦称为外源性HE,或氨性脑病,或称为门-体分流性脑病(porto-systemic
encephalopathy,PSE)。本型HE约50\%有诱因,消除诱因后,可使病情逆转,预后较好。急性肝性脑病(内源性HE)与慢性肝性脑病(外源性HE),无论在临床上或发病机制上,有时均难以截然区别。以前将无明显临床表现和生化异常,仅能用精细的智力实验和(或)电生理检测才能作出诊断的HE,称为亚临床HE(subclinical
HE,SHE)或隐性HE(latent
HE,LHE)。由于概念不清易被理解为发病机制不同的另外一种病症,故目前主张用轻微HE(mild
HE,MHE)较为合适(见下述)。

最近(2002年)国际消化病学大会(world congress of
gastroenterology,WCOG)工作小组将HE分为A、B和C三型,实际上也恰好取了分别代表急性(acute)、分流(bypass)和肝硬化(cirrhosis)的英文首字母以便记忆。

A型肝性脑病,即急性肝功能衰竭相关的HE(acute liver failure
associated-HE,ALFA-HE),用来代替原来代表一种急性HE的FHF的术语,因为FHF实际的意义远不仅指急性HE。采用ALFA-HE能够避免将急性肝功能衰竭伴发的HE与慢性肝病伴发的急性HE的概念进一步混淆。

B型肝性脑病,是存在明显门体分流但无内在肝病的脑病。很少见,分流的原因可以包括先天性血管畸形和在肝内或肝外水平门静脉血管的部分阻塞以及各种压迫产生的门静脉高压,而造成门体旁路。此时肝活组织检查提示肝组织学正常,但临床表现与肝硬化伴HE的患者相同。因此,只有在肝活检显示为正常的肝组织学特征,才能诊断这种类型的脑病。此外,特异性的确认此类型有助于医生诊断不明确的疾病。

C型肝性脑病,指在慢性肝病或肝硬化基础上发生的HE,不论其临床表现是否急性。包括了绝大多数的HE,即通常意义上的HE。认为肝功能不全是C型发生的主要因素,而循环分流居于次要地位,但两者可协同作用。沿用的“门体分流性脑病”基本都是此型。根据HE的不同表现、持续时间和特点,C型又可以分成发作性、持续性和轻微HE
3个亚型:

\hypertarget{text00101.htmlux5cux23CHP4-3-1}{}
(1) 发作性HE:

是在慢性肝病的基础上在短期内出现意识障碍或认知改变,不能用先前存在的有关精神异常来解释,并可在短期内自行缓解或在药物治疗后缓解。发作性HE根据有无诱因又可分为:①诱因型:有明确的可追踪的诱发因素;②自发型:无明确的诱发因素;③复发型:指1年内有2次或2次以上HE发作。

\hypertarget{text00101.htmlux5cux23CHP4-3-2}{}
(2) 持续性HE:

是在慢性肝病的基础上出现持续性的神经精神异常,包括认知力下降,意识障碍,昏迷甚至死亡。根据患者自制力和自律性受损的严重程度可进一步分为:①轻型:即HEⅠ级;②重型:即HEⅡ~Ⅳ级;③治疗依赖型:经药物治疗可迅速缓解,若间断治疗,症状又会加重。

\hypertarget{text00101.htmlux5cux23CHP4-3-3}{}
(3) 轻微(minimal)HE:

是指某些慢性肝病的患者无明显症状性HE(发作性或持续性HE的临床表现和生化异常),但用精细的智力实验和(或)神经电生理检测可见智力、神经、精神的异常而诊断的HE。轻微HE在肝硬化患者中的患病率约为30\%~80\%。此型越来越受到重视,因为患者虽形似正常,但操作能力和应急反应能力减低,在从事高空作业、机械或驾驶等工作时容易发生意外。以往所用的“亚临床HE”或“隐性HE”这个词有一定的误导性,易被误认为其发病机制独立于HE之外或临床意义不大,故近年已接受改称为轻微HE,以强调其作为HE发展过程中的一个特殊阶段。

\subsection{病因与发病机制}

\subsubsection{病因与诱因}

\paragraph{病因}

各种严重的急性和慢性肝病(病毒性肝炎肝硬化最多见)均可伴发肝性脑病。急性肝病时肝性脑病的病因是由于大量的肝细胞坏死,常为病毒性肝炎、药物或毒素引起的肝炎;也可由于大量肝细胞变性,如妊娠期脂肪肝、Reye综合征等。慢性肝病,如肝硬化和重症慢性活动性肝炎的肝性脑病是由于有功能的肝细胞总数减少和肝血流改变;慢性肝性脑病的发病与广泛的门-体静脉分流有关。肝脏被恶性肿瘤细胞广泛浸润时,也可导致肝性脑病。

\paragraph{诱因}

许多因素可促发或加剧肝性脑病(表\ref{tab38-1}),此种情况在慢性肝病时尤为明显。常见诱因有:①上消化道出血:尤其是食管静脉及胃底冠状静脉曲张破裂出血,是慢性肝性脑病最常见的诱因;急性胃黏膜病变出血则是急、慢性HE共有的常见诱因。②利尿剂使用不当或大量放腹水。③高蛋白饮食。④应用镇静安眠药(巴比妥类、氯丙嗪等)以及麻醉剂等。⑤给予含氨药物(氯化铵)、含硫药物(蛋氨酸、甲硫氨基酸、胱氨酸等)、输注库血、富含芳香族氨基酸的复合氨基酸注射液以及水解蛋白等。⑥感染:如自发性细菌性腹膜炎、脓毒症、肺炎以及泌尿系感染等。⑦电解质紊乱与酸碱平衡失调:常见者为低钠、低钾、低氯、碱中毒。⑧功能性肾衰竭。⑨其他:手术创伤、便秘或腹泻。无诱因的自发性肝性脑病往往是肝硬化的终末期表现,患者肝脏大多缩小,肝功能严重损伤,黄疸深,腹水多,预后恶劣。

\begin{table}[htbp]
\centering
\caption{肝性脑病的诱因及其机制}
\label{tab38-1}
\includegraphics[width=3.28125in,height=4.41667in]{./images/Image00148.jpg}
\end{table}

\subsubsection{发病机制}

肝性脑病的发病机制迄今尚未彻底阐明。一般认为产生HE的病理生理基础是肝细胞功能衰竭和门腔静脉之间有自然形成或手术造成的侧支分流。主要来自肠道的许多可影响神经活性的毒性产物,未被肝脏解毒和清除,经侧支进入体循环,透过血脑屏障而至脑部,引起大脑功能紊乱。虽然氨中毒学说在HE的发病机制中一直占有支配地位,但目前尚没有一种学说能完备的解释HE发病机制的全貌。由于肝脏是机体代谢的中枢,它所引起的代谢紊乱涉及多种环节和途径,因此HE的发病机制也是多因素综合作用的结果。

\hypertarget{text00101.htmlux5cux23CHP4-3-4-2-1}{}
(一) 氨中毒学说

氨中毒学说(ammonia intoxication
hypothesis)在肝性脑病的发病机制中仍占最主要的地位。PET显示肝性脑病患者血氨水平增高,血脑屏障对氨的通透表面积增大及大脑氨的代谢增高(\textsuperscript{13}
NH\textsubscript{3}
-PET)。严重肝脏疾病时,血氨增加的原因是由于氨的生成与吸收增加及(或)清除不足所致。

\paragraph{氨的生成与吸收增加}

①外源性产氨增加:指氨的来源为肠道含氮物质的分解代谢与吸收增加。肠道蛋白质的分解产物氨基酸,部分经肠道细菌的氨基酸氧化酶分解产生氨;另外,血液中的尿素约有25\%经胃肠黏膜血管弥散到肠腔内,经细菌尿素酶的作用而形成氨,后者再经门静脉重新吸收,是为尿素的肠肝循环。肝功能衰竭时,肠道菌丛紊乱且繁殖旺盛,分泌的氨基酸氧化酶及尿素酶增加;同时由于胃肠蠕动和分泌减少,消化和吸收功能低下,使肠内未经消化的蛋白质等成分增多,特别是在高蛋白饮食或上消化道出血后更是如此,以致结肠、小肠内产氨均相应增加;此外慢性肝病晚期,常伴有肾功能下降,血液中的尿素等非蛋白氮含量高于正常,因而弥散到肠腔内的尿素也大大增加,也使产氨增加。肠内氨的吸收决定于肠内容物的pH,pH
> 6时,生成的NH\textsubscript{3} 大量吸收,血氨增加;pH <
6时,以NH\textsubscript{4} \textsuperscript{+}
形式随粪便排出体外,血氨降低。可见,氨的来源主要取决于肠腔蛋白质及尿素肠肝循环的量,氨的生成取决于细菌酶的作用,氨的吸收则取决于肠腔内的pH。②内源性产氨增加:即体内蛋白质的分解代谢产氨增加。肝功能衰竭时,蛋白质分解代谢占优势,加之焦虑、烦躁等情况,肌肉及脑活动均增加,产氨量相应增加。

\paragraph{氨的清除不足}

氨在体内主要经肝脏内鸟氨酸循环合成尿素而被清除;其次在外周组织(如脑、肌肉)先后与α-酮戊二酸、谷氨酸结合生成谷氨酰胺,再经肾脏作用重新释放出氨,由尿排出。肝功能衰竭时,主要是肝脏消除氨的作用减退,其次为肌肉代谢氨减少,另外肾脏排出的氨亦减少。此外,门体分流存在时,肠道的氨未经肝脏解毒而直接进入体循环,亦可使血氨增高。

\paragraph{血氨增加引起脑病的机制}

氨对脑的毒性作用包括:①直接抑制神经细胞膜的电位活动:氨能干扰神经细胞膜上的Na\textsuperscript{+}
-K\textsuperscript{+}
-ATP酶的活性,即破坏血脑屏障的完整性,又损害膜的复极化作用,从而引起HE。②干扰脑的能量代谢:血氨增高使大量α-酮戊二酸转变为谷氨酸,而后者又能转变为谷氨酰胺,故致三羧酸循环中α-酮戊二酸耗竭,循环速度下降,高能磷酸盐和氧耗减低;同时在此过程中消耗大量的ATP和还原型辅酶Ⅰ(NADH),后者减少致呼吸链中的递氢过程受到阻碍,使ATP的生成亦减少;此外,氨还可通过促进磷酸果糖激酶的活性增加,使脑组织内糖酵解过程增强,并直接抑制丙酮酸脱羧酶与有氧代谢,从而增加乳酸的生成,减少ATP的产生。上述生化反应使脑组织中的ATP生成减少,脑组织生理活动受到影响并出现脑病。③增加了脑对中性氨基酸如酪氨酸、苯丙氨酸、色氨酸的摄取,这些物质对脑功能具有抑制作用。④脑星形胶质细胞功能受损:脑星形胶质细胞是氨神经毒性的主要靶细胞。脑星形胶质细胞含有谷氨酰胺合成酶,可促进氨与谷氨酸合成为谷氨酰胺,当脑内氨浓度增加,星形胶质细胞合成的谷氨酰胺增加。谷氨酰胺是一种很强的细胞内渗透剂,其增加不仅导致星形胶质细胞肿胀、功能受损,而且也使神经元细胞肿胀,这是HE时脑水肿发生的重要原因。星形胶质细胞为神经元提供乳酸、α-酮戊二酸、谷氨酰胺及丙氨酸等营养物质,其功能受损可以直接影响神经元的功能及代谢,并参与HE的发生发展过程。⑤通过PET研究发现PSE患者脑氨代谢率升高,氨从血中极易转移到脑中,因此即使血氨正常也会发生脑功能障碍,这可以部分解释血氨不高情况下发生HE以及降氨治疗不一定能完全达到预期目的原因。此外,血氨及其代谢的异常与其他发病机制有协同作用。

\hypertarget{text00101.htmlux5cux23CHP4-3-4-2-2}{}
(二) 脑星形胶质细胞功能异常学说

正常情况下突触前神经末梢释放的谷氨酸迅速被周围的星形胶质细胞摄取,并在谷氨酰胺合成酶的作用下与氨合成为谷氨酰胺,谷氨酰胺再循环至神经元内释放具有活性的谷氨酸,此谓脑中的谷氨酰胺循环。由于脑内缺乏鸟氨酸循环的酶,故脑内清除氨的主要途径依靠谷氨酰胺合成,故谷氨酸氨基化生成谷氨酰胺的“解氨毒”作用完成于星形胶质细胞。另外,谷氨酸是脑内重要的兴奋性神经递质,储存于突触小泡内,一旦释放即呈现神经递质的活性,能与其受体结合产生神经传导活性。而谷氨酰胺是一种很强的细胞内渗透剂,其增加可导致星形胶质细胞肿胀、功能受损。HE时,超量的氨经谷氨酰胺合成酶的作用,不仅使具有活性的谷氨酸形成减少,导致谷氨酸能突触异常,还耗费了大量能量,并可导致谷氨酰胺的蓄积使胞内渗透压增加使细胞肿胀,肿胀的星形胶质细胞的功能受损进一步影响氨的代谢和谷氨酸活性,出现HE的表现。

\hypertarget{text00101.htmlux5cux23CHP4-3-4-2-3}{}
(三) 假性神经递质学说

神经冲动的传导是通过递质来完成的。正常时兴奋与抑制两类递质保持生理平衡。兴奋性神经递质有儿茶酚胺中的多巴胺和去甲肾上腺素、乙酰胆碱、谷氨酸和门冬氨酸等;抑制性神经递质β-羟酪胺、苯乙醇胺等只在脑内形成。

食物中的芳香族氨基酸如苯丙氨酸及酪氨酸,经肠菌脱羧酶的作用生成苯乙胺及酪胺,该两种胺类正常在肝内被分解清除。严重肝病时,该两种物质在肝内清除发生障碍,经门-体侧支循环进入体循环,并透过血脑屏障进入脑组织,经β羟化酶的作用,分别生成苯乙醇胺和β-羟酪胺。这两种胺的化学结构与正常神经递质去甲肾上腺素极为相似,但不具有正常递质传递神经冲动的作用或作用很弱,因此称其为假性神经递质(false
neurotransmitters)。当假递质被脑细胞摄取并在神经突触堆积至一定程度时,则排挤或取代正常的真递质,使神经传导发生障碍,特别是影响脑干网状结构上行激活系统和大脑边缘系统的神经传递,从而造成精神障碍和昏迷。

但近年来研究结果并不支持假性神经递质学说,如给实验动物静脉注射β-羟酪胺或脑室内注入大量假性神经递质导致脑内β-羟酪胺浓度非常高,脑内去甲肾上腺素和多巴胺明显耗尽,并未引起昏迷;尸检研究发现死于肝性脑病的肝硬化患者脑内去甲肾上腺素和多巴胺水平增加,而β-羟酪胺浓度降低。因此,该学说已逐渐被氨基酸失衡学说(amino
acid imbalance hypothesis)所替代。

\hypertarget{text00101.htmlux5cux23CHP4-3-4-2-4}{}
(四) 氨基酸失衡学说

血浆氨基酸测定发现,某些晚期慢性肝病与HE患者,血浆芳香族氨基酸(AAA)包括酪氨酸、苯丙氨酸、游离色氨酸增高,支链氨基酸(BCAA)包括亮氨酸、异亮氨酸、缬氨酸减少,致血浆氨基酸比值异常。正常人血浆BCAA/AAA的比值为3.5
±
1.0(s),肝性脑病时比值下降至1.0~1.5左右,甚至低于1.0,其下降值与脑病程度有一定的相关性。

血浆氨基酸比值的变化是由于严重肝病所继发的高胰岛素和高胰高血糖素血症所致。在严重肝病时,肝脏对许多激素包括胰岛素、胰高血糖素的灭活作用减弱,使两者血中浓度均增高,但以胰高血糖素的增多更显著,使血中胰岛素/胰高血糖素比值降低,使体内的分解代谢增强。其中胰高血糖素的增多,使组织的蛋白分解代谢增强,致使大量AAA由肝和肌肉释放入血。AAA主要在肝脏降解,肝功能严重障碍,一方面致AAA的降解能力降低,另一方面肝脏的糖异生作用障碍致使AAA转为糖的能力降低,这些均可使血中AAA含量增高。正常时支链氨基酸不被肝脏代谢,主要被肌肉摄取利用,胰岛素有增加肌肉组织摄取和分解利用支链氨基酸的作用,所以当血中的胰岛素水平增高时,促使BCAA大量进入肌肉组织,故血中BCAA浓度减少。AAA和BCAA彼此竞争血脑屏障的同一载体而转运至脑组织内(竞争性抑制)。正常时,血中BCAA的浓度高,竞争力强,从而抑制AAA进入脑内的速度;肝功能衰竭时,由于血浆BCAA减少,高浓度的AAA不受抑制地迅速通过血脑屏障进入脑组织,故脑组织细胞内的AAA含量明显增加。

正常时,脑神经细胞内的苯丙氨酸在苯丙氨酸羟化酶作用下,生成酪氨酸;酪氨酸在酪氨酸羟化酶作用下生成多巴;多巴在多巴脱羧酶作用下生成多巴胺;多巴胺在多巴胺β-羟化酶作用下生成去甲肾上腺素,这是正常神经递质的生成过程。

当进入脑内的苯丙氨酸和酪氨酸增多时,增多的苯丙氨酸可抑制酪氨酸羟化酶的活性,使正常神经递质生成减少。增多的苯丙氨酸可在AAA脱羧酶作用下生成苯乙胺,进一步在β-羟化酶作用下生成苯乙醇胺。而增多的酪氨酸也可在AAA脱羧酶作用下生成酪胺,进一步在β-羟化酶作用下生成β-羟酪胺。因而,苯丙氨酸和酪氨酸进入脑内增多的结果可使脑内产生大量假性神经递质,而产生的假性神经递质又可进一步抑制正常神经递质的产生过程。

氨基酸失衡学说实际上是假性神经递质学说的补充和发展。但下列观察不支持该假说,如临床上发现血浆BCAA/AAA变化和肝性脑病程度并不一定有平行关系;临床上采用静脉或口服BCAA治疗对改善与逆转肝性脑病不一定有效。因此该假说也不能完整地阐明HE的发病机制。

\hypertarget{text00101.htmlux5cux23CHP4-3-4-2-5}{}
(五) GABA/Bz复合受体学说

γ-氨基丁酸(γ-aminobutyric
acid,GABA)是哺乳动物大脑主要的抑制性神经递质。脑内的GABA在突触前神经元内由谷氨酸经脱羟酶(GADI)催化下脱羟生成,并贮存在突触前神经元的囊泡内,此时无生物活性。只有被释放到突触间隙,并结合到突触后神经元膜面特异性的GABA受体上,引起氯离子(Cl\textsuperscript{−}
)转运通道开放,使Cl\textsuperscript{−}
经神经元细胞膜裂隙进入细胞质,原先静止的细胞膜电位即处于高极化状态,从而导致GABA神经递质起明显的突触后抑制作用。突触后神经元膜面的GABA受体不仅能与GABA结合,在受体表面的不同部位也能与巴比妥类(BARB)和苯二氮{}
类(benzodiazepines,Bz,即弱安定类)物质结合,故称为GABA/Bz复合受体或超级受体复合物。该复合受体包括三种配体,即GABA、Bz及BARB配体,彼此有协同性非竞争性结合位点,已证明GABA可引起Bz及BARB的催眠作用,反之亦然,故巴比妥类药能增加GABA的效应。Bz、BARB及GABA受体复合物的连接,通过增加GABA引起的Cl\textsuperscript{−}
通道开放而加强受体复合物对GABA的反应。

大脑抑制性神经递质GABA/Bz的增加可能是导致HE的重要原因。其机制可能有:①血浆内的GABA主要来源于肠道,系谷氨酸经肠道细菌酶作用催化而成。正常时肝脏能大量摄取来自门静脉的GABA,并迅速分解。在肝功能不全时,肝脏对GABA的清除明显减低,血浆内浓度因而明显增高。如果此时血脑屏障对血浆GABA透过性增加,而GABA又不能被神经元分解或摄取,则GABA可抵达GABA受体,使GABA能性神经传递增强。②肝功能不全时中枢神经系统GABA能活性增强尚可以是超级受体复合物上GABA受体密度和(或)亲和力增加的后果。无论GABA、Bz及BARB中任何一种与复合受体结合后,都能促进氯离子由神经元胞膜的离子通道进入突触后神经元的细胞质,使膜超极化,引起神经传导抑制。如有学者研究了动物和人体肝性脑病脑内GABA和Bz受体的数量和亲和性,在一些急性肝性脑病模型中,这些受体的数量成倍增加,而在其他模型中没有变化,这可能提示此时大脑对GABA能神经抑制性调节比Bz或BARB药物更为敏感;PET扫描揭示,脑病患者Bz复合物连接部位增加2~3倍,这可能是肝硬化时脑对镇静药敏感性增加的机制。但近年的研究表明,脑内GABA/Bz的浓度在HE时并没有改变,但在氨的作用下,脑星形胶质细胞Bz受体表达上调。临床上,肝功能衰竭患者对苯二氮{}
类镇静药及巴比妥类安眠药极为敏感,而Bz拮抗剂如氟马西尼对部分HE患者具有苏醒作用,支持该学说。

\hypertarget{text00101.htmlux5cux23CHP4-3-4-2-6}{}
(六) 其他

1.锰的毒性学说
MRI显示80\%以上的肝硬化患者大脑苍白球密度增高,组织学证实是锰沉积而造成的。肝脏是锰排泄的重要器官,当其功能受到影响或存在门体分流时均可使血锰浓度升高,并在苍白球沉积。锰沉积除直接对脑组织造成损伤外,还影响5-羟色胺、去甲肾上腺素和GABA等神经递质的功能。锰还影响多巴胺受体的结合,导致多巴胺氧化使多巴胺减少,使患者产生锥体外系的症状和体征。

2.硫醇与短链脂肪酸学说
①硫醇类:蛋氨酸在结肠内受细菌作用形成硫醇、甲基硫醇和二甲硫化物等,由于肝脏解毒功能减退,进入体循环和脑内,在肝性脑病时血浆浓度增高。硫醇类化合物可抑制神经细胞膜的Na\textsuperscript{+}
-K\textsuperscript{+}
-ATP酶,干扰线粒体的电子传递,以及抑制脑内氨的解毒。血中硫醇类浓度增加,从呼吸道呼出增多,医者可嗅到一种特征性气味,是为肝臭。②短链脂肪酸:肝性脑病患者血浆内C\textsubscript{4}
~C\textsubscript{8}
短链脂肪酸增多。它可抑制氧化磷酸化,使脑干网状结构激活系统的ATP和磷酸肌酸贮存减少,改变神经细胞膜的离子流通,从而抑制神经冲动的传递,诱发肝性脑病。

3.褪黑素(melatonin,MT)
MT是由松果腺分泌的一种激素,具有镇静、催眠、神经内分泌免疫调节等多种生理功能。松果腺细胞从血液中吸收色氨酸,经过一系列酶促反应合成MT。肝硬化时血液中色氨酸浓度升高,松果腺合成MT也增多。MT通过较多的途径增强GABA的中枢抑制,如MT可增加脑内GABA含量,2-吲哚MT可协同GABA神经元放电等。

4.内源性阿片类物质、脑中肌醇和磷酸酯浓度减少等变化对HE的发生有一定作用。

肝性脑病的发生与发展,是多种物质生化代谢紊乱的综合作用。Ziere等观察到氨、硫醇与脂肪酸三者间能互相增强毒物的作用,引起脑病。氨与GABA的协同作用,表现为氨对GABA转氨酶有抑制作用,使GABA不能转变成琥珀酸半醛并进而变为琥珀酸进入三羧酸循环,致脑组织中GABA蓄积并导致神经中枢抑制加深。在高氨血症时,可促进血浆中AAA增高,BCAA/AAA比值降低,血脑屏障对AAA转运增强,致使大量AAA进入脑内引起脑病。多种毒物的协同作用,可解释临床上血氨水平与肝性脑病之间不一定平行这一现象,也说明了为什么单纯消除氨毒性不一定均能逆转肝性脑病。

\subsection{诊断}

\subsubsection{病史}

有前述的病因与诱因存在。

\subsubsection{临床表现特点}

肝性脑病的临床表现往往由于肝病的病因、病程缓急、肝功能损害的程度及诱因等不同而表现不一。A型HE与急性肝功能衰竭相关,可无明显诱因,患者在起病数日内即进入昏迷直至死亡,昏迷前可无前驱症状。C型HE多见于肝硬化患者和(或)门腔分流手术后,以慢性反复发作性木僵与昏迷为突出表现,常有诱因,如上消化道出血等。在肝硬化终末期所见的HE起病缓慢,昏迷逐渐加深,最后死亡。最常见的C型HE,除了患者有性格、行为改变(见下述)外,还常有肝功能严重受损的表现,如明显黄疸、出血倾向等,随着疾病的进展,有些患者可并发各种感染、肝肾综合征、脑水肿和心、肾、肺等主要脏器损害,导致低血压、少尿、呼吸衰竭、DIC、昏迷等相应的复杂表现。B型HE少见,其临床症状的产生源自门体分流,故类似C型,但无肝病的表现,或由其导致门体分流的本身疾病的特征。

典型HE较早出现的症状包括性格改变、精神欣快、智力减退、睡眠习惯改变、说话缓慢而含糊、发音单调而低弱,以及不适当的行为等。个性方面的变化最为显著,原属活泼开朗者,则表现为抑郁,原属内向孤僻者,则可表现为欣快。自发性运动的减少、不动地凝视、表情淡漠、答语迟缓而简短,均系早期表现。早期的行为改变只限于有一些“不拘小节”的行为,如乱扔纸屑,随地便溺,寻衣摸床等毫无意思的动作;这些细微的行为改变只有经常接触患者并留心病情变化的医务人员才能觉察。睡眠过久较早出现,并进展至睡眠节律的倒置,白天昏沉嗜睡,夜间兴奋难眠,这提示患者中枢神经系统的兴奋与抑制处于紊乱状态,预示HE的来临,有人称此种现象为迫近昏迷(impending
coma)。智力衰退,可从轻度的器质性精神功能障碍直至明显的精神错乱,并可观察到这些情况逐日发生变化。局灶性障碍多出现于意识清醒者,常系空间性视觉障碍。其在运动方面的障碍最易识别,如构思性运用不能,患者不能用火柴梗或积木构造简单的图案。典型病例可有书写不整齐而出格的情况,每日的书写记录是观察病程演变的良好准绳。患者的运算能力和逻辑思维明显减退,不能区别相似体积、形态、作用及位置的物体,这是患者常在不适宜场所便溺的原因。进一步发展下去,患者出现骚动、不安、躁狂、幻听,有时表现为进行性精神萎靡和完全无力状。嗜睡和兴奋相互交替为特征之一。患者有谵妄和运动性不安,跃起,叫喊,或哭或笑,但对外界刺激仍有反应,再进一步只对强烈而有害的刺激才起反应。当骚动和谵妄加重,嗜睡期延长,逐渐由木僵状态而进入昏迷。

最具有特征性的神经系体征为“扑翼样震颤(flapping
tremor)”,但不是所有患者都出现此种现象,如在一个严重肝病患者出现这种体征,就具有早期诊断意义。但是扑翼样震颤在早期、中期直至完全昏迷前均可出现。所以应在其他症状出现前经常检查有无此种体征才具有早期诊断意义。扑翼样震颤须在一定的体位时才能显露或引出。嘱患者将上肢伸直,手指分开,或腕部过度伸展而前臂固定不动时可出现掌-指及腕关节呈快速的屈曲及伸展运动,每秒钟常达5~9次,且常伴有手指的侧位动作。有时上肢、颈部、面颌、伸出的舌头、紧缩的口以及紧闭的眼睑均被累及,而患者的步态变为共济失调。患者通常呈现双侧性震颤,虽然双侧的动作不一定完全同时发生,一侧的动作可较另侧明显。震颤在昏迷时消失,但偶尔将患者的一肢轻轻举起或移动时,震颤可重新出现。扑翼样震颤也可在尿毒症、呼吸衰竭及严重心力衰竭中见到。患者可取两腿交叉而贴于腹壁的姿势,四肢有交替性的肌肉强直和松弛。早期有肌腱反射亢进和踝阵挛,锥体束征常阳性,握持反应可阳性。局部或全身性抽搐常见于疾病末期。少数病例,尤其是儿童有舞蹈状动作或手足徐动等。肝性脑病时还可出现一种特征性的气味------肝臭,这种气味很难用语言、文字来形容,有人把其描述为鱼腥味、烂苹果味、变质鸡蛋或大蒜样味等。

\subsubsection{辅助检查}

\paragraph{肝病的实验室检查}

因各类型肝病而异,急性HE患者常以血清胆红素、凝血酶原时间异常为主;慢性HE多伴有低白蛋白血症、高γ-球蛋白血症;各型严重肝病的HE大多有一种或数种电解质异常;血清尿素氮、肌酐在伴有肝肾综合征时升高。

\paragraph{血氨测定}

慢性HE患者多有血氨升高,急性HE患者血氨可正常。

\paragraph{血浆氨基酸测定}

芳香氨基酸尤其色氨酸常呈明显增加,支链氨基酸浓度降低,两者比值常倒置。在慢性肝性脑病更明显。目前已少用。

\paragraph{脑脊液检查}

常规检查和压力均正常,谷氨酰胺、谷氨酸、色氨酸和氨浓度可增高。目前已少用。

\paragraph{脑电图(EEG)检查}

早在生化异常或精神异常出现前,脑电图即已有异常,其变化对诊断与预后均有一定意义。正常人的EEG呈α波,每秒8~13次。HE患者的EEG表现为节律变慢。Ⅱ~Ⅲ期患者表现为δ波或三相波,每秒4~7次;昏迷时表现为高波幅的δ波,每秒少于4次。

\paragraph{神经生理测试}

主要是各种诱发电位(EP)的测定,包括视觉诱发电位(VEP)、脑干听觉诱发电位(BAEP)、躯体感觉诱发电位(SSEP)和事件相关电位(ERPs)P300,被认为对MHE的筛选、诊断、疗效观察等方面优于常规EEG检查。最近研究认为,VEP在不同人、不同时期变化太大,缺乏特异性和敏感性,不如简短的心理或智力测试有效。

\paragraph{心理智能测验}

一般将木块图试验(block design)、数字连接试验(number connection
test,NCT A和B)及数字符号试验(digit symbol
test,DST)联合应用。对诊断早期HE最有价值,对Ⅱ级以上HE不适用。分析结果时应考虑年龄、教育程度等影响因素。

\paragraph{影像学检查}

头部CT或MRI检查时,急性HE患者可发现脑水肿,慢性HE患者则可发现有不同程度的脑萎缩。单光子发射计算机断层摄影(SPECT)可显示区域性的脑血流异常,如额颞部及基底节区的局部血流量降低。MRI还可显示基底神经节(苍白球等)T\textsubscript{1}
加权信号增强(可能与锰的积聚有关)。磁共振波谱学(magnetic resonance
spectroscopy,MRS)是一种在高磁场(1.5T)磁共振扫描机上测定活体某些区域代谢物含量的方法。可用于HE的动态监测和评估各种治疗方案的疗效。正电子发射断层摄影术(PET)可以以影像学形式反映脑的特殊生化或生理学过程,其影像主要取决于所用示踪剂。以\textsuperscript{15}
O-H\textsubscript{2} O可测脑血流;\textsuperscript{13}
N可测氨代谢;\textsuperscript{18} F-氟脱氧葡萄糖(\textsuperscript{18}
F-fluorodeoxyglucose)可测葡萄糖代谢。然而,这些检查费用昂贵,限制了应用。

\paragraph{临界视觉闪烁频率(critical fricker-fusion frequency,CFF)检测}

机制为:轻度星形细胞肿胀是HE的病理改变,而星形细胞肿胀(Alzheimer
Ⅱ型)会改变胶质-神经元的信号传导,视网膜胶质细胞在HE时形态学变化与Alzheimer
Ⅱ型星形细胞相似,故视网膜胶质细胞病变可作为HE时大脑胶质星形细胞病变的标志,通过测定临界视觉闪烁频率可定量诊断HE。该方法可用于发现及检测轻微HE。

\subsubsection{肝性脑病的临床分期}

为了观察HE的动态变化,根据意识障碍程度、神经系统表现和脑电图改变,采用West
Haven分法,将HE自轻度的精神改变到深昏迷分为四期(表\ref{tab38-2})。分期有助于早期诊断、预后估计及疗效判断。

但各期之间并无极其明确的界限,故相邻两期症状协同出现的机会比单独出现的为多。

\subsubsection{诊断注意事项}

Ⅰ~Ⅳ期HE的诊断可依据下列异常而建立:①有严重肝病和(或)广泛门体侧支循环形成的基础;②出现精神紊乱、昏睡或昏迷,可引起扑翼样震颤;③有肝性脑病的诱因;④反映肝功能的血生化指标明显异常和(或)血氨增高;⑤脑电图异常。

轻微HE的诊断依据可有:①有严重肝病和(或)广泛门体侧支循环形成的基础;②心理智能测验、诱发电位、头部CT或MRI检查及临界视觉闪烁频率异常。

HE应与下列疾病鉴别:①出现精神症状时应与精神病鉴别:肝病患者常先表现精神症状,极易误诊为精神病,尤多见于急性重型肝炎时。因此,凡有精神症状等应注意检查有无肝病体征(如黄疸、腹水)和作肝功能检测,以免漏误诊。②有扑翼样震颤时,应除外尿毒症、呼吸衰竭、严重心力衰竭和低钾性昏迷。这四种情况下均可引出扑翼样震颤。③已陷入昏迷的HE,应与引起昏迷的其他常见疾病,如脑卒中、颅内感染、尿毒症、糖尿病昏迷、低血糖昏迷及镇静剂中毒等鉴别。④有锥体束征或截瘫时,还应与脑或脊髓肿瘤、脊髓炎鉴别。

\subsection{治疗}

\subsubsection{及早识别及消除HE诱因}

\paragraph{慎用或禁用镇静药和损伤肝功能的药物}

禁用麻醉剂、巴比妥类、氯丙嗪及大剂量地西泮等。有躁狂、抽搐时,宜首选东莨菪碱(每次0.3~0.6mg肌肉注射),其次为抗组织胺药(如异丙嗪12.5~25mg/次肌肉注射,或苯海拉明10~20mg肌肉注射),或小剂量地西泮(5~10mg/次)。

\paragraph{止血和清除肠道积血}

上消化道出血是HE的重要诱因之一。止血措施参见第13章第1节“上消化道出血”治疗部分。清洁肠道可口服轻泻剂,以每日排出软便2~3次为宜,乳果糖、乳梨醇、大黄等均可酌情使用,剂量因人耐受性而异。对于胃肠道积血须立即排出者,可从胃管抽吸或清洁灌肠。灌肠液可用生理盐水500~700ml加适量的食醋,禁用碱性溶液(如肥皂水)灌肠。亦可口服或鼻饲25\%硫酸镁30~60ml导泻。右半结肠是产氨最多的地方,灌肠液应进抵右半结肠,才能有效地清除该处的内容物,并降低该处的pH,减少毒物在该处的生成和吸收。为此,灌肠时患者先采取臀部高位,使灌肠液进抵结肠脾曲,然后向右侧卧,这样才能使药液进入右半结肠。对急性门体分流性脑病昏迷者用乳果糖500ml加入500ml水或生理盐水中保留灌肠30~60分钟,每4~6小时一次,效果好,应作为首选治疗。

\paragraph{纠正电解质及酸碱平衡紊乱}

低钾性碱中毒是肝硬化患者在进食量减少、利尿过度及大量排放腹水后的内环境紊乱,是诱发或加重HE常见原因。因此,应重视患者的营养支持,慎用利尿剂或剂量不宜过大,大量排放腹水时应静脉输入足量的白蛋白以维持有效血容量和防止电解质紊乱。缺钾者补充氯化钾。若每日尿量超过500ml,即使无低钾血症,在输注高渗葡萄糖液或应用大量排钾性利尿剂时,也应于静脉输液中常规补钾,每日氯化钾补充3~6g。如出现明显低钾血症,应每日分次补充氯化钾共6~9g。稀释性低钠血症,以限制入水量为主,酌情静脉滴注28.75\%谷氨酸钠40m(l相当于生理盐水450ml)以补充钠盐,或酌情应用渗透性利尿剂如20\%甘露醇250ml,使排水多于排钠。长期营养不良、吸收不良、低蛋白血症和利尿剂应用可造成低镁血症,临床上可致肌肉兴奋性升高、手足徐动、谵妄和昏迷。如出现这些症状而给予钙剂(如10\%葡萄糖酸钙等)后无改善或反而加重,应考虑低镁血症。可用25\%硫酸镁5~10ml加入液体中静滴,或每次3~5ml深部肌肉注射,每日1~2次。若有门冬氨酸钾镁针剂宜首选,常用20~40ml加入液体中静滴。若患者有代谢性碱中毒,除补充氯化钾外,还可补充盐酸精氨酸。

\begin{table}[htbp]
\centering
\caption{肝性脑病的分期}
\label{tab38-2}
\includegraphics[width=6.78125in,height=3.40625in]{./images/Image00151.jpg}
\end{table}

\paragraph{控制感染}

应选用对肝损害小的广谱抗生素静脉给药。

\subsubsection{减少肠内氮源性毒物的生成与吸收}

\paragraph{控制与调整饮食中的蛋白质}

通常认为,在慢性肝细胞性疾病患者,应予高蛋白饮食,以维持正氮平衡。一旦发生肝性脑病,蛋白质的摄入即应限制并保证热量供给。Ⅲ~Ⅳ期HE患者应禁止从胃肠道补充蛋白质,可鼻饲或静脉注射25\%的葡萄糖溶液。Ⅰ~Ⅱ期患者开始数日应限制蛋白质在20g/d以内,如病情好转,每3~5天可增加10g蛋白质,以逐渐增加对蛋白质的耐受性。待患者完全恢复后每日可摄入0.8~1.0g/kg蛋白质,以维持基本的氮平衡。以植物蛋白为首选,动物蛋白质以乳制品如牛乳或乳酪为佳,如病情稳定可适量摄入。肉类蛋白质应尽量少摄入。少食多餐和睡前加餐可改善机体氮平衡,而不使HE恶化。

但最近的研究显示,与限制蛋白质的摄入相比,正常摄入蛋白1.2g/(kg•d)是安全的,对血氨和HE的恢复无负面影响。在摄入蛋白质的问题上应把握以下原则:①急性期首日患者禁蛋白饮食,给以葡萄糖保证供应能量,昏迷不能进食者可经鼻胃管供食,但短期(4天)禁食不必要;②慢性HE患者无禁食必要;③蛋白质摄入量为1~1.5g/(kg•d);④口服或静脉使用支链氨基酸制剂,可调整AAA/BCAA比值;⑤蛋白质加双糖饮食可增强机体对蛋白质的耐受;⑥植物和奶制品蛋白优于动物蛋白,前者含甲硫氨酸、芳香族氨基酸较少,含支链氨基酸较多,还可提供纤维素,有利于维护结肠的正常菌群及酸化肠道。

\paragraph{清洁肠道}

特别适用于上消化道出血或便秘患者,方法如前述。

\paragraph{抑制肠道菌丛}

肠道中的毒性代谢产物主要是肠道细菌酶作用于基质的结果,控制肠道菌丛,能有效地减少毒性代谢产物的生成。传统方法是给予广谱不吸收性抗生素口服,以减少肠内需氧菌和厌氧菌,使细菌分解蛋白质和尿素减少,从而减少氨的产生,使血氨降低,改善肝性脑病的症状。最常使用的是新霉素,本品从胃肠道吸收很小,仅3\%的口服量随尿排出,在大便中含量高,同时未破坏,故可作为胃肠道抗菌药。口服或鼻饲1.0~1.5g,每日3次。若不能口服时,亦可作保留灌肠,剂量相同,同时每日需做清洁灌肠1~2次。应用新霉素后,多数患者可有神经精神改善,部分患者昏迷清醒,伴有肝臭减轻或消失,动脉血氨下降和脑电图改善。对慢性肝性脑病效果较好。其副作用有:①影响肠黏膜对某些营养物质的吸收(如糖、氨基酸、长链脂肪酸、维生素A、维生素K等),对肠黏膜有一定刺激性并引起其损害;②虽然吸收很少,但仍有约3\%的被吸收,可引起肾及前庭脑神经的损害,血肌酐>
177μmol/L(2mg/dl)时不宜使用;③可引起肠内菌群失调。为此,长期应用宜以小剂量为宜。其他抗菌药物如甲硝唑(灭滴灵,0.8g/d)、氨苄西林、利福昔明(rifaximin)和氟喹诺酮类药物均可选用,可取得相似的效果,但亦应注意其不良反应。其中,利福昔明是一种口服后肠道吸收极少的广谱抗生素(利福平的衍生物),具有起效快、疗效好、耐受性好等优点,可作为Ⅰ~Ⅲ期HE的辅助治疗。抗生素使用期不宜超过1个月,其中急性HE以1~2周为宜,以免引起二重感染等副作用。此外,含有双歧杆菌、乳酸杆菌等的微生态制剂,可起到维持肠道正常菌群,抑制有害菌群,减少毒素吸收的作用。

\paragraph{改变肠道 pH}

常用乳果糖(lactulose)。它是人工合成的含酮双糖,在小肠内不被双糖酶水解,其吸收与排泄均在0.4\%以下,绝大部分进入结肠,主要在右侧结肠内被乳酸杆菌、厌氧杆菌、大肠埃希菌等分解形成乳酸、醋酸和少量甲酸,在结肠内增加发酵,减少腐败,有利于乳酸杆菌的生长。其对肝性脑病的治疗作用主要有:①能有效地降低下段肠内容物之pH。正常情况下,该处pH与血液近似,无梯度存在。应用本品后,由于1分子乳果糖可生成4分子酸,可使该处pH降至5.5以下,右半结肠内pH更低,这样有利于血液中的氨转移至肠腔,并在肠腔内与酸结合而沉淀。②渗透性腹泻作用。由乳果糖分解产生的小分子酸可使渗透压增高,减少结肠内水分吸收,小分子酸又能促进肠蠕动,从而引起腹泻,使粪便在肠腔内停留时间缩短,不利于氨及其他有毒物质的生成与吸收,增加从血液转移至粪便中的氨排出。③改变肠道菌群。肠道乳酸杆菌大量生长,大肠埃希菌和厌氧菌等受到抑制,使氨生成减少。④本品亦可使体内尿素、尿内尿素含量降低,粪内氮质排出增加。每日从胃肠道内尿素释放的氨,相当于25~100g食物蛋白质所释放者,故在降低血氨的情况下,能同时减少体内尿素的含量。⑤本品具有细菌的碳水化合物的底物的作用,能增加细菌对氧的利用,使氨进入细菌的蛋白质中,从而使氨降低。⑥在降低血氨时,可允许患者摄取较多的蛋白质,维持全身营养。乳果糖是目前公认有效的治疗急、慢性肝性脑病的药物,可使临床症状和脑电图均得以改善,对慢性肝性脑病的有效率达90\%,与新霉素合用可提高疗效。新霉素虽能杀灭细菌,但不影响乳果糖所致的肠内pH下降,这是因为新霉素对类杆菌属无作用,而这种细菌分解乳果糖,因而两者合用具有协同作用。

乳果糖有糖浆剂(60\%)和粉剂,可口服或鼻饲,日剂量30~100ml,分3次服用。从小剂量开始,视病情增减,以调整至每日排2~3次软便或糊状便,或使新鲜粪便的pH降至6.0以下。一般在用药后1~7天开始起作用。对不能口服或鼻饲者可予乳果糖灌肠。本品无毒性,很安全,主要的不良反应是腹泻、腹胀、纳差,少数可有呕吐、腹部痉挛性疼痛,可减量或停药后消失。尚有部分患者对其不耐受,因过甜而不喜欢服用。

乳梨醇(lactitol)是另一种双糖(β-半乳糖-山梨醇),系由乳糖还原而制备。作用与疗效和乳果糖类同。价格较乳果糖便宜,甜味也较轻,易于入口,可溶入果汁或水内饮服,易为患者接受。其剂量为每日30~40g,分3次口服。不良反应与乳果糖相同。

对于乳糖酶缺乏者亦可试用乳糖,由于有的人小肠内缺乏乳糖酶,口服乳糖后在小肠不被分解与吸收,进入结肠后被细菌分解而酸化肠道,并产生气体,使肠蠕动增加而促进排便。其剂量为每日100g。

\subsubsection{促进体内氨的代谢}

1.L-鸟氨酸-L-门冬氨酸(ornithine-aspartate,OA)为一种鸟氨酸和门冬氨酸的混合制剂,能促进体内的尿素循环(鸟氨酸循环)而显著降低HE患者血氨。鸟氨酸能增加氨基甲酰磷酸合成酶和鸟氨酸甲酰转移酶活性,其本身也是鸟氨酸循环的重要物质,促进尿素合成。门冬氨酸可促进谷氨酰胺合成酶的活性,促进脑、肝肾的利用和消耗氨以合成谷氨酸和谷氨酰胺而降低血氨。用法:每次口服5g,每天2~3次;静脉滴注10~20g/d,最多不超过80g/d,用量过大易致消化道反应。严重肾功能衰竭者禁用。

2.鸟氨酸 -α-酮戊二酸
鸟氨酸的作用机制如上所述。α-酮戊二酸可增加谷氨酰胺合成酶活性,其本身还是三羧酸循环上的重要物质,能与氨结合形成谷氨酸。其疗效不如OA。

3.苯甲酸钠和苯乙酸钠
苯甲酸钠可与甘氨酸作用产生马尿酸盐,苯乙酸钠可与谷氨酰胺作用形成苯乙酰谷氨酰胺,从尿中排出。排泄一分子的苯甲酸盐,肾脏即可排泄一分子的氮。苯甲酸钠每次口服5g,每日2次,其治疗HE的效果与乳果糖相当,但价格便宜,费用仅为乳果糖的1/30。苯乙酸钠的效果不如苯甲酸钠,但两者之间有协同作用。

4.谷氨酸钠(钾)
在理论上,谷氨酸可与氨结合生成谷氨酰胺而降低血氨。临床上常用28.75\%谷氨酸钠(每支5.75g/20ml,含钠34mmol)40~100ml和(或)31.5\%谷氨酸钾(每支6.3g/20ml,含钾34mmol)20~40ml加入5\%~10\%葡萄糖液中静滴。一般认为钠盐与钾盐混合或交替应用较单纯用钠盐或钾盐为好。谷氨酸钠与钾两者合用比例一般为2~3∶1,低钾时为1∶1。静滴过快可引起流涎、面色潮红、恶心等反应。由于谷氨酸与氨结合生成谷氨酰胺是在ATP与镁离子的参与下进行的,故应同时给予ATP和硫酸镁(或门冬氨酸钾镁)。但谷氨酸不易透过血脑屏障;各种肝病时血浆谷氨酸浓度增高而非降低;谷氨酸盐为碱性(可同时加入5~10g维生素C滴注),可使血pH升高;钠离子可加重腹水和脑水肿,临床上尚无对照研究证明其有肯定疗效。因此,目前倾向于认为此类药物应用价值可疑。

5.盐酸精氨酸
此药偏酸性,有碱血症时可选用。常用量为25\%盐酸精氨酸40~80ml加入液体中静滴。

6.L-卡尼汀(L-carnitine)是广泛存在于机体内的一种特殊氨基酸,是人体长链脂肪酸代谢产生能量必需的一种物质。临床试验证实本品有降低血氨和改善HE的作用。

7.硫酸锌
氨通过尿素循环转化为尿素的过程需要5种酶,其中2种酶是锌依赖性的。由于锌在尿中丢失增加,锌缺乏在肝硬化患者中常见。锌缺乏可诱发复发性HE的发作,加重病情,补充锌后病情缓解,同时锌在DNA和蛋白质合成、含金属酶的功能中具有广泛的重要作用。因此,对锌缺乏的肝硬化患者应予以适当补锌治疗。

\subsubsection{调节神经递质、改善神经传导}

\paragraph{支链氨基酸(BCAA)}

BCAA制剂是一种以亮氨酸、异亮氨酸、缬氨酸等BCAA为主的复合氨基酸。其机制为竞争性抑制芳香族氨基酸进入大脑,减少假神经递质的形成,其疗效尚有争议。现倾向于认为BCAA不宜作为肝性脑病的常规用药,但对治疗某些类型(门-体)脑病可能是有益的;在不能耐受蛋白食物或限制蛋白摄入的患者,为了维持正氮平衡,改善营养,BCAA的应用不仅有指征,也是安全的(BCAA比一般食用蛋白质的致昏迷作用小)。

\paragraph{氟马西尼(flumazenil)}

为GABA/Bz复合受体拮抗剂,对部分Ⅲ、Ⅳ期HE患者有促醒作用。用法为:0.5mg加入0.9\%氯化钠注射液10ml于5分钟内静脉推注完毕,继以1.0mg加入0.9\%氯化钠注射液250ml中静滴(约30分钟)。

\paragraph{阿片受体拮抗剂}

纳洛酮能使HE患者提前清醒,总有效率达90\%,可减少长期昏迷所导致的并发症,并且不良反应少,是治疗HE的有效药物。其机制是:①纳洛酮能消除大量内源性阿片肽释放对心血管功能和呼吸的抑制,改善脑组织微循环。②大剂量的纳洛酮直接作用于脑细胞,保护Na\textsuperscript{+}
-K\textsuperscript{+} -ATP酶,抑制Ca\textsuperscript{2+}
内流、自由基释放及脂质过氧化,从而保护脑细胞,减轻脑水肿。③对抗中枢性神经递质GABA,激活脑干网状结构上行激活系统,有中枢催醒作用。④抑制HE时巨噬细胞的趋化活性,减少炎症反应。⑤改善缺血时神经细胞内Ca\textsuperscript{2+}
、Mg\textsuperscript{2+} 的紊乱,恢复线粒体氧化磷酸化和能量供给。

\paragraph{左旋多巴}

本品为多巴胺的前体。能透过血脑屏障进入脑内,经多巴脱羧酶作用生成多巴胺,进而形成去甲肾上腺素,以排挤假性神经递质,恢复中枢神经系统的正常兴奋性递质,从而恢复神志;它还有提高大脑对氨的耐受性以及增加肝血流量,改善心肾功能使肾排泌氨增加,间接地降低血氨及脑脊液中的氨。用法:0.2~0.4g加入5\%葡萄糖液250ml中静滴,每日1~2次。亦可用2~4g/d,分4~6次口服或鼻饲。通常用药后24~30小时神志改善。由于维生素B\textsubscript{6}
是多巴脱羧酶的辅酶,在周围神经促使左旋多巴更多地变成多巴胺,以致中枢神经系统不能得到神经递质的补充,故在用左旋多巴时,不宜并用维生素B\textsubscript{6}
。既往对本品评价较高,认为其至少对部分患者有效,曾被作为治疗急性肝性脑病的首选药物之一。但随机对照研究显示,无论是口服抑或静脉注射,该药均不能促进昏迷患者苏醒。因此,目前对其疗效的评价持否定态度者居多,已少用。此外,左旋多巴的副作用较多,如:①食欲减退、恶心、呕吐,并使溃疡加重,甚至消化道出血;②烦躁不安、失眠及幻觉;③舌、口唇、面颊、下颌可发生不随意运动;④有拟肾上腺素作用,引起心悸、血压升高和期前收缩等。对有器质性心脏病患者应慎用或禁用。

\paragraph{溴隐亭(bromocriptine)}

为多巴胺受体激动剂,具有增强神经传导、增加脑血流和代谢的作用。开始剂量为2.5mg/d,与饮食同服,每3天增加2.5mg/d,最大剂量为15mg/d,8~12周1疗程,可用于慢性HE对其他治疗无反应者。其副作用有恶心、呕吐、腹绞痛、便秘或腹泻、疲倦、头痛、眩晕等。与左旋多巴一样,其疗效未得肯定,目前少用。

\subsubsection{肝硬化腹水的治疗}

肝硬化腹水形成是门静脉高压和肝功能减退共同作用的结果,为肝硬化肝功能失代偿时最突出的临床表现。肝硬化腹水形成机制主要涉及门静脉压力升高、血浆胶体渗透压下降及有效血容量不足等。

治疗腹水不但可减轻症状,且可防止在腹水基础上发展的一系列并发症如SBP、HRS等。腹水治疗措施如下:

\paragraph{限制钠和水的摄入}

限钠饮食和卧床休息是腹水的基础治疗。钠摄入量限制在60~90mmol/L(相当于食盐1.5~2.0g/d),应用利尿剂时,可适当放宽钠摄入量。有稀释性低钠血症(血钠低于125mmol/L)者,应同时限制水摄入,摄入水量在500~1000ml/d。

\paragraph{利尿剂}

对上述基础治疗无效或腹水较大量者应使用利尿剂。常用螺内酯和呋塞米合用:先用螺内酯40~80mg/d,4~5天后视利尿效果加用呋塞米20~40mg/d,以后再视利尿效果分别逐步加大两药剂量(最大剂量螺内酯400mg/d,呋塞米160mg/d)。理想的利尿效果为每天体重减轻0.3~0.5kg(无水肿者)或0.8~1.0kg(有下肢水肿者)。应监测体重变化及血生化。

\paragraph{提高血浆胶体渗透压}

对低蛋白血症者,每周定期输注白蛋白或血浆,可通过提高血浆胶体渗透压促进腹水消退。

\paragraph{难治性腹水的治疗}

难治性腹水(refractory
ascites)定义为使用最大剂量利尿剂(螺内酯400mg/d加上呋塞米160mg/d)而腹水仍无减退。对于利尿剂使用虽未达最大剂量,腹水无减退且反复诱发HE、低钠血症、高钾血症或高氮质血症者亦被视为难治性腹水。其治疗可选择下列方法:①大量排放腹水加输注白蛋白:在1~2小时内放腹水4~6L,同时输注白蛋白8~10g/L腹水,继续使用适量利尿剂,可重复进行。此法对大量腹水患者,疗效比单纯加大利尿剂剂量效果要好,对部分难治性腹水患者有效。但应注意不宜用于有严重凝血障碍、HE、上消化道出血等情况的患者。②经颈静脉肝内门体分流术(transjugular
intrahepatic portosystemic
shunt,TIPS):该法能有效降低门静脉压,但仅用于上述治疗无效的难治性腹水、肝性胸水及伴肾功能不全者。③肝移植:难治性腹水是肝移植优先考虑的适应证。

\subsubsection{病因治疗}

对A型HE患者,采取综合治疗措施(如抗病毒治疗、促进肝细胞再生等)治疗急性肝功能衰竭;对B型HE患者或C型某些与门体分流相关的自发型HE患者,临床上可用介入治疗技术或手术阻断门-体侧支循环,以降低HE的复发率;C型HE患者,病因治疗的重点是肝移植,包括原位肝移植和肝细胞移植。

\subsubsection{其他治疗}

包括人工肝支持治疗、驱锰治疗、肝移植、放射介入或直接手术的方法阻断门-体侧支循环、积极防治并发症等。

\subsection{预后}

HE的预后主要取决于肝细胞衰竭的程度和诱因是否可被去除。诱因明确且容易消除者(例如出血、缺钾等)的预后较好。肝功能较好,分流手术后由于进食高蛋白而引起PSE者预后较好。有腹水、黄疸、出血倾向的患者提示肝功能很差,其预后也差。暴发性肝功能衰竭所致的HE预后最差。
\protect\hypertarget{text00102.html}{}{}

\hypertarget{text00102.htmlux5cux23CHP4-3-8}{}
参 考 文 献

1.
王宇明.肝性脑病的定义、命名和诊断.中华肝脏病杂志,2004,12(5):305-306

2. Antoni M. Hepatic encephalopathy:from pathophysiology to treatment.
Digestion,2006,73(supplement):S86-93

3. 陈灏珠
,林果为.实用内科学.第13版.北京:人民卫生出版社,2009:2120-2126

4. 张文武 .急诊内科学.第2版.北京:人民卫生出版社,2008:477-492

5. Faint V. The pathophysiology of hepatic encephalopathy. Nurs Crit
Care,2006,2:69

\protect\hypertarget{text00103.html}{}{}

\chapter{低渗性脑病}

低渗性脑病(hypoosmolar
encephalopathy)系指细胞外液呈低渗状态,部分水分移入细胞内而导致脑细胞水肿,从而引起脑的代谢和功能障碍,出现一系列精神神经症状的综合征。

\subsection{病因与发病机制}

血液的渗透压一般可分为晶体渗透压(即一般所指的渗透压)和胶体渗透压,它们各自维持着人体细胞内外水的平衡和血管内外水的平衡。当血浆晶体渗透压降低或增高,则水移至细胞内引起细胞水肿;或水移至细胞外引起细胞脱水,此种改变均在短期内发生即可影响脏器的功能,甚至迅速发生危及生命的病情。而血浆胶体渗透压降低(低蛋白血症)引起的水肿常为慢性过程,不致短期内发生危及生命的病情。人体体液渗透压改变后是由下丘脑部位的渗透压中枢来调整,即当细胞外液(ECF)渗透压和容积增减时,影响下丘脑对内源性抗利尿激素(ADH,即精氨酸加压素AVP)释放和抑制释放,以调整体液容量和渗透压。ADH的作用是增加远端肾小管和集合管对水的通透性,从而使水重吸收增加,如血浆渗透压降低则下丘脑抑制释放ADH,尿量乃增多;反之,则释放ADH,尿量乃减少,这样维持着人体体液容量和渗透压于正常范围。当各种原因使血浆晶体渗透压严重降低时,则引起脏器细胞水肿,其中脑细胞更易受累及。一般说来,低渗状态,成年人是指血浆渗透压<
280mOsm/L,儿童< 270mOsm/L;引起脑病则应进一步降低,如成年人<
260mOsm/L,儿童< 250mOsm/L。

细胞外液低渗最常见原因是低血钠。血钠<
125mmol/L数小时,就可导致脑细胞水肿,形成低渗性脑病;而轻度低血钠,一般不会导致低渗性脑病。在常见的低血钠中,只有稀释性低血钠和缺钠性低血钠可以引起低渗性脑病,尤其是前者。稀释性低血钠最常见于ADH分泌过多,尤其是所谓ADH分泌失调综合征(syndrome
of inappropriate antidiuretic hormone
secretion,SIADH)。SIADH常见病因为:①恶性肿瘤:某些肿瘤组织合成并自主性释放AVP。最多见者为肺燕麦细胞癌,约80\%的SIADH患者是由此引起。其他肿瘤如胰腺癌、淋巴肉瘤、网状细胞肉瘤、十二指肠癌、霍奇金淋巴瘤、胸腺瘤等也可引起SIADH。②呼吸系统疾病:如肺结核、肺炎、阻塞性肺部疾病等有时也可引起SIADH,可能由于肺组织合成与释放AVP。另外,感染的肺组织可异位合成并释放AVP样肽类物质,具有AVP相似的生物特征。③中枢神经系统疾病:包括脑外伤、炎症、出血、肿瘤、多发性神经根炎、SAH等,可影响下丘脑-神经垂体功能,促使AVP释放而不受渗透压等正常调节机制的控制,从而引起SIADH。④药物:如氯磺丙脲、长春新碱、环磷酰胺、卡马西平、氯贝丁酯、三环类抗抑郁药、秋水仙碱等可刺激AVP释放或加强AVP对肾小管的作用,从而产生SIADH。AVP、DDAVP过量时也可造成SIADH。部分病因不明者称之为特发性SIADH。由于AVP释放过多,且不受正常调节机制所控制,肾远曲小管和集合管上皮细胞对水的重吸收增加,尿液不能稀释,游离水不能排出体外,如摄入水过多,水分在体内潴留,细胞外液容量扩张,血液稀释,血清钠浓度与渗透压降低。同时,细胞内液也处于低渗状态,细胞肿胀,当影响脑细胞功能时,可出现神经系统症状。SIADH一般不出现水肿,因为当细胞外液容量扩张到一定程度,可抑制近曲小管对钠的重吸收,使尿钠排出进一步增加,因此,钠代谢处于负平衡状态,加重低钠血症与低渗血症。同时,容量扩张,GFR增加,以及醛固酮分泌受到抑制,也增加尿钠的排出。尿渗透压高于血浆渗透压。

稀释性低血钠也见于肾功能不全时,未加限制地输入大量低渗液和葡萄糖液。缺钠性低血钠多发生于长期限制钠盐摄入,特别是同时应用利尿剂者;呕吐与腹泻也是常见原因。老年人和育龄妇女更易于发生低钠血症的脑损害。研究表明,雌性激素能促进血管加压素从垂体的释放,而雄性激素则能抑制其释放。雌激素和雄激素对脑Na\textsuperscript{+}
-K\textsuperscript{+}
-ATP酶有不同的效力:雌激素能显著降低Na\textsuperscript{+}
-K\textsuperscript{+} -ATP酶的活力,后者则相反,而Na\textsuperscript{+}
-K\textsuperscript{+}
-ATP酶对低钠血症时维持正常的脑容量是十分重要的。正是由于以上两个原因,育龄妇女对罹患低钠血症严重并发症有更高的危险。低渗性脑病的病理改变为脑细胞水肿,细胞间隙小,但常无血管损伤,血脑屏障相对完整。脑细胞水肿较重时则颅压增高,产生颅内压增高的临床表现。严重的颅内压增高导致脑组织向颅内阻力较小的区域移动而疝入硬脑膜间隙或颅骨生理孔道形成脑疝,造成受压脑组织阻碍CSF通路和脑血液循环,使颅内压更形增高,则受压的神经结构淤血、水肿、出血和软化。由于此种改变为细胞内外渗透压差增大所致,故可在补晶体后随渗透压升高能很快纠正,有起病快、恢复快的特点。这不同于脑瘤、炎症的血管源性脑水肿和各种脑积水的间质性脑水肿,此类脑水肿不能较快消除病因,而改善脑水肿或脑病亦较缓慢。

\subsection{诊断}

\subsubsection{具有低渗血症的临床特征}

低渗性脑病是在血浆晶体渗透压降低的基础上发生,故先有低渗血症,严重时发展为低渗性脑病。患者常存在有低钠、低氯、低钾、低钙、低镁和水失调,临床表现有头痛、头晕胀、注意力不集中、体力衰弱、疲乏、肌力降低、常卧床不起、纳差、恶心、呕吐、腹胀、腱反射迟钝等。

\subsubsection{低渗性脑病的临床表现特点}

按细胞外液渗透压降低的程度与临床表现,可将低渗性脑病分为以下三度。

\paragraph{轻度}

成人血浆渗透压为260~250mOsm/L,儿童为250~240mOsm/L,临床上出现表情淡漠、乏力、倦怠、纳差、恶心、腹胀等症状。

\paragraph{中度}

成人渗透压为250~240mOsm/L,儿童为240~230mOsm/L,临床表现为嗜睡、头晕、反应迟钝、定向力障碍等。

\paragraph{重度}

成人渗透压< 240mOsm/L,儿童<
230mOsm/L,临床表现为谵妄、浅昏迷或昏迷、抽搐、周围循环衰竭等,甚至发生脑疝。

\subsubsection{辅助检查}

\paragraph{血浆晶体渗透压测定}

血浆晶体渗透压正常范围为280~300mOsm/L。其测定方法有:①冰点下降法;②晶体计算渗透压法:血浆晶体渗透压由电解质、尿素、葡萄糖等低分子物质组成,故可以计算。公式为:血浆渗透压(mOsm/L)=
2(Na\textsuperscript{+} + K\textsuperscript{+}
)(mmol/L)+葡萄糖(mmol/L)+尿素氮(mmol/L)。根据此公式计算出的血浆渗透压和冰点下降法测定值基本相同。

\paragraph{ADH测定}

任何ADH分泌增加的疾病均可形成稀释性低钠血症,故测定ADH可确定由ADH分泌增多所致的低渗血症。由于ADH主要为使肾小管重吸收水增加,但对Na\textsuperscript{+}
、K\textsuperscript{+} 、Cl\textsuperscript{−}
等电解质的重吸收无明显作用,故尿内排电解质不减少。SIADH的特征为:①血清钠降低(常<
130mmol/L);②尿钠增高(常> 30mmol/L);③血浆渗透压降低(常<
270mOsm/L);④尿渗透压>血浆渗透压;⑤有关原发病或用药史;⑥血浆AVP增高对SIADH的诊断有重要意义。在正常情况下,当细胞外液处于低渗状态,AVP的释放被抑制,血浆AVP常明显降低或测不到;但在SIADH患者,血浆AVP常不适当增高;⑦无水肿,肾功能、肾上腺皮质功能正常。ADH正常值波动较大,主要和人体水负荷有关。正常参考值为1.0~9.2pg/ml,平均3.65pg/ml。

\paragraph{尿渗透压测定}

采用冰点下降法。也可采用尿比重粗略估计,即比重1.005 =
200mOsm/L,每增加0.05则渗透压增加200mOsm/L,如1.010 = 400mOsm/L,1.015 =
600mOsm/L,1.020 =
800mOsm/L,故ADH分泌正常的低渗血症尿比重低于1.010,SIADH的低渗血症因尿钠排出不减少,其渗透压在300~400mOsm/L以上,即尿比重高于1.010。

\paragraph{红细胞体积(MCV)和血细胞比容(Hct)测定}

低渗血症水移向细胞内而致MCV和Hct增大,可间接判断血浆渗透压。

\paragraph{低渗性脑病的有关检查}

低渗性脑病的诊断首先须有血浆渗透压降低,其次需有脑病的临床表现及实验室检查异常。这些辅助检查包括视乳头水肿、脑脊液压力增高、脑电图可出现广泛性慢波、颅脑CT扫描常无异常病变等,结合临床病情,有助于低渗性脑病的诊断。

\subsubsection{诊断注意事项}

\paragraph{单纯(ADH分泌正常者)低渗血症和ADH分泌增加低渗血症的鉴别}

如前所述,低渗血症可由缺钠性低钠血症和由SIADH水潴留引起的稀释性低钠血症所引起,两者的病因、病情不一,且治疗时SIADH性低渗血症还应限制入水量,故应予以鉴别。

\paragraph{低渗性脑病和肺性脑病的鉴别}

在肺心病Ⅱ型呼吸衰竭患者,常伴发有低渗血症和肺性脑病,此类病例仅纠正呼吸衰竭或仅纠正低渗血症常不能明显改善病情,只有针对两者并治才能有效。尤其是要注意低渗性脑病和肺性脑病的并存,或误将低渗性脑病作为肺性脑病来处理的情况。

\paragraph{低渗性脑病和其他脑病}

、脑水肿疾患的鉴别
脑病和脑水肿可由多种病因引起,也有部分病例初为其他原因引起,后因治疗不当促使低渗性脑病的发生。故不论何种原因的脑病,均应将血浆渗透压作为常规检查,以便及时发现低渗血症和低渗性脑病。

\subsection{治疗}

\subsubsection{病因治疗}

纠正基础疾病,药物引起SIADH者需立即停药。地美环素可拮抗AVP的作用,抑制肾小管重吸收水分,0.9~1.2g/d,分3次口服。苯妥英钠可抑制神经垂体加压素的释放,对某些患者有效。对重症糖尿病、肺心病、肾病、肝病、心脏病患者,在给予治疗时,应随时警惕和防止医源性低渗血症,纠正水、电解质紊乱。

\subsubsection{纠正低渗状态}

若是缺钠性低钠血症,可根据公式{[}135 −血清钠(mmol/L){]}×体重(kg)×
0.3,计算出缺钠总量(mmol),首先补给3\%氯化钠150~200ml(3\%氯化钠每100ml含Na\textsuperscript{+}
51.3mmol/L),使脑细胞内水移出,然后将所剩部分Na\textsuperscript{+}
换算成生理盐水(每100ml含Na\textsuperscript{+}
154mmol/L)补给即可。如为稀释性低血钠引起,此时机体并不缺钠,主要是严格控制水的摄入(通常每日入水量限于700ml左右)。但为了使细胞内的水移出,也常采取先补3\%氯化钠100~200ml,造成细胞外液瞬时“高渗状态”,使细胞内水被拉出,然后用快速强力利尿剂(如呋塞米20~40mg)静注,将过多的水、钠排出体外。补充3\%氯化钠时应注意心肺功能,滴速应<
20滴/分,必要时静滴前可静脉给予小剂量洋地黄制剂如毛花苷丙0.2~0.4mg,以防高渗盐水引起的左心衰和肺水肿。如有低钾、低钙、低镁时,也应同时补给相应的电解质,以防补钠后上述电解质进一步降低而引起的心律失常和抽搐。

纠正低钠血症的速度不可过快,否则有发生渗透性脱髓鞘作用的危险,主要是脑桥部损害,称为中央脑桥性脱髓鞘形成(central
pontine
myelinolysis,CPM)。发生机制可能与钠浓度升高导致渗透性内皮细胞损伤,使含血管较多的大脑灰质释放对髓鞘有害的物质所致;也可能与低钠血症时脑组织处于低渗状态,快速补充高渗盐水可使血浆渗透压迅速升高进而造成脑组织脱水,血脑屏障遭到破坏,有害物质透过血脑屏障使髓鞘脱失有关。CPM表现为低钠血症纠正后2~6天出现严重的神经系统症状,甚至出现截瘫、四肢瘫痪、失语等严重并发症,这些变化往往是不可逆的。因此在最初治疗的数小时内补充3\%高渗盐水时应注意:①低血钠伴有抽搐或昏迷及已有症状且血钠仍继续降低者,纠正血钠的速度应在最初的3~4小时每小时血钠升高不超过1.5~2mmol/L,直至症状缓解,但24小时不超过10~12mmol/L,48小时不超过18mmol/L;②如有症状,但未达上述标准者,纠正速度在最初的3~4小时不超过每小时1.0mmol/L,24小时不超过8mmol/L,48小时不超过18mmol/L。对于症状性低钠血症发生时间不明确者,24小时不超过8mmol/L。一般可先纠正到120~125mmol/L,或虽未达到该水平,但低钠血症症状已改善。③每2小时监测神经系统体征和血清电解质水平。

\subsubsection{对症支持疗法}

\paragraph{脱水治疗}

对存在明显颅内压增高者,应及时给予脱水剂治疗。由于低渗性脑病不似血管源性脑水肿破坏血脑屏障,其血脑屏障完整,故用脱水剂能增加脑组织和体液间的渗透压梯度,对脑组织有脱水作用。常用的脱水剂有20\%甘露醇等,其用法及注意事项参见第41章“颅高压危象”部分。

\paragraph{肾上腺皮质激素}

通过调节血脑屏障而增加脑脊液(CSF)回吸,减少CSF的产生,还具有抑制ADH分泌等作用,故可改善病情。可用地塞米松10~30mg/d加入液体静滴,连用3~5天即可。

\paragraph{防治感染}

注意翻身,保护皮肤,避免褥疮发生。作好口腔清洁卫生。应用抗生素防治感染。

\paragraph{支持疗法}

加强营养。血浆蛋白低者,可适量输入白蛋白、血浆制品。静滴复方氨基酸、维生素及微量元素供给机体代谢需要,并给予脑细胞代谢活化剂如辅酶A
(CoA)、ATP、脑蛋白水解物注射液(脑活素)等。
\protect\hypertarget{text00104.html}{}{}

\hypertarget{text00104.htmlux5cux23CHP4-4-4}{}
参 考 文 献

1. 陈灏珠 ,林果为.实用内科学.第13版.北京:人民卫生出版社,2009:981

2. Decaux G,soupart A. Treatment of symptomatic hyponatremia. Am J Med
Sci,2003,326(1):25

3. 陆再英 ,钟南山.内科学.第7版.北京:人民卫生出版社,2008:707

\protect\hypertarget{text00105.html}{}{}

\chapter{急性感染中毒性脑病}

急性感染中毒性脑病(acute infectious-toxic
encephalopathy,AITE)亦称急性中毒性脑炎(acute toxic
encephalitis,ATE),是指在全身性急性感染、传染性疾病(如肺炎、菌痢、流感、白喉、百日咳、猩红热、伤寒、肾盂肾炎等)的病程中(或恢复期),由于脑缺氧、微生物的毒性产物、体内复杂的代谢紊乱及毒性代谢产物的堆积而产生的中枢神经系统中毒性改变,伴大脑功能的障碍,临床上突出表现为意识障碍、昏迷、抽搐、轻瘫、病理反射等脑炎样神经与精神症状,并排除各种脑炎、脑膜炎的临床综合征。本病定义涉及以下内容:①所涉及的急性感染系指中枢神经系统以外的全身性急性感染;②病程中产生的毒性物质和(或)代谢紊乱引起脑功能障碍或造成继发性病理改变而出现精神神经症状;③中枢神经系统感染所致的精神神经症状则不属于本病的范畴。也有学者认为本病是非中枢神经系统感染性疾病过程中高级神经活动极度受抑制所导致的继发性或症状性临床综合征。本病的基本病理改变为脑水肿,脑脊液多无炎症改变,临床症状复杂多样,多呈可逆性或一过性表现,全身感染控制后,脑病症状常逐步好转。多发生于青少年和儿童,其中以1岁以内的婴儿发病最高,占43.3\%,可能与大脑发育不成熟、血脑屏障不完善有关。亦可见于老年人。本病若治疗及时且合理,则预后良好,亦不致造成后遗症。

\subsection{病因与发病机制}

本病并非各种感染性疾病的病原体直接侵犯中枢神经系统所致,而是人体对其毒素的一种中毒反应和继发性脑缺氧损伤;此外高热、脱水、电解质紊乱、惊厥及其他原因引起的缺氧也可引起,因此,凡能影响脑代谢和神经递质改变的疾病均可能引起感染中毒性脑病。早期改变主要是脑血管痉挛、脑缺氧和脑水肿,进而可能导致神经细胞变性,脑组织对毒素的过敏反应加重脑缺氧,毒素进入血液循环后机体的反应,导致内分泌和体液代谢紊乱。脑组织的耗氧量大(约占全身耗氧量的25\%,儿童约占45\%),而脑组织几乎没有氧和葡萄糖储备,大脑供氧量<
2ml/min,或血糖低于1.668mmol/L时均可使大脑皮质和脑中央灰质内神经细胞的代谢活动受到严重影响。感染可致水电解质代谢紊乱,当血渗透压>
320mmol/L时即可发生脑细胞脱水,而低钠血症(尤其是血钠<
125mmol/L)时又可导致脑细胞水肿。感染并发肝肾功能不全时体内蓄积的毒素对脑细胞均有毒性作用。各种因素(如全身性感染、病毒血症、毒素血症或血内毒性成分蓄积时,以及氧、葡萄糖等偏低时)导致的脑内缺血缺氧,必然发生活性氧浓度显著降低,于是引起系列的病理生理学改变。脑免疫组化法检查于脑血管壁神经发现,除儿茶酚胺(catecholamine)和乙酰胆碱(acetylcholine,Ach)外,还有神经肽Y(NPY)、血管活性肠多肽(VIP)、P物质(SP)和降钙素基因相关肽(calcitonin
geno-related
peptide,CGRP)等神经肽,后三者均具脑血管扩张作用。氨基酸神经递质具有兴奋性,起主要作用者为谷氨酸和天冬氨酸,因此称为兴奋性氨基酸(excitatory
aminoacid,EAA),但过度的兴奋却可发生毒性-兴奋毒性(excitotoxicity)。当脑血管缺血时引起谷氨酸浓度增高而导致Ca\textsuperscript{2+}
流入,造成线粒体损伤、蛋白分解和脂肪分解,结果均可使细胞发生坏死,这就是出现脑病症状的病理组织学基础。大脑细胞生物膜脂质代谢严重障碍,造成神经膜脂质过氧化作用增强,也是其发病机制之一。病理变化为脑实质充血、水肿、广泛小出血点,神经细胞混浊肿胀,染色质溶解,空泡变性;小血管充血、水肿,内皮细胞肿胀,血管阻塞,脑缺血,组织软化,但无明显炎性改变。

\subsection{诊断}

\subsubsection{临床表现特点}

本病的基本临床表现为原发病症状加类似脑炎的神经精神系统临床症状。例如伤寒时的感染中毒性脑病表现为持续高热,体温38.5~42℃,大多数呈稽留热型,同时尚有剧烈头痛、头昏、眩晕,食欲消失或频繁呕吐,或表现烦躁不安、谵妄、摸空或双目凝视、淡漠重听,重者可有意识障碍、神志模糊乃至昏迷不醒,并常有幻视、幻听或睡中惊叫。体检可见颈项强直、肌张力增高、腱反射亢进、脑膜刺激征阳性,或伴癫痫样抽搐、尿失禁等,甚至可有吞咽和眼球运动障碍,面瘫乃至偏瘫出现。以上精神神经症状一般与病情轻重密切相关,多发生于极期,随着病情改善及体温下降而恢复。

\subsubsection{辅助检查}

\paragraph{血 、尿常规}

依原发病的不同而异。末梢血象可有白细胞增高、核左移及白细胞中有中毒颗粒等严重感染证据。

\paragraph{脑脊液检查}

压力虽有增高,但糖和氯化物一般正常,白细胞计数正常或轻度增高,蛋白质可有极轻度增高,脑脊液中找不到致病菌。

\subsubsection{诊断注意事项}

在某种非中枢神经系统急性感染性疾病基础上,通常是在高热期中,有时是在退热之后复又突发高热,出现神经-精神症状,或精神错乱、谵妄狂躁,或嗜睡昏迷,神志不清,或惊厥抽搐,或瘫痪麻痹,而不能用低血糖、低血钙解释,并可除外中枢神经系统感染和占位病变者,即应考虑为感染中毒性脑病。在诊断时,须与各种类型的脑炎、脑膜炎鉴别。

\subsection{治疗}

治疗原则是采取病因治疗,脑病对症治疗,并配合以支持疗法的综合措施。在明确诊断的前提下,必须合理选择抗生素。

\subsubsection{原发病治疗}

对于全身性感染疾病,应根据可能的病原选用适宜的抗生素、抗病毒药物。具体应用参见有关章节。

\subsubsection{一般处理}

保持呼吸道通畅,清除分泌物,常规吸氧,供给易消化、高热量、高维生素食物,补充维生素B族及维生素C;适当控制水、钠的摄入以控制脑水肿;必要时行气管插管或气管切开及人工呼吸。

\subsubsection{对症处理}

\paragraph{抗惊厥治疗}

对有惊厥者,可选用地西泮(安定)0.25~1mg/kg(1次注射不宜超过10mg)静注;或苯巴比妥(鲁米那)5mg/kg,肌注,必要时4~6小时后可重复;异戊巴比妥(阿米妥,amytal)5~10mg/kg肌注,或溶于注射用水10ml中以1ml/min速度静注;氯丙嗪0.5~1mg/kg肌注,10\%水合氯醛0.5mg/kg保留灌肠。亦可用紫雪丹、至宝丹或羚羊角粉等药物。

\paragraph{降温}

降低脑代谢、减少脑耗氧量是基本的治疗措施。对高热病例体温每下降1℃,脑代谢率约可下降6.7\%,颅内压降低5.5\%。物理降温可用头枕冰袋或冰帽,腋下、腹股沟等大血管处的酒精擦浴等,和(或)降低室温,应用冰帽时要注意用棉垫或纱布保护双耳避免冻伤。药物降温除退热剂如复方氨基比林、柴胡注射液等和肾上腺皮质激素(如地塞米松)外,可考虑用人工冬眠药物如氯丙嗪、异丙嗪、哌替啶等,尤其是对高热伴惊厥者。具体用法参见本书第142章“人工冬眠疗法”。

\paragraph{脱水降颅内压}

对临床疑有或腰穿证实有颅内压增高者,可应用20\%甘露醇125~250ml,每6~8小时1次,静滴,有血尿、蛋白尿者禁用。甘油是较好的脱水剂,口服剂量为1~2g/(kg•d),静滴量为0.7~1g/(kg•d),成人可用10\%甘油500ml/d,以100~150ml/h速度输入。亦可应用利尿剂如呋塞米(速尿)和高渗葡萄糖液。激素对血管源性脑水肿具有明显的益处,在急性期可短期较大剂量应用,常用地塞米松或氢化可的松静滴。

\paragraph{促进脑细胞代谢药物}

常用的有三磷酸腺苷、胞磷胆碱、甲氯芬酯、辅酶A、细胞色素C、脑活素等。
\protect\hypertarget{text00106.html}{}{}

\hypertarget{text00106.htmlux5cux23CHP4-5-4}{}
参 考 文 献

1. 李梦东,王宇明.实用传染病学.第3版.北京:人民卫生出版社,2004:1400

2. 韩仲岩
,丛志强,唐盛孟.神经病治疗学.第2版.上海:上海科学技术出版社,2004:157

\protect\hypertarget{text00107.html}{}{}

\chapter{颅高压危象}

颅内压(intracranial
pressure,ICP)系指颅腔内容物,包括脑组织、颅内血液及颅内脑脊液对颅腔壁所产生的压力。它通常是以人的侧脑室内液体的压力为代表。在椎管蛛网膜下腔通畅的情况下,侧脑室内液体的压力与侧卧位时作腰椎穿刺所测得的压力大体相等,因此常以此压力作为代表。成年人的正常ICP为5.0~13.5mmHg,或70~180mmH\textsubscript{2}
O,平均为100mmH\textsubscript{2}
O,女性稍低;儿童为3.0~7.5mmHg,或40~100mmH\textsubscript{2}
O,平均为70mmH\textsubscript{2}
O。正常成人侧卧位腰椎穿刺脑脊液压力如超过200mmH\textsubscript{2}
O即为颅内压增高。颅高压危象系指因各种病因引起的患者急性或慢性颅内压增高,病情急剧加重出现脑疝症状而达到危及生命的状态。如不能及时诊断和解除颅内压增高的病因,或采取措施缓解颅内压力,则患者常因脑疝而致死。

\subsection{病因与发病机制}

\subsubsection{颅内压增高的病因}

凡能引起颅腔内容物体积增加的病变均可引起颅内压增高。常见的病因可分为颅内病变和颅外病变。

\hypertarget{text00107.htmlux5cux23CHP4-6-1-1-1}{}
(一) 颅内病变

\paragraph{颅内占位性病变}

颅内肿瘤、血肿、脓肿、囊肿、肉芽肿等,既可占据颅腔内一定的容积,又可阻塞脑脊液的循环通路,影响其循环及吸收。此外,上述病变均可造成继发性脑水肿,导致颅内压增高。

\paragraph{颅内感染性疾病}

各种脑膜炎、脑炎、脑寄生虫病,既可以刺激脉络丛分泌过多的脑脊液,又可以造成脑脊液循环受阻(梗阻性及交通性脑积水)及吸收不良;各种细菌、真菌、病毒、寄生虫的毒素可以损伤脑细胞及脑血管,造成细胞毒性及血管源性脑水肿;炎症、寄生虫性肉芽肿还可起到占位作用,占据颅腔内的一定空间。

\paragraph{颅脑损伤}

可造成颅内血肿及水肿。

\paragraph{急性脑血管病}

如脑出血、脑梗死、蛛网膜下腔出血及脑静脉窦血栓形成等。

\paragraph{脑缺氧}

各种原因造成的脑缺氧如窒息、麻醉意外、CO中毒,以及某些全身性疾病如肺性脑病、癫痫持续状态、重度贫血等,均可造成脑缺氧,进一步引起血管源性及细胞毒性脑水肿,导致颅内压增高。

\paragraph{脑积水}

当脑脊液分泌过多、循环过程受阻、吸收障碍或三者兼而有之引起脑积水,导致颅内压增高。脑脊液循环过程受阻引起脑积水叫阻塞性脑积水。脑脊液分泌过多或吸收障碍引起脑积水叫交通性脑积水。脑积水病变性质可以有先天性发育异常、炎症、出血、肿瘤和外伤等,一般在婴幼儿以先天性发育异常多见,在成人以继发性病变多见。

\hypertarget{text00107.htmlux5cux23CHP4-6-1-1-2}{}
(二) 颅外病变

\paragraph{心、肺、肾和肝功能障碍或衰竭}

心衰、休克、气道梗阻、急性肺损伤、ARDS、肝功能衰竭和肾功能衰竭均可并发脑水肿引起颅内高压。

\paragraph{中毒}

铅、锡、砷等中毒;某些药物中毒,如四环素、维生素A过量等;自身中毒如尿毒症、肝性脑病等,均可引起脑水肿,促进脉络丛分泌脑脊液等,并可损伤脑血管的自动调节作用,而形成高颅压。

\paragraph{内分泌功能紊乱}

年轻女性、肥胖者,尤其是月经紊乱及妊娠时,易于发生良性颅内压增高,可能与雌激素过多、肾上腺皮质激素分泌过少而发生的脑水肿有关。肥胖者可能与部分类固醇溶于脂肪组织中不能发挥作用而造成相对性肾上腺皮质激素过少有关。

\paragraph{其他}

如中暑、输血、输液反应、放射线脑病以及脊髓、马尾肿瘤等也可引起颅内高压。

\subsubsection{颅内压的生理调节}

颅腔是由颅骨组成的密闭腔隙,其容积不变。其内有三大内容物:脑组织、脑血流、脑脊液。当其中一个增大时,另两个或至少其中一个的体积就要缩小,以保持颅内压的稳定。颅内压与血压、呼吸关系密切,收缩期颅内压略有增高,舒张期颅内压稍下降;呼气时压力略增,吸气时压力稍降。

\paragraph{脑脊液的调节作用}

脑脊液占颅腔总体积的10\%,在颅腔三大内容物中活动性最大,最易被挤出颅腔,即通过脑脊液的转换作用可得到的最大调整空间为10\%。异常情况下,脑室壁可能发生异位吸收,使颅压在一定时期内保持正常(如正常颅压脑积水时)。脑脊液的吸收速度取决于蛛网膜下腔与静脉窦内的压差,当颅内压低于静脉压时,脑脊液吸收几乎停止,当颅压高于70mmH\textsubscript{2}
O时,脑脊液的吸收量与压力成正比增加,同时,其分泌减少,部分脑脊液被挤入脊腔,结果颅腔内脑脊液容量减少,使颅内压得到调节,若脑脊液生成过多或循环梗阻或吸收障碍,颅腔内脑脊液容积不断增加,超过其调节水平,即可发生颅内压增高。

\paragraph{脑血流的调节作用}

脑血流占颅腔总容积的2\%~7\%,平均每分钟1200ml的流量。

\includegraphics[width=2.59375in,height=0.39583in]{./images/Image00152.jpg}

从上述公式看出,颅内压增高时脑血流量减少;由于脑血流量减少,反射性地引起脑血管扩张,血管阻力减少,其结果又使脑血流量增加,从而保证了脑的供血。而在颅内压明显增高时,上述代偿机制失调,脑血流量随之减少,其结果一方面是使颅内压有所下降,但同时也使脑部供血受到影响。脑血流量对颅内压的调节作用不如脑脊液,其对颅内压增高的“容积代偿”能力有限。一般认为颅内压增高到需要依靠减少脑血流来调节时,则意味着病变的严重性及机体自动调节功能的损伤。

\paragraph{脑组织的调节作用}

在颅腔三大内容物中,脑组织最为稳定,它不易被挤压而让出空间来调整颅内压。急性颅内压增高时,脑组织不可能发生明显压缩以起代偿作用;但在慢性颅内压增高时,可以出现脑细胞坏死、纤维变性以至脑萎缩,从而腾出一部分空间缓冲颅内压增高。

\subsubsection{颅内压增高的发病机制}

颅内压的调节主要是颅内空间的调整,如通过脑脊液的转换作用,通过颅内静脉血被挤压出颅腔等而让出一定空间,使颅内压维持在一定水平而不至过高。但这种调节是有限的,若造成高颅压的病因持续存在,并不断扩张,则终将使所有可以代偿的空间全部利用,而出现显著的颅内压增高。从临床病情演变过程,可将颅内压增高的发生发展分为代偿期、早期、高峰期和晚期等四个阶段:

\paragraph{代偿期}

为病情初期发展阶段。因病变所致的颅腔内容物增高,尚未超过颅腔的代偿容积,颅内压仍可保持正常,亦常无颅内压增高的临床表现。

\paragraph{早期}

为病情早期发展阶段。因颅腔内容物体积增加的总和已超过颅腔的代偿容积,故可逐渐出现颅内压增高和相应临床症状如头痛、呕吐、视乳头水肿等。脑组织虽有轻度缺血缺氧,但脑血管的自动调节功能良好,而仍能获得足够血流量,如能及时解除病因,脑功能恢复较易,预后较好。

\paragraph{高峰期}

为病情严重发展阶段,脑组织缺血缺氧严重,脑功能损伤明显,出现较重的头痛、恶心、呕吐、视力减退和视乳头水肿,患者意识模糊甚至昏迷等相应的颅内压增高症状和体征。如脑干呼吸、心血管运动中枢功能受损,导致脉搏与呼吸深慢;同时因脑血管自动调节功能此时已有受损,主要靠全身性血管的加压反应来提高血压和维持脑部血流量,同时会出现心跳和脉搏缓慢,呼吸节律紊乱及体温升高等各项生命体征发生变化,这种变化即称为库欣反应(图\ref{fig41-1}中之A-B段),多见于急性颅内压增高病例,慢性者则不明显。如不及时采取有效治疗措施,常易迅速出现呼吸、心搏骤停等脑干功能衰竭症状。

\paragraph{晚期}

为病情濒死阶段。患者常处于深昏迷中,一切生理反应消失,双侧瞳孔散大和去大脑强直、血压下降(如图\ref{fig41-1}中之B-C段),心搏弱快,呼吸不规则甚至停止。脑组织缺血缺氧极严重,脑细胞功能已近停止,预后极差。

\subsection{诊断}

\subsubsection{有引起颅内压增高的病因存在}

\subsubsection{颅内压增高的临床表现}

典型临床表现为头痛、呕吐和视乳头水肿三联征。但三者同时出现者不多。

\paragraph{头痛}

系因颅内压增高刺激颅内敏感结构如脑膜、血管和脑神经受到牵扯、压迫所致。头痛为颅内高压的最常见症状,发生率约80\%~90\%。开始为阵发性,以后发展为持续性,以前额及双颞部为主,后颅凹病变头痛多位于枕部。咳嗽、喷嚏、用力等情况均可使头痛加重。头部活动时头痛也加重,患者常被迫不敢用力咳嗽、不敢转动头部。

\paragraph{恶心 、呕吐}

是因颅内压增高,使延髓呕吐中枢受激惹所引起。常在清晨空腹时发生,或于剧烈头痛同时发生,常与饮食无关,可呈喷射性,但不多见。位于后颅凹及第四脑室的病变较易引起呕吐。儿童头痛不显著,呕吐有时是唯一症状。

\paragraph{视神经乳头水肿}

\begin{figure}[!htbp]
 \centering
 \includegraphics[width=4.01042in,height=2.70833in]{./images/Image00153.jpg}
 \captionsetup{justification=centering}
 \caption{Cushing反射示意图}
 \label{fig41-1}
  \end{figure} 

视神经鞘为脑蛛网膜的延续。视网膜中央动、静脉位于视神经鞘内与视神经伴随而行,在视神经乳头处出入眼底。当颅内压增高时,蛛网膜下腔内的压力增高,视神经鞘内压力也增高,而使网膜中央静脉回流受阻,静脉内压力增高。检眼镜检查可见视乳头隆起、边缘不清、颜色发红,眼底静脉迂曲、怒张。由于毛细血管扩张、出血,检查时可见到点、片状,甚至火焰状出血。早期或轻度的视神经乳头水肿,一般不影响视力,如颅高压持续存在或继续发展,可出现盲点扩大,中心视力暗点及阵发性黑矇,病情再进一步发展,发生继发性视神经萎缩,视力持续下降直至失明。视神经乳头水肿虽是颅内压增高的特征性体征,但并非所有病例均有。

\paragraph{展神经麻痹与复视}

因展神经在颅内行走较长,颅内压增高时容易因挤压及牵拉受伤而出现单侧或双侧不全麻痹,出现复视。此症状无定位意义。故又称为“假定位征”。

\paragraph{意识障碍}

反应迟钝、嗜睡、昏睡至昏迷的各种意识障碍均可发生。系与颅内压增高时脑干网状结构上行激活系统及广泛大脑皮质受损有关。

\paragraph{抽搐}

、去大脑强直发作 与颅内压增高时脑干受压、脑供血不足、脑膜受刺激等有关。

\paragraph{生命指征的改变}

血压增高、脉搏缓慢、呼吸慢而深等;随着颅内压增高,可出现瞳孔缩小、对光反射迟钝、或忽大忽小、边缘不整、变化多端。常预示脑疝即将发生,应立即采取抢救措施。

\paragraph{并发全身其他系统病变的临床表现}

①胃肠功能紊乱及消化道出血:部分颅内压增高的患者可表现胃肠功能的紊乱,出现呕吐,胃及十二指肠出血及溃疡和穿孔等。这与颅内压增高引起下丘脑自主神经中枢缺血而致功能紊乱有关。也有人认为颅内压增高时,消化道黏膜血管收缩造成缺血,因而产生广泛的消化道溃疡。②神经源性肺水肿:在急性颅内压增高患者中,发生率高达5\%~10\%。这是由于下丘脑、延髓受压导致α-肾上腺素能神经活性增高,血压反应性增高,左心负荷过重,左心房及肺静脉压增高,肺毛细血管压力增高,液体外渗,引起肺水肿,患者表现为呼吸急促,痰鸣,并有大量泡沫状血性痰液。

\paragraph{小儿颅内压增高的表现}

小儿因不会诉说头痛,常表现为烦躁、哭闹或脑性尖叫,频繁呕吐、抽搐以至去脑强直发作,意识丧失。查体可见囟门隆起、扩大,颅缝裂开,头围增大,以及头皮静脉怒张;额、顶、颞及枕部突出膨大呈圆形,颈部静脉充盈,对比之下颜面很小;严重颅内压增高,压迫眼球,形成双目下视,巩膜外露的特殊表情,称落日征。

\subsubsection{脑疝的表现}

各种原因引起的颅内压增高,都可导致脑组织向压力相对较低的部位移位,形成脑疝。脑疝一般是逐渐形成的,但遇剧烈呕吐、咳嗽或腰穿等情况时,颅内压可急剧升高或颅腔与椎管间的压力失去平衡,可导致脑疝的骤然发生或原有脑疝加重。因此,在临床上怀疑慢性颅内压增高是因颅内占位性病变所引起时,作腰椎穿刺应慎重或尽量不做,以免致脑疝;确因诊断需要检查脑脊液时,腰穿前应使用一次高渗性脱水剂,穿刺放脑脊液时尽量不要拔出针芯且放液量宜少,穿刺后去枕平卧,头低位,并继续用高渗性脱水剂治疗。颅内可发生脑疝的部位虽多,但并非所有脑疝均有临床意义。临床上常见而危害大的有小脑幕裂孔下疝、枕骨大孔疝和小脑幕裂孔上疝,它们可单独存在或合并发生。详见本书第25章“急性脑功能衰竭”。

\subsubsection{颅内压监测}

利用各种颅内压监测技术对颅内压进行检测,可直接获得颅内压的数据为颅内高压诊断提供最直接的依据。目前颅内压监测技术分为有创颅内压监测技术和无创颅内压监测技术。有创颅内压监测技术包括脑室内插管法、硬脑膜外传感器、光纤探头监测ICP和腰椎穿刺检测ICP。有创颅内压监测技术准确性好,特别是脑室内插管法被认为ICP检测的“金标准”。但其缺点是有创、易感染、技术要求高、耗材贵不易临床推广。无创颅内压监测技术其优点是无创、技术要求低、不会引起任何不良反应、无耗材消耗、可以反复进行监测。但其准确性一般,能达到90\%。

\subsubsection{诊断性治疗}

用脱水药物如20\%甘露醇等静注,如颅内压增高症状缓解,则有诊断价值。

\subsubsection{辅助检查}

电子计算机X线断层扫描(CT)、磁共振成像(MRI)、脑血管造影(DSA)、头颅X线摄片等既可辅助判断颅内压增高,也可帮助明确颅内压增高的病因。腰椎穿刺测量脑脊液的压力可直接判断颅内压的高低。

\subsubsection{颅内压增高的分类与分级}

根据颅内压增高的范围可分为:①弥漫性颅内压增高:在颅内各分腔间没有大的压力差,其耐受限度较高,很少引起脑疝,压力解除以后神经的恢复较快。如见于蛛网膜下腔出血、弥漫性脑膜炎、脑水肿等。②局灶性颅内压增高:压力先在病灶附近增高然后传递到颅内各处,在颅内各分腔之间有较明显的压力差,其耐压限度较低,常有明显的脑组织移位(脑疝),超过一定时间以后解除压力,受损的脑组织功能恢复较慢。区别这两类颅内压增高对于估计预后与决定治疗有重要意义。根据ICP的增高程度可以分为三级:压力在200~260mmH\textsubscript{2}
O者为轻度增高;261~520mmH\textsubscript{2}
O者为中度增高;超过520mmH\textsubscript{2} O者为严重增高。

\subsection{治疗}

对颅内压增高的患者,既要及时治疗原发病变,又要尽可能降低颅内压,及时中断恶性循环,防治脑疝。

\subsubsection{一般疗法}

包括:①卧床休息,密切观察生命体征;②抬高头部约15°~30°,以利颅内静脉回流;③吸氧,保持呼吸道通畅,昏迷患者不能排痰者,应考虑气管切开;④呕吐频繁者,应暂禁食,静脉补足液体和热量或改给全胃肠外营养;⑤限制水盐摄入量,静滴液量成人每日不超过1500~2000ml(不包括脱水剂量),其中电解质液不超过500ml;⑥防止受凉、咳嗽、避免激动、生气,保持大便通畅,防止便秘;⑦对症处理:如疼痛、呕吐者,给以镇静止吐药物;⑧有条件时可行颅内压监测,以利于指导用药。

\subsubsection{并发高血压的处理}

当颅内压增高到一定程度时脑血管自动调节功能就受损,主要靠全身性血管的加压反应来提高血压以提高脑灌注压维持脑部血流量。因此,颅内压增高的患者血压升高是机体的一个自我保护性反应,不必要强行将血压降得过低,以免降低脑灌注压加重脑损害。对此类患者血压应控制在什么水平及如何控制目前还缺乏统一标准。借鉴急性脑血管病高血压处理方法,提出如下建议:①收缩压<
220mmHg或舒张压<
120mmHg时应观察,除非其他终末器官受损,如主动脉夹层分离、急性心肌梗死、肺水肿或高血压脑病;②收缩压>
220mmHg或舒张压121~140mmHg时用拉贝洛尔10~20mg静注,1~2分钟,每10分钟可重复或加倍使用,最大剂量300mg;或者尼卡地平5mg/h静滴,每5分钟增加2.5mg/h,直至最大剂量15mg/h,直到达到预期效果;目标是使血压降低10\%~15\%;最好应用微量输液泵,避免血压降得过低。③如有ICP检测,CPP应保持在60~70mmHg以上。但必须强调,对于颅内压增高并发高血压的处理,应重点针对病因治疗,以便有效降低颅内压,血压会自动下调。

\subsubsection{脱水疗法}

脑水肿是构成颅内压增高的主要因素,控制脑水肿的发生与发展对降低颅内压极为重要。采用脱水药物是最常用的降低颅内压力的方法。当颅内占位性病变的晚期突然发生脑疝时,也常需先用脱水疗法,待症状缓解后,再行手术治疗。常用的脱水剂有下列几种:

\hypertarget{text00107.htmlux5cux23CHP4-6-3-3-1}{}
(一) 渗透性脱水剂

包括各种高渗性晶体及大分子药物。使用后由于血脑屏障的选择性作用,药物进入血液后不能迅速转入脑与脑脊液中,致使血液呈现高渗状态,造成血液与组织间渗透压差,促使组织间液、细胞内液及脑脊液内的水分转移至血液内;且高渗物质由肾小球滤出时,在近端肾小管中造成高渗透压而产生利尿作用;同时因血液的高渗透压反射性的抑制脉络丛的分泌,使脑脊液分泌减少,结果均致颅内压下降。但该类药物只有在脑血管功能正常时才能很好地发挥作用,脑血管损伤时其疗效受到影响。常用药物有:

\paragraph{甘露醇}

甘露醇是单糖,分子量为182,在体内不被代谢,为广泛应用的渗透性脱水剂。甘露醇对血糖没有影响,因此糖尿病患者也可以使用。其作用机制:首先是组织的脱水作用,在血管壁完整的情况下,通过提高血浆渗透压,导致脑组织内细胞外液、脑脊液等水分进入血管内。其次是利尿作用,通过增加血容量,促进前列腺素Ⅰ分泌,从而扩张肾血管,提高肾小球滤过率;另外由于甘露醇在肾小管重吸收率低,故可提高肾小管内液渗透浓度,主要减少远端肾小管对水、Na\textsuperscript{+}
和其他溶质等的重吸收,从而将过多水分排出体外。它尚有清除自由基、减少其对细胞脂膜的破坏作用。虽然甘露醇的脱水作用强,是临床最常使用的脱水药物,但目前对使用甘露醇的剂量、次数及疗程等仍无统一意见,甚至存在较大争议。已知1g甘露醇可带出12.5ml水分,尿钠排泄0.5g。正常血浆渗透压范围是280~310mOsm/L,甘露醇高渗脱水的最佳作用区间是310~330mOsm/L,当渗透压超过330mOsm/L时就会产生肾和神经组织损害。甘露醇每次总量不宜超过60g,每日总量不宜超过300g。甘露醇治疗脑水肿的用量很关键,用量过少起不到脱水降颅压的作用,剂量过大又会产生不良反应,其量效关系非常明确。一般情况下,颅内压较轻或控制较好者用药剂量相应减少,取有效量至最佳有效量之间即可;对于严重颅内高压,甚至脑疝抢救时,即使最佳有效剂量也往往不够理想,此时就应以抢救生命为重,须短期快速静脉注射20\%甘露醇250ml甚至500ml才能取得疗效,或者配合其他脱水药物一起使用。甘露醇的临床常用剂量为每次0.25~0.5g/kg,浓度为20\%,于30~40分钟静滴完,进入血管后10~20分钟开始起作用,半衰期为71.15分钟±
27.02分钟,2~3小时降颅压效果最强,可维持作用4~6小时,大部分4小时左右经肾脏排出,故临床上间隔4~6小时用药一次。Marshall等监测8例脑损伤患者的ICP发现,不同剂量甘露醇间隔同样时间(8小时),小剂量(0.25g/kg)与大剂量(1g/kg)治疗后ICP降低的程度没有差异。所以,甘露醇用量不宜过大,用药时间不宜过长,停药时应逐渐减量。1999年美国心脏协会(AHA)方案建议,20\%甘露醇的用法为每次0.25~0.5g/kg,4~6小时1次。甘露醇的反跳现象:甘露醇的脱水作用有赖于血脑屏障的完整性,当血脑屏障的通透性增高时,甘露醇就可以逐步通过血脑屏障聚积于脑组织间隙,这样当停止静脉输入一段时间后,血浆渗透压就可能暂时低于脑组织的渗透压,此时水分由血浆反流入脑组织,使脑组织的含水量再度增高,脑水肿加重,颅内压回升,即出现所谓反跳现象,因此要严格控制用药间隔时间。还有学者在研究中发现,渗透性脱水剂从脑脊液清除的速率低于从血中清除的速率,所以停药后甘露醇在脑脊液和血中的渗透压梯度会短暂逆转,反而导致ICP较治疗前增高,形成所谓反跳现象。最常见不良反应为电解质紊乱,其他尚有排尿困难、血栓性静脉炎、过敏反应、甘露醇肾病等。其中甘露醇肾病常于大剂量快速静脉滴注时发生,往往会引起急性肾衰,一旦发生,立即停用甘露醇,改用其他脱水剂。轻者早期可应用血管扩张剂或利尿剂,病情严重者应透析治疗。

\paragraph{甘油果糖}

甘油果糖(10\%甘油、5\%果糖、0.9\%氯化钠)的渗透压是人体血浆的7倍,经静脉输液后能提高血浆渗透压,在血浆和脑之间形成渗透梯度,使水从脑转移向血浆,从而使脑组织脱水,并使脑脊液的产生减少,降低颅内压,消除脑水肿。甘油果糖不增加肾脏负担,无肾脏损害作用。甘油果糖进入体内参与代谢,产生水和二氧化碳,同时每500ml可提供1339kJ(320千卡)的热量。通过血脑屏障进入脑组织,氧化成磷酸化基质,参与脑代谢并提供热量,增强脑细胞活力,使脑代谢改善。同时甘油果糖能有效地改善血液流变学状态,改善微循环,增加脑血流量及供氧量。甘油果糖单用降颅压起效慢,作用维持时间长,费用大。现在多主张将甘油果糖和甘露醇联合应用,既迅速降颅压,改善症状,又减轻肾脏负担,保护肾功能,降低费用支出,也克服了甘露醇的颅内压反跳现象。

\paragraph{甘油}

一些学者认为,甘油有增加脑血流,改善脑代谢和减轻脑水肿的作用。其作用温和而持久,没有反跳现象,不会导致电解质紊乱,适用于肾功能不全或长期未控制的老年高血压患者。但它起效较慢,多在用药1周后效果显著,且在快速滴注时会出现溶血作用,导致血红蛋白尿,故滴速应控制在30滴/分钟以下,与甘露醇联合应用效果较好。汇总分析也表明,它能降低卒中后14天内的死亡率,但不能降低1年内的死亡率。它可以口服或静脉注射。①口服法:口服剂量为1~2g/(kg•d),用生理盐水配成50\%的甘油盐水,每次30~50ml口服,每日3次。副作用为恶心、呕吐、腹胀。②注射法:用复方甘油注射液,其中含10\%甘油,90\%生理盐水,为一种长效脱水剂。成人每次500ml,以100~150ml/h速度静脉输入,每日1~2次。注射后2~4小时发挥作用,持续18小时。

\paragraph{高渗盐水}

用高渗盐水降颅内压是目前学者们研究的热点之一。研究表明,高渗盐水能有效地减轻脑水肿、降低颅内压,其疗效甚至更优于目前临床最为常用的甘露醇。《ASA/AHA2007自发性脑出血治疗指南》亦明确将高渗盐水和甘露醇同时作为推荐的降颅压药物(Ⅱa类,证据水平C)。纳入6个随机对照试验的系统评价结果表明,高渗盐水在降颅压幅度、起效时间、最大效应时间和维持时间上均优于甘露醇,且不降低颅内灌注压和不增加全身副作用的发生率。高渗盐水减轻脑水肿、降低颅内压比甘露醇更安全有效,是一种可供选用的脱水剂。高渗盐水降低颅内压,提高脑灌注压的机制可能与下列因素有关:①提高血浆渗透压,使组织间液、脑细胞内液进入血液中,从而减轻脑水肿、降低颅内压力;②使血管内皮细胞、红细胞脱水,增加脑血流量。但静脉注射高渗盐水可能会导致血浆渗透压过高、充血性心力衰竭、电解质紊乱、酸碱失衡、脑桥中央髓鞘破坏等副作用。下一步需要解决的问题是:①用高渗盐水降颅压的最佳用量与时机;②如何避免副作用的发生;③该药能否成为一线降颅压药物。

\paragraph{人体白蛋白}

它是通过提高血浆胶体渗透压使脑组织间液的水分进入循环血中,达到脱水降颅压的作用。提高胶体渗透压可较长时间保持完好的血流动力学及氧的输送,而且扩张血容量后,使抗利尿激素分泌减少而利尿,对血容量不足、低蛋白血症的颅内高压、脑水肿患者尤为适用。因其增加心脏负荷,有心功能不全者须慎用。血脑屏障严重破坏的病变,白蛋白能漏出至毛细血管而加剧颅内高压,使用时须注意。另外,白蛋白价格昂贵,患者很难承担其费用。

\hypertarget{text00107.htmlux5cux23CHP4-6-3-3-2}{}
(二) 利尿性脱水剂

本类药物抑制肾小管对 Na\textsuperscript{+} 、Cl\textsuperscript{−}
、K\textsuperscript{+}
的重吸收,使尿量显著增加,循环血量减少,组织水分逸出,造成机体脱水而间接地使脑组织脱水,降低颅内压。但单独应用则其降低颅内压作用较弱;若与渗透性脱水剂合用,则可加强降颅内压效果。常用利尿剂有:呋塞米每次20~40mg,每日2~4次肌注或静注;布美他尼(丁尿胺)每次0.5~1mg肌注或静注,必要时30分钟后重复使用一次。呋塞米主要用于协助高渗性脱水剂的降颅压作用,心功能或肾功能不全的患者中应用此药可减轻心脏负荷,促进物质排泄,还可减少甘露醇的用量,从而减轻对肾小管的损害。一般建议与甘露醇交替使用。Roberts等通过动物实验研究呋塞米与甘露醇应用的最佳顺序,发现应用甘露醇15分钟后再用呋塞米可产生最明显和最持久降低ICP的效果。

\hypertarget{text00107.htmlux5cux23CHP4-6-3-3-3}{}
(三) 脱水疗法的注意事项

包括:①渗透性脱水剂可使钠、钾、氯的排出量稍有增加,但因其排出的水量很大,血清中电解质可无明显的变化,甚至血液浓缩反有相对增高的现象。1~2次用药可不必补电解质,如应用的时间较长或次数较多,则应严密观察电解质的变化并给予适量的补充。但利尿性脱水剂如呋塞米与布美他尼则易致电解质紊乱,不宜长期、频繁使用。②对颅内压增高并心功能不全、肺水肿、急性肾功能衰竭少尿期,一般不宜应用渗透性脱水剂,因可在短时间内使血容量急剧增加而加重心力衰竭;此时,最适宜用利尿性脱水剂。③在脱水剂疗法中,正确地掌握维持出入量的平衡是十分重要的,若入量过多则达不到脱水目的;反之,则可致血容量不足甚至发生低血容量性休克。一般应限制液体入量在1500~2000ml/d之内,其中包括盐水500ml。

\subsubsection{肾上腺皮质激素}

其减轻脑水肿、降低颅内压之作用机制是多方面的:①改善血脑屏障功能,降低毛细血管通透性,减轻血管源性脑水肿;②改善细胞膜的功能,重建细胞内外钾、钠离子的正常分布,减轻细胞毒性脑水肿;③抗氧化作用,对抗自由基,防止细胞膜磷脂的自由基反应,维持细胞膜的正常功能(自由基可使细胞膜上的多价不饱和脂肪酸产生脂质过氧化反应而失去功能);④抑制垂体后叶抗利尿激素的分泌,同时还能增加肾血流量抑制醛固酮的分泌。在降低颅内压力的效果上不及渗透性脱水剂,然而,其作用持久、温和,与其合用,能提高降压效果,防止反跳。常用地塞米松20~40mg/d或氢化可的松200~600mg/d,分次静滴。应注意防治其以下副作用:①抑制机体免疫力易导致感染;②使糖耐量降低,血糖升高;③诱发上消化道出血。目前在是否主张使用肾上腺皮质激素降低颅内压方面尚无统一意见。建议根据具体病情,权衡利弊作出选择。

\subsubsection{病因治疗}

各种原因所致的颅内压增高,均应采取积极而有效的方法对其原发病进行治疗,才能阻断恶性循环,使各种对症治疗收到良效。如对颅内肿瘤、各种炎症、脑血管病等,均应针对不同病因给以相应治疗。

\subsubsection{其他治疗}

包括:①人工冬眠疗法。②人工过度换气:采用短期控制性过度换气,使呼吸加深加快,降低PaCO\textsubscript{2}
至32~35mmHg,可诱导脑血管收缩,导致颅内压下降,停止过度换气后其效果可维持数小时。尤其用于外伤性颅内高压。③亚低温治疗:临床试验已经证实对外伤性颅内高压的患者实施亚低温治疗(32~35℃)可有效降低颅内压,未发现明显的心律失常、凝血机制障碍和感染等并发症。④脑保护剂及脑细胞代谢活化剂的运用,如ATP、COA、细胞色素C、脑活素等,均可酌情选用。⑤高压氧疗法:适用于缺氧引起的脑水肿病例。

\subsubsection{颅高压危象的外科手术治疗}

临床上颅高压危象可导致脑疝形成。脑疝症状一旦出现,除立即经静脉快速滴注或推注脱水剂、以期望缓解症状外,还应依不同情况尽可能做手术处理。

\paragraph{急性脑室扩张}

急性脑室扩张多见于小脑出血或梗死向前推压第四脑室、蛛网膜下腔出血、脑实质出血破入蛛网膜下腔等情况。一旦出现急性脑室扩张颅内压会急剧升高。在药物治疗无效时,应急诊行侧脑室穿刺引流术。

\paragraph{小脑幕裂孔下疝}

若病因诊断明确,应立即开颅手术,切除病变以达到缓解颅内压增高的目的;对于未能明确诊断的病例,应作紧急颞肌下减压术,如情况许可并应将小脑幕裂孔边缘切开,促使脑疝的复位。

\paragraph{枕骨大孔疝}

应紧急作脑室穿刺,缓慢放出脑室液,使颅内压慢慢下降,然后施行脑室持续引流术。待脑疝症状缓解后,对颅后凹开颅术,切除原发病变,对脑积水病例施行脑脊液分流术。
\protect\hypertarget{text00108.html}{}{}

\chapter{高血压危象}

在急诊工作中,常常会遇到一些血压突然和显著升高的患者,伴有症状或有心、脑、肾等靶器官的急性损害,如不立即进行降压治疗,将产生严重并发症或危及患者生命,称为高血压危象(hypertensive
crisis)。其发病率约占高血压患者的1\%~5\%左右。

有关高血压患者血压急速升高的术语有:高血压急症、高血压危象、高血压脑病、恶性高血压、急进型高血压等。美国高血压预防、检测、评价和治疗的全国联合委员会第七次报告(JNC7)对高血压急症(hypertensive
emergencies)和次急症(hypertensive
urgencies)的定义简单明了。高血压急症是以伴有即将发生或进展的靶器官功能障碍为特征的血压急剧升高(通常超过180/120mmHg),为防止或限制靶器官的受损,需要迅速降低血压(可以不达到正常范围)。如果仅有血压显著升高,但不伴靶器官新近或急性功能损害,则定义为高血压次急症。广义的高血压危象包括高血压急症和次急症;狭义的高血压危象等同于高血压急症。

高血压急症主要包括:①急性脑血管病:脑出血、脑动脉血栓形成、脑栓塞、蛛网膜下腔出血等。②主动脉夹层动脉瘤。③急性左心衰竭伴肺水肿。④急性冠状动脉综合征(不稳定心绞痛、急性心肌梗死)。⑤子痫前期、子痫。⑥急性肾功能衰竭。⑦微血管病性溶血性贫血。

高血压次急症主要包括:①高血压病3级(极高危)。②嗜铬细胞瘤。③降压药物骤停综合征。④严重烧伤性高血压。⑤神经源性高血压。⑥药物性高血压。⑦围术期高血压。

高血压急症与高血压次急症均可合并慢性器官损害,区别两者的唯一标准是有无新近发生的或急性进行性的严重靶器官损害。高血压水平的绝对值不构成区别两者的标准,因为血压水平的高低与是否伴有急性靶器官损害或损害的程度并非成正比。

高血压急症是一种严重危及生命的临床综合征,特别强调了心、脑、肾等重要靶器官的功能问题。在高血压急症治疗中,“降低血压”只是一种治疗手段,“保护或恢复靶器官的功能”才是“目的”。近年来,随着对自动调节阈的理解,临床上得以能够正确的把握高血压急症的降压幅度。尽管血压有显著的可变性,但血压的自动调节功能可维持流向生命器官(脑、心、肾)的血流在很小的范围内波动。例如,当平均动脉压(MAP)低到60mmHg或高达120mmHg,脑血流量可被调节在正常压力范围内。然而,在慢性高血压患者,其自动调节的下限可以上升到MAP
的100~120mmHg,高限可达150~160mmHg,这个范围称为自动调节阈。达到自动调节阈低限时发生低灌注,达到高限则发生高灌注。与慢性高血压类似,老年患者和伴有脑血管疾病的患者自动调节功能也受到损害,其自动调节阈的平均低限大约比休息时MAP低20\%~25\%。对高血压急症患者最初的治疗可以将MAP谨慎地下降20\%的建议就是由此而来。

\subsection{病因与发病机制}

\subsubsection{病因}

高血压危象的促发因素很多,最常见的是在长期原发性高血压患者中血压突然升高,约占40\%~70\%。另外,25\%~55\%的高血压危象患者有可查明原因的继发性高血压,肾实质病变占其中的80\%。高血压危象的继发性原因主要包括:①肾实质病变:原发性肾小球肾炎、慢性肾盂肾炎、间质性肾炎。②涉及肾脏的全身系统疾病:系统性红斑狼疮、系统性硬皮病、血管炎。③肾血管病:结节性多动脉炎、肾动脉粥样硬化。④内分泌疾病:嗜铬细胞瘤、库欣综合征、原发性醛固酮增多症。⑤药品:可卡因、苯异丙胺、环孢素、可乐定撤除、苯环利定。⑥主动脉狭窄。⑦子痫和子痫前期。

\subsubsection{发病机制}

各种高血压危象的发病机制不尽相同,某些机制尚未完全阐明,但与下列因素有关。

\paragraph{交感神经张力亢进和缩血管活性物质增加}

在各种应激因素作用下,交感神经张力、血液中血管收缩活性物质(如肾素、血管紧张素Ⅱ等)大量增加,诱发短期内血压急剧升高。

\paragraph{局部或全身小动脉痉挛}

①脑及脑细小动脉持久性或强烈痉挛导致脑血管继之发生“强迫性”扩张,结果脑血管过度灌注,毛细血管通透性增加,引起脑水肿和颅内高压,诱发高血压脑病。②冠状动脉持久性或强烈痉挛导致心肌明显缺血、损伤甚至坏死等,诱发急性冠脉综合征。③肾动脉持久性或强烈收缩导致肾脏缺血性改变、肾小球内高压力等,诱发肾功能衰竭。④视网膜动脉持久性或强烈痉挛导致视网膜内层组织变性坏死和血-视网膜屏障破裂,诱发视网膜出血、渗出和视神经乳头水肿。⑤全身小动脉痉挛导致压力性多尿和循环血容量减少,反射性引起缩血管活性物质进一步增加,形成病理性恶性循环,加剧血管内膜损伤和血小板聚集,最终诱发心、脑、肾等重要脏器缺血和高血压危象。

\paragraph{脑动脉粥样硬化}

高血压促成脑动脉粥样硬化后斑块或血栓破碎脱落易形成栓子,微血管瘤形成后易于破裂,斑块和(或)表面血栓形成增大,最终致动脉闭塞。在血压增高、血流改变、颈椎压迫、心律不齐等因素作用下易发生急性脑血管病。

\paragraph{其他}

引起高血压危象的其他相关因素尚有神经反射异常(如神经源性高血压危象等)、内分泌激素水平异常(如嗜铬细胞瘤高血压危象等)、心血管受体功能异常(如降压药物骤停综合征等)、细胞膜离子转移功能异常(如烧伤后高血压危象等)、肾素-血管紧张素-醛固酮系统的过度激活(如高血压伴急性肺水肿等)。此外,内源性生物活性肽、血浆敏感因子(如甲状旁腺高血压因子、红细胞高血压因子等)、胰岛素抵抗、一氧化氮合成和释放不足、原癌基因表达增加以及遗传性升压因子等均在引起高血压急症中起一定作用。

\subsection{诊断}

接诊严重的高血压患者后,病史询问和体格检查应简单而有重点,目的是尽快鉴别高血压急症和次急症。应询问高血压病史、用药情况、有无其他心脑血管疾病或肾脏疾病史等。除测量血压外,应仔细检查心血管系统、眼底和神经系统,了解靶器官损害程度,评估有无继发性高血压。如果怀疑继发性高血压,应在治疗开始前留取血和尿液标本。实验室检查至少应包括心电图和尿常规。高血压急症的临床特征见表\ref{tab42-1}。

\begin{table}[htbp]
\centering
\caption{高血压急症患者的临床特征}
\label{tab42-1}
\includegraphics[width=3.29167in,height=1.625in]{./images/Image00154.jpg}
\end{table}

高血压急症患者通常血压很高,收缩压> 210mmHg或舒张压>
140mmHg。但是,鉴别诊断的关键因素通常是靶器官损害,而不是血压水平。妊娠妇女或既往血压正常者血压突然增高、伴有急性靶器官损害时,即使血压测量值没有达到上述水平,仍应视为高血压急症。

单纯血压很高、没有症状也没有靶器官急性或进行性损害证据的慢性高血压患者(其中可能有一部分为假性高血压患者),以及因为疼痛、紧张、焦虑等因素导致血压进一步增高的慢性高血压患者,通常不需要按高血压急症处理。

\subsection{治疗}

\subsubsection{治疗原则}

治疗的选择应根据对患者的综合评价诊断而定,靶器官的损害程度决定血压下降到何种安全水平以限制靶器官的损害。治疗评价依据见表\ref{tab42-2}。

高血压急症应住院治疗,重症收入CCU(ICU)病房。酌情使用有效的镇静药以消除患者恐惧心理。在严密监测血压、尿量和生命体征的情况下,视临床情况的不同,应用短效静脉降压药物。定期采血监测内环境情况,注意水、电解质、酸碱平衡情况,肝、肾功能,有无糖尿病,心肌酶是否增高等,计算单位时间的出入量。降压过程中应严密观察靶器官功能状况,如神经系统的症状和体征,胸痛是否加重等。勤测血压(每隔15~30分钟),如仍然高于180/120mmHg,应同时口服降压药物。

降压目标不是使血压正常,而是渐进地将血压调控至不太高的水平,最大程度地防止或减轻心、脑、肾等靶器官损害。在正常情况下,尽管血压经常波动(MAP
60~150mmHg),但心、脑、肾的动脉血流能够保持相对恒定。慢性血压升高时,这种自动调节作用仍然存在。但调节范围上移,血压对血流的曲线右移,以便耐受较高水平的血压,维持各脏器的血流。当血压上升超过自动调节阈值之上时,便发生器官损伤。阈值的调节对治疗非常有用。突然的血压下降,会导致器官灌注不足。在高血压危象中,这种突然的血压下降,在病理上会导致脑水肿以及中小动脉的急慢性炎症甚至坏死。患者会出现急性肾衰、心肌缺血及脑血管事件,对患者有害无益。对正常血压者和无并发症的高血压患者的脑血流的研究显示,脑血流自动调节的下限大约比休息时MAP低20\%~25\%。因此初始阶段(几分钟到两个小时内)MAP的降低幅度不应超过治疗前水平的20\%~25\%。假如患者能很好耐受,且病情稳定,超过24小时后再把血压降至正常。无明显靶器官损害患者应在24~48小时内将血压降至目标值。

\begin{table}[htbp]
\centering
\caption{治疗评价的依据}
\label{tab42-2}
\includegraphics[width=6.64583in,height=1.91667in]{./images/Image00155.jpg}
\end{table}

上述原则不适用于急性缺血性脑卒中的患者。因为这些患者的颅内压增高、小动脉收缩、脑血流量减少,此时机体需要依靠MAP的增高来维持脑的血液灌注。此时若进行降压治疗、特别是降压过度时,可导致脑灌注不足,甚至引起脑梗死。因此一般不主张对急性脑卒中患者采用积极的降压治疗。关于急性出血性脑卒中合并严重高血压的治疗方案目前仍有争论,一般认为MAP
> 130mmHg时应该使用经静脉降压药物。

高血压次急症不伴有严重的靶器官损害,不需要特别的处理,可以口服抗高血压药物而不需要住院治疗。

高血压急症在临床上表现形式不同,治疗的药物和处理方法也有差异。高血压急症伴有心肌缺血、心肌梗死、肺水肿时,如果血压持续升高,可导致左室壁张力增加,左室舒张末容积增加,射血分数降低,同时心肌耗氧量增加。此时宜选用迅速降低血压,血压的目标值是使其收缩压下降10\%~15\%。此外,开通病变血管也是非常重要的。

高血压急症伴有神经系统急症是最难处理的。高血压脑病是排除性诊断,需排除出血性和缺血性脑卒中及蛛网膜下腔出血。以上各种情况的处理是不同的。①脑出血:在脑出血急性期,如果收缩压大于210mmHg,舒张压大于110mmHg时方可考虑应用降压药物,但要避免血压下降幅度过大,一般降低幅度为用药前血压20\%~30\%为宜,同时应脱水治疗降低颅内压。②缺血性脑卒中:一般当舒张压大于130mmHg时,方可小心将血压降至110mmHg。③蛛网膜下腔出血:首选降压药物以不影响患者意识和脑血流灌注为原则,蛛网膜下腔出血首期降压目标值在25\%以内,对于平时血压正常的患者维持收缩压在130~160mmHg之间。④高血压脑病:高血压脑病的血压值要比急性缺血性脑卒中要低。高血压脑病MAP在2~3小时内降低20\%~30\%。

高血压急症伴肾脏损害是非常常见的。有的患者尽管血压很低,但伴随着血压的升高,肾脏的损害也存在。尿中出现蛋白、红细胞、血尿素氮和肌酐升高,都具有诊断意义。高血压急症伴肾脏损害要在1~12小时内使MAP下降10\%~25\%,MAP在第1小时下降10\%,紧接2小时下降10\%~15\%。

高血压急症伴主动脉夹层有特殊处理。高血压是急性主动脉夹层形成的重要易患因素,因而降压治疗必须迅速实施,以防止主动脉夹层的进一步扩展。治疗时,在保证脏器足够灌注的前提下,应使血压维持在尽可能低的水平。首选静脉给药的β阻滞剂如艾司洛尔或美托洛尔,它可以减少夹层的发展。高血压伴主动脉夹层首期降压目标值将血压降至理想水平,在30分钟内使收缩压低于120mmHg。药物治疗只是暂时的,最终需要外科手术。

儿茶酚胺诱发的高血压危象,此症的特点是β肾上腺素张力突然升高。这类患者通常由于突然撤掉抗高血压药物造成。由于儿茶酚胺升高导致的高血压急症,最好用α受体阻滞剂,如酚妥拉明,其次要加用β受体阻滞剂。

怀孕期间的高血压急症,处理起来要非常谨慎和小心。硫酸镁、甲基多巴及肼屈嗪是比较好的选择。妊娠高血压综合征伴子痫前期使收缩压低于90mmHg。

围术期高血压处理的关键是要判断产生血压高的原因并去除诱因,去除诱因后血压仍高者,要降压处理。围术期的高血压的原因,是由于原发性高血压、焦虑和紧张、手术刺激、气管导管拔管、创口的疼痛等造成。手术前,降压药物应维持到手术前1天或手术日晨,长效制剂降压药宜改成短效制剂,以便麻醉管理。对于术前血压高的患者,麻醉前含服硝酸甘油、硝苯地平,也可用艾司洛尔300~500μg/kg静注,随后25~100μg/(kg•min)静点,或者用乌拉地尔(压宁定)首剂12.5~25mg,3~5分钟,随后5~40mg/h静点。拔管前用压宁定或艾司洛尔,剂量同前。

\subsubsection{降压药物的选择}

\hypertarget{text00108.htmlux5cux23CHP4-7-3-2-1}{}
(一) 急诊用药标准的考量

\paragraph{起效时间}

高血压急症急诊用药考虑的第一个因素是起效快。在常用降压药中,硝普钠起效最快,静注后“立即”起效;艾司洛尔和酚妥拉明起效时间为1~2分钟;硝酸甘油在5分钟内起效;拉贝洛尔和尼卡地平在5~10分钟起效;乌拉地尔稍慢,15分钟起效。从起效时间角度来衡量,除硝普钠起效最快,乌拉地尔起效稍慢外,上述所有药物都应符合高血压急症紧急降压的要求。

\paragraph{持续时间}

高血压急症急诊用药考虑的第二个因素是药物持续时间。其中持续时间较短的有:硝普钠(1~2分钟)、酚妥拉明(3~10分钟)、硝酸甘油(5~10分钟);居中的有:艾司洛尔(10~20分钟)、尼卡地平(1~4小时);较长的有:乌拉地尔(2~8小时)、拉贝洛尔(4~8小时)。药物持续时间主要与其半衰期有关。如药物持续时间很短,降压作用的平稳性就会很差,血压容易大起大落,需密切观察,随时调整药物的剂量和用药速度。临床上使用这类药物,比较麻烦,需密切监护,不太适合于急诊科使用。如药物持续时间较长,虽然降压作用的平稳性很好,但是一旦用药剂量过大,血压就会持续在较低水平,药物减量后需较长时间的等待,才能逐渐恢复,临床使用也不方便。故药物持续时间居中的降压药物,艾司洛尔和尼卡地平,有一定的优势。

\paragraph{常见且严重的不良反应}

药物的常见且严重的不良反应,主要决定于药物本身的特性。如β受体阻断药物艾司洛尔和拉贝洛尔,通过阻断心脏β受体,具有抑制心肌收缩力和减慢心率的作用。如果β\textsubscript{1}
受体阻断的选择性不强,还会有β\textsubscript{2}
受体阻断作用,使支气管收缩。钙离子拮抗剂中地尔硫{}
,也具有抑制心肌收缩力和减慢心率的作用。这些几乎是必然发生,和可能会很严重的不良反应,是临床医生选择药物时,常常不能容忍的问题,故只适用于高血压急症治疗中的一些特殊情况。

\hypertarget{text00108.htmlux5cux23CHP4-7-3-2-2}{}
(二) 高血压急症静脉降压药物

根据作用机制 ,经静脉降压药物主要分成以下几类:表\ref{tab42-3}。

\paragraph{血管扩张剂}

\hypertarget{text00108.htmlux5cux23CHP4-7-3-2-2-1-1}{}
(1) 硝普钠(sodium nitroprusside):

是一种起效快、持续时间短的强效静脉用降压药。静脉滴注数秒内起效,作用持续仅1~2分钟,血浆半衰期3~4分钟,停止注射后血压在1~10分钟内迅速回到治疗前水平。起始剂量0.25μg/(kg•min),其后每隔5分钟增加一定剂量,直至达到血压目标值。可用剂量0.25~10μg/(kg•min)。硝普钠应慎用或禁用于下列情况:①高血压脑病、脑出血、蛛网膜下腔出血。因该药可通过血-脑脊液屏障使颅内压进一步增高,影响脑血流灌注,加剧上述病情,故有颅内高压者一般不予应用。②急进型恶性高血压、高血压伴急性肾功能衰竭、肾移植性高血压、高血压急症伴严重肝功能损害等,因该药在体内与巯基结合后分解为氰化物与一氧化氮,氰化物被肝脏代谢为硫氰酸盐,全部需经肾脏排出。一般肾功能正常者硫氰酸盐排泄时间约为3天。故肝、肾功能不良患者易发生氰化物或硫氰酸盐中毒,产生呼吸困难、肌痉挛、精神变态、癫痫发作、昏迷、甚至呼吸停止等严重反应。③甲状腺功能减退和孕妇:因硫氰酸盐可抑制甲状腺对碘的摄取,加重甲状腺功能减退,且可通过胎盘诱发胎儿硫氰酸盐中毒和酸中毒。

\begin{table}[htbp]
\centering
\caption{治疗高血压急症的经静脉降压药物}
\label{tab42-3}
\includegraphics[width=6.69792in,height=4.47917in]{./images/Image00157.jpg}
\end{table}

过去认为硝普钠是高血压急症伴急性肺水肿、严重心功能衰竭、主动脉夹层的首选药物之一。其长期大剂量使用或患者存在肝、肾功能不全时,易发生氰化物中毒。故通常在初步控制病情后,应迅速改用其他药物。目前多数学者认为,由于硝普钠的严重副作用,它只用于无法获取其他降压药物时,和主动脉夹层等特殊情况,且患者的肝、肾功能正常的情况下;疗程尽可能短,输注速度应控制在2µg/(kg•min)以内,如大于4~10µg/(kg•min),必须同时给予解毒药物硫代硫酸盐。

\hypertarget{text00108.htmlux5cux23CHP4-7-3-2-2-1-2}{}
(2) 硝酸甘油(nitroglycerin):

能扩张静脉、动脉和侧支冠状动脉,特别适用于伴有中度血压增高的急性冠状动脉综合征或心肌缺血的患者。硝酸甘油起效快、消失也快,应注意监测静脉滴注的速率。该药小剂量时主要扩张静脉血管、较大剂量才能扩张小动脉,故可能需要每3~5分钟调快滴速,直到取得预期的降压效果。硝酸甘油静脉滴注2~5分钟起效,停止用药作用持续时间5~10分钟,可用剂量5~100µg/min。副作用有头痛、恶心呕吐、心动过速等。由于硝酸甘油是有效的扩静脉药物,只有在大剂量时才有扩动脉作用,能引起低血压和反射性心动过速,在脑、肾灌注存在损害时,静脉使用硝酸甘油可能有害。

\hypertarget{text00108.htmlux5cux23CHP4-7-3-2-2-1-3}{}
(3) 肼屈嗪(hydralazine):

通过直接舒张血管平滑肌降低血压。静脉注射每次10~20mg,10~15分钟起效,肌肉注射每次10~50mg,20~30分钟起效,血压持续下降可达12小时。虽然肼屈嗪循环半衰期只有3小时,但其效果减半的时间却达到了100小时,可能原因是肼屈嗪与肌性动脉壁长久结合。

由于肼屈嗪降压的效果持续和难于预测,不能控制其降压的强度,同时其会反射性引起每搏输出量和心率的增加,诱发或加重心肌缺血,应尽量避免在高血压急症时使用,仅用于子痫和惊厥患者。

\paragraph{钙拮抗剂}

\hypertarget{text00108.htmlux5cux23CHP4-7-3-2-2-2-1}{}
(1) 尼卡地平(nicardipine):

二氢吡啶类钙拮抗剂,通过抑制钙离子内流而发挥血管扩张作用。盐酸尼卡地平对血管平滑肌的作用比对心肌的作用强3万倍,其血管选择性明显高于其他钙拮抗剂。其扩张外周血管作用与硝苯地平相近,对冠脉的扩张比对外周血管更强。心脏抑制作用是硝苯地平的1/10,对心肌及传导系统无抑制作用。本品使心脏射血分数及心排血量增多,而左室舒张末压改变不多。能降低心肌耗氧量及总外周阻力,也可增加冠脉侧支循环,使冠状血流增加。5~15mg/h,缓慢静滴,直到出现预期反应。每5分钟可增加剂量2.5mg/h,最大剂量15mg/h。健康男性成年人,按0.01~0.02mg/kg盐酸尼卡地平静脉给予后,消除半衰期为50~63分钟。

尼卡地平与其他多数降压药物不同,在降低血压的同时,能增加重要器官的血流量,这是该药的重要特点之一。研究发现,尼卡地平可引起剂量依赖性的动脉血流量增加,程度为椎动脉>冠状动脉>股动脉>肾动脉。这是由于尼卡地平对椎-基底动脉及冠状动脉的选择性最高,这一特点不同于其他钙离子拮抗剂(如氨氯地平、非洛地平等就主要作用于周围血管),也有别于其他大多数降压药物。尼卡地平在降压的同时,可以改善脑、心、肾等重要器官的血流量,有效保护重要靶器官;故从保护靶器官角度考虑,尼卡地平可能是高血压急症治疗最佳的选择。

\hypertarget{text00108.htmlux5cux23CHP4-7-3-2-2-2-2}{}
(2) 地尔硫{} (diltiazem):

非二氢吡啶类钙拮抗剂,通过抑制钙离子向末梢血管、冠脉血管平滑肌细胞及房室结细胞内流,而达到扩张血管及延长房室结传导的作用。犬大剂量静脉注射盐酸地尔硫{}
可出现明显的心动过缓和房室传导改变。在犬和大鼠的亚急性和慢性毒性研究中,大剂量口服盐酸地尔硫{}
可引起肝脏损害。用法:10mg/次,静注或5~15µg(/kg•min)静滴。禁忌证主要为:①严重低血压或心源性休克患者。②Ⅱ度和Ⅲ度房室传导阻滞或病窦综合征(持续窦性心动过缓、窦性停搏和窦房阻滞等)。③严重充血性心衰患者。④严重心肌病患者。⑤对药物中任一成分过敏者。⑥妊娠或可能妊娠的妇女。⑦静脉给予盐酸地尔硫{}
和静脉给予β阻滞剂应避免在同时或相近的时间内给予(几小时内)。⑧室性心动过速患者,宽QRS心动过速患者(QRS≥0.12秒)使用钙通道阻滞剂可能会出现血流动力学恶化和室颤。静脉注射地尔硫{}
前,明确宽QRS波为室上性或室性是非常重要的。

\paragraph{肾上腺素受体阻滞剂}

\hypertarget{text00108.htmlux5cux23CHP4-7-3-2-2-3-1}{}
(1) 酚妥拉明(phentolamine):

是一种非选择性α受体阻滞剂,适用于伴有血液中儿茶酚胺过量的高血压急症,如嗜铬细胞瘤危象。静脉注射后1~2分钟内起效,作用持续10~30分钟。用法:每次5~15mg,静脉注射。但因其引起反射性心动过速,容易诱发心绞痛和心肌梗死,故禁用于急性冠状动脉综合征患者。副作用有心动过速、直立性低血压、潮红、鼻塞、恶心呕吐等。

\hypertarget{text00108.htmlux5cux23CHP4-7-3-2-2-3-2}{}
(2) 乌拉地尔(urapidil):

又名压宁定,对外周血管α\textsubscript{1}
受体有阻断作用,对中枢5-羟色胺受体有激动作用,因而有良好的周围血管扩张作用和降低交感神经张力作用。乌拉地尔扩张静脉的作用大于动脉,并能降低肾血管阻力,对心率无明显影响。其降压平稳,效果显著,有减轻心脏负荷、降低心肌耗氧量、增加心脏搏出量、抗心律失常、降低肺动脉高压和增加肾血流量等优点。目前特别适用于高血压急症伴急性左心衰竭、急性冠脉综合征、主动脉夹层、高血压脑病、急进型恶性高血压、妊娠高血压综合征伴子痫前期等患者。肾功能不全可以使用。缓慢静推10~50mg,监测血压变化,降压效果通常在5分钟内显示;若在10分钟内效果不够满意,可重复静推,最大剂量不超过75mg。静推后可持续静滴100~400μg/min,或者2~8μg/(kg•min)持续泵入。

在使用中,应注意:①血压骤然下降可能引起心动过缓甚至心脏停搏,这可能是存在抗高血压药物“首剂效应”的结果。②静脉使用乌拉地尔,治疗期限一般不超过7天,这可能是存在抗高血压药物“继发性耐受”的结果。③逾量可致低血压,主要机制可能为静脉扩张,回心血量减少;治疗可抬高下肢及增加血容量,必要时加升压药。④静脉注射乌拉地尔后,在体内分布成二室模型,血浆清除半衰期2.7(1.8~3.9)小时,蛋白结合率80\%。50\%~70\%的乌拉地尔通过肾脏排泄,其余由胆汁排出。故老年人及肝功能受损者可增强本品作用,应予注意。⑤乌拉地尔对大鼠具有中度的镇静作用,这一作用亦不受α\textsubscript{2}
受体阻滞剂的影响。故开车或操纵机器者应谨慎,可能影响其驾驶或操纵能力。

\hypertarget{text00108.htmlux5cux23CHP4-7-3-2-2-3-3}{}
(3) 拉贝洛尔(labetalol):

是联合的α和β肾上腺素能受体拮抗剂,静脉用药α和β阻滞的比例为1∶7,多数在肝脏代谢,代谢产物无活性。与纯粹的α受体阻滞剂不同的是,拉贝洛尔不降低心脏排血量,心率多保持不变或轻微下降。拉贝洛尔降低外周血管阻力,不降低外周血管血流量,脑、肾和冠状动脉血流保持不变。已经证明拉贝洛尔在治疗高血压危象和急性心肌梗死方面有效。静脉注射2~5分钟起效,5~15分钟达高峰,作用持续2~6小时。用法:首次静脉注射20mg,接着每10分钟20~80mg静脉注射,或者从2mg/min开始静脉滴注,最大累积剂量24小时内300mg,达到血压目标值后改口服。副作用有恶心、乏力,支气管痉挛,心动过缓,直立性低血压等。可见其不良反应中,还是存在β受体阻滞作用。

\hypertarget{text00108.htmlux5cux23CHP4-7-3-2-2-3-4}{}
(4) 艾司洛尔(esmolol):

是心脏选择性的短效β受体阻滞剂,起效快,500μg/kg静脉推注,在1~5分钟可迅速降低血压,单次注射作用持续时间15~30分钟。25~100μg/
(kg•min)持续静脉滴注,最大剂量可达300μg/(kg•min)。副作用有乏力、低血压、心动过缓、多汗等。故其应用时,必须评价β受体阻滞后,患者有可能出现的反应。Ⅰ度房室传导阻滞、充血性心力衰竭和哮喘慎用。

\paragraph{血管紧张素转换酶抑制剂}

依那普利拉(enalaprilat)是目前唯一可以注射给药的ACEI类药物。用法:每次1.25mg,5分钟内静脉注射,每6小时1次;每12~24小时增加1.25mg,最大剂量每6小时5mg。静脉注射15分钟内起效,作用持续12~24小时。降压效果与血浆肾素和血管紧张素浓度呈正相关。对于有慢性心力衰竭的高血压急症患者效果较好。副作用有低血压、肾功能衰竭(双侧肾动脉狭窄患者)。肾动脉狭窄和孕妇禁用。由于存在“首剂效应”,可能会出现严重低血压,尽可能不作高血压急症时的首选。

\paragraph{其他降压药}

非诺多泮(fenoldopam)是一种选择性外周多巴胺1受体拮抗剂,除扩张血管外,能增加肾血流、作用于肾近曲小管和远曲小管,促进尿钠排泄和改善肌酐清除率,故特别适用于合并肾功能损害的高血压急症患者。一些研究提示,非诺多泮的降压疗效与硝普钠相似,0.1~0.3μg/(kg•min)持续静脉滴注,5分钟快速起效,最大剂量1.6μg/(kg•min),撤药30分钟后作用消失。可能出现低血压、面部潮红、反射性心动过速、心电图异常、头痛、头晕、恶心、呕吐、眼内压增高、低钾血症。低起始剂量{[}0.03~0.1μg/(kg•min){]}可能避免反射性心动过速。给药期间需监测电解质。青光眼患者慎用。

\begin{table}[htbp]
\centering
\caption{治疗高血压(次)急症的口服降压药物}
\label{tab42-4}
\includegraphics[width=6.64583in,height=1.70833in]{./images/Image00163.jpg}
\end{table}

\hypertarget{text00108.htmlux5cux23CHP4-7-3-2-3}{}
(三) 高血压(次)急症口服降压药物

用于高血压(次)急症的口服降压药物主要有以下几种:表\ref{tab42-4}。

\paragraph{卡托普利(captopril)}

是口服血管紧张素转换酶抑制剂的代表药物,它也可舌下含服。15分钟起效,作用持续4~6小时。初次使用时极少引起急剧低血压效应,是治疗高血压次急症的最安全口服降压药。同时给予袢利尿剂如呋塞米可增强卡托普利的效果。常用剂量为12.5~50mg/次,每日2~3次。其他常用的口服ACEI还有:依那普利、蒙诺普利、苯那普利、培哚普利。

\paragraph{可乐定(clonidine)}

是中枢α肾上腺素能激动剂,口服后30~60分钟起效,2~4小时达到最大效应。单一剂量0.2mg疗效与0.1mg/h相当。可乐定的最常见副作用是倦睡(发生率高达45\%),可能会影响对患者精神状态的评估。

\paragraph{拉贝洛尔(labetalol)}

是联合的α和β肾上腺素能受体拮抗剂,口服200~400mg,2小时起效。与其他的β受体阻滞剂一样,拉贝洛尔可引起心脏传导阻滞,加重支气管痉挛。房室传导阻滞、心动过缓、慢性充血性心衰慎用。

\paragraph{哌唑嗪(prazosin)}

是α肾上腺素能阻滞剂,可用于嗜铬细胞瘤患者的早期处理。副作用包括晕厥(首剂时易发生)、心悸、心动过速和立位低血压。

\paragraph{呋塞米(furosemide)}

是袢利尿剂,每日40~120mg,分1~3次口服,最大剂量每日160mg。迅速降低心脏前负荷,改善心衰症状,减轻肺水肿和脑水肿,特别适合于心、肾功能不全和高血压脑病的患者。作用快而强,超量应用时,降压作用不加强,不良反应反而加重。可能出现水、电解质紊乱,以及与此有关的口渴、乏力、肌肉酸痛、心律失常。少尿或无尿患者应用最大剂量后24小时仍无效时应停药。

\paragraph{硝苯地平(nifedipine)}

是短效制剂,可口服、舌下含服或咀嚼,5~10分钟起效,持续3~5小时,常用剂量为每次5~10mg,每日3次。但因其可能引起急剧且不可控制的低血压效应,及反射性心动过速,增加心肌氧耗,恶化心肌缺血而可能危及生命。这种严重的副作用是不可预测的,故目前认为应慎用于高血压危象。
\protect\hypertarget{text00109.html}{}{}

\hypertarget{text00109.htmlux5cux23CHP4-7-4}{}
参 考 文 献

1. 沈潞华.高血压急症.中国循环杂志,2009,24(3):231-233

2.
郭树彬.高血压急症的处理与靶器官保护策略.中国急救医学,2009,29(1):76-79

3. 孟庆义.急诊临床思维.北京:科学技术文献出版社,2010

4. 田国祥
,孟庆义.高血压急症治疗-尼卡地平的兴起.中国循证心血管医学杂志,2010,2(1):6-8

5. 孟庆义
.高血压急症治疗应关注肺循环.中国急救医学,2009,29(2):145-147

6. Pergolini MS. The management of hypertensive crises:a clinical
review. Clin Ter,2009,160(2):151-157

7. Rodriguez MA,Kumar SK,De Caro M. Hypertensive crisis. Curr Drug
Targets,2009,10(8):788-798

\protect\hypertarget{text00110.html}{}{}

\chapter{垂 体 危 象}

腺垂体功能减退症是指腺垂体激素分泌减少,可以是单种激素减少如生长激素(GH)缺乏或多种促激素同时缺乏。由于腺垂体分泌细胞是在下丘脑各种激素(因子)直接影响之下,腺垂体功能减退可原发于垂体病变,或继发于下丘脑病变,表现为甲状腺、肾上腺、性腺等靶腺功能减退和(或)鞍区占位性病变。临床症状变化较大,但补充所缺乏的激素治疗后症状可迅速缓解。

垂体危象是在腺垂体功能减退的基础上,血循环中肾上腺皮质激素和甲状腺激素缺乏,对外界环境变化的适应能力下降,机体抵抗力下降,在各种应激情况下,如感染、腹泻、呕吐、失水、饥饿、受寒、中暑、手术、外伤、麻醉、酗酒及使用各种镇静安眠药、降糖药等,导致患者病情发生急剧变化,可表现为高热(>
40℃),低温(< 35℃),低血糖,循环衰竭,水中毒等的一种危及生命状态。

\subsection{病因与发病机制}

垂体前叶分泌六种激素,包括生长激素(GH)、泌乳素(PRL)、卵泡刺激素(FSH)、黄体生成素(LH)、促皮质激素(ACTH)和促甲状腺素(TSH),主要管辖三个靶腺及其相应靶组织:性腺、肾上腺皮质和甲状腺。

腺垂体功能减退症是临床上常见的内分泌疾病,系因腺垂体激素分泌功能部分或全部丧失。常见病因为以下几方面:

\paragraph{垂体及附近肿瘤压迫浸润}

垂体肿瘤,鞍上及鞍旁肿瘤,各种转移性癌,淋巴瘤,白血病,组织细胞增多症等均可浸润下丘脑和垂体,引起垂体前叶功能不全。

\paragraph{产后大出血所致垂体前叶破坏及萎缩}

产妇在分娩过程中大出血,可导致垂体前叶坏死,称为希恩(Sheehan)综合征。一般认为,随着妊娠,垂体呈生理性肥大,大出血时血管痉挛,血栓形成,或产后败血症引起垂体栓塞或DIC,导致腺垂体急性坏死。神经垂体的血流供应不依赖门脉系统,故产后出血一般不伴有神经垂体坏死。

\paragraph{感染和炎症}

各种病毒、结核、梅毒、真菌感染,化脓性脑膜炎,脑膜脑炎,流行性出血热,均可引起下丘脑-垂体损伤而导致功能减退。

\paragraph{自身免疫性疾病}

常见的是自身免疫性垂体炎,好发于妊娠及产后妇女,男性少见。自身免疫性垂体炎临床多表现为垂体功能减低,鞍区肿物或垂体柄增粗,有时伴高催乳素血症,金标准是病理诊断。

\paragraph{手术 、创伤和放射损伤}

垂体瘤摘除、放疗,或鼻咽癌等颅底及颈部放疗后均可引起本症。颅底骨折、垂体柄挫伤可阻断神经与门脉系统的联系而导致腺垂体及神经垂体功能减退。

\paragraph{其他}

空泡蝶鞍,动脉硬化可引起垂体梗死,颞动脉炎,海绵窦血栓常导致垂体缺血,糖尿病性血管病变引起缺血坏死等。长期大剂量糖皮质激素治疗也可抑制相应垂体激素的分泌,突然停药可出现单一性垂体激素分泌不足的表现。

上述多种病因均可引起下丘脑和(或)垂体功能减退,若为中度或重度垂体功能减退症,未经系统和正规激素补充治疗,或终止治疗,再遇感染、外伤、手术等应激状态或处理不当,常可诱发多种代谢紊乱和器官功能失调,出现精神失常、意识模糊、神志不清、谵妄甚至昏迷诱发垂体危象。

\subsection{诊断}

\subsubsection{临床表现特点}

\hypertarget{text00110.htmlux5cux23CHP4-8-2-1-1}{}
(一) 腺垂体功能减退

腺垂体功能减退症的临床表现与患者发病的年龄、性别、受累激素种类、分泌受损程度及原发病的病理性质有关。通常GH和FSH、LH的缺乏发生最早,其次是ACTH
及TSH缺乏。当肾上腺皮质激素和(或)甲状腺激素缺乏时,机体应激能力下降。

一般情况下,垂体破坏50\%以上才出现临床症状,破坏75\%出现较明显症状,破坏95\%出现严重症状。其中以LH、FSH和PRL受累最早最严重,其次分别为TSH、ACTH。

1.ACTH缺乏引起肾上腺皮质功能不全,表现为虚弱无力,肌肉松弛,肤色浅淡,食欲下降,体重减轻,血压下降或发生直立性低血压,易发生低血糖症。

2.TSH缺乏引起继发性甲减,患者水肿,表情淡漠,畏寒,皮肤干燥,心动过缓,体温低。

3.LH、FSH缺乏引起性腺功能减退,希恩综合征表现为产后无乳,闭经,腋毛阴毛脱落,性欲减退,乳房萎缩;男性表现为阳痿,睾丸萎缩,性欲低下等。

4.原发疾病表现
,垂体及附近肿瘤可有头痛、呕吐、视力减退、视野缺损等症状。

\hypertarget{text00110.htmlux5cux23CHP4-8-2-1-2}{}
(二) 垂体危象

在全垂体功能减退症基础上
,各种应激如感染、脓毒症、腹泻、呕吐、失水、饥饿、受寒、AMI、脑卒中、手术、外伤、麻醉及使用各种镇静安眠药、降糖药等均可诱发垂体危象。垂体危象临床主要表现为以下几种类型:

\paragraph{高热型(> 40℃)}

由于体内缺乏肾上腺皮质激素,患者抵抗力降低,容易感染,感染后发生高热。

\paragraph{低温型(< 35℃)}

由于患者甲状腺激素不足,全身代谢低下,产热不足,可低于35℃,昏迷逐渐发生,皮肤苍白、干冷,脉慢而细。

\paragraph{低血糖型}

由于患者缺乏肾上腺皮质激素和甲状腺激素,肝糖原储备不足,患者不耐受饥饿;同时患者对胰岛素敏感性增加,因而容易发生低血糖甚至昏迷。

\paragraph{低血压 、循环虚脱型}

糖皮质激素不足,容易发生低钠血症;胃肠道功能紊乱、手术、感染等,失钠,致血容量减低,容易发生周围循环衰竭和休克,患者表现为食欲不振、头痛、恶心、呕吐、软弱无力,严重者精神错乱、昏迷。

\paragraph{水中毒型}

当患者饮水过多或做水负荷实验可引起血容量增加,血液稀释,原有低钠血症时更容易发生,患者表现为全身无力、头痛、恶心、呕吐、意识模糊、嗜睡、抽搐甚至昏迷。

各种类型可伴有相应的症状,突出表现为消化系统、循环系统和神经精神方面的症状,如高热、循环衰竭、休克、恶心、呕吐、头痛、神志不清、谵妄、抽搐、昏迷等。①消化系统:可在原有的厌食、腹胀、腹泻的基础上,发展为恶心、呕吐,甚至不能进食。②循环系统:低钠血症,血容量降低,表现为脉搏细弱,皮肤干冷,心率过快或过缓,血压过低,直立性低血压,虚脱,甚至休克。③精神神经系统:患者可出现精神萎靡、烦躁不安、嗜睡、神志不清、谵妄或昏迷,低血糖患者可表现为无力、出汗、视物不清、复视或昏迷。

对于既往病史不清的患者,若出现严重的循环衰竭、低血糖、淡漠、昏迷、难以纠正的低钠血症、高热以及呼吸衰竭,应当考虑垂体危象。

\subsubsection{实验室检查}

\paragraph{血常规及血生化}

严重的低钠血症最为常见,血钠通常低于120mmol/L。而合并甲状腺功能减退的患者可出现贫血,表现为红系或三系均减低。患者空腹血糖降低,二氧化碳结合力降低。伴有严重感染的患者白细胞总数和中性粒细胞数明显升高。

\paragraph{靶腺激素水平减低}

肾上腺皮质激素(血皮质醇和尿游离皮质醇)及其代谢产物(17-羟类固醇,17-酮类固醇),甲状腺激素(T\textsubscript{3}
、T\textsubscript{4} 、FT\textsubscript{3} 、FT\textsubscript{4}
)及性腺激素(雌二醇、睾酮)均降低。

\paragraph{垂体激素减少}

生长激素(GH)、促肾上腺皮质激素(ATCH)、促甲状腺激素(TSH)、促性腺激素即黄体生成素(LH)和卵泡刺激素(FSH)降低。

\paragraph{兴奋试验}

在危象治疗好转后,可行兴奋试验进一步确诊。

\subsubsection{影像学检查}

\paragraph{磁共振成像}

(MRI)薄层扫描
通常作为首选的影像学检查,可以表现为下丘脑及垂体的占位病变、弥漫性病变、囊性变或空泡蝶鞍。

\paragraph{CT增强扫描}

对于有鞍底骨质破坏的患者及垂体卒中急性期的患者,CT比MRI有价值。

\paragraph{X线平扫及断层}

可表现为蝶鞍扩大、鞍底骨质破坏等。

\subsubsection{鉴别诊断}

\paragraph{黏液性水肿昏迷}

高血脂和黏液性水肿程度明显,甲状腺激素减低,垂体TSH明显增高,垂体促激素刺激试验始终不反应;而垂体前叶功能减退危象时,各相应的促激素正常或较低,垂体促激素刺激试验呈迟发反应。

\paragraph{胰岛素瘤或糖尿病用药引起的低血糖昏迷}

此种情况昏迷者,不伴腋毛、阴毛稀少等垂体功能减退表现,垂体及靶腺激素水平正常,血胰岛素水平升高。

\paragraph{脑血管意外}

病史、临床表现及激素测定有助鉴别。

\subsection{治疗}

一旦诊断垂体危象,及早应用糖皮质激素是抢救成功的关键,剂量为开始足量,根据病情的缓解程度逐渐减量直至替代剂量,补充了糖皮质激素才能有效纠正低血糖、低血压、低钠低氯血症。但对水中毒、失钠、低温型患者,糖皮质激素剂量不可过大。若同时合并甲状腺功能减退,甲状腺激素的替代应在糖皮质激素替代之后,小剂量开始,逐渐增加甲状腺激素的用量,直至生理替代剂量,若在使用糖皮质激素之前使用较大剂量的甲状腺激素,可能因加快糖皮质激素代谢而加重危象。

\paragraph{一般治疗}

一般先静注50\%葡萄糖40~60ml,继以10\%葡萄糖500~1000ml,内加氢化可的松100~300mg滴注,但低温型昏迷患者氢化可的松用量不宜过大。

\paragraph{低温型者}

治疗与黏液性水肿昏迷者相似,可用电热毯等将患者体温回升至35℃以上,但必须注意用甲状腺激素之前(至少同时)加用适量氢化可的松,此外,严禁使用氯丙嗪、巴比妥等中枢抑制剂。

\paragraph{严重低钠血症者}

需静脉补含钠液体,补钠时应缓慢,每小时血钠提高<
0.5mmol/L,但是最关键的措施仍是补充肾上腺皮质激素。

\paragraph{水中毒性昏迷者}

应立即给予小~中量的糖皮质激素,可口服泼尼松10~25mg或氢化可的松40~80mg,每6小时1次。不能口服者将氢化可的松50~200mg(地塞米松1~5mg),加入50\%葡萄糖液40ml,缓慢静脉注射,并适当限水。

\paragraph{加强诱因控制及对症支持治疗}

患者宜进高热量、高蛋白及富含维生素膳食,还需提供适量钠、钾、氯,但不宜过度饮水,防止劳累及应激刺激。病情平稳后,如果为育龄期妇女,可加用人工月经周期治疗,男性患者可补充雄激素以维持第二性征和性功能。

激素终生替代是治疗垂体功能减退的根本,遇感染等应激时激素应加量。一旦出现表情淡漠、嗜睡、定向力障碍、昏迷、低体温、低血压、低血钠、低血糖等症状体征即可认为垂体危象可能,应及时抢救,争取时间,抽血查垂体激素及靶腺激素水平、血常规及血生化后,立即使用糖皮质激素。如为垂体肿瘤内急性出血压迫视神经、出现垂体卒中,应尽快手术治疗。
\protect\hypertarget{text00111.html}{}{}

\hypertarget{text00111.htmlux5cux23CHP4-8-4}{}
参 考 文 献

1. 顾锋.垂体危象及垂体卒中.国外医学内分泌学分册,2005,25(6):433-435

2. Kearney T,Dang C. Diabetic and endocrine emergencies. Postgrad Med
J,2007 Feb,83(976):79-86

3. Arlt W. The approach to the adult with newly diagnosed adrenal
insufficiency. J Clin Endocrinol Metab,2009,94(4):1059-1067

\protect\hypertarget{text00112.html}{}{}

\chapter{甲状腺危象}

甲状腺危象(thyroid crisis,thyroid
storm)也称甲亢危象,是一种甲状腺毒症(thyrotoxicosis)病情极度加重的状态。甲亢危象是甲状腺功能亢进症(hyperthyroidism,甲亢)最严重的并发症,起病急、病情危重,不仅可导致多脏器功能衰竭,而且可导致死亡。早期诊断、及时正确治疗是成功抢救甲亢危象的关键,但积极预防甲亢危象的发生才是最重要的。

甲亢危象不常见。随着人们对该病症认识的提高、医疗条件和技术的改善,甲亢危象已经逐步减少。国外报道甲亢危象占甲状腺毒症患者的1\%左右。北京协和医院在20世纪70年代以前的44年间收治的甲亢危象36例次,占住院甲亢患者的1.45\%,在20世纪80年代初至2009年收治甲亢危象患者24例次,占同期住院甲亢患者的0.7\%。

甲亢危象与甲状腺毒症一样,好发于女性。可发生于任何年龄段,老年人多见,小儿少(罕)见。由各种原因导致甲状腺毒症的患者发生甲亢危象的危险都是存在的,其中以弥漫性毒性甲状腺肿(Graves病)最常见,其次为多结节性毒性甲状腺肿;也见于甲状腺损伤或甲状腺炎引起的甲状腺毒症。

\subsection{病因与发病机制}

\subsubsection{甲状腺毒症的病因}

甲状腺毒症是指血循环中甲状腺激素量过多,引起以神经、循环、消化等系统兴奋性增高和代谢亢进为主要表现的一组临床综合征。根据甲状腺的功能状态,甲状腺毒症可分为甲状腺功能亢进类型和非甲状腺功能亢进类型;前者的病因主要有Graves病、多结节性毒性甲状腺肿、甲状腺自主高功能腺瘤(Plummer病)、碘致甲状腺功能亢进症(碘甲亢)、桥本甲状腺毒症、TSH分泌性垂体腺瘤等,后者包括破坏性甲状腺毒症和服用外源性甲状腺激素。由于甲状腺滤泡被炎症(如亚急性甲状腺炎、无症状性甲状腺炎、桥本甲状腺炎、产后甲状腺炎等)破坏,滤泡内储存的甲状腺激素过量进入循环引起的甲状腺毒症称为破坏性甲状腺毒症。该类型甲状腺毒症的甲状腺功能并不亢进。

\subsubsection{甲亢危象的诱因}

多种原因可引发甲亢危象,这些原因可以是单一的,也可以由几种原因合并叠加引起。

\paragraph{内科方面的诱因}

①感染:感染是引发甲亢危象最常见的内科原因。主要包括上呼吸道感染、咽炎、扁桃体炎、气管炎、支气管肺炎,其次是胃肠道和泌尿系感染,脓毒病及其他感染如皮肤感染等均少见。②应激:精神极度紧张、工作过度劳累、高温、饥饿、药物反应(如药物过敏、白细胞明显减少、洋地黄中毒等)、心绞痛、心力衰竭、糖尿病酸中毒、低血糖、高钙血症、肺栓塞、脑梗死及其他脑血管意外、妊娠(甲亢患者妊娠后未治疗的,较给予治疗者发生危象几率多达10倍以上)、分娩及妊娠高血压疾病等,均可能导致甲状腺突然释放大量甲状腺激素,引起甲亢危象。③不适当停用碘剂药物:应用碘剂治疗甲亢中,突然停用碘剂,原有甲亢表现可迅速加重,因为碘化物可以抑制甲状腺激素结合蛋白质的水解,使甲状腺激素释放减少。此外,细胞内碘化物增加超过临界浓度时,可使甲状腺激素的合成受抑制,由于突然停用碘剂,甲状腺的滤泡上皮细胞内碘的浓度减低,抑制效应消失,甲状腺内原来贮存的碘又能合成甲状腺激素,释入血中,使病情迅速增重。不规则使用或停用硫脲类抗甲状腺药,偶尔也会引发甲亢危象,但这种情况并不多见。④少见原因:由于放射性碘治疗甲亢引起的放射性甲状腺炎、甲状腺活体组织检查,以及过多或过重或反复触摸甲状腺,使甲状腺损伤,均可使大量的甲状腺激素在短时间内释放进入血中,引起病情突然增重。也有称给碘剂(碘造影剂或口服碘)也可引发甲亢危象。此甲亢并发症也会发生于以前存在甲状腺毒症治疗不充分或始终未进行治疗的患者。

\paragraph{外科方面的诱因}

甲亢患者在手术后4~16小时内发生危象者,要考虑危象与手术有关;而危象在16小时以后出现者,尚需寻找感染病灶或其他原因。由手术引起甲亢危象的原因有:①甲亢病情未被控制而行手术:甲亢患者术前未用抗甲状腺药做准备;或因用药时间短或剂量不足,准备不充分;或虽用抗甲状腺药,但已经停药过久,手术时甲状腺功能仍处于亢进状态;或是用碘剂做术前准备时,用药时间较长,作用逸脱,甲状腺又能合成及释放甲状腺激素。②术中释放甲状腺激素:手术本身的应激、手术时挤压甲状腺,使大量甲状腺激素释放进入血中。另外,采用乙醚麻醉时也可使组织内的甲状腺激素进入末梢血中。③剖宫产或甲状腺以外的其他手术。

一般来说,内科方面的原因诱发的甲亢危象,其病情较外科方面的原因引起的甲亢危象更为常见,程度也严重。

\subsubsection{发病机制}

甲亢危象发生的确切机制尚不完全清楚,可能与下列因素有关,这些因素可以解释部分患者甲亢危象的发生原因,尚不能概括全部甲亢危象发生机制。

\paragraph{大量甲状腺激素释放至血循环}

它不是导致甲亢危象发生最主要的原因,但与服用大量甲状腺激素、甲状腺手术、不适当的停用碘剂以及放射性碘治疗后甲亢危象发生有关。

\paragraph{血中游离甲状腺激素增加}

感染、甲状腺以外其他部位的手术等应激,可使血中甲状腺激素结合蛋白质浓度减少,与其结合的甲状腺激素解离,血中游离甲状腺激素增多。这可以解释部分甲亢危象患者的发病。

\paragraph{周围组织对甲状腺激素反应的改变}

由于某些因素的影响,使甲亢患者身体各系统的脏器及周围组织对过多的甲状腺激素适应能力减低,由于此种失代偿而引起危象。临床上见到在甲亢危象时,有多系统的功能衰竭、血中甲状腺激素水平可不升高,以及在一些患者死后尸检所见无特殊病理改变,均支持对甲状腺激素反应的改变的这种看法。

\paragraph{儿茶酚胺结合和反应力增加}

在甲亢危象发病机制中儿茶酚胺起关键作用。甲亢危象患者的儿茶酚胺结合位点增加,对肾上腺素能刺激反应力增加,阻断交感神经或服用抗交感神经或β-肾上腺素能阻断剂后甲亢和甲亢危象的症状和体征可明显改善。

\paragraph{甲状腺素在肝中清除降低}

手术前、后和其他的非甲状腺疾病的存在、进食量减少,热量不足,均引起T\textsubscript{4}
清除减少,血中甲状腺素含量增加。

\subsection{诊断}

\subsubsection{临床表现特点}

多数患者原有明显甲状腺毒症相关临床表现,在诱发因素作用下出现临床表现明显加重为甲亢危象,少数患者起病迅猛,快速进入到甲亢危象。

甲亢危象典型临床表现包括:

\paragraph{高热}

本症发生体温急骤升高,多常在39℃以上,伴大汗淋漓,皮肤潮红,严重者,继而汗闭,皮肤苍白和脱水。高热是甲亢危象的特征性表现,是与重症甲亢的重要鉴别点。

\paragraph{中枢神经系统异常}

精神变态、焦虑,肢体震颤、极度烦躁不安、甚至出现谵妄、嗜睡,最后陷入昏迷状态。部分患者可伴有脑血管病发生,脑出血或脑梗死。

\paragraph{心血管功能异常}

心动过速,大于140次/分以上,甚至超过160次/分。伴有各种形式的快速心律失常,特别是快速房颤。有些患者可出现心绞痛,心力衰竭,收缩压增高、脉压显著增加。随病情恶化,最终血压下降,陷入休克。一般来说,甲亢伴有甲亢性心脏病的患者,容易发生甲亢危象,当发生危象以后,会促使心脏功能进一步恶化。

\paragraph{消化功能异常}

食欲极差,进食减少,恶心,呕吐频繁,腹痛,腹泻明显。腹痛及恶心、呕吐可发生在病的早期。病后体重锐减。肝脏可肿大,肝功能不正常,随病情的进展,肝细胞功能衰竭,常出现黄疸。黄疸的出现则预示病情严重及预后不良。

\paragraph{电解质紊乱}

由于进食差,呕吐、腹泻以及大量出汗,最终出现电解质紊乱,约半数患者有低血钾症,1/5的患者血钠减低。一些患者出现酸碱失衡。

有些患者甲亢危象临床征象不明显,称做“安静”类型。临床表现为行为改变,睡眠及记忆力障碍,痴呆,抑郁,嗜睡以及被动处事等。

很少一部分患者临床症状和体征甚至更不典型,表现为“淡漠型”。其特点是表情淡漠、木僵、嗜睡、反射降低、低热、明显乏力、心率慢、脉压小及恶病质,甲状腺常仅轻度肿大,最后陷入昏迷,甚而死亡。多见于老年及体质极度衰弱者。

\subsubsection{实验室检查}

危象时,血白细胞数可升高,伴轻度核左移。可有不同程度的肝功能异常、血清电解质异常,包括轻度的血清钙和轻度血糖水平升高。

危象时,血清甲状腺激素水平升高,但升高的程度不一致,多数升高程度与一般甲状腺毒症患者比较没有更显著增高,危象病程后期有些患者血清T\textsubscript{3}
水平甚至在正常范围。因此,通过血中甲状腺激素水平高低对甲亢危象的诊断帮助不大。

\subsubsection{诊断标准和注意事项}

任何一个甲状腺毒症的患者,特别是未经正规治疗、或治疗中断及有上述的内科及外科方面的诱因存在时,出现原有的甲亢病情突然明显增重,应考虑有甲亢危象的可能。

甲亢病史和一些特殊体征,如突眼,甲状腺肿大或其上伴血管杂音,以及胫骨前黏液性水肿、皮肤有白癜风及杵状指等表现提示存在甲亢可能,对诊断甲亢危象均有帮助。临床上怀疑有甲亢危象时,可先取血备查甲状腺激素。

甲亢危象尚无统一诊断标准。Wartofsky和Peele介绍用打分法(即根据体温高低,中枢神经系统影响,胃肠功能的损害,心率的增加,充血性心衰表现程度,心房纤颤的有无,诱因的存在与否来评分,依据打分后的最后积分<
25,25~44及>
45来判断为不能诊断、怀疑或确诊)。北京协和医院通过多年的临床实践,将甲亢危象大体分为两个阶段,即体温低于39℃和脉率在159次/分以下,多汗、烦躁、嗜睡、食欲减退、恶心以及大便次数增多等定为甲亢危象前期;而当患者体温超过39℃,脉率多于160次/分,大汗淋漓或躁动,谵妄,昏睡和昏迷,呕吐及腹泻显著增多等,定为甲亢危象。在病情处于危象前期时,如未被认识、未得到及时处理,会发展为危象。甲亢患者当因各种原因使甲亢的病情加重时,只要具备上述半数以上危象前期诊断条件,即应按危象处理。

\subsection{治疗}

不论甲亢危象前期或甲亢危象一经诊断,就应立即开始治疗,一定不要等待血清甲状腺激素的化验结果,才开始治疗。治疗的目的是纠正严重的甲状腺毒症和诱发疾病,保护脏器功能,维持生命指征。对怀疑有甲亢危象的患者,开始治疗时,应当在加强医疗病房(ICU)进行持续监护。

\subsubsection{保护机体脏器、防止功能衰竭}

改善危重病况,积极维护生命指征是救治的首要目标。

\paragraph{降温}

发热轻者,用退热剂,可选用对乙酰氨基酚(扑热息痛),冰袋,室内用电风扇(及)适当的空调也需要。不宜用阿司匹林。大剂量的阿司匹林可增高患者的代谢率,还可与血中的T\textsubscript{3}
及T\textsubscript{4}
竞争结合TBG及TBPA,使血中游离甲状腺激素增多。有高热者,须积极物理降温,如电风扇、冰袋、空调,必要时可用人工冬眠{[}哌替啶(度冷丁)100mg,氯丙嗪及异丙嗪各50mg,混后静脉持续泵入{]}。

\paragraph{给氧和支持治疗}

持续给氧是必要的。因高热,呕吐及大量出汗,极易发生脱水及高钠血症,需补充水及注意纠正电解质紊乱。补充葡萄糖可提供必需的热量和糖原。还应补充大量维生素。有心力衰竭或有肺充血存在,应积极处理,应用洋地黄及利尿剂。对有心房纤颤、房室传导阻滞、心率增快的患者,应当使用洋地黄及其衍生物或钙离子通道阻断剂。

\subsubsection{减少甲状腺激素的合成和释放}

\paragraph{抑制甲状腺激素的合成}

抑制甲状腺激素的合成可选用硫脲类抗甲状腺药。口服或经胃管鼻饲或必要时直肠给药,大剂量硫脲类药物(丙硫氧嘧啶,PTU
600~1000mg/d,分次用),在1小时内可阻止甲状腺内碘化物有机结合。此后每日给用维持剂量(相当于PTU
300~600mg/d,分次给药)。甲亢危象时选用PTU优于甲巯咪唑,PTU不仅可抑制甲状腺激素的合成,还可以抑制甲状腺外T\textsubscript{4}
向T\textsubscript{3} 转化。用PTU 1天以后,血中的T\textsubscript{3}
水平可降低50\%。

\paragraph{抑制甲状腺激素的释放}

用硫脲类抗甲状腺药1小时后,开始给碘剂,迅速抑制TBG水解,从而减少甲状腺激素的释放。一般每日口服复方碘溶液(Lugol液)30滴(也有用5滴每6小时一次),或静脉滴注碘化钠1~2g(或0.25g/6h),或复方碘溶液3~4ml/1000~2000ml
5\%葡萄糖溶液中。若碘化物的浓度过高或滴注过快易引起静脉炎。既往未用过碘剂者,使用碘剂效果较好。有报告在碘化物中用5'脱碘酶的强抑制剂胺碘苯丙酸钠(sodium
ipodate)0.5g,每日2次,或0.25g/6h,可减缓甲状腺激素从甲状腺的释放,或用碘番酸钠(sodium
iopanoate)替代碘化物更有效。

\subsubsection{降低循环中甲状腺激素水平}

硫脲类抗甲状腺药和碘化物只能减少甲状腺激素的合成和释放,不能快速降低已经释放到血中的甲状腺激素水平,尤其是T\textsubscript{4}
,它的半衰期为6.1天,且绝大部分是与血浆蛋白质结合的,在循环中保留的时间相当长。文献中介绍可迅速清除血中过多的甲状腺激素的方法有:换血法、血浆除去法和腹膜透析法,这些方法均较复杂,应用经验较少。

\subsubsection{降低周围组织对甲状腺激素的反应}

对已经释入血中的甲状腺激素,应设法减低末梢组织对其反应。抗交感神经药物可减轻周围组织对儿茶酚胺作用。常用的有:

\paragraph{β-肾上腺素能阻断剂}

对抗肾上腺素能的药物对循环中甲状腺激素能间接发挥作用。在无心功能不全时,β-肾上腺素能阻断剂用来改善临床表现。严重甲状腺毒症患者能发展为高排出量的充血性心力衰竭,β-肾上腺素能阻断剂的对抗可进一步减少心脏的排出。常用的是普萘洛尔(心得安),甲亢患者用本品后,虽然对甲状腺功能无改善,但用药后患者的兴奋、多汗、发热、心率增快等均有好转。目前认为本品有抑制甲状腺激素对交感神经的作用,也可较快的使血中T\textsubscript{4}
转变为T\textsubscript{3}
降低。用药剂量需根据具体情况决定,危象时一般每6小时口服40~80mg,或静脉缓慢注入2mg,能持续作用几小时,可重复使用。心率常在用药后数小时内下降,继而体温、精神症状,甚至心律失常(期前收缩、心房纤颤)也均可有明显改善。严重的甲状腺毒症患者可发展为高排出量的充血性心力衰竭,β-肾上腺素能阻断剂可进一步减少心排血量。但对有心脏储备功能不全、心脏传导阻滞、心房扑动、支气管哮喘等患者,应慎用或禁用。使用洋地黄制剂心力衰竭已被纠正,在密切观察下可以使用普萘洛尔或改用超短效的艾司洛尔(esmolol),静脉使用。

\paragraph{利血平}

消耗组织内的儿茶酚胺,大量时有阻断作用,减轻甲亢在周围组织的表现。首次可肌注2.5~5mg,以后每4~6小时注射2.5mg,约4小时以后危象表现减轻。利血平可抑制中枢神经系统及有降血压作用,用时应予考虑。

\subsubsection{糖皮质激素的使用}

甲亢危象时肾上腺皮质激素的需要量增加,此外,甲亢时糖皮质激素代谢加速,肾上腺存在潜在的储备功能不足,在应激情况下,激发代偿分泌更多的皮质激素,导致皮质功能衰竭。另外肾上腺皮质激素还可抑制血中T\textsubscript{4}
转换为T\textsubscript{3}
。因此,抢救甲亢危象时需使用糖皮质激素。皮质激素的用量是相当于氢化可的松200~300mg/d,或地塞米松4~8mg/d,分次使用。

\subsection{预后}

甲亢危象死亡率为20\%以上(20\%~50\%)。治疗后成功者多在治疗后1~2天内好转,1周内恢复。北京协和医院的36例次危象患者,平均在抢救治疗后3天内脱离危险,7(1~14)天恢复。开始治疗后的最初3天是抢救的关键时刻。危象消失以后,碘剂及皮质激素可逐渐减药、停用,做甲亢病的长期治疗安排。
\protect\hypertarget{text00113.html}{}{}

\hypertarget{text00113.htmlux5cux23CHP4-9-5}{}
参 考 文 献

1. 白耀.甲状腺病学:基础与临床.北京:科学技术文献出版社,2003:266-271

2. LW. Braverman,RD. Utiger. Werner & Ingbar's The Thyroid a
fundamental and clinical text. Ninth edition. Philadelphia:Lippincott
William & Wilkins,2005:651-657

\protect\hypertarget{text00114.html}{}{}

\chapter{甲状腺功能减退危象}

甲状腺功能减退症(hypothyroidism,甲减)是由各种原因导致的低甲状腺激素血症或甲状腺激素抵抗而引起的全身性低代谢综合征,其病理特征是黏多糖在组织和皮肤堆积,表现为黏液性水肿(myxedema)。甲状腺功能减退危象(hypothyroidism
crisis,HC),又称为黏液性水肿昏迷,是甲状腺功能低下失代偿的一种严重的临床状态,病情重笃,往往威胁患者生命,临床表现复杂,病史隐匿,易误诊误治。

本病的初始阶段往往伴有不同的诱发疾病与诱发因素,若不能及时诊断与治疗,进一步发展可使心血管系统与神经系统发生致命性的功能衰竭。在老年人,这种失代偿状况尤为突出,故本症多发生于老年患者。

\subsection{病因与发病机制}

\subsubsection{病因与分类}

\hypertarget{text00114.htmlux5cux23CHP4-10-1-1-1}{}
(一) 甲减的分类方法

\paragraph{根据病变发生的部位分类}

①原发性甲减(primary
hypothyroidism):由于甲状腺腺体本身病变引起的甲减,占全部甲减的95\%以上,且90\%以上原发性甲减是由自身免疫、甲状腺手术和甲亢\textsuperscript{131}
I治疗所致。②中枢性甲减(central
hypothyroidism):由下丘脑和垂体疾病引起的促甲状腺激素释放激素(TRH)或者促甲状腺激素(TSH)产生和分泌减少所致的甲减。垂体外照射、垂体大腺瘤、颅咽管瘤及产后大出血是其较常见的原因;其中由于下丘脑病变引起的甲减称为三发性甲减(tertiary
hypothyroidism)。③甲状腺激素抵抗综合征:由于甲状腺激素在外周组织实现生物效应障碍引起的综合征。

\paragraph{根据病变的原因分类}

有药物性甲减、\textsuperscript{131}
I治疗后甲减、手术后甲减、特发性甲减、垂体或下丘脑肿瘤手术后甲减等。

\paragraph{根据甲状腺功能减退的程度分类}

①临床甲减(overt
hypothyroidism):具有甲状腺功能减退的临床表现及血清甲状腺激素(T\textsubscript{3}
、T\textsubscript{4} 、FT\textsubscript{4}
)降低。②亚临床甲减(subclinical
hypothyroidism):指临床上无或仅有少许甲减症状,血清FT\textsubscript{3}
、及FT\textsubscript{4}
正常而TSH水平升高。需根据TSH测定和(或)TRH试验确诊。

\paragraph{以甲减起病时年龄分类}

①功能减退始于胎儿期或出生不久的新生儿者,称为呆小病(又称克汀病);②功能减退始于发育前儿童期者,称为幼年甲减;③功能减退始于成人期者,称为甲减。

\hypertarget{text00114.htmlux5cux23CHP4-10-1-1-2}{}
(二) 病因

成人甲减的主要病因是
:①自身免疫损伤:最常见的原因是自身免疫性甲状腺炎,包括桥本甲状腺炎、萎缩性甲状腺炎、产后甲状腺炎等;②甲状腺破坏:包括手术、\textsuperscript{131}
I治疗。甲状腺次全切除、\textsuperscript{131}
I治疗Graves病,十年的甲减累积发生率分别为40\%、40\%~70\%。③碘过量:可引起具有潜在性甲状腺疾病者发生一过性甲减,也可诱发和加重自身免疫性甲状腺炎。含碘药物胺碘酮诱发甲减的发生率是5\%~22\%。④抗甲状腺药物:如锂盐、硫脲类、咪唑类等。HC多见于年老长期未获治疗者,受寒及感染是最常见的诱因。

\subsubsection{发病机制}

\paragraph{氧耗与体热}

患者的机体氧耗和体热的产生均相应地下降,同时通过神经血管调节,限制皮肤的血液循环,以减少体热的丢失。在老年患者,氧耗与体热产生的下降更为明显,且代偿能力差,故易出现低体温,在冬季甚至在正常的温度下均可发生。

\paragraph{心血管系统}

心肌黏液性水肿导致心肌收缩力损伤、心动过缓、心排出量下降。ECG示低电压。由于心肌间质水肿、非特异性心肌纤维肿胀、左心室扩张和心包积液导致心脏增大,有学者称之为甲减性心脏病。冠心病在本病中高发。10\%的患者伴发高血压。HC晚期,血压转为持续性下降。

\paragraph{交感神经系统}

循环的儿茶酚胺(去甲肾上腺素和肾上腺素)水平一般正常,但终末器官对儿茶酚胺的反应性明显低下。β-肾上腺素能反应性的低下主要由于β-受体数量的减少,G蛋白调节异常和磷酸二酯酶活性增加。相反在β-肾上腺素能反应受损的同时,α-肾上腺素能活性却完整地保持正常。当输注小剂量肾上腺素时,正常人表现为心动过速与血管扩张;但在HC患者却表现为血管收缩和高血压反应。β-肾上腺素能活性低下,也损害了热能产生的反应能力,一旦热量丢失过多,就不能保持正常体温。

\paragraph{呼吸系统}

甲状腺功能低下,可以损害对高碳酸血症的呼吸反应能力,导致低通气量,极易发生CO\textsubscript{2}
潴留,当并发肺部感染时,CO\textsubscript{2} 潴留尤为加重。

\paragraph{肾脏功能}

水分排出受损,易发生低钠血症。水潴留主要由于血抗利尿激素(ADH)水平升高和肾脏血流量减少所致。后者与功能性血容量减少、心排血量下降有关。低钠血症,特别是在老年患者,常可进一步促使中枢神经系统损害,加重HC的神经精神异常。

\paragraph{代谢系统}

易发生低血糖。主要由于胰岛素清除率下降和糖原生成减少。另外,对肾上腺素与胰高血糖素的反应能力也受到了损害。血皮质醇虽然仍可维持在正常的基础水平,但其应激反应却严重受损。在一般情况下,几乎各种药物的清除率都是下降的,从而易致药物中毒,许多常用药物,如地高辛、利尿药与镇静药等,若给予常规剂量均可致药物中毒。此外,血浆肌酶包括转氨酶、磷酸肌酸激酶和乳酸脱氢酶及其异构形式,均易呈现升高。这些肌酶升高,估计是由于从骨骼肌向外渗漏及清除率下降而造成的。

\paragraph{血液系统}

由于下述四种原因发生贫血:①甲状腺激素缺乏引起Hb合成障碍;②肠道吸收铁障碍引起铁缺乏;③肠道吸收叶酸障碍引起叶酸缺乏;④恶性贫血是与自身免疫性甲状腺炎伴发的器官特异性自身免疫病。

\subsection{诊断}

\paragraph{病史}

HC多见于年老长期未获治疗者。昏迷前常有乏力、怠惰、反应迟缓、怕冷、食欲不振、便秘、体重增加、声音粗哑和听力下降,少数患者昏迷前有情绪抑郁或胡言乱语,类似精神分裂症的表现。如果患者有甲状腺疾病、甲状腺手术、放射碘治疗或其颈部放射线照射或分娩大出血与休克的病史,或其他垂体与下丘脑疾病史则有助于诊断。

\paragraph{临床表现特点}

HC见于病情严重的甲减患者,多在冬季寒冷时发病。诱因为严重的全身性疾病、甲状腺激素替代治疗中断、寒冷、手术、麻醉和使用镇静药等。临床表现为嗜睡、低体温(<
35℃)、呼吸徐缓、心动过缓、血压下降、四肢肌肉松弛、反射减弱或消失,甚至昏迷、休克、肾功能不全危及生命。

本病常有典型的甲减面容,如水肿、呆滞、唇厚、鼻宽、舌大、面色蜡黄、粗糙。全身皮肤发凉、水肿、弹性差,头发稀、干、缺少光泽,眉毛少。多数患者甲状腺不大。约1/3患者有心界扩大或心包积液,心音低钝而缓慢。胸腔积液或腹水也不少见。常有肝大。四肢肌张力低,腱反射低或消失。

\paragraph{实验室检查}

血清TSH升高、TT\textsubscript{4} 、FT\textsubscript{4}
降低为原发性甲减,在严重病例血清TT\textsubscript{3}
和FT\textsubscript{3}
减低;亚临床甲减仅有血清TSH升高,血清T\textsubscript{4}
或T\textsubscript{3} 正常。若TSH降低或正常,TT\textsubscript{4}
、FT\textsubscript{4}
降低,考虑中枢性甲减;做TRH刺激试验有助鉴别:静注TRH后,血清TSH不增高者提示为垂体性甲减,延迟增高者为下丘脑性甲减,在增高的基值上进一步增高者提示原发性甲减。

\paragraph{鉴别诊断}

血清TSH升高、TT\textsubscript{4} 、FT\textsubscript{4}
降低是诊断甲减的必备条件。鉴别诊断包括:①贫血:应与其他原因的贫血鉴别。②蝶鞍增大:应与垂体瘤鉴别。原发性甲减时TRH分泌增加可致高PRL血症、溢乳及蝶鞍增大,酷似垂体催乳素瘤,可行MRI检查鉴别。③心包积液:应与其他原因的心包积液鉴别。④水肿:主要与特发性水肿鉴别。⑤低T\textsubscript{3}
综合征:也称为甲状腺功能正常的病态综合征(euthyroid sick
syndrome,ESS),指非甲状腺疾病原因引起的伴有低T\textsubscript{3}
的综合征。严重的全身性疾病、创伤和心理疾病等都可导致甲状腺激素水平的改变,它反映了机体内分泌系统对疾病的适应性反应。主要表现在血清TT\textsubscript{3}
、FT\textsubscript{3} 水平降低,血清rT\textsubscript{3}
增高,血清T\textsubscript{4}
、TSH水平正常。疾病的严重程度一般与T\textsubscript{3}
降低的程度相关,疾病危重时也可出现T\textsubscript{4} 水平降低。

\subsection{治疗}

\paragraph{补充甲状腺激素}

HC患者都应给予甲状腺激素治疗,以逆转甲状腺功能低下状态,适应感染或其他原因的应激状况。首选L-T\textsubscript{3}
静脉注射,每4小时10μg,直至患者症状改善,清醒后改为口服;或L-T\textsubscript{4}
首次静脉注射300μg,以后每日50μg,至患者清醒后改为口服。如无注射剂可予片剂研细加水鼻饲,L-T\textsubscript{3}
20~30μg,每4~6小时1次,以后每6小时5~15μg;或L-T\textsubscript{4}
首次100~200μg,以后每日50μg,至患者清醒后改为口服。

\paragraph{对症支持治疗}

①保暖;②保持呼吸道通畅,供氧,必要时行气管插管或切开,机械通气;③静滴氢化可的松200~300mg/d,患者清醒后逐渐减量;④纠正水电解质紊乱,但入水量不宜过多,以避免水中毒、低钠血症及心力衰竭的发生或加重;⑤控制感染:细菌感染是HC最普通的诱发因素。在情况未明之前,所有的HC患者都应该疑及感染存在的可能。并在培养结果出来之前,静脉给予广谱抗生素治疗;⑥积极治疗原发疾病。

患者治疗中应注意:①L-T\textsubscript{4}
剂量不要随意增大,尤其中老年患者可能有并存的冠心病,如L-T\textsubscript{4}
剂量过大,有引致急性心肌梗死的危险;②遇有严重肺部感染的HC患者,应及时气管切开,并使用机械通气,以及早改善通气状况;③当HC患者血压下降时,切勿随意使用血管性升压药,而应静脉补充液体,以扩张血容量;④遇低体温的患者,切勿随意从外部加温。

HC患者经上述治疗,24小时左右病情有好转,则一周后可逐渐恢复;如24小时后不能逆转,多数不能挽救。

甲减患者,一般不能治愈,需要终生用甲状腺激素替代治疗。治疗的目标是将血清TSH和甲状腺激素水平恢复到正常范围内。治疗的剂量取决于患者的病情、年龄、体重和个体差异。首选左甲状腺素(L-T\textsubscript{4}
),其半衰期为7天,吸收缓慢,每天晨间服药一次即可维持较稳定的血药浓度。长期替代治疗维持量成年患者约50~200μg/d(1.6~1.8μg/kg)。儿童需要较高的剂量,大约2.0μg/(kg•d);老年患者则需要较低的剂量,大约1.0μg/(kg•d);妊娠时的替代剂量需要增加30\%~50\%;甲状腺癌术后的患者需要剂量大约2.20μg/(kg•d)。一般初始剂量为25~50μg/d,每1~2周增加12.5~25μg/d,直至达到治疗目标。在老年和缺血性心脏病患者,初始剂量为12.5~25μg/d,每2~4周增加12.5~25μg/d,以避免诱发或加重冠心病。补充甲状腺激素,重新建立下丘脑-垂体-甲状腺轴的平衡一般需要4~6周,所以治疗初期,每4~6周测定激素指标。然后依据检查结果调整L-T\textsubscript{4}
剂量,直到达到治疗目标。治疗达标后,需要每6~12个月复查一次激素指标。L-T\textsubscript{3}
(60~100μg/d)起效快、作用强,但持续时间短,一般不用于替代治疗。甲状腺片是动物甲状腺的干制剂,因其甲状腺激素含量不稳定和T\textsubscript{3}
含量过高已很少使用。

对亚临床甲减,在下述情况需要给予L-T\textsubscript{4}
治疗:高胆固醇血症、血清TSH > 10mU/L。
\protect\hypertarget{text00115.html}{}{}

\hypertarget{text00115.htmlux5cux23CHP4-10-4}{}
参 考 文 献

1. 陆再英,钟南山.内科学.第7版.北京:人民卫生出版社,2008

2. 陈灏珠 ,林果为.实用内科学.第13版.北京:人民卫生出版社,2009

3. 张文武 .急诊内科学.第2版.北京:人民卫生出版社,2007

\protect\hypertarget{text00116.html}{}{}

\chapter{肾上腺危象}

肾上腺危象(adrenal crisis)亦称急性肾上腺皮质功能减退症(acute
adrenocortical hypofunction)或艾迪生病危象(Addisonian
crisis),是指肾上腺皮质功能急性衰竭所致的危重综合征。病因多由于肾上腺皮质严重破坏致肾上腺皮质激素绝对不足,或慢性肾上腺皮质功能减低,患者在某种应激情况下肾上腺皮质激素相对不足所致。主要临床表现为高热、胃肠功能紊乱、循环虚脱、神志淡漠、萎靡或躁动不安、谵妄甚至昏迷,诊治稍失时机将威胁患者生命。

\subsection{病因与发病机制}

肾上腺危象的常见病因有:

1.急性肾上腺皮质出血 、坏死
①最常见的病因是感染。严重感染脓毒症合并全身和双侧肾上腺出血,如流行性脑脊髓膜炎合并的Waterhause-Friderichsen综合征、流行性出血热合并肾上腺出血等;②全身性出血性疾病合并肾上腺出血,如血小板减少性紫癜、DIC、白血病等;③癌瘤的肾上腺转移破坏;④外伤引起肾上腺出血或双侧肾上腺静脉血栓形成,以及抗凝药物治疗引起的肾上腺出血等。

2.肾上腺双侧全部切除
,或一侧全切、另侧90\%以上次全切除后,或单侧肿瘤切除而对侧已萎缩者,如术前准备不周、术后治疗不当或激素补给不足、停药过早等均可发生本症。

3.原发和继发性慢性肾上腺皮质功能不全患者
,在下列情况下可发生肾上腺危象:①Addison患者和肾上腺次全切除术后患者,在感染、劳累、外伤、手术、分娩、呕吐、腹泻和饥饿等应激情况下可致肾上腺危象;②长期激素替代治疗患者突然减停激素;③垂体功能减低患者如Sheehan征在未补充激素情况下给予甲状腺素或胰岛素时也可能诱发肾上腺危象。

4.长期大剂量肾上腺皮质激素治疗过程中
,由于患者垂体、肾上腺皮质已受重度抑制而呈萎缩,如骤然停药或减量过速,可引起本症。

肾上腺皮质激素是维持人的生命活动所必需的。正常人在严重应激情况下皮质醇分泌增加10倍于基础水平,但慢性肾上腺皮质功能减低、肾上腺皮质破坏的患者则不仅没有相应的增加,反而是肾上腺皮质激素严重不足。当盐类皮质激素不足时,肾小管回吸收Na\textsuperscript{+}
不足,失水、失Na\textsuperscript{+} ,K\textsuperscript{+}
、H\textsuperscript{+}
潴留;而糖皮质激素不足除糖原异生减弱致低血糖外,也有与盐皮质激素对水盐相同的作用,由于失Na\textsuperscript{+}
、失水引起血容量减少,血压下降以致虚脱和休克,引起肾上腺危象。

\subsection{诊断}

\subsubsection{临床表现特点}

肾上腺危象的发病可呈急性型,即可因皮质激素缺乏或严重应激而骤然发病;也可以呈亚急性型,主要是由于部分皮质激素分泌不足或轻型应激所造成,临床上发病相对缓慢,但疾病晚期也可以表现为严重的急性型。发生危象时,即具有共同的临床表现,也可因原发病不同而表现出各自的特点。

\paragraph{原发病的不同与起病特点}

各种病因所致的肾上腺危象本身的表现是相同的,它们之间的鉴别有赖于发生危象前各自的临床特征;危象的诱因和起病特点也有参考价值。①手术所致的肾上腺危象多于术后即发生,因失盐、失水有一个过程,常于48小时后症状明显。②难产分娩的新生儿若有肾上腺出血也常在出生后数小时至1~2天内发生危象。③弥散性血管内凝血所致者,常有严重的感染、休克、出血倾向、缺氧发绀及多器官栓塞等表现,凝血机制检查有异常发现。④流脑所致者,有烦躁、头痛呕吐、神志改变、颅内压增高、高热、皮肤黏膜紫斑、白细胞升高、脑脊液异常等。⑤慢性肾上腺皮质功能减退症常有明显色素沉着、消瘦、低血压、反复昏厥发作等病史。⑥长期应用肾上腺皮质激素者有向心性肥胖、多血质、高血压、肌肉消瘦、皮肤薄等库欣综合征表现。⑦肾上腺动静脉中血栓形成时,可出现骤然腹痛,疼痛位于患侧脐旁约在肋缘下6.5cm,一般早期无高热、休克与心率及呼吸呈显著加速等表现。

\paragraph{肾上腺危象的共同表现}

典型的肾上腺危象的表现有:

\hypertarget{text00116.htmlux5cux23CHP4-11-2-1-2-1}{}
(1) 循环系统:

心率快,可达160次/分钟以上,心律失常,脉搏细弱,全身皮肤湿冷、四肢末梢发绀,血压下降,虚脱,休克。

\hypertarget{text00116.htmlux5cux23CHP4-11-2-1-2-2}{}
(2) 消化系统:

食欲不振甚至厌食,恶心、呕吐,腹痛、腹泻、腹胀。部分病例的消化道症状特别明显,出现严重腹痛、腹肌紧张、反跳痛,酷似外科急腹症。

\hypertarget{text00116.htmlux5cux23CHP4-11-2-1-2-3}{}
(3) 神经系统:

极度孱弱,萎靡不振,烦躁不安、谵妄,逐渐出现淡漠、嗜睡、神志模糊,严重者乃至昏迷。有低血糖者常有出汗、震颤、视力模糊、复视,严重者精神失常、抽搐。

\hypertarget{text00116.htmlux5cux23CHP4-11-2-1-2-4}{}
(4) 泌尿系统:

因循环衰竭、血压下降,导致肾功能减退,血中尿素氮增高,出现少尿、无尿等。

\hypertarget{text00116.htmlux5cux23CHP4-11-2-1-2-5}{}
(5) 全身症状:

极度乏力,严重脱水(细胞外液容量丧失约1/5)。绝大多数有高热,亦可有体温低于正常者。最具特征性者为全身皮肤色素沉着加深,尤以暴露处、摩擦处、掌纹、乳晕、瘢痕等处为明显,黏膜色素沉着见于齿龈、舌部、颊黏膜等处,系垂体ACTH、黑素细胞刺激素(MSH)分泌增多所致。

\paragraph{肾上腺切除后发生本症的两种类型}

①糖皮质激素缺乏型:一般出现于停止补充可的松治疗1~2天后,有厌食、腹胀、恶心、呕吐、精神萎靡、疲乏嗜睡、肌肉僵硬、血压下降等表现。严重者可有虚脱、休克、高热等危象。②盐皮质激素缺乏型:由于术后补钠或摄入不足,加以厌食、恶心、呕吐、失水、失钠,常于症状发生5~6天出现疲乏软弱、四肢无力、肌肉抽搐,血压、体重、血钠、血容量下降而发生本症。

\subsubsection{实验室检查}

本症的实验室检查特点是三低(低血糖、低血钠、低皮质醇)、两高(高血钾、高尿素氮)和外周血嗜酸性粒细胞增高(常>
0.05 × 10\textsuperscript{9} /L,可高达0.3 × 10\textsuperscript{9}
/L,此与非肾上腺病引起的休克时常< 0.05 × 10\textsuperscript{9}
/L者不同。应除外合并寄生虫病及过敏性休克)。最具诊断价值者为ACTH兴奋试验,肾上腺皮质功能减退症患者示储备功能低下,而非本症患者,经ACTH兴奋后血、尿皮质类固醇明显上升。

\subsubsection{诊断注意事项}

肾上腺危象的诊断不难,关键在于能否想到本症的可能性和是否对本症有足够的认识。在临床急诊工作中,若患者有导致肾上腺危象的上述原因与诱因,又出现下列情况之一时应考虑到危象的可能:①不能解释的频繁呕吐、腹泻或腹痛;②发热、白细胞增高但用抗生素治疗无效;③顽固性休克;④顽固性低血钠(血钠/血钾<
30);⑤反复低血糖发作;⑥不能解释的神经精神症状;⑦精神萎靡、明显乏力、虚脱或衰弱与病情不成比例,且出现迅速加深的皮肤色素沉着。简言之,凡有慢性肾上腺皮质功能减退、皮质醇合成不足的患者,一旦遇有感染、外伤或手术等应激情况时,出现明显的消化道症状、神志改变和循环衰竭即可诊断为危象。

由于大多数肾上腺危象患者表现有恶心、呕吐、脱水、低血压、休克和意识障碍、昏迷,必须与其他病因的昏迷鉴别,如糖尿病酮症酸中毒昏迷、高血糖高渗状态、急性中毒及急性脑卒中等,此类患者血糖高或正常,嗜酸性粒细胞数不增加,而本症表现为血糖低,嗜酸性粒细胞增加等可资鉴别。由于本病患者常有显著的消化道症状,因此也必须和常见的急腹症鉴别,如胃肠穿孔、急性胆囊炎、重型急性胰腺炎、肠梗阻等,若患者同时有血钾高、嗜酸性粒细胞增加和血、尿皮质醇减低,则提示有肾上腺危象的可能。仔细询问病史在鉴别诊断中是关键。

\subsection{治疗}

\paragraph{补充皮质激素}

即刻静注氢化可的松注射液或注射用氢化可的松琥珀酸钠100mg,使血皮质醇浓度达到正常人在发生严重应激时的水平。继以氢化可的松100~200mg溶入5\%葡萄糖氯化钠注射液500ml中静滴2~4小时,此后依病情每4~8小时继续静滴100mg,因氢化可的松在血浆中半减期为90分钟,故应持续静滴。头24小时内氢化可的松用量可达300~500mg。若静滴地塞米松或甲泼尼龙,应同时肌注去氧皮质酮2mg。危象控制后可逐渐减少,第2天用第1天的2/3量,第3天用第1天的1/2量。为了避免静滴液中断后激素不能及时补充,可在静滴的同时,肌注醋酸可的松(需在体内转化为氢化可的松才起作用)100mg,以后每12小时1次,病情好转后,应迅速减量,约每日减量50\%。当病情稳定能进食时,糖皮质激素改为口服,每6小时口服氢化可的松200mg或醋酸可的松25mg,约半月减至维持量。一般情况下,醋酸可的松25~75mg/d或泼尼松5~10mg/d即可。上午用全量的2/3,下午用1/3。如仍有低钠血症或收缩压不能回升至100mmHg,可考虑加用盐皮质激素如9α-氟氢可的松0.05~0.2mg/d口服,或肌注醋酸去氧皮质酮1~3mg,每日1~2次。

\paragraph{纠正水和电解质紊乱}

典型的危象患者液化损失量约达细胞外液的1/5。根据尿量、尿比重、血压、血细胞比积、心肺功能状况补充血容量,一般头24小时补液量在2500~3000ml以上,以5\%葡萄糖盐水为主,有显著低血糖时另加10\%~50\%葡萄糖液。若治疗前有高钾血症,当脱水和休克纠正,尿量增多,补充糖皮质激素和葡萄糖后,一般都能降至正常,在输入第3L液体时,可酌情补钾20~40mmol。本病可有酸中毒,但一般不需补碱,当CO\textsubscript{2}
CP < 9.9mmol/L(22vol\%)时,可补充适量碳酸氢钠。

\paragraph{抗休克}

如血压在80mmHg以下伴休克症状者经补液及激素治疗仍不能纠正循环衰竭时,应及早给予血管活性药物。

\paragraph{去除诱因与病因治疗}

包括原发病与抗感染治疗等。

\paragraph{对症治疗}

包括给氧、使用各种镇静、止惊剂,但禁用吗啡、巴比妥类药物。
\protect\hypertarget{text00117.html}{}{}

\hypertarget{text00117.htmlux5cux23CHP4-11-4}{}
参 考 文 献

1. 陈灏珠 ,林果为.实用内科学.第13版.北京:人民卫生出版社,2009:1202

2. 陆再英,钟南山.内科学.第7版.北京:人民卫生出版社,2008:744

\protect\hypertarget{text00118.html}{}{}

\chapter{嗜铬细胞瘤危象}

嗜铬细胞瘤(pheochromocytoma,PHEO)是起源于肾上腺髓质、交感神经节或其他部位的嗜铬组织,这种瘤持续或间断地释放大量儿茶酚胺,引起持续性或阵发性高血压和多个器官功能及代谢紊乱。约10\%为恶性肿瘤。本病以20~50岁最多见。

在一些诱因包括情绪激动、运动、挤压肿瘤部位、创伤、麻醉、插管、手术、分娩、滥用某些药物(如拟交感神经药、单胺氧化酶抑制剂、吗啡、箭毒类、组织胺释放剂、β受体阻滞剂、骤停氯压定等)、吸烟以及作诊断性激发试验等情况下,患者可出现嗜铬细胞瘤危象,是指肿瘤短期分泌较多的肾上腺素和去甲肾上腺素造成急性高儿茶酚胺血症。嗜铬细胞瘤危象的临床表现有:①高血压危象;②高血压与低血压交替发作危象;③发作性低血压或休克;④急性左心衰和肺水肿;⑤心绞痛、心肌梗死、心律失常;⑥腹痛、恶心、呕吐等消化系统症状。

\subsection{病因与发病机制}

根据WHO
2004年的肿瘤分类,嗜铬细胞瘤(pheochromocytoma,PHEO)是起源于肾上腺髓质且能分泌儿茶酚胺的肾上腺内副神经节瘤。PHEOs多起源于肾上腺,9\%~23\%来自肾上腺外嗜铬组织,归于副神经节瘤。肾上腺和肾上腺外交感神经副神经节瘤均可产生大量的儿茶酚胺,引起典型的临床症状。而副交感神经副神经节瘤很少分泌大量儿茶酚胺。

肾上腺外的嗜铬细胞瘤主要位于腹部,多在腹主动脉旁(约占10\%~15\%),其他少见部位为肾门、肾上极、肝门区、肝及下腔静脉之间、近胰头部位、髂窝或近髂窝血管处如卵巢内、膀胱内、直肠后等。腹外者甚少见,可位于胸内(主要在后纵隔或脊柱旁,也可在心脏内)、颈部、颅内。

有文献报道在高血压就诊患者中,约0.1\%~1\%为嗜铬细胞瘤患者,其发病率为年0.8/10万人,各个年龄段均可发病,以20~50岁多见,男女发病率相似。典型症状是阵发性高血压或持续性高血压阵发性加重、心悸和大汗。嗜铬细胞瘤所致恶性高血压是导致心力衰竭、心肌梗死、脑卒中和肾功能受损的重要危险因素。与原发性高血压患者相比,此类患者心、脑、肾等高血压靶器官的损害更为严重,所以早期诊断、早期治疗显得尤为重要。

肾上腺髓质的嗜铬细胞瘤可产生去甲肾上腺素(noradrenaline,NE)和肾上腺素(epinephrine,E)及少量多巴胺,而肾上腺外的嗜铬细胞瘤只产生NE(主动脉旁嗜铬体例外)。嗜铬细胞瘤发病主要取决于肿瘤细胞分泌的儿茶酚胺成分中是以NE还是E为主,以及肿瘤释放儿茶酚胺是暴发性还是持续性的,这两方面决定了嗜铬细胞瘤的发病方式和临床表现的多样性。嗜铬细胞瘤危象发作则是肿瘤在某种诱因刺激下,呈暴发性地大量释放儿茶酚胺入血所致。此外,嗜铬细胞瘤可产生多种肽类激素,其中一部分可能引起嗜铬细胞瘤中一些不典型的症状,如面部潮红(舒血管肠肽、P物质)、便秘(鸦片肽、生长抑素)、腹泻(血管活性肠肽、血清素、胃动素)、面色苍白、血管收缩(神经肽Y)及低血压或休克(舒血管肠肽、肾上腺髓质素)等。

\subsection{诊断}

\subsubsection{临床表现特点}

本病变化多端。以心血管症状为主,兼有其他系统的表现。

嗜铬细胞瘤的典型症状包括突发血压增高,同时伴有头痛(80\%)、大汗(70\%)及心悸(60\%),每一个患者往往有不同的症状特点。典型的三联征(头痛、大汗、心悸)诊断PHEOs的敏感性为90.9\%,特异性为93.8\%。然而,8\%的患者可以完全没有症状,通常是有家族史的患者,或较大的囊性肿瘤。典型的临床症状和体征常发生于良性PHEOs患者。

嗜铬细胞瘤危象临床表现可有以下类型:

\paragraph{高血压危象}

持续性或阵发性高血压可以出现在90\%~100\%的患者中,是最常见的临床症状。常表现为突发血压升高,可达到200~300/130~180mmHg,其发作可由情绪激动、体位改变、创伤、灌肠、大小便、腹部触诊、某些药物等促发。头痛常较剧烈,为突然发作的双侧搏动性头痛。心悸常伴有胸闷、憋气、胸部压榨感或濒死感。多汗常呈大汗淋漓,伴有面色苍白,四肢发凉。症状严重者,可因此出现高血压脑病和(或)脑血管病综合征,如脑出血、蛛网膜下腔出血等,此时可出现剧烈头痛、躁动、抽搐、呕吐、颈强直、意识丧失,甚至死亡。发作终止后,患者可出现迷走神经兴奋的症状,如潮红、发热、流涎,瞳孔缩小,尿量增多等。

\paragraph{高血压与低血压交替发作危象}

高低血压交替发作可能是由于肿瘤组织分泌大量儿茶酚胺致血压骤升,同时导致小静脉及毛细血管前小动脉强烈收缩,使组织缺血缺氧,血管通透性增加,血浆外渗,血容量减少;加上强烈收缩的小动脉对儿茶酚胺敏感性下降,使血压降低。血压下降又反射性的引起儿茶酚胺释放增加,导致血压再度升高,如此反复,临床上即表现为高血压和低血压交替出现,血压在短时间内有大幅度而频繁的波动,同时心动过速和心动过缓交替出现,伴有大汗淋漓、面色苍白、四肢厥冷等循环衰竭表现。这种严重的血流动力学改变易引起脑血管意外、急性心衰、心梗、休克等严重并发症,如不及时处理可导致死亡。

\paragraph{发作性低血压或休克}

发病机制有如下几点:①大量的儿茶酚胺导致心律失常或心衰,心排血量锐减;②大量儿茶酚胺使血管强烈收缩,组织缺氧、微血管通透性增加,血容量减少,致血压下降,严重者发生休克;③由于肿瘤内发生出血、坏死,使儿茶酚胺分泌骤然减少或停止,突然失去儿茶酚胺作用后,血管床突然扩张,有效循环血容量不足;④应用α受体阻滞剂如酚妥拉明后血管床突然扩张,血容量相对不足,以低血压或休克为突出表现,易发生直立性低血压危象。

\paragraph{急性左心衰 、肺水肿}

大量儿茶酚胺所致的儿茶酚胺心脏病,包括扩张型心肌病或肥厚性心肌病,心肌发生退行性变、坏死、炎症细胞灶、弥漫性心肌水肿及心肌纤维变性等,心电图常有心肌损伤、缺血、ST段及T波变化、房室传导阻滞、期前收缩或心动过速等心律失常。危象时更易发生心力衰竭(主要是急性左心衰竭、肺水肿)。

\paragraph{心绞痛 、心肌梗死、心律失常}

由于大量儿茶酚胺突然释放,使心脏突然受到刺激而使冠状动脉负荷增大,或因为发作性的低血压期冠状动脉供血不足,致心肌缺血缺氧发生心绞痛及心肌梗死。表现为胸痛或心电图改变,包括ST段抬高或压低,T波倒置,其他心电图异常可有期前收缩、阵发性心动过速,心室颤动等。

\paragraph{腹痛 、恶心、呕吐等消化系统症状}

因儿茶酚胺可松弛胃肠平滑肌,使肠道张力减弱,引起便秘甚至结肠扩张;儿茶酚胺还可使胃肠小动脉痉挛、缺血,胃肠功能抑制,而导致肠出血、坏死、穿孔;另外还可抑制胆囊收缩。患者表现为剧烈腹痛、呕吐、呕血、血便,严重者出现休克。

\subsubsection{实验室检查}

诊断PHEOs的最佳生化方法值得商榷,但结合不同的化验会增加诊断的敏感性和特异性。常用高效液相电化学检测仪或ELISA的方法测定血、尿儿茶酚胺即去甲肾上腺素(NE)、肾上腺素(E)、多巴胺(DA)及其代谢产物香草扁桃酸(VMA)、3-甲氧基肾上腺素(MN)和3-甲氧基去甲肾上腺素(NMN)的浓度。

\paragraph{血浆 CA(E、NE)和二羟苯丙醇(DHPG)测定}

血浆CA值在本病持续性或阵发性发作明显高于正常,但其测定值仅反映取血样即时的CA水平,NE的正常值<
500pg/ml 和E < 100pg/ml。若NE > 1500pg/ml和E >
300pg/ml,具诊断价值。同时测定NE和DHPG可提高嗜铬细胞瘤的诊断特异性,因为DHPG仅来自神经元,而不是从外周循环血中NE代谢而来,因此若仅有DHPE水平增高或血浆NE/DHPE的比值>
2.0,则提示为嗜铬细胞瘤,若该比值≤0.5则可排除。

\paragraph{尿中 CA(E + NE)、VMA、HVA、MN和NMN测定}

本病所引起的持续性高血压患者,尿中儿茶酚胺及其代谢物、香草基杏仁酸3-甲氧基-4-羟基-扁桃酸(VMA)、高香草酸(HVA)、甲氧基肾上腺素(MN)和甲氧基去甲肾上腺素NMN皆升高,常在正常高限的两倍以上。阵发性者,仅在发作后才高于正常。同时测定去甲肾上腺素和它的代谢物二羟苯丙醇(DHPG),可提高其诊断的特异性。尿CA(NE
+
E)正常呈昼夜节律,且在活动时排量增多,正常值为591~890nmol/d(100~150μg/d),其中80\%为NE,20\%为E。大多数嗜铬细胞瘤患者,尿CA明显增高,往往大于1500nmol/d(250μg/d),阵发性发作者,应收集高血压发作期尿液,然后与未发作期同样时间和条件下收集尿,所测之值做对照,如增高3倍以上,才有临床诊断价值。

\subsubsection{抑制试验}

适用于持续性高血压、阵发性高血压发作的嗜铬细胞瘤患者,主要用于与其他病因高血压或原发性高血压者作鉴别诊断。一般当血压≥170/110mmHg时可应用下列试验以进一步明确诊断。

\paragraph{酚妥拉明(regitine)试验}

酚妥拉明为短效α-肾上腺素能受体阻断剂,可阻断CA的作用,因此可用以判断高血压是否是因高水平CA所致。方法:患者先安静平卧20~30分钟,每2~5分钟测一次血压和心率。待其稳定后,静脉滴注生理盐水,待血压平稳并≥170/110mmHg时,快速静注酚妥拉明5mg,然后每30秒测血压和心率一次,至3分钟,以后每1分钟测一次至10分钟,于15分钟、20分钟各测一次血压及心率。如注射酚妥拉明后2~3分钟内血压较用药前降低35/25mmHg以上且持续3~5分钟或更长时间,则为阳性反应,提示嗜铬细胞瘤可能。如能注射酚妥拉明前、后各抽血观察CA水平改变,如与血压改变一致,更有利于诊断的确立。为尽可能防治出现假阴、阳性结果,应在试验前,停用3~7天的任何降压、镇静、安眠药物。

\paragraph{可乐定试验}

可乐定是中枢性α\textsubscript{2}
肾上腺素能激动剂,可减少神经元CA的释放。而并不抑制嗜铬细胞瘤的CA释放,故可作鉴别,此试验非常安全但仅适用于试验前原血浆CA异常升高者。方法:受试者,先安静平卧,静脉穿刺并固定针头以备抽血样,于30分钟时采血作CA测定(对照),然后口服0.3mg可乐定,然后,在服药后的1小时、2小时、3小时分别取血样测CA。在大多数非本病的高血压病者,血压可下降,原发性高血压者的原CA高者可抑制到正常范围(按此分割点,灵敏度为87\%,特异性为93\%)或抑制至少为原水平的50\%(按此分割点其灵敏度为97\%,特异性仅67\%)。而大多数嗜铬细胞瘤患者血浆CA水平不受抑制,即用药前后血浆CA水平相同或反而更高,但也存在少数的假阴性或假阳性病例,必要时可结合胰高糖素激发试验或重复进行。

\subsubsection{影像学检查}

本病诊断一旦确立,必须进一步确定肿瘤所在部位,因为90\%的嗜铬细胞瘤为良性,切除可获痊愈。

\paragraph{肾上腺 CT扫描}

为首选的无创伤性影像学检查,CT诊断定位嗜铬细胞瘤的灵敏度为85\%~98\%,但特异性仅70\%,故必须结合临床表现和生化改变和术后肿块的病理检查,有条件时作免疫组化作综合判断。

\paragraph{磁共振显像(MRI)}

近年应用渐多,因其可显示肿瘤与周围组织的解剖关系及某些组织和结构特征,有较高的诊断价值而且具有不需注射造影剂、无放射性损害,可用于孕妇等优点。其灵敏度为85\%~100\%,但特异性仅67\%。

\paragraph{B型超声检查}

为无创性,方便,易行,价低的检测方法,但灵敏度不如CT和MRI,不易发现较小的肿瘤。可对肾上腺外,如腹腔、膀胱、盆腔处是否有肿瘤作初步的筛查,并对肿瘤的质地如囊性还是实体瘤有较大的鉴别价值。但不易识别胸腔、纵隔等部位的肿瘤。

\paragraph{\textsuperscript{131} I-间碘苄胺(MIBG)闪烁扫描}

MIBG因其结构与NE类似,是一种肾上腺能神经阻滞剂,故放射性核素标记的MIBG可被肾上腺素能囊泡所吸收,浓集,在其闪烁扫描时,既可显示分泌儿茶酚胺的肿瘤和转移病灶,也可显示其他的神经内分泌瘤,但对低功能的肿瘤现象较弱。其诊断的灵敏度为78\%~83\%,特异性却为100\%,既有定位又有定性诊断价值。

\subsubsection{诊断注意事项}

本病症状典型者诊断并不困难。但症状不典型者,易造成误诊。对以下患者需注意排除嗜铬细胞瘤危象:①高血压危象和脑病发生在年轻人,伴消瘦、心动过速、大汗和震颤者。②反复发生急性肺水肿,特别是非心源性肺水肿者。③反复发生急性左心衰而且用强心利尿剂不能缓解者。④高血压和低血压交替发生,或一般剂量的降压药即引起明显的低血压休克者。⑤不明原因的突发剧烈腹痛而无腹部体征者等。

\subsection{治疗}

嗜铬细胞瘤危象治疗时应注意以下几点:

1.首先应抬高床头,立即静脉注射酚妥拉明1~5mg,密切观察血压,当血压降至160/100mmHg左右时停止推注,继之以10~15mg溶于5\%葡萄糖生理盐水500ml中缓慢滴注。1~2分钟无效者根据血压情况可反复推注酚妥拉明1~5mg,直至危象可控制。由于酚妥拉明作用短暂(不超过30分钟),所以有人主张滴注硝普钠以维持降压,但有肾功能损害者要警惕氰化物中毒。

2.心律失常者根据其性质选用适当药物,最常见为频繁期前收缩及阵发性心动过速,可在应用α受体阻滞剂的基础上加用β受体阻滞剂,严重心律失常者常用普萘洛尔1~5mg缓慢推注(每分钟推注0.5~1mg)。血压波动大者为防非选择性β受体阻滞剂促进内源性儿茶酚胺释放,可用心脏选择性β受体阻滞剂,如氨酰心安和美托洛尔,对心动过速和心律失常有特效而很少引起血压波动。室性心律失常推注利多卡因50~100mg,继以静脉滴注每分钟1~2.5mg。

3.低血压
、休克的治疗须分析具体情况选用适当措施。血管过度收缩和血容量不足者应静脉滴注酚妥拉明并扩容(补林格液)。因心律失常引起者要纠正心律失常,为防血压骤升和血容量不足,同时也需滴注酚妥拉明及补充血容量。

4.嗜铬细胞瘤并危象患者一旦确诊,必须尽早实施手术,但风险性很大。因此,充分的术前准备极为重要,控制高血压、心律失常,纠正低血压休克、急性左心衰,维持水、电解质平衡,保持代谢正常是进行手术的基础,术中严密观察病情变化,术后精心护理,是手术成功的保证。
\protect\hypertarget{text00119.html}{}{}

\hypertarget{text00119.htmlux5cux23CHP4-12-4}{}
参 考 文 献

1.
曾正陪.嗜铬细胞瘤的诊断及其发病机制研究.中华内分泌代谢杂志,2005,21(5):395-397.

2. 潘东亮
,李汉忠,罗爱伦,等.嗜铬细胞瘤诊治50年回顾总结.中华泌尿外科杂志,2005,26(11):725-727.

3. Kizer JR,Koniaris LS,Edelman JD,et al. Pheochromocytoma
crisis,cardiomyopathy,and hemodynamic collapse.
Chest,2000,118(4):1221-1223.

4. Kearney T,Dang C. Diabetic and endocrine emergencies. Postgrad Med
J,2007,83(976):79-86.

\protect\hypertarget{text00120.html}{}{}

\chapter{低血糖危象}

正常情况下,循环血浆中葡萄糖的浓度通过一个复杂而相互联系的神经、体液和细胞调节系统维持在3.9~7.8mmol/L(70~140mg/dl),这是个相对狭窄的范围。当某些病理或生理原因导致非糖尿病患者血糖≤2.8mmol/L
(50mg/dl)、接受药物治疗的糖尿病患者血糖≤3.9mmol/L
(70mg/dl)而引起交感神经兴奋和中枢神经异常甚至意识障碍的症状及体征时,称为低血糖危象。低血糖危象临床表现多样,与血糖下降速度、程度和持续时间等相关。持续严重的低血糖可以导致患者脑细胞产生不可逆损害,甚至死亡,因此不管什么原因引起的低血糖危象均需紧急处理。

\subsection{病因与发病机制}

引起低血糖的原因很多,按其发生与进食的关系可分为空腹低血糖和餐后低血糖;按其进展速度可分为急性、亚急性和慢性低血糖;按症状可分为症状性低血糖和无症状性低血糖;按病因可以分为器质性、功能性及外源性低血糖;这些分类方法之间有一定的内在联系和交叉。就低血糖危象而言,依空腹和餐后低血糖来分类有助于指导诊断。

\subsubsection{空腹低血糖}

\hypertarget{text00120.htmlux5cux23CHP4-13-1-1-1}{}
(一) 血糖利用过多

\paragraph{存在高胰岛素血症}

常见于:①口服降糖药物,以胰岛素促泌剂最常发生,如:格列本脲(优降糖)、消渴丸(含优降糖)、D860、格列美脲、格列齐特、格列吡嗪、格列喹酮等。二甲双胍与胰岛素或促胰岛素分泌剂联合使用时可增加低血糖发生的危险性,特别是在老年患者和肝、肾功能不全、药量过大者更多见,甚至出现难治性低血糖。②胰岛β细胞瘤、异位胰岛素分泌瘤、胰岛素自身免疫综合征(IAS)及注射胰岛素等均可因内生或外源性胰岛素过多导致低血糖。③氯喹、奎尼丁、奎宁等可延缓胰岛素的降解,在血中胰岛素浓度升高从而加强其降血糖作用。④糖尿病母亲妊娠时,胎儿由于连续得到葡萄糖的过量供给,其胰腺细胞会受到高血糖的刺激而显著增生,出生后因失去母亲供给的葡萄糖,在自身的血糖调节机制未完全发挥作用之前极易发生低血糖。

\paragraph{正常血浆胰岛素浓度}

多见于:①胰外肿瘤:如胸腹腔肿瘤(纤维肉瘤、间皮瘤、黏液瘤)、胆管癌、肾上腺皮质癌、肾胚胎瘤、淋巴瘤、肝癌、胃肠癌及肺癌、卵巢癌等,这些癌肿可能分泌胰岛素样生长因子-Ⅰ、Ⅱ(plasma
insulin like growth
factorⅠ、Ⅱ,IGF-Ⅰ、IGF-Ⅱ)致使产生血糖。②对胰岛素过度敏感:见于Addison病、甲状腺功能低下、垂体前叶功能减低等。③全身性肉毒碱缺乏、脂肪氧化酶缺乏、3-羟基-3甲基戊二酸-COA分解酶缺乏等均可因影响糖代谢而导致低血糖。

\hypertarget{text00120.htmlux5cux23CHP4-13-1-1-2}{}
(二) 血糖生成不足

1.升糖激素缺乏
垂体功能减低、肾上腺功能不全、胰高血糖素缺乏等情况时,可因应激时升糖作用不足而导致严重的低血糖。

2.先天性葡萄糖酶缺乏
肝糖原累积症(Ⅰ、Ⅳ、Ⅵ、Ⅸ型)、半乳糖血症、遗传性果糖不耐受症、家族性半乳糖-果糖不耐受症、果糖1,6二磷酸缺乏症、儿童酮症性低血糖等。

3.肝脏疾病
如肝淤血、严重肝炎、肝硬化、急性黄色肝萎缩时,肝脏在血糖调节的作用缺陷,易发生低血糖。

4.药物
除了胰岛素和磺脲类药物外,酒精最常见。酒精可在空腹一定时间后将糖原储存耗尽,大量饮酒可因为肝糖原消耗以及糖原异生减少的缘故引起严重的低血糖。水杨酸是其次最常涉及的药物,它在与降糖药物联用时会使血药浓度增大,同时也有降糖作用,可以导致低血糖昏迷。此外,奎宁、β受体阻断剂、吲哚美辛和保泰松等也会导致血糖过低。

5.严重的营养不良
如小肠吸收不良综合征、克罗恩病、慢性肠炎、尿毒症、饥饿性营养不良症等。

6.新生儿因糖原储备不足或消耗过多、糖异生能力低下极易发生低血糖。

\subsubsection{餐后(反应性)低血糖}

\hypertarget{text00120.htmlux5cux23CHP4-13-1-2-1}{}
(一) 血糖利用过多

1.餐后营养性高胰岛素血症
包括:①胃大部切除术后低血糖(滋养性低血糖);由于胃迅速排空致使葡萄糖吸收加速,胰岛素反应性分泌增加。而葡萄糖的下降较胰岛素的下降更快,导致葡萄糖-胰岛素的不平衡而发生低血糖。②早期糖尿病反应性低血糖;糖尿病早期的胰岛β细胞分泌呈第一时相反应迟钝、第二时相高峰延迟的特点,致使在进食4~5小时出现低血糖。

2.特发性功能性低血糖症
由于迷走神经兴奋性增高,导致餐后3~4小时出现的低血糖反应。临床表现以肾上腺素分泌过多综合征为主。

3.由于降糖药物剂量偏大或用药后未进食所致。

4.亮氨酸过敏
亮氨酸对胰岛素分泌有很强的刺激作用。对亮氨酸过敏是导致婴幼儿低血糖的重要原因。

\hypertarget{text00120.htmlux5cux23CHP4-13-1-2-2}{}
(二) 血糖生成不足

1.慢性脏器功能不全及伴有自主神经病变的糖尿病患者,由于对低血糖反应的应激性下降造成血糖生成不足。

2.先天性糖代谢酶缺乏 如先天性果糖不耐受症、半乳糖血症。

\subsubsection{病理生理}

脑细胞所需要的能量几乎完全来自血液中的葡萄糖。当血糖下降至2.8~3.0mmol/L(50~55mg/dl)时,胰岛素分泌受抑制,升糖激素(胰升糖素、肾上腺素、生长激素和糖皮质激素)分泌增加,出现交感神经兴奋症状。血糖下降至2.5~2.8mmol/L(45~50mg/dl)时,大脑皮层受抑制,继而皮层下中枢包括基底节、下丘脑及自主神经中枢相继累及,最后延髓活动受影响。低血糖纠正后,按上述顺序逆向恢复。

\subsection{诊断}

\subsubsection{临床表现特点}

低血糖症的临床表现是非特异的,个体间差异也较大,并与低血糖的程度、患者的年龄、血糖下降的速度及持续的时间有关。临床表现多分为交感神经过度兴奋与脑功能障碍两个阶段。若无第1阶段即进入第2阶段且很快昏迷者,多为糖尿病患者注入过多的胰岛素所致。低血糖时先发生交感神经系统兴奋性增高反应的称之为急性神经性低血糖,主要见于血糖迅速降到域值时。该值在健康人为2.8mmol/L,新生儿为1.70mmol/L,接受药物治疗的糖尿病患者只要血糖水平≤3.9mmol/L就属低血糖范畴。继发于慢性器质性或代谢疾患的低血糖症状,常在不知不觉中出现,称为亚急性或慢性低血糖,表现为以大脑损害为主的中枢神经系统病症,这类患者的前驱症状不明显。总的来说,低血糖危象临床症状表现在交感神经兴奋症状和脑功能障碍两个阶段:

\paragraph{交感神经兴奋的表现}

低血糖发生后刺激肾上腺素分泌增多,出现周身乏力、出冷汗、皮肤苍白、心悸、饥饿感、四肢发凉、手颤动,此时神志尚清,若不能及时补充糖分,则进一步发展为第二阶段的脑功能障碍的表现。

\paragraph{脑功能障碍的表现}

低血糖若未得到及时纠正,可进一步出现意识障碍、精神神经症状、癫痫样表现。①大脑皮质功能受抑制:患者意识模糊,定向力及识别力明显减退,嗜睡、多汗、震颤、神志不清及语言障碍。②皮层下中枢受抑制:患者神志不清,躁动不安,痛觉过敏、阵挛性舞蹈动作、瞳孔散大,强制性抽搐及锥体束征阳性,神志不清。③大脑、中脑受累:患者肌张力下降,出现癫痫样抽搐。④脑干受损:表现为去大脑强直于心动过缓,体温不升及各种反射消失。严重而持久的低血糖也可成为永久的脑功能损害,如:偏瘫、截瘫、失语、共济失调、面神经麻痹、视野缺损等。

值得注意的是,老年人发生低血糖症状多不典型,经常容易误诊,尤其是昏迷、抽搐伴偏瘫为首发症状的低血糖现象,是一种暂时性偏瘫,常伴有意识不清,与脑卒中很相似。

\subsubsection{实验室检查}

\paragraph{血糖测定}

临床上一般用静脉血浆葡萄糖浓度表示血糖水平。多低于2.8mmol/L,但长期高血糖的糖尿病患者血糖突然下降时,虽然血糖高于此水平仍会出现低血糖反应的症状。

\paragraph{延长(5小时)口服葡萄糖耐量试验(OGTT)}

多用于餐后低血糖的诊断。不同原因的低血糖有不同的糖耐量曲线(表\ref{tab48-1})。

\begin{table}[htbp]
\centering
\caption{各种低血糖症糖耐量试验曲线的特点}
\label{tab48-1}
\includegraphics[width=3.27083in,height=1.84375in]{./images/Image00164.jpg}
\end{table}

\paragraph{血清胰岛素 、C肽、胰岛素原测定}

胰岛素测定在低血糖症的诊断中非常重要,尤其对内源性胰岛素分泌过多引起低血糖的诊断。由于胰岛素升高尚见于胰岛素抵抗、妊娠后期等。所以判断时必须结合同时测定的血糖值。①胰岛素释放指数:血浆免疫反应性胰岛素(μU/ml)与同时测定的血糖值(mg/dl)之比,正常<
0.3或> 0.4为异常,胰岛β细胞瘤的患者常>
1。②胰岛素释放修正指数:血浆胰岛素× 100/(血浆葡萄糖− 30mg/dl)。该值<
50为正常,> 80提示胰岛β细胞瘤。③低血糖时胰岛素测定值:放射免疫法>
6μU/ml或免疫化学发光法(ICMA)>
3μU/ml提示低血糖为胰岛素分泌过多所致。胰岛β细胞瘤患者的胰岛素水平很少超过100μU/ml(放免法),如超过1000mU/L提示为外源性胰岛素或存在胰岛素抗体。④C肽>
200pmol/L (ICMA)或胰岛素原>
5pmol/L(ICMA)也可以诊断为胰岛素分泌过多。

\paragraph{禁食试验}

正常人饥饿72小时血糖下降不< 3.1mmol/L,胰岛素不<
10μU/ml,而90\%胰岛β细胞瘤患者饥饿24小时内即有低血糖发作,发作时血糖<
2.5mmol/L而胰岛素水平不降,因而计算胰岛素释放指数增加。这项试验需在医生监护下进行,一旦出现低血糖症状,立即取血分测血糖和胰岛素,同时给患者注射葡萄糖或进食以终止试验。注意一次饥饿试验阴性不能完全排除胰岛素瘤。

\paragraph{激发试验}

是用药物加强和延长胰岛素分泌的刺激试验方法,有助于判断是否存在内源性胰岛素分泌过多。由于在胰岛β细胞瘤患者刺激试验有引起严重低血糖的危险,且有的患者无反应,临床诊断价值较低。临床多用于发作次数少、症状不典型及禁食试验无反应的疑难病例:

\hypertarget{text00120.htmlux5cux23CHP4-13-2-2-5-1}{}
(1) 甲苯磺丁脲刺激试验:

用甲苯磺丁脲钠1.0g溶于20ml注射用水中静脉推注,注射时间不能短于2分钟。注射后每5分钟测定一次血清胰岛素,持续15分钟,若胰岛素水平超过195μU/ml提示有胰岛素瘤存在。试验持续超过15分钟将会发现那些胰岛素延迟升高的胰岛素瘤患者,但可能会引起严重的低血糖。

\hypertarget{text00120.htmlux5cux23CHP4-13-2-2-5-2}{}
(2) 胰高血糖素刺激试验:

静脉注入1mg胰高血糖素,每5分钟测定一次血清胰岛素,连续15分钟。胰岛素水平超过135μU/ml提示有胰岛素瘤。当血清胰岛素过高时,胰高血糖素的升血糖作用作用可能较平常弱,但60分钟后可能接着出现严重的低血糖。

\paragraph{激素测定}

若由内分泌疾患引起的低血糖,根据不同的原因可测定生长激素、皮质醇、甲状腺激素、肾上腺素、性激素和IGF-Ⅰ、IGF-Ⅱ等。

\subsubsection{影像学检查}

影像学检查是定位诊断的主要手段,包括胰腺B超、CT、磁共振、选择性腹腔动脉血管造影、经皮肝门静脉插管测定激素法(PTPC)、选择性动脉钙刺激联合肝静脉取样、术前腹腔B超、内镜B超、术中B超等。世界范围报告显示各项定位检查的成功率为:CT
24\%,术前超声40\%,动脉造影43\%,经皮肝门静脉插管测定激素法(PTPC)88\%。结合检查费用和成功率,推荐首选超声检查,尤其是内镜B超或术中B超。因多数肿瘤太小(80\%<
2cm),故影像检查的阴性结果并不能排除诊断。有观点认为过多的术前定位检查并不必要,只要定性诊断明确,可直接由有经验的外科医生进行剖腹探查术,术中仔细探查配合IOUS(intraoperative
ultrasound)------“扪诊配合术中超声定位诊断”,可使肿瘤检出率达到95\%~100\%。

\subsubsection{诊断注意事项}

1.尚未确诊的低血糖昏迷
,应排除AEIOUH六大类昏迷原因,即:A.脑血管病;E.癫痫;I.感染;O.中毒;U.尿毒症;H.中暑。

2.已确诊低血糖症者,应与不同病因所致的低血糖症相鉴别。

(1)
糖尿病早期反应性低血糖:多在进食后3~5小时出现低血糖。患者多超重或肥胖,可有糖尿病家族史。5小时OGTT显示空腹血糖高于正常,服糖后0.5小时、1小时、2小时血糖升高,3~5小时可出现低血糖,血浆胰岛素高峰往往迟于血糖高峰。

(2) 特发性功能性低血糖症:为发生于餐后2~4小时或OGTT
2~5小时的暂时性低血糖。多见于女性,临床表现以肾上腺素分泌过多综合征为主,患者感心悸、心慌、出汗、面色苍白、饥饿、软弱无力、手足震颤、血压偏高等。一般无昏迷或抽搐,偶有昏厥、午餐及晚餐后较少出现。每次发作约15~20分钟,可自行缓解,病情非进行性发展,空腹血糖正常,发作时血糖可以正常或低至2.8mmol/L
(50mg/dl),但不会更低。口服葡萄糖耐量试验,在服糖后2~4小时其血糖可下降至过低值,然后恢复至空腹时水平。患者能耐受72小时禁食。没有胰岛素分泌过多的证据。糖尿病家族史常缺如。

(3)
肝源性低血糖:多有严重的肝脏疾患,肝功能异常。主要为空腹低血糖,饥饿、运动等可诱发出现,病情呈进行性。餐后可有高血糖。OGTT显示空腹血糖偏低,服糖后2小时血糖偏高,至3~5小时可能出现低血糖。

(4)
胰岛β细胞瘤:可见于任何年龄,女性约占60\%;起病缓慢,反复发作,进行性加重。多在清晨、饥饿及运动时发作低血糖,发作时血糖很低。OGTT呈低平曲线,血清胰岛素、C肽、胰岛素原浓度明显升高;禁食试验及激发试验均呈阳性反应。

(5)
酒精性低血糖:患者常有慢性肝病病史,在大量饮酒,尤其是空腹饮酒后出现低血糖;低血糖的临床表现常被醉酒状态所掩盖。没有胰岛素过多的证据;可伴有代谢性酸中毒、酮尿或酮血症。

(6)
胰外肿瘤:临床低血糖发作典型,多为空腹低血糖,有胰外肿瘤的依据、症状及体征。没有胰岛素分泌过多的证据,如血中IGF-Ⅱ增加有助于诊断。

(7)
胰岛素自身免疫综合征(IAS):常与其他自身免疫性疾病同时存在。实验室检查可发现低血糖的同时存在内源性胰岛素分泌过多的证据,血清中胰岛素自身抗体(IAA)阳性,少数可查出胰岛素受体抗体。

应注意低血糖危象,若以脑缺糖而表现为脑功能障碍为主者,可误诊为精神病、神经疾患(癫痫、短暂脑缺血发作)或脑卒中等。

3.确定低血糖危象 可依据Whipple三联征确定:①低血糖症状;②发作时血糖<
2.8mmol/L(50mg/dl);③补充葡萄糖后低血糖症状迅速缓解。少数空腹血糖降低不明显或处于非发作期的患者,应多次检测有无空腹或吸收后低血糖,必要时采用48~72小时禁食试验。

4.临床常用的词“低血糖反应”指有与低血糖相应的症状体征(主要是交感神经兴奋的表现),但血糖未低于2.8mmol/L,常见于药物治疗的糖尿病患者。“低血糖”则是一个生化诊断,指血糖低于2.8mmol/L的情况,往往伴有临床症状,无症状者称为“无症状低血糖”。患者没有自觉的前驱症状直接进入意识障碍状态者为“未察觉的低血糖症”(hypoglycemia
unawareness)。

\subsection{治疗}

低血糖危象为内科急症,如持续时间过长可使脑细胞不可逆损害以致脑死亡。因此应尽可能使血糖迅速恢复正常水平,防止低血糖的反复发作。

\subsubsection{急诊处理}

1.已明确低血糖危象而神志尚未完全丧失者,可给口服葡萄糖15~20g,同时采血测血糖浓度,每15分钟监测一次。

2.病情严重、神志不清者,立即静脉输入50\%葡萄糖20~60ml,通常10~15分钟后患者意识可以恢复。如效果不明显,可肌注胰高血糖素(成人1mg、儿童0.5mg),通常在10分钟内血糖即可升高。此后持续静脉滴注5\%~10\%葡萄糖液,根据病情调节葡萄糖液体量。血糖水平监测须追踪至少24~48小时。

3.如血糖恢复并维持正常水平后
,昏迷持续超过30分钟者,需考虑有脑水肿的可能。应加用20\%甘露醇液200ml或地塞米松10mg,根据情况再增减用量。

4.对胰岛素分泌过多所致的低血糖症可选择二氮嗪(氯甲苯噻嗪),其药理作用相似于氯噻嗪,但无利尿作用,有抑制胰岛素分泌作用,半衰期20~30小时。成人剂量200~600mg/d,儿童5~3mg/d。

\subsubsection{对症处理}

加强昏迷护理,对行为异常者要加强保护,以免出现意外,神志不清者可酌情加用抗生素,减少感染。垂体前叶功能低下或甲状腺功能低下引起的低血糖,应给予静滴氢化可的松或服用甲状腺素片。

\subsubsection{长期反复发作的低血糖}

此类患者的低血糖多为胰岛素瘤所致,应做手术切除;手术有禁忌证、拒绝手术以及术后未缓解或复发者,可服二氮嗪100~200mg/d,分2~3次服,与利尿剂合用可防止水潴留副作用。不能切除或已有转移的胰岛细胞癌,可用链脲佐菌素(streptozotocin),50\%的患者可缓解或延长存活时间。药物治疗同时应注意增加餐次,多吃含糖多脂的食物,必要时加用肾上腺皮质激素以防低血糖发作。胰岛素自身免疫综合征所引起的低血糖症,可服用泼尼松治疗。
\protect\hypertarget{text00121.html}{}{}

\chapter{糖尿病危象}

糖尿病(diabetes
mellitus,DM)是一组常见的以血糖水平增高为特征的代谢内分泌疾病,其基本病理生理变化是胰岛素绝对或相对分泌不足和胰高血糖素活性增高所引起的代谢紊乱,包括糖、蛋白质、脂肪、水及电解质等,严重时常导致酸碱平衡失常;其特征为高血糖、糖尿、葡萄糖耐量减低及胰岛素释放试验异常。临床上早期无症状,至症状期才有多食、多饮、多尿、烦渴、善饥、消瘦或肥胖、疲乏无力等症群,久病者常伴发心脑血管、肾、眼及神经等病变。严重病例或应激时可发生酮症酸中毒、高血糖高渗状态和乳酸性酸中毒等急性危象,若抢救及时,一般可以逆转,若延误诊治,死亡率均较高。因此,及早识别和诊断、正确处理这三类糖尿病危象是十分重要的。

糖尿病诊断是基于空腹血糖(FPG)、任意时间或口服葡萄糖耐量试验(OGTT)中2小时血糖值(2小时PG)。空腹指8~10小时内无任何热量摄入。任意时间指1日内任何时间,无论上一次进餐时间及食物摄入量。OGTT采用75g无水葡萄糖负荷。糖尿病症状指多尿、频渴多饮和难于解释的体重减轻。FPG
3.9~6.0mmol/L(70~108mg/dl)为正常;6.1~6.9mmol/L(110~125mg/dl)为空腹血糖调节受损(IFG);≥7.0mmol/L(126mg/dl)应考虑糖尿病。OGTT
2小时PG <
7.7mmol/L(139mg/dl)为正常糖耐量;7.8~11.0mmol/L(140~199mg/dl)为糖耐量减低(IGT);≥11.1mmol/L(200mg/dl)应考虑糖尿病。WHO(1999年)提出的并被我国糖尿病学会采纳的糖尿病诊断标准为:糖尿病症状加任意时间血浆葡萄糖≥11.1mmol/L(200mg/dl);或空腹血浆葡萄糖(FPG)≥7.0mmol/L(126mg/dl);或葡萄糖耐量试验(OGTT)中,2小时血浆葡萄糖≥11.1mmol/L
(200mg/dl)。

对于无糖尿病症状、仅一次血糖值达到糖尿病诊断标准者,必须在另一天复查核实而确定诊断。如复查结果未达到糖尿病诊断标准,应定期复查。IFG或IGT的诊断应根据3个月内的两次OGTT结果,用其平均值来判断。在急性感染、创伤或各种应激情况下可出现血糖暂时升高,不能以此诊断为糖尿病,应追踪随访。

关于糖尿病分型,目前国际上通用WHO糖尿病专家委员会提出的病因学分型标准(1999)。新的分类法将糖尿病分成四大类型,即1型糖尿病(T1DM)、2型糖尿病(T2DM)、其他特殊类型糖尿病和妊娠期糖尿病(GDM)。并取消胰岛素依赖型糖尿病(IDDM)和非胰岛素依赖型糖尿病(NIDDM)的医学术语;保留1、2型糖尿病的名称,用阿拉伯字,不用罗马字;糖耐量减低不作为一个亚型,而是糖尿病发展过程中的一个阶段;取消营养不良相关糖尿病。

\section{糖尿病酮症酸中毒}

糖尿病酮症酸中毒(diabetic
ketoacidosis,简称DKA)是由于体内胰岛素活性重度缺乏及升糖激素不适当增高,引起糖、脂肪和蛋白质代谢紊乱,以致水、电解质和酸碱平衡失调,出现高血糖、酮症,代谢性酸中毒和脱水为主要表现的临床综合征。是糖尿病的急性并发症,也是内科常见危象之一。当糖尿病代谢紊乱发展至严重阶段,脂肪分解加速,血清酮体积聚超过正常水平时称为酮血症,尿酮排出增多称为酮尿,此时临床表现统称酮症。酮体中酸基增多,大量消耗体内储备碱,而发生代谢性酸中毒,称为DKA;如病情严重发生昏迷时则称为糖尿病昏迷。

\subsection{病因与发病机制}

DKA的发生与糖尿病类型有关,与病程无关,约20\%以上新诊断的1型糖尿病和部分2型糖尿病患者可出现DKA。1型糖尿病有发生DKA的倾向,2型糖尿病在一定诱因下也可发生。在有的糖尿病患者,可以DKA为首发表现。DKA的临床发病大多有诱发因素,这些诱因多与加重机体对胰岛素的需要有关。常见的诱因有:①感染:是DKA最常见的诱因。常见有急性上呼吸道感染、肺炎、化脓性皮肤感染,胃肠道感染,如急性胃肠炎、急性胰腺炎、胆囊炎、胆管炎、腹膜炎等,以及泌尿道感染。②降糖药物应用不规范:由于体重增加、低血糖、患者依从性差等因素致使注射胰岛素的糖尿病患者,突然减量或终止治疗;或在发生急性伴发疾病的状态下,没有及时增加胰岛素剂量。③外伤、手术、麻醉、急性心肌梗死、心力衰竭、精神紧张或严重刺激引起应激状态等。④饮食失调或胃肠疾患,尤其是伴严重呕吐、腹泻、厌食、高热等导致严重失水和进食不足时,若此时胰岛素用量不足或中断、减量时更易发生。⑤妊娠和分娩。⑥胰岛素抗药性:由于受体和信号传递异常引起的胰岛素不敏感或产生胰岛素抗体,均可导致胰岛素的疗效降低。⑦伴有拮抗胰岛素的激素分泌过多,如肢端肥大症、皮质醇增多症或大量应用糖皮质激素、胰高血糖素、拟交感神经活性药物等。⑧糖尿病未控制或病情加重等。

胰岛素活性的重度或绝对缺乏和升糖激素过多(如胰高血糖素、儿茶酚胺类、皮质醇和生长激素)是DKA发病的主要原因。胰岛素缺乏和胰高血糖素升高是DKA发展的基本因素。胰岛素和胰高血糖素比率下降促进糖异生、糖原分解和肝酮体生成,肝的酶作用底物(游离脂肪酸、氨基酸)产生增加,导致高血糖、酮症和酸中毒。

\paragraph{酮症和酸中毒}

酮体包括β-羟丁酸、乙酰乙酸和丙酮。糖尿病加重时,胰岛素绝对缺乏,三大代谢紊乱,不但血糖明显升高,而且脂肪分解增加,脂肪酸在肝脏经β氧化产生大量乙酰辅酶A,由于糖代谢紊乱、草酰乙酸不足,乙酰辅酶A不能进入三羟酸循环氧化供能而缩合成酮体;同时由于蛋白合成减少,分解增加,血中成糖、成酮氨基酸均增加,使血糖、血酮进一步升高。β-羟丁酸、乙酰乙酸以及蛋白质分解产生的有机酸增加,循环衰竭、肾脏排出酸性代谢产物减少导致酸中毒。酸中毒可使胰岛素敏感性降低;组织分解增加,K\textsuperscript{+}
从细胞内逸出;抑制组织氧利用和能量代谢。严重酸中毒使微循环功能恶化,降低心肌收缩力,导致低体温和低血压。当血pH降至7.2以下时,刺激呼吸中枢引起呼吸加深加快;低至7.1~7.0时,可抑制呼吸中枢和中枢神经功能、诱发心律失常。

\paragraph{失水}

严重高血糖、高血酮和各种酸性代谢产物引起渗透压性利尿,大量酮体从肺排出又带走大量水分,厌食、恶心、呕吐使水分入量减少,从而引起细胞外失水;血浆渗透压增加,水从细胞内向细胞外转移引起细胞内失水。

\paragraph{电解质平衡紊乱}

渗透性利尿同时使钠、钾、氯、磷酸根等大量丢失,厌食、恶心、呕吐使电解质摄入减少,引起电解质代谢紊乱。胰岛素作用不足,物质分解增加、合成减少,钾离子(K\textsuperscript{+}
)从细胞内逸出导致细胞内失钾。由于血液浓缩、肾功能减退时K\textsuperscript{+}
滞留以及K\textsuperscript{+}
从细胞内转移到细胞外,因此血钾浓度可正常甚或增高,掩盖体内严重缺钾。随着治疗过程中补充血容量(稀释作用),尿量增加、K\textsuperscript{+}
排出增加,以及纠正酸中毒及应用胰岛素使K\textsuperscript{+}
转入细胞内,可发生严重低血钾,诱发心律失常,甚至心搏骤停。

\paragraph{携带氧系统失常}

红细胞向组织供氧的能力与血红蛋白和氧的亲和力有关,可由血氧离解曲线来反映。DKA时红细胞糖化血红蛋白(GHb)增加以及2,3二磷酸甘油酸(2,3-DPG)减少,使血红蛋白与氧亲和力增高,血氧离解曲线左移。酸中毒时,血氧离解曲线右移,释放氧增加(Bohr效应),起代偿作用。若纠正酸中毒过快,失去这一代偿作用,而血GHb仍高,2,3-DPG仍低,可使组织缺氧加重,引起脏器功能紊乱,尤以脑缺氧加重、导致脑水肿最为重要。

\paragraph{周围循环衰竭和肾功能障碍}

严重失水,血容量减少和微循环障碍未能及时纠正,可导致低血容量性休克。肾灌注量减少引起少尿或无尿,严重者发生急性肾衰竭。

\paragraph{中枢神经功能障碍}

严重酸中毒、失水、缺氧、体循环及微循环障碍可导致脑细胞失水或水肿、中枢神经功能障碍。此外,治疗不当如纠正酸中毒时给予碳酸氢钠不当导致反常性脑脊液酸中毒加重,血糖下降过快或输液过多过快、渗透压不平衡可引起继发性脑水肿并加重中枢神经功能障碍。

\subsection{诊断}

\subsubsection{病史与诱因}

有糖尿病病史或家族史,以及上述发病诱因。

\subsubsection{临床表现特点}

患者在出现明显DKA前,原有糖尿病症状加重如口渴、多饮、多尿、疲倦加重,并迅速出现食欲不振、恶心、呕吐、极度口渴、尿量剧增;常伴有头痛、嗜睡、烦躁、呼吸深快,呼气中含有烂苹果味。后期呈严重失水、尿量减少、皮肤干燥、弹性差、眼球下陷、脉细速、血压下降、四肢厥冷、反射迟钝或消失,终至昏迷。

由于DKA时心肌收缩力减弱、心搏出量减少,加以周围血管扩张、严重脱水、血压下降、周围循环衰竭。年长而有冠心病者可并发心绞痛、心肌梗死、心律不齐或心力衰竭等。

少数病例表现为腹痛(呈弥漫性腹痛),有的相当剧烈,可伴腹肌紧张、肠鸣音减弱或消失,极易误诊为急腹症。腹痛可能由于胸下部和上腹部辅助呼吸肌痉挛或因缺钾导致胃扩张和麻痹性肠梗阻所致;也可因肝脏迅速增大、DKA毒性产物刺激腹腔神经丛以及合并胰腺炎等所致;老年糖尿病患者出现腹痛和腹部体征时还应考虑与动脉硬化引起的缺血性肠病有关。

根据酸中毒的程度,可以将DKA分为轻度、中度和重度。轻度是指只有酮症,无酸中毒(糖尿病酮症);中度是指除酮症外,伴有轻~中度酸中毒(DKA);重度是指DKA伴意识障碍,或虽无意识障碍,但CO\textsubscript{2}
CP < 10mmol/L者。

\subsubsection{实验室检查}

\paragraph{血糖与尿糖}

血糖波动在11.2~112mmol/L(200~2000mg/dl),多数为16.7~33.3mmol/L(300~600mg/dl),有时可达55.5mmol/L(1000mg/dl)以上。如超过33.3mmol/L,应考虑同时伴有高血糖高渗状态或有肾功能障碍。尿糖强阳性,当肾糖阈升高时,尿糖减少甚至阴性。可有蛋白尿和管型。

\paragraph{血酮}

血酮体增高,定量一般>
4.8mmol/L(50mg/dl)。DKA时纠正酮症常比纠正高血糖缓慢。在DKA时,引起酸中毒作用最强、比例最高的是β-羟丁酸,而常用的亚硝酸铁氰化钠法仅仅可以测定乙酰乙酸和丙酮,无法检测β-羟丁酸。在治疗过程中,β-羟丁酸可以转化成乙酰乙酸,没有经验的医生可能误认为酮症恶化。因此监测DKA程度的最佳方法是直接测定β-羟丁酸。

\paragraph{尿酮}

当肾功能正常时,尿酮呈强阳性,但当尿中以β-羟丁酸为主时易漏诊(因亚硝酸铁氰化钠仅能与乙酰乙酸起反应,与丙酮反应较弱,与β-羟丁酸无反应)。肾功能严重损伤时,肾小球滤过率减少可表现为糖尿和酮尿减少甚至消失,因此诊断必须依靠血酮检查。若血pH明显降低而尿酮、血酮增加不明显者尚需注意有乳酸性酸中毒可能。

\paragraph{酸碱与电解质失调}

动脉血pH下降与血酮体增高呈平行关系,血pH≤7.1或CO\textsubscript{2} CP <
10mmol/L(< 20vol\%)时为重度酸中毒,血pH 7.2或CO\textsubscript{2} CP
10~15mmol/L为中度酸中毒,血pH > 7.2或CO\textsubscript{2} CP
15~20mmol/L为轻度酸中毒。血钠一般<
135mmol/L,少数正常,偶可升高达145mmol/L。血氯降低。血钾初期可正常或偏低,少尿而脱水和酸中毒严重期可升高至5mmol/L以上。血镁、血磷亦可降低。

\paragraph{血象}

血白细胞增多,无感染时可达(15~30)× 10\textsuperscript{9}
/L,尤以中性粒细胞增高较显著。血红蛋白、血细胞比容增高,反映脱水和血液浓缩情况。

\subsubsection{诊断注意事项}

早期诊断是决定治疗成败的关键,临床上对于原因不明的恶心呕吐、酸中毒、失水、休克、昏迷的患者,尤其是呼吸有酮味(烂苹果味)、血压低而尿量多者,不论有无糖尿病病史,均应想到本病的可能性。立即查末梢血糖、血酮、尿糖、尿酮,同时抽血查血糖、血酮、β-羟丁酸、尿素氮、肌酐、电解质、血气分析等以肯定或排除本病。如尿糖和酮体阳性,同时血糖增高,或血pH降低者,无论有无糖尿病病史即可诊断。DKA患者昏迷者只占少数,如发现有昏迷时尚应与糖尿病的另外几种危象情况相鉴别,详见表\ref{tab49-1}。

DKA患者可出现类似急腹症的临床表现,如呕吐、腹痛、腹部压痛与肌紧张、血白细胞增高等,与急腹症不易区别;急腹症患者也可因感染、呕吐不能进食而致酮症酸中毒,易与本症相混淆;而某些急腹症如急性胰腺炎、胆囊炎等有时可与DKA并存,使病情更为复杂。因此必须详询病史、细致的体检和必要的实验室检查,全面地加以分析判断。伴严重腹痛的DKA与急腹症的鉴别需注意以下特点:①病史:在疑似病例有时病史比体征更重要,若烦渴、多尿与厌食在腹部症状出现前早已存在,很可能患者全部临床表现是由DKA所致;如腹部症状较烦渴、多尿等症状出现为早,则急腹症的可能性较大。②体征:DKA时腹痛可急可缓,可伴有腹胀、腹部压痛,但反跳痛不明显,此种体征随酮症纠正很快改善;而急腹症时腹部压痛与反跳痛多明显,酮症纠正时,因病因未除去,临床症状不能好转。③腹痛特点:DKA时腹痛多呈弥散性,疼痛不固定,局限性压痛不明显;急腹症时均有相应的局限性压痛。

\subsection{治疗}

DKA的治疗原则是尽快补液以恢复血容量、纠正失水状态,降低血糖,纠正电解质及酸碱平衡失调,同时积极寻找和消除诱因,防治并发症,降低病死率。具体措施应根据病情轻重而定,如早期轻症,脱水不严重,酸中毒属轻度,无循环衰竭,神志清醒的患者,仅需给予足量正规胰岛素(RI),每4~6小时1次,每次皮下或肌肉注射10~20U,并鼓励多饮水,进半流汁或流汁饮食,必要时静脉补液,同时严密观察病情,随访尿糖、尿酮、血糖与血酮及CO\textsubscript{2}
CP、pH等,随时调整胰岛素量及补液量,并治疗诱因,一般均能得到控制,恢复到酮症前情况。对于中度和重症病例应积极抢救,具体措施如下。

\begin{table}[htbp]
\centering
\caption{糖尿病并发昏迷的鉴别}
\label{tab49-1}
\includegraphics[width=6.6875in,height=5.17708in]{./images/Image00165.jpg}
\end{table}

\subsubsection{一般处理}

一般处理措施包括:①立即抽血验血糖、血酮体、钾、钠、氯、CO\textsubscript{2}
CP、BUN、血气分析等。②留尿标本,验尿糖与酮体、尿常规,计尿量;昏迷者应留置导尿管。③昏迷患者应保持呼吸道通畅,吸氧,注意保暖与口腔、皮肤清洁。④严密观察病情变化与细致护理:每1~2小时查血糖、电解质与CO\textsubscript{2}
CP(或血气分析)1次,直至血糖< 13.9mmol/L
(250mg/dl),CO\textsubscript{2} CP >
15mmol/L(33vol\%),延长至每4小时测1次。由于静脉pH比动脉pH降低0.03U,可以用静脉pH换算,从而减少反复动脉采血。

\subsubsection{补液}

补液是治疗的关键环节。只有在有效组织灌注改善、恢复后,胰岛素的生物效应才能充分发挥。可建立两条静脉输液通道:一条用作补液,另一条用作补充胰岛素。由于静脉内应用胰岛素需要保持一定的浓度和滴速,因此,保证胰岛素单独静脉通路是十分必要的。胰岛素是蛋白质,输注液体的pH、液体成分及输注物的分子量等因素均可能降低胰岛素的生物学效价,因此用于静脉滴注的胰岛素可以是生理盐水或葡萄糖溶液,尽量不与其他药物配伍。最初补液治疗的目的:①迅速扩张血管内外液体容量;②恢复肾脏血流灌注;③纠正高渗状态;④通过肾脏排泄酮体。早期以充分补充生理盐水为主,避免输入低渗液而使血浆渗透压下降过速,诱发脑水肿。补液总量可按患者体重的10\%估算。可建立两条静脉输液通道:一条用作补液,另一条用作补充胰岛素。补液宜先快后慢,头4小时内补总量的1/4~1/3;头8~12小时内补总量的2/3;其余部分在24~48小时内补给。补液时:①对无心功能不全者,头2小时输注生理盐水1000~2000ml;第3、4小时内各输入300~500ml;以后每4~6小时输入1000ml或更多,争取12小时内输入4000ml左右。第一个24小时输入总量约达4000~5000ml,严重失水者可达6000~8000ml。②已发生休克或低血压者,快速输液不能有效升高血压,应考虑输入胶体液如血浆、全血或血浆代用品等,并按需要给予其他抗休克治疗。对年老或伴有心脏病、心力衰竭者,应在中心静脉压监测下调节输液速度与输液量。③当血钠>
155mmol/L,又无心功能不全或休克时,可慎重考虑输入0.45\%低渗盐水1000~2000ml。待血糖降至13.9mmol/L(250mg/dl)时,改输5\%葡萄糖液,并按每2~4g葡萄糖加入1U
RI。同时减少输液量,防止低血糖反应。液体损失严重又持续呕吐者,可输入5\%葡萄糖盐水。

对无明显呕吐、胃肠胀气或上消化道出血者,可同时采取胃肠道补液。胃肠道补液的速度在头2小时内约500~1000ml,以后依病情调整。胃肠道补液量可占总补液量的1/3~1/2。考虑输液总量时,应包括静脉和胃肠道补液的总和。

\subsubsection{胰岛素治疗}

目前均釆用小剂量(短效)胰岛素疗法(每小时给予胰岛素0.1U/kg)。该方法具有简便、有效、安全,较少引起脑水肿、低血糖、低血钾等优点。且血清胰岛素浓度可恒定达到100~200μU/ml。这一血清胰岛素浓度已有抑制脂肪分解及酮体生成的最大效应,相当强的降低血糖的生物效应,而促进K\textsuperscript{+}
转运的作用则较弱。用药途径以持续静滴法最常用,以每小时0.1U/kg静滴维持(可用50U
RI加入生理盐水500ml中,以1ml/min的速度持续静滴)。对伴有昏迷和(或)休克和(或)严重酸中毒的重症患者,可加用首次负荷量胰岛素10~20U静脉注射。血糖下降速度一般每小时约降低3.9~6.1mmol/L(70~110mg/dl)为宜,每1~2小时复查血糖。若治疗2小时后血糖无肯定下降,提示患者对胰岛素敏感性降低,则将单位时间内的胰岛素剂量加倍,加大剂量后仍须继续定时检测血糖(1~2小时一次)。当血糖降至13.9mmol/L(250mg/dl)时,可改用5\%葡萄糖液500ml加RI
6~12U(即1U胰岛素:2~4g葡萄糖)持续静滴,胰岛素滴注率下调至0.05U/(kg•h),此时仍需每4~6小时复查血糖。当血糖降至11.1mmol/L以下,血
,血pH >
7.3,尿酮体转阴后,可以开始\includegraphics[width=1.02083in,height=0.16667in]{./images/Image00166.jpg}
皮下注射胰岛素方案。但应在停静滴胰岛素前1小时皮下注射一次RI,一般注射量为8U以防血糖回跳。其他用药途径可采用间歇肌肉注射或间歇静脉注射,每小时注射1次,剂量仍为0.1U/kg。

DKA临床纠正的标准为:血糖<
11.1mmol/L(200mg/dl),血\includegraphics[width=0.96875in,height=0.15625in]{./images/Image00167.jpg}
,静脉血pH > 7.3。

\subsubsection{纠正电解质和酸碱平衡失调}

据估计一般较重病例可失钠500mmol、钾300~1000mmol、氯350mmol、钙及磷各50~100mmol、镁25~

50mmol、,
失水约5.6L,故补液中应\includegraphics[width=1.20833in,height=0.16667in]{./images/Image00168.jpg}
注意补充此损失量。当开始补生理盐水后钠、氯较易补足。

\paragraph{纠正低血钾}

DKA患者体内总缺钾量通常达300~1000mmol,但在治疗前,细胞内的K\textsuperscript{+}
大量转移到细胞外液,再加上失水、血液浓缩、肾功能减退等因素,血钾不仅不降低,有时反显增高,因此,治疗前血钾水平不能真实反映体内缺钾程度。治疗开始后因胰岛素发挥作用,大量钾转入细胞内,大量补液致血液浓缩改善,加上葡萄糖对肾脏渗透效应致钾与钠进一步丢失,治疗后4小时左右血钾常明显下降,有时达严重程度。因此,不论患者开始时血钾是否正常或略升高,在使用胰岛素4小时后,只要患者有尿排出(≥30ml/h),便应给予静脉补钾。如治疗前血钾水平已低于正常,开始治疗时即应补钾;如治疗前血钾正常,尿量≥40ml/h,可在输液和胰岛素治疗的同时即开始补钾;若尿量<
30ml/h,宜暂缓补钾,待尿量增加后即开始补钾。血钾<
3mmol/L时,每小时补钾26~39mmol(氯化钾2~3g);血钾3~4mmol/L时,每小时补钾20~26mmol(氯化钾1.5~2.0g);血钾4~5mmol/L时缓慢静滴,每小时补钾6.5~13mmol/L(氯化钾0.5~1.0g);血钾>
5.5mmol/L时应暂缓补钾。有条件时应在心电监护下,结合尿量与血钾水平,调整补钾量与速度。神志清醒者可同时口服钾盐。由于钾随糖、镁、磷等进入细胞较慢,补钾须继续5~7天方能纠正钾代谢。经充分补钾2~3天后低血钾难以纠正,或血镁<
0.72mmol/L(1.8mg/dl)时,应考虑补镁。用10\%~25\%硫酸镁1~2g肌肉注射,或加入液体中静滴;亦可用门冬氨酸钾镁20~60ml加入液体中滴注。

\paragraph{纠正酸中毒}

轻症患者经补液及胰岛素治疗后,钠丧失和酸中毒可逐渐得到纠正,不必补碱。重症酸中毒使外周血管扩张和降低心肌收缩力,导致低体温和低血压,并降低胰岛素敏感性,抑制呼吸中枢和中枢神经系统功能,故应给予相应治疗。但酮症酸中毒的基础是酮酸生成过多,非{}
损失过多;故必须采用胰岛素抑制酮体生成,促进酮体氧化,且酮体氧化后产生\includegraphics[width=1.03125in,height=0.15625in]{./images/Image00170.jpg}
30vol\%),无明显酸中毒大呼吸者可不予补碱或停止补碱。

\subsubsection{消除诱因与防治并发症}

\paragraph{抗感染}

感染既可作为诱因,又是DKA的常见并发症,应积极抗感染治疗。

\paragraph{防治并发症}

包括休克、心力衰竭、心律失常、肾功能不全、脑水肿等,详见有关章节。

\protect\hypertarget{text00122.html}{}{}

\section{高血糖高渗状态}

高血糖高渗状态(hyperglycemic hyperosmolar
state,HHS)是糖尿病急性代谢紊乱的另一临床类型。以严重高血糖、高血浆渗透压、脱水为特点,无明显酮症酸中毒,患者常有不同程度的意识障碍或昏迷。HHS与既往所称的“高渗性非酮症糖尿病昏迷”(hyperosmolar
nonketotic diabetic coma,HNDC)、高渗性昏迷(hyperosmolar
coma)略有不同,因为部分患者并无昏迷,部分患者可伴有酮症。与DKA相比,HHS失水更为严重,神经精神症状更为突出。本症多见于老年患者,好发年龄为50~70岁,但各年龄组均可发病,男女发病率大致相同。临床特点为无明显酮症与酸中毒,血糖显著升高,严重脱水甚至休克,血浆渗透压增高,以及进行性意识障碍等。

\subsection{病因与发病机制}

HHS的基本病因与DKA相同,但值得注意的是约2/3HHS患者发病前无糖尿病史,或者不知有糖尿病,有糖尿病史者也多为轻症2型糖尿病,偶也可发生于年轻的1型糖尿病患者。HHS除了原有的糖尿病基础外,几乎都有明显的诱发因素,临床上常见的诱因包括:①应激:如感染(尤其是呼吸道与泌尿道感染)、外伤、手术、急性脑卒中、急性心肌梗死、急性胰腺炎、胃肠道出血、中暑或低温等;②摄水不足:可见于口渴中枢敏感性下降的老年患者,不能主动进水的幼儿或卧床患者、精神失常或昏迷患者,以及胃肠道疾病患者等;③失水过多:见于严重的呕吐、腹泻,以及大面积烧伤患者等;④药物:如各种糖皮质激素、利尿剂(特别是噻嗪类和呋塞米)、苯妥英钠、氯丙嗪、普萘洛尔、西咪替丁、免疫抑制剂等;⑤高糖的摄入:见于大量服用含糖饮料、静脉注射高浓度葡萄糖、完全性静脉高营养,以及含糖溶液的血液透析或腹膜透析等。有时在病程早期因误诊而输入大量葡萄糖液或因口渴而摄入大量含糖饮料可诱发本病或使病情恶化。上述诸因素均可使机体对胰岛素产生抵抗、升高血糖、加重脱水,最终诱发或加重HHS的发生与发展。

HHS是体内胰岛素相对缺乏使血糖升高,并进一步引起脱水,最终导致的严重高渗状态。胰岛素相对不足、液体摄入减少是HHS的基本病因。胰岛素缺乏促进肝葡萄糖输出(通过糖原分解和糖异生)、损伤了骨骼肌对葡萄糖的利用,高血糖的渗透性利尿作用导致血容量不足,如补液不充分,患者病情加重。另外,HHS的发生发展受到一系列因素的影响:存在感染、外伤、脑血管意外等诱发因素的情况下,胰岛素分泌进一步减少,对抗胰岛素的激素水平升高,血糖明显升高;HHS多发生于老年患者,口渴中枢不敏感,加上主动饮水的欲望降低和肾功能不全,失水常相当严重,而钠的丢失少于失水,致血钠明显增高;脱水和低血钾一方面能引起皮质醇、儿茶酚胺和胰高血糖素等升糖激素的分泌增多,另一方面进一步抑制胰岛素分泌,继而造成高血糖状态的继续加重,形成恶性循环,最终导致HHS发生。

HHS与DKA都是由于胰岛素缺乏而引起的糖尿病急性并发症,DKA主要表现为高血糖、酮症和酸中毒,而HHS以严重高血糖和高渗透压为特征。这两种代谢紊乱临床表现的差别,可能的原因为:①HHS时胰岛素只是相对缺乏,分泌的胰岛素虽足以抑制脂肪分解和酮体生成,但却不能抑制糖异生,故主要为血糖的明显升高;但在DKA胰岛素是高度缺乏,已不能抑制酮体生成;②胰高血糖素等升糖激素升高较轻,促进脂肪分解和生酮作用较弱;③HHS时失水严重,不利于酮体产生;④部分HHS患者血浆非酯化脂肪酸水平高而无酮症,提示肝生酮功能障碍;⑤严重高血糖和酮体生成之间可能存在拮抗作用。由此可见,HHS与DKA是不同程度的胰岛素缺乏所导致的两种状态,两者可能同时存在,实际上,1/3的高血糖患者可同时表现出HHS和DKA的特征。

\subsection{诊断}

\subsubsection{病史}

患者多为老年人,部分患者已知有糖尿病,30\%患者有心脏疾病,90\%患者有肾脏病变。可有诱发疾病如肺炎、泌尿系感染、胰腺炎等的表现。

\subsubsection{临床表现特征}

\paragraph{前驱期特点}

HHS起病多隐蔽,在出现神经系统症状至进入昏迷前常有一段时间,即前驱期,时间一般为1~2周。表现为糖尿病症状如口渴、多尿和倦怠、乏力等症状的加重,反应迟钝,表情淡漠,引起这些症状的基本原因是由于渗透性利尿脱水。若能对本症提高警惕,在前驱期及时发现并诊断,则对患者的治疗和预后大有好处。但由于前驱期症状无特异性易被患者本人和医生所忽略,且常被其他合并症症状所掩盖和混淆,致使诊断困难和延误。

\paragraph{典型期的表现}

如前驱期得不到及时诊治,则病情继续发展,主要表现为严重的脱水和神经系统两组症状和体征。脱水表现为皮肤干燥和弹性减退,眼球凹陷、唇舌干裂、脉搏快而弱,卧位时颈静脉充盈不好,立位时血压下降。严重者出现休克,但因脱水严重,体检时可无冷汗。有些患者虽有严重脱水,但因血浆的高渗促使细胞内液外出,并补充了血容量,可能掩盖了失水的严重程度,而使血压仍保持正常。神经系统方面则表现为不同程度的意识障碍,从意识模糊、嗜睡直至昏迷。HHS患者的意识障碍与否,主要决定于血浆渗透压升高的程度与速度,与血糖的高低也有一定关系,而与酸中毒的程度关系不大。通常患者血浆有效渗透压>
320mOsm/L时,即可出现精神症状,如淡漠、嗜睡等;而当患者血浆有效渗透压>
350mOsm/L时,可有定向力障碍、幻觉、上肢拍击样粗震颤、癫痫样发作、偏瘫、偏盲、失语、视觉障碍、昏迷和阳性病理征等,这些提示患者可能有因脱水、血液浓缩和血管栓塞而引起的大脑皮质或皮质下的损害。出现神经系统症状常是促使患者前来就诊的原因,因此常被误诊为一般的脑卒中等颅内疾病而导致误诊误治,后果严重。和酮症酸中毒不一样,HHS没有典型的酸中毒大呼吸,如患者出现中枢性过度换气现象时,则应考虑是否合并有脓毒症和脑卒中。

\subsubsection{实验室检查}

\paragraph{血常规}

由于脱水血液浓缩,血红蛋白增高,白细胞计数多> 10 ×
10\textsuperscript{9} /L。

\paragraph{尿检查}

尿糖多强阳性,患者可因脱水及肾功能损害而致尿糖不太高,但尿糖阴性者罕见。尿酮体多阴性或弱阳性。

\paragraph{血糖}

常≥33.3mmol/L,一般为33.3~66.6mmol/L
(600~1200mg/dl),有高达138.8mmol/L(2500mg/dl)或更高者。血酮体多正常。另外,因血糖每升高5.6mmol/L,血钠下降1.6mmol/L左右,HHS时存在严重高血糖,可造成血钠水平假性降低。

\paragraph{血尿素氮(BUN)和肌酐(Cr)}

常显著升高,反映严重脱水和肾功能不全。BUN可达21~36mmol/L(60~100mg/dl),Cr可达124~663μmol/L(1.4~7.5mg/dl),BUN/Cr比值(按mg/dl计算)可达30∶1(正常人多在10~20∶1)。有效治疗后BUN及Cr多显著下降。BUN与Cr进行性升高的患者预后不佳。

\paragraph{血浆渗透压}

显著升高,多超过350mOsm/L,有效渗透压超过320mOsm/L。血浆渗透压可直接测定,也可根据血糖及电解质水平进行计算,公式为:血浆渗透压(mOsm/L)=
2({[}Na\textsuperscript{+} {]}+{[}K\textsuperscript{+}
{]})+血糖(mmol/L)+ BUN
(mmol/L),正常值为280~300mOsm/L;若BNU不计算在内,则为有效渗透压,因BUN可自由进出细胞膜。

\paragraph{电解质}

血Na\textsuperscript{+} 可升高>
145mmol/L,也可正常或降低。血K\textsuperscript{+}
正常或降低,有时也可升高。血Cl\textsuperscript{−}
情况多与Na\textsuperscript{+} 一致。血Na\textsuperscript{+}
、Na\textsuperscript{+} 、Cl\textsuperscript{−}
的水平取决于其丢失量,在细胞内外的分布情况及患者的血液浓缩程度。不论其血浆水平如何,患者总体Na\textsuperscript{+}
、K\textsuperscript{+} 、Cl\textsuperscript{−}
都是丢失的。有人估计,HHS患者Na\textsuperscript{+}
、K\textsuperscript{+} 和Cl\textsuperscript{−}
丢失分别为5~10mmol/kg、5~15mmol/kg和5~7mmol/kg。此外,还可有Ca\textsuperscript{2+}
、Mg\textsuperscript{2+} 和磷的丢失。

\paragraph{酸碱平衡}

约半数患者有轻、中度代谢性酸中毒,pH多高于7.3,{} 常高于15mmol/L。

\subsubsection{诊断注意事项}

HHS的病死率仍较高,能否及时诊断直接关系到患者的治疗和预后。从上述其临床表现来看,本症的诊断并不困难,关键是临床医生要提高对本症的警惕和认识,特别是对中、老年患者有以下临床情况者,无论其有无糖尿病病史,均应考虑有HHS的可能,应立即作实验室检查:①进行性意识障碍和明显脱水表现者;②中枢神经系统症状和体征,如癫痫样抽搐和病理反射征阳性者;③合并感染、心肌梗死、手术等应激情况下出现多尿者;④大量摄糖,静脉输糖或应用激素、苯妥英钠、普萘洛尔等可致血糖增高的药物时出现多尿并有意识改变者;⑤昏迷休克患者,休克未曾纠正而尿量多者。

HHS的诊断依据是:①中、老年患者,血糖≥33.3mmol/L
(600mg/dl);②有效渗透压≥320mOsm/L;③动脉血气分析示pH≥7.30或血{}
浓度≥15mmol/L;④尿糖强阳性,血酮体阴性或弱阳性。但应值得注意的是HHS有并发DKA或乳酸性酸中毒的可能性。个别病例的高渗状态主要是由于高血钠,而不是高血糖造成的。因此尿酮体阳性,酸中毒明显或血糖<
33.3mmol/L,并不能作为否定HHS诊断的依据。但HHS患者无一例外地存在明显的高渗状态,如昏迷患者血浆有效渗透压<
320mOsm/L,则应考虑到糖尿病并发其他急性并发症的可能性(参见表\ref{tab49-1})。

\subsection{治疗}

HHS的基本病理生理改变是高血糖、高渗透压引起脱水、电解质丢失和血容量不足,以致患者休克和肾、脑组织脱水与功能损害,而危及患者的生命。因此,其治疗原则是立即补液,使用胰岛素、纠正电解质紊乱和防治并发症,与DKA基本相同。

\subsubsection{补液}

HHS患者均有严重脱水,而高渗状态引起的脑细胞脱水是威胁患者生命的主要原因,单纯补液即可使血糖每小时下降1.1mmol/L(20mg/dl),可使血浆渗透压下降,减轻脑细胞水肿。因此,迅速补液以恢复血容量,纠正高渗和脱水是抢救成败的关键。本症患者脱水比DKA严重,失水量多在发病前体液的1/4或体重的1/8以上。补液时可根据患者的脱水程度,按其体重的10\%~15\%估算;也可以按血浆渗透压计算患者的失水量,计算公式为:患者的失水量(L)=(患者血浆渗透压−
300)÷ 300(为正常血血浆渗透压)×体重(kg)×
0.6。考虑到在治疗过程中将有大量液体自肾脏、呼吸道及皮肤丢失,补液总量可略高于估计的失液总量。一般在最初2小时可补液1000~2000ml,头4小时内输入补液总量的1/3,头12小时内补入总量的1/2加尿量,其余在以后24小时内补足。经积极补液4~6小时后仍少尿或无尿者,宜给呋塞米(速尿);若发现有显著的肾损害,则输液量要适当调整。在静脉输液的同时,应尽可能通过口服或胃管进行胃肠道补液,此法有效而且简单和安全,可减少静脉补液量,从而减轻大量静脉输液引起的副作用。在输液中,应注意观察患者的尿量、颈静脉充盈度并进行肺部听诊,有条件时应行中心静脉压监测,以指导补液。

补液后细胞脱水状态改善,葡萄糖利用率提高,肾功能改善,排糖能力增强,继而产生抗胰岛素水平下降等效应,可使血糖明显下降。一般每输入1500ml液体可使血糖降低18\%,但在有肾实质病变的患者充足补液尚不能恢复正常的排糖功能,血糖下降缓慢。

关于补液的种类和浓度,目前多主张治疗开始时用等渗盐水(308mmol/L),因大量输入等渗液不会引起溶血,有利于恢复血容量,纠正休克,改善肾血流量,恢复肾脏调节功能。休克患者应另予血浆或全血。如无休克或休克已纠正,在输入生理盐水1000~2000ml后,血浆渗透压仍>
350mOsm/L,血钠>
155mmol/L时,可考虑输入适量低渗液如0.45\%氯化钠溶液(154mmol/L)或2.5\%葡萄糖溶液(139mmol/L)。当血浆渗透压降至330mOsm/L时再改为等渗液。在治疗过程中,当血糖下降至16.7mmol/L(300mg/dl),应使用5\%葡萄糖液(278mmol/L)或5\%葡萄糖盐水(586mmol/L),并酌情加用胰岛素,以防止血糖及血浆渗透压过快下降。应注意:5\%葡萄糖液的渗透压为278mOsm/L,虽为等渗,但糖浓度约为正常血糖的50倍,5\%葡萄糖盐水的渗透压为586mOsm/L,在治疗早期两者均不适用,以免加重高血糖、高血钠及高渗状态。停止补液的条件是:①血糖<
13.9mmol/L(250mg/dl);②尿量>
50ml/h;③血浆渗透压降至正常或基本正常;④患者能饮食。

\subsubsection{胰岛素治疗}

其使用原则与方法和DKA大致相同,即在输液开始时同时给予小剂量胰岛素静脉滴注。HHS患者一般对胰岛素比DKA敏感,在治疗中对胰岛素需要量相对较少。经输液和用胰岛素后血糖降至≤16.7mmol/L(300mg/dl)、血浆渗透压下降至<
330mOsm/L时,将液体改为5\%葡萄糖液,同时按2~4g葡萄糖:1U胰岛素的比例加入胰岛素静滴(详见DKA的治疗),若此时血钠仍低于正常则宜用5\%葡萄糖盐水。在补充胰岛素时,应注意高血糖是维护患者血容量的重要因素,如血糖降低过快而液体又补充不足,将导致血容量和血压进一步下降,反而促使病情恶化。因此,应使血糖每小时以2.75~3.9mmol/L(50~70mg/dl)的速度下降,尿糖保持在“+~++”为宜。

\subsubsection{纠正电解质紊乱}

与DKA治疗相同。

\subsubsection{防治并发症}

各种并发症特别是感染,常是患者晚期死亡的主要原因,必须一开始就给予大剂量有效的抗生素治疗。其他并发症的治疗如休克、肾功能不全、心力衰竭等,参见有关章节。

\subsubsection{其他措施}

包括去除诱因、支持疗法和对症处理等。

\protect\hypertarget{text00123.html}{}{}

\section{乳酸性酸中毒}

乳酸性酸中毒(lactic
acidosis,LA)是由于各种原因导致组织缺氧,乳酸生成过多,或由于肝脏病变致使乳酸利用减少,清除障碍,血乳酸浓度明显升高引起。本症是糖尿病的急性并发症之一,多发生于伴有全身性疾病或大量服用双胍类药物的患者。可单独存在或与酮症酸中毒和高血糖高渗状态并存,其病情严重,病死率高达50\%以上,早期诊断与治疗非常重要。

\subsection{病因与发病机制}

乳酸是糖无氧酵解的终产物,在供氧正常时放出能量ATP,但当供氧不足时,丙酮酸不能进一步代谢而堆积在细胞内,在乳酸脱氢酶系的作用下,丙酮酸由NADH(还原型辅酶Ⅰ)获得H\textsuperscript{+}
而转变为乳酸,正常乳酸的产生与利用之间保持平衡,血乳酸浓度正常值为0.4~1.4mmol/L,约为丙酮酸的10倍。当全身或局部缺血、缺氧在细胞水平氧利用减低,糖酵解增强,丙酮酸生成增多,直接转变为乳酸也越多。随着血乳酸生成,血pH改变取决于:①组织产生乳酸的速度;②细胞外液的缓冲能力;③肝、肾对H\textsuperscript{+}
清除的能力。因此血乳酸堆积有两种情况,一种只是血乳酸水平暂时增加而无血pH降低的“高乳酸血症”(hyperlactacidemia),即Huckabee分型Ⅰ型;另一为乳酸性酸中毒,血乳酸增高同时有H\textsuperscript{+}
堆积、血pH降低,即Huckabee分型Ⅱ型。Ⅱ型按不同的病因机制又分为两个亚型:A型也叫“继发性乳酸性酸中毒”,继发于各种缺氧或缺血性疾病,如各种休克时。其发病机制是组织获得的氧不能满足组织代谢需要,导致无氧酵解增加,产生A型乳酸性酸中毒;B型也称“自发性乳酸性酸中毒”,因肝、肾疾病及白血病等全身性疾病以及某些药物(如苯乙双胍)引起乳酸代谢障碍所致。其发病机制与组织缺氧无关。B型可进一步分为三种亚型:B\textsubscript{1}
型与糖尿病、脓毒血症、肝肾功能衰竭等常见病有关;B\textsubscript{2}
型与药物或毒素有关;B\textsubscript{3}
型与肌肉剧烈活动、癫痫发作等其他因素有关。糖尿病乳酸性酸中毒常发生于2型糖尿病,其虽与上述各型都有联系,但更常见的是由口服双胍类降糖药(苯乙双胍即降糖灵,二甲双胍)引起的。苯乙双胍(DBI)引起乳酸性酸中毒的原因是:①DBI增加糖无氧酵解使乳酸产生增加;②减少了肝和肌肉对乳酸的摄取;③减少了肾脏排酸功能。已证实二甲双胍升高血乳酸的能力较DBI小,因而已逐渐代替DBI。过量饮酒、超量应用胰岛素等都有诱发乳酸性酸中毒的可能。另外,亦与糖尿病患者已合并有慢性心、肺疾病或肝、肾功能障碍有关。

\subsection{诊断}

\subsubsection{临床表现特点}

LA多见于50岁以上2型糖尿病,使用双胍类降糖药的过程中或伴发于急性重症并发症时。起病较急,主要表现为代谢性酸中毒引起的大呼吸,严重时神志模糊、精神恍惚、谵妄至昏迷,也可出现呕吐、腹泻等脱水症状,可有明显的腹痛,易误诊为急腹症。其临床过程又不能以肾功能衰竭或酮症酸中毒解释。

\subsubsection{实验室检查}

实验室检查是乳酸性酸中毒诊断的关键。除糖尿病的实验室检查外,还有:①血酸度明显增高:血pH
< 7.30,有的可降至7.0以下;血{} 明显降低,常< 10mmol/L。②血乳酸:常>
5mmol/L,有时可达35mmol/L(>
25mmol/L者大多不治);血丙酮酸相应增高,达0.2~1.5mmol/L;血乳酸/丙酮酸≥30。当乳酸浓度>
5mmol/L,HCO\textsubscript{3} \textsuperscript{−}
≤10mmol/L,乳酸/丙酮酸>
30而可除外其他酸中毒原因时,可确诊为本病。③血浆阴离子间隙(AG)∶AG常>
18mmol/L,可达25~45mmol/L(正常值12~16mmol/L)。AG增高常见于糖尿病酮症酸中毒或酒精性酮症酸中毒、尿毒症性酸中毒、乳酸性酸中毒及某些药物毒性所致如水杨酸盐等,临床上若排除前两者,又不存在药物毒性的可能,此时AG增高强烈支持乳酸性酸中毒。④血酮体一般不升高,或轻度升高。

\subsubsection{诊断注意事项}

糖尿病患者在服用双胍类降糖药过程中,呈现严重酸中毒,既无酮体增多(血酮、尿酮皆不增多),又无严重高血糖、血浆渗透压增高或高血钠等,即应疑及本症。凡有休克、缺氧、肝肾功能衰竭者,如酸中毒较重时,必须警惕LA的可能性。确诊依靠血乳酸测定,若无乳酸测定的设备条件,可根据AG增大,但先决条件是除外酮症酸中毒及高血糖高渗状态,其鉴别详见表\ref{tab49-1}。LA主要诊断标准为:①血乳酸≥5mmol/L;②动脉血pH≤7.35;③AG
> 18mmol/L;④{} < 10mmol/L;⑤CO\textsubscript{2}
CP降低;⑥丙酮酸增高,乳酸/丙酮酸≥30∶1;⑦血酮体一般不升高。

\subsection{防治}

\paragraph{预防为主}

LA病死率高,治疗难度大,故必须提高警惕,认真预防。双胍类药物如DBI可诱发LA,肝、肾、心功能不全时,可导致双胍类药物在体内蓄积,因此在应用双胍类药物前应查肝、肾、心功能,若存在功能不全则忌用双胍类药物。对于其他能诱发LA的药物,如水杨酸、异烟肼、山梨醇、乳糖等,也应尽量避免应用。休克、缺氧、肝肾功能衰竭状态下的危重患者,若伴有酸中毒,须警惕发生LA的可能性,并努力防治。

\paragraph{一般措施}

寻找和去除诱发LA的诱因,停用所有可诱发LA的药物与化学物质,有利于B型LA的治疗。畅通呼吸道,充分供氧,改善氧合功能。并加强监测。

\paragraph{纠正休克}

是治疗A型LA的重要措施。补液扩容可改善组织灌注,减少乳酸的产生,促进利尿排酸。输液宜用生理盐水,避免用含乳酸的液体。

\paragraph{纠正酸中毒}

高渗碳酸氢钠溶液可抑制HbO\textsubscript{2}
分离,加重组织缺氧,尤其有早期循环衰竭者;大剂量碳酸氢钠可引起血钠过高、血渗透压升高、容量负荷加重,血乳酸反而增高。故目前主张用小剂量等渗碳酸氢钠溶液持续静脉滴注的方式,使{}
上升4~6mmol/L,维持在14~16mmol/L,动脉血pH升至7.2。

缺乏\includegraphics[width=2.66667in,height=0.17708in]{./images/Image00176.jpg}
(mmol/L)× 0.5 ×体重(kg)。

糖尿病患者有DKA存在时仅需少量碳酸氢钠使pH恢复到7.0~7.1为宜。除补液补碱外,随时补充钾盐以防低钾或缺钾。

\paragraph{降低血乳酸}

①胰岛素治疗:胰岛素不足是导致糖尿病LA的诱因之一。胰岛素不足使丙酮酸脱氢酶活性降低,丙酮酸进入三羧酸循环减少。应用胰岛素治疗,减少糖无氧酵解,有利于血乳酸的清除。血糖不高的患者需同时静滴葡萄糖液。②亚甲蓝(美蓝):为氧化还原剂,其作用类似NAD\textsuperscript{+}
,可促使乳酸转化为丙酮酸,降低血乳酸的浓度。用法是1~5mg/kg静滴,2~6小时作用达高峰,可维持14小时。③二氯醋酸(dichloroacetate,DCA):是丙酮酸脱氢酶激活剂,能迅速增强乳酸的代谢,并可阻止肝细胞释放乳酸和丙酮酸,使血中浓度进一步降低。此外,DCA能增强心肌收缩力和心输出量,从而改善心脏灌注,明显提高患者生存水平。④血液净化疗法:用不含乳酸钠的透析液进行血液或腹膜透析治疗,可加速乳酸排泄,并可清除DBI等引起LA的药物,尤其适用于不能耐受钠过多的老年患者与肾功能衰竭患者,对双胍类药物引起的LA是最为有效的治疗方法。
\protect\hypertarget{text00124.html}{}{}

\hypertarget{text00124.htmlux5cux23CHP4-14-4}{}
参 考 文 献

1. 陆再英,钟南山.内科学.第7版.北京:人民卫生出版社,2008:770

2. 王姮 ,杨永年.糖尿病现代治疗学.北京:科学出版社,2005:265

3. 陈灏珠 ,林果为.实用内科学.第13版.北京:人民卫生出版社,2009:1018

4. Kitabchi AE,Umpierrez GE,Murphy MB,et al. Management of
hyperglycemic crises in patients with diabetes. Diabetes
Care,2001,24:131

5.
中华医学会糖尿病学分会.中国2型糖尿病防治指南.中华内分泌代谢杂志,2008,24(2):增录

\protect\hypertarget{text00125.html}{}{}

\chapter{痛 风 危 象}

高尿酸血症(hyperuricemia)与痛风(gout)是嘌呤代谢障碍引起的代谢性疾病,但痛风发病有明显的异质性,除高尿酸血症外可表现为急性关节炎、痛风石(tophus)、慢性关节炎、关节畸形、慢性间质性肾炎和尿酸性尿路结石。高尿酸血症患者只有出现上述临床表现时,才称之为痛风。痛风危象(gout
crisis)一般是指痛风性关节炎急性发作,以及因尿酸性尿路结石引起的肾绞痛和血尿。

\subsection{病因和发病机制}

临床上痛风可分为原发性和继发性两类,前者多由先天性嘌呤代谢异常所致,常与肥胖、糖脂代谢紊乱、高血压、动脉硬化和冠心病等聚集发生,后者则由某些系统性疾病或者药物引起。其具体病因和发病机制尚不清楚。作为嘌呤代谢的终产物,尿酸主要由细胞代谢分解的核酸和其他嘌呤类化合物以及食物中的嘌呤经酶的作用分解而来。人体中尿酸80\%来源于内源性嘌呤代谢,而来源于富含嘌呤或核酸蛋白食物仅占20\%。原发性高尿酸血症与痛风主要由尿酸排泄障碍引起(占80\%~90\%),包括肾小球滤过减少、肾小管重吸收增多、肾小管分泌减少以及尿酸盐结晶沉积,且以肾小管分泌减少最为重要;少数为尿酸生成增多,主要由酶的缺陷所致。继发性高尿酸血症与痛风则主要由于肾脏疾病致尿酸排泄减少,骨髓增生性疾病致尿酸生成增多,某些药物抑制尿酸的排泄等多种原因所致。

临床上仅有部分高尿酸血症患者发展为痛风,机制不清。当血尿酸浓度过高和(或)在酸性环境下,尿酸可析出结晶,沉积在骨关节、肾脏和皮下等组织,造成组织病理学改变,导致痛风性关节炎、痛风肾和痛风石等。急性关节炎是由于尿酸盐结晶沉积引起的炎症反应,因尿酸盐结晶可趋化白细胞,故在关节滑囊内尿酸盐沉积处可见白细胞显著增加并吞噬尿酸盐,然后释放白三烯B4(LTB4)和糖蛋白等化学趋化因子;单核细胞受尿酸盐刺激后可释放白介素l(IL-1)。长期尿酸盐结晶沉积招致单核细胞、上皮细胞和巨大细胞浸润,形成异物结节即痛风石。痛风性肾病是痛风特征性的病理变化之一,表现为肾髓质和锥体内有小的白色针状物沉积,周围有白细胞和巨噬细胞浸润。

\subsection{诊断}

\subsubsection{临床表现特点}

临床多见于40岁以上的男性,女性多在更年期后发病。常有家族遗传史。

\paragraph{无症状期}

仅有波动性或持续性高尿酸血症,从血尿酸增高至症状出现的时间可长达数年至数十年,有些可终身不出现症状,但随年龄增长痛风的患病率增加,并与高尿酸血症的水平和持续时间有关。

\paragraph{急性关节炎期}

特点是:①多在午夜或清晨突然起病,多呈剧痛,数小时内出现受累关节的红、肿、热、痛和功能障碍,单侧
{}
趾及第1跖趾关节最常见,其余依次为踝、膝、腕、指、肘;②秋水仙碱治疗后,关节炎症状可以迅速缓解;③发热;④初次发作常呈自限性,数日内自行缓解,此时受累关节局部皮肤出现脱屑和瘙痒,为本病特有的表现;⑤可伴高尿酸血症,但部分患者急性发作时血尿酸水平正常;⑥关节腔滑囊液偏振光显微镜检查可见双折光的针形尿酸盐结晶是确诊本病的依据。受寒、劳累、饮酒、高蛋白高嘌呤饮食以及外伤、手术、感染等均为常见的发病诱因。

\paragraph{痛风石及慢性关节炎期}

痛风石是痛风的特征性临床表现,常见于耳轮、跖趾、指间和掌指关节,常为多关节受累,且多见于关节远端,表现为关节肿胀、僵硬、畸形及周围组织的纤维化和变性,严重时患处皮肤发亮、菲薄,破溃则有豆渣样的白色物质排出。形成瘘管时周围组织呈慢性肉芽肿,虽不易愈合但很少感染。

\paragraph{肾脏病变}

主要表现在两方面:①痛风性肾病:起病隐匿,早期仅有间歇性蛋白尿,随着病情的发展而呈持续性,伴有肾浓缩功能受损时夜尿增多,晚期可发生肾功能不全,表现水肿、高血压、血尿素氮和肌酐升高。少数患者表现为急性肾衰竭,出现少尿或无尿,最初24小时尿酸排出增加。②尿酸性肾石病:约10\%~25\%的痛风患者肾有尿酸结石,呈泥沙样,常无症状,结石较大者可发生肾绞痛、血尿。当结石引起梗阻时导致肾积水、肾盂肾炎、肾积脓或肾周围炎,感染可加速结石的增长和肾实质的损害。

\subsubsection{辅助检查}

\paragraph{血尿酸测定}

男性和绝经后女性血尿酸> 420μmol/L (7.0mg/dl)、绝经前女性>
350μmol/L(5.8mg/dl)可诊断为高尿酸血症。

\paragraph{尿尿酸测定}

限制嘌呤饮食5天后,每日尿酸排出量超过3.57mmol(600mg),可认为尿酸生成增多。

\paragraph{滑囊液或痛风石内容物检查}

偏振光显微镜下可见针形尿酸盐结晶。

\paragraph{X线检查}

急性关节炎期可见非特征性软组织肿胀;慢性期或反复发作后可见软骨缘破坏,关节面不规则,特征性改变为穿凿样、虫蚀样圆形或弧形的骨质透亮缺损。

\paragraph{CT与MRI检查}

CT扫描受累部位可见不均匀的斑点状高密度痛风石影像;MRI的T1和T2加权图像呈斑点状低信号。

\subsubsection{诊断注意事项}

中老年患者尤其男性如出现上述特征性关节炎表现、尿路结石或肾绞痛发作,伴有高尿酸血症应考虑痛风或痛风危象。关节液穿刺或痛风石活检证实为尿酸盐结晶可做出诊断。X线检查、CT或MRI扫描对明确诊断具有一定的价值。急性关节炎期诊断有困难者,秋水仙碱试验性治疗有诊断意义。应注意与类风湿关节炎、化脓性关节炎与创伤性关节炎等鉴别。

\subsection{治疗}

原发性高尿酸血症与痛风的防治目的:①控制高尿酸血症预防尿酸盐沉积;②迅速终止急性关节炎的发作;③防止尿酸结石形成和肾功能损害。

\paragraph{一般治疗}

控制饮食总热量;限制饮酒和高嘌呤食物(如心、肝、肾等)的大量摄入;每日饮水2000ml以上以增加尿酸的排泄;慎用抑制尿酸排泄的药物如噻嗪类利尿药等;避免诱因和积极治疗相关疾病等。

\paragraph{高尿酸血症的治疗}

目的是使血尿酸维持正常水平。

\hypertarget{text00125.htmlux5cux23CHP4-15-3-2-1}{}
(1) 排尿酸药:

抑制近端肾小管对尿酸盐的重吸收,从而增加尿酸的排泄,降低尿酸水平,适合肾功能良好者;当内生肌酐清除率<
30ml/min时无效;已有尿酸盐结石形成,或每日尿排出尿酸盐>
3.57mmol(600mg)时不宜使用;用药期间应多饮水,并服碳酸氢钠3~6g/d;剂量应从小剂量开始逐步递增。常用药物:①苯溴马隆:25~100mg/d。②丙磺舒:初始剂量为0.25g,每日2次;两周后可逐渐增加剂量,最大剂量不超过2g/d。

\hypertarget{text00125.htmlux5cux23CHP4-15-3-2-2}{}
(2) 抑制尿酸生成药物:

别嘌呤醇通过抑制黄嘌呤氧化酶,使尿酸的生成减少,适用于尿酸生成过多或不适合使用排尿酸药物者。每次l00mg,每日2~4次,最大剂量600mg/d,待血尿酸降至360μmol/L以下,可减量至最小剂量或别嘌呤醇缓释片250mg/d,与排尿酸药合用效果更好
。

1. 陆再英,钟南山.内科学.第7版.北京:人民卫生出版社,2008:830

2. 中华医学会
.临床诊疗指南急诊医学分册.北京:人民卫生出版社,2009:305肾功能不全者剂量减半。

\hypertarget{text00125.htmlux5cux23CHP4-15-3-2-3}{}
(3) 碱性药物:

碳酸氢钠可碱化尿液,使尿酸不易在尿中积聚形成结晶。成人口服3~6g/d。

\paragraph{急性痛风性关节炎期的治疗}

绝对卧床,抬高患肢,避免负重,迅速给秋水仙碱,越早用药疗效越好。

\hypertarget{text00125.htmlux5cux23CHP4-15-3-3-1}{}
(1) 秋水仙碱(colchicine):

系治疗急性痛风性关节炎的特效药物,通过抑制中性粒细胞、单核细胞释放白三烯B4、糖蛋白化学趋化因子、白细胞介素-1等炎症因子,同时抑制炎症细胞的变形和趋化,从而缓解炎症。国内常用口服法给药:初始口服剂量为1mg,随后0.5mg/h或1mg/2h,直至症状缓解,最大剂量6~8mg/d。90\%的患者口服秋水仙碱后48小时内疼痛缓解。症状缓解后改为0.5mg,每天2~3次,维持数天后停药。不良反应为恶心、呕吐、厌食、腹胀和水样腹泻,发生率高达40\%~75\%,如出现上述不良反应及时调整剂量或停药,若用到最大剂量症状无明显改善时应及时停药。该药还可以引起白细胞减少、血小板减少等骨髓抑制表现以及脱发等。

\hypertarget{text00125.htmlux5cux23CHP4-15-3-3-2}{}
(2) 非甾体抗炎药:

通过抑制花生四烯酸代谢中的环氧化酶活性,进而抑制前列腺素的合成而达到消炎镇痛。常用药物:①吲哚美辛,初始剂量75~100mg,随后每次50mg,6~8小时1次。②双氯芬酸,每次口服50mg,每天2~3次。③布洛芬,每次0.3~0.6g,每天2次。④罗非昔布25mg/d。症状缓解应减量,5~7天后停用。禁止同时服用两种或多种非甾体抗炎药,否则会加重不良反应。

\hypertarget{text00125.htmlux5cux23CHP4-15-3-3-3}{}
(3) 糖皮质激素:

上述药物治疗无效或不能使用秋水仙碱和非甾体抗炎药时,可考虑使用糖皮质激素或ACTH短程治疗。如泼尼松0.5~1mg/(kg•d),3~7天后迅速减量或停用,疗程不超过2周;ACTH
50U溶于葡萄糖溶液中缓慢静滴。可同时口服秋水仙碱l~2mg/d。该类药物的特点是起效快、缓解率高,但停药后容易出现症状“反跳”。

\paragraph{发作间歇期和慢性期的处理}

治疗目的是维持血尿酸正常水平(见高尿酸血症治疗),较大痛风石或经皮溃破者可手术剔除。
\protect\hypertarget{text00126.html}{}{}

\hypertarget{text00126.htmlux5cux23CHP4-15-4}{}
参 考 文 献

\chapter{溶 血 危 象}

在慢性溶血性疾患病程中,突然出现急性溶血,或具有潜在溶血因素的患者,在某些诱因作用下,使红细胞寿命缩短、破坏增加,突然出现寒战、高热、烦躁不安、全身不适、胸闷、头痛、极度疲乏、剧烈的腰背及四肢酸痛,甚至出现少尿或尿闭,血红蛋白可骤然或大幅度下降,贫血、黄疸等表现急剧加重,网织红细胞增加,可伴有肝脾明显肿大,称之为溶血危象(hemolytic
crisis)。若不及时救治,常可危及生命。

\subsection{病因与发病机制}

溶血危象是在原有溶血性疾病的基础上,通过某种诱因而诱发。溶血性贫血的病因虽然很多,但引起溶血危象最常见病因是血型不合输血、药物性溶血、红细胞6-磷酸葡萄糖脱氢酶(G-6PD)缺乏症、自身免疫性溶血性贫血(AIHA)、阵发性睡眠性血红蛋白尿(PNH)、严重感染及动植物毒素等。常见诱因有感染(如呼吸道与胃肠道感染)、创伤、外科手术、妊娠、过度疲劳、情绪波动、大量饮酒、服酸性药物及食物等。

本病的发病机制尚不十分明了。正常红细胞的平均寿命约为100~120天,每天约有1\%的红细胞被破坏,而骨髓则不断相应地生成并释放新生的红细胞以维持动态平衡。如当平均红细胞寿命短于20天时,则红细胞破坏速度远远超过了骨髓的潜在代偿能力(正常的代偿能力为6~8倍),将出现溶血性贫血。溶血可以根据红细胞的破坏部位,分为血管内溶血和血管外溶血(表\ref{tab51-1})。大量溶血使血浆中游离血红蛋白(正常约为1~10mg/L)急骤增加,超过单核-巨噬细胞系统处理血红蛋白的能力,则发生游离血红蛋白血症。如游离血红蛋白大于0.7~1.4g/L时,超过珠蛋白所能结合的能力,溶血12小时后可以发生黄疸,并通过肾排泄而出现血红蛋白尿。大量血红蛋白刺激和沉淀,可以导致肾血管痉挛和肾小管梗阻,以致缺血坏死,发生急性肾衰竭;又由于大量红细胞破坏,患者出现严重贫血,甚至发生心功能不全、休克、昏迷。严重贫血时,骨髓又将大量幼稚红细胞释放入血,故危象发生时末梢血象可见大量不成熟红细胞。部分溶血危象患者病程中严重的黄疸可能突然有所减轻,血中网织红细胞急剧减少甚至完全消失,血清胆红素与尿中尿胆原降至正常范围,骨髓涂片呈现红细胞系列增生完全停滞,骨髓中出现巨大的原始细胞,这提示患者发生了急性骨髓功能衰竭(再生障碍性危象)。

\begin{table}[htbp]
\centering
\caption{血管内与血管外溶血的鉴别}
\label{tab51-1}
\includegraphics[width=6.79167in,height=3.69792in]{./images/Image00178.jpg}
\end{table}

\subsection{诊断}

\subsubsection{有溶血性贫血的病因和(或)诱因存在}

\subsubsection{临床表现特点}

起病急骤,突出的表现为严重的贫血、黄疸(间接胆红素增加),红细胞寿命缩短,网织红细胞增加,可伴有肝脾肿大。

\paragraph{常有慢性溶血性贫血的原发病的临床症状和体征}

如冷凝集素病患者出现雷诺现象、寒冷性荨麻疹及肢端麻木等;阵发性冷性血红蛋白尿症者,受冷后出现血红蛋白尿和黄疸;阵发性睡眠性血红蛋白尿常在睡眠后出现阵发性溶血等。此外,患者可有面色苍黄、不同程度的黄疸和贫血,轻度全身淋巴结肿大,肝脾肿大尤其以脾大更为明显。

\paragraph{溶血危象期的表现}

其严重程度与不同的病因和病种及溶血方式、溶血的速度等有关。

\hypertarget{text00126.htmlux5cux23CHP4-16-2-2-2-1}{}
(1) 寒战与发热:

大部分危象发生时,先有寒战,继之体温升高,达39℃左右,少数可超过40℃。可有不同程度的烦躁不安、胸闷、谵妄、神志不清。发热可能与红细胞急剧破坏、血红蛋白大量释放有关,有的病例亦可能与危象的感染诱因并存。

\hypertarget{text00126.htmlux5cux23CHP4-16-2-2-2-2}{}
(2) 四肢、腰背、腹部疼痛:

患者多有全身骨痛及腰背酸痛,尤以双肩及两侧肾区疼痛最为显著,腰背疼可以发生在急性肾衰竭之前或之中,并且症状出现越早,肾脏损害越严重。与此同时患者常可伴有腹痛,严重者出现明显的腹肌紧张,酷似急腹症,亦可有恶心、呕吐、腹胀、肠鸣等消化道症状。

\hypertarget{text00126.htmlux5cux23CHP4-16-2-2-2-3}{}
(3) 肾脏损害:

可有少尿或尿闭,高钾血症,氮质血症等,以致发生急性肾衰竭。

\hypertarget{text00126.htmlux5cux23CHP4-16-2-2-2-4}{}
(4) 血压下降:

危象发生后常出现血压下降,甚至休克,同时伴有心率增快,呼吸急促。这与抗原-抗体反应所致的过敏性休克、血管舒缩功能失调有关,尤其在血型不合的输血所致的溶血危象时,血压下降常不易纠正。此外,可因骤然大量溶血,导致高钾血症,心肌缺血缺氧,可引起心律失常,甚至发生心力衰竭。

\hypertarget{text00126.htmlux5cux23CHP4-16-2-2-2-5}{}
(5) 出血倾向与凝血障碍:

大量红细胞破坏可以消耗血液内的凝血物质,发生去纤维蛋白血症综合征(defibrination
syndrome),导致明显的出血倾向。部分患者常因感染、休克、肾功能衰竭、电解质紊乱、酸碱平衡失调并发DIC而使出血加重。

\hypertarget{text00126.htmlux5cux23CHP4-16-2-2-2-6}{}
(6) 贫血加重、黄疸加深:

患者贫血突然加重,全身乏力,心悸气短,危象发生12小时后,可见全身皮肤、黏膜黄疸急剧加深(因一次大量溶血,5~6小时后血中的胆红素浓度可以达到最高峰,但仍需5~6小时皮肤、黏膜才能黄染)。若溶血停止,一般在2~3天后黄疸消退,血中胆红素浓度恢复正常。

\hypertarget{text00126.htmlux5cux23CHP4-16-2-2-2-7}{}
(7) 肝、脾肿大:

溶血危象时,患者的肝脾均有明显肿大,尤其以脾大更为显著,这与贫血及黄疸轻重成正比。急剧肿大的肝、脾常有胀痛和压痛。因大量溶血,胆红素排泄过多,在胆道内沉积,易发生胆结石并发症。

\subsubsection{有溶血性贫血的实验室证据}

\paragraph{红细胞破坏增加的证据}

\hypertarget{text00126.htmlux5cux23CHP4-16-2-3-1-1}{}
(1) 血红蛋白代谢产物增加的表现:

①血清间接胆红素增高;②尿中尿胆原增加,每日可高达5~200mg(正常为0~3.5mg)。

\hypertarget{text00126.htmlux5cux23CHP4-16-2-3-1-2}{}
(2) 血浆血红蛋白含量增高的表现:

①血浆游离血红蛋白含量增高:正常人含量为1~10mg/L,大量溶血时,可高达1000mg/L以上,使血浆颜色变为琥珀色、粉红色或红色。这是血管内溶血最早可观察到的表现。②血清结合珠蛋白降低或消失:血清结合珠蛋白是血液中一组α\textsubscript{2}
糖蛋白,作用似血红蛋白的转运蛋白质。它是在肝脏内产生,正常血清中含量为0.5~1.5g/L(50~150mg/dl)。血管内溶血后,1分子的结合珠蛋白可结合1分子的游离血红蛋白,形成珠蛋白血红蛋白复合物,迅速被肝细胞摄取而从血中消失。大量溶血时,当血浆中游离血红蛋白过多,超过肝脏生成结合珠蛋白的能力,血清结合珠蛋白浓度降低,甚至消失。③血红蛋白尿:游离血红蛋白与结合珠蛋白相结合的产物,由于分子量大,不能通过肾小球排出,但当血浆中游离血红蛋白超过结合珠蛋白所能结合的量,多余的血红蛋白即可从肾小球滤出。经肾小球滤出的游离血红蛋白,在近端肾小管中可部分被重吸收,余下的血红蛋白形成临床所见的血红蛋白尿。所以,所谓血红蛋白的“肾阀”,实际上代表结合珠蛋白结合血红蛋白的能力和肾小管对血红蛋白重吸收能力之和。一般血浆中游离血红蛋白量大于1.3g/L(130mg/dl)时,临床出现血红蛋白尿,尿呈淡红色、红色、棕色或酱油色,尿隐血试验阳性。个别患者结合珠蛋白的表型与血红蛋白结合很差,结合量甚至低达0.025g/L,因而一旦有轻度血管内溶血,很容易出现血红蛋白尿。④含铁血黄素尿:被肾小管重吸收的游离血红蛋白,在肾近曲小管上皮细胞内被分解为卟啉、铁及珠蛋白。超过肾小管上皮细胞所能输送的铁,以铁蛋白或含铁血黄素形式沉积在上皮细胞内。当细胞脱落随尿排出,即成为含铁血黄素尿。血管内溶血后约数天含铁血黄素尿测定才转阳性,并可持续一段时间。⑤高铁血红素白蛋白血症(methemalbuminemia):血浆中游离血红蛋白很易氧化为高铁血红蛋白,然后分解出高铁血红素和珠蛋白,高铁血红素与白蛋白结合成高铁血红素白蛋白,使血浆呈棕色。⑥血清血结素水平降低:血结素系肝内合成,能结合循环中由高铁血红蛋白分解的游离血红素,最后被肝脏清除。血管内溶血时血结素被大量结合而消耗。

\hypertarget{text00126.htmlux5cux23CHP4-16-2-3-1-3}{}
(3) 红细胞寿命缩短:

红细胞的寿命缩短是溶血的最可靠指标。当一般检查不能肯定时,红细胞寿命测定常能显示溶血,且可以估计溶血的严重程度以及鉴别溶血是由于红细胞内缺陷还是红细胞外缺陷,或两者均有缺陷。目前常用有\textsuperscript{15}
Cr、\textsuperscript{3} P-DFP或\textsuperscript{3}
H-DFP(二异丙基氟磷酸)标记红细胞法。

\paragraph{红细胞系代偿性增生的表现}

①网织红细胞增加:溶血性贫血时,因血红蛋白的分解产物刺激造血系统,导致骨髓幼红细胞代偿性增生,网织红细胞一般可达5\%~20\%,如有肯定溶血的患者而无网织红细胞增生者,要考虑有再生障碍性危象的可能性。②周围血液中出现幼红细胞:一般不多,约1\%左右,主要是晚幼红细胞。此外,在严重溶血时尚可见豪-胶(Howell-Jolly)小体和幼粒细胞。由于网织红细胞及其他较不成熟红细胞自骨髓中大量释放至血液,故周围血液中大型红细胞增多。③骨髓幼红细胞增生:溶血性贫血时,幼红细胞显著增生,以中幼和晚幼红细胞最多,形态多正常。粒红比值明显降低(<
1.5)或倒置<
0.5)。④红细胞寿命的化学标志:最常用的代表红细胞寿命的化学标志是红细胞肌酐。较幼稚的红细胞肌酐是成熟型的6~9倍,且持续时间较网织红细胞长。⑤血浆铁转运率(PITR):被用来测定红细胞总的增生程度,且相关性较好。

\subsubsection{确定溶血性贫血的病因}

引起溶血性贫血的原因很多,下列几点可供参考:①若有肯定的化学、物理因素的接触史或明确的感染史,一般病因诊断容易肯定。②抗人球蛋白试验阳性者,应首先考虑免疫性溶血性贫血,进一步探究原因,并用血清学方法以探索抗体的性质。③抗人球蛋白试验阴性,血片中发现大量球形细胞,患者很可能为遗传性球形细胞增多症,可进一步检查红细胞渗透性脆性试验及自体溶血试验,同时进行直系亲属的血象检查以肯定诊断。但球形细胞增多也可见于免疫性溶血性贫血及某些化学及感染因素所致者。④周围血片发现有特殊红细胞畸形者,如椭圆形细胞、大量红细胞碎片、靶形及低色素细胞,可相应考虑遗传性椭圆形细胞增多症、微血管病性溶血性贫血及海洋性贫血,并进行有关的各项检查以肯定之。⑤患者既无红细胞畸形而抗人球蛋白试验又阴性,可进行血红蛋白电泳以除外血红蛋白病;热变性试验以除外不稳定血红蛋白;高铁血红蛋白还原试验以除外红细胞葡萄糖-6-磷酸脱氢酶缺陷症。

\subsubsection{诊断注意事项}

在诊断溶血危象时,应注意与以下疾病相鉴别:

\paragraph{急性再生障碍性贫血}

本病常多凶险,严重进行性贫血、出血、感染,常危及生命,但多无黄疸(除败血症外),网织红细胞明显减少,网织红细胞绝对计数减少,不伴肝脾肿大,骨髓象三系造血严重受抑,非造血细胞增多。

\paragraph{脓毒症}

常有原发或继发感染病灶;有阳性致病菌培养结果;白细胞计数增高且可见中性粒细胞内有中毒颗粒;即使有黄疸也较轻;无血浆中游离血红蛋白增高,无血红蛋白尿。

\paragraph{黄疸型肝炎}

溶血性贫血患者,当某种诱因激发溶血危象时,病情常常特别严重,患者严重乏力、深度黄疸、食欲极度减退伴肝脾肿大,易误诊为黄疸型肝炎,延误治疗。但本病除黄疸外,肝脾可肿大,多为低热,尿胆原可阳性,常无血红蛋白尿,胆红素升高多呈双相反应,网织红细胞多在正常范围内(很少超过5\%),骨髓增生无旺盛改变,末梢血不伴有红细胞受损所致的形态改变。

\paragraph{微血管病性溶血性贫血}

本综合征主要是微血管疾患包括血栓性血小板减少性紫癜、溶血性尿毒症综合征、暴发性紫癜(内毒素血症)等,除溶血表现外,主要是微血管本身病变疾病,各有其原发病特点,溶血只是其中表现之一。

\subsection{治疗}

\subsubsection{治疗病因、消除诱因}

首先应尽量去除已知的病因及各种诱因,如停止血型不合的输血,停用可疑引起溶血的药物、食物,控制感染等。

\subsubsection{肾上腺皮质激素的应用}

肾上腺皮质激素具有抑制单核-巨噬细胞系统合成抗体的作用,并能解脱致敏红细胞上的抗体。使用方便、安全、有效率高,应列为首选药物。主要用于温抗体型自身免疫溶血性贫血(AIHA)的溶血危象,对冷抗体型AIHA无效。对其他非免疫性溶血性贫血疗效不确定,不推荐使用。有适应证者可静脉快速滴注地塞米松20~40mg/d或氢化可的松300~1200mg/d,至少应用3~5天,待急性溶血控制或病情稳定后改用口服。常用泼尼松40~60mg/d口服,当Hb升至100g/L左右时,每周将泼尼松减少5~10mg,减至10~15mg/d时以此量维持1~2个月,最后以5~10mg/d再维持3个月。若在减量过程中,溶血性贫血又加重,应将剂量恢复至最后一次减量前的水平。但大剂量或长期激素治疗常合并高血压、糖尿病、感染,甚至可出现精神异常,必须引起注意。

\subsubsection{输注红细胞}

主要用于急性溶血危象及严重贫血或体质虚弱的患者,目的在于渡过危急难关,暂时改善严重贫血状态。一般输血后约12~48小时病情即可好转;但输血补给了补体有时反而加重溶血,因此,输血时应注意下列各点:①若因大量溶血发生休克、少尿、无尿、急性肾功能衰竭,应先解决少尿、无尿,输入低分子右旋糖酐改善微循环,纠正水、电解质失衡,待尿量增加、肾功能改善后,再进行输血。常需建立两条静脉通道,分别输液和缓慢输浓缩红细胞。②阵发性睡眠性血红蛋白尿接受输入的血浆可激活补体,诱发或加重溶血;严重贫血必须输血时,可谨慎输入经生理盐水洗涤的红细胞。③自体免疫性溶血性贫血患者体内抗体对正常供血者的红细胞易引起凝集现象,使输入的红细胞易于破坏,同时输血还提供了大量的补体,可使溶血加速,故应尽量避免输血。病情必须输血时,应先用配血试验凝集反应最小的供血者血液或经洗涤后红细胞悬液。若病情危急,又急需输血,又无分离或洗涤红细胞的条件,只有在输血的同时应用大量肾上腺皮质激素。输血速度应十分缓慢,密切观察,如有反应,应立即停止输血。④伯氨喹型药物性溶血性贫血及蚕豆病需输血时,献血员应作G-6-PD过筛试验。

\subsubsection{丙种球蛋白的应用}

静脉滴注丙种球蛋白{[}0.2~0.4g/(kg•d){]}对自身免疫性溶血性贫血有短期疗效。

\subsubsection{免疫抑制剂的应用}

免疫抑制剂多用于自身免疫性溶血性贫血对激素无效或需较大剂量维持者,常用环磷酰胺、环孢素和长春新碱等。

\subsubsection{血浆置换疗法}

发生严重贫血者,在静注或静滴皮质激素的同时,如未显效则应及时采取血浆置换疗法,以尽早去除存在于血浆中的抗体,特别适用于免疫性溶血性贫血危象发作时,常可较好较快改善疗效。有条件时应尽早试用。

\subsubsection{预防急性肾衰竭}

急性溶血发生少尿时,在纠正血容量后,为加快游离血红蛋白的排出,应尽早应用甘露醇,以增加肾血流量及尿量。先用20\%甘露醇250ml于15~30分钟内快速静滴完毕,使尿量维持在100ml/h以上。若尿量仍少,应每4~6小时重复1次。24小时尿量应达1500~2400ml。若24小时内仍无尿或少尿,则应停用。呋塞米(速尿)或布美他尼(丁尿胺)可以在用甘露醇的间歇期或甘露醇无效时应用。呋塞米剂量为40~80mg/次静脉注射,必要时可重复使用或加倍量,1天剂量可用至1000mg以上。已发生急性肾衰竭时,治疗原则与其他原因引起的急性肾衰竭相同。

既往处理溶血危象,强调补充碱性液体以碱化尿液,防止肾小管机械性阻塞。目前认为溶血引起肾功能衰竭的原理是反射性的肾血管痉挛,肾血流量减少,肾小管上皮细胞缺血、缺氧、坏死所致;或认为抗原-抗体复合物能引起肾功能损害;或与DIC有关。因此,过多补碱,尤其在少尿或无尿时,有引起碱中毒的潜在危险,使血液pH改变,导致氧解离曲线右移,更不利于组织的氧摄取,甚至可加速肺水肿的发生,故对碱化尿液防治肾衰竭的意义表示怀疑,认为不必列入常规治疗。但一般认为,有血红蛋白尿的患者,在利尿的基础上,适量给予碳酸氢钠来碱化尿液仍是必要的。

\subsubsection{防治其他并发症}

如防治休克、心力衰竭等,参见有关章节。

\subsubsection{脾切除术}

对某些溶血性贫血患者施行脾切除常可收到近期与远期效果,并能减少或防止溶血危象的发生,但须掌握脾切除适应证。对于遗传性球形红细胞增多症、地中海贫血综合征、丙酮酸激酶缺乏、不稳定血红蛋白病和原因不明的自身免疫性溶血性贫血所致的溶血危象,应用大剂量肾上腺皮质激素无效或因其严重副作用而不能耐受治疗,合并显著的脾功能亢进征象,甚至发生溶血危象而不易纠正者,可考虑脾切除术。
\protect\hypertarget{text00127.html}{}{}

\hypertarget{text00127.htmlux5cux23CHP4-16-4}{}
参 考 文 献

1. 顾静文.溶血性贫血概述//陈灏珠,林果为.实用内科学.
第13版.北京:人民卫生出版社,2009:2439

2. 谢毅
.溶血性贫血//陆再英,钟南山.内科学.第7版.北京:人民卫生出版社,2008:582

\protect\hypertarget{text00128.html}{}{}

\chapter{重症肌无力及其危象}

重症肌无力(myasthenia
gravis,MG)是一种神经-肌肉接头传递功能障碍的获得性自身免疫性疾病。主要由于神经-肌肉接头突触后膜上乙酰胆碱受体(acetylcholine
receptors,AChR)受损引起。临床主要表现为部分或全身骨骼肌无力和极易疲劳,具有活动后加重、休息后减轻和晨轻暮重等特点。若在其病程中急骤发生延髓肌和呼吸肌严重无力,出现呼吸困难,以致不能维持换气功能者为重症肌无力危象。其发生率约占MG患者的7.4\%~42.3\%,是神经内科常见急症之一,病死率较高,达19\%~43\%。如能及时、正确抢救,多数可挽回生命。

\subsection{病因与发病机制}

目前研究认为:重症肌无力是对自身Ach受体致敏的自身免疫病。70\%~90\%的重症肌无力患者血清中能检测到抗Ach受体抗体;且大多数患者血清中能检测到抗Ach受体抗体水平与疾病严重程度相一致;血浆置换治疗后,肌无力症状可以暂时好转。重症肌无力与胸腺异常关系密切,80\%以上的重症肌无力患者伴有胸腺异常,其中10\%~20\%的患者为胸腺肿瘤。而33\%~75\%的胸腺瘤患者合并有重症肌无力。胸腺切除以后,70\%的患者临床症状改善。重症肌无力与遗传因素有关,现在研究发现:重症肌无力与人类组织相容抗原(HLA-A,HLA-B,HLA-DR)明显相关。重症肌无力还与内分泌疾病有关,重症肌无力患者常伴发甲状腺功能亢进、类风湿关节炎、系统性红斑狼疮、多发性肌炎、多发性硬化等其他自身免疫性疾病。少数患者有家族性,称为家族性重症肌无力。

重症肌无力是一种主要累及神经-肌肉接头突触后膜AChR的自身免疫性疾病,主要由AChR抗体介导,在细胞免疫和补体参与下突触后膜的AChR被大量破坏,不能产生足够的终板电位,导致突触后膜传递功能障碍而发生肌无力。AChR抗体是一种多克隆抗体,主要成分为IgG,10\%为IgM。在AChR抗体中,直接封闭抗体可以直接竞争性抑制乙酰胆碱(acetylcholine,ACh)与AChR的结合;间接封闭抗体可以干扰ACh与AChR的结合。细胞免疫在MG的发病中也发挥一定的作用,MG患者周围血中辅助性T细胞增多,抑制性T细胞减少,造成B细胞活性增强而产生过量抗体。AChR抗体与AChR的结合还可以通过激活补体而使AChR降解和结构改变,导致突触后膜上的AChR数量减少。最终,神经-肌肉接头的传递功能发生障碍,当连续的神经冲动到来时,不能产生引起肌纤维收缩的动作电位,从而在临床上表现为易疲劳的肌无力。

\subsection{诊断}

\subsubsection{临床表现特点}

本病可见于任何年龄,发病年龄有两个高峰:20~40岁发病者女性多见;40~60岁发病者以男性多见,多合并胸腺瘤。

\paragraph{诱发因素}

感染、过度劳累、情绪波动、精神创伤、妊娠、月经期、系统性疾病、手术等为常见的诱因,甚至可使病情加重。另外一些药物如降低肌肉兴奋性的药物(奎宁、奎尼丁、普鲁卡因胺、利多卡因、苯妥英钠、青霉胺、普萘洛尔等)、止痛剂(吗啡、哌替啶等)、麻醉剂(乙醚、氯化琥珀胆碱、箭毒等)、抗生素(四环素、氨基糖苷类抗生素、新霉素、多黏菌素、巴龙霉素等)、镇静剂(苯二氮{}
类、苯巴比妥、氯丙嗪等)均可严重加重症状或抑制呼吸肌作用,应禁用。

\paragraph{肌无力特点}

受累的骨骼肌主要表现为病态疲劳,即持续活动后肌无力症状明显加重,经短暂休息后症状暂时缓解。肌无力另一特点是症状波动,不仅整个病程有波动,一天中的临床症状有波动,晨起症状较轻,下午和晚上症状逐渐加重,称为晨轻暮重现象。肌无力呈斑片状分布,程度随活动而变化,不能证明符合某一神经或神经根支配区,提示为神经肌肉传导障碍,是MG的典型临床特点。

\paragraph{受累肌的分布与表现}

全身骨骼肌均可受累,多以脑神经支配的肌肉最先受累。肌无力常从一组肌群开始,范围逐步扩大。首发症状常为一侧或双侧眼外肌麻痹,出现眼裂变小、睁眼困难、复视、眼球活动障碍等症状,严重者眼球完全固定,眼内肌(瞳孔括约肌)一般不累及,眼肌症状可以从单眼开始,而后波及对侧,也可双眼同时受累,但双眼症状多不对称。咀嚼肌受累则出现咀嚼无力,尤其在连续咀嚼坚硬食物时更明显,在进餐时常常因肌无力而需要休息,中断进餐,使进餐时间明显延长。咽喉部肌群无力时有吞咽困难,饮水咳呛,讲话时构音困难,常带有鼻音,或声音嘶哑,语音低弱。面肌受累则会有表情呆板,苦笑面容,闭眼和吸吮无力。胸锁乳突肌和斜方肌受累,则出现颈软、抬头困难、转头和耸肩无力。四肢肌肉受累以近端肌无力较远端明显,常呈对称性分布,表现为上臂抬举困难,尤其在做持续性抬举动作如梳头时更明显;下肢无力表现为不能长距离连续行走,常需要中途休息后方可继续前行,因抬腿无力而常需要用手拉住扶手上楼梯,下蹲后起立困难。呼吸肌和膈肌受累时出现咳嗽无力,呼吸困难,严重时可因呼吸肌麻痹而危及生命。偶尔会影响心肌,引起突然死亡。

\paragraph{重症肌无力危象}

大约10\%的重症肌无力出现危象。有三种表现形式:

\hypertarget{text00128.htmlux5cux23CHP4-17-2-1-4-1}{}
(1) 肌无力危象(myasthenic crisis):

在MG病程中,由于某种诱因而致肌无力症状加重,出现呼吸衰竭者为肌无力危象。为最常见的危象,多由于抗胆碱酯酶药物(ChEI)用量不足引起。其诱因多为合并感染、手术或外伤之后、精神创伤、分娩或月经、促皮质素(ACTH)或肾上腺皮质激素应用的早期,以及阻滞神经肌肉传递药物的应用等。上述因素可导致ACh去极化作用受到抑制而致神经兴奋传递障碍,从而使肌无力症状明显加重;咽喉肌及呼吸肌无力,吞咽困难甚至不能进食,呼吸困难,端坐呼吸,呼吸幅度表浅,呼吸频率加快;由于咳痰无力,气管内大量分泌物不能排除而加重缺氧,患者烦躁不安,甚至发生严重发绀。静脉注射依酚氯胺或肌肉注射新斯的明后可使症状明显缓解。

\hypertarget{text00128.htmlux5cux23CHP4-17-2-1-4-2}{}
(2) 胆碱能危象(cholinergic crisis):

由于长期应用ChEI和(或)用量过大,ACh在突触间隙处积聚过多,因而ACh持续作用于AChR,使突触后膜持续去极化,从而复极化过程受阻,而不能形成有效的动作电位,致全身肌力减弱,包括咽喉肌及呼吸肌无力,出现胆碱能危象。此种危象应用ChEI无效,甚至使症状更加严重。胆碱能危象除有呼吸衰竭等肌无力危象表现之外,尚可见有明显的ChEI副作用所致的症状,如流泪、全身大汗、唾液增多,咽喉及气管内大量分泌物,可见有肌束震颤或肌肉抽搐、痉挛,也可有瞳孔缩小、腹痛、腹泻、肠鸣音亢进、恶心、呕吐、尿便失禁等。患者焦虑不安、烦躁、精神错乱,甚至意识障碍、昏迷等。注射阿托品后可使症状改善。停止使用ChEI
24~72小时后临床症状好转。

(3) 反拗危象(brittle
crisis):又称为无反应性危象,是由于突触后膜大量AChR受损,对ChEI失去反应,残余的能与ACh发生反应的AChR太少,致突触后膜难以达到充分的去极化所致。此型可因长期应用ChEI或ChEI的剂量逐渐增大,或因感染、分娩、手术、创伤等诱因而致AChR过度疲劳,对ACh失去反应。临床表现与胆碱能危象相似,但发生此型危象时如应用或停用ChEI等均无效。

上述三种类型危象在病程中并非固定不变,肌无力危象患者在病程中也可能变为胆碱能危象或反拗危象,有的病例即具有胆碱能危象的表现,也有反拗危象的特点,某些病例在临床上不易辨识究竟属于何种类型危象。

\subsubsection{临床分型}

\paragraph{成年型肌无力(Osserman分型)}

Ⅰ型(眼肌型15\%~20\%):仅累及眼外肌,出现上睑下垂、斜视、复视,对肾上腺皮质激素治疗较敏感,大部分预后良好。

Ⅱa型(轻度全身型30\%):可累及眼、面、四肢肌肉,生活多可自理,无明显咽喉肌受累。进展缓慢,对药物敏感。

Ⅱb型(中度全身型25\%):症状较Ⅱa型重,除眼外肌、四肢肌无力外,还有吞咽困难、饮水呛咳、讲话含糊不清等延髓麻痹症状,呼吸肌常不受累,对药物的敏感性欠佳。

Ⅲ型(急性重症型15\%):急性发病,常在数周内累及延髓肌、肢带肌、躯干肌和呼吸肌,肌无力严重,易出现MG危象,此型病死率高。

Ⅳ型(迟发重症型10\%):自Ⅰ、Ⅱa和Ⅱb发展而来,2~4年后累及呼吸肌,症状同Ⅲ型,预后较差,常合并胸腺瘤。

Ⅴ型(伴肌萎缩型):少见,除肌无力外,合并肌萎缩。

\paragraph{少年型肌无力}

指14~18岁之间发病的MG患者,大部分以单纯眼外肌累及为主,仅少部分患者波及咽喉肌和四肢骨骼肌。

\paragraph{新生儿肌无力}

约10\%的MG母亲,其所生的婴儿可有短暂性的MG症状,如哭声低弱、吸吮无力、肌张力低、四肢少动等症状,严重者有呼吸困难,经抗胆碱酯酶药物治疗后,多于1周~3个月内症状消失,此系婴儿通过胎盘获得母体的AChR-Ab
IgG所致。

\paragraph{先天性肌无力}

极少见。婴儿在出生后短期内出现肌无力,持续存在的眼外肌麻痹症状,其母未患MG,但其家族中或同胞兄妹中有MG病史。

\paragraph{药物诱导的肌无力}

见于使用青霉胺治疗的肝豆状核变性、类风湿关节炎等患者中,有典型的MG临床症状,停药后症状可消失。

\subsubsection{辅助诊断试验}

下述试验有助于MG的诊断:

\paragraph{疲劳试验(Jolly试验)}

使受累肌肉在短时间内做重复收缩活动,如肌无力明显加重,经休息后又恢复者,为疲劳试验阳性。如对有上睑下垂者,嘱其持续向上注视,会出现眼睑下垂更明显,而后让其闭目休息数分钟后再睁眼,眼睑下垂症状又改善,为眼肌疲劳试验阳性。对肢体无力者,可令其双臂反复做平举动作,1分钟后出现上臂抬举困难,休息后恢复,为上肢疲劳试验阳性;做反复下蹲后起立动作,1分钟后出现起立越来越慢,甚至不能起立,休息后恢复,为下肢疲劳试验阳性。

\paragraph{抗胆碱酯酶药物试验}

①依酚氯胺(腾喜龙,tensilon)试验:依酚氯胺10mg用注射用水稀释至1ml,先静脉注射2mg,观察20秒,如无出汗、唾液增多等不良反应,再注射8mg(30秒内),1分钟内肌无力症状好转为阳性,持续10分钟后又恢复原状。②新斯的明试验:对依酚氯胺试验可疑者,可作本项试验,因其有较长时间供观察。肌肉注射新斯的明0.5~1mg,起效较慢,10~30分钟达高峰,作用持续2小时。若注射20分钟后肌无力症状好转,为新斯的明试验阳性。如出现恶心、呕吐、腹痛、腹泻、出汗、流涎、瞳孔缩小、心动过缓等毒蕈碱样反应,可肌肉注射阿托品0.5mg予以抵抗。

\subsubsection{辅助检查}

1.血 、尿、脑脊液检查正常。常规肌电图检查基本正常。神经传导速度正常。

2.重复神经电刺激(RNES)
为常用的具有确诊价值的检查方法。90\%的MG患者低频刺激时为阳性,且与病情轻重相关。

3.AChR抗体检测 对MG的诊断具有特征性意义。85\%以上全身型
MG患者血清中AChR抗体明显升高。

4.胸腺影像学检查
胸部X线尤其是胸腺CT和MRI有助于胸腺增生、肥大及胸腺瘤的发现。

\subsubsection{诊断注意事项}

MG的诊断要点:①病史特点:骨骼肌病态疲劳,症状波动,晨轻暮重,活动后加重,休息后减轻,没有神经系统其他阳性体征。②疲劳试验阳性。③新斯的明试验或依酚氯胺试验阳性。④神经重复频率刺激,动作电位波幅递减达10\%以上。⑤血AChR-Ab滴度增高。⑥胸部X线、CT
和MRI可显示胸腺增生或胸腺瘤。⑦服用抗胆碱酯酶药物有效。

MG须与Lambert-Eaton肌无力综合征、肉毒杆菌中毒、肌营养不良症、多发性肌炎等疾病鉴别。

\subsection{治疗}

临床上一旦明确MG诊断,应给予抗胆碱酯酶药物治疗,如单一抗胆碱酯酶药物疗效不明显,可联合应用肾上腺皮质激素或免疫抑制剂、胸腺切除、血浆置换疗法进行综合治疗。除病因及对症处理外,同时应尽量避免本病的各种诱发因素,防治各种感染,对可导致本病加重的药物应禁用或慎用。

\paragraph{抗胆碱酯酶药物}

应从小剂量开始,逐步加量,以能维持日常起居为宜。常用药物有:①溴吡斯的明30~120mg,每天3~4次,口服,作用持续时间6~8小时;②新斯的明15~30mg,每天3~4次,口服,作用维持3~4小时;③安贝氯铵5~10mg,每天3~4次,口服,作用维持4~6小时。以溴吡斯的明最为常用。氯化钾(1g,每天3次,口服)、麻黄碱(25mg,每天3次,口服)等能增强抗胆碱酯酶的作用,可作为辅助性用药。

\paragraph{肾上腺皮质激素}

可抑制自身免疫反应,减少AChR抗体的生成,增加突触前膜ACh的释放量及促使运动终板再生和修复,改善神经-肌肉接头的传递功能。适用于各种类型的MG。用法有两种:①冲击疗法:甲泼尼龙1000mg/d静脉滴注,3~5天后改用地塞米松10~20mg/d静脉滴注,连续7~10天。临床症状稳定改善后,改为口服泼尼松60~100mg隔日晨顿服。当症状基本消失后,逐渐减量至5~15mg长期维持,至少1年以上。适用于重症患者,特别是气管插管、使用人工呼吸机者,能在短期内获得满意疗效。②小剂量递增疗法:从小剂量开始,隔日晨顿服泼尼松20mg,每周递增10mg,直至隔日晨顿服60~80mg,待症状稳定改善4~5天后,逐渐减量至隔日5~15mg维持数年。此法优点是在用药初期一般不会出现MG症状加重。

\paragraph{免疫抑制剂}

适用于对肾上腺皮质激素疗效不佳或不能耐受,或因有高血压、糖尿病、溃疡病而不能用肾上腺皮质激素者。①环磷酰胺:成人口服50mg,每天2~3次,或200mg/次,每周2~3次静脉注射。儿童口服3~5mg/(kg•d)。可与肾上腺皮质激素合用。②环孢素A:6mg/(kg•d),口服,疗程12个月,治疗2周可见改善,6个月时可获最大改善。③硫唑嘌呤:适用于其他疗法无效的全身型MG。成人50~100mg/d,分2次服用,儿童1~3mg/(kg•d),长期服用,多在服药6~12周有效,6~15个月时达最佳疗效。

\paragraph{胸腺治疗}

①胸腺切除:适用于伴有胸腺肥大和高AChR抗体效价者;伴胸腺瘤的各型MG患者;年轻女性全身型MG患者;对ChEI治疗反应不满意者。约70\%的患者术后症状缓解或治愈。②胸腺放射疗法:对于年老体弱、有严重并发症不宜行胸腺摘除术者或手术后又复发者,可行胸腺深部\textsuperscript{60}
Co放射治疗。

\paragraph{血浆置换疗法}

主要清除血浆中的AChR-Ab及其他免疫复合物等致病因素,使症状迅速缓解。具有起效快、作用显著的特点,但维持时间短,价格昂贵,仅适用于危象和难治性MG。

\paragraph{静脉注射免疫球蛋白(IVIG)}

外源性IgG可以干扰AChR抗体与AChR的结合从而保护AChR不被抗体阻断。IVIG
400mg/(kg•d),每日或隔日一次,5次一个疗程,尤其适用于MG加重期、难治性MG及MG危象的治疗。

\paragraph{危象的处理}

处理的关键主要是:①保持呼吸道通畅,改善通气量,使动脉血氧维持正常水平。一旦发现有呼吸肌麻痹,应立即行气管插管和加压人工呼吸,如短期内症状不改善,则及时行气管切开,给予人工呼吸机辅助呼吸。②应注意避免或减少诱发因素。③积极对症处理,防治肺部感染,维持水电解质平衡。④症状治疗:皮质激素治疗,可给予大剂量甲泼尼龙冲击治疗,500~1000mg/d静脉滴注,3~5天后再逐步递减;如条件允许可行血浆置换疗法或静脉注射免疫球蛋白,争取短期内改善症状。同时应根据不同类型的危象采取相应的抢救措施:①肌无力危象:增加CHEI的剂量,静脉注射依酚氯胺10mg或肌肉注射新斯的明0.5~1.0mg,好转后逐渐改口服剂量,亦可用新斯的明2mg加入500ml液体中静脉滴注。②胆碱能危象:立即停用ChEI,阿托品1~2mg肌肉或2mg/h静脉注射,根据病情可重复使用,直至轻度阿托品化,症状改善后重新调整ChEI剂量,或改用皮质激素等其他治疗方案。③反拗性危象:主要维持生命体征的稳定,积极对症处理,避免或防治感染。停用ChEI,经过一段时间后,如对ChEI有效,则重新调整药物剂量;如对ChEI仍不起反应,则改用其他治疗方案。
\protect\hypertarget{text00129.html}{}{}

\hypertarget{text00129.htmlux5cux23CHP4-17-4}{}
参 考 文 献

1. 张文武.急诊内科学.第2版.北京:人民卫生出版社,2008

2. 贾建平.神经病学.第7版.北京:人民卫生出版社,2008

3. Keesey JC. Clinical evaluation and management of myasthenia gravis.
Muscle Nerve,2004,29(4):404-405

4. Rowland LP. Merritts Neurology. 11\textsuperscript{th} ed. New
York:Lippincott Williams & Wilkins,2005

\protect\hypertarget{text00130.html}{}{}



\appendix
\part{参考文献}

1.郭继鸿。新概念心电图。第3版。北京:北京大学医学出版社,2007

2.吴祥。心律失常梯形图解法。杭州:浙江大学出版社,2006

3.黄宛。临床心电图学。第5版。北京:人民卫生出版社,2004

4.陈新。黄宛临床心电图学。第6版。北京:人民卫生出版社,2009

5.郭继鸿。心电图学。北京:人民卫生出版社,2002

6.中国心电学会,中国心律学会编译。心电图标准化和解析的建议与临床应用国际指南2009。北京:中国环境科学出版社,2009

7.吴晔良,龚仁泰。临床心电图鉴别诊断。南京:江苏科学技术出版社,1999

8.吴晔良,龚仁泰。危重症心电图及临床处理。合肥:安徽科学技术出版社,2003

9.刘尚武,杨成悌。心电图诊断实践指南。北京:人民军医出版社,2007

10.朱力华,方炳森,张文篪。专题心电图精解。天津:天津科学技术出版社,2004

11.耿仁义,朱林中。人工心脏起搏心电图。北京:中国医药科技出版社,2001

12.刘仁光,徐兆龙。跨R波的房室传导。临床心电学杂志,2005,14(2):81

13.郭继鸿。Osborn波及临床应用意义。临床心电学杂志,1999,8(1):54

14.马劲林,马业新。致心律失常性右室心肌病/发育不良。中华内科杂志,2000,39(7):491

15.吴祥,蔡思宇。“巨R型”ST段抬高的特性及其临床意义。中华心血管病杂志,2004,32(8):762

16.何方田。急性心肌梗死墓碑型ST段抬高的预后意义。浙江医学,2003,25(11):661

17.吴祥。急性心肌梗死ST段抬高形态的临床意义。心电学杂志,2001,20(3):189

18.郭继鸿。Niagara瀑布样T波。临床心电学杂志,2001,10(4):233

19.王立群,郭继鸿。T波交替及T波变异性分析在临床中的应用。心电学杂志,2002,21(4):239

20.VincentGM,张莉,崔长琮。先天性长QT综合征的QT间期不均一性诊断上的意义。中华心律失常学杂志,2001,5(1):6

21.方炳森,龚仁泰。Q-T间期缩短的危重症5例报告。心电学杂志,2003,22(2):89

22.鲁端。短Q-T间期与继发性短Q-T间期综合征。心电学杂志,2008,27(2):181

23.何方田,尹小妹,李成。短Q-T间期综合征3例。临床心血管病杂志,2007,23(2):155

24.鲁端。心电图U波的新视野。心电学杂志,2009,28(3):211

25.何方田。房室旁路性并行心律及加速性逸搏心律2例。临床心血管病杂志,2003,19(3):176

26.龚仁泰,梁群。旁道心律。心电学杂志,2007,26(1):56

27.戚文航。心房颤动基础研究和临床治疗中的新视点。中华心血管病杂志,2002,30(10):577

28.汪康平。心房颤动合并二度房室阻滞的争议。临床心电学杂志,2004,13(1):2

29.鲁端。心房颤动合并房室传导阻滞的若干认识。心电学杂志,2007,26(2):186

30.何方田。心房颤动合并三度房室传导阻滞心电图分析。心电学杂志,2000,19(1):27

31.何方田。心室内折返性心律失常的心电图表现。中华心律失常杂志,2001,5(5):315

32.何方田。室性早搏伴折返径路内双径路传导的心电图表现。临床心血管病杂志,2001,17(9):424

33.胡大一。室上性心动过速的研究进展。心电学杂志,2002,21(2):77

34.赵昜。房性心动过速。心电学杂志,2006,25(3):179

35.赵昜。文氏现象。心电学杂志,2002,21(2):123

36.龚仁泰,中文宣。阵发性房室传导阻滞。心电学杂志,2004,23(3):168

37.刘仁光,郭莲怡。房室传导阻滞心电图分析中应注意的几个问题。心电学杂志,2003,22(4):210

38.赵昜。多层传导阻滞现象。心电学杂志,2002,21(4):244

39.何方田。双结病合并房室结双向性三径路传导的心电图表现。中华心血管病杂志,2003,31(5):382

40.何方田。早搏揭示房室结内多径路传导二例。中国心脏起搏与心电生理杂志,2002,16(5):397

41.何方田。房室结快径路隐匿性结-窦逆传致假性窦性静止2例。临床心血管病杂志,2002,18(7):344

42.赵昜。房室传导系统内的多径路传导。心电学杂志,2002,21(2):52

43.何方田。房性二联律心电图及24h动态心电图分析。浙江医学,2003,25(4):215

44.汪康平。病态窦房结综合征。心电学杂志,2003,22(4):198

45.何方田。103例心室停搏的动态心电图分析。浙江医学,2005,27(3):195

46.郭继鸿。韦金斯基现象。临床心电学,2007,16(1):57

47.何方田。心源性猝死高危患者的心电图特征。浙江医学,2004,26(4):246

48.赵昜。隐匿性传导。心电学杂志,2004,23(1):53

49.赵昜。蝉联现象。心电学杂志,2005,24(1):52

50.黄织春。容易掩盖急性心肌梗死的几种心电图诊断识别。临床心电学杂志,2006,15(3):170

51.许原。预激综合征旁道对共存束支传导阻滞心电图的影响。临床心电学杂志,2002,21(1):49

52.何方田,赵嵘,尹小妹,等。一种心电现象揭示另一种心电现象的若干心电图表现。心电学杂志,2007,26(2):86

53.张澍。起搏器综合征。中国循环杂志,2000,15(3):197

54.许原。起搏器介导性心动过速心电图。心电学杂志,2003,22(1):49

55.刘晓健。起搏器感知功能低下的表现形式。心电学杂志,2005,24(1):46

56.刘晓健。起搏器起搏功能障碍的心电图表现。心电学杂志,2005,24(2):111

57.刘晓健。起搏器所致心律失常的表现形式。心电学杂志,2005,24(4):231

58.刘晓健。起搏器若干特殊功能的心电图表现。心电学杂志,2006,25(1):47

59.何方田,尹小妹,赵林水。加强起搏心电图诊断中的医-技对话与沟通。心电学杂志,2006,25(4):232

60.何方田,刘岚,尹小妹。具有特殊功能DDD起搏器的心电图分析。心电学杂志,2008,27(4):270

61.金华斌,莫丹霞,何方田,等。VVI、DDD起搏器在心房扑动、颤动时容易误诊的心电图表现。心电学杂志,2009,28(2):88

62.王立群,郭继鸿。起搏模式的自由转换(1)。心电学杂志,2009,28(3):200

63.郭继鸿。连缀现象。临床心电学杂志,2000,9(1):52

64.赵昜。心脏电交替现象。心电学杂志,2005,24(2):116

65.赵昜。拖带现象。心电学杂志,2005,24(3):184

66.常学伟,魏毅东,张晓晓,等。心电图在诊断急性心肌梗死相关动脉中的价值。临床心电学杂志,2007,16(6):416

67.刘仁光。心肌缺血与心肌梗死心电图诊断的新概念。临床心电学杂志,2009,18(4):257

68.郭继鸿。急性冠脉综合征心律失常。临床心电学杂志,2009,18(4):302

69.高红,王红宇,王瑞英。运动试验变时性与冠心病相关性的研究。临床心电学杂志,2007,16(6):423

70.郭继鸿。心脏的变时性。临床心电学杂志,2003,12(4):267

71.鲁端。恶性心室早复极综合征的识别与处理。心电学杂志,2009,28(5):293

72.何方田,尹小妹。隐匿性传导在心房扑动、颤动时的心电图表现。心电学杂志,2009,28(5):317

73.唐继志。窦性心律震荡检测技术的临床应用及评价。心电学杂志,2009,28(5):373

74.赵晖,徐秋萍。心电图在急性肺栓塞诊治中的应用。心电学杂志,2006,25(1):82

75.赵昜。心室内传导阻滞。心电学杂志,2008,27(4):298

76.刘仁光,陈亮。宽QRS波群心动过速的鉴别诊断和处理原则。心电学杂志,2002,21(2):85

77.郭继鸿。宽QRS波群心动过速诊断中无人区心电轴的应用价值。心电学杂志,2007,26(1):38

78.赵昜。宽QRS波群心动过速。心电学杂志,2008,27(2):162

79.赵昜。室性心动过速。心电学杂志,2007,26(3):182

80.赵昜。室性心动过速(续)。心电学杂志,2007,26(4):244

81.何方田,林洁,尹小妹,等。规范心电图诊断报告。心电学杂志,2007,26(4):215


\end{document}
