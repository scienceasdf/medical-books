\chapter{绪 言}

\subsection{药理学的研究内容和任务}

药物(drug):指可以改善(或查明)生理功能及病理状态,达到预防、诊断、治疗疾病和计划生育目的的物质。

药品(中国药品管理):可用于预防、诊断、治疗疾病,有目的地调节生理功能并规定有适应证和主治功能、用法和用量的化学物质。

毒物:指损害机体的一类化学物质。

很多药物在大剂量时会对机体造成损害,“是药三分毒”,无病勿乱用。

$\text{\large 药物}\Leftrightarrow\text{\large 毒物}$

保健品:介于药物与食物之间的有利于机体健康但并不具体治疗某种疾病的物质。不以治疗疾病为目的,且对人体不产生任何急性、亚急性或者慢性危害的食品。

食物:维持机体生命活动需要的物质。

药理学(pharmacology)是研究药物与机体(含病原体)相互作用及作用规律的科学,为临床防治疾病、合理用药提供理论基础、基本知识和科学的思维方法。

药理学是连接医学与药学、基础医学与临床医学的桥梁学科,属于药学的分支。

(一)药理学研究的内容

药物效应动力学(pharmacodynamics,
PD):简称药效学,研究药物对机体的作用及作用机制。

药物代谢动力学(pharmacokinetics,
PK):简称药动学,研究机体对药物处置的过程,研究药物在机体的影响下所发生的变化及其规律。

\includegraphics{./images/Image00002.jpg}

\includegraphics{./images/Image00003.jpg}

(二)药理学的学科任务

(1)阐明药物的作用及作用机制,为临床合理用药、发挥最佳疗效、防治不良反应提供理论依据。

(2)研究开发新药,发现药物新用途(老药新用)。

(3)为生命科学提供重要的科学依据和研究方法。

\subsection{药理学与新药的研究开发}

新药:指化学结构、药品组分或药理作用不同于现有药品的药物。

我国《药品注册管理办法》规定:新药指未曾在中国境内上市销售的药品。已生产的药品,若改变剂型、给药途径、制造工艺或增加新的适应证,亦按新药管理。

新药研究的三个步骤(研究费用2亿~3亿美元):

1.临床前研究

药学研究:生产工艺、质量控制和稳定性等。

动物实验:药理学研究,药动学和药效学研究,毒理学研究,包括急性毒性、慢性毒性和特殊毒性等的研究。

2.临床研究

Ⅰ期(20~30例):主要观察正常人体对药物的作用、体内过程等药物代谢动力学参数。

Ⅱ期(不少于100例):主要观察对疾病是否有效。

Ⅲ期(不少于300例)------(可以上市):主要观察药物的治疗作用和不良反应。

Ⅳ期临床实验:在较大范围内观察、评价药物的治疗作用和不良反应。

3.售后调研

\subsection{药理学的学习方法}

(1)对所学的药物按照作用和作用机制进行分类,及时归纳、小结和简化所学的药理学知识。

(2)重点掌握各类药物中的代表性药物,如肾上腺素激动剂的代表性药物肾上腺素、胆碱能受体阻断剂的代表性药物阿托品、抗精神病药物中代表性药物氯丙嗪等。

(3)注意掌握药理作用与临床应用和不良反应的关系,特别是对临床应用意义较大的作用、不良反应和特点。

(4)比较同类药物的共性和个别药物的特性。

(5)及时、反复记忆是学好药理学的基本方法。

(6)理论联系实际,带着问题学。

\section*{大纲要求}

1.理解药理学、药物的概念及药理学研究的内容。

2.了解新药的研究开发。

