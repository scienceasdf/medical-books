\chapter{呼吸系统疾病的药物治疗}

\section{急性气管-支气管炎}

急性气管-支气管炎(acute
tracheobronchitis)是由生物、物理、化学刺激或过敏等因素引起的急性气管-支气管黏膜炎症。多为散发,无流行倾向,年老体弱者易感。临床症状主要为咳嗽和咳痰。常发生于寒冷季节或气候突变时,也可由急性上呼吸道感染迁延不愈所致。

\subsection{病因}

\subsubsection{微生物}

病原体与上呼吸道感染类似。常见病毒为腺病毒、流感病毒(甲、乙)、冠状病毒、鼻病毒、单纯疱疹病毒、呼吸道合胞病毒和副流感病毒。常见细菌为流感嗜血杆菌、肺炎链球菌、卡他莫拉菌等,近年来衣原体和支原体感染明显增加,在病毒感染的基础上继发细菌感染亦较多见。

\subsubsection{物理、化学因素}

冷空气、粉尘、刺激性气体或烟雾(如二氧化硫、二氧化氮、氨气、氯气等)的吸入,均可刺激气管-支气管黏膜引起急性损伤和炎症反应。

\subsubsection{过敏反应}

常见的吸入致敏原包括花粉、有机粉尘、真菌孢子、动物毛皮排泄物;或对细菌蛋白质的过敏,钩虫,蛔虫的幼虫在肺内的移行均可引起气管-支气管急性炎症反应。

\subsection{临床表现}

起病较急,通常全身症状较轻,可有发热。初为干咳或少量黏液痰,随后痰量增多,咳嗽加剧,偶伴血痰。咳嗽、咳痰可延续2~3周,如迁延不愈,可演变成慢性支气管炎。伴支气管痉挛时,可出现程度不等的胸闷气促。

查体可无明显阳性表现;也可在两肺闻及散在干、湿啰音,部位不固定,咳嗽后可减少或消失。

\subsection{诊断}

根据病史、咳嗽和咳痰等呼吸道症状,两肺散在干、湿性啰音等体征,结合血象和X线胸片,可作出临床诊断。病毒和细菌检查有助于病因诊断。

\subsection{治疗}

\subsubsection{一般治疗}

多休息、多饮水,避免劳累,避免吸入粉尘及刺激性气体。

\subsubsection{对症治疗}
\paragraph{镇咳祛痰药物}

(1)右美沙芬:中枢性镇咳药物,可抑制延脑咳嗽中枢而产生镇咳作用,与可待因的镇咳效果相当,长期服用不会引起依赖性和耐受性。

(2)氨溴索:祛痰药,可促进呼吸道内黏液分泌物的排出及减少黏液的滞留,促进排痰改善呼吸道状况。同时促进肺表面活性物质的分泌,增加支气管纤毛运动,使痰液易于咳出。

(3)桃金娘油提取物:目前较常应用的口服祛痰药物,通过重建上、下呼吸道黏液纤毛清除系统的清除功能,从而稀化和碱化痰液,增强黏液纤毛运动,黏液移动速度显著增加,促进痰液排出。服用该药物后排痰次数有所增加。除此之外,标准桃金娘油还具有抗感染作用。该药物能清除呼吸时的恶臭气味,长期用药后呼吸道的慢性炎症可获得改善。

(4)棕色合剂:较为常用的兼顾止咳和化痰的药物。该药物为复方制剂,含有甘草流浸膏、复方樟脑酊、愈创木酚甘油醚。甘草流浸膏为保护性祛痰药;复方樟脑酊为镇咳药物;愈创木酚甘油醚为祛痰剂,能使呼吸道腺体分泌增加、痰液稀释,易于咳出。
\paragraph{支气管扩张药物}

对于支气管痉挛的患者可用茶碱类药物如氨茶碱、二羟丙茶碱以及β{2}
受体激动剂等解痉止喘。
\paragraph{发热的患者可用解热镇痛药进行对症处理}

\subsubsection{抗菌药物治疗}

有细菌感染证据时应及时使用。可以首选新大环内酯类、青霉素类,亦可选用头孢菌素类或喹诺酮类等药物。多数患者口服抗菌药物即可,症状较重者可经肌内注射或静脉滴注给药,少数患者需要根据病原体培养结果指导用药。

\section{肺炎}

肺炎(pneumonia)是指终末气道、肺泡和肺间质的炎症,可由病原微生物、理化因素、免疫损伤、过敏及药物所致。细菌性肺炎是最常见的肺炎,也是最常见的感染性疾病之一。在抗菌药物应用以前,细菌性肺炎对儿童及老年人的健康威胁极大,抗菌药物的出现及发展曾一度使肺炎病死率明显下降。但近年来,尽管应用强力的抗菌药物和有效的疫苗,肺炎总的病死率不再降低,甚至有所上升。

\subsection{流行病学}

20世纪90年代欧美国家社区获得性肺炎和医院获得性肺炎年发病率分别约为12/1000人口和(5~10)/1000住院患者,近年发病率有增加的趋势。门诊肺炎患者的病死率<1%,住院患者平均为12%,入住重症监护病房(ICU)者约40%。发病率和病死率高的原因与社会人口老龄化、吸烟、伴有基础疾病和免疫功能低下有关,如慢性阻塞性肺病、心衰、肿瘤、糖尿病、尿毒症、神经疾病、药瘾、嗜酒、艾滋病、久病体衰、大型手术、应用免疫抑制剂和器官移植等。此外,亦与病原体变迁、医院获得性肺炎发病率增加、病原学诊断困难、不合理使用抗菌药物导致细菌耐药性增加等有关。

\subsection{病因及发病机制}

引起肺炎的病原体主要有细菌、真菌、衣原体、支原体、立克次体、病毒等微生物,其中细菌性肺炎占全部肺炎的半数左右,在我国成人肺炎中约占80%。近年来,尽管新的高效抗生素不断被发现,人类非但没有消灭肺炎,反而由于病原体的变迁、人口老龄化、特定高危人群的增加(如重大手术、机械通气、器官移植、肿瘤放化疗及AIDS患者等)以及抗生素的不合理应用、耐药菌株的不断增加等因素而面临严峻的挑战。肺炎的发病取决于宿主和病原体两方面的因素。

\subsubsection{宿主防御功能障碍}

任何原因造成全身免疫功能和呼吸道局部防御功能受损都是发生肺炎的高危因素。在医疗机构外,肺炎的发病中上呼吸道感染、受凉、疲劳、醉酒等都是常见的诱因。老年人机体防御功能减退,是细菌性肺炎的好发人群;一些慢性疾病患者,如癌症、慢性阻塞性肺疾病、心衰、高血压、糖尿病、肾病等好发肺炎。久住ICU及应用广谱抗生素、糖皮质激素、免疫抑制剂、细胞毒性药物时可引起机体内菌群失调、免疫功能低下,也易发生肺炎。建立人工气道和机械通气可破坏呼吸道局部防御功能,可促发通气相关肺炎。

\subsubsection{病原体侵入下呼吸道}
\paragraph{吸入污染的空气}

患者咳嗽、打喷嚏、说话时口鼻溅出飞沫,将呼吸道中的病原体播散到空气中,携带病原体的空气、飞沫、尘粒经呼吸进入呼吸道中可引起感染。支原体肺炎常流行于学校等集体或家庭中,空气飞沫传播是主要传播途径。
\paragraph{误吸上呼吸道病原菌}

健康人熟睡时可能不同程度地吸入咽喉部分分泌物,但通常不至于发生感染性病变。当上呼吸道有机会致病菌或其他病原体大量繁殖,再加上昏迷、休克、多痰、气管插管甚至雾化吸入治疗等因素,易使病原体进入下呼吸道,这是医疗机构内肺炎发病的重要途径。
\paragraph{血行播散及直接蔓延}

机会致病菌或其他病原体亦可为身体其他部位的感染病灶,通过血源播散或直接蔓延而侵入肺部。

\subsection{临床表现和分类}

\subsubsection{临床表现}

新近出现的咳嗽、咳痰或原有呼吸道疾病症状加重,并出现脓性痰,伴或不伴胸痛;发热,血白细胞计数增多;肺实变体征和湿性啰音;胸部X线检查显示片状、斑片状浸润性阴影或间质性改变,伴或不伴胸腔积液。上述系肺炎的共同表现,需要指出的是医院获得性肺炎的临床表现往往不典型,如粒细胞缺乏、严重脱水患者并发医院获得性肺炎时X光检查可以阴性,卡氏肺孢子虫肺炎有10%~20%患者的X线检查结果完全正常。

\subsubsection{分类}
\paragraph{按解剖分类}

(1)大叶性肺炎:炎症沿肺泡孔向其他肺泡扩散,引起肺段或肺叶广泛实变,支气管一般不受累,故又称作肺泡性肺炎,多为原发改变。X线显示呈叶、段或片状分布阴影,伴或不伴有空洞。

(2)小叶性肺炎:炎症随气管、支气管分布一直累及双侧肺泡,又称为支气管肺炎,多继发于其他肺部疾病。X线显示沿肺纹理分布的不规则斑片阴影。

(3)间质性肺炎:炎症主要侵犯肺间质,多见于病毒、支原体和过敏因素,X线显示肺内网状条索样分布阴影。
\paragraph{按病因分类}

1)细菌性肺炎

(1)需氧革兰阳性球菌:常见的有肺炎链球菌、金黄色葡萄球菌、甲型溶血性链球菌等。

(2)需氧革兰阴性杆菌:常见的有肺炎克雷伯菌、铜绿假单胞菌、大肠埃希菌、变形杆菌、军团杆菌、流感嗜血杆菌等。

(3)厌氧菌:如棒状杆菌、梭状杆菌等。

2)真菌性肺炎

致病性真菌如组织胞质菌、皮炎芽生菌等,条件致病菌如假丝酵母菌属、隐球菌属、曲霉菌属等。卡氏肺孢子虫也是一种真菌,常在免疫力低下的宿主中引起肺炎,是获得性免疫缺陷综合征患者最常见的直接致死原因。

3)病毒性肺炎

病毒性肺炎多为病毒性上呼吸道感染向下蔓延所致,在非细菌性肺炎中占25%~50%,好发于冬春季节,儿童多见,其中以流感病毒最为常见。

4)非典型病原体肺炎

由嗜肺军团菌、肺炎支原体和肺炎衣原体等感染引起。
\paragraph{按获病方式分类}

(1)社区获得性肺炎(community acquired
pneumonia,CAP):是指在社会环境中所患的感染性肺实质炎症,包括具有明确潜伏期的病原体在医疗机构外感染而入院后平均潜伏期发病的肺炎。肺炎链球菌感染占40%~70%,其次为金黄色葡萄球菌等。

(2)医院获得性肺炎(hospital acquired
pneumonia,HAP):是指患者入院时不存在、也不处于感染潜伏期,而于入院48h后在医院内发生的肺炎。我国医院获得性肺炎发病率为1.3%~3.4%,是第一位的医院内感染,需氧革兰阴性杆菌感染占70%;其次为金黄色葡萄球菌等。

\subsection{治疗原则}

\subsubsection{抗感染治疗}
\paragraph{抗生素经验治疗和针对性治疗的统一}

根据病原微生物选择相应的抗微生物化学治疗(化疗)是肺炎治疗的原则。但微生物学诊断从采集到病原体的分离鉴定需要时间,且诊断的敏感性和特异性不高,为等待病原学诊断延迟初始抗病原微生物治疗会延误治疗时机,从而影响预后。另一方面肺炎的病原微生物以细菌最为常见,抗菌药物的发展使抗菌治疗足以覆盖可能的病原菌,获得治疗成功。因此,细菌性肺炎在获得病原学诊断前,应尽早开始经验性抗菌治疗。经验性治疗应当参考不同类型肺炎病原谱的流行病学资料结合患者具体的临床与影像学特征,估计最可能的病原菌,选择药物和制定治疗方案。在48~72h后对病情再次评价,根据治疗反应和病原学结果调整治疗方案。若病原学检查结果无肯定临床意义,而初始治疗有效则继续原方案治疗。若获得特异性病原学诊断结果,而初始经验治疗方案明显不足或有错,或治疗无反应,则应根据病原学诊断结合药敏试验结果选择敏感抗菌药物,重新拟定治疗方案,此即靶向治疗。所以经验治疗与靶向治疗是治疗过程中的两个不同阶段,是有机的统一。
\paragraph{熟悉抗生素的药理学知识是合理应用抗菌治疗的基础}

每种抗生素的抗菌谱、抗菌活性、药动学、药效学参数、组织穿透力以及在肺泡上皮衬液和呼吸道分泌物中的浓度、不良反应及药物经济学评价是正确选择药物和治疗方案的基础。近年来关于药动学/药效学(pharmacokinetics/pharmacodynamics,PK/PD)的理论对抗生素的临床应用有重要的指导意义。β-内酰胺类和大环内酯类(阿奇霉素除外)属时间依赖性杀菌药物,要求血药浓度高于最低抑菌浓度(minimal
inhibitory
concentration,MIC)的时间占给药间歇时间(T>MIC%)至少达到40%~50%,此类药物半衰期短,抗生素后效应时间短或无抗生素后效应,因此须按半衰期所折算的给药间歇时间每日多次给药,不能任意减少给药次数。氨基糖苷类和喹诺酮类药物属浓度依赖性杀菌药物要求血药峰值浓度与最低抑菌浓度之比(C{max}
/MIC)达到8~10倍,或药时曲线下面积(areas under the
curves,AUC)与最低抑菌浓度之比(AUC/MIC,即AUIC)在G{+}
球菌均达到30、G{-} 杆菌达100以上,才能取得预期临床疗效,避免产生耐药性。
\paragraph{结合本地区耐药情况,参考指南选择药物}

目前包括中国在内的许多国家都制定和颁布了社区获得性和医院获得性肺炎诊治指南,提供了初始经验性治疗的抗菌药物推荐意见。但在各国或同一国家的不同地区,耐药情况不同,因此肺炎的经验性抗生素选择应当结合本国、本地区细菌耐药的流行病学资料认真选择。以下是中华医学会呼吸病学分会《社区获得性肺炎诊断和治疗指南》(2006年修订版)针对部分人群CAP患者初始经验性抗感染治疗的建议(见表\ref{tab12-1})。

\begin{longtable}[]{p{4cm}p{4cm}p{4cm}}
    \caption{CAP患者初始经验性抗感染治疗建议}
    \label{tab12-1}\\
\toprule
\endhead
CAP人群 & 常见病原菌 & 抗菌药物\tabularnewline
\midrule
青壮年、无基础疾病患者 &
肺炎链球菌、肺炎支原体、流感嗜血杆菌、肺炎衣原体等 &
①青霉素类;②多西环素(强力霉素);③大环内酯类;④第一代或第二代头孢菌素;⑤呼吸喹诺酮类(如左氧氟沙星、莫西沙星等)\tabularnewline
老年人或有基础疾病患者 &
肺炎链球菌、流感嗜血杆菌、需氧革兰阴性杆菌、金黄色葡萄球菌、卡他莫拉菌等
&
①第二代头孢菌素(头孢呋辛、头孢丙烯、头孢克洛等)单用或联合大环内酯类;②β内酰胺类/β内酰胺酶抑制剂(如阿莫西林克拉维酸钾、氨苄西林舒巴坦)单用或联合大环内酯类;③呼吸喹诺酮类(如左氧氟沙星、莫西沙星等)\tabularnewline
需入院治疗但不必收住ICU的患者 &
肺炎链球菌、流感嗜血杆菌、混合感染、需氧革兰阴性杆菌、金黄色葡萄球菌、肺炎支原体、肺炎衣原体、呼吸道病毒等
&
①静脉第二代头孢菌素单用或联合静脉大环内酯类;②静脉呼吸喹诺酮类;③静脉β-内酰胺类/β-内酰胺酶抑制剂单用或联合静脉大环内酯类;④头孢噻肟、头孢曲松单用或联合静脉大环内酯类\tabularnewline
需入住ICU的重症患者(无铜绿假单胞菌感染危险因素)
&
肺炎链球菌、需氧革兰阴性杆菌、嗜肺军团菌、肺炎支原体、流感嗜血杆菌、金黄色葡萄球菌等
&
①头孢曲松或头孢噻肟联合静脉注射大环内酯类;②静脉注射呼吸喹诺酮类联合氨基糖苷类;③静脉注射β内酰胺类/β内酰胺酶抑制剂(如阿莫西林克拉维酸钾、氨苄西林舒巴坦)联合静脉注射大环内酯类;④厄他培南联合静脉注射大环内酯类\tabularnewline
需入住ICU的重症患者(有铜绿假单胞菌感染危险因素) & 上组常见病原体+铜绿假单胞菌 &
①具有抗假单胞菌活性的β内酰胺类抗生素(如头孢他啶、头孢吡肟、哌拉西林/他唑巴坦、头孢哌酮/舒巴坦、亚胺培南、美罗培南等)联合静脉注射大环内酯类,必要时还可同时联合氨基糖苷类;②具有抗假单胞菌活性的β内酰胺类抗生素联合静脉注射喹诺酮类;③静脉注射环丙沙星或左氧氟沙星联合氨基糖苷类\tabularnewline
\bottomrule
\end{longtable}

对重症、体弱和昏迷患者,仰卧位会增加胃食管反流和误吸的危险,若无禁忌证,患者均宜采用半卧位可显著减少患者发生胃内容物。

\subsubsection{支持治疗}
\paragraph{一般治疗}

患者应卧床休息,改善营养状况;同时,注意补充水分,维持水电解质和酸碱平衡。高热患者宜用物理降温,必要时可用退热药。
\paragraph{氧疗}

轻症患者无须氧疗,重症患者氧疗是综合治疗的有效措施之一。
\paragraph{雾化、湿化治疗}

由于呼吸道急慢性炎症气管分泌物比较多,有时痰液黏稠不易咳出,同时由于支气管痉挛等因素的存在,除保持呼吸道通畅外,保持呼吸道充分湿化亦是提高抗感染治疗效果的重要措施之一。常用的气溶胶雾化剂有支气管扩张剂、黏稠分泌物溶解剂和肾上腺皮质激素。支气管扩张剂多选用β{2}
受体兴奋剂、茶碱类;黏液溶解剂可选用碳酸氢钠(4%~5%)、α-糜蛋白酶;肾上腺皮质激素选用地塞米松、布地奈德等。
\paragraph{体位痰液引流}

\section{支气管哮喘}

支气管哮喘(bronchial
asthma,简称哮喘)是由多种细胞(如嗜酸性粒细胞、肥大细胞、T淋巴细胞、中性粒细胞、气道上皮细胞等)和细胞组分参与的气道慢性炎症性疾病。这种慢性炎症与气道高反应性相关,通常出现广泛多变的可逆性气流受限,并引起反复发作性的喘息、气急、胸闷或咳嗽等症状,常在夜间和(或)清晨发作、加剧,多数患者可自行缓解或经治疗缓解。哮喘如诊治不及时,随病程的延长可产生气道不可逆性缩窄和气道重塑。而当哮喘得到控制后,多数患者很少发作,严重哮喘发作则更少见。

\subsection{流行病学}

全球约有1.6亿患者。各国患病率不等,国际儿童哮喘和变应性疾病研究显示13~14岁儿童的哮喘患病率为30%以下,我国五大城市的资料显示同龄儿童的哮喘患病率为3%~5%。一般认为儿童患病率高于青壮年,老年人群的患病率有升高的趋势。成人男女患病率大致相同,发达国家高于发展中国家,城市高于农村。约40%的患者有家族史。

\subsection{病因}

哮喘的病因还不十分清楚,患者个体过敏体质及外界环境的影响是发病的危险因素。哮喘与多基因遗传有关,同时受遗传因素和环境因素的双重影响。许多调查资料表明,哮喘患者亲属患病率高于群体患病率,并且亲缘关系越近,患病率越高;患者病情越严重,其亲属患病率也越高。目前,哮喘的相关基因尚未完全明确,但有研究表明存在有与气道高反应性、IgE调节和特应性反应相关的基因,这些基因在哮喘的发病中起着重要作用。

环境因素中主要包括某些激发因素,如尘螨、花粉、真菌、动物毛屑、二氧化硫、氨气等各种特异和非特异性吸入物;感染,如细菌、病毒、原虫、寄生虫等;食物,如鱼、虾、蟹、蛋类、牛奶等;药物,如普萘洛尔、阿司匹林等。同时,气候变化、运动、妊娠等也可能是哮喘的激发因素。

\subsection{临床表现}

临床表现为发作性伴有哮鸣音的呼气性呼吸困难或发作性胸闷和咳嗽。严重者被迫采取坐位或呈端坐呼吸,干咳或咳大量白色泡沫痰,甚至出现发绀等,有时咳嗽可为唯一的症状(咳嗽变异型哮喘)。哮喘症状可在数分钟内发作,经数小时至数日,用支气管舒张药或自行缓解。某些患者在缓解数小时后可再次发作。在夜间及凌晨发作和加重常是哮喘的特征之一。有些青少年的哮喘症状表现为运动时出现胸闷、咳嗽和呼吸困难(运动性哮喘)。

发作时胸部呈过度充气状态,有广泛的哮鸣音,呼气音延长。但在轻度哮喘或非常严重哮喘发作时,哮鸣音可不出现。心率增快、奇脉、胸腹反常运动和发绀常出现在严重哮喘患者中。非发作期体检可无异常。

\subsection{诊断}

\subsubsection{诊断标准}

(1)反复发作喘息、气急、胸闷或咳嗽,多与接触变应原、冷空气、物理、化学性刺激、病毒性上呼吸道感染、运动等有关。

(2)发作时在双肺可闻及散在或弥漫性,以呼气相为主的哮鸣音,呼气相延长。

(3)上述症状可经治疗缓解或自行缓解。

(4)除外其他疾病所引起的喘息、气急、胸闷和咳嗽。

(5)临床表现不典型者(如无明显喘息或体征)应有下列三项中至少一项阳性:①支气管激发试验或运动试验阳性;②支气管舒张试验阳性;③昼夜PEF变异率≥20%。

符合(1)~(4)条或(4)(5)条者,可以诊断为支气管哮喘。

\subsubsection{支气管哮喘的分期及控制水平分级}

支气管哮喘可分为急性发作期和非急性发作期。
\paragraph{急性发作期}

急性发作期是指气促、咳嗽、胸闷等症状突然发生或症状加重,常有呼吸困难,以呼气流量降低为特征,常因接触变应原等刺激物或治疗不当所致。哮喘急性发作时程度轻重不一,病情加重可在数小时或数日内出现,偶尔可在数分钟内即危及生命,故应对病情做出正确评估,以便给予及时有效地紧急治疗。哮喘急性发作时严重程度可分为轻度、中度、重度和危重4级。
\paragraph{非急性发作期(亦称慢性持续期)}

许多哮喘患者即使没有急性发作,但在相当长的时间内仍有不同频度和(或)不同程度出现症状(喘息、咳嗽、胸闷等),肺通气功能下降。过去曾以患者白天、夜间哮喘发作的频度和肺功能测定指标为依据,将非急性发作期的哮喘病情严重程度分为间歇性、轻度持续、中度持续和重度持续4级,目前则认为长期评估哮喘的控制水平是更为可靠和有用的严重性评估方法,对哮喘的评估和治疗指导意义更大。哮喘控制水平分为控制、部分控制和未控制3个等级。

\subsection{治疗}

目前尚无特效的治疗方法,但长期规范化治疗可使哮喘症状能得到控制,减少复发乃至不发作。长期使用最少量或不用药物能使患者活动不受限制,并能与正常人一样生活、工作和学习。

\subsubsection{脱离变应原}

部分患者能找到引起哮喘发作的变应原或其他非特异刺激因素,立即使患者脱离变应原的接触是防治哮喘最有效的方法。

\subsubsection{药物治疗}

治疗哮喘药物主要分为两类。
\paragraph{缓解哮喘发作}

此类药物主要作用为舒张支气管,故也称支气管舒张药。

1)β{2} 肾上腺素受体激动剂(简称β{2} 激动剂)

β{2} 激激动剂主要通过激动呼吸道的β{2}
受体,激活腺苷酸环化酶,使细胞内的环磷酸腺苷(cAMP)含量增加,游离\ce{Ca^2+}
减少,从而松弛支气管平滑肌,是控制哮喘急性发作的首选药物。常用的短效β受体激动剂有沙丁胺醇(Salbutamol)、特布他林(Terbutaline)和非诺特罗(Fenoterol),作用时间约为4~6h。长效β{2}
受体激动剂有福莫特罗(Formoterol)、沙美特罗(Salmaterol)及丙卡特罗(Procaterol),作用时间为10~12h。长效β{2}
激动剂尚具有一定的抗气道炎症,增强黏液-纤毛运输功能的作用。不主张长效β{2}
受体激动剂单独使用,须与吸入激素联合应用。但福莫特罗可作为应急缓解气道痉挛的药物。肾上腺素、麻黄碱和异丙肾上腺素,因其心血管不良反应多而已被高选择性的β{2}
激动剂所代替。

用药方法可采用吸入,包括定量气雾剂(MDI)吸入、干粉吸入、持续雾化吸入等,也可采用口服或静脉注射。首选吸入法,因药物吸入气道直接作用于呼吸道,局部浓度高且作用迅速,所用剂量较小,全身性不良反应少。常用剂量为沙丁胺醇或特布他林MDI,每喷100μg,每日3或4次,每次1或2喷。通常5~10min即可见效,可维持4~6h。长效β{2}
受体激动剂如福莫特罗4.5μg,每日2次,每次1喷,可维持12h。应教会患者正确掌握MDI吸入方法。干粉吸入方法较易掌握。持续雾化吸入多用于重症和儿童患者,使用方法简单,易于配合。如沙丁胺醇5mg稀释在5~20mL溶液中雾化吸入。沙丁胺醇或特布他林一般口服用法为2.4~2.5mg,每日3次,15~30min起效,但心悸、骨骼肌震颤等不良反应较多。β{2}
激动剂的缓释型及控制型制剂疗效维持时间较长,用于防治反复发作性哮喘和夜间哮喘,为注射用药,用于严重哮喘。一般每次用量为沙丁胺醇0.5mg,滴速2~4μg/min,易引起心悸,只在其他疗法无效时使用。

2)抗胆碱药

吸入抗胆碱药如异丙托溴铵(Ipratropine
Bromide),为胆碱能受体(M受体)拮抗剂,可以阻断节后迷走神经通路,降低迷走神经兴奋性而起舒张支气管作用,并有减少痰液分泌的作用。与β{2}
受体激动剂联合吸入有协同作用,尤其适用于夜间哮喘及多痰的患者。可用MDI,每日3次,每次25~75μg或用100~150μg/mL的溶液持续雾化吸入。约10min起效,维持4~6h。不良反应少,少数患者有口苦或口干感。近年发展的选择性M{1}
、M{3}
受体拮抗剂如噻托溴铵作用更强,持续时间更久(可达24h),不良反应更少。

3)茶碱类

茶碱类除能抑制磷酸二酯酶,提高平滑肌细胞内的cAMP浓度外,还能拮抗腺苷受体;刺激肾上腺分泌肾上腺素,增强呼吸肌的收缩;增强气道纤毛清除功能和抗感染作用,是目前治疗哮喘的有效药物。茶碱与糖皮质激素合用具有协同作用。

口服给药:包括氨茶碱和控(缓)释茶碱,后者因其昼夜血药浓度平稳,不良反应较少,可维持较好的治疗浓度,平喘作用可维持12~24h,可用于控制夜间哮喘。一般剂量每日6~10mg/kg,用于轻度或中度哮喘。静脉注射氨茶碱首次剂量为4~6mg/kg,注射速度不宜超过0.25mg/(kg·min),静脉滴注维持量为0.6~0.8mg/(kg·h),日注射量一般不超过1.0g。静脉给药主要应用于重、危症哮喘。

茶碱的主要不良反应为胃肠道症状(恶心、呕吐)、心血管症状(心动过速、心律失常、血压下降)及尿多,偶可兴奋呼吸中枢,严重者可引起抽搐乃至死亡。最好在用药中监测血浆氨茶碱浓度,其安全有效浓度为6~15μg/mL。发热、妊娠、小儿或老年患者,以及患有肝、心、肾功能障碍及甲状腺功能亢进者尤须慎用。合用西咪替丁(甲氰咪胍)、喹诺酮类、大环内酯类药物等可影响茶碱代谢而使其排泄减慢,应减少用药量。
\paragraph{控制或预防哮喘发作}

此类药物主要治疗哮喘的气道炎症,亦称抗炎药。

1)糖皮质激素

由于哮喘的病理基础是慢性非特异性炎症,糖皮质激素是当前控制哮喘发作最有效的药物。主要作用机制是抑制炎症细胞的迁移和活化;抑制细胞因子的生成;抑制炎症介质的释放;增强平滑肌细胞β{2}
受体的反应性。糖皮质激素可分为吸入、口服和静脉用几种剂型。

(1)吸入剂型。吸入治疗是目前推荐长期抗炎治疗哮喘的最常用方法。常用吸入药物有倍氯米松(Beclometasone,BDP)、布地奈德(Budesonide)、氟替卡松(Fluticasone)、莫米松(Momethasone)等,后两者生物活性更强,作用更持久。通常需规律吸入1周以上方能生效。根据哮喘病情,吸入剂量轻度持续者一般每日(BDP或等效量其他皮质激素)为200~500μg,中度持续者一般每日500~1000μg,重度持续者一般每日>1000μg(不宜超过每日2000μg)(氟替卡松剂量减半)。吸入治疗药物全身性不良反应少,少数患者可引起口咽假丝酵母菌感染、声音嘶哑或呼吸道不适,吸药后用清水漱口可减轻局部反应和胃肠吸收。长期使用较大剂量(>每日1000μg)者应注意预防全身性不良反应,如肾上腺皮质功能抑制、骨质疏松等。为减少吸入大剂量糖皮质激素的不良反应,可与长效β{2}
受体激动剂、控释茶碱或白三烯受体拮抗剂联合使用。

(2)口服剂:有泼尼松(强的松)、泼尼松龙(强的松龙)。用于吸入糖皮质激素无效或需要短期加强的患者。起始每日30~60mg,症状缓解后逐渐减量至≤10mg/d,然后停用,或改用吸入剂。

(3)静脉用药:重度或严重哮喘发作时应及早应用琥珀酸氢化可的松,注射后4~6h起效,常用量为100~400mg/d,或甲泼尼龙(甲基泼尼松龙80~160mg/d)起效时间更短(2~4h)。地塞米松因在体内半衰期较长、不良反应较多,宜慎用,一般为10~30mg/d。症状缓解后逐渐减量,然后改口服和吸入制剂维持。

2)LT调节剂

通过调节LT的生物活性而发挥抗炎作用,同时具有舒张支气管平滑肌。可以作为轻度哮喘的一种控制药物的选择。常用半胱氨酸LT受体拮抗剂,如孟鲁司特(Montelukast)10mg每日1次,或扎鲁司特(Zafirlukast)20mg每日2次,不良反应通常较轻微,主要是胃肠道症状,少数有皮疹、血管性水肿、转氨酶升高,停药后可恢复正常。

3)其他药物

酮替酚(Ketotifen)和新一代组胺H{1}
受体拮抗剂阿司咪唑、曲尼斯特、氯雷他定对轻症哮喘和季节性哮喘有一定效果,也可与β{2}
受体激动剂联合用药。

\subsubsection{急性发作期的治疗}

急性发作的治疗目的是尽快缓解气道阻塞,纠正低氧血症,恢复肺功能,预防进一步恶化或再次发作,防止并发症。一般根据病情的严重程度进行综合性治疗。
\paragraph{轻度}

每日定时吸入糖皮质激素(200~500μg BDP),出现症状时吸入短效β{2}
受体激动剂,可间断吸入。效果不佳时可加用口服β{2}
受体激动剂控释片或小量茶碱控释片(每日200mg),或加用抗胆碱药如异丙托溴铵气雾剂吸入。
\paragraph{中度}

吸入剂量一般为每日500~1000μg BDP;规则吸入β{2}
激动剂或联合抗胆碱药吸入或口服长效β{2}
受体激动剂。亦可加用口服LT拮抗剂,若不能缓解,可持续雾化吸入β{2}
受体激动剂(或联合用抗胆碱药吸入),或口服糖皮质激素(每日<60mg)。必要时可用氨茶碱静脉注射。
\paragraph{重度至危重度}

持续雾化吸入β{2}
受体激动剂,或合并抗胆碱药;或静脉滴注氨茶碱或沙丁胺醇;加用口服LT拮抗剂。静脉滴注糖皮质激素如琥珀酸氢化可的松或甲泼尼龙或地塞米松(剂量见前)。待病情得到控制和缓解后(一般3~5d),改为口服给药。注意维持水、电解质平衡,纠正酸碱失衡,当pH值<7.2且合并代谢性酸中毒时,应适当补碱;可给予氧疗,如病情恶化缺氧不能纠正时,进行无创通气或插管机械通气。若并发气胸,在胸腔引流气体下仍可机械通气。此外,应预防下呼吸道感染等。

\subsubsection{哮喘非急性发作期的治疗}

一般哮喘经过急性期治疗症状得到控制,但哮喘的慢性炎症病理生理改变仍然存在,因此,必须制定哮喘的长期治疗方案。根据哮喘的控制水平选择合适的治疗方案。对哮喘患者进行哮喘知识教育和控制环境、避免诱发因素贯穿于整个治疗阶段。

由于哮喘的复发性以及多变性,需不断评估哮喘的控制水平,治疗方法则依据控制水平进行调整。如果目前的治疗方案不能够使哮喘得到控制,治疗方案应该升级直至达到哮喘控制为止。当哮喘控制维持至少3个月后,治疗方案可以降级。通常情况下,患者在初诊后1~3个月回访,以后每3个月随访1次。如出现哮喘发作时,应在2周~1个月内进行回访。对大多数控制剂来说,最大的治疗效果可能要在3~4个月后才能显现,只有在这种治疗策略维持3~4个月后,仍未达到哮喘控制,才考虑增加剂量。对所有达到控制的患者,必须通过常规跟踪及阶段性减少剂量来寻求最小控制剂量。大多数患者可以达到并维持哮喘控制,但一部分难治性哮喘患者可能无法达成同样水平的控制。

以上方案为基本原则,但必须个体化,联合应用,以最小量、最简单的联合,不良反应最少,达到最佳控制症状为原则。

\section{肺结核}

肺结核(pulmonary
tuberculosis)仍然是严重危害人类健康的主要传染病,是全球关注的公共卫生和社会问题,也是我国重点控制的主要疾病之一。从20世纪60年代起,结核病化学治疗已取代过去消极的“卫生营养疗法”,成为公认的控制结核病的主要武器,使新发现的结核病治愈率达到95%以上。但20世纪80年代中期以来,结核病出现全球性恶化趋势,大多数结核病疫情很低的发达国家发现结核病卷土重来,众多发展中国家的结核病疫情出现明显回升。结核病在许多国家和地区失控的主要原因:一方面是人免疫缺陷病毒感染的流行、多重耐药(至少耐异烟肼和利福平)结核分枝杆菌感染的增多、贫困、人口增长和移民等客观因素;另一方面则是由于缺乏对结核病流行回升的警惕性和结核病控制复杂性的深刻认识,误认为结核病问题已解决,因而放松和削弱对结核病控制工作的投入和管理等主观因素所致。

\subsection{流行病学}

\subsubsection{全球疫情}

全球有三分之一的人(约20亿)曾受到结核分枝杆菌的感染。结核病的流行状况与经济水平大致相关,结核病的高流行与国民生产总值的低水平相对应。世界卫生组织把印度、中国、俄罗斯、南非、秘鲁等22个国家列为结核病高负担、高危险性国家。全球80%的结核病例集中在这些国家。无疑这些国家的结核病控制将对全球的结核病形势产生重要影响。

\subsubsection{我国疫情}

当前我国的结核病疫情特点如下。

(1)高感染率:年结核分枝杆菌感染率为0.72%。全国有近半的人口,约5.5亿曾受到结核分枝杆菌感染,城市人群的感染率高于农村。

(2)高肺结核患病率:2000年活动性肺结核患病率、痰涂片阳性(简称涂阳)及或培养阳性(简称菌阳)肺结核患病率分别为367/10万、122/10万和160/10万,估算病例数分别约为500万、150万和200万。中青年患病多,15~59岁年龄段的涂阳肺结核患者数占全部涂片阳性患者的61.6%。

(3)死亡人数多:每年约有13万人死于结核病。

(4)地区患病率差异大:西部地区活动性肺结核患病率、涂片阳性肺结核和培养阳性肺结核患病率明显地高于全国平均水平,而东部地区低于平均水平。

\subsection{临床表现}

各型肺结核的临床表现不尽相同,但有共同之处。
\paragraph{呼吸系统症状}

(1)咳嗽咳痰:是肺结核最常见症状。咳嗽较轻,干咳或少量黏液痰。有空洞形成时,痰量增多,若合并其他细菌感染,痰可呈脓性。若合并支气管结核,表现为刺激性咳嗽。

(2)咯血:约1/3~1/2的患者有咯血。咯血量多少不定,多数患者为少量咯血,少数为大咯血。

(3)胸痛:结核累及胸膜时可表现胸痛,为胸膜性胸痛。随呼吸运动和咳嗽加重。

(4)呼吸困难:多见于干酪样肺炎和大量胸腔积液患者。
\paragraph{全身症状}

发热为最常见症状,多为长期午后潮热,即下午或傍晚开始升高,翌晨降至正常。部分患者有倦怠乏力、盗汗、食欲减退和体重减轻等。育龄女性患者可以有月经不调。

\subsection{结核病的化学治疗}

\subsubsection{化学治疗的原则}

肺结核化学治疗的原则是早期、规律、全程、适量、联合。整个治疗方案分强化和巩固两个阶段。

(1)早期:对所有检出和确诊患者均应立即给予化学治疗。早期化学治疗有利于迅速发挥早期杀菌作用,促使病变吸收和减少传染性。

(2)规律:严格遵照医嘱要求规律用药,不漏服,不停药,以避免耐药性的产生。

(3)全程:保证完成规定的治疗期是提高治愈率和减少复发率的重要措施。

(4)适量:严格遵照适当的药物剂量用药,药物剂量过低不能达到有效的血浓度,影响疗效和易产生耐药性,剂量过大易发生药物不良反应。

(5)联合:联合用药系指同时采用多种抗结核药物治疗,可提高疗效,同时通过交叉杀菌作用减少或防止耐药性的产生。

\subsubsection{化学治疗的生物学机制}
\paragraph{药物对不同代谢状态和不同部位的结核分枝杆菌群的作用}

结核分枝杆菌根据其代谢状态分为A、B、C、D四群。

(1)A菌群:快速繁殖,大量的A菌群多位于巨噬细胞外和肺空洞干酪液化部分,占结核分枝杆菌群的绝大部分。由于细菌数量大,易产生耐药变异菌。

(2)B菌群:处于半静止状态,多位于巨噬细胞内酸性环境中和空洞壁坏死组织中。

(3)C菌群:处于半静止状态,可有突然间歇性短暂的生长繁殖,许多生物学特点尚不十分清楚。

(4)D菌群:处于休眠状态,不繁殖,数量很少。抗结核药物对不同菌群的作用各异。

抗结核药物对A菌群作用强弱依次为异烟肼>链霉素>利福平>乙胺丁醇;对B菌群依次为吡嗪酰胺>利福平>异烟肼;对C菌群依次为利福平>异烟肼。随着药物治疗作用的发挥和病变变化,各菌群之间也互相变化。通常大多数结核药物可以作用于A菌群,异烟肼和利福平具有早期杀菌作用,即在治疗的48h内迅速发挥杀菌作用,使菌群数量明显减少,传染性减少或消失,痰菌阴转。这显然对防止获得性耐药的产生有重要作用。B和C菌群由于处于半静止状态,抗结核药物的作用相对较差,有“顽固菌”之称,杀灭B和C菌群可以防止复发,抗结核药物对D菌群无作用。
\paragraph{耐药性}

耐药性是基因突变引起的药物对突变菌的效力降低。治疗过程中如单用一种敏感药,菌群中大量敏感菌被杀死,但少量的自然耐药变异菌仍存活,并不断繁殖,最后逐渐完全替代敏感菌而成为优势菌群。结核病变中结核菌群数量愈大,则存在的自然耐药变异菌也愈多。现代化学治疗多采用联合用药,通过交叉杀菌作用防止耐药性产生。联合用药后中断治疗或不规律用药仍可产生耐药性。其产生机制是各种药物开始早期杀菌作用速度的差异,某些菌群只有一种药物起灭菌作用,是因菌群再生长期间和菌群延缓生长期药物抑菌浓度存在差异所造成的结果。因此,强调在联合用药的条件下也不能中断治疗,短程疗法最好应用全程督导化疗。
\paragraph{间歇化学治疗}

间歇化学治疗的主要理论基础是结核分枝杆菌延缓生长期。结核分枝杆菌接触不同的抗结核药物后产生不同时间的延缓生长期。如接触异烟肼和利福平24h后分别可有6~9d和2~3d的延缓生长期。药物使结核分枝杆菌产生延缓生长期,就有间歇用药的可能性,而氨硫脲没有延缓生长期,就不适于间歇应用。
\paragraph{顿服}

抗结核药物血中高峰浓度的杀菌作用要优于经常性维持较低药物浓度水平的情况。每日剂量1次顿服要比每日2次或3次分服所产生的高峰血浓度高3倍左右。临床研究已经证实顿服的效果优于分次口服。

\subsubsection{常用抗结核病药物}
\paragraph{异烟肼(Isoniazid,INH,H)}

异烟肼仍然是单一抗结核药物中杀菌力,特别是早期杀菌力最强者。INH对巨噬细胞内外的结核分枝杆菌均具有杀菌作用。最低抑菌浓度为0.025~0.05μg/mL。口服后迅速吸收,血中药物浓度可达最低抑菌浓度的20~100余倍,脑脊液中药物浓度也很高。用药后经乙酰化而灭活,乙酰化的速度决定于遗传因素。成人剂量每日300mg,顿服;儿童为每日5~10mg/kg,最大剂量每日不超过300mg。结核性脑膜炎和血行播散型肺结核的用药剂量可加大,儿童20~30mg/kg,成人10~20mg/kg。偶可发生药物性肝炎,肝功能异常者慎用,需注意观察。如果发生周围神经炎可服用维生素B{6}
(吡哆醇)。
\paragraph{利福平(Rifampicin,RFP,R)}

最低抑菌浓度为0.06~0.25μg/mL,对巨噬细胞内外的结核分枝杆菌均有快速杀菌作用,特别是对C菌群有独特的杀灭菌作用。INH与RFP联用可显著缩短疗程。口服1~2h后达血高峰浓度,半衰期为3~8h,有效血浓度可持续6~12h,药量加大持续时间更长。口服后药物集中在肝脏,主要经胆汁排泄,胆汁药物浓度可达200μg/mL。未经变化的药物可再经肠吸收,形成肠肝循环,能保持较长时间的高峰血浓度,故推荐早晨空腹或早饭前0.5h服用。利福平及其代谢物为橘红色,服后大小便、眼泪等为橘红色。成人剂量为每日8~10mg/kg,体重在50kg及以下者为450mg,50kg以上者为600mg,顿服。儿童每日10~20mg/kg。间歇用药为600~900mg,每周2或3次。用药后如出现一过性转氨酶上升可继续用药,加保肝治疗观察,如出现黄疸应立即停药。流感样症状、皮肤综合征、血小板计数减少多在间歇疗法出现。妊娠3个月以内者忌用,超过3个月者要慎用。其他利福霉素类药物有利福喷丁(Rifapentine,RFT),该药血清峰浓度(C{max}
)为10~30μg/mL,半衰期为12~15h。RFT的最低抑菌浓度为0.015~0.06μg/mL,比RFP低很多。上述特点说明RFT适于间歇使用,使用剂量为450mg~600mg,每周2次。RFT与RFP之间完全交叉耐药。
\paragraph{吡嗪酰胺(Pyrazinamide,PZA,Z)}

吡嗪酰胺具有独特的杀灭菌作用,主要是杀灭巨噬细胞内酸性环境中的B菌群。在6个月标准短程化疗中,PZA与INH和RFP联合用药是第3个不可或缺的重要药物。对于新发现初治涂阳患者PZA仅在前2个月使用,因为使用2个月的效果与使用4或6个月的效果相似。成人用药剂量为每日1.5g,每周3次用药为每日1.5~2.0g,儿童每日为30~40mg/kg。常见不良反应为高尿酸血症、肝功能损害、食欲不振、关节痛和恶心。
\paragraph{乙胺丁醇(Ethambutol,EMB,E)}

乙胺丁醇对结核分枝杆菌的最低抑菌浓度为0.95~7.5μg/mL,口服易吸收,成人剂量为每日0.75~1.0g,每周3次用药为每日1.0~1.25g。不良反应为视神经炎,应在治疗前测定视力与视野,治疗中密切观察,提醒患者发现视力异常应及时就医。鉴于儿童无症状判断能力,故不用。
\paragraph{链霉素(Streptomycin,SM,S)}

链霉素对巨噬细胞外碱性环境中的结核分枝杆菌有杀菌作用。肌内注射,每日量为0.75g,每周5次;间歇用药每次为0.75~1.0g,每周2或3次。不良反应主要为耳毒性、前庭功能损害和肾毒性等,严格掌握使用剂量,儿童、老人、孕妇、听力障碍和肾功能不良者要慎用或不用。

\subsubsection{统一标准化学治疗方案}

为充分发挥化学治疗在结核病防治工作中的作用,便于大面积开展化学治疗,解决滥用抗结核药物、化疗方案不合理和混乱造成的治疗效果差、费用高、治疗期过短或过长、药物供应和资源浪费等实际问题,在全面考虑到化疗方案的疗效、不良反应、治疗费用、患者接受性和药源供应等条件下,且经国内外严格对照研究证实的化疗方案,可供选择作为统一标准方案。实践证实,严格执行统一标准方案确能达到预期效果,符合投入效益的原则。
\paragraph{初治涂阳肺结核治疗方案(含初治涂阴有空洞形成或粟粒型肺结核)}

1)每日用药方案

(1)强化期:异烟肼、利福平、吡嗪酰胺和乙胺丁醇,顿服,2个月。

(2)巩固期:异烟肼、利福平,顿服,4个月。简写为2HRZE/4HR。

2)间歇用药方案

(1)强化期:异烟肼、利福平、吡嗪酰胺和乙胺丁醇,隔日1次或每周3次,2个月。

(2)巩固期:异烟肼、利福平,隔日1次或每周3次,4个月。简写为2H3R3Z3E3/4H3R3。
\paragraph{复治涂阳肺结核治疗方案}

1)每日用药方案

(1)强化期:异烟肼、利福平、吡嗪酰胺、链霉素和乙胺丁醇,每日1次,2个月。

(2)巩固期:异烟肼、利福平和乙胺丁醇,每日1次,4~6个月。巩固期治疗4个月痰菌未转阴,可继续延长治疗期2个月。简写为2HRZSE/4-6HRE。

2)间歇用药方案

(1)强化期:异烟肼、利福平、吡嗪酰胺、链霉素和乙胺丁醇,隔日1次或每周3次,2个月。

(2)巩固期:异烟肼、利福平和乙胺丁醇,隔日1次或每周3次,6个月。简写为2H3R3Z3S3E3/6H3R3E3。
\paragraph{初治涂阴肺结核治疗方案}

1)每日用药方案

(1)强化期:异烟肼、利福平、吡嗪酰胺,每日1次,2个月。

(2)巩固期:异烟肼、利福平,每日1次,4个月。简写为2HRZ/4HR。

2)间歇用药方案

(1)强化期:异烟肼、利福平、吡嗪酰胺,隔日1次或每周3次,2个月。

(2)巩固期:异烟肼、利福平,隔日1次或每周3次,4个月。简写为2H3R3Z3/4H3R3。

上述间歇方案为我国结核病规划所采用,但必须采用全程督导化疗管理,以保证患者不间断地规律用药。

\subsubsection{耐药肺结核}

耐药结核病,特别是耐多药结核病:至少耐异烟肼和利福平,和当今出现的超级耐多药结核病(extensive
drug resistarit or extreme drug
resistant,XDR-TB)。除耐异烟肼和利福平外,还耐二线抗结核药物,对全球结核病控制构成严峻的挑战。世界卫生组织估算全球MDR-TB约有100万例,治愈率低,病死率高(特别是发生在HIV感染的病例),治疗费用昂贵,传染危害大。我国为耐多药结核病高发国家之一,初始耐药率为18.6%,获得性耐药率为46.5%,初始耐多药率和获得性耐多药率分别为7.6%和17.1%。

制定MDR-TB治疗方案应注意:详细了解患者的用药史,尽量用药敏试验结果指导治疗,治疗方案至少含4种可能的敏感药物,药物至少每周使用6d。吡嗪酰胺、乙胺丁醇、氟喹诺酮应每天用药,二线药物根据患者耐受性也可每天1次用药或分次服用;药物剂量依体重决定;氨基糖苷类或卷曲霉素注射剂类药物至少使用6个月;治疗期在痰涂片和培养阴转后至少治疗18个月,有广泛病变的应延长至24个月;吡嗪酰胺可考虑全程使用。

MDR-TB治疗药物第1组药为一线抗结核药,依据药敏试验和用药史选择使用。第2组药为注射剂,如菌株敏感链霉素为首选,次选为卡那霉素和阿米卡星,两者效果相似并存在百分之百的交叉耐药;如对链霉素和卡那霉素耐药,应选择卷曲霉素。卷曲霉素和链霉素效果相似并有高的交叉耐药。第3组为氟喹诺酮类药,菌株敏感按效果从高到低选择是莫西沙星=加替沙星>左氧氟沙星>氧氟沙星=环丙沙星。第4组为口服抑菌二线抗结核药,首选为乙硫异烟胺/丙硫异烟胺,该药疗效确定且价廉,应用从小剂量250mg开始,3~5d后加大至足量。PAS也应考虑为首选,只是价格贵些。环丝氨酸国内使用较少。氨硫脲不良反应较多,因而使用受到限制。第5组药物,疗效不确定,只有当1~4组药物无法制定合理方案时,方可考虑。

MDR-TB治疗方案通常含两个阶段:强化期(注射剂使用)和继续期(注射剂停用),治疗方案采用标准代码,例如6Z-Km(Cm)-Ofx-Eto-Cs/12Z-Ofx-Eto-Cs,初始强化期含5种药,治疗6个月,注射剂停用后,口服药持续至少12个月,总疗期18个月。注射剂为卡那霉素(Km),也可选择卷曲霉素(Cm)。

预防耐药结核的发生最佳策略是加强实施DOTS策略,使初治涂阳患者在良好管理下达到高治愈率。另一方面加强对MDR-TB的及时发现和给予合理治疗以阻止其传播。

\subsection{其他治疗}

\subsubsection{对症治疗}

肺结核的一般症状在合理化疗下很快减轻或消失,无须特殊处理。咯血是肺结核的常见症状,在活动性和痰涂阳肺结核患者中,咯血症状分别占30%和40%。咯血处置要注意镇静、止血,患侧卧位,预防和抢救因咯血所致的窒息并防止肺结核播散。

一般少量咯血,多以安慰患者、消除紧张、卧床休息为主,可用氨基己酸、氨甲苯酸(止血芳酸)、酚磺乙胺(止血敏)、卡络柳钠(安络血)等药物止血。大咯血时先用垂体后叶素5~10IU加入25%葡萄糖液40mL中缓慢静脉注射,一般为15~20min,然后将垂体后叶素加入5%葡萄糖液按0.1IU/(kg·h)速度静脉滴注。垂体后叶素收缩小动脉,使肺循环血量减少而达到较好止血效果。高血压、冠心病、心衰患者和孕妇禁用。对支气管动脉破坏造成的大咯血可采用支气管动脉栓塞法。在大咯血时,患者突然停止咯血,并出现呼吸急促、面色苍白、口唇发给、烦躁不安等症状时,常为咯血窒息,应及时抢救。置患者头低足高45°的俯卧位,同时拍击健侧背部,保持充分体位引流,尽快使积血和血块由气管排出,或直接刺激咽部以咳出血块。有条件时可进行气管插管,硬质支气管镜吸引或气管切开。

\subsubsection{糖皮质激素}

糖皮质激素在结核病的应用主要是利用其抗炎、抗毒性作用。仅用于结核毒性症状严重者。必须确保在有效抗结核药物治疗的情况下使用。使用剂量依病情而定,一般用泼尼松口服每日20mg,顿服,1~2周,以后每周递减5mg,用药时间为4~8周。

\subsubsection{肺结核外科手术治疗}

当前肺结核外科手术治疗主要的适应证是经合理化学治疗后无效、多重耐药的厚壁空洞、大块干酪灶、结核性脓胸、支气管胸膜瘘和大咯血保守治疗无效者。

\subsection{肺结核与相关疾病}

\subsubsection{HIV/AIDS}

截至2004年底,全球共有HIV/AIDS约3940万例,其中2004年HIV新感染者约为490万例,因HIV/AIDS死亡者310万例。在HIV/AIDS死亡病例中,至少有1/3病例是由HIV/AIDS与结核的双重感染所致。HIV/AIDS与结核病双重感染病例的临床表现是症状和体征多,如体重减轻、长期发热和持续性咳嗽等,全身淋巴结肿大,可有触痛,肺部X线经常出现肿大的肺门纵隔淋巴结团块,下叶病变多见,胸膜和心包有渗出等,结核菌素试验常为阴性,应多次查痰。治疗过程中常出现药物不良反应,易产生获得性耐药。治疗仍以6个月短程化疗方案为主,可适当延长治疗时间,一般预后差。

\subsubsection{肝炎}

异烟肼、利福平和吡嗪酰胺均有潜在的肝毒性作用,用药前和用药过程中应定期监测肝功能。严重肝功能损害的发生率为1%,但约20%患者可出现无症状的轻度转氨酶升高,无须停药,但应注意观察,绝大多数的转氨酶可恢复正常。如有食欲不良、黄疸或肝大应立即停药,直至肝功能恢复正常。在传染性肝炎流行区,确定肝炎的原因比较困难。如肝炎严重,肺结核又必须治疗,可考虑使用2SHE/10HE方案。

\subsubsection{糖尿病}

糖尿病合并肺结核的发病率有逐年升高趋势。两病互相影响,糖尿病对肺结核治疗的不利影响比较显著,必须在控制糖尿病的基础上肺结核的治疗才能奏效。肺结核合并糖尿病的化疗原则与单纯肺结核相同,只是治疗期可适当延长。

\subsubsection{矽肺(硅沉着病)}

矽肺患者是并发肺结核的高危人群。近年来,随着矽肺合并肺结核的比例不断上升,Ⅲ期矽肺患者合并肺结核的比例可高达50%以上。矽肺合并结核的诊断强调多次查痰,特别是采用培养法。矽肺合并结核的治疗与单纯肺结核的治疗相同。Ⅰ期和Ⅱ期矽肺合并肺结核的治疗效果与单纯肺结核的治疗相同。药物预防性治疗是防止矽肺并发肺结核的有效措施,使用方法为INH每日300mg,6~12个月,可减少发病约70%。

\subsection{结核病控制策略与措施}

\subsubsection{全程督导化学治疗}

全程督导化疗是指肺结核患者在治疗过程中,每次用药都必须在医务人员的直接监督下进行,因故未用药时必须采取补救措施以保证按医嘱规律用药。督导化疗可以提高治疗依从性,保证规律用药,因而能够显著提高治愈率,降低复发率并减少死亡,能够使患病率快速下降并减少多耐药病例的发生,符合投入效益的原则。

\subsubsection{病例报告和转诊}

按《中华人民共和国传染病防治法》,肺结核属于乙类传染病。各级医疗预防机构要专人负责,做到及时、准确、完整地报告肺结核疫情。同时要做好转诊工作,转诊对象为肺结核、疑似肺结核患者。乡镇卫生院和没有能力进行X线诊断和痰结核分枝杆菌检查的医疗机构应将肺结核可疑症状者推荐到结核病防治机构进行检查。

\subsubsection{病例登记和管理}

由于肺结核病程较长、易复发和具有传染性等特点,必须长期随访,掌握患者从发病、治疗到治愈的全过程。通过对确诊肺结核病例的登记达到掌握疫情和便于管理的目的。通过病例登记,医务人员就能够在督促规律用药、按时复查、指导预防家庭内传染以及动员新发现患者的家庭接触者检查等方面采取主动措施。

\subsubsection{卡介苗接种}

迄今,卡介苗问世已80余年,在182个国家和地区约40多亿儿童接种了卡介苗。卡介苗接种的效果远不如脊髓灰质炎糖丸和牛痘在预防小儿麻痹和天花方面那么理想。目前新结核疫苗的研究正在积极进行之中。普遍认为卡介苗接种对预防成年人肺结核的效果很差,但对预防由血行播散引起的结核性脑膜炎和粟粒型结核有一定作用。新生儿进行卡介苗接种后,仍须注意采取与肺结核患者隔离的措施。

\subsubsection{预防性化学治疗}

主要应用于受结核分枝杆菌感染易发病的高危人群,包括HIV感染者、涂阳肺结核患者的密切接触者、肺部硬结纤维病灶(无活动性)、矽肺、糖尿病、长期使用糖皮质激素或免疫抑制剂者、吸毒者、营养不良者、35岁以下结核菌素试验硬结直径达15mm者等。常用异烟肼300mg/d,顿服6~8个月,儿童用量为4~8mg/kg,或利福平和异烟肼3个月,每日顿服或每周3次。效果经观察与对照组比较可减少发病60%~80%。在一些资金短缺的国家和地区应优先把资金用于涂阳肺结核患者的治疗和管理,而不开展预防性化学治疗。

(诸 慧)

\protect\hypertarget{text00020.html}{}{}

