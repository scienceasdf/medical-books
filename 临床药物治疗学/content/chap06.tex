\chapter{特殊人群用药}

\section{妊娠期和哺乳期妇女用药}

\subsection{妊娠期临床用药}

妊娠期由于母体变化、胎儿胎盘的存在及激素的影响,药物代谢和转运与非妊娠时期有很大差别。在全妊娠过程中,母体、胎盘、胎儿三者相互关联组成一个生物学、药物代谢动力学的组合单位。除极少数药物(例如胰岛素、肝素)不通过胎盘到胎儿,大多数药物均能通过胎盘进入胎儿体内。因此,孕妇用药必须了解药物的药代动力学,了解药物经胎盘到胎儿体内对胎儿及新生儿的药理作用,选择安全有效的药物,适时、适量地用药。

\subsubsection{药物在胎盘的转运与代谢}

胎盘由羊膜、属于子体部分的绒毛膜和属于母体部分的底蜕膜构成,将母血与胎儿血分开,称“胎盘屏障”。胎盘通透性与一般的血管生物膜相似,相当多的药物能够通过“胎盘屏障”进入胎儿体内。感染、缺氧常能破坏“胎盘屏障”,能使正常情况下不易通过“胎盘屏障”的抗生素容易通过。
\paragraph{影响药物通过胎盘的因素}

药物多以被动转运方式经胎盘转运,其速度受以下因素影响。①药物脂溶性高低。脂溶性药物,如安替比林及硫喷妥钠,能很快地以扩散方式通过胎盘。②药物分子的大小。较小分子量药物比大分子量药物扩散速度快。③药物离子化程度。④与蛋白结合能力。药物与蛋白质结合能力的高低与通过胎盘的药量成反比。⑤胎盘血流量。合并先兆子痫、糖尿病等全身性疾病的孕妇,麻醉或脐带受压迫时引起子宫胎盘血流量的改变,也可以使胎盘输送功能受到不同程度的影响,减缓药物转运。
\paragraph{药物在胎盘的代谢}

有些药物需要在胎盘经过代谢转化,才能成为容易输送的物质。胎盘有无数有活力的酶系统,具有生物合成及降解药物的功能。有些药物通过胎盘代谢降低活性,有些药物则增加活性。如天然或人工合成的肾上腺皮质激素,皮质醇及泼尼松通过胎盘转化为失活的11-酮衍化物;地塞米松通过胎盘则不需要经过代谢就能进入胎儿体内。因此,为了治疗孕妇疾病可用泼尼松,治疗胎儿疾病宜应用地塞米松。胎盘能代谢的仅限于几类酶所作用的物质,主要承担甾体类及多环碳氢化合物的代谢。

\subsubsection{母体药代动力学}
\paragraph{药物吸收}

妊娠期因孕激素影响胃肠系统的张力及活动力减弱,胃酸分泌减少,使口服药物的吸收延缓,达峰时间延长。但难溶性药物(如地高辛)因药物通过肠道的时间延长而生物利用度提高。

妊娠妇女由于肺潮气量和每分通气量明显增加,心排出量和肺血流量也增加,可使呼吸道吸入给药经肺泡摄取的药量增加。在妊娠妇女吸入麻醉时麻醉药的剂量通常应减少。
\paragraph{药物分布作用}

药物吸收后进入较非孕期增多的血浆、体液及脂肪组织中,使药物的分布容积增大,血药浓度低于非妊娠期。
\paragraph{药物与蛋白结合}

妊娠期虽然生成白蛋白的速度加快,但因血浆容积增加,形成生理性血浆蛋白低下。同时妊娠期很多蛋白结合部位被内泌素等物质所占据,所以使妊娠期药物蛋白结合能力下降,游离药物增多,药效和不良反应增强。
\paragraph{肝的代谢作用}

肝微粒体酶降解的药物可能减少,妊娠期高雌激素水平使胆汁在肝脏郁积,药物从胆汁排出减慢,从而使药物在肝脏清除减慢。
\paragraph{药物排出}

肾血流量及肾小球滤过率均增加,肾排泄药物或代谢产物加快,使主要以原形从尿中排出的药物消除加快,血药浓度不同程度降低(妊高征除外)。晚期妊娠期仰卧位时肾血流量减少而使由肾排出的药物作用延长,孕妇可采用侧卧位以促进药物的消除。

\subsubsection{药物在胎儿体内的转运与代谢}

胎儿体内的药物大部分经胎盘转运而来,也有少量药物经羊膜转运进入羊水中,而被胎儿吞饮经胃肠道吸收,或直接经皮肤吸收。
\paragraph{肝脏中的代谢}

药物通过胎盘经脐静脉进入胎儿血循环中。胎儿肝脏中酶的水平为成年人的30%~50%,胎儿对药物代谢能力较成年人低,所以胎儿体内药物浓度较母体高。因胎儿肝细胞缺乏催化葡萄糖醛酸苷类生成的酶,对药物解毒能力很差,如巴比妥、水杨酸类和激素等,易在胎儿体内达到毒性浓度。
\paragraph{肝外的代谢}

胎儿肝脏以外的代谢部位为肾上腺,胎儿肾上腺有很高活性的细胞色素P-450,在胎儿肾上腺内代谢的酶作用物质可能与肝脏是相同的。
\paragraph{排泄}

胎儿的肾小球滤过率甚低,肾排泄药物功能极差。许多药物在胎儿体内排泄缓慢,容易造成蓄积,如氯霉素、四环素等药物在胎儿体内排泄速度较母体明显减慢。胎儿进行药物消除的主要方式是将药物或其代谢物经胎盘返运回母体,由母体消除。

\subsubsection{胎儿治疗学}

胎儿治疗学指妊娠期孕妇用药,其目的不为治疗孕妇,而是为了给胎儿用药。胎儿治疗学所选用药物应注意其药代动力学,必须是经胎盘转运到胎儿,未经胎盘代谢,保持原有药效作用。已证实有效的治疗药物,如预计要早产的孕妇,妊娠期用肾上腺皮质类固醇促使胎儿肺提前成熟,选用肾上腺皮质类固醇时用地塞米松而不用泼尼松。

\subsubsection{妊娠期合理用药的条件}

鉴于许多药物可以通过胎盘,故在用药前应考虑以下几点。

(1)采用疗效肯定、不良反应小且对于药物代谢有清楚说明的药物,避免使用尚难确定有无不良影响的新药。

(2)已证明药物对灵长目动物胚胎是无害的。但没有任何一种药物对胎儿的发育是绝对安全的。

(3)用药时需清楚地了解妊娠周期。因为很难确定何时是胚胎器官形成的最终时刻,所以用药最好能在妊娠足4个月以后开始,在怀孕的前3个月内应避免应用任何药物。

(4)用药需有明确指征。用可能对胎儿有影响的药物时,要权衡利弊后给药,只有药物对母亲的益处多于对胎儿的危险时才考虑在孕期用药。

\subsubsection{FDA颁布的药物对妊娠的危害等级标准}

(1)A级:在有对照组的研究中,妊娠3个月的妇女未见到对胎儿有危害的迹象(并也没有在其后的6个月有危害性的证据),可能对胎儿的影响甚微。

(2)B级:在动物繁殖性研究中(并未进行孕妇的对照研究),未见到对胎儿的不良影响。在动物繁殖性研究中发现有不良反应,但这些不良反应并未在妊娠3个月的妇女得到证实(也没有对其后6个月的危害性证据)。

(3)C级:动物研究证明对胎儿有危害性(致畸或杀死胚胎),但并未对对照组妇女进行研究,或没有对妇女和动物平行地进行研究。本类药物只有在权衡了对孕妇的好处大于对胎儿的危害后方可应用。

(4)D级:对胎儿的危害性有明确的证据,尽管有危害性,但孕妇用药后有绝对好处。例如孕妇受到死亡的威胁或患有严重疾病,应用其他药物虽然安全但无效,因此需要用此类药物。

(5)X级:在动物或人的研究中均表明它可造成胎儿异常,或根据经验认为对人或动物是有危害性的,给孕妇应用这类药显然无益。本类药物禁用于妊娠或即将妊娠的患者。

\subsection{哺乳期临床用药}

大部分药物均能从乳汁排出并能测出药物浓度,一般药物由乳汁排出的浓度低,不超过母体1d内药量的1%。如果哺乳期需要用药,而且是一种比较安全的药,应在婴儿哺乳后(即下次哺乳前3~4h)用药。个别药物在乳汁中可达到较高浓度,如甲硝唑、异烟肼、红霉素及磺胺类等药物,它们在乳汁中的浓度可达到乳母血药浓度的50%。有时也可利用药物进入乳汁来治疗乳儿疾病。如用苯海拉明治疗婴儿皮肤过敏性疾患时,可让母亲服用常用量(25~50mg),通过哺乳,乳儿可获得治疗量的药物。

\section{老年人用药}

老年人由于年龄的增长,其生理功能处于逐渐衰退的状况,肌体对于药物的吸收、生物转化和排泄功能等各项指标都在下降,对药物处置能力及药物的反应性相应降低。在用药过程中由于多种疾病的存在使药物的体内过程复杂化,且多种疾病的并存往往需要同时使用多种药物治疗,由此产生的药物相互作用不仅影响老年人的药物治疗效果,同时药物不良反应所带来的用药风险性也随之增加。

\subsection{老年人药代动力学改变}

\subsubsection{吸收}

老年人胃肠吸收功能减退,药物吸收减少。但由于胃肠蠕动减慢,药物在胃肠中停留时间及与肠道吸收表面接触时间均延长,故对大多数药物(被动转运吸收的药物)总吸收的影响不明显,老年人和成年人相比无明显差异。但对靠主动转运来吸收的药物(如铁、木糖、钙以及维生素B{1}
、B{2} 、B{12}
、C等),由于老年人吸收这些药物所需的酶和糖蛋白等载体分泌减少,故吸收机能减弱。由于药物在胃肠内滞留时间延长,对胃肠道刺激增加,胃肠道反应增加。

\subsubsection{分布}

老年人机体组成成分发生改变,细胞内液减少,身体总水量减少,脂肪组织增加。故水溶性药物分布容积减少,血药浓度增加,如吗啡、乙醇、水杨酸盐、青霉素等;脂溶性药物分布容积增大,作用持续较久,半衰期延长,易在体内蓄积中毒。如老年人使用利多卡因时毒性反应明显增加,70岁以上者发病率为80%。

老年人血浆白蛋白含量减少,病情严重或极度虚弱的老年人下降尤为明显。应用血浆白蛋白结合率高的药物时,血中游离型药物浓度增大,易出现不良反应。如华法林,老年人用成人剂量时不良反应大,有引起出血的危险。此类药物还有普萘洛尔、苯妥英钠、安定、保泰松、地高辛和水杨酸盐,用时应注意减量。

\subsubsection{代谢}

老年人肝细胞减少,肝微粒酶的活性降低,肝血流量减少,使代谢能力下降,药物代谢减慢,造成药物蓄积,引起不良反应。对肝清除率高、首过效应明显的药物影响尤为显著。如老年人服用利多卡因、咖啡因、氨基比林、普萘洛尔等,要注意减少用量,或延长服药的间隔时间。

\subsubsection{排泄}

老龄所致的最大药代动力学改变在于药物的排泄,是老年人发生药物中毒反应的最重要因素。人的年龄达到40岁后,肾小球滤过率和肾小管排泄能力按每年1%的速度降低。因此,老年人药物清除率降低,即使无肾脏疾病,使用主要经肾排泄的药物时,易在体内蓄积而造成中毒。如地高辛、氨基糖苷类抗生素、青霉素G、苯巴比妥、西咪替丁及磺酰脲类降糖药等,都可因肾功能减退而排泄减少,半衰期显著延长,并有蓄积中毒的危险,均应相应减少用量或延长给药间隔时间。

\subsection{与增龄相关的系统改变}

\subsubsection{药物的相互作用}

老年人多病,常多药并用,药物的相互作用不仅影响老年人的药物治疗效果,同时使不良反应发生率上升,用药风险性随之增加。

\subsubsection{疾病因素}

老年人某些疾病的发病率亦迅速增加,使药物在体内过程复杂化。

(1)神经系统:衰老时中枢神经有某种病理变化的缓慢发展,对用药的影响表现在:①因记忆力差,引起服药的差错增多,对需要有稳态血药浓度的药物易因漏服而出现症状或因过量而出现不良反应;②应慎用对神经有毒性的药物,以防毒性叠加;③对药物的反应性有变化,如服用地西泮有引起脑功能失调的报道。

(2)心血管系统:老年人应激时调节最大心律的能力下降;平均收缩压较高,对血压调节功能降低,易出现体位性低血压。应慎用降压药和利尿药,避免引起体位性低血压;应注意控制甲状腺功能亢进、感染(特别是肺部)等疾病和输液用量,避免加重充血性心力衰竭。

(3)肾脏:主要由肾脏消除的药物应调整剂量,同时应关注体液和电解质平衡的紊乱。

(4)消化系统:肝脏清除药物减慢,主要经肝脏消除的药物必要时需调整剂量;慎用易引起便秘的药物。

(5)血液系统:造血组织的总量有所减少,但血液成分的变化不明显。对用药的影响主要为慎用有骨髓抑制不良反应的药物。

\subsection{老年人在临床治疗中需特别注意的常用药物}

\subsubsection{抗菌药物}

(1)青霉素类:主要经肾清除,老年人肾功能减退引起其消除半衰期延长,血药浓度增高,易出现神经精神症状,如幻觉、抽搐、昏睡、知觉障碍等。当控制感染需较大剂量青霉素类时,必须减少剂量或延长给药间隔时间。肌酐清除率可以作为其可靠的衡量指标。老年人处理电解质平衡的能力低,要注意避免处方含钠青霉素类而致钠过多,而处方羧苄西林或替卡西林时应注意有无血钾过低。

(2)头孢菌素类:所有头孢菌素都会抑制肠道菌群产生维生素K,因此具有潜在的致出血作用。服用阿司匹林、华法林等抗凝药物的老年人在给予头孢菌素类药物抗感染时,尤其需密切监测凝血酶原时间的变化,以免发生出血等严重不良反应。

(3)氨基糖苷类:均有不可逆的耳毒性和不同程度的肾毒性,肌酐清除率降低,使药物排泄受到一定限制;对耳毒药物更为敏感,更易发生上述毒性反应。65岁以上老年人应慎用此类药物,临床上对确需使用氨基糖苷类药物的老年患者应考虑采用每日1次的给药方案,以减小其耳肾毒性。同时注意避免与呋塞米、依他尼酸、顺铂等其他耳、肾毒性药物联合应用。

(4)喹诺酮类:该类药物具有脂溶性,脑脊髓中浓度高,并抑制脑内抑制性递质γ-氨基丁酸与其受体的结合,从而增加中枢神经系统的兴奋性。老年人存在不同程度的脑萎缩或脑动脉硬化,且肾清除药物的能力降低,因此老年人静脉滴注喹诺酮类药物,引起精神紊乱或中枢神经系统兴奋等不良反应的发生率较年轻人高。

\subsubsection{地高辛}

地高辛是临床上治疗充血性心力衰竭的常用药物,但治疗窗窄,中毒反应严重。地高辛中毒的发生率随年龄增加而增高,因此,老年人使用地高辛时,需监测地高辛血药浓度,且老年人的地高辛血药浓度的治疗范围可适当降低(<2.0ng/mL)。

\subsubsection{镇静催眠药}

老年人感觉较为迟钝,智力反应减低,应用镇静催眠药更易发生不良反应。老年人使用巴比妥类药物会发生兴奋激动,不宜常规应用。老年人对地西泮的中枢抑制作用比年轻人更敏感,应用时需谨慎,给药的时间间隔要加长。

\subsubsection{氨茶碱}

氨茶碱是慢性支气管炎和心源性哮喘患者的常用药,被肝脏的混合功能酶代谢。老年人肝功能都有不同程度的降低,半衰期因此延长。所以老年人服用氨茶碱后容易出现氨茶碱中毒,表现出烦躁、呕吐、忧郁、记忆力减退、定向力差、心律失常、血压急剧下降等现象乃至死亡。静脉注射速度过快或浓度太高可引起心悸、惊厥等严重反应。因此,对于急性心肌梗死、低血压、甲状腺功能亢进的患者禁用。老年人应用氨茶碱一定要慎重,开始用药要小剂量试用,询问氨茶碱的用药史。一旦发现有胃部不适或兴奋失眠,可用安定、复方氢氧化铝等药物来对抗或停药。氨茶碱主要通过肝药酶CYP1A2代谢,当与CYP1A2酶抑制剂(如环丙沙星等喹诺酮类抗菌药物)联合用药时,适当减少茶碱给药剂量或调整给药间隔,并密切监测茶碱血药浓度,以避免茶碱血药浓度过高而引起不良反应。

\subsubsection{HMG-CoA还原酶抑制剂(他汀类)}

他汀类药物是目前最强有力的调脂药物,肌病和横纹肌溶解是他汀类的最严重不良反应。他汀类通过CYP3A4药酶代谢(普伐他汀类除外),如果联用CYP3A4抑制剂,如大环内酯类的红霉素或克拉霉素、唑类抗真菌药(伊曲康唑、氟康唑、酮康唑等)、贝特类调脂药,可潜在地引起肌病和随后的横纹肌溶解。应控制剂量,对高龄老人慎用或减量,尽量避免与CYP3A4酶抑制剂或贝特类药物联用,如必须与CYP3A4酶抑制剂合用可选择普伐他汀。用药期间定期检测肝肾功能及血清肌酸激酶,他汀类药物也是可以安全使用的。

\subsection{老年人临床合理用药原则}

\subsubsection{不应当随意加服药物}

如因情绪不稳定、过度紧张、过度疲劳、睡前用脑过度而影响睡眠,出现失眠的情况,可以通过改变生活方式、心理慰藉来改善睡眠障碍,不应滥用安眠药。

\subsubsection{减少用药数量和剂量}

《中国药典》规定60岁以上老年人用药剂量为成年人的3/4,中枢神经系统抑制药应当以成年剂量的1/2或1/3作为起始剂量。

\subsubsection{加强宣传教育}

加强对老年人合理用药的宣传教育,应告知其按医师嘱咐合理用药。

\subsubsection{必要的血药浓度监测指导用药}

一些安全范围很窄的药物,应当做血药浓度监测,以调整用药剂量或更换药物治疗,并做到给药方案的个体化,如抗心律失常药普鲁卡因胺。

\subsubsection{合理使用抗生素}

老年人抗生素应用频率很高,但由于老年人肾功能呈生理性退行性改变,药物排出减少,血药浓度易在体内增高,易产生不良反应,一般用正常治疗剂量的2/3~1/2为宜,包括头孢菌素、青霉素等β-内酰胺类,应尽量减少用毒性大的抗菌药物如万古霉素及氨基糖苷类等品种。

\subsubsection{传统医药的应用}

老年人常服用补虚扶正中成药,但也不应太过或随意服用,一般提倡应用调补药品。但所用补药剂量不应过大。老年人用中药也宜从小剂量开始,因人、因时、因地不同而辨证论治。

\subsubsection{中西药的相互作用问题}

临床配合应用中西药物的现象很多,但对其相互作用研究不多。如在中药汤药中,习惯用甘草调和诸药,如果患者同时用呋塞米等利尿药,血钾浓度可能下降;并用降糖药者,其效用可能减低。

\section{儿科用药}

从胎儿到青春期(14岁)为儿科范围,药物治疗是儿科防病治病的主要手段。因小儿正处于生长发育的重要时期,所以用药时须特别注意。小儿时期具体包括新生儿期(出生至生后28d)、婴儿期、幼儿期、学龄前期、学龄期和少年期,应注意不同年龄分期,结合儿童的具体情况,如营养状况、体质等,根据药物的性质、用药方式作调整,才能取得满意效果。给药途径取决于病情的轻重缓急、用药目的及药物本身性质。一般情况下,有小儿剂型的药物不使用成人剂量分成几等份后服用。

\subsection{小儿药代动力学特点}

小儿时期,其器官和组织均处在不断发育和成熟的过程,新生儿期尤其是一个特殊阶段。为保证用药安全,应根据小儿生理的特征及药物在体内的药动学和药效学特点合理选择药物。

\subsubsection{吸收}

新生儿胃排空时间长,通过胃肠道吸收的药物比成人慢,肠壁相对长而薄,通透性高,可使一些药物的吸收增加。各种消化液分泌量少,胃液及酶的浓度小,消化能力弱。胃酸pH值较高,对遇酸不稳定的青霉素分解少,吸收好;对弱酸性药物,由于在胃液中解离增加,吸收少。

\subsubsection{分布}

小儿的体液总量和细胞外液量较成人比例高,可影响药物的分布。新生儿体脂含量低,脂溶性药物不能充分与之结合,因而分布容积小,游离药物浓度高,易出现中毒。同时,新生儿脑组织占身体比重较大,血脑屏障发育不完善,使脂溶性药物易进入大脑,出现神经系统反应。新生儿药物血浆蛋白的结合力低于成年人,营养不良和低蛋白血症的新生儿更低,对某些药物的敏感性增加。因此,在成人被认为是安全的、很低的血浆药物浓度,对新生儿可能引起不良反应。

\subsubsection{代谢}

新生儿肝容积与体重的比例较成人大,部分酶类较成人多,使一些完全在肝脏代谢的药物血浆半衰期缩短。同时,新生儿肝内混合功能氧化酶和化合酶代谢药物的活性比成人低得多,又使很多药物代谢缓慢,血浆半衰期延长。除了代谢程度,新生儿药物代谢与成人相比还有本质上的差别,如新生儿使用茶碱,将有一部分代谢为咖啡因,还需考虑咖啡因的药理作用。所以,对新生儿用药时,应考虑品种和剂量的选择,以防药物蓄积中毒。

\subsubsection{排泄}

新生儿的肾小球滤过功能和肾小管的分泌功能均不足,对主要在肾脏排泄的药物清除慢,引起中毒的危险,药物剂量需进行调整。

\subsection{小儿用药的特殊反应}

\subsubsection{药物敏感性和耐受性改变}

新生儿对酸、碱和水电解质平衡的调节能力差,过量水杨酸类药物可致酸中毒,利尿剂可致缺钠或缺钾,氯丙嗪易引起麻痹性肠梗死,氯霉素致灰婴综合征和再生障碍性贫血,长期使用皮质激素易引起胰腺炎等。

儿童对铁盐耐受性很差,成年人可耐受50g之多,而婴儿口服1g即可引起严重中毒反应,2g以上可致死,原因是可溶性铁盐引起婴幼儿肠道黏膜的损伤,甚至严重呕吐、腹泻、胃肠出血导致失水、休克。

小儿应用解热镇痛药后,可因体温骤降、出汗引起虚脱。应注意的是,阿司匹林、吲哚美辛可收缩血管,使新生儿动脉导管迅速关闭,致肺动脉高压,使新生儿病死率增加。解热镇痛药之间有交叉过敏反应,如对阿司匹林过敏,应用吲哚美辛、萘普生等也可能过敏。所以,在用药的过程中,要密切观察,防止少数患儿因过敏致死。

儿童因苯巴比妥过敏反应较多,故很少用。使用苯妥英钠常见的不良反应是癫痫
发作频率增加,如果此时未检测血药浓度,则往往认为是剂量不足,再增加剂量则症状更显著,所以使用也相对较少。通常用丙戊酸钠,其不良反应发生率较低,但有肝毒性,2岁以下儿童在合用其他抗癫痫
药时较易发生,用药期间应注意监测肝功能。

\subsubsection{溶血反应}

主要发生在红细胞G-6-PD缺乏的新生儿,使用维生素K、磺胺类、噻嗪类、利尿类、萘啶类、呋喃唑酮等有氧化性的药物时,可使红细胞膜发生破裂,引起溶血。

已知在妊娠后期,临床应用容易引起新生儿溶血或黄疸的药物包括较大剂量的解热镇痛药,如非那西丁、阿司匹林、氨基比林、安替比林、辛克芬;抗疟疾药,如奎宁、伯氨喹等;抗微生物药物,如头孢菌素类、青霉素、新生霉素、金霉素、氯霉素、四环素;中枢神经系统抑制剂,如吩噻嗪类、地西泮、苯巴比妥、苯妥英、乙醇、氯仿、水合氯醛;洋地黄毒苷类;性激素类等。其中,可引起免疫性溶血的药物包括青霉素、头孢菌素、磺胺药、异烟肼、奎宁、甲基多巴、非那西丁等。

\subsubsection{核黄疸}

新生儿本来就有黄疸的因素,在使用一些与胆红素竞争血浆蛋白的药物时,血中游离的胆红素升高,进入脑内与基底核结合导致胆红素脑病或核黄疸。维生素K{3}
、磺胺类、新生霉素都能影响胆红素代谢,加重新生儿黄疸,故慎用。

\subsubsection{神经系统反应}

新生儿血脑屏障发育不成熟,药物易透过血脑屏障直接作用于脆弱的中枢神经系统,引起神经系统反应。如阿片类药物易引起呼吸抑制;抗组胺药、苯丙胺、氨茶碱、阿托品可致昏迷及惊厥;皮质激素易引起手足抽搐;氨基糖苷类抗生素易引起听神经损伤等。

\subsubsection{牙色素沉着}

四环素、多西环素、米诺环素等可沉积于骨组织和牙齿,引起永久性色素沉着,如牙齿发黄,四环素还可抑制骨的生长发育。故妊娠4个月后,哺乳期妇女和8岁以下儿童除眼科局部用药外,不得应用四环素。