\chapter{心血管系统疾病的药物治疗}

\section{高血压}

高血压是最常见的心血管疾病之一,是一种以动脉压升高为主要特点的临床综合征。2005年《中国高血压指南》定义高血压的诊断标准为收缩压≥140mmHg和(或)舒张压≥90mmHg。患者既往有高血压病史,目前正在使用抗高血压药,血压虽低于140mmHg/90mmHg,亦可诊断为高血压。动脉压的持续升高可导致靶器官如心脏、肾脏、脑和血管的功能性或器质性病变,并伴有全身代谢性改变,不仅严重影响患者的生活质量,还直接威胁患者的生命。

临床上高血压可分为原发性高血压和继发性高血压两大类。原发性高血压是一种以血压升高为主要临床表现而病因尚未明确的独立疾病,占高血压的95%以上;继发性高血压又称为症状性高血压,在这类疾病中病因明确,而血压升高仅是该种疾病的临床表现之一,约占高血压的5%,如能及时治愈原发病,可能使血压恢复正常。

\subsection{流行病学}

高血压的患病率有地域、年龄和种族的差别,各国的情况也不尽相同,总体上发达国家高于发展中国家。近年来,我国的原发性高血压患病率与年俱增,2002年卫生部调查资料显示,我国成人高血压患病率达18.8%,较1991年全国普查成人患病率的11.26%有了明显增长。

\subsection{病因及发病机制}

\subsubsection{遗传}

高血压发病有明显的遗传性,并具有家族倾向。双亲均有高血压的子女,成年后发生高血压的比例亦增高。目前认为高血压是多基因的遗传病,具有遗传背景的患者占整个高血压人群的比例达30%~50%。

\subsubsection{精神、神经作用}

流行病学调查提示,从事经常处于应激状态、需高度集中注意力的工作,长期精神紧张、焦虑或长期处于受噪音、不良视觉刺激的环境下可引起高血压。这可能与大脑皮质兴奋、抑制平衡的机制失调,以致不能正常行使调节和控制皮质下中枢活动的功能,交感神经活动增强,从而使小动脉收缩,周围血管阻力上升,血压增高。

\subsubsection{肾素-血管紧张素-醛固酮系统(RAAS)激活}

血管紧张素Ⅱ是RAAS系统中最重要的活性成分,它可促使血管收缩,醛固酮分泌增加,水钠潴留,增加交感神经活力,最终导致血压上升。

\subsubsection{体内钠盐过多}

体内钠盐过多除了与钠摄取过多外,肾脏排钠障碍也是重要原因之一。钠潴留使细胞外液量增加,引起心排血量增高;血管平滑肌细胞内钠水平增高又可致细胞内钙离子浓度升高,使血管反射性收缩、外周血管阻力升高、血压升高。

\subsubsection{代谢综合征}

近年来研究发现,约50%的原发性高血压患者存在胰岛素抵抗。胰岛素抵抗导致血压升高的作用机制可能是促使细胞内钠、钙浓度升高;促进肾小管对水、钠的重吸收;交感神经活性增加;提高血压对钠盐的敏感性等。

\subsubsection{肥胖}

肥胖是代谢综合征的表现之一,肥胖患者常伴有高胰岛素血症,交感系统活性增高,且脂肪细胞可产生过多的血管紧张素原等,因此肥胖者易出现高血压。

\subsection{临床表现}

原发性高血压早期多无症状,一般在体检时偶然发现,少数患者在发生心、脑、肾等靶器官的并发症时才明确高血压的诊断。高血压患者可有头痛、眩晕、耳鸣、眼花、失眠、乏力、心悸等症状。体检时可听到主动脉瓣区第二心音亢进和主动脉瓣区杂音。持续高血压可有左心室肥厚并可闻及第四心音。高血压病初期只是在劳累、精神紧张、情绪波动后出现血压暂时升高,随后可恢复正常。随着病情的进展,血压持续升高,后期出现与心、脑、肾等器官相关的并发症。

\subsection{诊断}

成人高血压的诊断标准:在未用抗高血压药物的情况下,收缩压≥140mmHg和(或)舒张压≥90mmHg(见表\ref{tab11-1}\footnote{若患者的收缩压与舒张压分属不同级别时,则以较高的分级为准。单纯收缩期高血压也可按照收缩压水平分为1、2、3级。})。
由于血压的波动性,应至少2次在非同日静息状态下测得血压升高时方可诊断为高血压,而血压值应以连续测量3次的平均值,注意情绪激动、体力活动时会引起一过性的血压升高。高血压危险分层如表\ref{tab11-2}\footnote{SBP:收缩压;DBP:舒张压。1级高血压:SBP为140~159mmHg或DBP为90~99mmHg;2级高血压:SBP为160~179mmHg或DBP为100~109mmHg;3级高血压:SBP≥180mmHg或DBP≥110mmHg}所示。

\begin{longtable}{llll}
\caption{2005年中国高血压指南血压水平的定义及分类}
\label{tab11-1}\\
\toprule
类别&收缩压/mmHg&&舒张压/mmHg\\
\midrule
正常血压&<120&和&<80\\
正常高值&120~139&和(或)&80~89\\
高血压&≥140&和(或)&≥90\\
\quad 1级高血压(轻度)&140~159&和(或)&90~99\\
2级高血压(中度)&160~179&和(或)&100~109\\
3级高血压(重度)&≥180&和(或)&≥110\\
单纯收缩期高血压&≥140&和&<90\\
\bottomrule
\end{longtable}

\begin{longtable}[]{@{}llll@{}}
\caption{高血压危险分层}
\label{tab11-2}\\
\toprule
\endhead
其他危险因素及病史 & 1级高血压 & 2级高血压 & 3级高血压\tabularnewline
\midrule
Ⅰ:无其他危险因素 & 低危 & 中危 & 高危\tabularnewline
Ⅱ:1或2个危险因素 & 中危 & 中危 & 极高危\tabularnewline
Ⅲ:≥3个危险因素、靶器官损害或糖尿病 & 高危 & 高危 &
极高危\tabularnewline
Ⅳ:并存临床情况 & 极高危 & 极高危 & 极高危\tabularnewline
\bottomrule
\end{longtable}

\subsection{治疗}

\subsubsection{治疗目标}

治疗原发性高血压的最终目的是最大限度地降低心、脑、肾和周围血管等靶器官的损害及死亡。大量临床试验结果表明,降低高血压患者的血压水平并长期维持,可以减少靶器官损害和心脑血管事件及其相关死亡。2005年中国高血压防治指南根据现有的临床试验证据,推荐将普通高血压患者的目标血压定为<140mmHg/90mmHg,伴有糖尿病或肾病的高血压患者定为<130mmHg/80mmHg,老年人收缩压<150mmHg,如能耐受,还可以进一步降低。

\subsubsection{治疗原则}
\paragraph{非药物治疗原则}

对所有1~3级高血压患者,都应尽快给予非药物治疗。非药物治疗主要是改善生活方式,如限制钠盐摄入、减少总脂肪和饱和脂肪摄入、减轻体重、适当增加体育运动、减少精神紧张和心理压力、戒烟限酒等。
\paragraph{抗高血压药物治疗原则}

对3级高血压患者,或心血管危险为高危、极高危的1级和2级高血压患者,应当立即开始药物治疗。抗高血压的药物治疗原则如下。

(1)采用最小的有效剂量以获得最佳疗效,并使不良反应最小,如疗效不满意,可逐步增加剂量以获得最佳疗效。

(2)为了防止靶器官损害,要求24h内血压稳定于目标范围内。因此,宜选用一天1次给药,持续24h有降压作用的药物。

(3)联合应用降压药物治疗,有利于增加降压效果而不增加或减少不良反应的发生。

\subsubsection{常用降压药物}

近年来,抗高血压药物发展迅速。根据抗高血压的作用部位或机制,可将其分为以下几类:利尿剂、β受体阻滞剂、钙通道阻滞剂(CCB)、血管紧张素转换酶抑制剂(ACEI)和血管紧张素Ⅱ受体拮抗剂(ARB)。其他还有:α受体阻滞剂、交感神经抑制剂以及直接血管扩张剂等。
\paragraph{利尿剂}

(1)噻嗪类利尿剂:通过排钠利尿使细胞外液和血容量减少,从而使心排量减少和血压下降。噻嗪类利尿剂是治疗高血压的基础药物,临床首选用于单纯收缩期高血压(老年人)、充血性心力衰竭、黑人高血压患者;伴有代谢综合征、糖耐量受损、妊娠患者慎用;痛风患者禁用。常用制剂有氢氯噻嗪,12.5~25mg,每日1次。

(2)袢利尿剂:作用于肾小管髓袢升支,抑制其对\ce{Na^+} 、\ce{Cl^-}
的重吸收,利尿、降压作用强而迅速。临床首选用于终末期肾病、心力衰竭患者。常用制剂有呋塞米,每日20~100mg,每日2次;托拉塞米,2.5~10mg,每日1次。

(3)醛固酮拮抗剂:阻断醛固酮的作用,干扰钠的重吸收,促进\ce{Na^+} 、\ce{Cl^-}
的排出而产生利尿作用。临床首选用于心力衰竭、心肌梗死的患者。肾功能衰竭、高血钾患者禁用。常用制剂有螺内酯,每日20~50mg,每日1或2次。

(4)保钾利尿剂:选择性阻断肾小管上皮钠转运通道,减少远曲小管钠钾交换,使尿钠排泄增多而起到利尿作用,同时导致钾相对潴留。常用制剂有氨苯蝶啶,每日50~100mg,每日1或2次;阿米洛利,每日5~10mg,每日1或2次。

噻嗪类利尿药和袢利尿剂可引起血钾、血钠降低,血尿酸升高,长期应用者应适量补钾,建议多吃含钾的水果及其他食物。
\paragraph{β受体阻滞剂}

阻断交感神经β受体、减慢心率、降低心排血量、抑制肾素释放等。临床首选用于轻中度高血压患者,尤其是伴有劳力性心绞痛、心肌梗死或快速心律失常者。代谢综合征、糖耐量受损、运动员和从事重体力活动者慎用,哮喘、房室传导阻滞(Ⅱ或Ⅲ度)、慢性阻塞性肺疾病患者禁用。常用制剂有普萘洛尔每日10~30mg,每日1或2次;索他洛尔每日80~160mg,每日1或2次;美托洛尔每日25~50mg,每日1或2次;比索洛尔每日2.5~10mg,每日1次;阿罗洛尔每日10~30mg,每日1或2次。

β受体阻滞剂常见的不良反应有心动过缓、支气管痉挛、恶心、腹泻、抽搐、头晕、乏力等,还可能影响糖代谢、脂代谢以及诱发高尿酸血症。冠心病患者突然停药可诱发心绞痛,故应逐步减量。

本品不宜与地尔硫卓、维拉帕米合用。与利舍平合用可导致重度心动过缓甚至晕厥,与伪麻黄碱、麻黄碱或肾上腺素合用可至血压升高。
\paragraph{钙通道阻滞剂(CCB)}

抑制钙通过细胞膜的钙通道进入周围动脉平滑肌细胞,降低外周血管阻力,使血压下降。单用或与其他降压药合用可有效控制血压,减少心脑事件。孕妇忌用,有窦房结功能低下或心脏传导阻滞者以及充血性心力衰竭患者慎用地尔硫卓和维拉帕米。常用制剂有二氢吡啶类:硝苯地平每日30~60mg,每日3次或缓(控)释片每日30~60mg,每日1次;氨氯地平每日2.5~10mg,每日1次;非洛地平每日2.5~10mg,每日1次;拉西地平每日2~6mg,每日1次;非二氢吡啶类:地尔硫卓普通片每日90~180mg、每日3次或缓释片每日90~180mg、每日1次;维拉帕米普通片每日120~240mg、每日3次或缓释片每日90~180mg、每日1次等。

钙通道阻滞剂常见的不良反应有颜面潮红、头痛、眩晕、心悸、胃肠道不适、体位性低血压等。二氢吡啶类还可致踝部水肿、齿龈增生。维拉帕米的负性肌力和负性频率作用较明显,可抑制心脏传导系统和引起便秘。
\paragraph{血管紧张素转换酶抑制剂(ACEI)}

抑制血管紧张素Ⅰ转变为血管紧张素Ⅱ,减慢有扩血管作用的缓激肽的降解,促进有扩血管作用的前列腺素的释放。适用于轻中度高血压者,尤其是伴有心功能不全者;妊娠、高血钾及双侧肾动脉狭窄者禁用。常用制剂有卡托普利每日25~100mg,每日2次;贝那普利每日10~30mg,每日1次;福辛普利每日10~40mg,每日1次;赖诺普利每日5~40mg,每日1次;雷米普利每日2.5~10mg,每日1次;培哚普利每日2~4mg,每日1次等。

ACEI最常见的不良反应是持续性干咳,多见于用药早期,常导致患者不能耐受而换药。其他少见不良反应有低血压、高钾血症、血管神经性水肿、皮疹以及味觉异常。
\paragraph{血管紧张素Ⅱ受体拮抗剂(ARB)}

阻断血管紧张素Ⅱ的作用而达到降压目的。适用于轻中度高血压者,对伴有糖尿病、肾病和慢性心功能不全者以及不能耐受ACEI所致咳嗽患者有良好疗效。妊娠、高血钾及双侧肾动脉狭窄者禁用。常用制剂有氯沙坦每日25~100mg,每日1次;缬沙坦每日80~160mg,每日1次;厄贝沙坦每日150~300mg,每日1次;奥美沙坦每日20~40mg,每日1次;坎地沙坦每日8~32mg,每日1次;替米沙坦每日20~80mg,每日1次等。

ARB的不良反应少而轻微,主要为低血压、血管性水肿、皮疹、高钾血症、粒细胞减少等。由于没有咳嗽的副作用而优于ACEI。

\subsection{治疗策略}

\subsubsection{特殊情况用药}

抗高血压药的选用应考虑患者的个体情况、药物的作用、代谢、不良反应和药物的相互作用。

(1)合并心力衰竭者,宜选用ACEI和利尿剂。

(2)老年人收缩期高血压者,宜选用利尿剂、长效CCB。

(3)合并糖尿病、肾病者,宜选用ACEI、ARB。

(4)心肌梗死后患者,可选用β受体阻滞剂、ACEI。

(5)对合并支气管哮喘、糖尿病、抑郁症者不宜用β受体阻滞剂;痛风患者不宜用利尿剂。

\subsubsection{常用的联合用药}

当足量的单药治疗不能使血压达标时,需联合应用降压药。联合用药有利于血压在相对较短时期内达到目标值,也有利于减少不良反应。目前常用的5类一线降压药,有以下几种联合应用方案:ACEI或ARB+利尿剂;CCB+ACEI或ARB;CCB+噻嗪类利尿剂;β受体阻滞剂+CCB。部分患者需要联合3种降压药才能控制血压,而难治性高血压则常常需要联合3种以上的降压药,3种抗高血压药联合的治疗方案中必须包含利尿剂。

\section{冠状动脉粥样硬化性心脏病}

冠状动脉粥样硬化性心脏病简称冠心病,是指由于冠状动脉粥样硬化使管腔狭窄或阻塞导致心肌缺血、缺氧而引起的心脏病,亦称缺血性心脏病。根据冠状动脉病变的部位、范围和程度不同,有不同的临床特点。近年来,从提高诊治效果和降低病死率出发,将本病分为两类:慢性心肌缺血综合征和急性冠状动脉综合征。慢性心肌缺血综合征包括隐匿型冠心病、稳定性心绞痛和缺血性心肌病等;急性冠状动脉综合征包括不稳定型心绞痛、非ST段抬高型心肌梗死和ST段抬高型心肌梗死。

\subsection{流行病学}

本病的发病率存在着明显的地区和性别差异,多发于40岁以上,男性多于女性,工业发达国家发病率高于发展中国家,是西方发达国家死亡和致残的主要因素。目前我国冠心病的发病率和病死率仍低于发达国家,然而和一些发展中国家一样,近年来有升高趋势。

\subsection{病因及发病机制}

本病是冠状动脉粥样硬化所致,动脉粥样硬化始发于内皮损伤,损伤的原因不仅包括修饰的脂蛋白,还有病毒和其他可能的微生物,但目前与微生物存在的因果关系还未确定。粥样硬化可累及冠状动脉中的1、2或3支,亦可4支同时受累,其中以左前降支最为多见,病变也最严重。

\subsubsection{慢性稳定型心绞痛}

心绞痛是由于暂时性心肌缺血引起的以胸痛为主要特征的临床综合征,是冠心病最常见的症状。通常见于至少1支主要冠状动脉分支管腔狭窄的患者,当体力或精神应激时,冠状动脉血流不能满足心肌代谢的需要,导致心肌缺血,引起心绞痛发作,休息或含服硝酸甘油可缓解。慢性稳定性心绞痛是指心绞痛发作的程度、频度、性质及诱发因素在数周内无显著变化的患者。

\subsubsection{不稳定型心绞痛}

目前已趋向于将稳定性心绞痛以外的缺血性胸痛统称为不稳定型心绞痛(UA),这不仅是基于对不稳定的粥样斑块的深入认识,也表明了这类心绞痛患者临床上的不稳定性,进展至心肌梗死的危险性,必须予以足够的重视。

\subsection{临床表现}

稳定型心绞痛表现为发作性胸痛,有压迫性、发闷或紧缩感,一般在3~5min内缓解。疼痛部位在胸骨中段或上段,可波及心前区,常放射至左肩、左臂至无名指和小指,或至下颌部、咽部或颈部。不发作时无异常体征,发作时常见心率增快、血压升高、出汗及情绪变化,有些患者会出现心律失常及心脏杂音。

不稳定型心绞痛的临床表现一般具有以下三个特征之一:①静息时或夜间发生心绞痛常持续20min以上;②新近发生心绞痛(病程在2个月内)且程度严重;③近期心绞痛逐渐加重(包括发作的频度、持续时间、严重程度和疼痛放射到新的部位)。发作时可有出汗、皮肤苍白湿冷、恶心、呕吐、心动过速、呼吸困难、出现第三或第四心音等表现。

\subsection{治疗}

\subsubsection{慢性稳定型心绞痛}
\paragraph{治疗原则}

减少发作次数和减轻发作时心肌缺血的症状,提高患者生存治疗。
\paragraph{一般治疗}

发作时立即休息,稳定型心绞痛患者一般在停止活动后疼痛消除。
\paragraph{药物治疗目的}

预防心肌梗死和猝死,缓解症状和缺血发作,改善生活质量。在选择治疗药物时,应首先考虑预防心肌梗死和死亡。
\paragraph{常用治疗药物}

(1)硝酸酯类:为内皮依赖性血管扩张剂,能减少心肌需氧和改善心肌灌注,从而改善心绞痛症状。硝酸酯类药会反射性增加交感神经张力使心率加快,因此常联合负性心率药物如β受体阻断剂或非二氢吡啶类钙拮抗剂治疗慢性稳定性心绞痛。联合用药的抗心绞痛作用优于单独用药。

舌下含服或喷雾用硝酸甘油仅作为心绞痛发作时缓解症状用药,也可在运动前数分钟使用,以减少或避免心绞痛发作。长效硝酸酯制剂用于降低心绞痛发作的频率和程度,并可能增加运动耐量。长效硝酸酯类不适宜用于心绞痛急性发作的治疗,而适宜用于慢性长期治疗。每天用药时应注意给予足够的无药间期,以减少耐药性的发生。如劳力型心绞痛患者日间服药,夜间停药;皮肤敷贴片白天敷贴,晚上除去。常用制剂:硝酸甘油,舌下含服,每次0.5~0.6mg,一般连用不超过3次,每次相隔5min;硝酸异山梨酯,普通片,每次10~30mg,每日3或4次口服,缓释片剂,每次20~40mg,每日1或2次口服;单硝酸异山梨酯,普通片,20mg,每日2次口服,缓释片剂,40~60mg,每日1次口服。

硝酸酯类药物的不良反应包括头痛、面色潮红、心率反射性加快和低血压,以上不良反应以给予短效硝酸甘油更明显。第1次含用硝酸甘油时,应注意可能发生体位性低血压。使用治疗勃起功能障碍药物西地那非者24h内不能应用硝酸甘油等硝酸酯制剂,以避免引起低血压,甚至危及生命。对由严重主动脉瓣狭窄或肥厚型梗阻性心肌病引起的心绞痛,不宜用硝酸酯制剂,因为硝酸酯制剂降低心脏前负荷和减少左室容量能进一步增加左室流出道梗阻程度,而主动脉瓣狭窄严重的患者应用硝酸酯类制剂也因前负荷的降低进一步减少心搏出量,有造成晕厥的危险。

(2)β受体阻断剂:能抑制心脏β肾上腺素能受体,从而减慢心率、减弱心肌收缩力、降低血压,以减少心肌耗氧量,可以减少心绞痛发作和增加运动耐量。用药后要求静息心率降至55~60次/min,严重心绞痛患者如无心动过缓症状,可降至50次/min。只要无禁忌证,β受体阻断剂应作为稳定性心绞痛的初始治疗药物。

β受体阻断剂能降低心肌梗死后稳定型心绞痛患者死亡和再梗死的风险。目前可用于治疗心绞痛的β受体阻断剂有多种,当给予足够剂量时,均能有效预防心绞痛的发作。但具有内在拟交感活性的β受体阻断剂心脏保护作用较差,故推荐使用无内在拟交感活性的β受体阻断剂,如美托洛尔、阿替洛尔及比索洛尔。

在有严重心动过缓和高度房室传导阻滞、窦房结功能紊乱、有明显的支气管痉挛或支气管哮喘的患者,禁用β受体阻断剂。外周血管疾病及严重抑郁是应用β受体阻断剂的相对禁忌证。慢性肺心病的患者可小心使用高度选择性β受体阻断剂。没有固定狭窄的冠状动脉痉挛造成的缺血,如变异性心绞痛,不宜使用β受体阻断剂,而应将钙拮抗剂作为首选药物。β受体阻断剂的使用剂量应个体化,从较小剂量开始。常用药物见高血压病的药物治疗章。

(3)钙拮抗剂:通过改善冠状动脉血流和减少心肌耗氧起缓解心绞痛作用,对变异性心绞痛或以冠状动脉痉挛为主的心绞痛,钙拮抗剂是一线药物。地尔硫卓和维拉帕米能减慢房室传导,常用于伴有心房颤动或心房扑动的心绞痛患者,这两种药不应用于已有严重心动过缓、高度房室传导阻滞和病态窦房结综合征的患者。长效钙拮抗剂还可减少心绞痛的发作。常用药物见高血压病的药物治疗章。

外周水肿、便秘、心悸、面部潮红是所有钙拮抗剂常见的不良反应,低血压也时有发生,其他不良反应还包括头痛、头晕、虚弱无力等。当稳定性心绞痛合并心力衰竭必须应用长效钙拮抗剂时,可选择氨氯地平或非洛地平。β受体阻断剂和长效钙拮抗剂联合用药比单用一种药物更有效。此外,两药联用时,β受体阻断剂还可减轻二氢吡啶类钙拮抗剂引起的反射性心动过速的不良反应。非二氢吡啶类钙拮抗剂地尔硫卓或维拉帕米可作为对β受体阻断剂有禁忌的患者的替代治疗。但非二氢吡啶类钙拮抗剂和β受体阻断剂的联合用药能使传导阻滞和心肌收缩力的减弱更明显,要特别警惕。老年人、已有心动过缓或左室功能不良的患者应避免合用。

(4)血管紧张素转化酶抑制剂。在稳定性心绞痛患者中,合并糖尿病、心力衰竭或左心室收缩功能不全的高危患者应该使用血管紧张素转化酶抑制剂。常用药物见高血压病的药物治疗章。

(5)阿司匹林。随机对照研究证实了慢性稳定型心绞痛患者服用阿司匹林可以降低心肌梗死、脑卒中或心血管性死亡的风险。阿司匹林的最佳剂量范围为每日75~100mg。其主要不良反应为胃肠道出血或过敏,故在治疗期间应严密监测血常规;不能耐受阿司匹林的患者,可改用氯吡格雷作为替代治疗。

(6)氯吡格雷:主要用于支架植入以后及阿司匹林有禁忌证的患者。该药起效快,首剂顿服300mg后2h即能达到有效血药浓度。常用维持剂量为每日75mg,1次口服。

(7)调脂治疗。从总胆固醇浓度(TC)<4.68mmol/L开始,TC水平与发生冠心病事件呈连续的分级关系,最重要的危险因素是LDL-C。他汀类药物能有效降低TC和LDL-C,并因此降低心血管事件。为达到更好的降脂效果,在他汀类药物治疗基础上,可加用胆固醇吸收抑制剂依折麦布每日10mg。高三酰甘油血症或低高密度脂蛋白血症的高危患者可考虑联合服用降低LDL-C浓度的药物和一种贝特类药物(非诺贝特)或烟酸。高危或中度高危者接受降LDL-C药物治疗时,治疗强度应足以使LDL-C水平至少降低30%~40%。

他汀类药物可能引起的肝脏损害和横纹肌溶解,在应用该类药物时,应严密监测转氨酶及肌酸激酶等生化指标。采用强化降脂治疗时,更应注意监测药物的安全性。

\subsubsection{急性冠状动脉综合征(acute coronaly syndrome,ACS)}

ACS是一大类包含不同临床特征、危险性及预后的临床综合征,它们有共同的病理机制,即冠状动脉硬化斑块破裂、血栓形成,并导致病变血管不同程度的阻塞。根据心电图有无ST段持续性抬高,可将ACS区分为ST段抬高和非ST段抬高两大类,前者主要为ST段抬高心肌梗死(大多数为Q波心肌梗死,少数为非Q波心肌梗死),后者包括不稳定型心绞痛(UA)和非ST段抬高性心绞痛(NSTEMI)。NSTEMI大多数为非Q波心肌梗死,少数为Q波心肌梗死。这种划分临床上较为实用,这不仅反应两类疾病的病理机制有所差异,而且治疗对策也有明显不同。

UA/NSTEMI标准的强化治疗包括抗缺血治疗、抗血小板和抗凝治疗。有些患者经过强化的内科治疗,病情即趋于稳定。另一些患者经保守治疗无效,可能需要早期介入治疗。UA/NSTEMI治疗主要有两个目的:即刻缓解缺血和预防严重不良反应后果(即死亡、心肌梗死或再梗死)。其治疗包括抗缺血治疗、抗血小板与抗血栓治疗以及根据危险度分层进行有创治疗。
\paragraph{抗缺血治疗}

抗缺血治疗的主要原则:采用舌下含服或口喷硝酸甘油后静脉滴注以迅速缓解缺血及相关症状。在硝酸甘油不能即刻缓解症状或出现急性肺充血时,静脉注射硫酸吗啡。如果有进行性胸痛,并且没有禁忌证,口服β受体阻断剂,必要时静脉注射。对于频发性心肌缺血并且β受体阻断剂为禁忌时,在没有严重左心室功能受损或其他禁忌时,可以进行非二氢吡啶类钙拮抗剂,如维拉帕米或地尔硫卓治疗。ACEI用于左心室收缩功能障碍或心力衰竭、高血压患者,以及合并糖尿病的ACS患者。
\paragraph{抗血小板与抗凝治疗}

抗血小板治疗中,如果既往未用过阿司匹林,可以嚼服首剂阿司匹林0.3g,或口服水溶性制剂,以后每日75~150mg。除非有禁忌证,每位UA/NSTEMI患者均应使用阿司匹林。

氯吡格雷是二磷酸腺苷受体拮抗剂,对血小板的抑制是不可逆的。氯吡格雷的疗效等于或大于阿司匹林,因而对不能耐受阿司匹林者,氯吡格雷可作为替代治疗。在PCI患者中应常规使用氯吡格雷。阿司匹林+氯吡格雷可以增加择期冠脉旁路移植术患者术中、术后大出血危险,因而准备行此手术者,应停用氯吡格雷5~7d。

在UA/NSTEMI中早期使用肝素可以降低患者AMI和心肌缺血的发生率,联合使用阿司匹林获益更大。低分子肝素与普通肝素疗效相似,依诺肝素疗效还优于普通肝素。低分子肝素可以皮下注射,无需监测部分凝血活酶时间,较少发生肝素诱导的血小板计数减少,因此在某些情况下可以替代普通肝素。普通肝素和低分子肝素在UA/NSTEMI治疗中都被推荐使用。

此外,研究表明在ACS早期应用他汀类药物可以改善预后。
\paragraph{AMI的治疗}

AMI的治疗原则是保护和维持心脏功能,挽救濒死的心肌,防止梗死面积扩大,缩小心肌缺血范围,及时处理严重心律失常、泵衰竭和各种并发症,防止猝死,使患者不但能度过急性期,且康复后还能保持尽可能多的有功能的心肌。AMI的治疗原则如下:

(1)休息和护理:卧床休息1周,保持环境安静。

(2)吸氧:间断或持续吸氧。

(3)监测:进行心电图、血压、呼吸监测,随时调整治疗措施。

(4)饮食:进食不宜过饱,保持排便通畅。

(5)缓解疼痛:能降低心肌耗氧量。硝酸甘油是缓解心肌梗死疼痛最常用的药物。如硝酸酯类药物不能使疼痛迅速缓解,应立即用吗啡3~5mL静脉注射或5~10mL皮下注射,每4~6h可重复应用。
\paragraph{常用治疗药物}

(1)抗血小板治疗。如无禁忌证,起始3d均应服用阿司匹林,首次剂量至少300mg,使其通过口腔黏膜迅速吸收,随后每日1次,3d后改为50~100mg,每日1次。氯吡格雷首剂至少300mg,以后每日75mg。

(2)抗凝治疗。在梗死范围大、梗死范围再扩大、复发性梗死或有梗死先兆时可考虑抗凝治疗。患者有出血倾向、新近手术、活动性溃疡及严重肝肾功能不全者禁用。抗凝药物多选用肝素,由于普通肝素出血的发生率高,近年来逐渐被低分子肝素取代。一般使用方法是普通肝素静脉推注70IU/kg,然后静脉滴注15IU/kg维持,48~72h后改为皮下注射7500IU,每12h一次,注射2~3d。目前临床较多应用低分子肝素,每日皮下注射1次,每次0.3mL左右,5~7d为一疗程。使用肝素期间应监测血小板计数和部分凝血酶原时间,及时发现肝素诱导的血小板减少症和出血的发生。

(3)溶栓治疗。AMI溶栓治疗限于发病后6h内,发病距溶栓时间越短,梗死相关血管越容易再通,保护更多的心肌,更大程度地降低患者的病死率。常用药物为尿激酶,30min内静脉滴注150万IU;链激酶或重组链激酶,60min内静脉滴注150万IU;重组组织型纤维蛋白溶酶原激活剂,先静脉注射15mg,继而30min内静脉滴注50mg,其后60min内再静脉滴注35mg。

(4)β受体阻断剂。AMI最初几小时,使用β受体阻断剂可以限制梗死面积,并能缓解疼痛,减少镇痛剂的应用,对降低急性期病死率有肯定疗效。

(5)血管紧张素转换酶抑制剂(ACEI)。ACEI主要通过影响心肌重构、减轻心室过度扩张而减少充盈性心力衰竭的发生,降低病死率。AMI后长期服用ACEI,还可减少心肌再梗死和冠状动脉搭桥术的需要,提高远期存活率。ACEI的禁忌证:STEMI急性期动脉收缩压<90mmHg、临床表现严重的肾功能衰竭、双侧肾动脉狭窄、移植肾或孤立肾伴肾功能不全、对ACEI类制剂过敏或导致严重咳嗽者及妊娠期、哺乳期妇女等。

\section{高脂血症}

高脂血症是指由于脂肪代谢或运转异常导致血浆中一种或几种脂质或脂蛋白浓度高出正常范围,包括血脂的含量和(或)组分异常。血脂异常主要是指:血清TC水平过高;血清三酰甘油(TG)水平过高;血清高密度脂蛋白胆固醇(HDL-C)水平过低。

\subsection{病因及发病机制}

高脂血症通常分为原发性和继发性,由于遗传缺陷所致者称为原发性高脂血症,系统性疾病所致者称为继发性高脂血症。继发性高脂血症主要见于糖尿病、肾病综合征、甲状腺功能减低、肥胖症、慢性乙醇中毒及肝胆胰疾病等。

高胆固醇血症的原因包括基础值偏高、高胆固醇和高饱和脂肪酸摄入、体重增加、年龄及女性更年期影响、遗传基因异常、多基因缺陷与环境因素的相互作用等。高胆固醇血症的主要危害是易引起冠心病及其他动脉粥样硬化性疾病。

高脂血症根据血浆脂蛋白谱的变化可分5型:Ⅰ、Ⅱ、Ⅲ、Ⅳ、Ⅴ型。

(1)Ⅰ型高脂蛋白血症:又称高乳糜微粒血症,临床非常罕见,是一种染色体隐性遗传性疾病,其主要是外源性脂肪处理发生障碍,造成血中乳糜微粒增多。此类型常导致胰腺炎反复发作。

(2)Ⅱ型高脂蛋白血症:又称家族性高胆固醇血症、家族性高β脂蛋白血症等,是一种常染色体显性遗传疾病。患者血清胆固醇明显升高,可高达1000mg/mL,但血清TG水平基本正常范围。此型是导致冠心病的主要类型。

(3)Ⅲ型高脂蛋白血症:又称高胆固醇血症兼内源性高三酰甘油血症。此型的TG浓度变化很大,与饮食高脂、高糖有密切关系,故此类型患者平时应控制高脂和高热量的饮食。

(4)Ⅳ型高脂蛋白血症:又称为内源性高三酰甘油血症、糖致高三酰甘油血症。Ⅳ型高脂蛋白血症是一种典型的脂类代谢紊乱性疾病,它是高糖饮食引起的,常见于成年人,其心脑血管病发病率很高。此病当限制糖类饮食时,能使三酰甘油和前β脂蛋白的水平迅速下降。

(5)Ⅴ型高脂蛋白血症:又称混合型高脂肪血症。患者在青年或成年早期之前不出现症状,大多数在50岁以后开始发病。

\subsection{临床表现}

高脂血症的临床表现主要包括两大方面:脂质在真皮内沉积所引起的黄色瘤和脂质在血管内皮沉积所引起的动脉粥样硬化。由于高脂血症时黄色瘤的发生率并不高,动脉粥样硬化的发生和发展则需要相当长的时间,因此许多高脂血症患者并无任何症状和异常体征发现,往往是在血液生化检验时被发现。

\subsection{诊断}

关于高脂血症的诊断标准,目前国际和国内尚无统一的方法。国内多数学者认为血浆TC浓度大于5.2mmol/L可定为高胆固醇血症,血浆TG浓度大于2.3mmol/L可定为高三酰甘油血症。

\subsubsection{高脂血症的诊断}

明确血脂异常的诊断应对下列对象进行血脂检查:①已有冠心病、脑血管病或周围动脉粥样硬化的病者;②有高血压、糖尿病、肥胖、吸烟者;③有皮肤黄瘤者;④有家族性高脂血症者;⑤有冠心病或动脉粥样硬化病家族史者;⑥40岁以上的男性及绝经期后女性。首次检查发现血脂异常,应在2~3周内复查,若仍然属异常,则可确立诊断。

\subsubsection{高脂血症的分类}

临床上可简单地分为四类:高胆固醇血症、混合型高脂血症、高三酰甘油血症、低高密度脂蛋白血症。

\subsubsection{明确高胆固醇以外的冠心病危险因素及危险状态}

(1)性别:男性发病率比女性高,在中年时约高3~4倍,绝经期后妇女发病增高,但男女比例仍在1倍左右。

(2)年龄:随年龄增加,冠心病发病增加。

(3)高血压:无论收缩压或舒张压长期增高,均使冠心病危险性增加。

(4)吸烟:吸烟者的危险性比不吸烟者可高1倍。

(5)冠心病家族史:直系亲属中有冠心病史,冠心病危险性增加。

(6)糖尿病及糖耐量减低:无论胰岛素依赖与否,危险性都增加。

(7)肥胖:肥胖的分布与程度很重要,躯干及腹内脏器脂肪过度增加,是冠心病的独立危险因素。

\subsection{治疗}

\subsubsection{一般治疗}

饮食控制和生活方式改善是治疗任何情况下血脂异常的基础,无论治疗哪一型高脂血症,应首先调整生活方式,如控制饮食和适当锻炼,饮食治疗是首要的基本措施。心血管病高危人群必须进行调脂药物治疗。

\subsubsection{药物治疗}

无冠心病患者经过3~6个月的一般治疗后,或有冠心病的患者在进行了1~2个月的一般治疗后,其血脂水平仍未达到控制标准,应选用降血脂药物。

临床上常用的降血脂药物包括他汀类、胆汁酸螯合剂、烟酸类、贝特类及其他降血脂药。
\paragraph{他汀类}

他汀类是当前防治高胆固醇血症和动脉粥样硬化性疾病非常重要的药物。这类药物是细胞内胆固醇合成限速酶即三羟基三甲基戊二酰辅酶A(HMG-CoA)还原酶的抑制剂,是目前临床上应用最广泛的一类降脂药。目前国内临床上可供选用的他汀类药物有:洛伐他汀(Lovastatin)、辛伐他汀(Simcastatin)、普伐他汀(Pracastatin)、氟伐他汀(Flucastatin)、阿托伐他汀(Atorvastatin)和瑞舒伐他汀(Rosucastatin)。

各种他汀的具体使用剂量并无定论。在某些特殊情况下,各种他汀的最大允许使用剂量,如阿托伐他汀每日80mg是可以采用的,但需慎重监控药物的不良反应。有些他汀类可发生横纹肌溶解并发肾功能衰竭,故在应用他汀类治疗时,要定期监测谷丙转氨酶(ALT)、谷草转氨酶(AST)和肌酸肌酶(CK)。轻度的转氨酶升高(少于3倍正常上限值)并不看作是治疗的禁忌证。无症状的轻度CK升高常见。建议患者在服用他汀期间出现肌肉不适或无力症状以及排褐色尿时应及时报告,并进一步检测CK浓度。如果发生或高度怀疑心肌炎,应立即停止他汀治疗。
\paragraph{贝特类}

亦称苯氧芳酸类药物。此类药物通过去除血液循环中富含TG的脂蛋白,降低血浆TG水平和提高HDL-C水平,促进胆固醇的逆向转运,并使LDL亚型由小而密颗粒向大而疏松颗粒转变。临床上可供选择的贝特类药物有:非诺贝特(片剂0.1g,每日3次;微粒化胶囊0.2g,每日1次)、苯扎贝特(0.2g,每日3次)、吉非贝奇(0.6g,每日3次)。其适应证为高三酰甘油血症或以TG水平升高为主的混合型高脂血症和低HDL-C血症。临床试验证实,贝特类药物可能延缓冠心病的进展,减少主要冠状动脉事件。

此类药物的常见不良反应为消化不良、胆石症等,也可引起肝酶升高和肌病。绝对禁忌证为严重肾病和严重肝病。吉非贝齐虽有明显的调脂疗效,但安全性不如其他贝特类药物。由于贝特类单用或与他汀类合用时也可发生肌病,应用贝特类药物时也需监测肝酶与肌酶,以策安全。
\paragraph{烟酸类}

烟酸属B族维生素,当用量超过作为维生素作用的剂量时,可有明显的降脂作用。烟酸的降脂作用机制尚不十分明确,可能与抑制脂肪组织中的脂解和减少肝脏中极低密度脂蛋白合成和分泌有关。此外,烟酸还具有促进脂蛋白酯酶的活性,加速脂蛋白中TG的水解,因而其降低TG水平的作用明显。临床观察到,烟酸既降低TC水平又能降低TG水平,同时还具有升高HDL-C浓度的作用。

烟酸常用缓释片剂,常用量为1或2g,每日1次。一般临床上建议,开始用量为0.375~0.5g,睡前服用;4周后增量至每日1g,逐渐增加至最大剂量每日2g。适用于高三酰甘油血症,低高密度脂蛋白血症或以TG水平升高为主的混合型高脂血症。临床试验证实,烟酸能降低主要冠脉事件,并可能减少总病死率。

烟酸的常见不良反应有颜面潮红、高血糖、高尿酸(或痛风)、上消化道不适等。这类药物的绝对禁忌证为慢性肝病和严重痛风;相对禁忌证为溃疡、肝毒性和高尿酸血症。缓释型制剂的不良反应轻,易耐受。
\paragraph{胆酸螯合剂}

主要为碱性阴离子交换树脂,在肠道内能与胆酸呈不可逆结合,因而阻碍胆酸的肠肝循环,促进胆酸随大便排出体外,阻断胆汁酸中胆固醇的重吸收。通过反馈机制刺激肝细胞膜表面的LDL受体,加速LDL血液中LDL的清除,结果使血清LDL-C水平降低。

常用的胆酸螯合剂有考来烯胺(每日4~16g,分3次服用),考来替泊(每日5~20g,分3次服用)。胆酸螯合剂主要降低TC水平,对TG水平无降低作用甚至或稍有升高。临床试验证实这类药物能降低主要冠脉事件和冠心病死亡。

胆酸螯合剂常见不良反应有胃肠不适、便秘等消化道不良反应,还可干扰脂溶性维生素、叶酸、铁、镁、锌的吸收。
\paragraph{胆固醇吸收抑制剂}

胆固醇吸收抑制剂依折麦布(Ezetimibe)口服后被迅速吸收,且广泛地结合成依折麦布-葡萄糖苷酸,作用于小肠细胞的刷状缘,可有效抑制胆固醇和植物固醇的吸收。由于减少胆固醇向肝脏释放,促进肝脏LDL受体的合成,又加速了LDL的代谢。常用剂量为每日10mg,与他汀类合用对LDL-C、HDL-C和TG的作用进一步增强,安全性和耐受性良好。最常见的不良反应为头痛和恶心。
\paragraph{其他调脂药}

普罗布考:此药通过掺入到脂蛋白颗粒中,影响脂蛋白代谢,而产生调脂作用。常用剂量0.5g,每日2次。主要适用于高胆固醇血症尤其是纯合子型家族性高胆固醇血症。该药虽使HDL-C水平降低,但可使黄色瘤减轻或消退,动脉粥样硬化病变减轻,其确切作用机制未明。

降血脂药物的主要治疗对象是冠心病患者和心血管病高危人群,首选他汀类药物,所采用的药物剂量是以LDL-C达标为度,当单一降脂药物疗效欠佳时,可考虑联合用药。降脂药物治疗需要个体化,治疗期间必须监测安全性。依据患者的心血管病状况和血脂水平选择药物和起始剂量。在药物治疗时,必须监测不良反应,主要是定期检测肝功能和血CK。如肝酶(AST/ALT)超过3倍正常上限值,应暂停给药。停药后仍需每周复查肝功能,直至恢复正常。在用药过程中应询问患者有无肌痛、肌压痛、肌无力、乏力和发热等症状,血CK浓度升高超过5倍正常上限值应停药。用药期间如有其他可能引起肌溶的急性或严重情况,如败血症、创伤、大手术、低血压和抽搐等,应暂停给药。

\section{心力衰竭}

心力衰竭(heart failure)是各种心脏疾病导致心功能不全(cardiac
insufficiency)的一种综合征,是各种病因所致心脏疾病的终末阶段。在绝大多数情况下是指心肌收缩力下降、心排血量不能满足机体代谢的需要,造成器官、组织血液灌注不足,同时出现肺循环和(或)体循环瘀血的表现。由于心力衰竭时常伴有肺循环和(或)体循环的被动性充血,因此又常称为充血性心力衰竭。

\subsection{病因}

引起心力衰竭的原因众多,包括原发性心肌舒缩功能障碍及心脏负荷过度。原发性心肌舒缩功能障碍主要见于心肌病变,如心肌梗死、心肌缺血、心肌炎和原发性或继发性心肌病;心肌代谢障碍性疾病,如冠心病、肺心病、糖尿病心肌病、高原病、休克和严重贫血等;心脏负荷过度引起的疾病有高血压、心脏瓣膜病、先天性心脏病及甲亢等。

有基础心脏病的患者,其心力衰竭症状往往由一些增加心脏负荷的因素所诱发,常见的诱因包括感染、心律失常、肺栓塞、劳力过度、妊娠及分娩、贫血及出血、输液过多或过快、水电解质紊乱、不恰当地应用洋地黄类药物或抗高血压药等。

\subsection{发病机制}

心力衰竭的病理生理改变十分复杂,当心脏病开始损害心功能时,机体首先发生多种代偿机制,可使心功能在一定时间内维持在相对正常水平。当各种不同机制相互作用形成恶性循环时,造成失代偿,出现充血性心力衰竭,其病理生理改变更加复杂。

\subsubsection{心肌功能及结构变化}

心力衰竭时,心肌收缩力减弱,心率加快,心脏前、后负荷及心肌耗氧均增加,心脏收缩及舒张功能发生障碍。结构变化表现为心肌细胞发生凋亡和(或)坏死、心肌细胞肥大、心肌细胞外基质各种成分增多、堆积,心肌组织纤维化,进而发生心室重塑。

\subsubsection{神经内分泌变化}

心力衰竭时,机体全面启动神经内分泌调节:交感神经系统激活,长期的交感神经系统激活可使心肌后负荷及耗氧量增加,促进心肌肥厚,诱发心律失常甚至猝死。肾素-血管紧张素-醛固酮系统(RAAS)激活,长期的RAAS激活因增加心脏负荷而使心力衰竭恶化。其他神经内分泌变化还包括精氨酸加压素、内皮素、肿瘤坏死因子、利钠肽类、肾上腺髓质素等的增多和内皮细胞松弛因子、降钙素基因相关肽等的减少。

\subsection{临床表现}

\subsubsection{慢性心力衰竭}

(1)左心衰竭:主要表现为肺循环淤血引起的呼吸困难,最先发生在体力活动时(劳力性呼吸困难),休息时即可缓解。患者常在夜间突发胸闷、气急,为缓解呼吸困难常采取半坐位或坐位(端坐呼吸)。其他症状还包括咳嗽、咳痰和咯血,体力下降、乏力和虚弱,早期夜尿增多,严重时出现少尿并有肾功能不全的相应表现。

(2)右心衰竭:主要表现为长期胃肠道淤血所致的食欲减退、恶心及呕吐;肝淤血引起上腹饱胀感、黄疸、心源性肝硬化;肾淤血引起尿量减少、夜尿增多、蛋白尿和肾功能减退;四肢肌肉供血不足所致的乏力等。

(3)全心衰竭:多见于心脏病晚期,同时具有左、右心衰竭的临床表现。

\subsubsection{急性心力衰竭}

急性心力衰竭主要表现为急性肺水肿,患者常突发重度呼吸困难,采取坐位,极度烦躁不安、大汗淋漓、皮肤湿冷、面色灰白、发绀、咳嗽并咳粉红色泡沫痰。在肺水肿早期,血压可一度升高,随着病情持续,血压下降,如肺水肿不能及时被纠正,则可最终导致心源性休克。

\subsection{诊断}

典型的心力衰竭诊断并不困难。左侧心力衰竭的诊断依据为原有心脏病的证据和肺循环充血的表现。右侧心力衰竭的诊断依据为原有心脏病的证据和体循环瘀血的表现,且患者大多有左侧心力衰竭的病史。

要注意心力衰竭的早期诊断:劳力性气促和阵发性夜间呼吸困难是左心衰竭的早期症状;颈静脉充盈和肝区肿大是右心衰竭的早期症状。

\subsection{治疗}

目前慢性心力衰竭的治疗已从过去短期血流动力学、药理学措施转变为长期的、修复性的策略,治疗目标从改善症状、提高生活质量,到更重要的针对心肌重构的机制,防止和延缓心肌重构的发展,从而降低心力衰竭的病死率和住院率。

\subsubsection{心力衰竭的一般治疗}

(1)去除或缓解基本病因。应对所有患者都进行导致心力衰竭的基本病因评价。进行病因治疗,包括:原发性瓣膜病的手术修补或瓣膜置换,冠心病的冠状动脉血管重建术,有效控制血压,甲状腺功能亢进的治疗等。

(2)消除心力衰竭的诱因:如控制感染,治疗心律失常,纠正贫血、电解质紊乱等。

(3)改善生活方式:如戒烟、戒酒,肥胖患者减轻体重,低盐、低脂饮食等。

(4)密切观察病情演变及定期随访。

(5)避免应用某些药物。

如非甾体类抗炎药、Ⅰ类心律失常药及大多数钙拮抗剂等。

\subsubsection{心衰的药物治疗机制}
\paragraph{利尿剂}

利尿剂在心衰治疗中起关键作用。与任何其他治疗心衰药物相比,利尿剂能更快地缓解心衰症状,使肺水肿和外周水肿在数小时或数天内消退。其他治疗心衰的药物,包括洋地黄类、ACEI类或β受体阻断剂可能需要较长时间方能显效。利尿剂是唯一能充分控制心衰患者体液潴留的药物,也是标准治疗中必不可少的组成部分。合理使用利尿剂是其他治疗心衰药物取得成功的关键因素之一。

1)利尿剂治疗的适应证

所有心衰患者,有体液潴留的证据或原先有过液体潴留者,均应给予利尿剂。NYHA心功能Ⅰ级患者一般不需应用利尿剂。利尿剂一般应与ACEI类和β受体阻断剂联合应用。即使在应用利尿剂后心衰症状得到控制,也不能将利尿剂作为唯一治疗药物。

2)利尿剂的剂量

利尿剂应用的目的是控制心衰的体液潴留,一旦病情控制,即应以最小有效量长期维持,一般需无限期使用。利尿剂的给药剂量通常从小剂量开始,如呋塞米每日20mg,氢氯噻嗪每日25mg,并逐渐增加剂量直至尿量增加,体重每日减轻0.5~1.0kg。在长期维持治疗期间,仍应根据体液潴留情况随时调整剂量。每日体重的变化是最可靠的监测利尿剂效果和调整利尿剂剂量的指标。不恰当地大剂量使用利尿剂会导致血容量不足,增加ACEI类药物和血管扩张剂发生低血压的危险以及ACEI类和ARB类药物出现肾功能不全的危险。

3)利尿剂品种的选择

常用的利尿剂有袢利尿剂和噻嗪类利尿剂。袢利尿剂如呋塞米或托拉塞米是多数心衰患者的首选药物,特别适用于有明显体液潴留或伴有肾功能受损的患者。呋塞米的剂量与效应呈线性关系,故剂量不受限制。噻嗪类利尿剂仅适用于有轻度体液潴留、伴有高血压而肾功能正常的心衰患者。氢氯噻嗪每日100mg已达到最大效应,再增量亦无效。

4)对利尿剂的反应和利尿剂抵抗

对利尿剂的治疗反应取决于药物浓度和进入尿液的时间过程。轻度心衰患者即使小剂量利尿剂也反应良好,随着心衰的进展,因肠管水肿或小肠的低灌注,药物吸收延迟,且肾血流和肾功能减低,药物转运受到损害。因而当心衰进展和恶化时常需加大利尿剂剂量,最终则在大剂量时也无反应,即出现利尿剂抵抗。此时,可用以下方法克服:①静脉应用利尿剂,如呋塞米静脉注射40mg,继以持续静脉滴注(10~40mg/h);②2种或2种以上利尿剂联合使用;③应用增加肾血流的药物,如短期应用小剂量的多巴胺100~250μg/min。

非甾体类抗炎剂吲哚美辛能抑制多数利尿剂的利钠作用,应避免使用。

5)利尿剂的不良反应

长期应用利尿剂应严密观察不良反应,主要有电解质紊乱、症状性低血压以及肾功能不全,特别是在服用剂量大和联合用药时。
\paragraph{ACEI类}

ACEI对心衰、CHD、动脉粥样硬化、糖尿病等具有多种有益的机制。ACEI有益于CHF主要通过两个机制:

(1)抑制RAAS。ACEI能竞争性地阻断血管紧张素AngⅠ转换为AngⅡ,从而降低循环和组织的AngⅡ水平,还能阻断Ang1-7的降解,使其水平升高,进一步起到扩张血管及抗增生作用。组织RAAS在心肌重构中起关键作用,当心衰处于相对稳定状态时,心脏组织RAAS仍处于持续激活状态;心肌ACE活性增加,血管紧张素原mRNA水平上升,AngⅡ受体密度增加。

(2)作用于激肽酶Ⅱ,抑制缓激肽的降解,提高缓激肽水平,通过缓激肽-前列腺素-NO通路而发挥有益作用。ACEI促进缓激肽的作用与抑制AngⅡ产生的作用同样重要。

长期应用ACEI时,尽管循环中AngⅡ水平不能持续降低,但ACEI仍能发挥长期效益。这些资料清楚表明,ACEI的有益作用至少部分是由缓激肽通路所致。ACEI是已被证实能降低心衰患者病死率的第一类药物,公认为治疗心衰的基石。

1)ACEI类的适应证

(1)所有左室收缩功能不全的患者,均可应用此类药物,除非有禁忌证或不能耐受。NYHA心功能Ⅰ级患者也应使用,可预防和延缓发生心衰。

(2)不同程度慢性心力衰竭患者的长期治疗。

2)禁忌证或慎用ACEI的情况

应用此类药物曾出现过严重不良反应的患者,如血管神经性水肿、无尿肾功能衰竭,以及妊娠妇女,绝对禁用ACEI类。有以下情况的患者慎用:双侧肾动脉狭窄;血肌酐水平显著升高(>225.2μmol/L,尚有争论);高血钾症(>5.5mmol/L);低血压患者待血流动力学稳定后再决定是否应用ACEI类。

3)ACEI的剂量与品种选择

(1)起始剂量和递增方法:治疗前利尿剂应已维持在最合适剂量。此类药物应用的基本原则是从很小剂量开始,逐步增加,直到达到目标剂量。一般每隔3~7d剂量倍增1次,剂量调整快慢取决于每个患者的临床状况。

(2)临床上小剂量应用现象十分普遍,以为小剂量也同样有效而且更好是一种误解。一些研究表明,大剂量应用较小剂量应用对血流动力学、神经内分泌、症状和预后产生更大的作用。应该尽量将剂量增加到目标剂量或最大耐受剂量。

(3)维持应用:一旦剂量调整到目标剂量或最大耐受剂量,应终生使用。ACEI良好治疗反应通常要1~2个月,撤出ACEI有可能导致临床状况恶化,应避免。

4)ACEI的不良反应

患者体液潴留可减弱ACEI的疗效,而容量不足又可加剧ACEI的不良反应。

(1)低血压:在治疗开始几天或增加剂量时容易发生。RAAS激活明显的患者发生早期低血压反应的可能性最大,这些患者往往有显著的低钠血症或快速利尿。防治方法:坚持以极小剂量开始,先停用利尿剂1或2d,以减少对RAAS的依赖性。多数患者经适当处理后仍适合应用ACEI长期治疗。

(2)肾功能恶化:NYHA心功能Ⅳ级或低钠血症的患者易导致肾功能恶化。减少利尿剂剂量,肾功能通常会改善,不需停用ACEI。

(3)高血钾:ACEI阻滞醛固酮合成而减少钾的丢失,心衰患者可能发生高钾血症,严重者可引起心脏传导阻滞。如血钾≥5.5mmol/L,应停用ACEI。

(4)咳嗽:ACEI引起的咳嗽特点为干咳,见于治疗开始的几个月内。咳嗽不严重者可鼓励继续使用,影响正常生活者可考虑停用,或换用ARB类。

(5)血管性水肿:较为罕见(<1%),但可出现声带水肿,危险性较大,应予以注意,血管性水肿多见于首次用药或治疗最初24h内。由于可能是致命性的,因此临床上一旦怀疑为血管神经性水肿,该患者应终生避免应用所有的ACEI类药物。
\paragraph{β受体阻断剂}

β受体阻断剂是一种很强的负性肌力药,过去一直禁用于心衰的治疗。在应用ACEI和利尿剂的基础上加用β受体阻断剂,长期治疗慢性心衰,能改善临床情况、左室功能,降低病死率和住院率。目前有证据用于心衰的β受体阻断剂有选择性β{1}
受体阻断剂,如美托洛尔,比索洛尔,兼有β{1} 、β{2} 和α{1}
受体阻断作用的制剂,如卡维地洛、布新洛尔。

1)β受体阻断剂的适应证

所有NYHA心功能Ⅱ、Ⅲ级患者病情稳定,LVEF<40%者,均必须应用β受体阻断剂,除非有禁忌证或不能耐受。应在ACEI和利尿剂的基础上加用β受体阻断剂,洋地黄也可应用。病情不稳定的或NYHA心功能Ⅳ级患者,如病情已稳定,无体液潴留,体重恒定,且不需要静脉用药者,可考虑在严密监护下,由专科医师指导应用。β受体阻断剂是一作用强大的负性肌力药,治疗初期对心功能有抑制作用,但长期治疗(3个月以上)则可改善心功能。因此,此类药物只是用于慢性心衰的长期治疗,不能作为急性失代偿性心衰的急救用。虽然β受体阻断剂能掩盖低血糖症状,但有资料表明糖尿病患者获益更多,所以心衰伴糖尿病患者仍可使用。

2)β受体阻断剂的禁忌证

支气管痉挛性疾病、心动过缓(心率<60次/min)、Ⅱ度及以上房室传导阻滞(除非已安装起搏器)均不能使用。

3)β受体阻断剂的剂量选择

在此类药物起始治疗前和治疗期间患者体重必须恒定,已无明显体液潴留,利尿剂已维持在最合适剂量。此类药物应从极低剂量开始应用,如美托洛尔12.5mg,每日1次;比索洛尔1.25mg,每日1次;卡维地洛3.125mg,每日2次。如患者能耐受,可每隔2~4周将剂量加倍。如前一低剂量出现不良反应可延迟加量直至不良反应消失。确定β受体阻断剂治疗心衰的剂量,原则与ACEI类相同,并不按照患者治疗反应决定剂量,而应增加到事先设定的靶剂量。如患者不能耐受靶剂量,也可用较低剂量及最大耐受量。β受体阻断剂的个体差异很大,治疗宜个体化。应避免突然停药,以避免引起病情显著恶化。

4)β受体阻断剂制剂的选择

临床试验表明,选择性β受体阻断剂与非选择性β兼α{1}
受体阻断剂同样可降低病死率和罹患率。目前的观点是选择性β{1}
受体阻断剂美托洛尔、比索洛尔和非选择性的卡维地洛均可用于慢性心衰。

5)β受体阻断剂应用注意事项

此类药物在使用期间,必须监测血压,避免低血压的出现,特别是有α受体阻断作用的品种更容易发生。一般在首剂或加量的24~48h内发生,通常在重复用药后可自动消失。为减少低血压的危险,可将ACEI类药物或血管扩张剂的剂量减少或与β受体阻断剂在每日不同时间内应用。一般情况下,不主张减少利尿剂的用量,避免出现体液潴留。此外,在β受体阻断剂增加剂量的过程中,应注意避免出现心动过缓和房室传导阻滞。如果心率<55次/min或出现Ⅱ、Ⅲ度房室传导阻滞,应将β受体阻断剂减量或停用。同时避免此类药物与具有相同作用的其他药物间的相互作用。
\paragraph{正性肌力药}
正性肌力药(inotropic
agents)是指选择性增强心肌收缩力,主要用于治疗心力衰竭的药物。近年来,随着分子生物学的进展,对心力衰竭发生发展的研究和认识不断深化。目前研究表明,各种原因引起的心肌重塑是心力衰竭发生发展的基本机制。对心力衰竭发生机制的新认识使心力衰竭治疗的观念发生了根本性的转变,阻止神经激素的激活、逆转心肌重塑成为治疗的关键。目前,心力衰竭的治疗已从短期的血流动力学改善转变为长期的修复性策略,目的是改变心力衰竭心脏的生物学性质,降低病死率,改善预后。

1)洋地黄类

洋地黄类并非只是正性肌力药物,也是一种神经内分泌活动的调节剂。此类药物中地高辛是唯一经临床研究证实长期治疗不会增加病死率的药物。

(1)地高辛的适应证:一般而言,急性心衰并非地高辛的应用指征,除非伴有快速心室率的心房颤动。急性心衰应使用静脉给药,而地高辛仅可作为长期治疗措施的开始阶段而发挥部分作用。由于地高辛没有降低心衰病死率的作用,不存在推迟使用会影响存活率的可能性,因此地高辛的早期应用并非必要。一般推荐先使用那些能降低死亡率和住院危险的药物,如ACEI类和β受体阻断剂。

(2)地高辛的禁忌证:地高辛不能用于窦房阻滞、Ⅱ度或高度房室传导阻滞无永久起搏器保护的患者,与能抑制窦房结或房室结功能的药物(如胺碘酮、β受体阻断剂)合用时,尽管患者可耐受地高辛治疗,但仍需谨慎。

(3)地高辛的剂量调整:目前多采用自开始即用固定的维持量给药方法,即维持量疗法,每日0.125~0.25mg;对于70岁以上或肾功能受损者,地高辛宜采用小剂量(0.125mg)每日1次或隔日1次。必要时,可采用较大剂量(每日0.375~0.5mg),但不宜作为窦性心律心衰患者的治疗剂量。此外还可通过测定地高辛血清浓度,指导地高辛剂量的选择。但此法多用于帮助判断洋地黄中毒而非疗效的评估。

(4)地高辛的不良反应:主要包括①心律失常;②胃肠道症状(厌食、恶心和呕吐);③神经精神症状(视觉异常、定向力障碍、昏睡及精神错乱)。患者在低血钾、低血镁、甲状腺功能低下时更易出现不良反应。奎尼丁、维拉帕米、普鲁卡因胺、胺碘酮、丙吡胺、普罗帕酮等与地高辛合用时,可使血清地高辛浓度增加,从而增加洋地黄中毒的发生率,此时地高辛应减量。

2)磷酸二酯酶抑制剂

通常只能短期使用,不推荐常规间歇静脉滴注。常用药物有米力农。

(1)米力农的适应证:用于对洋地黄、利尿药、血管扩张剂治疗无效或欠佳的急、慢性顽固性充血性心力衰竭。

(2)注意事项:①下列情况慎用:肝肾功能损害、低血压、心动过速、急性心肌梗死、急性缺血性心脏病、孕妇及哺乳期妇女、儿童。不宜用于严重瓣膜狭窄病变、肥厚型梗阻性心肌病。②本品仅限于短期使用,长期使用增加病死率。③用药期间应监测心率、心律、血压,必要时调整剂量。④对房扑、房颤患者,因可增加房室传导作用导致心室率增快,宜先用洋地黄制剂控制心室率。⑤合用强利尿剂时,可使左室充盈压过度下降,且易引起水、电解质失衡。

(3)剂量调整:静脉注射负荷量25~75μg/kg,5~10min缓慢静脉注射,以后每分钟0.25~1.0μg/kg速度维持。每日最大剂量1.13mg/kg。

(4)不良反应:少见头痛、室性心律失常、无力、血小板计数减少;过量时可有低血压、心动过速。

\subsubsection{不同类型心衰的药物选择}
\paragraph{急性左心衰竭药物治疗}
急性心力衰竭(acute heart failure)是指由于急性心脏病变引起心
血量显著、急骤降低导致组织器官灌注不足和急性肺淤血综合征。常
病因为急性心肌梗死、高血压危象、心脏瓣膜病、心肌病等。临床上
急性左心衰竭常见,主要表现为急性肺水肿或心源性休克。 
急性左心衰处理必须刻不容缓,立即给氧、使用正
肌力药物利尿剂等药物。

1)镇静剂

主要应用吗啡,用法为2.5~5.0mg静脉缓慢注射,亦可皮下或肌内注射。应密切观察疗效和呼吸抑制的不良反应。

2)支气管解痉剂

一般应用氨茶碱0.125~0.5mg/(kg·h)静脉滴注。亦可应用二羟丙茶碱0.25~0.5g静脉滴注,速度为25~50mg/h。此类药物不宜用于冠心病,如AMI或不稳定型心绞痛所致的急性心衰患者;也不可用与伴心动过速或心律失常的患者。

3)利尿剂

(1)应用指征:适用于急性心衰伴肺循环和体循环明显瘀血以及容量负荷过重的患者。

(2)药物种类和用法:应采用静脉利尿制剂,首选呋塞米,先静脉注射20~40mg,继以静脉滴注5~40mg/h,其总剂量在起初6h不超过80mg,起初24h不超过200mg。临床研究表明,利尿剂小剂量联合应用,其疗效优于单一利尿剂的大剂量,且不良反应也更少。

4)血管扩张药物

(1)应用指征:此类药可应用于急性心衰早期阶段。收缩压水平能够使评估此类药是否安全的重要指标。收缩压>110mmHg的急性心衰患者通常可以安全使用;收缩压在90~110mmHg之间的患者应谨慎使用;而收缩压<90mmHg的患者则禁忌使用。

(2)药物种类和用法:主要有硝酸酯类、硝普钠、乌拉地尔、酚妥拉明,但钙拮抗剂不推荐用于急性心衰的治疗。

①
硝酸酯类药物:临床研究证实,硝酸酯类静脉制剂与呋塞米合用治疗急性心衰有效。静脉应用硝酸酯类药物应十分小心滴定剂量,经常测量血压,防止血压过度下降。硝酸甘油静脉滴注起始剂量5~10μg/min,每5~10min递增5~10μg/min,最大剂量100~200μg/min;亦可每10~15min喷雾1次(400μg),或舌下含服每次0.3~0.6mg。硝酸异山梨酯静脉滴注剂量5~10mg/h,亦可舌下含服每次2.5mg。

②
硝普钠:适用于严重心衰、原有后负荷增加以及伴心源性休克患者。临时应用宜从小剂量10μg/min开始,可酌情逐渐增加剂量至50~250μg/min,静脉滴注,不要超过72h。由于其强效降压作用,应用过程中要密切监测血压,根据血压调整合适的维持剂量。停药应逐渐减量,并加用口服血管扩张剂,以避免反跳现象。

③
乌拉地尔:该药具有外周和中枢双重扩血管作用,可有效降低血管阻力,降低后负荷,增加心输出量,但不影响心率,从而减少心肌耗氧量。适用于高血压性心脏病、缺血性心肌病(包括急性心肌梗死)和扩张性心肌病引起的急性左心衰。通常静脉滴注100~400μg/min,可逐渐增加剂量,并根据血压和临床状况予以调整。伴有严重高血压者可缓慢静脉注射12.5~25.0mg。

5)正性肌力药物

应用指征和作用机制,此类药物适用于低心排血量综合征,可缓解组织低灌注所致的症状,保证重要脏器的血流供应。血压较低和对血管扩张药物及利尿剂不耐受或反应不佳的患者尤其有效。

(1)洋地黄类:一般应用毛花苷C
0.2~0.4mg缓解静脉注射,2~4h后可以再用0.2mg,伴快速心室率的房颤患者可酌情适当增加剂量。

(2)多巴胺:250~500μg/min静脉滴注。此药应用个体差异较大,一般从小剂量开始,逐渐增加剂量,短期应用。

(3)多巴酚丁胺:该药短期应用可以缓解症状,但并无临床证据表明对降低病死率有益。用法:100~250μg/min静脉滴注。使用时注意监测血压,常见不良反应有心律失常,心动过速,偶尔可因加重心肌缺血而出现胸痛。正在应用受体阻断剂的患者不推荐应用多巴酚丁胺和多巴胺。

(4)磷酸二酯酶抑制剂:米力农,首剂25~50μg/kg静脉注射(>10min),继以0.25~0.5μg/(kg·min)静脉滴注。常见不良反应有低血压和心律失常。
\paragraph{急性右心衰的药物选择}
急性右心衰竭较少见,可发生于急性右心室梗
死,或由大面积肺梗死引起的急性肺源性心脏病。 

1)右心室梗死伴急性右心衰

扩容治疗:如存在心源性休克,在监控中心静脉压的基础上首要治疗是大量补液,可应用706羧钾淀粉、低分子右旋糖酐或生理盐水20mL/min静脉滴注,直至PCWP上升至15~18mmHg,血压回升和低灌注症状改善。24h的输液量大约在3500~5000mL。对于充分扩容而血压仍低者,可给予多巴酚丁胺或多巴胺。如在补液过程中出现左心衰,应立即停止补液。此时若动脉血压不低,可小心给予血管扩张药。

禁用利尿剂、吗啡、硝酸甘油等血管扩张剂,以避免进一步降低右心室充盈压。

如果右心室梗死同时合并广泛左心室梗死,则不宜盲目扩容,防止造成急性肺水肿。如果存在严重的左心室功能障碍和PCWP升高不宜使用硝普钠,应该考虑主动脉内球囊反搏治疗。

2)急性大块肺栓塞所致急性右心衰

止痛:吗啡或哌替啶。

溶栓治疗:常用尿激酶或人重组组织型纤溶酶原激活剂(rt-PA)。停药后应该继续肝素治疗。用药期间监测凝血酶原时间,使之延长至正常对照的1.5~2.0倍。持续滴注5~7d,停药后改用华法林口服数月。
\paragraph{慢性心衰的药物治疗}

慢性心衰的常规治疗包括联合使用3大类药物,即利尿剂、ACEI(或ARB)和β受体阻断剂。为进一步改善症状和控制心率,地高辛应该是第4个联用的药物。醛固酮受体拮抗剂则可应用于重度心衰患者。

1)利尿剂

临床应用:利尿剂缓解症状最为迅速,数小时或数日内即见效,而ACEI、β受体阻断剂则需数周或数月,故利尿剂必需最早应用。在利尿剂治疗的同时,应适当限制钠盐的摄入量。

2)血管紧张素转换酶抑制剂

所有慢性收缩性心衰患者,包括B、C、D各个阶段人群和NYHAⅠ、Ⅱ、Ⅲ、Ⅳ各级心功能患者(LVEF<40%),都必须使用ACEI,而且需要终身使用,除非有禁忌证或不能耐受。ACEI治疗早期可能出现一些不良反应,但一般不会影响长期应用。

3)β受体阻断剂

所有慢性收缩性心衰、NYHA
Ⅱ、Ⅲ级病情稳定患者,以及阶段B、无症状性心力衰竭或NYHA
Ⅰ级的患者(LVEF<40%),均必须应用β受体阻断剂,而且需终身使用,除非有禁忌证或不能耐受。NYHA
Ⅳ级心衰患者需待病情稳定(4d内未静脉用药,已无液体潴留并体重恒定)后,在严密监护下由专科医师指导应用。

在应用ACEI和利尿剂的基础上加用β受体阻断剂可使死亡危险性下降更明显,提示同时抑制两种神经内分泌系统可产生相加的有益效应。应尽早开始应用β受体阻断剂,不要等到其他疗法无效时才用。

4)地高辛

适用于已在应用ACEI(或ARB)、β受体阻断剂和利尿剂治疗,而仍持续有症状的慢性收缩性心衰患者。重症患者可将地高辛与ACEI(或ARB)、β受体阻断剂和利尿剂同时应用。另一种方案是先将醛固酮受体拮抗剂加用于ACEI、β受体阻断剂和利尿剂的治疗上,仍不能改善症状时,再应用地高辛。如患者已在应用地高辛,则不必停用,但必须立即加用神经内分泌抑制剂ACEI和β受体阻断剂治疗。

5)血管紧张素Ⅱ受体拮抗剂(ARB)

ACEI一直是治疗心衰的首选药物,而今年来随着ARB临床观察资料的积累,提高了ARB类药物在心衰治疗中的地位。

6)神经内分泌抑制剂的联合应用

(1)ACEI和β受体阻断剂的联合应用:临床试验已证实二者有协同作用,可进一步降低CHF患者的病死率,已是心衰治疗的经典常规,应尽早合用。

(2)ACEI与醛固酮受体拮抗剂合用:醛固酮受体拮抗剂的临床试验均是与以ACEI为基础的标准治疗作对照,证实ACEI加醛固酮受体拮抗剂可进一步降低CHF患者的病死率。

(3)ACEI加用ARB:ARB是否能与ACEI合用以治疗心衰,目前仍有争论。根据临床试验,AMI后并发心衰的患者不宜联合使用这两类药物。

(4)ACEI、ARB与醛固酮受体拮抗剂三药合用:三者合用的安全性证据尚不足,且肯定会进一步增加肾功能异常和高钾血症的危险。

(5)AECI、ARB与β受体阻断剂三药合用:不论是ARB与β受体阻断剂合用,或ARB+ACEI与β受体阻断剂合用,目前并无证据表明对心衰或MI后患者不利。