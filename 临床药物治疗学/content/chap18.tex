\chapter{内分泌系统及代谢性疾病的药物治疗}

\section{甲状腺功能亢进症}

甲状腺功能亢进症(hyperthyroidism),简称甲亢,指甲状腺呈现高功能状态,产生和释放过多的甲状腺激素所致的一组疾病,其共同特征为甲状腺激素分泌增加而导致的高代谢和交感神经系统的兴奋性增加。本病多见于女性,男女得病之比为1∶4,各种年龄均可发病,但以中青年发病者最多。

\subsection{甲状腺功能亢进的病因及临床表现}

引起甲状腺功能亢进的病因很多,最常见的是Graves病(弥漫性毒性甲状腺肿),占所有甲状腺功能亢进的85%。Graves病与自身免疫反应有关,其免疫失调的突出特征是患者的B淋巴细胞产生抗体,其中一些可以与甲状腺滤泡细胞上的促甲状腺激素(TSH)受体结合并使受体活化,刺激甲状腺的增长并产生过多的甲状腺激素。

甲状腺功能亢进的主要临床症状有甲状腺肿大、食欲亢进、体重减轻、心动过速、情绪激动、怕热、出汗、手抖。甲状腺危象(thyroid
storm)又称甲状腺功能亢进危象,为甲状腺功能亢进患者可危及生命的严重表现,通常见于严重的甲状腺功能亢进者在合并其他疾病时,如感染、败血症、精神应激和重大手术时;严重的甲状腺功能亢进同时合并其他疾病与甲状腺危象之间很难截然区分,因此严重甲状腺功能亢进同时合并感染、败血症等其他疾病的患者如不能区分是否是甲状腺危象,应按甲状腺危象处理。

\subsection{甲状腺功能亢进的诊断和治疗原则}

甲状腺功能亢进的诊断主要包括临床高代谢的症状和体征、甲状腺肿和(或)甲状腺结节以及TT$_{4}$、FT$_{4}$ 、TT$_{3}$ 、FT$_{3}$
升高,TSH降低。目前尚无有效地针对病因和发病机制的根治方案,对症治疗主要是控制高代谢症状,促进器官特异性自身免疫的消退。适当休息,避免过度紧张和精神刺激,补充足够的热量和营养,限制碘摄入是甲状腺功能亢进的一般治疗原则。此外,常用的治疗方法有三种:抗甲状腺药物治疗、放射性同位素碘$^{131}$
治疗和甲状腺次全切除术。药物治疗是甲状腺功能亢进治疗最常用的方法。放射性同位素碘$^{131}$
治疗和手术治疗前必须用药物控制甲状腺功能亢进,以免发生甲状腺危象。

\subsection{治疗药物的选用}

\subsubsection{硫脲类}

硫脲类药物优点是口服用药易被接受,在使病情缓解的同时不会引起腺体损伤。但此类药物疗程长、依从性差、儿童用药需家长和医师的严密监护、复发率较高、不良反应危险性大。此类药物目前常用的包括丙硫氧嘧啶(PTU)、甲巯咪唑(MMI)和卡比马唑等。

硫脲类药物可用于甲状腺功能亢进内科治疗和术前准备用药。初始剂量:PTU每日300~400mg,分3或4次服用;MMI每日30~40mg,每日1或3次服用。儿童与成人服法相同,单剂量需调整。甲状腺危象或严重甲状腺功能亢进宜选用PTU,因其可阻断外周组织T$_{4}$
向T$_{3}$
转化,较MMI起效为迅速。虽然PTU的使用非常广泛,但MMI也有其特别优势。该药可以每日给药1次,可提高患者依从性。与PTU相比,等剂量条件下MMI的药效当10倍的PTU,所以MMI药效更强,所需剂量更小,但PTU不良反应低,且与剂量相关。因此,在PTU不耐受或其他临床不适用PTU的情况下可选用MMI。妊娠期间进行甲状腺功能亢进治疗时,多选用PTU,而非MMI。并且由于PTU几乎不透过胎盘屏障,也不经乳汁分泌,PTU也适用于哺乳期妇女。儿童甲状腺功能亢进的治疗应选用MMI。

\textbf{丙硫氧嘧啶(Propylthiouracil,PTU):}

【适应证】

(1)甲状腺功能亢进的内科治疗:适用于病情轻,甲状腺轻、中度肿大的甲状腺功能亢进患者;年龄<20岁、妊娠甲状腺功能亢进、年老体弱或合并严重心、肝、肾疾病不能耐受手术者。不适宜手术或放疗者、手术后复发而不适于放射性碘治疗者均宜采用药物治疗,也可作为放射性碘治疗时的辅助治疗。

(2)甲状腺危象的治疗:作为辅助治疗以阻断甲状腺素的合成。

(3)术前准备:为了减少麻醉和术后并发症,防止术后发生甲状腺危象。

【用法和用量】

口服:用药剂量应个体化。根据病情、治疗反应及甲状腺功能检查结果随时调整。每日剂量分次口服,间隔时间尽可能平均。

(1)用于甲状腺功能亢进。成人开始剂量为每次100mg,每日3次,每日最大剂量为600mg,通常在4周以后发挥作用。当症状消失,血中甲状腺激素水平接近正常后逐渐减量。大约每2~4周减药1次,减量至每日50~100mg,减至最低有效剂量每日50~100mg时维持治疗,总疗程一般为1.5~2年。治疗过程中出现甲状腺功能减退或甲状腺明显增大时可酌情加用左甲状腺素或甲状腺片。儿童开始剂量为每日按体重4mg/kg,分次口服,维持量酌减。

(2)用于甲状腺危象。每日400~800mg,分3次服用,疗程不超过1周,作为综合性治疗措施之一。

(3)甲状腺功能亢进术前准备。每次100mg,每日3次,使甲状腺功能恢复到正常或接近正常,然后加服2周碘剂再进行手术。

【不良反应】

多发生在用药初始的2个月。一般不良反应为胃肠道反应、关节痛、头痛、皮疹、药物热等;血液不良反应为轻度白细胞计数减少,严重粒细胞缺乏、血小板计数减少、脉管炎和红斑狼疮样综合征;罕见间质性肺炎、肾炎、黄疸、肝功能损害、免疫功能紊乱等。

【注意事项】

(1)本品可透过胎盘屏障,并引起胎儿甲状腺功能减退及甲状腺肿大,甚至在分娩时造成难产、窒息。因此,对患甲状腺功能亢进的妊娠妇女宜采应用最小有效剂量的抗甲状腺药。本品可由乳汁分泌,可引起婴儿甲状腺功能减退,在哺乳期间应停止哺乳。

(2)小儿用药应根据病情调节用量,老年人尤其肾功能减退者,用药量应减少。甲状腺功能亢进控制后及时减量,用药过程中应加用甲状腺素,避免出现甲状腺功能减退。

(3)外周血白细胞计数偏低、对硫脲类药过敏者慎用。如出现粒细胞缺乏或肝炎的症状和体征,应停止用药。

(4)老年患者发生血液不良反应的危险性增加,若中性粒细胞少于1.5×10$_{9}$
/L应即停药。

【禁忌证】

(1)对本品及其他硫脲类药过敏者。

(2)严重肝肾功能损害、严重粒细胞缺乏、结节性甲状腺肿伴甲状腺功能亢进者、甲状腺瘤者。

\textbf{甲巯咪唑(Thiamazole):}

【适应证】 【注意事项】

同丙硫氧嘧啶。

【用法和用量】

口服。

(1)用于甲状腺功能亢进。成人开始每日30mg,可按病情轻重调节为每日30~45mg,每日最大量60mg,一般均分3次口服,但也可每日单次顿服。病情控制后,逐渐减量,1次减量每日5~10mg。维持量为每日5~15mg,疗程一般1~1.5年。

(2)用于儿童甲状腺功能亢进。开始时剂量为每日按体重0.4mg/kg,最大剂量为30mg,分次口服。维持量约减半或按病情轻重调节。

【不良反应】

常见皮疹、瘙痒、白细胞计数减少;少见严重粒细胞缺乏、血小板计数减少、凝血因子Ⅱ和Ⅶ降低;可见味觉减退、恶心、呕吐、上腹不适、关节痛、脉管炎等。

【禁忌证】

对本品过敏者、哺乳期妇女。

\subsubsection{碘剂}

碘剂起效迅速,对于甲状腺危象患者是较好的选择。碘剂按常规应用于甲状腺手术前10~14d,可减少腺体血供并增加腺体硬度使其易于切除。手术前碘剂可与β受体阻断剂或硫脲类药物先后或联合使用。此外,碘剂在硫脲类药物起效之前,可用于缓解甲状腺功能亢进的症状,用于甲状腺危象的抢救。

\section{糖尿病}

糖尿病是以慢性高血糖为特征的一组异质性代谢性疾病,由胰岛素分泌缺陷和(或)胰岛素作用缺陷所引起,以慢性高血糖伴碳水化合物、脂肪和蛋白质的代谢障碍为特征。在过去20多年,世界糖尿病患者数量持续增长,并且预计将来还会继续增加。2007---2008年应用糖耐量筛查全国部分城市20岁以上人群结果显示2型糖尿病患病率高达11%以上。

\subsection{糖尿病的病因及临床表现}

大部分糖尿病患者可按照病因、发病机制分为1型和2型糖尿病。1型糖尿病的主要病因是由于自身免疫对胰岛β细胞破坏后造成胰岛素分泌的绝对缺乏,故1型糖尿病患者需要以胰岛素治疗来维持生命。2型糖尿病的发生是由于胰岛素分泌减少或是外周胰岛素抵抗,可表现为以胰岛素抵抗为主伴胰岛素相对缺乏,或胰岛素分泌缺陷为主伴或不伴胰岛素抵抗。

引起糖尿病的主要病因有遗传因素、环境因素和免疫因素等。1型糖尿病绝大多数为自身免疫性糖尿病,遗传缺陷是1型糖尿病的发病基础;而少部分1型糖尿病患者为特发性糖尿病,以病毒感染最为重要,已经发现的有腮腺炎病毒、柯萨奇病毒、风疹病毒、巨细胞病毒等。2型糖尿病有明显的遗传异质性,并受到多种环境因素的影响,如肥胖、年龄增加、生活方式等。妊娠也可能导致糖尿病,但多数患者于分娩后可恢复正常,近30%的患者于5~10年随访中转变为糖尿病。

临床上早期无症状,至症状期才有多食、多饮、多尿、烦渴、善饥、消瘦或肥胖、疲乏无力等症群,即“三多一少”症状。严重病例或应激时可发生酮症酸中毒(diabetic
ketoacidosis,DKA)和高渗性高血糖状态;慢性并发症包括大血管和微血管病变,大血管病变如动脉粥样硬化、冠心病、高血压、脑血管疾病、周围血管疾病、糖尿病足等;微血管病变如糖尿病肾病、糖尿病视网膜病变、糖尿病神经病变等。

\subsection{糖尿病的诊断和治疗原则}

糖尿病的诊断依据血糖水平确定,常用标准为1999年WHO糖尿病诊断标准,如表\ref{tab18-1}所示。 

\begin{longtable}[]{p{6cm}p{6cm}}
    \caption{糖尿病的诊断标准}
    \label{tab18-1}\\
\toprule
诊断标准 & 静脉血浆葡萄糖水平/(mmol/L)\tabularnewline
\midrule
\endhead
典型糖尿病症状(多饮、多尿、多食、体重下降)+随机血糖检测 &
≥11\tabularnewline
空腹血糖(FPG) & ≥7.0\tabularnewline
葡萄糖负荷后2h血糖 & ≥11.1\tabularnewline
\bottomrule
\end{longtable}

注:空腹状态指至少8h没有进食热量;随机血糖指不考虑上次用餐时间,1d中任意时间的血糖浓度,不能用来诊断空腹血糖受损(IFG)或糖耐量异常(IGT)

限于目前的医学水平,糖尿病还是一种不可根治的慢性疾病,因此糖尿病的治疗应是综合性治疗。“综合性”的第一层含义是:糖尿病的治疗是包括饮食控制、运动、血糖监测、糖尿病自我管理教育和药物治疗;第二层含义是:虽然糖尿病主要是根据高血糖确诊因而需要医疗照顾,但对大多数的2型糖尿病患者而言,往往同时伴有“代谢综合征”的其他表现,如高血压、血脂异常等,所以糖尿病的治疗应是包括降糖、降压、调脂和改变不良生活习惯如戒烟等措施的综合治疗。

\subsection{常用降糖药物的选择及其作用机制}

根据患者的糖尿病类型选择降糖药物。1型糖尿病确诊后,在生活方式干预的基础上应立即给予胰岛素治疗。2型糖尿病确诊后,在生活方式干预基础上采用口服药物治疗。使用口服降糖药物治疗的前提是:患者必须有较好的分泌胰岛素功能,只是相对不足而已,口服降糖药物只起协助自身分泌胰岛素的降糖作用,如果自身分泌胰岛素的功能很差,任何口服降糖药物的治疗效果都不会好。因2型糖尿病是进展性疾病,多数患者在采用单一的口服降糖药物治疗一段时间后都可出现治疗效果下降。因此常采用两种不同作用机制的口服降糖药物进行联合治疗,如口服降糖药物的联合治疗仍不能有效地控制血糖,可采用胰岛素与一种口服降糖药物联合治疗。口服降糖药按其作用机制分为四大类。

\subsubsection{促胰岛素分泌剂}
\paragraph{磺脲类胰岛素促泌剂}

\textbf{格列本脲(Glibenclamide):}

【适应证】 用于轻、中度2型糖尿病。

【用法和用量】 口服:一般患者开始每次2.5mg,早餐前或早餐及午餐前各1次。轻症者每次1.25mg,每日3次,三餐前服。用药7日后剂量递增(每周增加2.5mg)。一般用量为每日5~10mg,最大用量每日不超过15mg。

【不良反应】 常见腹泻、恶心、呕吐、头痛、胃痛或胃肠不适;少见皮疹、严重黄疸、肝功能损害、骨髓抑制、粒细胞减少(表现为咽痛、发热、感染)、血小板减少症(表现为出血、紫癜)等。

【注意事项】 ①体质虚弱、高热、恶心和呕吐、甲状腺功能亢进症、老年人慎用。②用药期间应定期测血糖、尿糖、尿酮体、尿蛋白和肝、肾功能,并进行眼科检查等。③乙醇本身具有致低血糖作用,可延缓本品的代谢。与乙醇合用,可引起腹痛、恶心、头痛、呕吐、面部潮红,且更易发生低血糖反应,用药期间应忌酒。

【禁忌证】 1型糖尿病、糖尿病低血糖昏迷、酮症酸中毒者。妊娠及哺乳期妇女。严重的肾或肝功能不全者。对本品及其他磺酰脲类、磺胺类或赋形剂过敏者。

\textbf{格列喹酮(Gliquidone):}

【适应证】 用于2型糖尿病。

【用法和用量】 口服:应在餐前30min服用。一般每日剂量为15~20mg,酌情调整,通常每日剂量为30mg以内者可于早餐前1次服用;更大剂量应分3次,分别于三餐前服用;最大日剂量不得超过180mg。

【不良反应】 有极少数报道皮肤过敏、胃肠道反应、轻度低血糖反应及血液系统改变。

【注意事项】 糖尿病合并肾病者,当肾功能轻度异常时尚可使用,但严重肾功能不全时,则应改用胰岛素治疗。治疗中若出现不适,如低血糖、发热、皮疹、恶心等应从速就医,一旦发生皮肤过敏反应应停用本品。

【禁忌证】 1型糖尿病、糖尿病低血糖昏迷或昏迷前期、糖尿病合并酮症酸中毒、晚期尿毒症者;对本品及磺胺药过敏者;妊娠及哺乳期妇女。

\textbf{格列吡嗪(Glipizide):}

【适应证】 用于经饮食控制及体育锻炼2~3个月疗效不满意的轻、中度2型糖尿病,但此类患者的胰岛β细胞尚有一定的分泌功能且无急性并发症,不合并妊娠、无严重的慢性并发症。

【用法和用量】 口服:治疗剂量因人而异,根据血糖浓度监测调整剂量。①控释片:常用起始剂量为每日5mg,与早餐同服;对降糖药敏感者可由更低剂量起始;使用本品3个月后测定糖化血红蛋白水平,若血糖未能满意控制可加大剂量;多数患者每日服10mg,部分患者需15mg,最大日剂量20mg。②速释片:一般推荐剂量为每日2.5~20mg,早餐前30min服用;初始剂量每日2.5~5mg,逐渐调整至合适剂量;每日剂量超过15mg时,应分成2或3次,餐前服用。老年、体弱或营养不良者、肝肾功能损害者的起始和维持剂量均应采取保守原则,以避免低血糖发生。

【注意事项】 ①患者用药时应注意饮食、剂量和用药时间。②治疗中注意早期出现的低血糖症状,应及时采取措施,静脉滴注葡萄糖。③必须在进餐前即刻或进餐中服用;治疗时不定时进餐或不进餐会引起低血糖。④肝肾功能不全者会影响本品的排泄,增加低血糖反应发生的危险,应慎用。⑤虚弱或营养不良者应慎用。⑥65岁以上老年人达稳态时间较年轻人约延长1或2d。⑦控释片需整片吞服,不能嚼碎分开和碾碎。⑧对严重胃肠道狭窄的患者(病理性或医源性)应慎用。⑨速释片对体质虚弱、高热、恶心、呕吐、有肾上腺皮质功能减退或垂体前叶功能减退症者慎用。⑩避免饮酒,以免引起戒断反应。

【禁忌证】 ①1型糖尿病、糖尿病低血糖昏迷或昏迷前期、糖尿病合并酮症酸中毒、晚期尿毒症者。②严重烧伤、感染、外伤和大手术、肝肾功能不全者、白细胞计数减少者。③对本品及磺胺药过敏者。④妊娠及哺乳期妇女。
\paragraph{非磺脲类胰岛素促泌剂}

此类药物作用位点与磺脲类类似,也是胰岛β细胞的KATP,通过与SURI的结合导致Kir6.2关闭,最终导致细胞的胞吐作用,促进胰岛素的分泌。但是其与SURI的结合部位与磺脲类不同,它与SU受体I结合和解离速度更快、作用时间更短,加上此类药物吸收速度更快,这些特点决定了此类药物恢复餐后早期胰岛素分泌时相的作用更显著、更符合生理需求、控制餐后血糖的效果更好、发生低血糖的机会更低。属于超短效药物。因此又被称为餐时血糖调节剂。此外,其促胰岛素分泌作用与血糖浓度有关,具有血糖依赖性,血糖浓度高时其作用增强,血糖浓度低时其作用则减弱。即具有“按需促泌”的特点。因而降低餐后高血糖的作用较强,同时低血糖发生率较低。

\textbf{瑞格列奈(Repaglinide):}

【适应证】 用于2型糖尿病。

【用法和用量】 口服:在主餐前15min服用,剂量因人而异。推荐起始剂量为0.5mg,以后如需要可每周或每2周作调整。接受其他口服降血糖药治疗的患者转用本品时的推荐起始剂量为1mg;最大的推荐剂量为4mg,但最大日剂量不应超过16mg。

【不良反应】 偶见瘙痒、皮疹、荨麻疹;罕见低血糖、腹痛、恶心、皮肤过敏反应;非常罕见腹痛、恶心、呕吐、便秘、视觉异常、AST及ALT升高。

【注意事项】 服用本品可引起低血糖,与二甲双胍合用会增加发生低血糖的危险性。乙醇可加重本品导致的低血糖症状,并延长低反应持续时间。

【禁忌证】 ①已知对本品成分过敏者。②1型糖尿病、伴随或不伴昏迷的糖尿病酮症酸中毒、严重肝功能不全者。③妊娠及哺乳期妇女。④12岁以下儿童。⑤严重的肝肾功能不全者。
\paragraph{二肽基肽酶-4(DPP-4)抑制剂}

DPP-4抑制剂通过抑制DPP-4而减少胰高糖素样多肽(GLP-1)在体内的失活,使内源性GLP-1的水平升高。GLP-1以葡萄糖浓度依赖的方式增强胰岛素分泌,抑制胰高血糖素分泌。目前在国内上市的DPP-4抑制剂为西格列汀、沙格列汀和维格列汀。

\textbf{西格列丁(Sitagliptin):}

【适应证】 用于2型糖尿病。

【用法和用量】 口服:每次100mg,每日1次。中至重度肾功能不全者(肌酐清除率为30~50mL/min):每次50mg,每日1次。严重肾功能不全者(肌酐清除率<30mL/min):每次25mg,每日1次。

【不良反应】 服用本品可能出现上呼吸道感染、鼻咽炎、头痛。

【注意事项】 单独使用DPP-4抑制剂不增加低血糖发生的风险。DPP-4抑制剂对体重的作用为中性。

【禁忌证】 ①已知对本品成分过敏者。②1型糖尿病、糖尿病酮症酸中毒者。③妊娠及哺乳期妇女。④18岁以下儿童。

\subsubsection{双胍类药}

主要代表药物有二甲双胍、苯乙双胍。药理作用:抑制肝糖原异生(脂肪、蛋白质变成葡萄糖),降低肝糖从细胞内输出到血液中,增加组织对胰岛素的敏感性,增加胰岛素介导的葡萄糖利用,增加非胰岛素依赖组织(如脑、血细胞、肾髓质、肠道、皮肤)对葡萄糖的利用;从而使基础血糖降低,基础血糖与餐后血糖叠加的高度也随之降低,间接地降低了餐后血糖浓度;该药对肠细胞摄取葡萄糖有抑制作用,对直接降低餐后血糖浓度有一定的作用;该药还有抑制胆固醇的生物合成和贮存,降低血三酰甘油和总胆固醇的作用。

\textbf{二甲双胍(Metformin):}

【适应证】 首选用于单纯饮食控制及体育锻炼治疗无效的2型糖尿病,特别是肥胖的2型糖尿病。对磺酰脲类疗效较差的糖尿病患者与磺酰脲类口服降血糖药合用。

【用法和用量】 口服:从小剂量开始渐增剂量。通常起始剂量为每次0.5g,每日2次;或0.85g,每日1次;随餐服用;可每周增加0.5g,或每2周增加0.85g,逐渐加至每日2g,分次服用。10~16岁的2型糖尿病患者本品的每日最高剂量为2000mg;成人最大推荐剂量为每日2550mg;对需进一步控制血糖浓度的患者,剂量可以加至每日2550mg(即每次0.85g,每日3次);每日剂量超过2g时,为了更好地耐受,最好随三餐分次服用。

【不良反应】 常见腹泻、恶心、呕吐、胃胀、乏力、消化不良、腹部不适及头痛症状;少见大便异常、低血糖、肌痛、头昏、头晕、指甲异常、皮疹、出汗增加、味觉异常、胸部不适、寒战、流感症状、潮热、心悸、体重减轻等症状;罕见乳酸性酸中毒。

【注意事项】 ①定期检查肾功能,可减少乳酸酸中毒的发生,尤其是老年患者更应定期检查。65岁以上老人慎用。②接受外科手术和碘剂X线摄影检查前患者需暂停口服本品。③肝功能不良、既往有乳酸酸中毒史者应慎用。④应激状态:如发热、昏迷、感染和外科手术时,应暂时停用本品,改用胰岛素,待应激状态缓解后再恢复使用。⑤对1型糖尿病患者,不宜单独使用本品,而应与胰岛素合用。⑥本品可减少维生素B$_{12}$
的吸收,应定期监测血常规及血清维生素B$_{12}$
水平。⑦老年、衰弱或营养不良的患者,以及肾上腺和垂体功能低减、乙醇中毒的患者更易发生低血糖。⑧单独接受本品治疗的患者在正常情况下不会产生低血糖,但与其他降糖药联合使用(如磺酰脲类和胰岛素)、饮酒等情况下会出现低血糖,须注意。⑨服用本品治疗血糖控制良好的2型糖尿病患者,如出现实验室检验异常或临床异常(特别是乏力或难于言表的不适),应迅速寻找酮症酸中毒或乳酸酸中毒的证据,测定包括血清电解质、酮体、血糖、血酸碱度、乳酸盐、丙酮酸盐和二甲双胍水平,如存在任何类型的酸中毒都应立即停用本品。

【禁忌证】 ①10岁以下儿童、80岁以上老人、妊娠及哺乳期妇女。②肝肾功能不全者或肌酐清除率异常者。③心功能衰竭(休克)、急性心肌梗死及其他严重心、肺疾病。④严重感染或外伤、外科大手术、临床有低血压和缺氧等。⑤急性或慢性代谢性酸中毒,包括有或无昏迷的糖尿病酮症酸中毒。⑥并发严重糖尿病肾病或糖尿病眼底病变。⑦酗酒者、维生素B$_{12}$
及叶酸缺乏未纠正者。⑧需接受血管内注射碘化造影剂检查前,应暂停用本品。⑨对本品过敏者。

\subsubsection{α-葡萄糖苷酶抑制剂}

主要代表药物有阿卡波糖、伏格列波糖。药理作用为抑制碳水化合物的消化酶(α-葡萄糖苷酶),减缓碳水化合物(如粮食、蔬菜、水果)在肠道消化成葡萄糖的速度,延长吸收时间,降低餐后血糖。该药不被吸收,只在肠道发挥作用。

\textbf{阿卡波糖(Acarbose):}

【适应证】 配合饮食控制用于2型糖尿病;降低糖耐量低减者的餐后血糖浓度。

【用法和用量】 口服:用餐前即刻整片吞服或前几口食物一起咀嚼服用,剂量需个体化。一般推荐剂量为每次50mg,每日3次,以后逐渐增加至每次100mg,每日3次;个别情况下可增至每次200mg。

【不良反应】 常见胃肠胀气和肠鸣音;偶见腹泻、腹胀和便秘,极少见腹痛,个别可能出现红斑、皮疹和荨麻疹等。每日150~300mg用药者个别人发生与临床相关的肝功能检查异常,为一过性的(超过正常高限3倍),极个别情况出现黄疸和(或)肝炎合并肝功能损害。

【注意事项】 ①如果服药4~8周后疗效不明显,可以增加剂量;但如坚持严格的糖尿病饮食仍有不适时不能再增加剂量,有时还需减少剂量。②个别患者尤其是使用大剂量时可发生无症状的肝氨基转移酶升高,应考虑在用药的前6~12个月监测AST及ALT的变化,停药后,肝氨基转移酶值会恢复正常。③本品可使蔗糖分解为果糖和葡萄糖的速度更加缓慢,因此如果发生急性低血糖,不宜使用蔗糖,而应用葡萄糖纠正低血糖反应。④本品应于餐中整片(粒)吞服,若服药与进餐时间间隔过长,则疗效较差,甚至无效。

【禁忌证】 ①妊娠及哺乳期妇女。②有明显的消化和吸收障碍的慢性胃肠功能紊乱患者。③患有由于胀气可能恶化的疾患(如Roemheld综合征、严重的疝气、肠梗阻和肠溃疡)者。④严重肾功能不全(肌酐清除率<25mL/min)者。⑤18岁以下患者。⑥对本品过敏者。

\subsubsection{噻唑烷二酮类(胰岛素增敏剂)}

主要代表药物有罗格列酮、吡格列酮。药理作用:增加组织细胞对胰岛素的敏感性,克服胰岛素与细胞膜胰岛素受体结合障碍或(和)结合后细胞内部的活动障碍,使现有的胰岛素(包括自身分泌的和注射的)发挥更大的作用,并减少高胰岛素血症的不良反应。该类药降低胰岛素抵抗的作用对动脉硬化形成的多种因素有抑制作用,从而降低了患心脑血管病的危险度。

\textbf{罗格列酮(Rosiglitazone):}

【适应证】 用于2型糖尿病。也可与磺酰脲类或双胍类药合用治疗单用时血糖控制不佳者。

【用法和用量】 口服。①单药治疗,初始剂量为每日4mg,单次或分2次口服,8~12周后如空腹血糖下降不满意,剂量可加至每日8mg,单次或分2次口服。②与二甲双胍合用治疗,初始剂量为每日4mg,单次或分2次口服,12周后如空腹血糖下降不满意,剂量可加至每日8mg,单次或分2次口服。③与磺酰脲类合用治疗,剂量为每日2mg或4mg,单次或分2次口服。可空腹或进餐时服用。

【不良反应】 常见上呼吸道感染、外伤、头痛、背痛、高血糖、疲劳、鼻窦炎、腹泻、低血糖;偶见贫血、水肿、充血性心衰、肺水肿和胸腔积液;罕见肝功能异常、血管神经性水肿和荨麻疹;非常罕见黄斑水肿。

【注意事项】 ①心功能衰竭及心功能不全者慎用,对有心衰危险者应严密监测其症状和体征;老年患者可能有轻至中度水肿及轻度贫血。②单药治疗或与其他降糖药合用时可见血红蛋白和血细胞比容下降,轻度白细胞计数减少,可能与治疗后引起血容量增加有关,也可能与剂量相关。③本品可使伴有胰岛素抵抗的绝经前期和无排卵型妇女恢复排卵,随着胰岛素敏感性的改善,女性患者有妊娠的可能。④罕见肝功能异常报告,治疗前应该监测肝功能,此后应当定期检测肝功能。

【禁忌证】 孕妇及哺乳期妇女;Ⅲ级和Ⅳ级(HYHA)心力衰竭者;儿童和未满18岁的青少年;2型糖尿病有活动性肝脏疾患的临床表现或AST及ALT升高大于正常上限2.5倍时;对本品过敏者。

\subsection{2型糖尿病的药物治疗}

\subsubsection{肥胖或超重的2型糖尿病患者的药物选择和治疗程序}

肥胖或超重的2型糖尿病患者在饮食和运动不能满意控制血糖的情况下,应首先采用非胰岛素促分泌剂类降糖药物治疗(有代谢综合征或伴有其他心血管疾病危险因素者应优先选用双胍类药物或格列酮类,主要表现为餐后高血糖的患者也可优先选用α-糖苷酶抑制剂),两种作用机制不同的药物间可联合用药。如血糖控制仍不满意可加用或换用胰岛素促分泌剂。如在使用胰岛素促分泌剂的情况下血糖仍控制不满意,可在口服药基础上开始联合使用胰岛素或换用胰岛素。

此外,肥胖或超重的2型糖尿病患者还可选用能显著降低体重和减少心血管危险因素的作用GLP-1受体激动剂。GLP-1受体激动剂通过激动GLP-1受体而发挥降低血糖的作用,以葡萄糖浓度依赖的方式增强胰岛素分泌、抑制胰高血糖素分泌,并能延缓胃排空,通过中枢性的食欲抑制来减少进食量。可单独使用,单独使用不明显增加低血糖发生的风险。目前国内上市的GLP-1受体激动剂为艾塞那肽和利拉鲁肽,均需皮下注射。

\subsubsection{体重正常的2型糖尿病患者的药物选择和治疗程序}

非肥胖或超重的2型糖尿病患者在饮食和运动不能满意控制血糖的情况下,可首先采用胰岛素促分泌剂类降糖药物或α-糖苷酶抑制剂。如血糖控制仍不满意可加用非胰岛素促分泌剂(有代谢综合征或伴有其他心血管病危险因素者优先选用双胍类药物或格列酮类,α-糖苷酶抑剂适用于无明显空腹高血糖而餐后高血糖的患者)。在上述口服药联合治疗的情况下血糖仍控制不满意,可在口服药基上开始联合使用胰岛素或换用胰岛素。

\subsection{糖尿病合并妊娠的治疗}

在糖尿病诊断之后妊娠者为糖尿病合并妊娠,在妊娠期间发现糖尿病者为妊娠糖尿病。一般来讲,在糖尿病患者合并妊娠时血糖水平波动较大,血糖较难控制,绝大多数患者需要使用胰岛素控制血糖。相反,妊娠糖尿病患者的血糖波动相对较轻,血糖易于控制,绝大多数患者可通过严格的饮食计划和运动使血糖得到满意控制,仅部分患者需要使用胰岛素控制血糖。

\section{骨质疏松症}

骨质疏松综合征(osteoporosis
syndrome)是由各种原因引起的一组以骨强度受损,骨折危险增加为特征的骨骼代谢性疾病。骨质疏松症在代谢性骨病中最为常见,是一种重要的老年性疾病。据估计,20世纪90年代全世界约有2亿人受到骨质疏松症的威胁,大约有7500万人患骨质疏松症。

\subsection{骨质疏松症的诊断和治疗原则}

骨质疏松较轻时常无症状或仅表现为腰背、四肢疼痛、乏力等。严重者机体活动受到明显障碍,日久下肢肌肉往往有不同程度萎缩。可无明显诱因或轻微外伤即可发生骨折,骨折的部位以椎体、股骨颈和尺、挠骨远端为多见。临床上用于诊断骨质疏松症的通用指标是发生了脆性骨折和(或)骨密度低下,但尚缺乏直接测定骨强度的临床手段。

骨质疏松症应及早识别、及早治疗。治疗以促进骨矿化类药物为治疗的基础用药;当骨密度减少但仍在骨折阈值以上时,建议选择骨吸收抑制剂。骨密度下降明显且低于骨折阈值时,建议选择骨吸收抑制剂和骨形成促进剂的联合用药,对于继发性骨质疏松的治疗应以治疗原发病为根本。预防和治疗骨质疏松药物可分为促进骨矿化药物、骨吸收抑制剂、促进骨细胞形成的药物和中药四大类。

\subsection{不同类型骨质疏松症的药物选择}

抗骨质疏松的药物有多种,作用机制也有所不同,或以抑制骨吸收为主;或以促进骨形成为主,也有一些多重作用机制的药物。临床上抗骨质疏松药物的疗效判断包括是否能提高骨量和骨质量,最终降低骨折风险。常用抗骨质疏松药物有以下几种。

\subsubsection{双膦酸盐类(Bisphosphonates)}

双膦酸盐是焦膦酸盐的稳定类似物,其特征是含有P-C-P基团,双膦酸盐与骨骼羟磷灰石有高亲和力的结合,特异性地结合到骨转化活跃的骨细胞表面上抑制破骨细胞的功能,从而抑制骨吸收。不同双膦酸盐抑制骨吸收的效力差很大,因此临床上不同双膦酸盐药物使用的剂量及用法也有所差异。

\textbf{阿仑膦酸钠(Alendronate Sodium):}

【适应证】 适用于治疗绝经后妇女的骨质疏松症,以预防髋部和脊柱骨折。适用于治疗男性骨质疏松症以预防髋部和脊椎骨折。

【用法和用量】 口服:用于骨质疏松症,每次10mg,每日1次,每日早餐前至少30min空腹用200mL温开水送服;或每次70mg,每周1次。连续6个月为一疗程。

【不良反应】 腹痛、腹泻、恶心、便秘、消化不良、食管炎、食管溃疡,无症状性血钙降低、短暂血白细胞计数升高、尿红细胞和白细胞计数升高。

【注意事项】 ①轻、中度肾功能减退者慎用。②有消化不良、吞咽困难、上消化道疾病的患者慎用。③孕妇及哺乳期妇女、儿童均不宜使用。④咖啡、橘子汁可使本品的生物利用度降低60%,在服药2h内,不宜服用钙剂、咖啡、橘子汁等。⑤服用30min内及当日首次进食前,避免躺卧,以防引起食管不良反应(食管炎、糜烂、食管狭窄)。

【禁忌证】 导致食管排空延迟的食管异常,如食管弛缓不能、食管狭窄者禁用;不能站立或坐直至少30min者;对本品任何成分过敏者;低钙血症者。

\subsubsection{降钙素(Calcitonin)}

降钙素是一种钙调节激素,能抑制破骨细胞的活性并能减少破骨细胞的数量,从而减少骨量丢失并增加骨量。降钙素类药物另一突出的特点是能明显缓解骨痛,对骨质疏松骨折或骨骼变形所致的慢性疼痛及骨肿瘤等疾病引起的骨痛均有效,更适合有骨痛的骨质疏松症患者。CFDA批准的适应证为治疗绝经后骨质疏松症。

\textbf{降钙素(Calcitonin):}

【适应证】

(1)骨质疏松症早期和晚期绝经后骨质疏松。为防止骨质进行性丢失,应根据个体的需要给予足量的钙和维生素D。

(2)变形性骨炎。

(3)高钙血症和高钙血症危象。

(4)痛性神经营养不良症。

【用法与用量】

(1)皮下或肌内注射:

① 骨质疏松症,标准维持量,每次50IU,每日1次,或100IU,隔日1次。

② 变形性骨炎,每次100IU,每日1次,治疗时间应至少持续3个月或更长。

③ 高钙血症,高钙血症危象时,每日5~10 IU/kg,溶于500mL
0.9%葡萄糖注射液中,静脉滴注至少6h。慢性高钙血症,每日5~10IU/kg,1次或2次,皮下或肌内注射。

④
痛性神经营养不良症或Sudeck氏病,每次100IU,每日1次,皮下或肌内注射,持续2~4周,以后每周3次,每次100IU,维持6周以上。

(2)鼻喷给药:

① 骨质疏松症,每次100IU,每日1次,或200IU隔日1次。

②
Paget's病,每次200IU,每日1次或分次给药,治疗时间应至少持续3个月或更长。高钙血症或慢性高钙血症时,每日200~400IU,单次给药最高剂量为200IU,需要更大剂量时应分次给药。

③
痛性神经营养不良症或Sudeck氏病,每次200IU,每日1次,持续2~4周,以后隔日1次200IU,连续6周。

【不良反应】 常见面部潮红、头晕、头痛、面部及耳部刺痛、手足刺痛、腹泻、恶心、呕吐、胃痛、过敏、皮疹、荨麻疹、注射部位红肿及胀痛,少见尿频、高血压、视觉障碍,偶见AST及ALT异常、耳鸣、抽搐、低钠血症、出汗、哮喘发作,极少见过敏反应、皮疹、寒战、胸闷、鼻塞、呼吸困难和血糖升高。

【注意事项】

(1)妊娠和哺乳妇女不宜使用。

(2)部分患者在用药中会出现抗体而对治疗产生抵抗性。

(3)鼻炎可加强鼻喷剂的吸收。

(4)鼻喷剂的全身不良反应少于针剂。

(5)可疑对本品或蛋白质过敏者,用药前需做皮试。

【禁忌证】 对本品过敏者或14岁以下儿童禁用。

\subsubsection{雌激素类}

雌激素类药物能抑制骨转换,阻止骨丢失。包括雌激素(ET)和雌、孕激素(EPT)补充疗法,能降低骨质疏松性椎体、非椎体骨折风险;是防治绝经后骨质疏松的有效手段。在各国指南中均被明确列入预防和治疗绝经妇女骨质疏松的药物。

适用于60岁以前围绝经和绝经后妇女,特别是有绝经症状(如潮热、出汗等)及泌尿生殖道萎缩症状的妇女。

【禁忌证】 雌激素依赖性肿瘤(乳腺癌、子宫内膜癌)、血栓性疾病、不明原因阴道出血、活动性肝病及结缔组织病为绝对禁忌证。子宫肌瘤、子宫内膜异位症、乳腺癌家族史、胆囊疾病和垂体泌乳素瘤者慎用。

建议激素补充治疗遵循以下原则:

(1)明确的适应证和禁忌证(保证利大于弊)。

(2)绝经早期(<60岁)开始用,收益更大、风险更小。

(3)应用最低有效剂量。

(4)治疗方案个体化。

(5)局部问题局部治疗。

(6)坚持定期随访和安全性监测(尤其是乳腺和子宫)。

(7)是否继续用药应根据每位妇女的特点每年进行利弊评估。

\subsubsection{选择性雌激素受体调节剂(SERMs)}

SERMs不是雌激素,其特点是选择性地作用于雌激素靶器官,与不同的雌激素受体结合后,发生不同的生物效应。如已在国内外上市的SERMs雷洛昔芬,在骨骼上与雌激素受体结合,表现出类雌激素的活性,抑制骨吸收。

\subsubsection{甲状旁腺激素(PTH)}

PTH是当前促进骨形成药物的代表性药物:小剂量的rhPTH(1-34)有促进骨形成作用。

\subsubsection{氯化物}

对氯化物的研究有很长的历史,体外研究表面氯化物是骨祖细胞强有效的刺激物,但因为多个研究结果不一致,所以目前还只是试验性药物。

\section{痛风和高尿酸血症}

痛风(Gout)是一种可逆性的尿酸盐晶体异常沉积性疾病。嘌呤代谢长期紊乱可导致高尿酸血症,并可由此引起反复发作性痛风性急性关节炎、痛风石沉积、痛风石性慢性关节炎和关节畸形,常累及肾脏,引起慢性间质性肾炎和尿酸肾结石形成。

\subsection{痛风的临床表现及治疗原则}

痛风的临床表现包括急性或慢性痛风性关节炎、痛风性肾病、尿酸性肾结石、痛风石和高尿酸血症。男性和绝经后女性血尿酸浓度>420(mol/L(7.0mg/dL)、绝经前女性血尿酸浓度>350(mol/L(5.8mg/dL)可诊断为高尿酸血症。急性关节炎期确诊有困难时,可试用秋水仙碱作为诊断治疗。如为痛风,服秋水仙碱后症状迅速缓解,具有诊断意义。

流行病学调查发现,原发性痛风患者中,约15%~20%有痛风阳性家族史,从患者近亲中发现15%~25%有高尿酸血症,因此认为原发性痛风是常染色体显性遗传,而其他因素如年龄、性别、饮食习惯及肾功能异常等可能影响痛风遗传的表现形式。

原发性痛风缺乏病因治疗,不能根治;继发性痛风需要在治疗原发疾病的基础上降低血尿酸。不论原发性或者是继发性,本病的临床治疗要求达到以下四个目的:①尽快终止急性关节炎发作;②防止关节炎复发;③纠正高尿酸血症,防治尿酸盐沉积于肾脏、关节等处引起的并发症;④防止尿酸肾结石形成。痛风的治疗原则如下。

(1)急性痛风性关节炎:以控制关节炎的症状(红、肿、痛)为目的。常用药有NSAIDs(阿司匹林及水杨酸钠禁用)和秋水仙碱。如上述两类药效果差或不宜用时可考虑用糖皮质激素。

(2)高尿酸血症的治疗:痛风性关节炎症状基本控制后2~3周开始采取降血尿酸措施,目的是预防急性关节炎复发,导致关节骨破坏和肾结石形成。降血尿酸药物会有抑制尿酸生成的别嘌醇和促使尿酸通过肾脏排出的苯溴马隆及丙磺舒。

(3)非药物治疗:如禁酒、饮食控制、生活调节极为重要。如能遵照可避免或减少口服降尿酸药的许多不良反应和应用剂量。

(4)抗痛风治疗具有终生性。

(5)无症状的高尿血症不一定需要治疗。

\subsection{痛风急性期和发作间期治疗药物的选择}

\subsubsection{急性期治疗药物的选择}
\paragraph{秋水仙碱(Colchicine)}

最古老的治疗急性痛风的药物之一,对急性痛风性关节炎有选择性消炎作用。其作用机制是与粒细胞的微管蛋白结合,从而妨碍粒细胞的活动,抑制粒细胞浸润。它不影响尿酸盐的生成、溶解及排泄,因而没有降低血尿酸的作用。

秋水仙碱治疗急性痛风适用于肝功能或骨髓功能正常,尤其是非甾体类抗炎药禁忌或不能耐受的患者。秋水仙碱用药后12h关节红、肿、热、痛症状即行减轻,48h内症状缓解。秋水仙碱用量参照《英国国家药典》(BNF)的推荐剂量,即起始剂量1mg,继之以每2~3h
50μg的剂量给药直至疼痛缓解或出现呕吐、腹泻等症状或总剂量达6mg,该疗程在3d内不得重复进行。国内某些专家认为BNF推荐的剂量和用法不能为我国痛风患者所耐受,我国患者的用法及用量:首日1.5~3.0mg,分2或3次服用,以后每日0.5~1.5mg,分2或3次服用,连续7~14d为一个疗程。虽然最佳用量尚不能完全统一,但据报道,小剂量秋水仙碱治疗急性痛风具有较好的疗效,不良反应较少见。
\paragraph{NSAIDs}

本类药物逐渐成为治疗急性痛风的一线用药,对缓解急性发作期的各项指标都有明确的作用。虽不及秋水仙碱作用迅速,但也有很好的抗炎镇痛作用,且药源充足,不良反应相对较少,是一种很好的替代药物。可选用其中任何一种,一旦症状减轻即逐渐减量,5~7d后停用。
\paragraph{糖皮质激素(Glucocorticoids)}

治疗急性痛风有效,但一般只是在患者不能耐受秋水仙碱和NSAIDs或有相对禁忌证时使用。应该强调的是,当肾功能不全(血肌酐水平>20mg/L或肌酐清除率<50mL/min)的患者急性痛风发作时,不应选用秋水仙碱或NSAIDs,而应选用糖皮质激素。关节腔内注射也可缓解症状。

\subsubsection{发作间期治疗药物的选择}

在间歇期及慢性期的治疗,主要是维持血清尿酸水平在正常范围和预防急性发作。预防治疗需用秋水仙碱,在刚有症状时,即给予每日0.5~2mg,常可免受急性发作之苦。在体内和体外试验发现,秋水仙碱对快速生长的细胞有抑制作用,如对胃肠道上皮细胞、毛发细胞的影响,会造成胃肠不适和脱发,因此不宜长期应用。为维持正常血清尿酸,则需用促进尿酸排泄药和抑制尿酸生成药,为保证有效,药量要足,并终生维持。

在肾功能正常或有轻度损害,及24h尿液中尿酸排出量在600mg以下时,可用排尿酸药;在中度以上肾功能障碍(肌酐清除率<35mL/min),24h尿液中尿酸水平明显升高时应用别嘌醇;血尿酸明显升高及痛风石大量沉积的患者,可联用以上两种药物,以防止渐进性痛风性并发症。为预防促发急性关节炎发作,开始时用小剂量,在7~10d内逐渐加量。
\paragraph{排尿酸药}

为防止尿酸在肾脏大量排出时引起肾脏损害及肾结石的不良反应,均应从小剂量开始,并考虑碱化尿液。

(1)丙磺舒(Probenecid):开始剂量为每次0.25g,每日2次,2周后增至0.5g,每日3次,每日最大剂量不超过2g。有皮疹、发热、胃肠刺激、肾绞痛、激发急性发作等不良反应。

(2)苯溴马隆(Benzbromarone):排尿酸作用较丙磺舒更强。开始剂量为每次25mg,每日1次,逐渐可增至100mg。不良反应轻微,可有胃肠道反应,极少数有皮疹、发热、肾绞痛及激发急性发作,对肝肾功能无影响。
\paragraph{抑制尿酸生成药}

别嘌醇(Allopurinol):可迅速降低血尿酸值,抑制痛风石和肾结石形成,并促进痛风石溶解。每次100mg,每日2或3次,每日最大剂量低于600mg。不良反应有过敏性皮疹、药热、肠胃不适及消化道出血、白细胞和血小板计数减少、肝功能损害等。因此,在用药过程中,应定期检查白细胞、血小板计数及肝功能,溃疡病者慎用。