\chapter{急性中毒的药物治疗}

急性中毒是指毒物短时间内经皮肤、黏膜、呼吸道、消化道等途径进入人体,使机体受损并发生器官功能障碍甚至死亡的过程。根据毒物种类将急性中毒分为化学性毒物中毒(包括药物、农药、有害气体和有机溶剂)、植物性毒物中毒和动物性毒物中毒。急性中毒起病急骤,症状严重,病情变化迅速,不及时治疗常危及生命,必须尽快做出诊断与急救处理。

急性中毒患者的救治一般采取以下原则:立即将患者脱离中毒现场;清除体内尚未被吸收的毒物,如催吐、洗胃、导泻、灌肠等;采取强效利尿剂、吸氧、增加血容量、血液透析、血液灌流等措施促进已吸收毒物的排出;结合临床症状进行对症支持治疗,维持生命体征;尽早采用特异性拮抗毒物的药物治疗。

\section{常见药物中毒}

\subsection{巴比妥类药物中毒}

\subsubsection{中毒机制}

巴比妥类药物主要阻断脑干网状结构上行激活系统,有剂量-效应关系:因剂量而异,而依次产生镇静、催眠、抗惊厥和中枢麻痹作用。急性中毒时以中枢神经抑制为主,且明显影响呼吸、心血管及消化系统。

\subsubsection{临床表现}

急性巴比妥类中毒患者,中枢神经系统高度抑制,患者感觉迟缓,言语不清,定向力障碍至深度昏迷,并出现周期性脑电图异常,瞳孔扩大,角膜、咽及腱反射消失。大剂量巴比妥类可降低延髓呼吸中枢对二氧化碳分压和pH值变化的敏感性,使呼吸抑制;后期可使心肌收缩力减弱,心排出量下降,且降低外周血管肌弹性,增加血管通透性和静脉瘀血,减少回心血量,进一步降低心排出量,导致体温下降、低血压、少尿或无尿,最终因呼吸、循环和肾衰竭死亡。

\subsubsection{治疗原则}
\paragraph{洗胃及导泻}

对服药量大且未超过4~6h者应用1∶5000高锰酸钾溶液或清水洗胃。用硫酸钠20~30g导泻(忌用硫酸镁,因镁离子吸收而加重中枢神经系统的抑制),也可用20%的药用炭悬液。
\paragraph{一般处理}

常规吸氧,补液维持水、电解质及酸碱平衡,循环不稳定可使用血管活性药物;呼吸衰竭可气管插管及机械通气。防治肺部感染、脑水肿、休克等并发症。
\paragraph{血液净化}

血液透析是有效方法,对服药剂量大、昏迷程度深、洗胃不彻底的患者更应尽早实施,有条件可行血液灌流,如患者不宜搬动也可用腹膜透析。

\subsubsection{药物治疗}
\paragraph{中枢兴奋剂}

对深度昏迷、呼吸浅或不规则,选用下列药物。

(1)贝美格(乙甲哌啶二酮):能直接兴奋呼吸中枢及血管运动中枢,使呼吸增加,血压微升。50mg稀释于葡萄糖注射液中静脉滴注,密切观察,如出现恶心、呕吐、肌肉颤抖等中毒症状需减量或停药。

(2)尼可刹米:静脉注射0.375~0.75g/h,直至角膜反射与肌肉颤抖出现。

(3)纳洛酮:0.4~0.8mg加入葡萄糖注射液中静脉滴注,直至呼吸或(和)意识状态明显改善。中枢兴奋剂量过大可引起惊厥或出现心律失常,加重呼吸、循环衰竭,凡遇肌张力及反射恢复或出现肌肉震颤等情况均应减量或停药。
\paragraph{利尿剂}

巴比妥钠和苯巴比妥主要从肾脏排出,在补充血容量后用20%甘露醇250mL静脉滴注,每8~12h
1次,对合并颅内高压者尤为适合;亦可间断静脉注射呋塞米20~40mg;静脉滴注5%碳酸氢钠碱化尿液,可促进药物排泄。

\subsection{阿片类药物中毒}

\subsubsection{中毒机制}

阿片类药物主要包括阿片,吗啡、可待因、复方樟脑酊和罂粟碱等,以吗啡为代表。吗啡可抑制大脑高级中枢而昏迷,随即抑制延脑呼吸中枢导致呼吸抑制。同时兴奋催吐化学感受器产生恶心、呕吐,兴奋脊髓,使肠道平滑肌、括约肌张力增强,降低胃肠蠕动产生便秘,对支气管、输尿管和胆管平滑肌的收缩作用导致尿潴留、胆绞痛。兴奋动眼神经缩瞳核,产生针尖样瞳孔。当大剂量吗啡中毒后抑制延脑血管运动中枢,释放组胺,使周围组织血管扩张导致心动过缓、低血压状态以至休克。

\subsubsection{临床表现}

轻者有头痛头晕、恶心呕吐、兴奋或抑制、幻想、时间和空间感消失,以及血糖浓度增高。重者发绀、面色苍白、肌肉无力,有昏迷、针尖样瞳孔、呼吸抑制“三联征”特点,惊厥、牙关紧闭、角弓反张,呼吸先浅而慢,后叹息样或潮式呼吸、肺水肿、休克、瞳孔对光反射消失,死于呼吸衰竭。

\subsubsection{治疗原则}
\paragraph{清除毒物}

首先明确中毒途径,以便迅速排出毒物。采取洗胃和导泻的方法阻止毒物的吸收;应用利尿剂或高渗葡萄糖注射液尽快促使毒物排出体外。
\paragraph{对症支持疗法}

(1)维持水、电解质、酸碱、血气平衡,调整体循环容量,保持足够尿量。

(2)纠正非心源性肺水肿。因纳洛酮对非心源性肺水肿无作用,主要靠维持呼吸道通畅,确保氧疗效果,必须不失时机的气管插管,气管切开,应用呼吸机,给以呼气末正压人工呼吸方可有效纠正肺水肿。

\subsubsection{药物治疗}
\paragraph{特效解毒剂}

(1)纳洛酮:阿片受体拮抗剂,对阿片受体亲和力大于吗啡类药物,阻止吗啡样物质与受体结合,用药后迅速逆转阿片类药物所致的昏迷,呼吸抑制、缩瞳等毒性作用。首次量为每次0.4~0.8mg,肌肉注射或静脉注射,10~20min重复,直至呼吸恢复正常或总量达10mg后,视病情减量。

(2)烯丙吗啡:别名烯丙吗啡,对抗吗啡作用,首次剂量为5~10mg肌肉注射或静脉注射,必要时20min重复1次,总量不超过40mg。
\paragraph{中枢呼吸兴奋剂}

(1)安钠咖(苯甲酸钠咖啡因):0.5g肌肉注射,每4小时1次;尼可刹米0.25~0.5g静脉注射或肌肉注射。

(2)严格禁用士的宁、印防己毒素等中枢兴奋剂,因其与吗啡对脊髓兴奋起协同作用而引起惊厥。

\subsection{吩噻嗪类药物中毒}

\subsubsection{中毒机制}

吩噻嗪药物主要作用于网状结构,抑制中枢神经系统多巴胺受体,以减轻焦虑紧张、幻觉妄想和病理性思维等精神症状;抑制脑干血管运动和呕吐反射,以及阻断α肾上腺素能受体、抗组胺及抗胆碱能等作用。药物过量引起过度抗多巴胺作用使乙酰胆碱相对占优势,出现锥体外系兴奋症状;抗α肾上腺素能受体作用引起低血压甚至休克。

\subsubsection{临床表现}

神经系统可表现为抽搐、昏迷和反射消失,锥体外系兴奋症状如震颤麻痹综合征、静坐不能及强直反应等。心血管系统主要表现为心动过速、心律失常、低血压甚至休克等。

\subsubsection{治疗原则}

(1)口服中毒未超过6h者,用1∶5000高锰酸钾溶液或温开水洗胃,硫酸镁导泻。

(2)应平卧,尽量少搬动头部,避免体位性低血压。血压过低时,可选用间羟胺等升压药物。

(3)静注高渗葡萄糖或右旋糖酐,促进利尿,排泄毒物,输出量不可过多,以预防心力衰竭和肺水肿。

\subsubsection{药物治疗}

本类药物尚无特效解毒剂,治疗以对症及支持为主。中枢神经系统抑制较重时可用苯丙胺、咖啡因等。如进入昏迷状态,可用哌甲酯(40~100mg)肌注,必要时每0.5~1h重复应用,直至苏醒。如有震颤麻痹综合征可选用苯海索(安坦)、东莨菪碱等。若有肌肉痉挛及张力障碍,可用苯海拉明25~50mg口服或肌注20~40mg。血压过低可选用间羟胺及去氧肾上腺素等α受体兴奋剂,禁用异丙基肾上腺素、多巴胺等β受体兴奋剂,避免加重低血压(因周围β肾上腺素能有血管扩张作用)。心律失常应用利多卡因纠正。

\section{急性乙醇中毒}

乙醇又称酒精,一次饮入过量乙醇,可引起以神经精神症状为主的疾病即为急性乙醇中毒。

\subsection{中毒机制}

乙醇具有脂溶性,可迅速透过大脑神经细胞膜,并作用于膜上的某些酶而影响细胞功能。乙醇对中枢神经系统的抑制作用,随着剂量的增加,由大脑皮质向下,通过边缘系统、小脑、网状结构到延髓。小剂量出现兴奋作用,极高浓度乙醇抑制延髓中枢引起呼吸或循环衰竭。乙醇在肝细胞内代谢生成大量还原型烟酰胺腺嘌呤二核苷酸(NADH),使还原型与氧化型比值增高(NADH/NAD),相继发生乳酸增高、酮体蓄积导致的代谢性酸中毒以及糖异生受阻所致低血糖。长期大量饮酒进食减少,可使维生素缺乏导致周围神经麻痹。乙醇刺激腺体分泌,可引起食管炎、胃炎或胰腺炎。乙醇主要在肝脏代谢、降解,代谢产生大量自由基,可引起细胞膜脂质过氧化,造成肝细胞坏死,肝功能异常。

\subsection{临床表现}

乙醇中毒者呼出气中有浓厚的乙醇味,临床表现与患者的饮酒量、耐受性和血乙醇浓度有关。可分为兴奋期、共济失调期、昏迷期三期。

(1)兴奋期:表现为头痛、欣快感、健谈、情绪不稳定、易激怒,有时可沉默、孤僻或入睡。

(2)共济失调期:表现为言语不清、视物模糊,复视、眼球震颤、步态不稳、行动笨拙、共济失调等,易并发外伤。

(3)昏迷期:表现为昏睡、瞳孔散大、体温降低、心率增快、血压降低、呼吸减慢并有鼾音,严重者因呼吸、循环衰竭而死亡。无乙醇耐受者清醒后,可有头痛、头晕、无力、恶心、震颤等症状,耐受者症状较轻。重症中毒患者常发生轻度酸碱平衡失调、低血糖和肺炎等并发症,严重者可发生急性肌病,表现为肌肉肿胀、疼痛或伴有肌球蛋白尿。

\subsection{治疗原则}

(1)对一般较轻的酒醉者无须特殊治疗,可使其静卧、保暖,待自行恢复。

(2)对烦躁不安、过度兴奋者可压迫舌根催吐,并肌注地西泮5~10mg。

(3)对昏迷者,可静脉注射纳洛酮。

(4)重度中毒者,可静脉注射50%葡萄糖100mL,同时皮下注射普通胰岛素20IU,肌肉注射维生素B{6}
和烟酸各100mg,加速乙醇在体内氧化。极严重者可予透析治疗。

(5)呼吸表浅缓慢而呈呼吸衰竭现象者,肌肉注射尼可刹米或洛贝林,必要时进行人工呼吸。

\subsection{药物治疗}

纳洛酮为羟-2-氢吗啡酮衍生物,是阿片样物质的特异性拮抗剂。纳洛酮与阿片受体亲和力远大于吗啡及脑啡肽,静脉用纳洛酮能迅速逆转阿片样物质的作用,拮抗内源性吗啡样物质介导的各种效应,使交感神经及肾上腺髓质分泌释放儿茶酚胺、前列腺素增加,呼吸兴奋,血压上升,促进急性乙醇中毒患者清醒。纳洛酮可以拮抗急性乙醇中毒时增高的β-内啡肽对中枢神经系统的抑制作用,可以防止和逆转乙醇中毒,从而催醒与解除乙醇中毒而达到治疗作用。

\section{农药中毒}

农药是指在农业生产中,为保障、促进植物和农作物的成长,所施用的杀虫、杀菌、杀灭有害动物(或杂草)的一类药物统称。

\subsection{有机磷农药中毒}

有机磷农药属有机磷酸酯或硫化磷酸酯类化合物,多呈黄色或棕色油状脂溶性液体,少数为结晶固体,易挥发,遇碱易分解,有蒜臭。目前使用的种类很多,如:甲拌磷(3911)、内吸磷(1059、E1059)、对硫磷(1605、E605)、敌敌畏(DDV)、乐果、敌百虫等,敌百虫在碱性溶液中还可变成毒性较强的敌敌畏。有机磷农药对人、畜均有毒性,可经皮肤、黏膜、呼吸道、消化道侵入人体,引起中毒。

\subsubsection{中毒机制}

有机磷酸酯进入机体后,其磷酸根与胆碱酯酶活性部分紧密结合,形成磷酰化胆碱酯酶,使其丧失水解乙酰胆碱的能力,导致乙酰胆碱蓄积,产生一系列中毒症状。

\subsubsection{临床表现}

(1)毒蕈碱样症状:主要表现平滑肌痉挛及腺体分泌亢进、瞳孔缩小、视物模糊、光反应消失、面色苍白、多汗、流涎、恶心、呕吐、腹泻、支气管痉挛、胸闷、呼吸困难,肺水肿。

(2)烟碱样毒症状:表现为肌纤维颤动,常自小肌群开始,有眼睑、颜面、舌肌颤动、渐及全身如:牙关紧闭,腓肠肌痉挛,全身肌肉抽搐,严重时有肌力减退,甚至瘫痪。

(3)中枢神经系统症状:头痛、头晕、烦躁、嗜睡、神志恍惚、共济失调、抽搐、昏迷等。

(4)其他:部分患者有中毒性心肌损害,心律失常,心力衰竭,局部可有接触性皮炎,皮肤出现红肿、水泡等。慢性中毒症状较轻,表现为头昏、乏力、记忆力下降、厌食、恶心等。有机磷农药中毒治疗后,一般不留后遗症,个别患者可发生下肢瘫痪及周围神经炎。

\subsubsection{治疗原则}

(1)迅速清除毒物,移离现场,脱去被污染的衣物,彻底清污染的头发、皮肤等。除敌百虫中毒外,均可用冷肥皂水或2%碳酸氢钠溶液,彻底清洗污染部位。敌百虫中毒可用清水清洗,防止残余毒物继续被吸收,口服中毒时应立即洗胃,忌用碳酸氢钠洗胃,因敌百虫遇碱性溶液可迅速转化为毒性更强的敌敌畏,故选用温水洗胃。洗胃后灌入50%硫酸镁或硫酸钠40~50mL导泻。

(2)尽早使用解毒特效药。

(3)积极应对并发症,如休克、肺水肿、脑水肿、感染等。

\subsubsection{药物治疗}

解毒治疗:尽早使用抗胆碱药和胆碱酯酶复能剂。

(1)阿托品:具有拮抗乙酰胆碱的作用,可消除或减轻毒蕈碱样症状。

①
给药原则:早期足量直至阿托品化、注意防止因剂量不足至病情反复,而影响预后。患者对阿托品的耐受量及阿托品化所需剂量有较大个体差异应密切观察病情变化,增减剂量,注意观察和判断阿托品化与中毒的临床表现。

②
阿托品化:瞳孔较前逐渐扩大、不再缩小,但对光反应存在,流涎、流涕停止或明显减少,面颊潮红,皮肤干燥,心率加快而有力,肺部啰音明显减少或消失。阿托品化后,注意逐渐减少药量或延长用药间隔时间,防止阿托品中毒或病情反复。

③
阿托品中毒表现:烦躁不安、甚至出现幻觉、狂躁等精神症状,瞳孔明显散大,对光反应迟钝或消失,无汗性高热可达40℃以上,心动过速,160次/min,尿潴留。严重阿托品过量患者,可转为抑制状态,出现昏迷、呼吸中枢衰竭。遇有阿托品中毒可选用拟胆碱药、毛果芸香碱、毒扁豆碱、新斯的明等拮抗剂,并增加输液量,促使排泄。

(2)胆碱酯酶复活剂:氯解磷定、碘解磷定是肟类化合物,使被抑制的乙酰胆碱酯酶活力恢复,有解除烟碱样毒作用,但只对形成不久的磷酰化胆碱酯酶有作用。数日后,磷酰化胆碱酯酶“老化”,其酶的活性即难以恢复。此类药物中毒早期使用效果较好,对慢性中毒无效,解磷定对1605、1059、特普、乙硫磷疗较好,而对敌敌畏、乐果、敌百虫、马拉硫磷效果差或无效。须与阿托品合用,可提高疗效。

\subsection{氨基甲酸酯杀虫剂中毒}

氨基甲酸酯类农药的中毒原因与其他农药中毒类似。生产性中毒多见于未采取适当防护措施或高温湿热环境下施洒农药,主要经皮肤和呼吸道吸入中毒。生活性中毒多见于服毒自杀或误服,为经口中毒。

\subsubsection{中毒机制}

氨基甲酸酯类杀虫剂的立体结构式与乙酰胆碱相似,可与乙酰胆碱酯酶阴离子部位和酯解部位结合,形成可逆性的复合物,即氨基甲酰化,使其失去水解乙酰胆碱的活力,引起乙酰胆碱蓄积,刺激胆碱能神经兴奋,产生相应的临床表现。但氨基甲酰化酯酶易被水解,酶活性常在数小时内自然恢复,故临床症状较有机磷农药中毒轻且恢复较快。

\subsubsection{临床表现}

氨基甲酸酯类农药中毒表现与有机磷农药中毒类似,症状相对较轻。中毒症状的开始时间与严重程度与进入体内的毒物量有关。生产性中毒一般在连续工作3h后开始出现,而生活性中毒则可在较短的时间内出现中毒症状。生产性中毒者开始时感觉不适并可能有恶心、呕吐、头痛、眩晕、疲乏、胸闷等;以后患者开始大量出汗和流涎,视觉模糊,肌肉自发性收缩、抽搐,心动过速或心动过缓,少数患者出现阵发痉挛和进入昏迷。一般在24h内完全恢复(极大剂量的中毒者除外),无后遗症和遗留残疾。经口中毒者,症状进展迅速,短时间内出现呕吐、流涎、大汗等毒蕈碱样症状;服毒量大者可迅速出现昏迷、抽搐,甚至呼吸衰竭而死亡。

\subsubsection{治疗原则}

(1)毒物清除:用清水或肥皂水清洗全身,注意清洗毛发、腋窝、腘窝、会阴部等部位。经口中毒者立即洗胃,直至洗出液无色无味。

(2)尽快应用特效解毒药。

\subsubsection{药物治疗}

阿托品为治疗氨基甲酸酯类农药中毒首选药物,疗效极佳,能迅速控制由胆碱酯酶受抑制所引起的症状和体征,首剂2~5mg口服或肌注,必要时重复1或2次,不必应用过大剂量。

由于氨基甲酸酯类农药在体内代谢迅速,胆碱酯酶活性恢复很快,肟类胆碱酯酶复活剂需要性不大;有些氨基甲酸酯类农药如急性西维因中毒,使用肟类胆碱酯酶复活剂会增强毒性和抑制胆碱酯酶活性,影响阿托品治疗效果,故氨基甲酸酯类农药中毒禁用肟类胆碱酯酶复活剂治疗。

\subsection{灭鼠药中毒}

灭鼠药种类较多,包括有机氟类、磷化锌类、毒鼠磷类和氰化物类,中毒机制不尽相同,常用者为二苯茚酮(敌鼠钠)等。

\subsubsection{中毒机制}

二苯茚酮为抗凝血杀鼠剂,化学结构与香豆素类相似,可与维生素K竞争肝脏中有关酶,从而抑制凝血因子生成,亦可损伤毛细血管,增加管壁通透性,造成内脏及皮下出血。

\subsubsection{临床表现}

误食中毒者有恶心、呕吐、食欲缺乏、腹痛等。1~2d后有全身出血症状,凝血时间、凝血酶原时间延长。

\subsubsection{治疗原则}

中毒者应立即清洗肠胃,排除毒物;同时采用药物对抗中毒反应。

\subsubsection{药物治疗}

维生素K{1}
为特效解毒药,维生素K是肝脏合成因子Ⅱ、Ⅶ、Ⅸ、Ⅹ所必需的物质,参与凝血因子的合成及凝血作用的活化。肌肉或静脉注射维生素K,每次10~20mg,每日2或3次;严重时加大剂量连续治疗数日。还可适量给予注射用血凝酶每日1~2kU,同时大剂量使用维生素C和糖皮质激素降低毛细血管通透性。

\section{有害气体和化学物质中毒}

\subsection{急性一氧化碳中毒}

一氧化碳(CO)为无色、无味、无臭的气体,凡是碳或含碳物质不充分时燃烧,均可产生CO。在使用柴炉、煤炉时,如通风系统不畅通,尤其是近年来煤气取暖器和煤气热水器使用不当使CO中毒大为增加。

\subsubsection{中毒机制}

CO经呼吸吸入肺后,通过肺泡壁弥散入血与血红蛋白(Hb)结合成碳氧血红蛋白(COHb);由于CO与Hb的亲和力比氧大,COHb离解却比正常Hb慢。因此,血液中CO与氧竞争Hb时,大部分血红蛋白成为HbCO。COHb携氧能力差,引起组织缺氧,而COHb解离曲线左移,血氧不易释放更加重组织缺氧。

此外,CO还可与还原型细胞色素氧化酶的二价铁结合,抑制该酶活性,影响组织细胞呼吸与氧化过程,阻碍对氧利用。由于中枢神经系统对缺氧耐受性最差,首先受累,严重者发生缺氧窒息死亡或造成永久性神经系统损害。

\subsubsection{临床表现}

脑缺氧和中毒的症状体征是CO中毒的主要表现。轻度脑缺氧可表现为头晕、眼花、头痛、全身疲乏无力、恶心呕吐、胸闷、心悸等。重度脑缺氧患者表现为昏迷伴有肌张力增高,并出现意识障碍,严重者可死于呼吸循环衰竭。严重中毒患者经抢救存活后可留有不同程度的后遗症。

\subsubsection{治疗原则}

(1)CO中毒的治疗目的是尽快避免继续吸入CO和及早排出CO。高压氧治疗可以增加血液中物理溶解氧,供组织、细胞利用,加速COHb解离,促进CO清除。

(2)采用利尿药防治脑水肿及并发症。

\subsubsection{药物治疗}

药物治疗的目的是减轻或消除脑水肿,防治并发症。静脉注射高渗甘露醇能迅速增加尿量及排除\ce{Na^+}
、\ce{K^+}
,并迅速提高血浆渗透压,使组织间液水分向血浆转移而产生组织脱水作用。呋塞米为高效利尿剂,排除大量等渗尿液。皮质激素类药物可减轻组织反应,与甘露醇合用对脑水肿的防治有效,但其临床价值尚有待验证。有频繁抽搐者,首选地西泮10~20mg静脉注射。抽搐停止后再静脉滴注苯妥英钠0.5~1g,剂量可在4~6h内重复应用。

\subsection{甲醇中毒}

\subsubsection{中毒机制}

甲醇对神经系统有麻醉作用;甲醇经脱氢酶作用,代谢转化为甲醛、甲酸,一方面抑制某些氧化酶系统,致需氧代谢障碍,体内乳酸及其他有机酸积聚,引起酸中毒;另一方面其代谢物甲醛、甲酸在眼房水和眼组织内含量较高,致视网膜代谢障碍,易引起视网膜细胞、视神经损害及视神经脱髓鞘。

\subsubsection{临床表现}

中毒早期呈酒醉状态,出现头昏、头痛、乏力、视力模糊和失眠,严重时谵妄、意识模糊、昏迷甚至死亡,双眼可有疼痛、复视甚至失明,眼底检查视网膜充血、出血、视神经乳头苍白及视神经萎缩等,血液中甲醇、甲酸浓度升高,个别有肝、肾功能损害,二氧化碳结合力降低,血气分析可见pH值降低,SB减少及BE负值增加等指标的改变,慢性中毒可出现视力减退、视野缺损、视神经萎缩以及伴有神经衰弱综合征和自主神经功能紊乱等。

\subsubsection{治疗原则}

对症支持治疗。口服急性中毒者以3%~5%碳酸氢钠洗胃,严重者作血液透析或腹膜透析,以清除体内甲醇,并纠正酸中毒。

\subsubsection{药物治疗}

甲醇中毒无特效解毒药。以20%甘露醇250mL加地塞米松5~10mg静脉滴注防治脑水肿等措施抢救。慢性中毒及视神经损害、视神经萎缩者,给予地巴唑、烟酸及维生素B{1}
、B{12}
等血管扩张剂、神经营养药每日3次口服治疗;必要时配用肾上腺糖皮质激素,如泼尼松或地塞米松口服治疗,保护视神经,促进其恢复。

\subsection{苯中毒}

\subsubsection{中毒机制}

苯的亲脂性很强,且多聚集于细胞膜内,使细胞膜的脂质双层结构肿胀,影响细胞膜蛋白功能,干扰细胞膜的脂质和磷脂代谢,抑制细胞膜的氧化还原功能,致中枢神经麻醉。

苯代谢产物抑制骨髓基质生成造血干细胞,干扰细胞增殖和分化的调节因子,阻断造血干细胞分化过程而诱发白血病。同时苯的酚类代谢产物,可直接毒害造血细胞,并通过巯基作用使维生素C和谷胱甘肽代谢障碍。

\subsubsection{临床表现}

急性中毒主要为中枢神经系统抑制症状,轻者酒醉状,伴恶心、呕吐、步态不稳、幻觉、哭笑失常等表现,重者意识丧失、肌肉痉挛或抽搐、血压下降、瞳孔散大、呼吸麻痹、心室颤动等。

\subsubsection{治疗原则}

对于吸入中毒患者,可以立即脱离现场至空气新鲜处,脱去污染的衣着,并用肥皂水或清水冲洗污染的皮肤。口服中毒者,用清水或3%~5%碳酸氢钠洗胃,清除毒物。同时注意维持水、电解质及酸碱平衡。

\subsubsection{药物治疗}

葡萄糖醛酸,可与体内苯的代谢物酚类等结合成为低毒的苯基葡萄糖醛酸钠而起解毒作用。对于呼吸抑制者给予尼可刹米等呼吸兴奋剂,血压下降者给予多巴胺、间羟胺等血管活性药物。

\subsection{亚硝酸盐中毒}

\subsubsection{中毒机制}

亚硝酸盐是氧化剂,进入人血循环后,使血红蛋白中的二价铁氧化为三价铁,产生大量高铁血红蛋白。高铁血红蛋白无携氧功能,使组织缺氧。此外,亚硝酸盐还有松弛小血管平滑肌的作用,使血管扩张,大量摄入可致血压下降。

\subsubsection{临床表现}

主要表现为缺氧和发绀。常有头痛、头晕、乏力、胸闷、气短、心悸、恶心、呕吐、腹痛、腹泻、腹胀等。全身皮肤及黏膜呈现不同程度青紫色。严重者出现烦躁不安、精神萎靡、反应迟钝、意识丧失、惊厥、昏迷、呼吸衰竭甚至死亡。

\subsubsection{治疗原则}

口服中毒者应催吐、洗胃及导泻。尽快给予特效解毒药,吸氧等。同时采取对症支持治疗,应用血管活性药物(间羟胺)、呼吸兴奋剂(尼可刹米)、镇静剂、维生素C等。

\subsubsection{药物治疗}

亚甲蓝是亚硝酸盐中毒的特效解毒剂,能还原高铁血红蛋白,恢复正常输氧功能。用药时注意亚甲蓝为氧化还原剂,只有在低浓度(1~2mg/kg)时才使高铁血红蛋白还原为血红蛋白,而高浓度时则使血红蛋白氧化为高铁血红蛋白。因此,治疗时严格控制亚甲蓝的剂量及注射速度,否则会使病情加重。

\section{动植物中毒}

\subsection{毒蛇咬伤}

\subsubsection{中毒机制}

(1)蛇毒:可分为神经毒素、血液循环毒素和混合毒三类。

(2)神经毒素:对中枢、周围神经、神经肌肉传导功能等产生损害作用,可引起惊厥、瘫痪和呼吸麻痹。

(3)血液循环毒素:包含凝血毒、抗凝血毒、纤溶酶、溶血毒等,可引起凝血、出血、溶血、毛细血管损伤等,对心血管和血液系统造成损害,引起心律失常、循环衰竭、溶血和出血。

(4)混合毒:包括神经毒素和血液循环毒素。

\subsubsection{临床表现}

(1)神经毒表现:伤口局部出现麻木,知觉丧失,或仅有轻微痒感,伤口红肿不明显,出血不多,约在伤后半小时出现全身中毒症状。首先感觉头昏、嗜睡、恶心、呕吐及乏力,接着出现神经症状并迅速加剧,主要为眼睑下垂、视力模糊、斜视、言语障碍、吞咽困难、流涎、眼球固定和瞳孔散大。重症患者呼吸由浅而快且不规则,最终出现中枢或周围性呼吸衰竭。

(2)血液循环毒表现:咬伤的局部迅速肿胀,并不断向近侧发展,伤口剧痛,流血不止,伤口周围的皮肤常伴有水泡或血泡、皮下瘀斑、组织坏死,严重时全身广泛性出血,如结膜下瘀血、鼻衄、呕血、咯血及尿血等,部分蛇毒还会出现胸腔、腹腔出血及颅内出血,最后导致出血性休克。大量溶血引起血红蛋白尿,出现血压下降、心律失常、循环衰竭和急性肾衰竭。

(3)混合毒表现:常见于眼镜蛇、眼镜王蛇、蝮蛇等咬伤,兼有上述两类表现但又侧重不同,眼镜蛇以神经毒为主,蝮蛇以血液循环毒性为主。

\subsubsection{治疗原则}

毒蛇咬伤后应采取各种措施,迅速排出毒液并防止毒液的吸收与扩散。为减少毒液吸收,在伤口近心端有效绷扎、冲洗和吸毒外,采用糜、胰蛋白酶4000IU以2%利多卡因5mL溶解在伤口及周围皮下进行局封。及早应用抗蛇毒血清,应用破伤风抗毒素预防破伤风,抗生素预防感染;同时,采取对症支持治疗,防治休克、肾功能衰竭、呼吸衰竭等。

\subsubsection{药物治疗}

抗蛇毒血清是中和蛇毒的特效药。抗蛇毒血清主要成分为经胃酶消化后的马蛇毒免疫球蛋白,一般有抗蝮蛇毒血清、抗五步蛇毒血清、抗眼镜蛇毒血清和抗银环蛇毒血清。使用时根据毒蛇类型选用相应抗蛇毒血清,对无特异性抗蛇毒血清的蛇毒伤,可选用相同亚科的抗蛇毒血清。应用抗蛇毒血清前要做皮肤过敏试验,阴性者才可应用。

对症治疗需要注意,神经毒和混合毒蛇类咬伤后,忌用中枢神经抑制药如吗啡、氯丙嗪、苯海拉明等以及横纹肌抑制药。血液循环毒类蛇咬伤,忌用肾上腺素和抗凝血药。

\subsection{蜂螫伤}

\subsubsection{中毒机制}

蜂类毒液中主要含有多种酶、肽类、蚁酸、神经毒素和组胺等,进入体内能引起局部严重的炎性反应,如大量毒素吸收可引起全身炎性反应,严重者出现溶血、出血、急性肾功能衰竭。

\subsubsection{临床表现}

蜂蜇伤后,轻者仅局部出现红肿、疼痛、灼热感,也可有水泡、瘀斑、局部淋巴结肿大。如果被蜂群螫伤多处,常引起发热、头痛、头晕、恶心、烦躁不安、昏厥等全身症状。蜂毒过敏者,可引起荨麻疹、鼻炎、唇及眼睑肿胀、腹痛、腹泻、恶心、呕吐,个别严重者可致喉头水肿、气喘、呼吸困难、昏迷,终因呼吸、循环衰竭而死亡。

\subsubsection{治疗}

局部被蜇者寻找到蜂针并拔除,减少毒素的吸收。局部用3%氨水、5%碳酸氢钠溶液或肥皂水洗净。

全身中毒症状严重者,可在伤口周围涂蛇药片。全身过敏尤其是过敏性休克者,应迅速给予肾上腺素皮下注射,并静脉注射甲泼尼龙琥珀酸钠或其他肾上腺皮质激素药物,同时口服抗组胺药物;出现肌肉痉挛可予葡萄糖酸钙缓慢静注;支气管痉挛致严重呼吸困难者吸入支气管扩张剂,并静脉给予氨茶碱。

\subsection{毒蕈中毒}

\subsubsection{中毒机制}

毒蕈种类较多,不同毒蕈所含毒素不同,同一种毒蕈也可能含多种毒素。按各种毒蕈中毒的主要表现,大致分为四型。

(1)毒蕈型:是类似于乙酰胆碱的生物碱,毒性极强,能兴奋胆碱能节后纤维,引起一系列中毒症状。

(2)神经精神型:某些毒蕈中含有毒蝇碱、蟾蜍毒素等毒素,可引起幻觉,谵妄及精神异常等症状。

(3)溶血型:溶血表现,可引起贫血,肝脾肿大等体征。

(4)中毒性肝炎型:所含毒素包括毒伞毒素及鬼笔毒素两大类共11种,鬼笔毒素作用快,主要作用于肝脏,毒伞毒素作用较迟缓,但毒性较鬼笔毒素大20倍,能直接作用于细胞核,有可能抑制RNA聚合酶,并能显著减少肝糖原而导致肝细胞迅速坏死,导致中毒性肝炎。

\subsubsection{临床表现}

毒蕈中毒的临床表现虽各不相同,但起病时多有吐泻症状,易发生水和电解质紊乱,严重者可致休克。毒蕈碱中毒可出现多汗、流涎、流泪、脉搏缓慢、瞳孔缩小等。溶血毒素可引起贫血,肝脾肿大等体征。神经毒素可引起幻觉,谵妄及精神异常等症状。毒伞、白毒伞中毒病情危急,常致肝、脑、心、肾等器官损害。以肝脏的损害最为严重,有黄疸、转氨酶升高、肝肿大、出血倾向等表现。

\subsubsection{治疗原则}

尽快给予洗胃,及时催吐、导泻促进毒物排泄;积极纠正脱水、酸中毒及电解质紊乱。对有肝功能损害者应给予保肝支持治疗,并使用解毒药解毒。

\subsubsection{药物治疗}

阿托品,主要用于含毒蕈碱的毒蕈中毒。可根据病情轻重,采用0.5~1mg皮下注射,每0.5~6h
1次。必要时可加大剂量或改用静脉注射。阿托品尚可用于缓解腹痛、吐泻等胃肠道症状。对因中毒性心肌炎而致房室传导阻滞亦有作用。

毒伞、白毒伞等毒蕈中毒用阿托品治疗常无效,用含巯基的解毒药治疗此类毒蕈中毒,有一定的效果。其作用机理是巯基解毒药与某些毒素如毒伞肽等相结合,使其毒力减弱,而保护了体内含巯基酶的活性,甚至恢复部分已与毒素结合的酶的活力。二巯丁二钠0.5~1g稀释后静脉注射,每6h
1次,首剂加倍,症状缓解后改为每日注射2次,5~7d为一疗程。

糖皮质激素具有抗炎、稳定溶酶体及细胞膜等多种作用,适用于溶血型毒蕈中毒及其他重症中毒患者,特别是有中毒性心肌炎、中毒性脑炎、严重的肝功能损害及有出血倾向的患者。