\chapter{参考文献}

[1]姜远英,许建华.临床药物治疗学[M].3版.北京:人民卫生出版社,2012.

[2]徐建国.疼痛药物治疗学[M].北京:人民卫生出版社,2007.

[3]黄幼霞,周勤.临床药物治疗学概论[M].北京:人民卫生出版社,2012.

[4]肖忠革,周曾同.口腔药理学与药物治疗学[M].上海:上海世界图书出版公司,2009.

[5]肖忠革.口腔临床药理学[M].上海:上海交通大学出版社,2011.

[6]史宗道.口腔临床药物学[M].4版.北京,人民卫生出版社,2013.

[7]尹音,王峰.实用口腔药物学[M].北京:人民卫生出版社,2006.

[8]史久成,史俊南.口腔药物治疗学[M].西安:世界图书出版西安公司,2002.

[9]中华医学会.临床诊疗指南妇产科学分册[M].北京:人民卫生出版社,2007.

[10]乐杰.妇产科学[M].北京:人民卫生出版社,2008.

[11]葛均波,徐永健.内科学[M].8版.北京:人民卫生出版社,2013.

[12]姜远英.临床药物治疗学[M].北京:人民卫生出版社,2003.

[13]杨世杰.药理学[M].北京:人民卫生出版社,2010.

[14]胡晋红.临床药物治疗学[M].北京:高等教育出版社,2012.

[15]陈立,赵志刚.临床药物治疗学[M].北京:清华大学出版社,2012.

[16]姜远英,李俊.临床药物治疗学[M].北京:人民卫生出版社,2003.

[17]王吉耀,廖二元,胡品津.内科学[M].北京:人民卫生出版社,2005.

[18]韩星海.现代风湿病药物治疗学[M].北京:人民军医出版社,2005.

[19]黄幼霞,周勤.临床药物治疗学概论[M].北京:人民卫生出版社,2012.

[20]孙明.内科治疗学[M].2版.北京:人民卫生出版社,2006.

[21]陈文彬,罗德诚.临床药物治疗学[M].3版.北京:人民卫生出版社,2004.

[22]黄从新,李庚山,高尚志.临床治疗学实践[M].北京:人民卫生出版社,2001.

[23]国家药典委员会.中华人民共和国药典-临床用药须知(化学药和生物制品卷)[M].北京:人民卫生出版社,2005.

[24]中华医学会.临床诊疗指南肾脏病学分册[M].北京:人民卫生出版社,2013.

[25]吴飞华,袁克俭,杨婉花.社区医院用药指导[M].浙江:浙江科学技术出版社,2010.

[26]陈灏珠,林果为,王吉耀.实用内科学[M].14版.北京:人民卫生出版社,2013.

[27]殷立新,王绵,张力辉.内分泌科常见病用药处方分析[M].北京:人民卫生出版社,2013.

[28]徐叔云.临床药理学[M].北京:人民卫生出版社,2003.

[29]陆再英,钟南山.内科学[M].7版.北京:人民卫生出版社,2007.

[30]孙淑娟.抗菌药物治疗学[M].北京:人民卫生出版社,2008.

[31]杨世杰.药理学和药物治疗学[M].北京:人民卫生出版社,2005.

[32]曹红.临床药物治疗学[M].北京:人民卫生出版社,2010.

[33]李俊.临床药物治疗学[M].北京:人民卫生出版社,2007.

[34]陆晓彤,张健.儿科临床用药手册[M].上海:上海科学技术出版社,2006.

[35]李文航,胡仪吉.儿科临床药理学[M].北京:人民卫生出版社,1998.

[36]王祖承,方贻儒.精神病学[M].上海:上海科技教育出版社,2011.

[37]江开达,周东丰.精神病学[M].北京:人民卫生出版社,2005.

[38]郝伟,于欣.精神病学[M].北京:人民卫生出版社,2013.

[39]程德云,陈文彬.临床药物治疗学[M].北京:人民卫生出版社,2012.

[40]江开达.精神药理学[M].北京:人民卫生出版社,2011.

[41]李俊.临床药理学[M].北京:人民卫生出版社,2013.

[42]赵靖平,翟金国.精神科-常见病用药[M].北京:人民卫生出版社,2008.

[43]Hankim A,Clunie
G.牛津临床风湿病手册[M].吴东海,译.北京:人民卫生出版社,2006.

[44]Goldman,Bennett.西氏内科学[M].王贤才,译.21版.西安:世界图书出版公司,2004.

[45]Jameson
JL.哈里森内分泌学[M].胡仁明,李益明,童伟,译.北京:人民卫生出版社,2010.

[46]Sanford
JP.桑福德抗微生物治疗指南.42版.北京:中国协和医科大学出版社,2012.

[47]Ekberg H,Tedesco-Silva H,Demirbas A,{et al} .Reduced exposure
to calcineurin inhibitors in renal transplantation[J].N Engl J
Med,2007,357(25):2562-2575.

[48]Ekberg H,Grinyo J,Nashan B,{et al} .Cyclosporine sparing with
mycophenolate mofetil,daclizumab and corticosteroids in renal allograft
recipients:The CAESAR Study[J].Am J
Transplant,2007,7(3):560-570.

[49]Salvadori M,Holzer H,de Mattos A,{et al} .Enteric-coated
mycophenolate sodium is therapeutically equivalent to mycophenolate
mofetil in de novo renal transplant patients[J].Am J
Transplant,2004,4(2):231-236.

[50]Ferrer F,Machado S,Alves R,{et al} .Induction with basiliximab
in renal transplantation[J].Transplant Proc,2010,42(2):467-470.

[51]McKeage K,McCormack PL.Basiliximab:A review of its use as
induction therapy in renal
transplantation[J].Biodrugs,2010,24(1):55-76.

[52]Saghafi H,Rahbar K,Nobakht Haghighi A,{et al} .Efficacy of
anti-interleukin-2 receptor antibody(daclizumab)in reducing the
incidence of acute rejection after renal
transplantation[J].2012,4(2):475-477.

[53]Kidney Disease:Improving Global Outcomes(KDIGO)Transplant Work
Group.KDIGO clinical practice guideline for the care of kidney
transplant recipients[J].Am J Transplant,2009,9(Suppl3):S1-157.

[54]Webster AC.The CARI guidelines.Calcineurin inhibitors in renal
transplantation:the addition of anti-CD25 antibody induction to
standard immunosuppressive therapy for kidney transplant
recipients[J].Nephrology(Carlton),2007,12(Supp11):S75-84.

[55]Hammond EB,Taber DJ,Weimert NA,{et al} .Efficacy of induction
therapy on acute rejection and graft outcomes in African American kidney
transplantation[J].Clin Transplant,2010,24(1):40-47.

[56]Knight SR,Morris PJ.Steroid avoidance or withdrawal after renal
transplantation increases the risk of acute rejection but decreases
cardiovascular risk.A
meta-analysis[J].Transplantation,2010,89(1):1-14.

[57]O'Dell JR.Therapeutic Strategies for Rheumatoid Arthritis
[J].N Engl Med,2004,350(25):2591-2602.

[58]中华医学会呼吸病学分会哮喘学组.支气管哮喘控制的中国专家共识[J].中华内科杂志,2013,52(5):440-443.

[59]中华医学会呼吸病学分会.社区获得性肺炎诊断和治疗指南[J].中华结核和呼吸杂志,2006,29(10):651-655.

[60]中华医学会呼吸病学分会慢性阻塞性肺疾病病学组.慢性阻塞性肺疾病诊治指南(2013年修订版)[J].中华结核和呼吸杂志,2013,36(4):1-10.

[61]唐孝达.皮质激素在防治尸肾移植排斥反应三联药物治疗中的实践[J].肾脏病与透析肾移植杂志,2002,11(2):142-143.

[62]闵志廉.肾移植后排斥治疗中如何用好激素[J].肾脏病与透析肾移植杂志,2002,11(2):144-145.

[63]郭建军,赵华,王晶晶.CYP介导的抑制性药物相互作用的体外定量预测[J].中国临床药理学与治疗学,2009,14(8):841-848.

[64]中华医学会风湿病学会.类风湿关节炎诊治指南(草案)[J].中华风湿病学杂志,2003,7(4):250-254.

[65]中华医学会风湿病学会.系统性红斑狼疮诊治指南(草案)[J].中华风湿病学杂志,2003,7(8):508-513.

[66]中华医学会风湿病学会.系统性硬化病诊治指南(草案)[J].中华风湿病学杂志,2004,8(6):377-379.

[67]胡荣.妊娠高血压的治疗-解读2011ESC妊娠期心血管疾病治疗指南(高血压部分)[J].中国全科医学,2011,14(9):11-13.

[68]中华医学会妇产科学分会妊娠期高血压疾病学组.妊娠期高血压疾病诊治指南(2012版)[J].中华妇产科杂志,2012,47(6):476-480.

[69]黄志琨,傅鹰.循证性诊疗指南综述-子宫肌瘤的药物治疗[J].药物流行病学杂志,2004,13(2):100-103.

[70]中华消化杂志编委会.消化性溃疡病诊断与治疗规范(2013年,深圳)[J].中华消化杂志,2014,34(2):73-76.

[71]抗血小板药物消化道损伤的预防和治疗中国专家共识组.抗血小板药物消化道损伤的预防和治疗中国专家共识(2012更新版)[J].中华内科杂志,2013,52(3):264-270.

[72]中华医学会消化病学分会幽门螺杆菌学组,全国幽门螺杆菌研究协作组.第四次全国幽门螺杆菌感染处理共识报告[J].胃肠病学,2012,17(10):618-625.

[73]刘文忠.2013年美国胃肠病学院胃食管反流病诊断和处理指南解读[J].胃肠病学,2013,18(4):193-199.

[74]《中华内科杂志》编委会,《中华消化杂志》编委会,《中华消化内镜杂志》编委会.急性非静脉曲张性上消化道出血诊治指南[J].中国实用乡村医生杂志,2012,19(24):6-9.

[75]中国医师学会急诊医学分会.急性上消化道出血急诊诊治流程专家共识(修订稿)[J].中国急救医学,2011,31(1):1-8.

[76]袁耀宗,杨云生,吴开春.2012年《溃疡出血患者处理指南》的解读与质子泵抑制剂的临床应用[J].中华消化杂志,2012,32(7):487-489.

[77]中华医学会消化病学分会炎症性肠病学组.炎症性肠病诊断与治疗的共识意见(2012年,广州)[J].中华消化杂志,2012,32(12):796-813.

[78]中华医学会消化病学分会炎症性肠病学组.英夫利西治疗克罗恩病的推荐方案(2011年)[J].中华消化杂志,2011,31(12):822-824.

\protect\hypertarget{text00032.html}{}{}

\chapter{英汉缩略语}

\begin{longtable}[]{@{}lll@{}}
\toprule
\endhead
5-ASA & 5-aminosalicylic acid & 5-氨基水杨酸\tabularnewline
5-ASA & 5-aminosalicyli acid & 5-氨基水杨酸\tabularnewline
5-HT & 5-hydroxy tryptamine & 5-羟色胺\tabularnewline
6-MP & 6-mercaptopurine & 6-巯基嘌呤\tabularnewline
AA & Aplastic anemia & 再生障碍性贫血\tabularnewline
ACEI & Angiotensin converting enzyme inhibitors &
血管紧张素转换酶抑制剂\tabularnewline
ACR & Accelerated rejection & 加速性排异反应\tabularnewline
ACS & Acute coronary syndrome & 急性冠状动脉综合征\tabularnewline
AD & Alzheimer's disease & 阿尔茨海默病\tabularnewline
ADR & Adverse drug reaction & 药品不良反应\tabularnewline
ADM & Doxorubicin & 多柔比星\tabularnewline
ALG & Anti lymphocyte globulin & 抗淋巴细胞球蛋白\tabularnewline
AR & Acute rejection & 急性排异反应\tabularnewline
AR & Angiotensin receptor blockers & 血管紧张素受体阻滞剂\tabularnewline
Ara-C & Arabinoside cytosine & 阿糖胞苷\tabularnewline
ATG & Anti thymocyte globulin & 抗胸腺细胞球蛋白\tabularnewline
AUC & Areas under the curves & 曲线下面积\tabularnewline
AZA & Azathioprine & 硫唑嘌呤\tabularnewline
BDP & Beclomethasone dipropionate & 倍氯米松\tabularnewline
BE & Barrett's esophagus & Barrett食管\tabularnewline
cAMP & Cyclic adenosine monophosphate & 环磷腺苷\tabularnewline
CAP & Community acquired pneumonia & 社区获得性肺炎\tabularnewline
CBP & Carboplatin & 卡铂\tabularnewline
CCB & Calcium channel blocker & 钙通道阻滞剂\tabularnewline
CD & Crohn's disease & 克罗恩病\tabularnewline
CKD & Chronic kidney disease & 慢性肾脏疾病\tabularnewline
COX & Cyclooxygenase & 环氧化酶\tabularnewline
CPT & Camptothecin & 喜树碱\tabularnewline
CPT-11 & Irinotecan & 伊立替康\tabularnewline
CR & Chronic rejection & 慢性排异反应\tabularnewline
CRP & C-reactive protein & C反应蛋白\tabularnewline
CsA & Cyclosporin & 环孢素\tabularnewline
CTLA-4 & Cytotoxic T lymphocyte antigen 4 &
T淋巴细胞相关抗原4\tabularnewline
CTX & Cyclophosphamide & 环磷酰胺\tabularnewline
DA & Dopamine & 多巴胺\tabularnewline
DC-ART & Disease controlling anti-rheumatic therapy &
控制疾病的抗风湿治疗药\tabularnewline
DDI & Drug drug interaction & 药物相互作用\tabularnewline
DDP & Cisplatin & 顺铂\tabularnewline
DKA & Diabetic ketoacidosis & 酮症酸中毒\tabularnewline
DMARDs & Disease-modifying anti-rheumatic drugs &
改变病情抗风湿药\tabularnewline
DNR & Daunorubicin & 柔红霉素\tabularnewline
DU & Duodenal ulcer & 十二指肠溃疡\tabularnewline
EBM & Evidence based medicine & 循证医学\tabularnewline
EE & Erosive esophagitis & 糜烂性食管炎\tabularnewline
EIs & Entry inhibitors & 病毒进入抑制剂\tabularnewline
EMB & Ethambutol & 乙胺丁醇\tabularnewline
EPI & Epirubicine & 表柔比星\tabularnewline
ESR & Erythrocyte sedimentation rate & 红细胞沉降率\tabularnewline
FIs & Fusion inhibitors & 融合抑制剂\tabularnewline
FSGS & Focal segmental glomerular sclerosis &
局灶节段性肾小球硬化\tabularnewline
GABA & γ-aminobutyric acid & γ-氨基丁酸\tabularnewline
GERD & Gastroesophageal reflux disease & 胃食管反流病\tabularnewline
GPM & Good pain management & 规范化疼痛处理\tabularnewline
GCS & Glucocorticoid & 糖皮质激素\tabularnewline
GU & Gastric ulcer & 胃溃疡\tabularnewline
HAP & Hospital acquired pneumonia & 医院获得性肺炎\tabularnewline
HAR & Hyperacute rejection & 超急性排异反应\tabularnewline
HCPT & Hydroxycamptothecine & 羟基喜树碱\tabularnewline
HHRT & Homobarringtonie & 高三尖杉酯碱\tabularnewline
HLA & Human leukocyte antigen & 人类白细胞抗原\tabularnewline
IBD & Inflammatory bowel disease & 炎症性肠病\tabularnewline
IFG & Impaired fasting glycemia & 空腹血糖受损\tabularnewline
IFO & Ifosfamide & 异环磷酰胺\tabularnewline
IFX & Infliximab & 英夫利昔单抗\tabularnewline
IGT & Impaired glucose tolerance & 糖耐量异常\tabularnewline
IIs & Integrase inhibitors & 整合酶抑制剂\tabularnewline
IL & Interlukin & 白细胞介素\tabularnewline
IMPDH & Hypoxanthine dehydrogenase monophosphate &
单磷酸次黄嘌呤脱氢酶\tabularnewline
INH & Isoniazid & 异烟肼\tabularnewline
LES & Lower esophageal sphincter & 食管下括约肌\tabularnewline
L-OHP & Oxaliplatin & 奥沙利铂\tabularnewline
MCD & Minimal change disease & 微小病变\tabularnewline
MDR & Multidrug resistance & 多药耐药\tabularnewline
MIC & Minimal inhibitory concentration & 最低抑菌浓度\tabularnewline
MMC & Mitomycin C & 丝裂霉素C\tabularnewline
MMF & Mycophenolate mofetil & 吗替麦考酚酯\tabularnewline
MP & Methylprednisolone & 甲基泼尼松龙\tabularnewline
MRP & Multidrug resistance associated-protein &
多药耐药相关蛋白\tabularnewline
MIT & Mitoxantrone & 米托蒽醌\tabularnewline
MTX & Methotrexate & 甲氨蝶呤\tabularnewline
NERD & Non-erosive reflux disease & 非糜烂性反流病\tabularnewline
NMDA & N-methyl-D-aspartic acid & N-甲基-D-天门冬氨酸\tabularnewline
NNRTIs & Non-nucleoside reverse transcriptase inhibitors &
非核苷类反转录酶抑制剂\tabularnewline
NRTIs & Nucleoside reverse transcriptase inhibitors &
核苷类反转录酶抑制剂\tabularnewline
NSAIDs & Non-sieroidal anti-inflammatory drugs &
非甾体类抗炎药\tabularnewline
NVB & Navelbine & 长春瑞滨\tabularnewline
PC & Pharmaceutical care & 药学服务\tabularnewline
PD & Parkinson disease & 帕金森病\tabularnewline
P-gp & P-glycoprotein & P-糖蛋白\tabularnewline
PIs & Protease inhibitors & 蛋白酶抑制剂\tabularnewline
PK & Pharmacokinetics & 药物代谢动力学\tabularnewline
PPI & Proton pump inhibitions & 质子泵抑制药\tabularnewline
PTU & Propylthiouracil & 丙硫氧嘧啶\tabularnewline
PU & Peptic ulcer & 消化性溃疡\tabularnewline
PZA & Pyrazinamide & 吡嗪酰胺\tabularnewline
RA & Rheumatoid arthritis & 类风湿关节炎\tabularnewline
RAAS & Renin-angiotensin-aldosteronesystem &
肾素-血管紧张素-醛固酮系统\tabularnewline
RF & Rheumatoid factor & 类风湿因子\tabularnewline
RFP & Rifampicin & 利福平\tabularnewline
RFT & Rifapentine & 利福喷丁\tabularnewline
rHuEPO & Recombinant human erythropoietin &
重组人促红细胞生成素\tabularnewline
SAARDs & Slow acting anti-rheumatic drugs &
慢作用抗风湿药\tabularnewline
SLE & Systemic lupus erythematosus & 系统性红斑狼疮\tabularnewline
SM & Streptomycin & 链霉素\tabularnewline
SM-ARDs & Symptom modifying anti-rheumatic drugs &
改善症状抗风湿药\tabularnewline
TAX & Paclitaxel & 紫杉醇\tabularnewline
TLESR & Transient LES relaxation & 一过性LES松弛\tabularnewline
TPT & Topotecan & 托泊替康\tabularnewline
TXT & Docetaxel & 多西他赛\tabularnewline
UC & Ulcerative colitis & 溃疡性结肠炎\tabularnewline
VAS & Visual analogue scale & 视觉模拟评分法\tabularnewline
VCR & Vincristine & 长春新碱\tabularnewline
VLB & Vinblastine & 长春碱\tabularnewline
VRS & Verbal rating scale & 疼痛强度简易描述量表\tabularnewline
VDS & Vindesine & 长春地辛\tabularnewline
VM-26 & Teniposide & 替尼泊苷\tabularnewline
VP-16 & Etoposide & 依托泊苷\tabularnewline
\bottomrule
\end{longtable}

