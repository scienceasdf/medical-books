\chapter{咯 血}

\section{12 咯血}

咯血是指喉及喉以下呼吸道或肺组织任何部位的出血,经口腔咳出者。咯血大多数为呼吸系统及(或)循环系统疾病所致,口腔、鼻腔或上消化道的出血有时易和咯血混淆。鼻腔出血多从前鼻孔流出,并常在鼻中隔前下方发现出血灶,较易诊断。有时鼻后部的出血量较多,特别是在睡眠时不自觉地坠入气道而于清晨咳出,较易误诊为咯血;如见血液从后鼻孔沿软腭或咽后壁下流,用鼻咽镜检查可以确诊。此外,还须检查有无鼻咽癌、喉癌、口腔溃疡、咽喉炎及牙龈出血的可能性。

呕血为上消化道出血,经口腔呕出,出血灶多位于食管、胃及十二指肠。咯血和呕血可根据病史、体征及其他检查方法进行鉴别,参见表\ref{tab4-1}。

\begin{table}[htbp]
\centering
\caption{咯血与呕血的鉴别}
\label{tab4-1}
\includegraphics[width=5.89583in,height=2.59375in]{./images/Image00038.jpg}
\end{table}

区别咯血和呕血一般不难,但如患者出血急骤,量多或病史诉说不清时,有时鉴别并不容易;因此须详细询问有关病史,作细致的体格检查,及时作出诊断。

如已明确为咯血,须进一步探索其原因。引起咯血的原因很多(表\ref{tab4-2}),其中最常见的疾病是肺结核、支气管扩张、肺脓肿、支气管肺癌。此外支气管结石、肺寄生虫病、心血管疾病(特别是二尖瓣狭窄)、结缔组织病、钩端螺旋体病等也可引起咯血。

\begin{table}[htbp]
\centering
\caption{引起咯血的常见疾病分类}
\label{tab4-2}
\includegraphics[width=5.89583in,height=4.80208in]{./images/Image00039.jpg}
\end{table}

如咯血量较大,应即采取急救措施,以尽早确定出血的部位。当X线检查的条件未具备时,可应用听诊法以确定。如咯血开始时一侧肺部呼吸音减弱或(及)出现湿啰音,而对侧肺野呼吸音良好,常提示出血即在该侧。气管和支气管疾病所致出血,全身症状一般不严重,胸部X线检查基本正常,或仅有肺纹理增粗;肺部病变所致出血,有比较明显的全身症状,胸部X线检查常发现病变阴影;必须指出,咯血可为全身疾病表现的一部分,临床医生必须对咯血患者做全身检查,以作出正确的诊断。

对于咯血患者,全面分析病史资料常可对咯血原因做出初步估计,同时还需要进一步做下列有关检查:

\subsection{1.病史}

须询问出血为初次或多次。如为多次,与以往有无不同。发生于幼年可见于先天性心脏病;儿童少年慢性咳嗽伴小量咯血和低色素性贫血,须注意特发性肺含铁血黄素沉着症;青壮年咯血多注意肺结核、支气管扩张等疾病;40岁以上有长期大量吸烟史(纸烟20支/
日×20年以上)者,要高度警惕支气管肺癌的可能性;年轻女性反复咯血也要考虑支气管结核和支气管腺瘤。在既往史上需注意幼年是否曾患麻疹、百日咳。在个人史中须注意结核病接触史、多年吸烟史、职业性粉尘接触史、生食螃蟹与蝲蛄史、月经史等。

细致观察咯血的量、颜色,有无带痰。肺结核、支气管扩张、肺脓肿、支气管结核、出血性疾病咯血颜色鲜红;肺炎球菌大叶性肺炎、肺卫氏并殖吸虫病和肺泡出血可见铁锈色血痰;烂桃样血痰为肺卫氏并殖吸虫病最典型的特征;肺阿米巴病可见脓血样痰呈棕褐色,带腥臭味;砖红色胶冻样血痰主要见于克雷伯杆菌肺炎;二尖瓣狭窄肺淤血咯血一般为暗红色;左心衰竭肺水肿时咳浆液性粉红色泡沫样血痰;并发肺梗塞时常咳黏稠暗红色血痰。大量咯血常由于空洞型肺结核、支气管扩张、慢性肺脓肿、动脉瘤破裂等所致;国内文献报告,无黄疸型钩端螺旋体病也有引起致命的大咯血。而痰中带血持续数周或数月应警惕支气管肺癌;慢性支气管炎咳嗽剧烈时可偶有血性痰。

详细询问伴随症状如发热、胸痛、咳嗽、痰量等。咯血伴有急性发热、胸痛常为肺部炎症或急性传染病,如肺出血性钩端螺旋体病、流行性出血热;咯血、发热同时伴咳嗽、咳大量脓痰多见于肺脓肿;长期低热、盗汗、消瘦的咯血应考虑肺结核;反复咳嗽、咳脓痰不伴有发热多见于支气管扩张。

\subsection{2.体格检查}

活动期肺结核和肺癌患者常有明显的体重减轻,而支气管扩张患者虽反复咯血而全身情况往往较好。有些慢性心、肺疾病可伴有杵状指(趾)。锁骨上淋巴结肿大在中老年患者要注意肺内肿瘤的转移。肺部闻及局限性哮鸣音提示支气管有狭窄、阻塞现象,常由肿瘤引起。肺部湿性啰音可能是肺部炎症的体征,也应考虑是否为血液存积在呼吸道所致。对咯血患者还应注意有无全身的出血表现。

\subsection{3.实验室检查}

痰检查有助于发现结核杆菌、真菌、癌细胞、肺吸虫卵等。出血时间、凝血时间、凝血酶原时间、血小板计数等检查,有助于出血性疾病的诊断。外周血红细胞计数与血红蛋白测定可推断出血的程度。外周血中嗜酸性粒细胞增多提示寄生虫病的可能性。

\subsection{4.X线检查}

对于咯血患者,除个别紧急情况不宜搬动外,均应做胸部X线检查。肺实质病变一般都能在X线胸片上显示阴影,从而及时作出诊断。如疑有空洞、肿块,或见肺门、纵隔淋巴结肿大,可加做胸部X线体层摄片或CT检查,CT还有助于发现细小的出血病灶。对疑有支气管扩张者,可做高分辨CT检查等协助诊断。对疑为支气管动脉性出血所致大咯血,必要时可行CT支气管动脉造影(CTA)或数字减影血管成像(DSA)检查,明确出血部位,后者尚可同时进行栓塞介入治疗。

\subsection{5.纤维支气管镜检查}

原因未明的咯血,尤其伴有支气管阻塞者,应考虑纤维支气管镜检查,可发现气管和支气管黏膜的非特异性溃疡、黏膜下层静脉曲张、结核病灶、肿瘤等病变,并可在直视下钳取标本作病理组织检查,吸取分泌物或灌洗液送细菌学和细胞学检查。

\subsection{6.其他检查}

先天性心脏病的诊断往往借助右心导管检查。放射性核素67镓对恶性肿瘤组织较健康组织有更大的亲和力,因而枸橼酸67镓肺部扫描可能有助于肺癌与其他肺部肿物的鉴别诊断。PET/CT对肺部肿瘤引起的咯血的诊断也有帮助。

咯血量的多少视病因和病变性质而不同,但与病变的严重程度并不完全一致,少则痰中带血,多则大口涌出,一次可达数百或上千毫升。临床上常根据患者咯血量的多少,将其分为少量咯血、中量咯血和大量咯血。但界定这三种情况的咯血量多少的标准尚无明确的规定,但一般认为24小时内咯血量少于100ml者为小量咯血;100~500ml/d者为中量咯血;>500ml或一次咯血量>100ml者为大量咯血。

临床上无异常肺部X线征象的咯血病例并不少见,诊断较为困难,其主要原因可能为:①气管或大支气管的非特异性溃疡,一般表现为小量咯血或血痰,支气管镜检查可以发现。②气管或支气管的静脉曲张,多见于右上叶支气管开口处或隆突部分,常引起大咯血,无痰,可经支气管镜检查而发现。③肺动脉瘤、支气管小动脉粥样硬化破裂,肺动静脉瘘破裂。④小块肺栓塞,常不易发现,一般有心脏病、下肢深静脉血栓形成、外伤史、长时间卧床或处于产褥期病史。⑤钩虫蚴、蛔虫蚴、血吸虫毛蚴、比翼线虫在肺内游移引起咯血。⑥早期支气管肿瘤,轻度支气管扩张、支气管结核,肺结核早期等。纤维支气管镜的广泛应用,结合胸部X线检查大大提高咯血病因的确诊率,国内一组917例经胸部X线与纤维支气管镜检查而确定的咯血病因如表\ref{tab4-3}所示:\footnote{*既有临床表现又有X线表现}

\begin{table}[htbp]
\centering
\caption{917例咯血的病因分析(X线诊断与纤支镜诊断比较)}
\label{tab4-3}
\includegraphics[width=5.91667in,height=3.58333in]{./images/Image00040.jpg}
\end{table}

由表\ref{tab4-3}所示,有部分咯血患者虽经X线和纤维支气管镜检查,仍未能发现阳性结果,且患者亦无引起咯血的全身性疾病,此类咯血可称为特发性咯血。但仍有可能在以后随诊中,在这类“特发性咯血”患者的一部分中,检出呼吸系统疾病。

国内曾有报告一组390例胸片无明显异常的咯血患者,作纤维支气管镜检查,结果发现肺癌16例(4.1\%)、支气管结核2例、支气管腺瘤1例,支气管囊性静脉曲张出血l例。作者认为咯血患者40岁以上,吸烟指数(吸烟年限×每天吸烟支数)>400,咯血时间长,且为痰血而非纯咯血者,尤须警惕肺癌的可能性。

X线胸片是咯血患者的常规检查,但可能无阳性发现。X线胸片正常的咯血患者应进一步作病因学诊断。有作者推荐应先作CT检查,以期发现潜在的肺部病灶,并有助于以后做纤维支气管镜检查时,有目标地进行刷检、活检取材,提高咯血的病因学诊断率。

\protect\hypertarget{text00057.html}{}{}

\subsection{12.1 气管和支气管疾病}

\subsubsection{一、急、慢性支气管炎}

急、慢性支气管炎患者有时也可咯血,一般为小量或痰中带血,不需治疗,可在数天内自行停止,但易于再发。如出血量大,需注意其他原因。本病的咯血与支气管炎症加剧有一定的关系,故咯血前常有病情加重的表现。慢性支气管炎患者发生持续的小量咯血时,须小心寻找其他原因,特别是支气管肺癌。

\subsubsection{二、非结核性支气管扩张}

非结核性支气管扩张可分为原发性与继发性。继发性者是由于支气管内或支气管外阻塞,引起支气管腔与支气管壁的感染,从而损害支气管壁的各层组织所引起。原发性支气管扩张则无明显的引起支气管阻塞的因素,但多数有肺炎病史,特别是麻疹、百日咳、流感等所继发的支气管肺炎史。

咯血是非结核性支气管扩张的常见症状,文献报告约90\%患者有不同程度的咯血,并作为提示诊断的线索。咯血可从童年即开始,常伴有杵状指(趾)。

此病的咯血有两种不同表现:

\paragraph{1.小量咯血}

在经常有慢性咳嗽、脓痰较多情况下,同时有小量咯血;有时在咯血前先有咳嗽较剧烈的一段感染加重阶段。因感染导致支气管内肉芽组织充血及损伤小血管而出现咯血。

\paragraph{2.大咯血}

由于支气管有炎症病变,血管弹性纤维被破坏,管壁厚薄不匀或形成假血管瘤,加上炎症影响,易破裂引起大咯血。咯血量每次达300~500ml以上,色鲜红,常骤然止血(因此类出血常来自支气管动脉系统,压力高,而动脉血管壁弹性好,收缩力强,故可较快止血)。

患者病程虽长,但全身情况尚好。咳嗽和咳痰也为常有的症状,咳嗽可轻微,也可相当剧烈;咳嗽和咳痰常与体位改变有关,如在晨起或卧床后咳嗽可加剧,咳痰增多。痰量可为大量,每天达数百毫升(湿性型)。痰液静置后可分为三层:上层为泡沫状黏液,中层为较清的浆液,下层为脓液及细胞碎屑沉渣。有些患者痰量甚少(干性型),如合并感染,痰量随之增多,并有发热、咯血等。

支气管扩张的好发部位是下肺,以左下叶较右下叶为多见,最多累及下叶基底段,病灶可延伸至肺边缘。病变部位出现呼吸音减弱和湿性啰音,位置相当固定,体征所在的范围常能提示病变范围的大小。

胸部X线平片检查不易确诊本病。国内一组84例非结核性支气管扩张中,只1/3病例在胸部X线平片上有少许的征象,大部分甚至没有任何改变。胸部X线平片检查对排除慢性肺脓肿及慢性纤维空洞型肺结核颇有帮助。如患者有支气管扩张的临床表现,X线胸片又显示一侧或双侧下肺纹理增粗、紊乱以及蜂窝状小透亮区,或见有液平面则支气管扩张的可能性最大,胸部CT检查可确定诊断,并对明确病变部位及决定治疗方案有重要意义。

全内脏转位、支气管扩张、鼻窦病变三联症,又称Kartagener综合征,国内有少数病例报告。此综合征有咳嗽、咳痰、咯血等症状。咯血可从童年开始,反复发作,量不多。

\subsubsection{三、结核性支气管扩张}

结核性支气管扩张的症状因肺内结核病灶的情况而定,如肺结核病灶不严重,则可无明显症状。有时或可闻及少许干、湿性啰音。X线胸片上显示病灶似已硬结,而患者仍有或多或少的咯血,应考虑结核性支气管扩张的可能性。国内一组64例患者中,发病大多在30岁以上,90\%有咯血(痰中带血或大量咯血)。病灶部位大都在两肺上叶,尤以右上叶的后段、左上叶的尖后段多见。

结核性支气管扩张与非结核性支气管扩张的鉴别见表\ref{tab4-4}。

\subsubsection{四、支气管结核}

支气管结核一般为继发性,原发性者罕见。患者大多有咯血,其他常见症状为阵发性剧烈咳嗽、喘鸣、阵发性呼吸困难等。有时轻度动作即可引起呼吸困难与发绀。如发生支气管阻塞,则引起突然的发热、痰量减少,而阻塞解除后痰量突然增加,体温也下降。

\begin{table}[htbp]
\centering
\caption{结核性与非结核性支气管扩张的鉴别要点}
\label{tab4-4}
\includegraphics[width=5.94792in,height=1.65625in]{./images/Image00041.jpg}
\end{table}

支气管结核是发生于气管、支气管黏膜或黏膜下层的结核病变。据国内报告,肺结核合并支气管结核者占23.6\%~57.1\%。患者以青壮年为多,文献报告女性罹患多于男性,常发生于慢性纤维空洞型肺结核、慢性血行播散型肺结核、支气管淋巴结结核、浸润型肺结核及干酪样肺炎等基础之上。这些患者有下列情况提示有支气管结核的可能:①反复小量咯血或血痰而X线胸片未见明显病变者;②药物难以控制的刺激性咳嗽;③有喘鸣音;④有不同程度的呼吸困难而不能用肺实质病变解释者;⑤肺无明显病变而痰结核菌屡为阳性;⑥肺内有新播散病灶而不能用其他原因解释;⑦肺结核并发肺不张;⑧某些肺野空洞:在萎陷疗法后产生的张力性空洞;空洞时大时小;出现圆形、薄壁空洞;肺门附近的空洞等。

支气管结核的确诊须依靠纤维支气管镜检查。如临床症状典型,虽纤维支气管镜检查阴性,也不能除外此病的存在。

近年国内一组单纯气管、支气管结核病28例报道,误诊颇多,原因为:①胸片无异常发现;②胸片虽出现局限性肺气肿、肺纹理密集、肺纹理粗乱、叶间胸膜影移位等异常表现,但又非特异性而未加注意。作者建议对干咳、胸闷、喘息、咳黏液痰或咯血患者,经抗感染及对症治疗2周未见好转时.应及早作纤维支气管镜检查,镜下刷检涂片染色找抗酸杆菌,或钳取组织做病理检查。镜下所见仍疑似结核而实验室检查阴性时,2周后应再做纤维支气管镜检查。

目前将刷检标本或支气管肺泡灌洗液进行PCR检测结核分枝杆菌,可大大提高病原学诊断率。

结核感染T细胞斑点(T-SPOT.TB)试验是近年来的一项新诊断技术,通过检测外周血分泌γ-干扰素的T淋巴细胞数量来诊断结核感染,具有较高的特异性和敏感性,且不受卡介苗接种和环境分枝杆菌感染的影响,在肺结核的筛查和诊断中有较好的应用价值。

\subsubsection{五、支气管结石}

本病的特点为反复咯血,而肺部除有钙盐沉积之外,无其他原因可解释。患者或曾有咳出结石史。咯血通常为小量,但有些患者可有大咯血。X线检查发现有支气管结石阴影,以右中叶根部较为多见,结石远端可发现有阻塞性肺不张或肺部感染,CT检查可见支气管阻塞远端有钙化影。纤维支气管镜检查可帮助诊断。支气管结石常由肺结核病灶钙化引起。X线胸片上如炎症病变相应部位有钙化结节,在炎症消退后而咯血不断者,则支气管结石的可能性甚大。

据国内一组20例的报告中,以咯血为主要症状者占95\%,其中威胁生命的大咯血占40\%。误诊率60\%。并发症(占85\%)表现为肺不张、支气管扩张、阻塞性肺炎等。但手术治疗效果好。X线断层摄片、胸部CT和纤维支气管镜检查等综合检查对诊断有较大帮助。

\subsubsection{六、原发性支气管肺癌(肺癌)}

本病大多见于40岁以上男性,文献报告有咯血者占50\%~70\%,国内一组141例报告中,第一个月内出现咯血者有38.3\%。中央型肺癌较周围型肺癌易引起咯血。癌组织内小血管较多,患者又常有刺激性咳嗽,易引起癌组织损伤而致出血。其特点是痰中带血或小量咯血多见,而大量咯血少见,但晚期可有致命性大咯血。咳嗽是较常见的早期症状,无痰或有少量的白色黏液痰,可伴有胸痛。间断的或持续的小量咯血,对提示此病的诊断有重要意义。痰中可混有小颗粒状灰白色坏死组织,其中较易找到癌细胞。X线胸片、纤维支气管镜、胸部CT及活组织病理检查有助于诊断。

国内一组确诊肺癌患者1105例中,纤维支气管镜下直接见到肿瘤病灶(直接征象)者638例(57.7\%),只见肿瘤间接征象,即支气管黏膜改变者412例(37.3\%)。肺癌多见于段以上的支气管(中央型肺癌约占3/4),有些病例第一次活检及刷检均未能确诊,需做第二次,偶尔还要第3~4次检查。当发现有间接征象的可疑病例,应尽可能多部位活检多取标本,甚至看到癌体也应多点活检。

\subsubsection{七、支气管类癌}

支气管类癌罹患多为中年人,男女性别无差异,生长慢,具有恶性程度低和较少发生转移的特点。早期症状常为咯血,术后长期生存率较高。国内一组17例报告中,患者40岁以下者占64.7\%,中央型12例,周围型5例,2例有支气管旁淋巴结转移,主要临床表现为咳嗽、咳痰、咯血或痰中带血、发热和反复发作肺部炎症。临床上本病易误诊为肺癌、结核球或良性支气管肿瘤。X线检查与纤维支气管镜下活检对诊断帮助较大。

\subsubsection{八、良性支气管瘤}

良性支气管瘤少见,发病多在30~40岁之间。全身情况良好的中年患者如有反复的小量咯血或痰中带血,或类似哮喘发作,或屡次发作的呼吸道阻塞及感染症状,应考虑此病的可能。由于肿瘤生长缓慢,临床症状可延续多年。早期可无症状,或仅有气喘、干咳,有时甚至被误诊为支气管哮喘。肿瘤逐渐增大而堵塞支气管时,可发生相应肺叶的肺不张,并在肿瘤的远侧发生感染与支气管扩张。X线体层摄片、胸部CT,可了解较大的支气管内肿瘤的范围及部位,气道阻塞情况及继发性支气管扩张,对诊断有重要帮助。由于良性瘤多发生于较大的支气管内,纤维支气管镜检查的检出率可达85\%~90\%。

良性支气管瘤有腺瘤、平滑肌瘤、乳头状瘤等,此外较罕见的有纤维瘤、软骨瘤、脂肪瘤等。其中腺瘤比较多见,典型X线征象为肺门附近有圆形或类圆形阴影,密度均匀一致,边缘锐利,体层摄片或胸部CT检查更易于发现;由于多数腺瘤位于主支气管或肺叶支气管内,纤维支气管镜检查易作出诊断。

\protect\hypertarget{text00058.html}{}{}

\subsection{12.2 肺部疾病}

\subsubsection{一、肺结核}

咯血是肺结核患者常见的症状,且常为提示此病诊断的线索。咯血量多寡不一,少可仅为痰中带血,多则一次可达500ml以上,血色鲜红。咯血与结核病变的类型有一定关系,多见于浸润型肺结核、慢性纤维空洞型肺结核、干酪样肺炎,而少见于原发综合征(原发型肺结核)和急性血行播散型肺结核。咯血严重程度并不一定与病灶大小成正比,小的病灶可有较多的咯血,而病灶广泛的也可无咯血。出血量常和血管损害程度有关,血管壁渗透性增高所致的咯血,出血量少,但持续时间较长;小血管的破裂则多引起小量出血,这往往由于慢性活动性肺结核所致;大咯血多为肺动脉分支破损所致,其中以空洞内形成的动脉瘤破裂所致的大咯血多见,此类出血来势甚急,而由于洞壁纤维化不易收缩止血,或血凝块虽能填塞空洞压迫血管暂时止血,但又可因血块溶解而再次出血。

肺结核患者以青壮年占大多数,不少患者以咯血为初发症状而就诊。咯血之后常有发热,是由于病灶播散及病情发展所致。患者常同时出现疲乏、纳差、体重减轻、午后潮热、盗汗、脉快和心悸等全身中毒症状。

肺结核的诊断主要依靠症状、体征、X线胸片和痰结核菌检查。如在青壮年人一侧肺尖部经常听到湿性啰音,又有上述全身性中毒症状,则支持活动性肺结核的诊断。X线胸片是诊断肺结核的重要方法,可以发现早期轻微的结核病变,确定病灶的范围、部位、形态、密度、与周围组织的关系,判断病变的性质、有无活动性、有无空洞、空洞大小和洞壁特点等。因此,定期进行胸部X线检查能及早发现病灶,有助于早期治疗。

痰结核菌检查阳性可确诊为肺结核,且可肯定病灶为活动性。但痰结核菌阴性并不能否定肺结核的存在,对可疑病例需反复多次痰液涂片检查,如有需要,可采用浓集法、培养法、PCR法等,在咯血前后,因常有干酪性坏死物脱落,此时的痰菌阳性率较高。

长期被误诊为肺结核咯血的肺部疾病并非少见,文献报道有支气管扩张、支气管囊肿、肺癌、肺脓肿、肺吸虫病等。

年轻患者反复咯血,痰结核菌检查阴性,全身情况较好,而病灶又处于中、下肺野,用一般抗菌药物治疗能改善炎症表现者,则可认为是非结核性支气管扩张,胸部CT检查有助于确定诊断。支气管囊肿在胸部平片及透视下一般可确定诊断。非结核性支气管扩张或支气管囊肿合并普通细菌感染时,其症状的出现通常较早,可追溯到童年时期,特别是在患麻疹、百日咳之后常有咯血及呼吸道炎症症状,其与肺结核病的鉴别是前两者在长期的病患过程中,全身一般状况仍较好,无结核病的全身中毒症状,可伴有杵状指(趾),痰结核菌阴性。

肺癌被误诊为肺结核者颇为常见。在下列情况下,应考虑肺癌的可能:①年龄在40岁以上,尤其是长期重度吸烟的男性患者,新近出现反复的咯血或持续的痰中带血,或近肺门处有致密的异常阴影,或出现肺不张合并感染,而多次痰液检查未发现结核菌者,应首先考虑肺癌的可能;但痰中结核菌阳性也不能除外肺结核与肺癌并存。②肺癌组织内部发生坏死破溃,坏死组织排出后形成空洞,其X线征象可酷似结核性空洞。但癌性空洞常呈偏心性,其内侧壁凹凸不平,外缘多呈毛刺状、分叶状,常无病灶周围卫星灶,多次痰结核菌检查阴性,经规律的抗结核治疗无效,病灶逐渐增大,这些均可与肺结核鉴别。③肺癌和肺结核并存,肺癌可发生在陈旧性肺结核瘢痕的基础上,而肺癌又能促使结核病灶恶化。如在陈旧性或活动性结核病灶处出现新的、致密的圆形病灶,且经积极抗结核治疗一个月后,病灶仍逐渐增大,或出现肺不张、肺门阴影增大,癌性空洞等改变,应考虑并存肺癌的可能。

\subsubsection{二、肺 炎}

在急性肺炎时,肺实质处于高度充血状态,小血管通透性增加并可发生破裂而致咯血。由于小血管可发生血栓性脉管炎,致血管腔闭塞,通常不易引起大量咯血。

肺炎链球菌肺炎的患者,痰中混有血液者并不少见,有时血量可达20~30ml,病期第2、3天转为铁锈色痰。在整个病程中均呈血性痰的甚少。

肺炎杆菌性肺炎多为砖红色稠胶样痰;化脓性链球菌肺炎咳粉红色稀痰;葡萄球菌肺炎可为血性痰、脓性痰;绿脓杆菌肺炎咯血少见,典型者咳翠绿色脓痰。军团菌肺炎少量黏液痰中可带血丝,并有发热、咳嗽、肌痛、关节痛、腹泻、蛋白尿、转氨酶升高、直接荧光抗体阳性或间接荧光抗体1∶256。肺炎支原体肺炎约1/4病例有血性痰,但绝无铁锈色痰;流感病毒性肺炎常引起反复的小量咯血。

\subsubsection{三、肺脓肿}

肺脓肿多由于吸入感染或血源性感染所引起,约50\%患者伴有咯血,常伴有大量脓痰或脓血样痰。急性肺脓肿的早期可有大量的咯血而无脓痰,但此时有寒战、高热、胸痛、血白细胞和中性粒细胞增高,提示急性细菌性感染,1周后可出现大量脓性痰。慢性肺脓肿常有大量的脓痰或脓血痰,痰量每天可达300~500ml,带臭味,痰静置分层,多数患者伴有杵状指。慢性肺脓肿常被误诊为肺结核病,前者可根据急性发病史、X线胸片见大片浓密模糊浸润阴影,脓腔内出现圆形透亮区及气液平面,痰培养可有致病菌生长以及抗菌治疗有效,一般鉴别不难。慢性肺脓肿与肺癌的区别,可根据肺脓肿过去的急性发病史、空洞的特点及痰中癌细胞检查等加以鉴别,X线胸片、纤维支气管镜、胸部CT扫描有利于诊断。癌性空洞与肺脓肿空洞、结核性空洞的鉴别参见表\ref{tab4-5}。

\begin{table}[htbp]
\centering
\caption{癌性空洞与肺脓肿空洞、结核性空洞的鉴别}
\label{tab4-5}
\includegraphics[width=5.97917in,height=3.53125in]{./images/Image00042.jpg}
\end{table}

\subsubsection{四、肺部真菌感染}

肺部真菌感染是最常见的深部真菌病,主要由念珠菌、曲霉、毛霉、新型隐球菌等真菌感染所致。老年、幼儿及体弱者易患此病,多为痰中带血或小量咯血。常见症状有发热、乏力、盗汗、纳差、消瘦、咳嗽、胸痛;痰的特点是量少,不同种类的真菌感染时,其痰的性状不一。肺白念珠菌感染,为胶样黏稠痰,带乳块或血丝;肺曲霉病反复咯血或咳出大量泡沫痰(可带酒味);肺新型隐球菌病则咳小量黏液性痰或血丝痰。病原学检查可找到致病菌,胸部X线检查,肺组织病理检查有助于诊断。

\subsubsection{五、肺寄生虫病}

\paragraph{1.肺阿米巴病}

阿米巴性肺脓肿为肝阿米巴病并发症之一,也可来自肠道病灶。多数起病较慢,常有发热、乏力、消瘦、咳嗽、咳痰、右下胸痛并放射至右肩,少数呈急性发病,高热、胸痛、呼吸困难等,可有肝脏肿大体征。典型的痰液呈棕褐色而带腥臭味,有助于此病的诊断。如合并出血或混合感染,可呈血性或黏液脓血痰。痰液、胸腔积液或纤维支气管镜取病变坏死组织中查找到溶组织阿米巴滋养体可确诊。

\paragraph{2.肺吸虫病}

本病有严格的地区性,患者都曾有在疫区进食未煮熟的含有肺吸虫囊蚴的石蟹或蝲蛄史。病程中常反复的小量咯血,痰血混合多呈特殊的棕黄色或铁锈色,烂桃样血痰是肺吸虫病最典型特征。早期症状有畏寒、发热、脐周隐痛、腹泻,并有乏力、盗汗、纳差,2~3周后出现咳嗽、胸痛、咯血等,患者虽有长期的反复咯血,但全身情况尚好,胸部体征多不明显,可有皮下结节。常有血嗜酸性粒细胞增多,痰中发现肺吸虫卵即能确诊,阳性率90\%以上。粪便虫卵检查、肺组织病理检查、免疫学检查、X线检查等有助于诊断。

胸部X线检查有较特别的征象,病灶多位于中、下肺野及内侧带,因病变的不同时期而有下列的表现:①早期呈边缘模糊的弥漫阴影,大小约为1~2cm;②中间期为边缘清楚、多房或单房的、实质或囊状的大小不等的阴影,多房性囊状阴影是本病的X线特征;③晚期为纤维增殖性变及硬结钙化阴影。此外,可有肺门增大、肺纹理增粗紊乱、胸腔积液或胸膜增厚等征象。对一些疑难的、不典型的病例,流行病学调查和免疫学检查,在诊断上有重要意义。

如患者有上述的流行病学史和胸痛、咳铁锈色痰等症状,血嗜酸性粒细胞增多,痰中虽未发现肺吸虫卵,而肺吸虫抗原皮内试验阳性,并已除外血吸虫病、华支睾吸虫等感染时,则大致可作出肺吸虫病的临床诊断,并应进行特效药物(如吡喹酮)的诊断性治疗;如疗效显著,可进一步确立诊断。肺吸虫病主要须与肺结核相鉴别(表\ref{tab4-6})。

\begin{table}[htbp]
\centering
\caption{肺吸虫病与肺结核鉴别}
\label{tab4-6}
\includegraphics[width=5.94792in,height=4.91667in]{./images/Image00043.jpg}
\end{table}

在四川及福建发现的肺吸虫病,临床表现较特别,其症状轻,咯血量较小,痰中虫卵检出率较低,有游走性皮下结节者甚多(多分布于胸壁及上腹壁),血中嗜酸性粒细胞显著增多。

国内报道肺吸虫病误诊率较高。主要原因是由于肺吸虫病临床表现及X线胸片大多无特异性,且典型的游走性皮下结节,肺空泡、结节和隧道线条等X线表现又少见。一组报道34例肺吸虫病在入院前全部误诊为肺结核。

由于肺吸虫病免疫学诊断的敏感性和特异性高,且简便易行,在流行区内患者有生食石蟹或蝲蛄史,而反复出现咳嗽、咯血或痰中带血、发热等症状,应即作肺吸虫抗原皮内试验,上述一组误诊为肺结核的34例患者,肺吸虫抗原皮内试验全部为1∶2000以上阳性。应用混合单抗双抗体夹心ELISA法诊断疑难肺吸虫病,效果更佳。

\paragraph{3.肺包虫病}

肺包虫病是棘球绦虫的幼虫(棘球蚴)寄生于人体肺内所引起,主要流行于畜牧区,以青壮年农民和牧民为主,早期可无症状。当包囊肿大破裂时可出现咯血或痰中带血,并可咳出类似粉皮样的角皮膜;合并感染时则出现咳嗽、咳痰、胸部不适,或胸痛及劳力后气促等症状。可有肝脏或其他部位囊肿征象。囊肿破裂,囊液亦可阻塞气管而引起窒息。X线胸片或CT扫描有助于诊断,可显示包虫圆形或卵圆形,略呈分叶状阴影,边缘清晰,密度均匀,壁可钙化,阴影随呼吸而变形;包虫囊壁破裂,空气进入,则顶部呈现半月形透亮带。包虫抗原皮内试验及补体结合试验对本病有重要的诊断意义,阳性率可达90\%以上。此外,痰检查、B超检查、放射性核素扫描等对诊断也有帮助。

\subsubsection{六、恶性肿瘤的肺转移}

恶性肿瘤转移至肺部时,可引起咯血。较常发生肺部转移的恶性肿瘤有鼻咽癌、乳腺癌、食管癌、胃癌、肝癌、结肠直肠癌、前列腺癌、睾丸畸胎瘤、精原细胞瘤、绒毛膜上皮细胞癌、恶性葡萄胎及类癌等。以绒毛膜上皮细胞癌、睾丸畸胎瘤和恶性葡萄胎的肺转移的咯血发生率最高。对成年女性原因未明的咯血,患者阴道曾排出水泡样胎块,或兼有流产后持续的不规则阴道出血,应考虑恶性葡萄胎或绒毛膜上皮细胞癌的可能性,尿妊娠诊断试验有助于此病的诊断。

转移性肺恶性肿瘤常为多发,但也可为单发,后者较少见。X线胸片显示多发性肺转移肿瘤的形态多为圆形、卵圆形或粟粒状阴影,大小相仿,边缘不整,发展较快。转移性肺恶性肿瘤原发病灶的诊断有时不易,须设法寻找。

\subsubsection{七、肺梅毒}

本病极为少见,国内仅有二例感染报告,均有咯血。病程进展缓慢,往往有咳嗽、咯血、胸痛等症状,虽然X线胸片显示下肺野呈大片状实质模糊阴影,但全身情况良好。诊断须依据梅毒感染史、梅毒血清反应阳性与驱梅治疗的疗效。肺梅毒须与其他肺部疾病相鉴别,特别是肺结核病。

\subsubsection{八、肺囊肿}

肺囊肿可区分为先天性与后天性两类,以前者较为多见,后者是由肺部感染性疾病或寄生虫所引起。多发性先天性肺囊肿常伴有支气管扩张,多在儿童期出现症状,其临床表现与支气管扩张相似,患者往往因突然小量咯血或痰中带血而就诊,由于病变多位于中、上肺,引流较好,较少出现发热。X线胸片检查呈圆形透亮影,其壁菲薄而整齐;多发者大小不一,可分布于任何肺野,但以中、上肺野较多见。

多发性先天性肺囊肿需与支气管扩张鉴别。肺囊肿继发感染时可出现大片状模糊阴影,类似浸润性肺结核,但经抗生素治疗后感染较快消退,而有别于肺结核。肺囊肿合并感染时,其临床症状和X线胸片的改变与肺脓肿相似,需加以鉴别。

国内曾报告一组82例的成年患者中,52例(63.4\%)有咯血,多数在200ml以下,认为如有下列情况应考虑本病:①肺部阴影长期存在;②阴影在同一部位反复出现;③无播散灶;④阴影新旧程度一致;⑤肺门及纵隔淋巴结不大。患者虽反复咯血而无结核中毒症状。胸部CT检查有助于对本病的诊断。

近年肺囊肿有一组38例报告,发病常在青少年期,5例初发年龄在10岁以下。全部患者(38例)在院外均被误诊为肺结核。初发症状以咯血或痰中带血多见(21/38),咳嗽、咳痰、发热次之。6例无症状患者经体检而发现本病。21例有长期反复出现咯血或痰中带血史,最长者达20年。

肺囊肿的胸部X线表现常缺乏特征性。作者认为以下几点有利于肺囊肿的早期诊断:①发病年龄在30岁以下,特别是男性,有反复咯血或痰中带血、咳嗽、发热史。②动态观察X线胸片阴影的形态变化不大,无肿瘤的淋巴结、淋巴道远处转移的改变。③结核菌素试验阴性或经积极的抗结核治疗无效。④胸部X线显示其病变虽无固定部位但与支气管走向有关;虽经反复感染但病变部位固定不变;在非囊肿部位不出现新的病灶。⑤有条件时应作高分辨胸部CT检查加以鉴别。⑥经上述不能确诊时应考虑肺活检或外科手术探查。

\subsubsection{九、尘 肺}

包括硅沉着病和其他尘肺,是由于长期吸入某种粉尘所致的以肺实质弥漫性纤维病变为主的疾病。主要发生于从事粉尘作业的工人,可有慢性、顽固性咳嗽,咳泡沫状痰、咯血或痰中带血,气短和胸痛。早期症状不明显,常为干咳或带黏稠痰,晚期咳嗽加重,痰多,如合并肺结核或支气管扩张,可反复大咯血。晚期病情重,有发绀、杵状指、肺气肿、肺源性心脏病等表现。胸片可见中、下肺野呈网状、条索状或结节状阴影改变,肺门淋巴结肿大。其诊断主要依据为职业病史、临床表现、胸部X线征象及肺功能检查。

硅沉着病合并肺结核比无硅沉着病的发病率高4~5倍,其特点是肺结核的发生率和严重程度与硅沉着病的发展程度成正比,硅沉着病愈严重,其并发肺结核的可能性愈大,肺结核病的病变发展也愈迅速,病情愈严重。石棉肺合并肺结核比较少见,但合并肺癌者却较多。

\protect\hypertarget{text00059.html}{}{}

\subsection{12.3 肺血管及其他循环系统疾病}

\subsubsection{一、肺血栓栓塞症}

肺血栓栓塞症是肺栓塞的最常见类型,占肺栓塞中的绝大多数,多继发于右心或体循环深静脉系统的血栓形成,偶尔也见于肺动脉炎、感染性心内膜炎等病例中,是以各种栓子阻塞肺动脉系统为其发病原因的一组疾病或临床综合征。主要症状为呼吸困难及气促、胸痛、小量咯血(大咯血少见)、咳嗽、心悸、发热等。心瓣膜病(特别是合并心房颤动)患者发生咯血、未能解释的短期发热时,须考虑肺血栓栓塞症的可能。X线胸片可显示区域性肺纹理变细、稀疏或消失,肺野透亮度增加;也可显示肺组织的继发改变,如肺野局部的片状阴影,尖端指向肺门的楔形阴影,肺不张或膨胀不全,有肺不张侧可见横膈抬高。X线胸片对鉴别其他胸部疾病有重要帮助。螺旋CT、放射性核素肺通气/血流灌注扫描、磁共振显像(MRI)及肺动脉造影都是肺血栓栓塞症的重要确诊方法。

\subsubsection{二、肺动脉高压}

肺动脉高压可由许多心、肺和肺血管疾病引起,根据发病的原因是否明确,曾被习惯性分为“原发性”和“继发性”肺动脉高压。2008年世界卫生组织第4届肺动脉高压会议重新修订了肺动脉高压分类,目前按照病因或发病机制、病理与病理生理学特点分为五个大类:①动脉性肺动脉高压;②左心疾病所致肺动脉高压;③肺部疾病及(或)低氧所致肺动脉高压;④慢性血栓栓塞性肺动脉高压;⑤未明多因素机制所致肺动脉高压。继发性肺动脉高压较原发性肺动脉高压常见,早期临床表现以基础疾病如慢性支气管炎、COPD等的临床表现为主,晚期以右心功能不全的表现为主。原发性肺动脉高压是一少见病,被世界卫生组织改称为“特发性肺动脉高压”,是一种不明原因的肺动脉高压。早期通常无症状,仅在剧烈活动时感到不适;随着肺动脉压力的增高,可逐渐出现呼吸困难、胸痛、头晕或晕厥、咯血。咯血量通常较少,有时也可因大咯血而死亡。其他症状还包括疲乏、无力,雷诺现象,声嘶(Ortner综合征)等。胸部X线检查、超声心动图和多普勒超声检查、放射性核素肺通气/灌注扫描、右心导管术、肺活检都对诊断有重要的作用。

\subsubsection{三、肺动静脉瘘}

肺动静脉瘘是先天性肺血管的血管瘤样畸形,也可为获得性,临床上少见。可有咳嗽、间歇小量咯血、发绀、杵状指(趾)及红细胞增多症。体格检查可发现在相应胸壁部位触及震颤,闻及来回性血管杂音。X线胸片和胸部CT在诊断上起重要作用,可见边缘整齐、密度均匀的圆形或卵圆形阴影,多位于中下肺野,且与肺门之间有条索状阴影,病变无钙化,无空洞形成。肺动静脉瘘发病常与遗传性出血性毛细血管扩张症有关,可误诊为肺结核球或支气管肺癌,肺动脉造影可协助明确诊断。

\subsubsection{四、单侧肺动脉发育不全}

本病少见,患者大多有不同程度的咳嗽、咳痰、痰中带血、胸痛、气促等表现,体格检查可发现患侧胸廓扩张稍受限、语颤及呼吸音减弱、多可听到干、湿性啰音,可被误诊为肺气肿、气胸、支气管扩张等。诊断主要依靠胸部X线检查,尤其是胸部CT的肺动脉造影对诊断极有帮助。

\subsubsection{五、肺淤血}

常见于风湿性心脏病二尖瓣狭窄,且多发生在较严重的瓣口狭窄的慢性充血期,也可见于急性左心衰、复张后肺水肿、高原性肺水肿等,多表现为痰带血丝、小量咯血或咳出粉红色泡沫样痰。结合心脏病史、胸腔快速抽液(气)及快速登山等病史,心尖部舒张期隆隆样杂音,超声心动图和多普勒超声检查等可作出诊断。二尖瓣关闭不全较少引起咯血。

\subsubsection{六、高血压}

在恶性或急进型高血压,由于血压持续增高时,可引起肺毛细血管破裂而出现咯血。也可由于并发急性肺水肿而咳粉红色泡沫样痰。

\subsubsection{七、先天性心脏病}

某些有血液分流的先天性心血管病如房间隔缺损、室间隔缺损、艾森曼格综合征等,均可伴有显著的肺动脉高压,由此可引起咯血。

\protect\hypertarget{text00060.html}{}{}

\subsection{12.4 全身性疾病及其他原因}

\subsubsection{一、急性传染病}

1.肺出血型钩端螺旋体病
肺出血型钩端螺旋体病也称钩端螺旋体性出血性肺炎,是钩端螺旋体病的严重类型,如不注意常易误诊。钩端螺旋体病以夏秋季多发,以青壮年为主,从事牧、渔业劳动者发病率高,大多起病急骤,症状主要有畏寒或寒战、高热、头痛、全身肌肉酸痛、衰弱无力、眼结膜充血、腓肠肌疼痛、淋巴结肿大,多在毒血症过程中出现心悸、烦躁、呼吸和心律逐渐增快。初为痰中带血,以后咯血量增多。严重的肺弥漫性出血型者,可引起致命的大咯血,其特点是发生突然、发展迅猛,临终时多数患者出现从口鼻涌出大量血液,立即窒息而死。X线胸片和胸部CT显示双侧肺野斑片状模糊阴影,以中、下肺野尤其显著。需与其他原因的肺炎、肺结核鉴别。病原学和血清学检查有助于明确诊断。

2.流行性出血热
由汉坦病毒引起,经呼吸道、消化道、母婴、虫媒及动物源传播,流行季节以3~5月份或5~7月份以及11月份至次年1月份间为高峰,青壮年多见,主要损害全身小动脉和毛细血管。患者起病急,典型病例具有发热、出血与肾损害三大主要特征以及发热期、低血压休克期、少尿期、多尿期和恢复期五期经过。其主要临床表现为发热、头痛、腰痛、眼眶痛、口渴、呕吐、酒醉貌、球结膜充血水肿,皮肤和黏膜广泛出血、鼻出血、咯血、呕血、便血、血尿,软腭及腋下有出血点,肾区有叩击痛。早期外周血白细胞正常或偏低,有尿蛋白,尿红、白细胞及管型改变;肾功能损害,约有半数出现肝功能损害。特异性血清学试验有助于确诊。

\subsubsection{二、血液病}

某些血液病如血小板减少性紫癜、白血病、再生障碍性贫血、血友病等均可出现咯血,与原发病有关。除咯血外,尚伴有其他部位的出血倾向。血常规、骨髓细胞学检查、血小板功能与凝血因子检查可确诊。

\subsubsection{三、白塞病(Behcet disease)}

本病由病毒感染、遗传因素、免疫功能及体内微量元素异常等因素引起。初发年龄主要是16~40岁的青壮年,以女性多见。基本病变是血管炎,可累及毛细血管和细小动静脉,可因肺部血管受累而反复咯血,多为小量咯血,也有因肺脉管炎而引起多发性肺栓塞。主要的临床表现为反复发作的口腔黏膜、舌尖及其边缘、齿龈、上下唇内侧等处的痛性小溃疡;外生殖器损害与口腔基本相似;眼部损害主要为结膜炎、角膜炎、虹膜炎、视网膜炎,其他表现有皮肤损害、消化道损害、关节损害及神经系统损害等。活动期多有血沉、黏蛋白、唾液酸、α2球蛋白增高,部分患者血浆铜蓝蛋白及冷球蛋白阳性。本病并发肺脉管炎引起咯血时的胸部X线表现,可类似肺炎支原体肺炎、肺部转移癌,或出现大片密度增高的圆形阴影。病理学检查及有关脏器病变的相应检查有助于诊断。针刺反应阳性是一特征性表现,若能同时检查HLA-B5,对本病的诊断更有帮助。

\subsubsection{四、结缔组织病}

该类疾病如系统性红斑狼疮、结节性多动脉炎、重叠综合征等,其发病与遗传、某些药物、物理因素、病毒感染、内分泌因素、免疫异常等有关,多见于20~40岁的女性。可有小量咯血,如伴有肺动脉受侵害,可发生大咯血。若患者有多个器官系统功能的损害,胸部X线检查见肺部有阴影,而抗菌药物治疗效果不佳时,应考虑结缔组织病伴有肺部损害的可能性。诊断结缔组织病所致的咯血时,须认真除外肺结核、支气管扩张、肺肿瘤等疾病。实验室检查和肺组织病理检查有助于诊断。

\subsubsection{五、肺出血-肾炎综合征}

国外文献曾有多例重度肺出血合并肾小球性肾炎报告,并命名为Goodpasture综合征,病因未明,也有将此病归入结缔组织病,临床上少见。此病多见于20~40岁男性,病程数月至数年,预后不良。临床经过可分为两个阶段:①肺部病变阶段,87\%~96\%以上的病例的首发症状表现为间歇的咯血,轻者为血痰,重者出现大咯血,反复出血可致贫血;病变广泛者可有呼吸困难、发绀与胸痛,X线胸片上显示短暂的弥漫性细小或大片状阴影,但X线检查也可呈阴性。②肾脏病变阶段:多数患者在咯血后数周或数月出现肾炎症状,肾脏病变表现为肾小球性肾炎,起病隐袭,当肺部病变显著时,尿检查发现蛋白尿、镜下血尿与管型尿,早期肾功能正常,当肾脏病变为进行性,尿毒症症状迅速出现,并掩盖肺部症状。X线、痰涂片、尿常规、血液检查、免疫学检查、肺功能检查都有助于诊断。肾组织活检免疫荧光检查,发现抗肾小球基底膜抗体则可明确诊断。通常由于尿毒症导致死亡。

除肺、肾两脏器之外,其他脏器很少受累。高血压少见。

\subsubsection{六、肉芽肿性多血管炎(韦格纳肉芽肿)}

是一种原因未明的综合征,被认为是机体对未知抗原的异常超敏反应所致。以30~50岁男性多见,具有上下呼吸道坏死性肉芽肿性血管炎,肾小球肾炎和小血管炎的临床表现。常有小量咯血,严重者可发生大量肺泡性出血,患者可伴有发热、乏力、纳差、关节痛、肌痛;上呼吸道症状有鼻分泌物增多、口咽部溃疡、声嘶;肺部症状有咳嗽、胸痛、呼吸困难;肾损害表现有不同程度的蛋白尿、镜下血尿或红细胞管型;其他表现可有皮肤黏膜损害、血疱、结节、红斑、结膜角膜炎、多发性神经炎、心肌炎、耳部损害。胸部X线检查表现为肺单侧或双侧多发或孤立结节影,大小不等,边界清楚,多有空洞形成,空洞壁薄而形态不规则,罕有液平存在。口咽、肺、肾组织活检及实验室检查有助于诊断,典型病例胞浆型抗中性粒细胞胞浆抗体(c-ANCA)阳性。

\subsubsection{七、弯刀综合征}

弯刀综合征为一种罕见的先天性血管畸形,其特征为右肺静脉开口于下腔静脉,X线胸片显示血管形态类似古代土耳其武土佩带的弯刀。患者常有反复咳嗽、咯血、右肺感染,易被误诊为“支气管扩张”。胸部X线检查为:①右肺发育不全;②X线检查沿右心缘的肺静脉呈弯刀样阴影;③心脏向右移位,状似右位心,据此可作出诊断。

\subsubsection{八、替代性月经}

成年女性发生与月经期相应的周期性咯血,须考虑为“替代性月经”。国内文献报告一例每次咯血都在月经周期前2~3天开始,待月经过后即能自行停止。此种异常现象罕见,原因未明,有人认为体内雌激素的周期性浓度增高,引起肺毛细血管充血、出血所致。部分患者在咯血周期前1周,应用黄体酮治疗可预防出血。此外,气管或支气管子宫内膜异位也可引起此现象,但罕见。

对此类与月经周期有明显关系的周期性咯血,须经细致检查与长期观察,而不能发现咯血

的其他原因时,方可下“替代性月经”的诊断。

(周燕斌 谢灿茂)

\protect\hypertarget{text00061.html}{}{}

\section{参考文献}

1.孙书明,等.138例咯血患者的胸部X线检查与纤维支气管镜检查对照分析.中华结核和呼吸杂志,1995,18(4):226

2.来孺牛.纤维支气管镜、螺旋CT对咯血的诊断价值.现代中西医结合杂志,2004,6:352

3.姜静波,等.CTA与DSA对支气管动脉性咯血临床应用价值的比较.医学临床研究,2012,29(7):1334-1337

4.宋美君,等.多层螺旋CT支气管动脉造影与数字减影下经股动脉支气管动脉造影在咯血诊治中的对比.中国呼吸与危重监护杂志,2012,11(4):378-381

5.胡华成,等.X线胸片正常咯血患者病因的进一步诊断.中华结核和呼吸杂志,1994,17(6):377

6.陈志烈,等.胸片无明显异常的支气管扩张症青年患者咯血的诊断与鉴别.宁夏医学院学报,2001,23
(1):28

7.林金学,等.单纯气管、支气管结核病28例临床分析.中华结核和呼吸杂志,1997,20(6):368

8.杨远,等.纤维支气管镜检讨胸片正常的支气管内膜结核.中华结核和呼吸杂志,1995,18(1):12

9.陈文彬,等.酷似肺癌的支气管结核六例.中华结核和呼吸杂志,1995,18(4):246

10.张立华,等.T-SPOT与结核菌素试验对结核病患者的临床诊断价值.中华临床医师杂志(电子版),2012,6(14):4107-4108

11.高明乐,季卫星.支气管壁内结石致大出血介入治疗1例.临床荟萃,1998,13(22):1040

12.林耀广,等.肺癌在支气管镜下的特征.中华内科杂志,1998,37(4):235

13.张逊,等.支气管类癌外科治疗11例报告.中华结核和呼吸杂志,1995,18(1):40

14.张兵,雷发国.34例肺吸虫病误诊临床分析.中华传染病杂志,1998,16(1):54

15.刘云霞,等.混合单抗双抗体夹心ELISA法诊断疑难肺吸虫病二例.中华内科杂志,1997,36(6)420

16.王维山,等.38例支气管囊肿临床及X线形态分析.中华结核和呼吸杂志,1994,17(5):314

17.易善国,等.弯刀综合征一例.中华内科杂志,1992,31(3):176

18.陈庆荣,等.Goodpasture综合征(附7例)报告及文献复习.中华肾脏病杂志,1991,7(2):93

19.毕经瑞.支气管子宫内膜异位症引起咯血1例报告.吉林医学,1998,19(6):333

\protect\hypertarget{text00062.html}{}{}

