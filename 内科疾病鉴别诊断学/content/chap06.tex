\chapter{胸腔积液}

健康人胸膜腔的两层胸膜之间被以薄层的浆液,此薄层润滑性浆液的量保持相当稳定,是由于胸腔内液体持续滤出和吸收并处于动态平衡。目前认为液体由于压力梯度从壁层和脏层胸膜的体循环血管通过有渗漏性的胸膜进入胸膜腔,然后通过壁层胸膜的淋巴管微孔(stomas)经淋巴管回吸收,而任何因素使胸膜腔内液体形成过快或吸收过缓,即产生胸腔积液。胸腔积液可由于胸膜炎症、肿瘤、结缔组织病、局部淤血以及全身性疾病(例如慢性肾炎与营养不良所致的低蛋白血症)等所引起。目前报道有50种以上的疾病可以产生胸腔积液,不同病因可引起渗出性或漏出性胸腔积液(表\ref{tab6-1})。

\begin{table}[htbp]
\centering
\caption{胸腔积液疾病的分类}
\label{tab6-1}
\includegraphics[width=5.95833in,height=4.55208in]{./images/Image00049.jpg}
\end{table}

胸腔积液又可区分为原发性与继发性两类。前者起因于胸膜本身的病变;后者则起因于其他器官的或全身性病变,例如大叶性肺炎、慢性充血性心力衰竭,各种原因的低蛋白血症等。

\section{【胸腔积液的诊断步骤】}

\subsection{(一)确定有无胸腔积液}

中量以上的胸腔积液诊断不难,症状和体征均较明显。少量积液(0.3L)仅表现为肋膈角变钝,有时易与胸膜粘连混淆,可行患侧卧位胸片,液体可散开于肺外带。体征上需与胸膜增厚鉴别,胸膜增厚叩诊浊音,听诊呼吸音减弱,但往往伴有胸廓扁平或塌陷,肋间隙变窄,气管向患侧移位,语音传导增强等体征。B超、CT等检查可确定有无胸腔积液。

\subsection{(二)区别漏出液和渗出液}

尽快行诊断性胸腔穿刺区别积液的性质,见表\ref{tab6-2}\footnote{\textsuperscript{*}
Light标准:①胸腔积液/血清蛋白比值>0.5;②胸腔积液/血清LDH比值>0.6;③胸腔积液LDH水平>血清正常值高限的2/3。符合3条中任何1条可诊断为渗出液,无1条符合者为漏出液}。

\begin{table}[htbp]
\centering
\caption{渗出液与漏出液的鉴别}
\label{tab6-2}
\includegraphics[width=5.94792in,height=4.71875in]{./images/Image00050.jpg}
\end{table}

Light标准被认为是鉴别漏出液与渗出液的“金标准”,其敏感性和特异性分别达到99\%
和98\%。但也有例外。胸腔积液的蛋白质含量与比重常和积液存在时间的长短有关。胸水存在时间较短,蛋白质含量与比重可能较低,反之则可较高。强烈利尿也可引起假性渗出液,在这种情况下,有作者提出血清与胸腔积液蛋白浓度差>31g/L,或血清-胸腔积液白蛋白梯度>12g/L作为进一步判断是否漏出液的标准。

以下漏出液的临床特点对病因的分析和确定是重要的。充血性心力衰竭多表现为双侧胸腔积液,积液量右侧多于左侧,强烈利尿可引起假性渗出液。肝硬化胸腔积液多伴有腹水。肾病综合征胸腔积液多为双侧,可表现为肺底积液。低蛋白血症的胸腔积液多伴有全身水肿。心包积液引起的胸腔积液多在左侧。如不符合以上特点,或伴有发热、胸痛等症状应行诊断性胸腔穿刺进行积液分析,以避免漏诊其他原因引起的渗出液。

有些积液难以确切地划入漏出液或渗出液,见于恶性胸腔积液,系由于多种机制参与积液的形成。

\subsection{(三)寻找胸腔积液的病因}

如确定为漏出液,应寻找引起漏出液的原因,其主要原因有充血性心力衰竭、肝硬化、肾病综合征、低蛋白血症等,可根据临床特点和检查作出诊断。如确定为渗出液,则应通过积液的检查和分析,或借助器械检查作出诊断。

\section{【胸腔积液的症状、体征与X线征】}

\subsection{(一)小量胸腔积液}

小量胸腔积液常无症状,但如为胸膜急性炎症所致,则多有胸痛与干咳,肺部听诊可闻胸膜摩擦音。此种小量的积液只有在X线透视下方能确诊。立位透视可见肋膈角变钝或填平,患侧膈肌呼吸运动减弱;仰卧位透视则积液散开,肋膈角复呈锐利。超声检查能发现小量胸腔积液。

\subsection{(二)中等量胸腔积液}

缓慢增长的中等量胸腔积液,患者较能适应,但往往在活动后出现气促与心悸。视诊发现患侧胸廓较为饱满、呼吸运动减弱、肋间隙增宽,触觉语颤减弱或消失,叩诊呈浊音或实音,听诊呼吸音减弱或消失,根据这些典型的胸腔积液体征,往往不需做X线检查而能确定诊断,但有时仍须与下列两种情况相区别:

\subsubsection{1.胸膜增厚}

胸膜增厚时,叩诊与听诊的体征可与胸腔积液相同;胸膜增厚不严重时,语颤可增强或与健侧相等,据此可作为与胸腔积液鉴别之助。如慢性胸膜增厚兼有患侧胸廓下陷,气管、心脏向患侧移位等表现,则一般不难与胸腔积液相鉴别。

\subsubsection{2.一侧膈肌升高}

在膈肌升高的胸部下方叩诊可呈现浊音,平静呼吸时呼吸音也可减弱,可与胸腔积液混淆。嘱患者坐位,作腹式深呼吸,如浊音部位呼吸音仍然减弱或消失,语颤也减弱,则为胸腔积液;如深呼吸时呼吸音不减弱,则为膈肌升高。如为膈肌麻痹,则须经X线透视方能区别。

中等量积液时,X线透视下或后前位胸片见患侧胸下部或中下部显示密度较高的均匀阴影,积液上缘呈现向外、向上的弧形。有时渗液的上缘可呈圆顶状阴影,可被误认为膈肌升高,如嘱患者仰卧位透视,此时积液散开,患侧整个肺野的透亮度降低,并可观察到膈肌的真正位置。

\subsection{(三)大量胸腔积液}

如积液为大量,则由于纵隔向健侧移位、肺呼吸面积减少,患者常有心率加快、气促、呼吸困难。体检发现患侧胸廓饱满,气管与心浊音界向健侧移位,胸腔积液征也更明显。X线透视下或胸片见患侧胸部大部分呈均匀的致密阴影,肺尖一般仍可见到含气的肺组织,气管和纵隔向健侧移位,膈肌下降,患侧肋间隙增宽。

\subsection{(四)不典型的胸腔积液}

如包裹性积液,叶间隙、肺底等处的局限性积液,易与肺内或胸膜肿瘤相混淆,须利用X线检查与超声检查,结合临床表现加以鉴别。

\subsubsection{1.肺下积液}

系指积聚于肺底与膈上的胸腔积液,立位X线表现与横膈升高相似,须与膈高位及膈肌麻痹相鉴别。肺下积液可区分为流动型与包裹型两类,流动型可采用变换体位的X线透视法诊断之,嘱患者向患侧倾斜,可见液体部分溢入肋膈角或腋缘前方。包裹性肺下积液时,则可用胸部CT检查等方法而与膈下病变鉴别,并经诊断性穿刺阳性而确定为积液。超声波探查也有助于积液的诊断,且能提示穿刺的定位。

\subsubsection{2.叶间积液}

在侧位片上呈梭形的致密阴影,上下有明显的边缘,按解剖位置可决定积液发生在哪些肺叶之间。

\subsection{(五)多发性浆膜腔积液}

除了双侧胸腔积液外,其他浆膜腔也可同时或先后出现积液,如胸腔积液伴心包积液、胸腔积液伴腹腔积液或胸腔积液伴心包和腹腔积液。心包疾病引起的胸腔积液多在左侧,或为双侧胸腔积液。

\section{【诊断性胸腔穿刺】}

对明确积液性质及病因诊断均十分重要。疑为渗出性积液必须做胸腔穿刺,如有漏出液病因者则避免胸腔穿刺。胸腔积液可从外观、气味等估计胸腔积液的病因。

\subsection{(一)外观}

浆液性渗出液常见于结核性胸膜炎、脓胸的早期和胸膜转移癌的早期,有时见于风湿热与结缔组织病。

血性胸腔积液多见于胸膜创伤,偶尔见于自发性气胸、主动脉瘤破裂等。文献报道恶性积液中血性者占50\%~85\%。胸腔积液中红细胞数>100×10\textsuperscript{9}
/L者,多由恶性肿瘤、创伤和肺梗死引起。

浆液血性胸腔积液可见于胸膜转移癌、胸膜间皮瘤、血液病、炭疽杆菌性胸膜炎,有时见于肺吸虫病、结缔组织病、结核性胸膜炎等。

脓胸常由于葡萄球菌、肺炎链球菌等引起,也可由于结核分枝杆菌、化脓性链球菌等引起。结核性脓胸、慢性非特异性脓胸、肺胸膜放线菌病及肋骨骨髓炎所致的脓胸,均可有胸壁瘘管形成。

乳糜性胸腔积液,常由于胸导管阻塞、破裂或受压所引起,病因为胸部手术、恶性肿瘤、丝虫病或淋巴结核、创伤等。也可由于肺淋巴管肌瘤病(LAM)引起。

巧克力色积液考虑阿米巴肝脓肿破溃入胸腔的可能。

黑色积液多为曲霉感染。

黄绿色积液见于类风湿关节炎。

\subsection{(二)气味}

大多数胸腔积液并无特殊气味,厌氧菌感染的胸腔积液则有恶臭,尿毒症胸腔积液可有氨味。大肠杆菌、粪产碱杆菌性脓胸,脓液常带有粪臭味。

诊断性胸腔穿刺对多浆膜腔积液也有重要意义。如果临床上有漏出液的病因,可不作诊断性穿刺,疑为渗出液者则必须行诊断性穿刺。多浆膜腔积液临床上并不少见,一般各浆膜腔积液的性质相同,故诊断性穿刺在一个胸膜腔已足够提供资料诊断分析。文献报道以结核病占大多数(62.5\%),其次为肿瘤(31.2\%),再次为结缔组织病(6.3\%)。

\section{【胸腔积液分析】}

对诊断性胸腔穿刺得到的胸腔积液进行化验检查,结果的分析可明确约70\%的病因。

\subsection{(一)渗出液和漏出液鉴别的常用化验检查}

参见表\ref{tab6-2}。

\subsection{(二)葡萄糖pH}

漏出液与大多数渗出液葡萄糖含量正常。而脓胸、类风湿关节炎、系统性红斑狼疮、结核和恶性胸腔积液中含量可<3.3mmol/L。肿瘤患者胸腔积液葡萄糖含量很低者提示肿瘤广泛浸润,预后不佳。pH在上述情况下也降低,在食管破裂引起的胸腔积液和脓胸时一般<7.0。由于葡萄糖和pH与感染关系较大,如炎症性积液pH<7.0,葡萄糖<2.2mmol/L应肋间插管引流。如肿瘤性积液则预后不良。

\subsection{(三)酶学}

乳酸脱氢酶(LDH)是反映胸膜炎症程度的指标,其值越高,表明炎症越明显。胸腔积液LDH>500U/L,胸腔积液/血清LDH比值>3.0,LDH的同工酶LDH\textsubscript{2}
升高者常提示为恶性胸腔积液。

胸腔积液溶菌酶(LZM)活性<65mg/L时,提示可能为恶性胸腔积液,而>80mg/L时提示可能为结核性胸腔积液;数值愈高,则结核的可能性愈大。结核性胸腔积液时溶菌酶活性与LDH活性同时升高,而恶性胸腔积液时则溶菌酶活性降低而LDH活性升高,这种矛盾分离现象是恶性胸腔积液的特点。

腺苷脱氨酶(ADA)在结核性胸膜炎时胸腔积液中ADA多高于45U/L,其诊断结核性胸膜炎的敏感性和特异性均较高。但HIV感染或AIDS合并结核性胸膜炎时胸腔积液ADA可不升高。

胸腔积液淀粉酶升高可见于急性胰腺炎、恶性肿瘤等。急性胰腺炎伴胸腔积液时,淀粉酶溢漏致使该酶在胸腔积液中含量高于血清中含量。淀粉酶同工酶测定有助于肿瘤的诊断,如唾液型淀粉酶升高而非食管破裂,则恶性肿瘤可能性极大。

\subsection{(四)细胞因子和免疫学检查}

γ-干扰素在结核性胸腔积液增高,多>200pg/ml,对区别良恶性胸腔积液有重要价值,且敏感性和特异性均较高。胸腔积液中细胞因子在良恶性病变的变化国内已经作了大量的研究,有报道胸腔积液中肿瘤坏死因子(TNF)和白介素8(IL-8)在结核性胸腔积液明显高于恶性胸腔积液,而胸腔积液中外周血多形核白细胞上黏附分子CD11b/CD18在结核性胸腔积液则明显低于肿瘤性胸腔积液,但这些患者血清中CD11b/
CD18在结核性则明显高于肿瘤性胸腔积液,呈胸腔积液和血清含量的分离现象。这三项指标都有很高的敏感性、特异性和正确性,如2项或3项指标联合检测,其敏感性、特异性和正确性达96\%~100\%。

系统性红斑狼疮及类风湿关节炎引起的胸腔积液中补体C3、C4成分降低,免疫复合物的含量增高。系统性红斑狼疮胸腔积液中抗核抗体滴度可达1∶160以上。

\subsection{(五)肿瘤标志物}

癌胚抗原(CEA)在恶性胸腔积液的早期即可升高,且比血清更显著。若胸腔积液CEA升高或胸腔积液/血清CEA>1,常提示为恶性胸腔积液,其敏感性约60\%,特异性则90\%以上。近年来还开展了许多肿瘤标志物检测,端粒酶是近年备受重视的一种特殊的逆转录酶,国内报道一组30例恶性胸腔积液中90\%含量升高,而良性胸腔积液5.7\%升高,但活性较低。诊断恶性胸腔积液的敏感性、特异性和正确性均超过90\%,明显高于CEA。其他的肿瘤标志物如肿瘤糖链相关抗原(CA50、CA125、CA153、CA19-9等)、细胞角蛋白19
(CK19)片段、神经元特异性烯醇酶等,可作为鉴别诊断的参考。联合检测多种肿瘤标志物,可提高阳性检出率。

\subsection{(六)细胞学}

胸腔积液找癌细胞的阳性率为40\%~90\%,平均约60\%。而胸腔积液沉淀细胞染色体检查阳性率达81\%,但技术要求较高。胸膜间皮细胞有时易误诊为肿瘤细胞,组织化学染色可鉴别。

\section{【其他辅助诊断检查】}

\subsection{(一)结核菌素皮内试验}

阳性特别是强阳性反应时,常见于结核性胸腔积液;但阴性反应并不能排除结核性胸腔积液。

\subsection{(二)超声检查}

用于估计胸腔积液的深度和积液量,可以鉴别胸腔积液、胸膜增厚、胸腔积液兼有胸膜增厚以及液气胸等不同情况。对包裹性积液、肺下积液、叶间积液和少量积液也能提示相当准确的定位诊断,有助于诊断性和治疗性穿刺。

\subsection{(三)胸部CT}

能检出常规胸片分辨困难的病变,显示肿块、结节、胸膜斑块、钙化和包裹性积液的程度和部位。壁层胸膜增厚往往是渗出液的征象。胸部CT见到以下征象:①外周胸膜增厚;②结节状胸膜增厚;③壁层胸膜增厚>1cm;④纵隔胸膜受累或有原发肿瘤的证据,则可以提示恶性胸腔积液,据报道其特异性22\%~56\%,敏感性88\%~100\%。

\subsection{(四)\textsuperscript{18} F-FDG PET/CT}

在肿瘤早期诊断、临床分期、疗效检测等方面已得到广泛应用。应用于良恶性胸膜病变的鉴别时,胸膜恶性病变SUV值明显高于良性病变,以体重较正后SUV>2.2为切点,诊断的准确率为82.3\%。文献报道PET/CT显像鉴别恶性胸膜疾病的灵敏度、特异性、阳性预测值、阴性预测值、准确度为95\%、80\%、91\%、89\%、90\%,PET上胸膜FDG的异常摄取是鉴别良、恶性胸腔积液最精确的标准。使用注射\textsuperscript{18}
F-FDG后1小时、2小时的图像(双时相\textsuperscript{18} F-FDG
PET)用于鉴别良恶性胸膜病变,其发现良恶性胸膜病变在不同时相SUV值改变的比例(\%SUV)有明显差别。以\%SUV增加9\%作为鉴别诊断的标准,其敏感性67\%,特异性94\%。或最大SUV≥2.4及(或)\%SUV≥9\%作为判断标准,其敏感性增加至100\%,特异性94\%,阴性预测值100\%。

\subsection{(五)经皮闭式胸膜活检}

如疑为胸膜结核、胸膜转移癌或胸膜间皮瘤等,可考虑经皮闭式胸膜活检以助诊断。恶性病变的阳性率约为40\%~80\%,结核性胸膜病变阳性率约为60\%,多次活检可提高阳性率。拟诊结核病时,活检标本除做病理检查外,还应作结核分枝杆菌培养。CT或B超引导下活检可提高成功率。

\subsection{(六)胸腔镜检查}

胸腔镜能窥视整个胸膜腔,在直视下活检,安全且阳性率高,是疑难胸膜疾病病因诊断的最佳方法。国内一组245例原因不明的胸腔积液经胸腔镜检查后92.7\%得到确诊。

\subsection{(七)开胸探查}

经上述方法仍未能确诊者如无特殊禁忌可考虑开胸探查。

\protect\hypertarget{text00070.html}{}{}

\section{18 感染性胸腔积液}

\subsection{一、结核性胸膜炎}

在我国渗出性胸膜炎以结核性占最多数。患者多为青壮年,发病颇急,发热、患侧胸刺痛,呼吸、咳嗽时加剧,并有干咳、全身不适、潮热、盗汗、消瘦等结核中毒症状。热型多为不规则型或弛张型,个别不典型病例可无发热。早期积液量少时无明显体征,或有胸膜摩擦音。如积液大量时,胸痛减轻或消失,出现气促、心悸,体检发现胸腔积液体征,甚至出现心脏、气管向健侧移位的征象。慢性胸腔积液量虽不少,但由于患者逐渐适应,可无明显的气促、心悸等症状。

实验室检查血象白细胞总数正常或轻度增多,急性期常有中等度中性粒细胞增多与核左移。胸腔积液为渗出液,常呈透明的草黄色,胸腔积液有易凝固的倾向,比重常高于1.018,Rivalta试验阳性,蛋白质含量30~60g/L,多>40g/L,细胞数为数百~数千,以淋巴细胞为主,间皮细胞<5\%。胸腔积液ADA及γ干扰素增高。胸腔积液中结核分枝杆菌检出率不高(包括培养与动物接种),应用PCR技术可大大提高检出率。结核菌素试验强阳性,但老年患者可无发热,结核菌素试验亦常阴性。通常根据患者的临床表现与胸腔积液常规检查,即能确定结核性渗出性胸膜炎的诊断,而抗结核治疗的良好疗效能进一步证实此诊断。经皮闭式胸膜针刺活检阳性率可达60\%~80\%。内科胸腔镜诊断阳性率更高。X线胸片检查可同时发现有无肺结核病灶存在,疗程中前后对比,也有助于观察病情变化。

结核性胸腔积液有时可为浆液血性,须与肺癌转移所致血性胸腔积液相区别。前者经抗结核治疗后颜色渐转为草黄色,抽液后液量渐减少;胸腔积液ADA和γ干扰素升高。后者有明显的血性,积液量和颜色不因抽液及抗结核治疗而好转,并有发展的倾向;胸腔积液CEA或其他肿瘤标志物升高。此外,根据患者的年龄较大、血痰史、消瘦、持续性胸部闷痛、淋巴结转移、痰及胸腔积液中癌细胞或胸膜活检发现癌细胞浸润等情况,通常可与结核性血性积液相鉴别。必要时可行内科胸腔镜检查确诊。

\subsection{二、结核性脓胸}

结核性脓胸主要由于肺内结核病灶向胸腔穿溃,或未及时治疗的结核性浆液性胸膜炎转变所致。结核性自发性气胸也可并发结核性脓胸。自从抗结核药物的广泛应用以来,结核性脓胸已少见。结核性脓胸临床主要表现为结核性全身性中毒症状与胸腔积液体征。急性脓胸和混合感染性脓胸的中毒症状较重,慢性脓胸的症状则较轻,可不发热,但贫血及消瘦较明显,患侧胸廓可塌陷,肋间隙变窄,患侧呼吸音减低,常有杵状指(趾)。

胸腔穿刺脓性积液为淡黄色,稀薄,含有干酪样物质,白细胞数在10
000×10\textsuperscript{6}
/L以上,脓细胞多,应用涂片法、培养法或PCR可发现结核分枝杆菌,其中尤以后一项检查阳性率较高。

慢性结核性脓胸常并发支气管胸膜瘘、胸壁瘘,最易混合感染。有支气管胸膜瘘时,胸腔穿刺注入1\%亚甲蓝溶液或乙醚1ml,则咳出带蓝色痰液或痰中有乙醚气味,有助于此并发症的诊断。支气管胸膜瘘患者咳嗽多频繁,痰有臭味,咳嗽与咳痰常与一定的体位有关。

患者有下列条件之一项,可确诊为结核性脓胸:①脓液中结核分枝杆菌检查(涂片法、培养法或PCR)阳性;②胸膜活检证明有结核性病变。如脓胸患者有肺结核存在,而无肺与邻近器官化脓性病变及外伤史,虽脓液中未发现结核菌,结核性脓胸的可能性仍甚大。有时结核性脓胸混合普通致病菌感染而形成混合性脓胸。

\subsection{三、类肺炎性胸腔积液与脓胸}

类肺炎性胸腔积液(parapneumonic
effusions)多继发于肺炎、肺脓肿、支气管扩张、外伤感染,以及邻近器官的化脓性感染,如肝脓肿、膈下脓肿、化脓性心包炎、化脓性纵隔炎的蔓延。早期表现为类肺炎性胸腔积液,继后才有化脓性变称为脓胸。败血症也可经由血行播散而引起脓胸。

此病的临床表现为先有肺炎、肺脓肿等原发病的表现,然后出现胸腔积液,积液量一般不多,患者有发热、咳嗽、咳痰、胸痛等症状,血白细胞升高,中性粒细胞增加伴核左移。若肺部感染未能控制,致病菌直接侵袭、穿破入胸腔则造成胸腔积脓。脓胸的致病菌大多为肺炎链球菌、金黄色葡萄球菌、化脓性链球菌,少数为大肠杆菌、副大肠杆菌、肺炎克雷伯杆菌、伤寒杆菌、粪产碱杆菌、布氏杆菌、真菌、放线菌、奴卡菌等,且多合并厌氧菌感染。急性脓胸常表现为高热、畏寒、寒战、剧烈胸痛、气促、咳嗽等;慢性脓胸有胸膜增厚、胸廓塌陷、慢性消耗和杵状指(趾)等。符合下列任何一项者可诊断为脓胸:①胸腔积液呈脓性、黏稠;②涂片革兰氏染色找到细菌;③或脓液细菌培养阳性。

继发于肺炎的肺炎链球菌性脓胸,其胸腔积液多为黄色或黄绿色而黏稠。链球菌性脓胸,其脓液较稀薄而呈淡黄色。金黄色葡萄球菌性脓胸,脓液稠厚而带黄色,并有脓块形成。铜绿假单胞菌性脓胸,脓液呈淡绿色。大肠杆菌、粪产碱杆菌性脓胸,脓液常带有粪臭味。厌氧性链球菌、梭状杆菌、螺旋体性腐败性脓胸,脓液常具有强烈的腐败恶臭味。如为产气性细菌性脓胸,则形成脓气胸。

\subsection{四、放线菌胸膜炎}

肺放线菌病在我国少见,此病常累及胸膜、肋骨、胸壁,易误诊为肺结核、脓胸、肺癌。最初表现为肺炎性浸润,病变多发生于肺下部,但也可在肺任何部分。肺部病变发展缓慢而直接向周围蔓延,侵犯胸膜则产生渗出液与脓胸,累及肋骨与胸壁则引起肋骨损害与胸壁瘘管形成。

胸膜放线菌病主要须与结核性脓胸、肺癌胸膜转移等相区别。放线菌病的肺、胸膜与肋骨三部分病变直接连在一起,发生在同一个部位,是此病的特点,与结核病、肺癌有所不同。结核性脓胸虽可并发肋骨损害与胸壁瘘管,但胸壁炎症反应较轻;放线菌病引起的胸壁病变,除红肿与压痛之外,其周围组织变硬,有胸壁脓肿和瘘管形成,分泌物中可找到“硫磺颗粒”,其中含有放线菌,有助于此病的确诊。肺癌虽可侵及胸膜与肋骨,但罕有形成胸壁瘘管者。

\subsection{五、真菌性胸膜炎}

真菌感染引起的胸腔积液。较为罕见,仅占所有胸腔积液病因的1\%以下。引起真菌性胸膜炎的病原体主要有烟曲霉、皮炎芽生菌、组织胞浆菌、粗球孢子菌、新型隐球菌、念珠菌和马内菲青霉等。多继发于肺部真菌病,或肺部真菌性肺脓肿穿破入胸腔,少数由血流或邻近器官的真菌感染引起。念珠菌性胸膜炎部分是由食管穿孔所致。曲霉感染多因手术后感染所致,如肺癌、肺结核和曲霉肿肺叶切除后,多伴有支气管胸膜瘘。部分真菌性胸膜炎合并细菌感染。

大多患者有真菌感染的危险因素,如中性粒细胞减少、接受免疫抑制剂和糖皮质激素治疗、创伤、大手术等。表现为发热(可高热或低热)、咳嗽、咳痰或咯血,胸痛和呼吸困难。除了侵袭性肺曲霉病外,变应性支气管肺曲霉病(ABPA)也可引起胸腔积液。粗球孢子菌胸膜炎多为原发性感染,胸痛症状较为突出。组织胞浆菌病多伴有肺门纵隔淋巴结肿大和心包炎。病程长者还有中毒症状如消瘦、疲乏等。如继发于艾滋病、白血病、淋巴瘤等有基础病的相应症状。

有胸腔积液相应的体征,但大多数患者积液量较少,积液体征并不明显。

外周血白细胞可升高,粗球孢子菌病和APBA嗜酸性粒细胞可升高。胸腔积液中涂片或培养可发现真菌。隐球菌和接合菌以外的侵袭性真菌病其血清1,3-β-D葡聚糖抗原检测(G试验)可阳性。侵袭性曲霉病的血、尿、脑脊液及肺泡灌洗液曲霉半乳甘露聚糖测定(GM试验)可阳性。皮炎芽生菌、荚膜组织胞浆菌和粗球孢子菌病补体结合试验的滴度在部分患者升高,前者不足25\%,后两者则大多数阳性。隐球菌和组织胞浆菌胸膜炎血清和胸腔积液的隐球菌抗原可阳性。

胸腔积液中涂片或培养真菌阳性,或活体组织标本病理检查发现真菌可确诊。本病需与其他感染性胸腔积液如结核性胸腔积液鉴别。

\subsection{六、阿米巴病胸腔积液}

阿米巴病并发胸膜炎比较少见,主要由于阿米巴性肝脓肿向右膈肌破溃所致;其次是继发于肺阿米巴病。阿米巴肝脓肿可向上刺激胸膜,引起胸膜反应,造成胸膜渗液。这种积液清晰,且不含阿米巴,此时可被误诊为结核性胸膜炎。如阿米巴肝脓肿向胸腔穿破,则形成巧克力色积液。国内一组737例阿米巴肝病并发脓胸者58例次(7.9\%)。如同时累及肺组织则称为肺、胸膜阿米巴病。偶尔也有左叶肝脓肿向左侧胸腔穿破,引起左侧胸膜阿米巴病者。

常见的临床表现是:患者先有发热、消瘦,肝脓肿的症状与体征,继而出现一侧(多为右侧)胸痛、咳嗽、胸膜炎的症状与体征。肝脓肿向胸腔穿破时往往有突然发生的剧烈胸痛与呼吸困难。如有胸膜支气管瘘形成,则有咯血及咳出棕褐色(巧克力色)痰,痰液中可检出溶组织阿米巴。注入1\%亚甲蓝溶液于胸腔后,痰液迅速染成蓝色。

此病可被误诊为结核性胸膜炎、脓胸等疾病,肝脏病变也可能被忽略。临床上应注意到膈下病变易蔓延到膈上,而膈上病变少蔓延到膈下的规律,如患者有肝脏病变,继而出现胸膜或肺、胸膜病变,须警惕阿米巴病的可能性。

\subsection{七、肺吸虫性胸膜炎}

肺吸虫病患者胸片检查常有胸膜病变,有的报告约15\%并发胸腔积液。积液为草黄色、透明,也可为乳白色,个别呈血性或脓性。胸腔积液含嗜酸性粒细胞较多,有时可发现夏科-雷登晶体。胸腔积液或痰液中发现肺吸虫卵可确定诊断。抗肺吸虫药物治疗及穿刺抽液后可治愈胸膜炎症。

此病须与结核性胸膜炎相区别,后者在机体过敏反应显著时,积液中也可有多数嗜酸性粒细胞。据临床报告,肺吸虫性胸膜炎被误诊为结核性胸膜炎者并非少见。如患者曾在流行地区居住,并有生食螃蟹、蝲蛄的历史,经抗结核治疗无效,应考虑此病的可能。近年应用ELISA法、放射免疫法(RIA)检测,结果阳性率较高且较特异性。

\subsection{八、恙虫病性胸膜炎}

恙虫病肺部受累常见,表现有肺充血,伴有支气管肺炎和胸腔积液。国内报道一组31例患者中累及胸膜者有8例(26\%),胸腔积液量大多为少到中等量,可单侧或双侧,一般随病情好转可自行吸收(参见第3.1.5节)。

\protect\hypertarget{text00071.html}{}{}

\section{19 恶性胸腔积液}

恶性胸腔积液为恶性肿瘤胸膜转移或恶性胸膜间皮瘤引起者,前者占恶性胸腔积液病因的90\%以上,其中最主要的是肺癌、乳腺癌和淋巴瘤直接侵犯或转移至胸膜所致,约占恶性胸腔积液的75\%。其他如泌尿生殖系统肿瘤、胃肠道肿瘤、血液系统肿瘤等也可引起,但多有原发肿瘤的症状和体征,有些病例以胸腔积液作为首发表现,则应仔细寻找病因。有些恶性肿瘤患者其胸腔积液并非由肿瘤直接侵犯或转移至胸膜,而是其他原因引起者,如淋巴管阻塞,则称为“类恶性胸腔积液”(paramalignant
pleural effusions)。

\subsection{一、肺癌合并胸膜转移}

肺癌合并胸膜转移颇为常见,易被误诊为结核性胸膜炎,特别是当有大量积液而肺实质情况未明时,易于误诊。凡40岁以上患者出现胸腔积液,无发热,有胸部钝痛、咳血丝痰和消瘦等症状,胸腔积液多呈血性、量大、增长迅速,而结核中毒症状不明显,胸腔积液中未找到抗酸杆菌,胸腔积液CEA升高、胸腔积液/血清CEA>1、LDH>500U/L时应考虑恶性胸腔积液的诊断。其他多种肿瘤标志物如糖类抗原CA19-9(CA19-9)、鳞状细胞癌抗原(SCC)、神经元特异性烯醇化酶(NSE)、细胞角蛋白19片段(CYFRA21-1)及端粒酶测定也有助诊断。胸腔积液细胞学检查、经皮针刺闭式胸膜活检、胸部影像学、纤维支气管镜以及胸腔镜等检查,有助于进一步明确诊断和鉴别。

\subsection{二、乳腺癌合并胸膜转移}

乳腺癌合并胸膜转移为晚期征象,文献报道其发生率仅次于肺癌,约占恶性胸腔积液的25\%。国内报告一组乳腺癌170例,其中有5例合并肺与胸膜转移。

\subsection{三、恶性淋巴瘤}

淋巴瘤的胸膜侵犯率为7\%~30\%,霍奇金淋巴瘤(HL)、非霍奇金淋巴瘤(NHL)、淋巴肉瘤等均可引起胸、腹水。淋巴肉瘤所致的胸、腹水可为血性,可被误诊为恶性肿瘤的胸、腹膜转移或结核性多发性浆膜腔炎。HL和NHL并发的胸腔积液多数为少量积液,积液的发生与淋巴管、静脉阻塞有关;部分为中到大量的血性积液则为胸膜转移所致。国内报告一组50例HL,并发胸水及腹水者各9例,其中胸、腹水并存者4例,2例胸、腹水均找到史-雷(Sternberg-Reed)细胞。

\subsection{四、恶性胸膜间皮瘤}

恶性胸膜间皮瘤(MPM)是原发性胸膜肿瘤,来源于间皮组织,可区分为局限型与弥漫型。本病临床上少见,多在40岁以上发病。目前认为发病可与石棉接触有关。但国内综合报告310例中,只2.9\%有石棉接触史;60.3\%病例有胸腔积液,其中3/4为血性胸腔积液。弥漫型胸膜间皮瘤多有胸腔积液,其临床特点是进行性胸痛、呼吸困难、血性胸腔积液及胸膜增厚,此外尚有乏力、体重减轻与刺激性咳嗽;X线胸片检查胸膜呈不规则状、波浪状起伏,或结节状增厚,而不伴有胸廓凹陷,相反甚至可能凸出,与慢性炎症引起胸膜增厚不同。胸腔积液检查可发现肿瘤细胞。胸腔积液透明质酸增高>250mg/L。

据国内一组30例报告,平均年龄为40.1岁,有下列几种不同的X线征:①胸部孤立性肿块,X线定位来自胸膜,且密度高、边缘光滑,呈分叶状;②多发性胸膜分叶状肿块;③胸膜肿块伴同侧胸腔积液;④胸腔积液在抽液注气之后,X线显示形态不规则的明显胸膜增厚和胸膜上结节影;⑤胸膜肿块并侵蚀邻近软组织或(及)骨质。B超或CT亦显示胸膜有实质性肿块、或胸腔积液伴有明显的胸膜不规则增厚。

如患者有长期石棉接触史,或病变区胸痛进行性加重,或有骨关节疼痛、低血糖症、发热和贫血,或血性胸腔积液抽液后又迅速增长,并有胸腔积液中透明质酸含量增高,或胸膜肿块活检或胸腔积液细胞学检查发现间皮瘤细胞,则诊断更为明确。电视胸腔镜或内科胸腔镜胸膜肺活检术对病变范围观察完整,对组织标本获取确切,并且微创,已成为MPM的最佳诊断和鉴别诊断方法之一。

\subsection{五、类恶性胸腔积液}

类恶性胸腔积液(paramalignant
effusion)是指在伴发于恶性肿瘤而没有胸膜直接受侵袭或胸腔积液中未找到恶性细胞的胸腔积液。目前关于该类胸腔积液的数据及病因有限。国外报道在975例胸腔积液患者中,恶性胸腔积液占15\%,类恶性胸腔积液占16\%。后者可表现为漏出液或渗出液,最常见于肺癌患者,其机制可能与淋巴管堵塞有关。

\protect\hypertarget{text00072.html}{}{}

\section{20 结缔组织病与变态反应疾病}

\subsection{一、结缔组织病并发胸膜炎}

结缔组织病并发胸膜炎,以类风湿关节炎(RA)、系统性红斑狼疮(SLE)较多见,其他如Churg-Strauss综合征、进行性系统硬化、混合性结缔组织病(MCTD)、Wegener肉芽肿、皮肌炎、白塞病、结节性多动脉炎等也可导致胸腔积液,临床上较少见。

\subsubsection{(一)类风湿胸膜炎}

胸膜炎是RA最常见的胸部病变,多见于年龄60岁以上男性患者,RA史多10年以上。临床上有咳嗽、胸痛、活动后气促、关节疼痛;或无明显症状而于胸部X线检查时发现。胸腔积液多为单侧,或双侧性,多为少量或中等量,为渗出液,黄色或黄绿色,也可为乳状或血性。胸腔积液白细胞增高,(1000~3000)×10\textsuperscript{6}
/L,以淋巴细胞为主,蛋白质含量常>40g/L,LDH增高,多>1000IU/L。较为特征的是胸腔积液葡萄糖含量明显降低,常<1.68mmol/L,静脉输注葡萄糖也不能提高胸腔积液中葡萄糖含量;胸腔积液pH<7.30;胸腔积液补体C4升高,类风湿因子常阳性(比血清阳性率高),滴度常>1∶320。少数患者可发生无菌性脓胸。1/3患者合并有肺间质或肺实质病变。

\subsubsection{(二)狼疮性胸膜炎}

SLE合并胸膜炎者约50\%~75\%,胸腔积液可发生于病程的任何阶段,通常为双侧性,单侧较少,积液常为小量或中等量,大量者需考虑继发感染。胸腔积液为草黄色渗出液,有易凝的倾向,但有时也可为血性。胸腔积液白细胞数较类风湿胸膜炎低,常<1500×10\textsuperscript{6}
/L,单核细胞为主。蛋白质含量多>35g/L,LDH升高,但不超过500IU/L,胸腔积液葡萄糖含量高于类风湿性胸膜炎,常>3.36mmol/L,胸腔积液pH>7.30,补体含量甚低。有诊断价值的指标包括:胸腔积液抗ANA、ds-DNA抗体阳性,ANA滴度>1∶160或胸腔积液/血清ANA比值>1。

年轻女性的胸腔积液为中等量与血性,伴有多个器官损害的表现,应考虑狼疮性胸膜炎的可能性(参见第5.3节)。病程中可发生肺部病变,X线检查呈片状、块状或小结节状阴影,多侵犯下肺,常为双侧性,有游走与复发的倾向,对抗生素治疗无效,而对糖皮质激素治疗有良效。胸膜炎如出现于其他病征(尤其是面部蝶形红斑)之前常易被误诊为结核性。

药物性狼疮综合征最多由普鲁卡因胺与肼屈嗪引起。常见的临床表现是急性胸膜炎,出现胸腔积液与纤维性变。血清中抗核抗体滴度常增高。

\subsection{二、风湿性胸膜炎}

风湿性胸膜炎为急性风湿热表现之一,临床上比较少见,常见症状为发热、咳嗽、胸痛和呼吸困难。半数病例的胸腔积液为双侧性,为纤维素性浆液,量一般不多。急性期胸腔积液显示为渗出液,细胞以中性粒细胞为主,并混有红细胞,后期淋巴细胞及嗜酸性粒细胞增多。个别病例呈血性积液。

风湿性胸膜炎的诊断依据为:①患者有风湿热的临床表现;②胸膜炎的体征与X线征;③抗风湿药物治疗有良好效应;④除外其他原因的胸膜炎。

风湿性胸膜炎主要须与结核性胸膜炎相区别,前者对抗风湿药物治疗有良好效应,病程短,吸收快而无典型的胸膜粘连过程。风湿性胸膜炎常为风湿热的部分表现,可伴有风湿性肺炎,肺炎多为双侧性、游走性,且多呈局限性炎症性改变。

\subsection{三、嗜酸性粒细胞增多性胸膜炎}

嗜酸性粒细胞增多性胸膜炎实际上并非独立的疾病,临床上少见。文献报道此病可见于大叶性肺炎、肺栓塞、胸廓创伤、人工气胸、肺结核、吕弗勒综合征、Churg-Strauss综合征、肺包虫病、霍奇金淋巴瘤、全身性皮炎、支气管肺癌、花粉症、阿米巴性肺脓肿、肺吸虫病、蛔虫病、脓胸等。以气胸及血胸多见,既往认为恶性肿瘤引起者少见。但文献报道约占20.5\%~26.1\%。

通常胸腔积液中嗜酸性粒细胞计数与周围血的数值是相近的,一般不超过1\%~5\%,而本病的胸腔积液中嗜酸性粒细胞显著增多,大部分达50\%以上,有时可达90\%以上,而周围血中嗜酸性粒细胞计数正常或仅轻度增多。胸腔积液可澄清、混浊或血性。

\protect\hypertarget{text00073.html}{}{}

\section{21 心血管疾病}

\subsection{一、心脏疾病引起的胸腔积液}

胸腔积液是心脏病的常见表现,当基础心脏病严重到足以引起充血性心力衰竭时,大多数患者将会产生漏出性胸腔积液。心包疾病较少见,但是无论是急性或慢性心包疾病也可以引起胸腔积液,甚至在没有心力衰竭的情况下也可发生。心包疾病原因不同,胸腔积液可呈漏出液或渗出液。

住院的充血性心力衰竭患者常规X线胸片检查发现,58\%~73\%患者有胸腔积液。对CCU患者的超声检查则有半数患者有胸腔积液。如用敏感的CT技术,充血性心力衰竭患者高达87\%有胸腔积液。早期研究尸解发现死于心力衰竭的患者90\%有胸腔积液。88\%的积液是双侧的,8\%和4\%仅表现是右侧或左侧积液。

心包疾病引起胸腔积液决定于其病因,各种病因的心包疾病常规X线检查约35\%有胸腔积液。仅37\%为双侧积液,大多数(60\%)为左侧积液。胸内恶性肿瘤、结核病和结缔组织病如类风湿关节炎等均可累及心包和胸膜,引起心包和胸腔积液。

患者有典型的充血性心力衰竭表现,呼吸困难、端坐呼吸和水肿。较少有胸膜炎性疼痛。如积液量大,呼吸困难更加明显。体征有颈静脉怒张、肺部啰音、收缩期第3心音奔马律和外周水肿。肺基底叩诊浊音、震颤减弱,可闻支气管呼吸音。

心包炎患者有明显胸痛以及呼吸困难。典型者可闻及心包摩擦音,如有低血压、心率快、奇脉则要警惕心脏压塞。如心包炎是系统性疾病的表现时,如系统性红斑狼疮或类风湿关节炎,有基础疾病的其他表现。

由于充血性心力衰竭引起的胸腔积液为漏出液,蛋白低(胸腔积液/血清蛋白比<0.5),LDH不高,细胞数<1.0×10\textsuperscript{9}
,大多为淋巴细胞和间皮细胞。必须注意的是,如病程中强烈利尿可引起假性渗出液。

心包炎产生的积液多为渗出液。一般来说,心包液和胸腔积液性质相同,检查胸腔积液可能发现病因。如胸腔积液涂片发现恶性细胞或抗酸杆菌,心包液和胸腔积液ANA升高提示红斑狼疮。心力衰竭本身可伴有心包积液,但和胸腔积液相比较少发生,失代偿心衰约20\%有心包积液,为漏出液。后前位胸片示大多数患者双侧胸腔积液和心影增大。心力衰竭的其他征象包括肺血管淤血、间隔纹理增多和肺泡水肿。胸腔积液多为游离,中等量,不超过半个胸腔,积液量右侧多于左侧。超声检查有助于局限性或包裹性积液的诊断,对心脏疾病、心功能、心包积液和有无心包增厚的诊断优于常规X线检查。胸部CT检查可鉴别或排除肺内疾病。

\subsection{二、血管疾病引起的胸腔积液}

\subsubsection{(一)肺栓塞}

国内文献报道其发生率40.5\%,首诊误诊率达44\%。可引起渗出液或漏出液。肺栓塞患者主要表现为3组症状:①胸膜炎性胸痛或咯血;②呼吸困难;③循环障碍引起的晕厥。体格检查依发生频率可见气促,啰音,心率快,出汗,喘息,发热,胸膜摩擦音和Homans征。胸腔积液外观黄色或血性,WBC可<0.1×10\textsuperscript{9}
或>50.0×10\textsuperscript{9}
,细胞分类无特异性。虽然对已诊断为肺栓塞的患者胸腔积液检查意义不大,但对于合并发热的患者排除胸腔感染是必要的。诊断主要依赖影像学和其他检查。

\subsubsection{(二)脓毒性肺栓塞}

当血栓含有微生物时可发生脓毒栓子,可以是细菌、真菌或寄生虫。多数脓毒栓子来源于心脏的三尖瓣或室间隔缺损的心内膜炎或血栓性静脉炎。常见细菌为金黄色葡萄球菌。

患者多有系统性炎症反应综合征(SIRS)的临床表现。胸部X线检查肺周边可见多发的、边界不清的、圆形或楔状的阴影,阴影大小可相似或不一致,反映栓子栓塞的情况。空洞常见,许多发展迅速,通常是薄壁空洞没有气-液平面。

右心内膜炎的患者约25\%有胸腔积液,为渗出液,多数细菌培养阴性。积液细胞多为中性粒细胞、淋巴细胞或单个核细胞。积液特征性改变是乳酸脱氢酶明显升高。

\subsubsection{(三)Lemierre综合征}

此综合征是脓毒栓子的一种特殊情况,系口咽部感染后引起颈内静脉血栓性静脉炎和转移性栓子,常常转移到肺和关节。症状包括呼吸困难、胸膜炎性胸痛、发热和咯血。最常见细菌是厌氧的G\textsuperscript{-}
菌坏死梭杆菌。如有转移性感染和颈内静脉血栓性静脉炎可诊断。辅助诊断方法包括颈-胸的对比增强CT,典型影像学表现为静脉曲张静脉壁增厚、腔内充盈缺损和局部软组织水肿。约32\%患者有胸腔积液,14\%合并脓胸,12\%有气胸。

\subsubsection{(四)肺静脉阻塞性疾病}

较为罕见。可有反复的肺静脉血栓形成,临床表现为肺动脉高血压和肺水肿,病因不清。典型表现为缓慢发展的呼吸困难和急性肺水肿引起的强迫性端坐呼吸。胸部X线检查显示大多数患者有少量胸腔积液和Kerley
B线,可能系慢性肺毛细血管高血压使液体进入肺间质腔所致。胸腔积液为漏出液,此病诊断只能通过外科肺活检。预后较差,多在诊断后2年死亡。

\protect\hypertarget{text00074.html}{}{}

\section{22 消化系统疾病引起的胸腔积液}

胸膜腔与腹腔解剖上仅间隔几毫米,中间有膈肌分开。因为膈肌有很小的淋巴管穿过,所以任何腹腔液体只要接触到膈肌的腹腔侧可引起胸腔积液;同时,膈肌血管丰富,腹腔感染或非感染的炎症也可引起膈肌血管的炎症而产生胸腔积液。另外,由于手术、导管或外伤引起膈肌任何部位的破裂,也可使腹腔液体通过膈肌裂口进入胸腔而引起胸腔积液。

\subsection{一、肝脏疾病}

\subsubsection{(一)肝硬化}

肝性胸水定义为肝硬化和门脉高压的患者合并胸腔积液,并排除原发于心脏、肺或胸膜疾病的胸腔积液。伴有腹水的肝硬化患者约6\%有胸水。一般认为,如没有腹水,诊断肝性胸水必须小心。但仍有20\%左右的肝性胸水患者没有检测到有腹水。肝性胸水大多发生在右侧,约79.5\%,左侧17.5\%,双侧6\%。胸水性质为漏出液。

肝硬化患者尤其是伴有腹水者如产生胸水时应考虑为肝性胸水。患者可无症状或症状变化范围很大,如活动后气促到呼吸衰竭,决定于积液量的大小,如腹水量,胸水产生的速度和是否伴有肺部疾病等。

胸腹水性质相同,均为漏出液。证实胸腹腔之间有无相通最好的方法是闪烁显像,用放射性标志物质如\textsuperscript{99m}
锝硫磺胶体或\textsuperscript{99m}
锝巨聚白蛋白直接注入腹水,然后进行胸部间歇扫描,24小时内间歇扫描胸水显像阳性,大多在45分钟内阳性。无腹水者可腹腔内置入导管,注入含有放射性标志物的生理盐水500ml后进行胸部显像。

\subsubsection{(二)肝脓肿}

20\%伴有胸腔积液,胸腔积液的病因和胸腔积液的性质与膈下脓肿相似。由于未经治疗的肝脓肿死亡率极高,因此,对右侧胸腔积液为渗出性、白细胞占优势者应考虑是否有肝脓肿。

多数患化脓性肝脓肿的患者有发热,厌食,约50\%患者有寒战。约半数患者有肝胆道疾病。腹痛最为常见,但并不一定在右上腹部。大多数患者肝大质软。实验室检查可有白细胞升高、贫血、碱性磷酸酶升高和高胆红素血症。诊断依靠腹部CT或超声,确诊可在CT或超声引导下经皮穿刺抽吸脓液做出。

\subsubsection{(三)肝移植后}

几乎所有接受肝移植的患者(77\%~100\%)术后都发生胸腔积液。一般在移植后72小时内95\%发生右侧胸腔积液,5\%发生左侧积液。也有报道1/3是双侧,其余是右侧,右侧的积液量明显多于左侧。

\subsection{二、胰腺疾病}

胰腺炎是最常引起胸腔积液的疾病。胰腺炎相关胸腔积液有两个不同的类型。

\subsubsection{(一)急性胰腺炎}

最常见伴有胸腔积液的类型。主要表现腹痛、恶心、呕吐及腹胀,多数患者有发热,如重症胰腺炎时可有低血压或休克和水、电解质和酸碱平衡紊乱。50\%患者可伴有胸腔积液,多同时伴有腹腔积液和心包积液。由于有些患者积液时可伴有肺不张。胸腔积液多数是小量,当胰腺炎好转时可自行吸收。积液检查为渗出液,中性粒细胞占优势,淀粉酶明显升高,一般为血清浓度的2倍以上。

\subsubsection{(二)胰管破裂}

大多由于慢性胰腺炎伴有胰管狭窄或结石引起,也可因胰腺手术后、胰管内肿瘤或外伤引起。胰管破裂早期胰液通过网膜引起胰性腹水;后期腹水则位于腹膜后继而进入胸膜腔。这些瘘管会持续存在直到胰液排除减慢,胰管引流重建,或异常部分的外科切除。胰管破裂引起的胸水其淀粉酶极高,往往超过100
000IU/L。

有时,胰腺假性囊肿可移行到膈肌表面,虽然完整的假性囊肿可移行到纵隔,或罕见地移行到胸腔,表现为肿块性病变,但是,更常见的是囊肿渗漏或破裂入膈肌下、纵隔或胸膜腔。由于胸腔是负压,假性囊肿液进入胸腔没有阻力。来自假性囊肿或胰腺瘘管的胸水为渗出液,量大,中性粒细胞占优势,血性胸水嗜酸性粒细胞可升高。必须注意的是胸水淀粉酶在某些情况下可升高,如食管破裂、肿瘤、卵巢癌或异位妊娠,临床上应鉴别诊断。

\subsection{三、脾脏疾病}

脾梗死、脾血肿和脾脓肿常引起胸腔积液。临床和实验室特点表现积液多在左侧并往往伴有胸膜炎性疼痛;胸水为渗出液,小量,中性粒细胞升高,往往不需要特殊处理。但是,脾的基础疾病临床上应仔细鉴别,罕见的脾病变有脾肿瘤、上皮样囊肿、脾静脉栓塞和淀粉样变时的脾自发性破裂。

脾梗死是血管阻塞引起,脾动脉是单一的动脉,没有侧支循环。这种解剖上的特点使脾容易被镰状细胞所阻塞,在高原镰状细胞杂合子可以发生。治疗性脾栓塞处理脾功能亢进时当80\%脾梗死时可发生胸腔积液。

脾血肿常由外伤引起,胸腔积液常伴有膈下血肿。有些患者可能自诉无脾外伤史,因为很轻的外伤可引起脾血肿。胸腔积液可以是血性的或非血性的,如血肿持续较长时间时,积液可能超过2周才能完全吸收。

脾脓肿较罕见,有报道镰状细胞贫血和静脉吸毒者可发生脾脓肿。其他原因还有脾外伤、邻近部位感染和心内膜炎。脾结核也有报道。

\subsection{四、食管破裂}

常伴外伤,多由内镜或手术引起。自发性食管破裂少见,有食管肿瘤、Barrett溃疡、放疗、食管克罗恩病、疱疹性食管炎或结核性食管炎,笔者还见过急速吞食肉块引起食管破裂的患者。

医源性食管破裂伴有胸腔积液最常见的原因是食管狭窄接受食管扩张后引起的,其他的内镜检查或治疗引起的还有食管异物的取出,食管静脉硬化治疗。主要的症状是胸痛,可持续数小时。如内镜检查或治疗后出现胸痛,应马上进行食管造影以建立诊断,然后禁食和使用抗生素。外伤或术后食管也可穿孔,如食管癌手术易有吻合口渗漏,胸腔手术导致食管的损伤等。

胃食管连接部贴近纵隔胸膜,任何原因引起的食管破裂可使食管液进入纵隔。由于食管黏液不是无菌的,可引起纵隔炎和脓胸。食管穿孔死亡者多由于需氧和厌氧菌引起纵隔炎所致。如发生脓胸则应积极引流。

胸腔积液的特征为淀粉酶升高,大多来自唾液淀粉酶。胸腔积液pH低,系由于细菌代谢产物和中性粒细胞代谢所致。积液可大量、单侧或双侧,单独右侧积液少见。有时胸腔积液还可见食物。

应尽可能迅速做出诊断,如患者有干呕或食管检查治疗操作并有肺部症状应马上评价有无食管破裂。站立位胸片可确定是否有胸腔积液或纵隔是否有气体,并可提示食管穿孔的部位。食管下段的穿孔引起左侧液气胸,食管中段的穿孔引起右侧胸腔积液。胸部CT比胸部平片对纵隔积液和积气诊断敏感性和特异性高。

\subsection{五、膈下脓肿}

约80\%有胸腔积液,多数来自外科手术后,腹部手术约1\%发生膈下脓肿,发生时间约为术后1~3周。一些膈下脓肿并没有腹部手术史,如胃、十二指肠和阑尾穿孔,憩室炎,胆囊炎,胰腺炎或外伤,这些患者往往容易漏诊。胸腔积液系膈下炎症引起膈肌毛细血管通透性增加所致,胸腔积液培养多阴性。

患者可有胸部或腹部症状,大多数患者术后有发热、白细胞增多和腹痛;胸部症状是胸膜炎性胸痛,X线表现为胸腔积液、下肺炎症和压缩性肺不张,以及患侧膈肌抬高。胸腔积液大多为少量,也可大量,性质为渗出液,多形核白细胞占优势。

\subsection{六、其他少见原因}

胃淋巴瘤、胃癌、小肠和大肠病(如Menetrier病、HIV肠病、重度Crohn病)、胆囊疾病、炎症性肝病、食管静脉曲张硬化治疗等均可伴有胸腔积液,诊断时须注意鉴别。

\protect\hypertarget{text00075.html}{}{}

\section{23 妇产科疾病引起的胸腔积液}

各种妇产科疾病可伴有胸腔积液。除了怀孕期间肺炎、病毒感染或肺栓塞等常见原因外,妇科和产科有些疾病可发生胸腔积液,如卵巢过度刺激综合征、产后胸腔积液等。

\subsection{一、卵巢过度刺激综合征}

卵巢过度刺激综合征(OHSS)是绒毛膜促性腺激素(hCG)促排卵的一种严重并发症,氯米芬偶然也可引起。此综合征以卵巢增大和液体移位为特征,引起血管内血容量严重不足。严重的OHSS可威胁生命,临床表现为大量腹水或胸水,以及呼吸困难、血黏度升高、肾/肝功能障碍或全身性水肿。重症OHSS也包括肾衰竭、血栓栓塞、大量腹水或急性呼吸窘迫综合征。

使用促性腺激素的患者约3\%~7\%发生严重OHSS,其中29\%有胸腔积液。胸腔积液的原因尚未清楚,可能与:①卵巢增大伴有卵泡、黄体、出血性卵巢囊肿和间质水肿形成;②短期内液体从血管内移出到血管外有关。OHSS的危险因素包括年龄<35岁,虚弱体质,多囊卵巢综合征,妊娠,使用hCG,高雌二醇血症(>2500pg/ml)和多滤泡。

发生OHSS最初多主诉腹部不适或腹胀,继而恶心、呕吐和腹泻。如果症状加重,多与发生腹水、胸水或呼吸困难有关。呼吸症状在注射hCG后7~14天发生,在最严重阶段,患者血黏度增高,凝血功能异常和肾功能减退。胸腔积液多发生在右侧(53\%),左侧18\%,双侧29\%。积液为渗出液。

\subsection{二、产后胸腔积液}

产后胸腔积液可在产后立即发生或1周后发生,其胸腔积液的病因不同。

\subsubsection{1.产后早发性胸腔积液}

系产后24小时内发生的胸腔积液。目前报道经阴道分娩早发性积液发生率约46\%~67\%,其中双侧积液55\%~75\%,这些病例均在分娩后24小时内摄正侧位胸片诊断。但是,如用超声诊断,仅2\%~17.6\%有胸腔积液。因所报道的病例数不多,其差别的原因还不清楚。产后早发性积液的原因还不清楚,推测与下列因素有关:①妊娠期间胶体渗透压降低,胸腔积液形成增加;②第二产程用力时憋气(Valsalva呼吸方式)由于系统静脉压升高可影响胸膜腔的淋巴回流。大多数患者无症状,不需要治疗。

\subsubsection{2.产后迟发性胸腔积液}

指产后数周内发生的胸腔积液。发生率较低,目前仅有少数病例报道。其原因可能与抗磷脂抗体的存在有关。典型的临床表现是产后数周内发热、胸腔积液和肺部渗出。患者往往有胸痛,有些患者有心包炎及(或)静脉血管栓塞。抗磷脂抗体阳性,但不符合SLE的诊断标准;所有报道的患者梅毒试验呈假阳性反应。必须注意的是抗磷脂综合征其中的一个并发症是自发性流产,多发生在妊娠中期。

抗磷脂抗体阳性的孕妇产后应仔细观察,早期发现胸膜、肺、心脏或血栓的症状或征象。如发生这种并发症时,应产后再测定抗磷脂抗体,如阳性,排除肺栓塞和感染后可给予免疫抑制治疗。

\subsection{三、Meigs综合征}

此综合征是Meigs于1937年首次描述,指良性实体卵巢肿瘤的患者伴有腹水和胸腔积液。而且,如卵巢肿瘤切除后,胸腹水应同时吸收。但是,此定义目前已经扩大到盆腔的其他疾病,包括良性囊性卵巢肿瘤、子宫良性肿瘤(纤维肌瘤)、无转移的恶性程度低的卵巢肿瘤和子宫腺肌瘤。因此,伴有胸腹水的盆腔肿瘤只要外科切除后胸腹水永远消失者可诊断为Meigs综合征。

腹水可能与原发肿瘤大量分泌液体有关,大的肿瘤分泌的液体更多。胸腔积液则可能是腹水通过膈肌的小孔进入胸膜腔,类似于肝硬化发生胸腹水的机制。因为:①胸腹水的性状一样;②胸腔穿刺抽液后胸水很快再发;③有些卵巢肿瘤的患者只有腹水没有胸水。也有认为腹水由淋巴管经膈肌进入胸腔而产生胸腔积液。

Meigs综合征患者以消瘦、胸腔积液、腹水和盆腔肿块为特征,与胸水有关的唯一症状是气短。胸腔积液发生在右侧70\%,左侧10\%,双侧20\%。胸腔积液蛋白多>30g/L,因此是渗出液。必须认识到的最重要的一点是并非有胸腹水的患者都有不能切除的盆腔恶性肿瘤,同时,有部分Meigs综合征患者血清肿瘤标志物CA-125升高,但并不是恶性肿瘤。

女性有盆腔肿块、胸腹水多认为是患有不能切除的肿瘤,因此,应在体液或活检标本中寻找恶性肿瘤的证据。最初的评价应包括胸腹水的细胞学,如果细胞学阴性,而且没有其他部位恶性肿瘤的依据,患者应接受手术探查。如果是良性肿瘤,可切除原发肿瘤。如手术后胸腹水吸收而且不再发生,则诊断可以成立。胸腹水一般在术后2周内吸收。

\subsection{四、其 他}

子宫内膜异位症曾有报道30\%有腹水和胸腔积液,其发生机制和Meigs综合征相似。积液多发生在右侧,呈血性或巧克力样,CA-125可升高。

月经期血气胸多伴有盆腔或腹腔子宫内膜异位症,右侧血气胸多发。其机制推测子宫内膜组织经膈肌进入胸腔,患者经期来临时,内膜组织脱落入胸膜腔从而产生血气胸。患者多为未产妇,平均年龄32岁。症状为呼吸困难或胸痛。

\protect\hypertarget{text00076.html}{}{}

\section{24 其他原因胸腔积液}

\subsection{一、胆固醇性胸膜炎}

胆固醇性胸腔积液是以胸腔积液中含有大量胆固醇为特点的一种慢性脓胸,是一种少见的慢性胸膜疾病。因病因及发病机制尚不清楚,其发生与结核病关系最大,此外有人认为也与糖尿病、梅毒,或二者合并结核时,以及慢性酒精中毒、肺吸虫病或肿瘤有关。发病以青壮年人较多,男性多于女性,国内报告24例中,男性占20例。右侧胸腔积液多于左侧,双侧受累较少见。临床症状较轻,多数表现为轻度咳嗽、胸闷、胸部钝痛,仅少数患者有午后低热、盗汗及乏力等结核中毒症状。X线胸片多示包裹性胸腔积液。也有以纤维板形成为主要表现。

积液吸收缓慢而有再发的可能性。积液无色、混浊或带血性,或呈乳白、淡黄、橙黄、黄绿、暗褐等不同色泽,而以黄白色较多见。比重大多在1.020~1.030之间。其突出的特点是积液中常混有浮动的鳞片状、绢丝状、带有光泽、折光性强的胆固醇结晶,静置后可见结晶沉积于管底。沉淀物镜检可见众多的板状、针状、斜方形结晶体,也可见红细胞、脂肪颗粒。有些病例积液沉淀涂片检查结核菌为阴性,但豚鼠接种试验可呈阳性。

本病的诊断依据为在胸腔积液中检出大量胆固醇结晶。胸腔积液中胆固醇含量>150mg/dl可诊断为本病。必须指出,胆固醇结晶存在与否,与积液中胆固醇总量高低无相应因果关系。

\subsection{二、乳糜性胸腔积液}

乳糜聚集于胸腔即称为乳糜性胸腔积液,外观呈乳状无味。乳糜性胸腔积液少见,常系胸导管直接破裂或阻塞引起淋巴液反流,使胸膜淋巴管扩张造成许多小瘘管所致,病因可归纳为:①良性病变,如炎症、寄生虫(丝虫病性肉芽肿)、免疫性疾病等;②肿瘤,如纵隔肿瘤、恶性淋巴瘤等;③胸导管外伤或手术损伤破裂;④特发性,如先天性淋巴管病;⑤肺淋巴管肌瘤病(LAM);⑥不明原因。

乳糜定性试验可诊断乳糜胸,但偶有假阳性;甘油三酯>1.11g/L为乳糜液,<0.5g/L可排除之;介于两者间若脂蛋白分析有乳糜微粒亦可确诊。胸腔积液找抗酸杆菌、癌细胞、微丝虫及细菌培养等可鉴别病因。参见第97节。

\subsection{三、血胸与血气胸}

血胸临床上多见于外伤。血性胸腔积液则多见于恶性肿瘤,有时也见于结核病、肺炎、结缔组织病与风湿热。风湿热所致的血性胸腔积液国内有个别病例报告。炭疽杆菌性胸膜炎时也出现血性胸腔积液。血气胸可见于胸壁外伤与自发性气胸。国内曾有一组7例血气胸报告,患者并无活动性肺结核病灶或外伤,只因大声讲话、背部入针过深、或无明显诱因而发生。白血病所致血性胸腔积液也有时见到。

血胸、血气胸与血性胸腔积液,须经胸部体格检查、X线胸部透视与诊断性胸腔穿刺而确诊。

\subsection{四、黄甲综合征}

黄甲综合征(yellow nails
syndrome,YNS)是一种罕见的疾病,其特征是甲色变黄伴淋巴水肿。1972年Hiler等提出黄甲、淋巴水肿、胸腔积液三联症中出现2个症状即可诊断黄甲综合征,淋巴管造影显示淋巴管发育异常或淋巴结缺如可明确诊断。到目前为止全世界报道了100多例,国内报道10余例。50\%以上病例为双侧胸腔积液,积液为淡黄色透明渗出液,蛋白含量常>40g/L,白细胞分类以淋巴细胞为主,胸水呈慢性进行性生长,胸水出现前常有呼吸道反复感染史。

\subsection{五、其 他}

有报道登革热、猫抓病、痛风等可引起胸腔积液。膈下脓肿、急性胰腺炎等有时可引起胸腔积液,国内也称反应性胸膜炎,积液一般为少量。急性胰腺炎可并发渗出性胸膜炎或腹膜炎,也有同时并发胸、腹膜炎者。胸(腹)水中淀粉酶含量增高,基本上与血清淀粉酶相平行。

目前由于介入技术和治疗的发展,医源性胸腔积液也时有报道,医源性原因包括药物、放射治疗、消化内镜检查和治疗(胃镜检查和食管静脉曲张硬化剂治疗)、支气管动脉栓塞术,卵巢过度刺激综合征、液体负荷过大、冠脉搭桥手术、中心静脉置管穿破和腹膜透析等,都可以引起胸腔积液,临床上应加以注意。

\protect\hypertarget{text00077.html}{}{}

\section{25 不明原因胸腔积液}

特发性胸膜炎

偶有胸腔积液患者即使经过全面的胸腔积液检查和胸腔镜活检,或开胸胸膜活检,临床上仍不能明确病因,而病理诊断为非特异性胸膜炎。综合国外数篇文献报道有194例胸腔积液患者经开胸手术或胸腔镜检查仍不明原因,经34~62个月的随访,其中有70例(36.6\%)仍未能明确原因,诊断为特发性胸膜炎。最近有文献报道在间皮瘤高发地区对142例接受了内科胸腔镜检查的患者进行了为期58个月的追踪观察,结果发现44例(31\%)的患者诊断为非特异性胸膜炎/胸膜纤维化,这些患者再经过9.8(±4.6)个月的随访后5例(12\%)诊断为恶性病变。仅26例患者未能发现病因。因此,大多数胸腔镜病理诊断为非特异性胸膜炎的患者经过密切随访可以找到病因,仅有小部分患者无明确病因。临床上称为真正的“特发性胸膜炎”,其病程呈良性过程,其特点如下:①一般情况稳定;②无体重减轻;③PPD阴性;④无发热(T<38℃);⑤胸腔积液中淋巴细胞不超过95\%;⑥积液量不超过胸腔的50\%。

\protect\hypertarget{text00078.html}{}{}

\section{参考文献}

1.何长清.关于胸水的鉴别诊断问题.中华结核和呼吸系疾病杂志,1980,3(1):59

2.高育瑶.恶性胸腔积液81例临床分析.中华结核和呼吸系疾病杂志,1983,6:211

3.李基业.溶菌酶对胸膜渗出液诊断的探讨.中华内科杂志,1984,23:392

4.陈继烈.原发性肺放线菌病三例.中华内科杂志,1959,7:188

5.邵令方.胸膜肺阿米巴病.中华医学杂志,1964,50:34

6.刘顾岗.胸膜间皮瘤30例报告.中华结核和呼吸系疾病杂志,1983,6:89

7.胡秀媛.米格氏综合征二例报告.中华妇产科杂志,1978,13:128

8.计威康,等.结缔组织疾病的肺脏表现(综述).中华结核和呼吸系疾病杂志,1980,3:191

9.容中生.自发性血气胸10例临床分析.新医学,1984,15:210

10.彭志军,等.结核性多浆膜腔积液25例临床分析.中华结核和呼吸杂志,1998,21(5):283

11.曾三如.猫抓热合并急性非特异性心包炎和胸膜炎一例.中华传染病杂志,1998,16(1):62

12.曾传生.肺吸虫病胸腔积液37例临床分析.中华传染病杂志,1993,11(2):118

13.杨志仁,等.胸腔积液端粒酶和癌胚抗原测定对良恶性胸腔积液诊断价值对比研究.中华结核和呼吸杂志,2001,24(4):201

14.黄毓东,等.肿瘤坏死因子、白细胞介素8、可溶性细胞间黏附分子1和CD11b/CD18在良恶性胸腔积液的价值.中华结核和呼吸杂志,2001,24(1):35

15.李嫣红,等.Th1/Th2免疫应答系统在结核性胸膜炎患者中的表达.中华结核和呼吸杂志,2004,27
(5):324

16.唐可京,等.多指标检测对结核性和恶性胸积液的鉴别诊断.中华结核和呼吸杂志,1999,22(8):508

17.刘运秋,等.糖链抗原153与癌胚抗原测定对良恶性胸腔积液的诊断价值.中华结核和呼吸杂志,2003,26
(8):508

18.徐峰,等.细胞角蛋白19mRNA定量检测在鉴别胸腔积液性质中的作用.中华结核和呼吸杂志,2003,26
(8):501

19.薛立福,等.胸腔镜术在内科的应用价值.中华结核和呼吸杂志,2001,24(4):198

20.崔姝梅,等.胸腔积液中间皮细胞在鉴别结核性和恶性胸腔积液的价值.中华结核和呼吸杂志,1998,21
(10):622

21.江涛,等.白念珠菌感染胸腔积液二例.中华结核和呼吸杂志,2001,24(11):654

22.沈庄夫,等.33例肺吸虫性胸腔积液临床分析.中华结核和呼吸杂志,1997,20(3):188

23.瞿跃进.恙虫病致肺损害31例临床分析.中华结核和呼吸杂志,2000,23(2):98

24.中华结核和呼吸杂志编辑委员会.胸膜间皮瘤310例综合分析.中华结核和呼吸杂志,1990,13:216

25.王仁本,等.恶性胸膜间皮瘤12例临床分析.中华结核和呼吸杂志,2001,24(8):454

26.陈应泰,等.恶性胸膜间皮瘤及其诊疗进展.中华胸心血管外科杂志,2004,20(4):253

27.胡红,等.肺淋巴管肌瘤病的临床分析.中华医学杂志,2001,81(20):1256

28.喻延,等.重度卵巢过度刺激综合征合并大量胸腔积液23例临床分析.中国医师进修杂志,2007,30(15):14-16.

29.徐伟健,等.26例非创伤性乳糜胸临床分析.中华结核和呼吸杂志,1999,22(12):754

30.王秋月,等.结节病伴双侧血性胸腔积液一例.中华结核和呼吸杂志,2001
24(2):110

31.刘昌起.胸膜疾病的病因和发病机制.中华结核和呼吸杂志,2001,24(1):15

32.Alkhawaldeh K,Biersack HJ,Henke A,et al.Impact of dual-time-point
F-18FDG PET/CT in the assessment of pleural effusion in patients with
non-small-cell lung cancer.Clin Nucl Med.2011;36:423-428.

33.Davies HE,Nicholson JE,Rahman NM,et al.Outcome of patients with
nonspecific pleuritis/fibrosis on thoracoscopic pleural biopsies.Eur J
Cardiothorac Surg.2010,38:472-477.

34.Romero-Candiera S,Fernandez C,Martin C,et al.Influence of
diuretics on the concentration of proteins another components of pleural
transudates in patients with heart failure.Am J Med 2001,110:681-686

35.Gotsman I,Fridlender Z,Meirovitz A,et al.The evaluation of pleural
effusions in patients with heart failure.Am J Med 2001,111:375-378

36.Heffner JE.Diagnosis and management of malignant pleural
effusions.Respirology.2008;13:5-20

37.Toaff JS,Metser U,Gottfried M,et al.Differentiation between
malignant and benign pleural effusion in patients with extra-pleural
primary malignancies:assessment with positron emission
tomography-computed tomography.Invest Radiol.2005;40:204-209

38.Khaleeq G,Musani AI.Emerging paradigms in the management of
malignant pleural effusions.Respir Med,2008;102:939-948

39.A.Sahn.Plueral fluid characteristics of paramalignant
effusion.Chest,2009,136:44S-45S

40.戴新建,等.胸水嗜酸性粒细胞增多症23例临床分析.杭州医学高等专科学校学报,2002,23:14-16.

41.洪澜,等.胆固醇性脓胸1例.四川医学,2008,29(8):1101.

42.崔喜梅,等.肺栓塞合并胸腔积液36例分析.热带医学杂志,2007,7:885-887

43.王丰,等.肺栓塞合并胸腔积液临床分析.右江民族医学院学报,2011,33:432-433.

\protect\hypertarget{text00079.html}{}{}

