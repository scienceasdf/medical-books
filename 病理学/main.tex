\PassOptionsToPackage{unicode=true}{hyperref} % options for packages loaded elsewhere
\PassOptionsToPackage{hyphens}{url}
%
\documentclass[
  12pt,
  twoside,
  banjiao]{ctexbook}
\usepackage{lmodern}
\usepackage{amssymb,amsmath}
\usepackage{ifxetex,ifluatex}
\ifnum 0\ifxetex 1\fi\ifluatex 1\fi=0 % if pdftex
  \usepackage[T1]{fontenc}
  \usepackage[utf8]{inputenc}
  \usepackage{textcomp} % provides euro and other symbols
\else % if luatex or xelatex
  %\usepackage{unicode-math}
 \defaultfontfeatures{Scale=MatchLowercase}
 \defaultfontfeatures[\rmfamily]{Ligatures=TeX,Scale=1}
\fi
% use upquote if available, for straight quotes in verbatim environments
\IfFileExists{upquote.sty}{\usepackage{upquote}}{}
\IfFileExists{microtype.sty}{% use microtype if available
  \usepackage[]{microtype}
  \UseMicrotypeSet[protrusion]{basicmath} % disable protrusion for tt fonts
}{}
\makeatletter
\@ifundefined{KOMAClassName}{% if non-KOMA class
  \IfFileExists{parskip.sty}{%
    \usepackage{parskip}
  }{% else
    \setlength{\parindent}{0pt}
    \setlength{\parskip}{6pt plus 2pt minus 1pt}}
}{% if KOMA class
  \KOMAoptions{parskip=half}}
\makeatother
\usepackage{xcolor}
\IfFileExists{xurl.sty}{\usepackage{xurl}}{} % add URL line breaks if available
\usepackage{bookmark}
\usepackage{hyperref}
\hypersetup{
  pdftitle={病理学},
  pdfauthor={github},
  pdfborder={0 0 0},
  breaklinks=true,
  bookmarksdepth=4}
  \PassOptionsToPackage{hyphens}{url}
\urlstyle{same}  % don't use monospace font for urls
\usepackage{graphicx,grffile}
\makeatletter
%\def\maxwidth{\ifdim\Gin@nat@width>\linewidth\linewidth\else\Gin@nat@width\fi}
%\def\maxheight{\ifdim\Gin@nat@height>\textheight\textheight\else\Gin@nat@height\fi}
\makeatother
% Scale images if necessary, so that they will not overflow the page
% margins by default, and it is still possible to overwrite the defaults
% using explicit options in \includegraphics[width, height, ...]{}
%\setkeys{Gin}{width=\maxwidth,height=\maxheight,keepaspectratio}
%\setlength{\emergencystretch}{3em}  % prevent overfull lines
\providecommand{\tightlist}{%
  \setlength{\itemsep}{0pt}\setlength{\parskip}{0pt}}
\setcounter{secnumdepth}{3}
% Redefines (sub)paragraphs to behave more like sections
\ifx\paragraph\undefined\else
  \let\oldparagraph\paragraph
  \renewcommand{\paragraph}[1]{\oldparagraph{#1}\mbox{}}
\fi
\ifx\subparagraph\undefined\else
  \let\oldsubparagraph\subparagraph
  \renewcommand{\subparagraph}[1]{\oldsubparagraph{#1}\mbox{}}
\fi

\usepackage{framed}
\usepackage{caption}
\usepackage{booktabs}
\usepackage{tabularx}
\usepackage{longtable}
\usepackage{tablefootnote}
\usepackage[version=4]{mhchem}
\renewcommand {\thetable} {\thechapter{}-\arabic{table}}
\renewcommand {\thefigure} {\thechapter{}-\arabic{figure}}
\newcommand{\chapterabstract}[1]{{\CJKfontspec{楷体} \small #1}}
\newcommand{\authorinfo}[1]{{\hspace*{\fill} #1}}
\ctexset{
  paragraph/runin=false
}


\newcommand{\mychapter}[1]{
    \chapter*{#1}
    \addcontentsline{toc}{chapter}{#1}
}

% set default figure placement to htbp
\makeatletter
\def\fps@figure{htbp}
\makeatother

\setCJKmainfont{思源宋体}
\setCJKfallbackfamilyfont{\CJKrmdefault}{宋体}
\setmainfont{思源宋体}
\usepackage[a4paper,top=1in, bottom=1in, left=0.8in, right=0.8in]{geometry}
\setlength{\parindent}{2em}
\setlength{\parskip}{0em}

\newfontfamily\apostrophefont[Ligatures=TeX]{Liberation Serif}
\XeTeXinterchartokenstate=1
\newXeTeXintercharclass \apostrophe

% Assign the new class to all Latin capital letters
\makeatletter
\@tempcnta=`'
\loop\unless\ifnum\@tempcnta>`'
  \XeTeXcharclass \@tempcnta \apostrophe
  \advance \@tempcnta by 1
\repeat
\makeatother

% Setup font change
\XeTeXinterchartoks 0 \apostrophe   = {\begingroup\apostrophefont}
\XeTeXinterchartoks \apostrophe 0   = {\endgroup}
\XeTeXinterchartoks 4095 \apostrophe = {\begingroup\apostrophefont}
\XeTeXinterchartoks \apostrophe 4095 = {\endgroup}




\title{病理学}
\author{github \\ 项目主页:\url{https://github.com/scienceasdf/medical-books}\\ 新书下载:\url{https://github.com/scienceasdf/medical-books/releases/latest}}


\begin{document}
\maketitle
{
\setcounter{tocdepth}{1}
\tableofcontents
\addcontentsline{toc}{chapter}{目录}
}

\mychapter{绪论}

\subsection*{一、病理学的概念和任务}

病理学(pathology)是研究疾病的病因、发病机制、病理变化(包括代谢、功能、形态结构)、转归、结局的一门医学基础学科。

疾病是在致病因子的作用下机体局部或全身所发生的代谢、功能和形态结构的变化。病理学的主要任务有二:一是应用科学的方法研究疾病发生、发展和转归的规律,从而阐明疾病的本质,为防治疾病提供理论依据;二是根据患病机体的病理形态学改变对疾病作出诊断。由此可见,病理学是医学中居于核心地位的学科。作为医生,要认识疾病防治疾病就必须有坚实的病理学基础。

\subsection*{二、病理学在医学中的作用}

病理学在医学教学体系中居于核心地位。人们把病理学形象地比喻为基础医学和临床医学之间的桥梁,是因为在医学教学体系中,解剖学、组织胚胎学、生理学、生物化学、病原生物学和免疫学等基础医学课程是让医学生了解和掌握正常人体形态结构、代谢、功能及其调节机制,而病理学的教学目的是引导医学生用上述基础知识来辨别患病机体所出现的各种病理现象并掌握其发展规律,为后续临床学科(主要阐述疾病的诊断、治疗和预防)的学习打下基础,其桥梁作用就体现在这里。因此,一名医学生只有很好地掌握病理学基础知识,才能学好临床各学科的课程。医学生毕业后诊治疾病的能力高低,也与病理知识掌握的好坏密切相关,只有那些对疾病的病因、发病机制、病理变化等有深刻理解、融会贯通的医学生才能成为高水平的临床医生。

病理学在临床诊疗中发挥至关重要的作用。尽管临床影像学、生化检验技术的发展突飞猛进,大大提高了疾病诊断的时效性和准确率,但活体组织的病理诊断仍然是临床诊断的金标准,而通过尸体解剖可对死因作出准确回答。尤其是分子病理诊断技术的进步,一大批肿瘤标志物、病原标志物的发现和检测为临床靶向治疗提供了重要靶点,从而推动了精准医学的发展。

病理学在疾病的研究中扮演重要角色。这主要体现在三个方面:一是病理工作者不仅从事教学和医疗工作,而且通过细胞培养、动物试验对疾病的发生发展机制和防治进行研究;二是从日常工作中积累的大量组织标本着手,开展旨在提高临床诊疗水平的系列研究工作;三是借助病理形态学知识和技术为其他学科开展的研究工作提供帮助。

\subsection*{三、如何学好病理学}

要学好病理学应该注意以下几个方面:首先必须有正常人体形态结构、代谢、功能及其调节机制的知识基础。其次是了解教科书的内容编排和学习要领。一般来说,病理学均包括总论和各论两部分内容,本书也不例外。总论阐述细胞和组织的适应、损伤与修复,局部血液循环障碍,炎症和肿瘤等基本病理变化,也就是疾病的普遍规律(共性)。各论,如循环系统疾病、消化系统疾病等章节则介绍各系统常见疾病的原因、发病机制、病变及其发生、发展的特殊规律(个性)。学好总论是学习各论的必要基础,学习各论必须联系、运用总论的知识,两者之间有着密切的内在联系。第三,学会运用病理知识去解释临床现象,可以起到巩固病理知识的作用。第四,及时地复习和总结也很重要。另外,病理学网络教学资源很丰富,在条件许可情况下,应该积极利用。

\subsection*{四、病理学的研究方法}

病理学的研究方法很多,有些是经典的,还有些是近30年发展起来的,概括起来,主要有以下几种:

\subsubsection*{尸体剖验}

尸体剖验(简称尸检或尸解,autopsy)可以直接观察疾病的病理改变,进行详细的组织学检查,结合临床资料明确诊断,查明死因,总结经验,提高临床医疗水平。此为病理学基本研究方法之一。通过大量尸体剖验资料的积累,不仅可以研究疾病发生、发展的规律,而且还能及时发现和确诊某些新的传染病、地方病、流行病,为防治疾病提供依据。尸体剖验积累的标本,也为培养医务工作者提供了大量有价值的资料。由于人们观念陈旧以及相关法规不健全,所以我国尸检率很低,对病理学及医学的发展极为不利。

\subsubsection*{活组织和细胞学检查}

从患者活体采取组织进行病理检查,以确定诊断,称为活组织检查(简称活检,biopsy)。这是临床广泛开展的病理检查方法。目前采取病变组织的方法有钳取、切取、穿刺、局部切除等。这些方法的优点是组织新鲜,不仅可供常规病理诊断,而且可用于各种细胞化学、组织化学、超微结构以及分子病理学研究。在临床上,活检对判断病变性质、确定治疗方案有重要意义。对性质不明的肿瘤,还可在手术过程中,切取病变组织作冰冻切片或快速石蜡切片,迅速确定肿瘤的良性、恶性,决定手术范围,这对肿瘤病人的治疗和预后尤为重要。

细胞学(cytology)检查又称脱落细胞学,是指采集病变处脱落细胞或细针吸取的细胞,涂片染色后进行诊断。其具有方法简便、创伤小、可重复等优点,适于对人群进行大规模普查。但由于没有组织结构、细胞常有变性,所以易出现假阴性结果,有时需结合其他检查结果综合判断。

\subsubsection*{动物试验}

在适宜动物体内复制某些人类疾病的模型,以了解该疾病或某一病理过程的发生、发展。

这种方法也可用以研究疾病的病因、发病机制及进行药物治疗试验,观察疗效及药物不良反应。但要注意,动物和人类之间存在种系差异,不能将动物实验的结果直接套用于人类。

\subsubsection*{组织培养与细胞培养}

将人体或动物的某种组织或细胞在体外培养,以观察组织或细胞病变的发生、发展,也可观察药物等外来因子对培养细胞的影响。这种方法周期短、见效快,但要注意体内和体外有差异,孤立的体外环境缺少体内存在的整体环境中众多因素间的相互影响,因此不能将体外研究结果与体内病变过程等同对待。

\subsubsection*{其他技术在病理研究中的应用}

除上述四种基本研究方法外,随着生物学和相关学科的发展,近数十年特别是近二十年来高新技术也相继进入病理领域,目前在病理研究中应用较多的有组织和细胞化学、免疫组织化学、电子显微镜技术。①组织和细胞化学:组织和细胞化学法是应用某些化学试剂,在组织及细胞上进行特异性化学反应,呈现出特异的颜色,从而了解和鉴定组织细胞中的各种蛋白质、脂类、糖、酶和核酸等化学成分的状况。②免疫组织化学法:是应用酶标抗体(或抗原)和相应的抗原(或抗体)接触,形成特异性抗原抗体复合物,催化底物后,可呈现颜色变化,在原位检测组织细胞内的抗原或抗体的技术。组织细胞中凡是能作为抗原或半抗原的物质,如蛋白质、多肽、氨基酸、多糖、磷脂、受体、酶、激素及病原体等都可用相应的特异性抗体进行检测。③电子显微镜技术(简称电镜):应用透射电镜或扫描电镜对细胞内部和表面的超微结构进行更细微的观察,即从亚细胞(细胞器)水平上认识和了解细胞的病变。除上述三种常用方法外,进入病理领域的其他高新技术还有流式细胞仪技术、图像分析技术、分子原位杂交、聚合酶链反应(PCR)、共聚焦显微镜技术、组织芯片技术、二代测序技术等,使病理学对疾病的研究从定性进入定量,从细胞水平进入分子水平,并使形态结构和代谢、机能的研究联系起来,其结果不仅加深了对疾病的理解和认识,又推动了病理学发展。

\subsection*{五、病理学发展简史}

在我国,早在公元前700年,《黄帝内经》中就有以阴阳、脏腑和经络之间功能失调作为疾病病因的论述,这是源于当时阴阳五行(金、木、水、火、土)的哲学思想,这种思想延续至今仍然是中医诊治疾病的理论基础。隋唐时代巢元方所著的《诸病源候论》对疾病的病因和征候的记载十分详细,可以说他是我国古代第一个病理学家。但是由于中西医研究疾病的角度不同,所以祖国医学中的病理学和现代病理学分属不同的理论体系,前者与古希腊名医Hippocrates建立的体液病理学相似。

现代病理学的建立源于尸体解剖,其发展与人们对疾病的认识息息相关,特别是与基础医学学科的发展和技术进步有密切联系,主要分为三个阶段:

初期即器官病理学阶段。这一阶段是建立在尸解的基础上的。该时期最有影响的代表人物是意大利名医Morgagni(1682---1771)。他根据700例尸解肉眼观察材料,结合临床资料,对照分析,著成《疾病的位置与原因》一书,提出了疾病的器官定位的观点,为病理学的发展奠定了基础。此后Rokitansky在掌握了大量尸解资料的基础上,于1843年完成了《病理解剖学》巨著,丰富了器官病理学的内容,但病理学向广度和深度发展主要得益于其他科学技术的发展。

中期即细胞病理学阶段。19世纪中叶,德国病理学家Rudolf
Virchow(1821-1902)在进行大量尸检的同时,借助显微镜对尸检材料进行观察研究,于1858年出版了著名的《细胞病理学》一书,提出了细胞形态和功能的变化是疾病的基础的观点,使病理学从器官的模糊阶段进入细胞的微观水平。他对近代病理学的发展作出了卓越的贡献,而且也为所有医学基础学科的建立和发展奠定了基础。

繁荣阶段即现代病理学阶段。20世纪40年代末期电子显微镜应用于医学生物领域,60年代开展免疫组织化学的研究,70年代分子生物学崛起,研究方法和技术日趋进步,使病理学取得突破性的进展,不仅极大丰富了细胞病理学的内容,而且使病理学从经典的形态学范畴进入亚分子和分子水平。随着研究内容的拓宽与深入,在病理学范畴内又出现了新的病理学科分支。从临床医学上分出了外科病理学、妇产科病理学、儿科病理学、神经病理学、皮肤病理学、眼科病理学、耳鼻喉病理学。随着边缘学科的兴起及研究方法的互相渗透,又出现了超微病理学、免疫病理学、实验病理学、定量病理学、遗传病理学及分子病理学等。

精准病理学阶段已经到来。为了在治疗过程中尽可能减少对正常组织的损伤,近年提出了精准医学的概念,介入治疗、靶向治疗即属此范畴。要真正实现精准治疗,必须有可靠靶标。病理科开展乳腺癌激素受体、癌基因HER-2/neu表达的检测,以及其他肿瘤的标志物检测,即可提供上述靶标,从而指导临床采用内分泌治疗、单克隆抗体靶向治疗等。可以说,病理学已进入了一个崭新的发展阶段。

20世纪初,现代病理学传入我国,经过几代病理学家的艰苦努力,造就了一大批优秀的病理学人才,积累了具有我国疾病特点的病理资料,编写了富有特色的病理教材。当前摆在新一代病理工作者面前的任务是不仅要继承老一辈病理学家的研究方法,还要将生命科学研究中的新方法、新技术用于病理学的研究中,承前启后,继往开来,为赶上世界先进水平、发展我国病理事业作出贡献。
\chapter{总论}

\section{CT机的基本构造和原理}

1895年11月8日,德国著名物理学家威·康·伦琴(W.K.Roentgen)在一次阴极真空射线管放电实验中偶然发现了X线,它不仅是对物理学的巨大贡献,也为放射诊断学的创立和发展奠定了基础。100多年来放射诊断学获得了迅猛的发展。1969年英国物理学家G.N.Hounsfield利用人体对X线的吸收原理,结合计算机的图像重建和处理功能设计了计算机断层扫描机(computed
tomography,简称CT),这一成果于1972年向全世界宣告。这种图像质量好、诊断价值高的成像方法,使放射诊断学发生了重大突破,是对现代医学的卓越贡献。为此,Hounsfield获得了1979年诺贝尔医学生物奖。

\subsection{CT的发展简史}

1967年CT的基本组成部分即重建数学、计算机技术和X线探测器都已具备。那时,Hounsfield在EMI实验研究中心,从事图像识别和利用计算机存储手写字技术的研究。他证实了有可能采用一种与直接电视光栅方式不同的另一种存储方法,这种方法使信息检索更为有效。

首先,有人提议从三维物体的各个方向取读数。但后来的断层方法似乎更适用于图像重建和诊断。它意味着仅需从单一平面里获取透射的读数。因此,每个光束通路都可以看作联立方程中的许多方程之一,必须解这组联立方程才能获得该平面的图像。此原理用数学模拟法加以研究。然后经过反复实验,并用X线进行人脑组织标本扫描研究,于1969年Hounsfield设计成功CT机。第一个原型设备于1971年9月安装在Atkinson
Morley医院,1971年10月4日检查了第一个病人。在1972年4月英国放射学家研究年会上宣告EMI扫描机诞生了,接着在同年11月芝加哥的北美放射学会(RSNA)年会上向全世界宣布。1973年在英国放射学杂志上报道。1974年美国George
Town医学中心工程师Ledley设计了全身CT机,从此CT告别了只限于头颅检查的时代。1985年开发了滑环技术,1989年单方向连续螺旋型CT即螺旋CT的问世,是滑环技术的体现,是CT发展的重大突破。1991年开发了亚毫米扫描和双螺旋CT。1998年多层螺旋CT机问世,SIEMENS、GE、PICKER、TOSHIBA公司都相继生产。1983年超高速CT(ultrafast
CT,UFCT)又称电子束CT(electron beam technology
CT,EBCT)由美国Imatron公司率先研制成功,并于1993年推向市场。EBCT进一步开拓了CT的应用范围,例如心脏功能和形态学研究,心、脑、肾、冠状动脉的血流量测定等。2004年GE公司率先推出64层(64排探测器)螺旋CT,并在北美放射学会年会发布这一信息。此后SIEMENS公司亦研制出64层螺旋CT(探测器为40排,机架每旋转1周利用中间的32排探测器即可获得64层图像)。在2005年北美放射学会上,SIEMENS公司又推出了双源CT(SOMATOM
Definition),SOMATOM
Definition基于西门子64层螺旋CT的成熟技术,配备了两个同步旋转的X射线源、探测器,每组X射线源、探测器组合只需转动90\textsuperscript{o}
就可以获得质量很好的心脏图像;基于0.33s的机架旋转时间,它可获得83ms的时间分辨率,使心脏成像不受心率影响。

\subsection{CT机的基本结构和成像原理}

\subsubsection{基本结构}

1.X线发生系统:高压发生器、X线球管、冷却系统、前准直器(去除散射线,使X线呈束状排列)。

2.X线探测部分:探测器、模数转换器(将探测器形成的电信号转换成数字信号,输入计算机)、后准直器(去除被照物体后的散射线)。探测器分为气体和固体两大类。固体探测器由闪烁晶体构成,有碘化钠、碘化铯、钨酸镉、锗酸铋等;气体探测器采用气体电离室的原理,一般多用氙气。固体探测器灵敏度较高,但其几何利用率较低,而气体探测器则与之相反。

3.支架部分:扫描架和检查床。

4.计算机系统:第3、第4代CT机包括阵列处理机(图像处理计算机)和主计算机(中央处理系统)。

5.图像贮存、显示和记录部分:主要指磁盘(硬盘)、光盘、磁带或软盘、显示器和照相机等。

6.操作控制部分:主要指操作台的键盘。

\subsubsection{成像原理}

CT的成像原理与普通X线相仿,只是图像的载体用探测器代替了胶片或荧光屏。CT扫描时用高度准直的X线束扫描人体的某部位,并围绕该部位做360°匀速转动,穿过人体的X线再经过准直后,由探测器接受。探测器接受的大量信息经模数(A/D)转换器将模拟量转换成数字输入计算机,计算机计算出该断面上各单位体积的X线吸收值(CT值)并排列成数字矩阵。数字矩阵再经数模转换(D/A),用灰白不同的灰度等级在监视器荧屏上显示,就获得了该部位的横断解剖结构图像。不同密度的组织对X线的吸收量不同。探测器分辨X线量的敏感程度较X线透视和X线摄影高的多,其对组织的密度分辨力较常规X线检查约高10~30倍。

\subsection{CT机的分代}

CT机的发展通常以“代”来划分,主要依据X线球管和探测器的关系、探测器的数目、排列方式和两者的运动方式来划分。其实“代”并不完全反映CT机的性能优劣,即并非代数越高CT机性能越好,如第3、第4代CT各有其优点。

1.第1代CT机:X线为单射束,单个或数个探测器,运动方式为平移加旋转,扫描一层需数分钟,只能限于头颅扫描。

2.第2代CT机:它与第1代无质的区别。X线为小角度(3°~30°)扇形X线束,探测器从数个至几十个。运动方式仍为平移加旋转,扫描时间缩短至18秒左右。虽已扩大至全身应用,但运动伪影很明显,故实际仍限于头颅扫描。

3.第3代CT机:X线为角度较大(30°~45°)的扇形X线束,探测器也相应呈扇形排列,数目多达数百个。运动方式为旋转式,扫描时间一般为2~5秒,最快可达1秒,使CT检查应用于全身。滑环技术及随之应用的螺旋CT是第3代机型的重大突破。

4.第4代CT机:与第3代基本相同。探测器排列呈圆周状,固定在扫描架四周,仅为X线球管旋转。实际为第3代的变型,并无明显优越性,仅有少数厂家生产。

5.第5代CT机:即超高速CT机(UFCT),又称电子束CT(EBCT),与以前的CT机已有根本区别。采用电子枪结构,使每次扫描时间可缩短至30ms,大大有利于心脏CT扫描。

EBCT即UFCT,主要由电子枪、聚集线圈、偏转线圈、8排探测器群、台面高速移动的检查床和控制系统组成。采用电子枪发射电子束,经聚焦后由偏转线圈控制,使电子线旋转,并轰击四个平行的钨靶环,从而获得旋转的X线源,再采用8排探测器群来收集扫描数据。目前,扫描速度可达50ms。由于有4个靶环,一次可进行4次扫描,最快每秒可扫描24次,故对心脏、冠状动脉等心血管的研究有特殊的作用。它的优点是:①扩大了影像诊断范围;②提高了图像质量(减轻了运动伪影);③减少了造影剂用量,并提高了高峰显影质量;④增加了单位时间的检查人数;⑤可做血流量、血流速度和弥散等功能检查。

\section{CT的应用和进展}

\subsection{CT图像重建的常用数学演算方式}

通常使用的演算方式有:标准演算法(standard
resolution)、边缘演算法(edge resolution)和平滑演算法(smooth
resolution)等。可根据受检部位的组织成分及密度差异,选择合适的数学演算方式。标准算法适于分辨率要求一般的普通CT图像重建如头颅等。软组织演算法对密度差异很近的组织分辨率较好,常用于腹部脏器的检查。骨密度演算法的图像分辨率最佳,可以分辨密度差异很大的组织,适用于观察骨质及内耳、乳突等,也可用于肺部小病灶的高分辨率CT检查(HRCT)。

\subsection{影响CT成像的因素}

1.窗宽和窗位:形成CT图像的数字矩阵都是CT值,即组织密度的代表。空气的CT值约为-1000Hu,骨皮质的CT值约为1000Hu。而人眼大约能分辨16个灰阶。如果某幅图像内既含有空气,亦有骨皮质,则上下CT值范围约2000Hu差值,每个灰阶所包含的CT值范围为125Hu。那么,CT值相差在125Hu内的组织结构显示为同一灰阶(即同一灰度)而不能在CT图像上各自显示。所以就要在观察某幅图像或观察某部位的组织结构时,选择合适的CT值范围和该范围的中点,来观察或显示某幅图像,这一CT值范围即窗宽,其中点即为窗位。如观察脑组织窗宽为100Hu,窗位为35Hu;肺部窗宽为1000~1500Hu,窗位为-700Hu左右;纵隔窗宽为200~300Hu,窗位30~50Hu。

2.伪影与噪声:①伪影:在扫描中由于某种因素的影响而产生的被检物体不存在的假象。有机器因素造成的环状伪影、条状伪影、点状伪影等;亦有因人体内密度差异(例如骨骼、手术金属物、胃肠内钡剂)造成的伪影;还有运动性(如胃肠蠕动、呼吸、病人身体移动)伪影。扫描条件不当亦造成伪影。②噪声:分为光子噪声和组织噪声。前者亦称扫描噪声,即X线穿透人体后到达检测器的光子量有限,其在矩阵内各图像点(像素)上的分布不是绝对均匀所造成,以致均质组织或水在各图像点上的CT值不是相等的,为减少噪声需增加X线量。组织噪声为各种组织(如脂肪和脑组织)的平均CT值变异所致,即同一组织CT值常在一定范围内变化,以致不同组织可具有同一CT值。

3.部分容积效应:像素代表一个体积,此体积内可能含有各种不同组织。其CT值实际代表的是单位体积内各种组织CT值的平均数。例如骨骼和气体加在一起可以类似肌肉密度(CT值)。因此在较高密度区域中间的较小低密度灶的CT值常偏高,反之亦然。

4.空间分辨率与密度分辨率:前者是指影像中能显示的最小细节,后者是指能显示的最小密度差别。

\subsection{CT的检查方法}

\subsubsection{常用检查方法}

常用的CT检查方法有平扫描、增强扫描、动态CT扫描、靶扫描(亦称目标CT扫描、放大CT扫描)、高分辨率CT技术(HRCT)。

用于肝脏的特殊增强扫描即所谓的肝脏CT血管造影有两种方式,包括肝动脉插管的动脉造影CT(CTA)和经脾动脉或肠系膜上动脉注入造影剂的门静脉期扫描,又称经动脉门脉血管造影CT(CTAP)。

\subsubsection{特殊检查方法}

1.CT透视与实时螺旋CT扫描:其原理相同,即在接近0.6秒的延迟时间后CT图像以6帧/秒速度显示,能达到实时观察的目的。CT透视主要用于CT介入穿刺;实时螺旋扫描能在扫描期间评价增强程度、选择扫描时机等。

2.CT血管造影术:或称CT血管成像术(CT
angiography,CTA)是螺旋CT三维(3D)重建技术的应用结果,主要用于颈动脉、颅内动脉、胸主动脉、腹主动脉、髂动脉、肺动脉、肾动脉、肠系膜动脉及内脏静脉(如门静脉)成像。CT仿真内镜如CT胆管成像、泌尿系尿路成像等亦是螺旋CT三维重建的应用体现。

3.仿真内镜术(Virtual
endoscopy,VE):是将CT或MR获得的原始容积数据与计算机三维图像技术相结合,借助导航技术(navigation)或漫游技术(flythrough)以及伪彩技术来逼真的模拟腔道内镜检查的一种方法。于1994年Vining等首次报道应用于结肠CTVE。

目前CTVE主要用于:①气管和支气管;②鼻咽腔、鼻窦、喉和中耳;③胃和结肠;④大血管;⑤胆道;⑥肾盂、输尿管和膀胱;⑦脑室和椎管;⑧关节腔等。但它存在着不能显示病变的颜色、不能显示腔内扁平病变、定性能力差等缺点。目前虽处于初步认识阶段,但值得进一步深入研究。

4.CT灌注成像(CT perfusion
imaging):是指静脉注射造影剂的同时对选定的层面进行连续多次扫描,以获得该层面内每一像素的时间-密度曲线。根据该曲线利用不同的数学模型计算出血流量、血容量、对比剂的平均通过时间、造影剂峰值时间等参数,以此评价组织器官的灌注状态。它反映的是生理功能的改变,因此是一种功能成像,可用于了解脑、肝、肾、胰腺、心脏的灌注状态。灌注参数还能较准确的反映头颈部、肝、肾、胰腺和肺等部位的肿瘤内血管变化和血液动力学改变,对肿瘤的诊断及恶性肿瘤的分级有重要意义,且为治疗方案的选择提供有价值的信息。

5.CT定量骨密度测定:定量骨密度测定为骨矿物质含量测定的重要方法。其方法很多,如单光子吸收法和双光子吸收法等,其中以CT双效能定量测量法(定量CT法)比较可靠。

\subsection{常用的螺旋CT三维重建技术}

常用的三维重建技术有:多平面重建法(multi-planar
reformation,MPR)、最大强度投影法(maximum intensity
projection,MIP)、最小强度投影法(minimum intensity
projection,MinIP)、遮蔽表面显示法(shaded surface
display,SSD)、容积再现法(volume rendering,VR)以及曲面重建法(curved
planar reformation,CPR)、透明重建(Ray sum)等。

\subsection{应用和副反应}

\subsubsection{药理}

CT增强所用的造影剂主要为经肾脏排泄的含碘造影剂,但也有用硫酸钡制剂和胆道造影剂者。钡剂用于胃肠道检查;经肝脏排泄的胆道造影剂(包括口服和静脉用药)只用于胆道增强。

目前,CT检查使用的经肾脏排泄的造影剂多为水溶性造影剂,且均为三碘苯环的衍生物(图\ref{fig1-1})。根据其结构分为4型:①离子型单体:常用的有复方泛影葡胺、安其格纳芬(Angiografin);②离子型双聚体:常用的有碘卡明;③非离子型单体:常用的有优维显(碘普罗胺)、欧乃派克(碘苯六醇)、碘必乐(碘异酞醇)等;④非离子型双聚体:常用的有伊索显(碘曲仑)。

\begin{figure}[!htbp]
 \centering
 \includegraphics[width=.7\textwidth,height=\textheight,keepaspectratio]{./images/Image00002.jpg}
 \captionsetup{justification=centering}
 \caption{三碘苯环的基本分子式}
 \label{fig1-1}
  \end{figure} 

离子型者苯环上1位侧链为羧基盐(---COOR),具此结构的造影剂水溶性高,在水溶液中可离解成阴离子(含有三碘的苯环)及阳离子(葡甲胺、钠、钙、镁)。非离子型者苯环上1位侧链为酰胺衍生物(---CONH),其水溶性很高,但不离解于水中。单体造影剂是指一分子造影剂含有一个三碘化苯环,双聚体则有两个三碘化苯环。

经肾脏排泄造影剂的临床应用主要受下列因素影响:①碘浓度:与造影剂的增强效果有关。根据其浓度可分为4类:特高浓度(400mg/ml)、高浓度(350~400mg/ml)、中浓度(280~320mg/ml)、低浓度(80~240mg/ml)造影剂。特高浓度偶用于心脏、大血管造影和经静脉注射的动脉造影。中、高浓度造影剂尤其高浓度造影剂适用于快速静脉注射后的CT动态扫描。CT脑室或CT脊髓造影适用低浓度造影剂。②渗透压:高渗造影剂的副作用高于低渗造影剂;离子型渗透压高于非离子型;单体造影剂的渗透压高于双聚体者。③黏度:分子大、浓度高、温度低者黏度高。黏度越高给大剂量快速注射带来困难,且易形成微小血管的阻塞而引起局部的缺血缺氧。

\subsubsection{给药方式}

理想造影剂应具备以下条件:①显影清楚;②无毒、副作用;③易于吸收和排出;④使用方便;⑤性质稳定,易储存;⑥价格低廉。

除离子型造影剂碘卡明和非离子型造影剂优维显(碘普罗胺)、欧乃派克(碘苯六醇)、碘必乐(碘异酞醇)、伊索显(碘曲仑)可用于脑室及脊髓造影外,其他肾脏排泄造影剂禁用于脑室和椎管造影。因为这类造影剂容易进入蛛网膜下腔,可损害血脑屏障,引起抽搐及至死亡。

给药方式和用药量如下:

1.静脉团注法:亦称快速注射法。将某一剂量的高浓度造影剂加压快速注入静脉,在造影剂经血循环大量进入靶器官的供血动脉时开始CT扫描。这种方法可提供CT所需的高质量增强情况,现已成为常规增强方式。一般情况下造影剂用量为1.5~2ml/kg体重(成人一般注入80~100ml),注射流率为1~8ml/s。

2.静脉滴注法:以20~30ml/min的速度注入含碘量约为300mg/ml的造影剂100ml后,再行CT扫描的方法。

3.动脉注射给药法:主要用于肝脏肿瘤的诊断,即选择性注入肝动脉、脾动脉及肠系膜上动脉的CTA和CTAP扫描术,以0.7~1ml/s的流率注入70~100ml。

4.胆系造影增强:30ml胆影葡胺缓慢注射(大于5分钟)或100ml胆影葡胺静脉滴注给药(快速团注易引起严重副反应)。给药后30~60分钟达最佳强化。

5.蛛网膜下腔给药:由腰穿注入水溶性碘造影剂后做脊髓或脑室扫描。椎管造影浓度一般为200~300mg/ml,剂量10~15ml。脑室造影浓度为150
mg/ml,剂量5ml。

6.胃肠充盈造影:常用2%的碘水造影剂,用量无统一标准。①胃十二指肠检查前20~30分钟服500~800ml,上床前再服200~300ml。②小肠检查前2~3小时服800~1000ml以充盈结肠,检查前1~2小时服300~500ml以充盈小肠远端,检查前15~30分钟再服300~500ml充盈胃及小肠近端。③结肠检查一般灌肠注入1500~1800ml。

\subsubsection{含碘造影剂的副反应}

一般根据反应的轻重和需治疗的程度进行分类(见表\ref{tab1-1})。离子型和非离子型造影剂副反应发生率有明显差异,前者约为5%,后者约为1.3%,但后者重度反应明显少,约为0.01%
。所以对有肝、肾、心疾病、糖尿病、虚弱、恶病质和过敏体质者等高危人群尽可能选用非离子型造影剂。离子型和非离子型造影剂对肝肾功能的影响区别不大。

\begin{table}[htbp]
\centering
\caption{造影剂副反应的分类}
\label{tab1-1}
\includegraphics[width=\textwidth,height=\textheight,keepaspectratio]{./images/Image00003.jpg}
\end{table}

\subsection{CT的发展方向}

目前推出基于CT的肿瘤放疗系统即CT模拟定位系统,其软件、硬件专门为CT模拟设计制造,还配有立体定位介入引导系统,可帮助医师模拟和介入(活检或引流等),并配有组织间近距离放疗的CT立体定位机械手臂系统。

介入性CT可用于脑、肺、纵隔、肝、胰、肾、肾上腺、腹膜后淋巴结、盆腔肿块的穿刺活检及肿瘤治疗,亦可用于骨骼肌肉的穿刺活检。对胸腹部脓肿、腹部和盆腔囊肿(如肝、肾囊肿)进行穿刺引流、硬化治疗,其定位准确性更高。尤其对颅内血肿和脓肿的穿刺抽吸引流更具独到之处。在颈臂神经丛和腹腔神经丛神经阻断术中是其他影像学手段所不及。CT亦可用于植物神经阻断术。

\section{常用的CT技术术语}

\subsection{平扫描和增强扫描}

扫描(scan):CT机扫描架内的X线球管围绕人体旋转进行X线照射,探测器接收到衰减程度不同的X线,转换成电信号,输入计算机重建成图像,每旋转一次的动作称为扫描。

平扫描(simple
scan):不向血管内注射造影剂的一般扫描程序称为平扫。检查腹部虽然多口服造影剂以充盈胃肠,但仍叫平扫。

增强扫描(contrast enhancement
scan):即造影增强,以CE或+C表示。应用碘水造影剂注射入静脉或动脉内,使心血管、组织器官或病灶密度增加,有利于对病变或正常组织器官的显示。

\subsection{快速连续扫描和延迟扫描}

动态扫描(dynamic
scan):按设定的部位,自扫描起始位到终止位,自动地进行逐层扫描,扫描后自动处理并显示图像,可分为动床式和同层动态扫描。此种方法主要用于增强扫描。

快速连续扫描(fast continuous
scan):对感兴趣的某区,自动地进行多次快速扫描,了解器官功能活动情况、造影剂充盈与排泄情况,显示同一层次在不同时间的变化。实际属于同层动态扫描。

延迟扫描(delay
scan):部分病例需要在团注增强扫描结束后一段时间内再做病灶区域或整个脏器扫描。如疑肝血管瘤,可在团注造影剂后5~15分钟再做局部扫描,以观察病灶是否被造影剂充填。

\subsection{间隔扫描和重叠扫描}

薄层扫描(thin slice
scan):一般指层厚≤5mm的扫描。主要为了减少部分容积效应而进行此扫描。

间隔扫描(interval
scan):亦称间断扫描,即层距大于层厚的扫描。其扫描不是连续扫描,可以按一定间隔进行隔层扫描,减少了扫描层数。如层厚为10mm,层距为12mm、15mm、20mm,则为间隔扫描。

重叠扫描(overlap
scan):层距小于层厚的扫描。如层厚为5mm,而层距(间隔)为3mm,则为重叠扫描。

\subsection{定位扫描}

定位扫描(scan ogram)又称为topgraph或scout
view。即在X线球管固定时扫描出一幅人体正位或侧位像,用以做出扫描层次、方向、层距、间隔及扫描次数等计划。定位扫描像有时可代替普通X线片,供诊断参考。

\subsection{靶CT扫描}

靶CT扫描(target scan)亦称目标扫描(object scan)、放大扫描(magnify
scan),是针对某一感兴趣区做局部的CT扫描,即应用小显示野(DFOV)扫描。由于被显示的范围小而矩阵不变,在一定单位体积的区域内像素相对增多,故可明显提高空间分辨率。也可以在普通CT扫描结束后,利用收集的原始扫描数据做局部的靶图像重建。后一种方法因有扫描数据保存,故可做各种部位、大小和成像方式的图像靶重建。靶扫描或靶重建与CT图像的单纯放大不同。后者仅是把图像的某部分放大,并无从根本上改变像素的大小和成像方式,所以其分辨率未提高,其清晰度反而下降。

\subsection{螺旋扫描}

螺旋扫描(spiral scan or helical
scan)是建立在滑环技术上,是在一次数据扫描过程中X线管和探测器不停地向一个方向旋转(第4代CT机只是X线管旋转),检查床亦同时向前推进,整个扫描的轨迹呈螺旋形。故螺旋扫描采集的数据是某一器官的容积数据,因此重建时可以采用任意的重建距离来进行重建而获得相应的图像幅数。

在扫描过程中X线管每旋转一周,检查床推进的距离不一定要和层厚相等,检查床推进距离可以等于、大于或小于层厚。床推进距离和层厚之比称为螺距指数(pitch
index)简称螺距。螺距指数=床推进距离/层厚。床推进距离和层厚一致时螺距为1,床推进距离大于层厚时螺距>1,反之则<1。

\subsection{图像的重建和重组}

重建(reconstraction):是利用图像的原始数据来进行处理所形成的图像。

重组(refomatting):是利用已经形成的图像进行重新组合,如用来形成冠状面、矢状面、多平面(MPR)、三维(3D)图像。故重组与重建两者含义是有区别的,但多习惯于将重组亦称为重建。

\subsection{高分辨率CT技术}

高分辨率CT(hight resolntion
CT,HRCT)技术,即利用CT机具有的特殊软件,专为显示肺部弥漫性间质性病变以及结节病变等细微结构的重建方法,可使空间分辨率显著提高。一般采用1~2mm的薄层扫描,故亦可称为薄层高分辨率CT。实际上多采用骨密度演算法重建,所以也适用于观察骨质情况及内耳、中耳、鼻窦、眼眶等结构。

\subsection{窗功能和双窗}

窗宽(window
width):以W.W或W表示,即在观察某幅图像时所选择的CT值范围。观察不同的组织器官应选择合适的窗宽。

窗位(window level或window
center):以W.L、L或C表示,即所选择窗宽的CT值范围的中心值。

窗功能(window
function):即在观察某幅图像时,通过窗宽及窗位的调节,使所需观察的组织、器官及病变清楚显示称为窗功能。

双窗(dual
window):即为显示不同组织器官使用双窗位及双窗宽,具体数字在监视器或CT片上分别显示。例如胸部双窗显示,可在显示纵隔图像的同时,也显示肺的细节,有利于观察肺部病变与纵隔的关系。

\subsection{CT值及其换算公式}

CT值是指X线穿过人体后,探测器检测并换算出的组织、器官的衰减值,它所反应的是组织、器官的密度。其换算公式如下:

CT值=(μ组织-μ水)/μ水×α

μ组织为人体组织的吸收系数,μ水为水的吸收系数,α为分度因数。在换算时将水的吸收系数调节为1,空气的吸收系数为0,μ组织是相对水和空气而言的。α目前均采用Hunsfield的单位,其分度因数为1000,故水的CT值为0Hu,空气约为-1000Hu,骨的吸收系数最高可达水的两倍,即μ骨为2,故其CT值可高达1000Hu。最早的CT机采用EMI单位,其分度因数为500,故其CT值是Hu单位的一半,如空气为-500EMI单位。

\subsection{感兴趣区}

感兴趣区(region of
interest,ROI)即我们对图像某部分进行CT值测量分析的区域,其中有3项指示在画面上。

1.平均值(mean,m):即ROI内的CT值。

2.标准偏差(standard deviation,SD):即ROI内CT值的标准偏差。

3.面积(area,a):即ROI内的面积,以mm\textsuperscript{2} 表示。

\subsection{层厚和层距}

层厚(thickness or
slice)是指CT断层每个层面的厚度。层距(interval)是指每个扫描层面间的距离。根据层厚和层距的关系可分为连续扫描、间隔扫描和重叠扫描。

\subsection{像素和像体素}

矩阵(matrix):是一个数学概念,它表示一个横成行、纵成列的数字阵列。如320×320,512×512,1024×1024等。CT机将计算的人体断面各点的CT值以矩阵排列,构成分布图。矩阵由极小的方格所组成,其格数越多即矩阵越大,则显示的图像越清晰细致。

像素和像体素(pixel
voxel):一幅CT图像是由许多矩阵排列的小单元(小方格)组成,这些组成图像的基本单元称为像素。像素所表示的每一个小单元内具有一定宽度和一定厚度(层厚)的立方体称为像体素。体素是一个三维概念,而像素是一个二维概念,像素实际是体素成像时的表现,矩阵越大像素越小。

\subsection{周围间隙现象和伪影}

部分容积效应(partial volume
effect):亦称为平均值效应。每个像素的CT值为此像素或体素内各种物质CT值的平均值,故如果层厚过大,则一个像素内常含有两种或两种以上密度互不相同的物质,从而不能确切地反映其组织密度,这一现象称为部分容积效应。可通过减小层厚减轻部分容积效应的影响。

周围间隙现象(peripheral space
phenomenon):在同一层面内,与层面垂直的两个相邻且密度不同的物体,其物体边缘部的CT值不能准确测得,结果在CT图像上也不能清晰地分辨出两者的交界,这种现象亦称为边缘效应。

伪影(artifacts):由于某些因素的影响,图像中产生实际并不存在的各种形状的假象,称为伪影。

\subsection{空间分辨率和密度分辨率}

空间分辨率(spatial
resolution):又称高对比分辨率,是指CT对物体空间大小(几何尺寸)的分辨能力。通常用每厘米内的线对数或用可辨别最小物体的直径来表示。空间分辨率与被检物体的密度差别也有关,密度差别小则空间分辨率亦也相应减小。影响空间分辨率的主要因素为探测器的几何尺寸、探测器间的间隙和总的原始数据。重建算法也是影响空间分辨率的重要因素。

密度分辨率(density
resolution):又称低对比分辨率,是指CT对密度差别的分辨能力。以百分比表示,如密度分辨率为0.35%,即表示两个物质的密度差>0.35%时,CT即可将它们分辨出来。噪声和信噪比是影响密度分辨率的重要因素。

以上二者是相互制约的,空间分辨率与像素大小密切相关,一般为像素宽度的1.5倍。像素越小,数目越多,空间分辨率提高,图像越清晰,但在X线源总能量不变的条件下,每个单位容积所获得的光子数却按比例减少,使密度分辨率下降。

\subsection{“多层”和“多排”}

“多层”(multi-slice)和“多排”(multi-row)是两个完全不同的概念。1998年全球各相关公司相继推出了4层螺旋CT,然而不同的厂家采用了不同的探测器设计理念。如探测器的排列有对称和不对称之分,有8、16、34排不同的排列,但均为同步获得4层图像的扫描能力。2001年西门子公司率先推出了16层螺旋CT,而探测器的排列是24排。GE公司64层螺旋CT,为64排探测器;SIEMENS公司64层螺旋CT,探测器为40排,机架每旋转1周利用中间的32排探测器即可获得64层图像。故
“排”是指探测器的物理排列数目;而“层”是指数据采集系统同步获得图像的能力,即机架每旋转一周能够同步采集几层图像。所以,“多层螺旋CT”更加符合人们通常的理解且更趋合理。

\protect\hypertarget{text00009.html}{}{}


\chapter{损伤的修复}

\chapterabstract{本章主要介绍机体组织损伤后的修复过程,要求掌握修复、再生、纤维性修复、肉芽组织的概念,不同类型细胞的再生能力,肉芽组织的构成及其在修复过程中的作用,熟悉常见组织的再生过程,瘢痕组织的形态及对机体的影响,创伤愈合的基本过程和皮肤的创伤愈合,了解细胞再生的影响因素,骨折愈合过程和影响创伤愈合的因素。}
\begin{framed}
	{案例2-1}

	{【病例摘要】}

	患者,男,65岁,因意识不清,突发倒地入院。CT检查示右侧基底节出血灶,外科行血肿清除术后,生命体征平稳,但患者仍无自主意识,长期卧床。术后20天,患者左侧肩胛部见一压疮灶,直径约4
	cm,深部组织坏死明显,清创术后数日,见压疮灶内有红色颗粒状组织覆盖。

	{【问题】}

	(1)该压疮灶内红色颗粒状组织是什么?由哪些成分构成?

	(2)该红色颗粒状组织有何功能?
\end{framed}
机体对损伤所造成的缺损进行修补恢复的过程,称为修复(repair)。修复过程可包括两种不同的形式:由损伤周围邻近的同种细胞来修复,称为再生(regeneration);由纤维结缔组织来修复,最后局部纤维化,形成瘢痕,称为纤维性修复。

\section{再生}

\subsection{再生的类型}

\paragraph{生理性再生}
生理过程中,许多组织细胞不断衰老、死亡,同时又由同种细胞通过分裂增生补充,这种再生称为生理性再生。例如皮肤表层角化细胞经常脱落,表皮基底层细胞不断增生分化,予以补充,胃黏膜上皮三天左右更新一次,血细胞也在不断更新等,皆属生理性再生。

\paragraph{病理性再生}
在病理状态下,组织细胞坏死或缺损后,通过周围同种细胞增生来恢复原有的结构和功能,称为病理性再生。如皮肤表皮损伤后,基底层以上各层细胞坏死,由基底层细胞增生、分化,恢复表皮的结构和功能。

\subsection{不同类型细胞的再生能力}

按再生能力不同,将人体组织细胞分为三类。

\paragraph{不稳定细胞(Labile cells)}
这类细胞再生能力强,在生理状态下经常进行周期活动,不断分裂增生,以补充衰老死亡的细胞,在病理状态下也具有强大的再生能力。例如全身的上皮细胞、淋巴造血细胞。上皮细胞包括皮肤表皮、胃肠道和呼吸道的黏膜上皮、泌尿道的移行上皮以及腺体的导管上皮等。

\paragraph{稳定细胞(stable cells)}
这类细胞在生理状态下增生现象不明显,处于细胞增殖周期的静止期(G{0}
期),但具有潜在的再生能力,在损伤的刺激下,则进入DNA合成前期(G{1}
期),表现出较强的再生能力。属于这类细胞的有各种腺体及腺样器官的实质细胞,如肝、胰、内分泌腺、汗腺、皮脂腺及肾小管的上皮细胞等;还包括间叶细胞及其衍生的各种细胞,例如成纤维细胞、骨、软骨、脂肪、平滑肌细胞等。

\paragraph{永久性细胞(permanent cells)}
这类细胞在生理状态下较为恒定,基本上无再生能力,故不能分裂增生,一旦遭受损伤则成为永久性缺失。属于这类的细胞有神经细胞、心肌细胞及骨骼肌细胞。心肌细胞和骨骼肌细胞虽有微弱的再生能力,但因速度极慢,以至损伤处被快速增生的纤维结缔组织替代,通过瘢痕修复。

\subsection{常见组织的再生过程}

\paragraph{上皮组织的再生}
(1)被覆上皮再生:皮肤的复层鳞状上皮受损伤时,创缘或基底部残存的基底细胞则分裂、增生,向缺损中心移动。初起为单层,完全覆盖缺损后,细胞开始分化,形成多层,以后角化。黏膜上皮也以同样的方式再生,新生的黏膜上皮细胞初起为立方形,以后增高变为柱状。

(2)腺上皮再生:腺体受损伤后,若基底膜未被破坏,残存的腺上皮分裂增生,可恢复原有的结构和功能。若腺体(包括基底膜)完全破坏,则难以再生。肝细胞有活跃的再生能力,但如肝内网状支架塌陷,再生的肝细胞则形成结构紊乱的肝细胞结节。

\paragraph{血管的再生}
毛细血管多以出芽方式再生。原有毛细血管内皮细胞肥大、分裂增生,形成向血管外突起的幼芽。开始幼芽为实心的细胞条索,在血流冲击下形成管腔,并有血液通过,进而互相吻合构成毛细血管网(图\ref{fig2-1})。为适应功能需要,毛细血管不断改建,部分管腔关闭消失,部分管壁增厚,成为小动脉、小静脉,其平滑肌等成分可由血管外未分化的间叶细胞分化而来。

大血管离断后需手术吻合,吻合处两侧的内皮细胞分裂增生,互相连接,恢复原来的内膜结构。离断处的肌层难以再生,由结缔组织连接,通过瘢痕修复。

\paragraph{纤维组织再生}
纤维组织普遍分布于机体各部位,再生能力很强,是病理性再生中最常见的现象。在损伤的刺激下,局部静止状态的纤维细胞,或未分化的间叶细胞分化形成幼稚的纤维母细胞。幼稚的纤维母细胞胞体大、胞浆丰富略嗜碱性,两端常有突起。电镜下胞浆内有丰富的粗面内质网和高尔基器,提示其合成蛋白的功能活跃。当纤维母细胞停止分裂后,开始合成并分泌原胶原蛋白,在细胞周围形成胶原纤维。随着细胞的成熟,周围胶原纤维逐渐增多,于是胞体大、有突起的纤维母细胞则变成长梭形的半静止状态的纤维细胞(图\ref{fig2-2})。
\begin{figure}[!h]
	\begin{center}
		\includegraphics{./images/Image00024.jpg}
	\end{center}
	\captionsetup{justification=centering}
	\caption{毛细血管再生模式图 \\ {\small 毛细血管内皮细胞增生;增生的内皮细胞形成条索,并出现管腔;新生的毛细血管相互连接、沟通}}
	\label{fig2-1}
\end{figure}
%\FloatBarrier


\begin{figure}[!h]
	\begin{center}
		\includegraphics{./images/Image00025.jpg}
	\end{center}
	\captionsetup{justification=centering}
	\caption{纤维母细胞产生胶原纤维并转变为纤维细胞的模式图}
	\label{fig2-2}
\end{figure}
%\FloatBarrier

\paragraph{神经组织的再生}
脑和脊髓内的神经细胞破坏后不能再生,由再生能力较强的胶质细胞形成胶质纤维填补,形成胶质瘢痕。但神经纤维断离后,如果与其相连的神经细胞仍然存活,则可再生。首先断处远侧端的神经髓鞘及轴突崩解吸收,断处近侧一小段神经纤维亦发生同样变化。然后两端的神经膜细胞增生,将断端连接,并产生髓磷脂将轴突包绕,形成髓鞘。近端新生的轴突伸向远端髓鞘内,最终达到该神经末稍,可以完全恢复其功能。由于神经轴突生长缓慢(每天延长1~2
mm),再生过程常需数月以上才能完成。如果近端再生的神经轴突未能向远端髓鞘内伸展,只在断裂处长出很多细支,与周围增生的纤维组织缠绕在一起,可形成瘤状物,即创伤性神经瘤(traumatic
neuroma),可引起顽固性疼痛。为防止上述情况发生,临床上常施行神经吻合术或对截肢神经断端作适当处理。

\section{纤维性修复}

纤维性修复开始于肉芽组织增生,填补组织缺损,以后肉芽组织经过纤维化的过程,转化为胶原纤维为主的瘢痕组织,这种修复便告完成。

\subsection{肉芽组织}

肉芽组织(granulation
tissue)由新生的毛细血管、增生的纤维母细胞及多少不等的炎细胞组成,在创伤表面常呈鲜红色,颗粒状,柔软湿润,似新鲜肉芽(图\ref{fig2-3}),故此得名。组织损伤后24小时内,血管内皮细胞及纤维母细胞开始增生,新生的毛细血管管壁的基底膜和胶原纤维尚不完整,故血管通透性大,富有蛋白的液体甚至红细胞漏出到血管外间隙,使肉芽组织呈水肿样外观。新生的毛细血管常呈平行排列,与创面垂直生长,近伤口表面处互相吻合,形成弓状突起。与此同时,局部组织的纤维母细胞受刺激,分裂增生,并产生胶原纤维(图\ref{fig2-4})。毛细血管与血管之间增生的纤维母细胞一起构成小团块,均匀分布,突起于创面,呈颗粒状。肉芽组织中有些细胞外形似纤维母细胞,除能产生胶原纤维外,胞浆中还含有丰富的肌动蛋白和肌凝蛋白,电镜下胞浆内具有丰富的肌微丝,具有类似平滑肌的收缩能力,这种变异的细胞被称为肌纤维母细胞,在创伤收缩中起重要作用。肌纤维母细胞的起源不明,可能来自未分化的间叶细胞,也可能是一种特殊分化的纤维母细胞。炎细胞中以巨噬细胞为主,也可有中性粒细胞及淋巴细胞。巨噬细胞和中性粒细胞具有吞噬细菌和组织碎片的作用,这些细胞坏死后释放的蛋白水解酶能分解坏死组织及纤维蛋白。

\begin{figure}[!h]
	\begin{center}
		\includegraphics{./images/Image00026.jpg}
	\end{center}
	\captionsetup{justification=centering}
	\caption{创口表面颗粒状肉芽组织}
	\label{fig2-3}
\end{figure}


\begin{figure}[!h]
	\begin{center}
		\includegraphics{./images/Image00027.jpg}
	\end{center}
	\captionsetup{justification=centering}
	\caption{肉芽组织镜下观}
	\label{fig2-4}
\end{figure}

肉芽组织在修复过程中有抗感染及保护创面,机化血凝块、坏死组织及其他异物,填补伤口或其他组织缺损等作用。

\subsection{瘢痕组织}

肉芽组织形成的初期呈鲜红色、颗粒状,如嫩芽,以后细胞间水分逐渐减少,纤维母细胞合成胶原纤维,并逐渐转变为纤维细胞。随着细胞外胶原纤维增多,多数毛细血管逐渐关闭、退化、消失,少数改建为小动脉、小静脉。肉芽组织中的炎细胞也先后消失。经过上述纤维化过程,肉芽组织转变为血管稀少,主要由胶原纤维组成的瘢痕组织(图\ref{fig2-5})。肉眼观:呈灰白色,质硬,缺乏弹性。瘢痕组织中胶原纤维经过不断的溶解、形成和改建,最终排列方向与创面平行,以适应伤口修复后的强度需要。

\begin{figure}[!htbp]
	\centering
	\includegraphics{./images/Image00028.jpg}
	\caption{肉芽组织转变为瘢痕组织镜下观(HE染色,低倍) \\ {\small 毛细血管明显减少,胶原纤维沉积增多}}
	\label{fig2-5}
\end{figure}

瘢痕组织中血管少,细胞少,胶原纤维较多较粗,常有玻璃样变性。由于瘢痕组织内肌纤维母细胞的收缩及后期瘢痕内水分明显减少,引起病灶体积缩小,此即瘢痕收缩。瘢痕收缩可引起组织、器官表面凹陷或器官变形,还可造成腔道狭窄。关节附近的瘢痕可致关节运动障碍。瘢痕愈大,影响愈甚。发生在重要器官的瘢痕收缩后将造成严重后果。例如,心瓣膜上的瘢痕可引起瓣膜闭锁不全或瓣膜口狭窄,造成血流动力学的改变,严重者可导致心力衰竭。一般情况下,瘢痕中的胶原纤维在胶原酶的作用下逐渐降解吸收,瘢痕缓慢变小、变软,偶尔瘢痕中胶原纤维形成过多,可成为大而不规则的硬结。少数“瘢痕体质”者,轻微创伤后就可形成明显的瘢痕,过度的瘢痕形成称为瘢痕疙瘩。

\section{创伤愈合}

创伤愈合(wound
healing)是机体组织遭受创伤后进行再生修复的过程,它包括创伤周围特异性组织细胞再生,以及肉芽组织形成、纤维化,最后形成瘢痕组织的复杂过程。

\subsection{创伤愈合的基本过程}

\paragraph{伤口的早期变化}
伤口局部有不同程度的组织损伤、出血及炎症反应。血液和炎性渗出物中的纤维蛋白凝固成血凝块充满缺口,血凝块表面脱水、干燥形成痂皮。血凝块和痂皮对伤口起填充和保护作用,血凝块中的血小板及单核细胞等具有促进局部细胞再生的作用。

\paragraph{伤口收缩}
2~3天后,伤口边缘的皮肤和皮下组织向中心移动,创面缩小。动物实验证明,有些部位的创面在15天内可缩小80%,对愈合十分有利。创面缩小与肉芽组织中肌纤维母细胞收缩有关。

\paragraph{肉芽组织增生及瘢痕形成}
大约从第3天开始,自创缘长出肉芽组织,并向伤口中的血凝块内延伸,机化血凝块。第5~6天起,纤维母细胞产生胶原纤维,其后一周胶原纤维形成极为活跃,以后逐渐缓慢下来。随胶原纤维增多,形成瘢痕组织,大约在伤后一个月瘢痕完全形成。瘢痕组织抗拉力的强度只有正常皮肤的70%~80%,因此腹壁和心脏等部位的较大瘢痕,在内压的作用下可膨出形成腹壁疝或室壁瘤。

\paragraph{表皮及其他组织再生}
表皮再生经过细胞移动、细胞增生及细胞分化三个连续过程。受伤后24小时内,创缘上皮基底层细胞,开始在血凝块下面向伤口中心移动、增生,伤后48小时连接成片,形成菲薄的单层上皮,然后分化。伤后5天内就可恢复原有上皮层厚度并具有角化层的正常表皮结构。

如伤口过大(直径>20
cm)再生表皮难以将创口完全覆盖,往往需要植皮。毛囊、汗腺、皮脂腺等组织若完全破坏,则不能再生,由瘢痕修复。

\subsection{皮肤的创伤愈合}

根据创伤程度及有无感染可分为三种类型。

\paragraph{一期愈合(healing by first intention)}
见于组织缺损少、创缘整齐、创面对合好、无感染、炎症反应轻微的伤口。例如手术切口,切口内只有少量血凝块,创缘炎症反应轻微,第二天表皮再生,在48小时内形成连续的上皮细胞层,覆盖创面,将之与炎性渗出物及血凝块分开。第三天肉芽组织从创缘长出并很快填满伤口,5~6天胶原形成(此时可拆线),2~3周完全愈合,留下一条线状瘢痕(图\ref{fig2-6})。

\begin{figure}[!htbp]
	\centering
	\includegraphics{./images/Image00029.jpg}
	\caption{皮肤一期愈合}
	\label{fig2-6}
\end{figure}

\paragraph{二期愈合(healing by second intention)}
见于创伤组织缺损大,创缘不整齐,伴有感染,炎症反应明显的伤口。愈合由创伤底部向上进行,由于创伤大,需要较多的肉芽组织才能填补缺损,这类创伤坏死组织出血多,并有感染,影响上皮细胞增生移行及肉芽组织的生长,需要清除坏死组织,控制感染,创伤才能愈合。二期愈合和一期愈合的基本过程相同,但需时较长。由于二期愈合肉芽组织增生明显,愈合后形成的瘢痕较大(图\ref{fig2-7}),常影响脏器的外形和功能。若条件允许,可行清创术以达到一期愈合的目的。

\paragraph{痂下愈合(healing under scar)}
创伤表面的血液、渗出液及坏死组织凝固干燥,形成黑褐色硬痂,在痂下进行上述的愈合过程(图\ref{fig2-8}),待上皮再生完成后,硬痂脱落。其愈合时间通常较无痂者长。如痂下有较多的渗出液,易继发感染,不利于愈合。
\begin{figure}[!htbp]
	\centering
	\includegraphics{./images/Image00030.jpg}
	\caption{创伤愈合}
	\label{fig2-7}
\end{figure}

\begin{figure}[!htbp]
	\centering
	\includegraphics{./images/Image00031.jpg}
	\caption{痂下愈合(HE染色,低倍){\small 皮肤创面有血痂形成,上皮已经再生完成,肉芽组织内仍有较多的炎细胞浸润}}
	\label{fig2-8}
\end{figure}


\subsection{骨折愈合}

骨折通常可分为外伤性骨折和病理性骨折两大类。骨的再生能力很强,骨折后大都能完全恢复,其愈合基础是骨膜细胞再生。因其结构和功能的特殊性,愈合过程较复杂,可分为以下几个阶段(图\ref{fig2-9})。

\begin{figure}[!htbp]
	\centering
	\includegraphics{./images/Image00032.jpg}
	\caption{骨折愈合过程}
	\label{fig2-9}
\end{figure}

\paragraph{血肿形成}
骨折时,局部骨和软组织受损伤,血管破裂出血,填充在骨折两端及其周围组织间,形成血肿。骨折局部还可见轻度的炎症反应。

\paragraph{纤维性骨痂形成}
骨折2~3天后,血肿开始由肉芽组织取代而机化,增生的肉芽组织填充和桥接骨折断端,使局部呈梭形膨大,继而纤维化,称为纤维性骨痂,起到初步固定作用。

\paragraph{骨性骨痂形成}
骨折愈合过程进一步发展,纤维性骨痂逐渐分化出骨母细胞及软骨母细胞。骨母细胞分泌基质,逐渐成熟为骨细胞,形成类骨组织,类骨组织经钙盐沉着后变为骨组织,即骨性骨痂。此过程约需几周。骨性骨痂中骨小梁排列紊乱,结构不够致密,仍达不到正常功能需要。软骨母细胞也可经过软骨内化骨形成骨性骨痂,但所需时间较长。软骨的形成与骨折后断端固定不良有关。

\paragraph{骨痂改建或再塑}
上述骨痂形成后,骨折断端被幼稚的、排列不规则的编织骨连接起来,属临床愈合。为了适应生理要求,还需要进一步改建为成熟的板状骨,并重新恢复皮质骨和骨髓腔的正常关系。改建是在破骨细胞的骨质吸收及骨母细胞新骨形成协调作用下进行的。改建后新骨的排列将适应该骨活动时承受压力的方向。骨痂的改建过程在儿童需1~2年,成人需要更长时间。

\subsection{影响创伤愈合的因素}

影响创伤愈合的因素多种多样,了解的目的是为了避免不利因素,创造有利条件,加速组织再生修复。

\subsubsection{全身因素}

\paragraph{年龄}
儿童和青少年较老年人组织再生能力强,愈合快。这可能与老年人常有动脉粥样硬化、血液供应减少、代谢减慢、免疫力降低等有关。

\paragraph{营养}
营养物质缺乏,特别是蛋白质和维生素C,对愈合有很大影响。长期蛋白质缺乏,其中含硫氨基酸蛋氨酸、胱氨酸缺乏时影响前胶原分子形成,不仅使创面愈合速度减慢,而且抗张力强度减低。锌缺乏时将影响DNA和RNA的合成,细胞增生缓慢,延缓创伤愈合。

\paragraph{疾病}
某些疾病,如糖尿病、尿毒症、肿瘤恶病质及一些免疫缺陷病等均可影响再生修复。糖尿病患者白细胞功能降低,对细菌微生物的易感性增加。此外,凡引起小血管闭塞及神经的病变都将影响愈合。

\paragraph{激素}
特别是皮质醇类激素能抑制炎症的渗出反应。临床上用皮质醇处理的病人,创伤处巨噬细胞稀少,影响肉芽组织的形成和创伤收缩。因此,在炎症修复过程中皮质醇类激素的使用要慎重。

\subsubsection{局部因素}

\paragraph{感染和异物}
感染使渗出物增多,从而增加局部创口的张力,甚至引起伤口裂开。许多化脓菌产生的毒素和酶能引起组织坏死,基质和胶原纤维溶解,加重局部损伤,因此只有当创伤局部感染被控制后,修复才能顺利进行。异物(如丝线等)可对局部组织有刺激作用,引起异物反应,妨碍修复。

\paragraph{局部血循环障碍}
血液供应对创伤愈合很重要,凡是引起动脉血供应不足,或静脉血流不畅的疾病都将影响局部创伤的愈合。如下肢静脉曲张患者,小腿发生溃疡后,常迁延不愈,变为慢性溃疡。X线长期照射的部位,小动脉壁增厚,管腔变窄,局部组织供血不良,损伤后修复缓慢。

\paragraph{神经支配}
正常的神经支配对维持组织结构及功能极为重要,失去神经支配的组织就失去了对损伤的反应。正常的神经功能与再生修复亦有一定关系,例如麻风病引起的溃疡不易愈合,这与麻风病患者肢体神经受累有关。

\section{再生修复的机制}

组织损伤修复的机制极为复杂,涉及损伤局部的炎症反应、各种化学因子的释放、干细胞和纤维母细胞的激活和增殖、细胞外基质的产生以及与细胞之间的相互作用、增生程度的控制、修复后重塑等(图\ref{fig2-10})。

\begin{figure}[!htbp]
	\centering
	\includegraphics{./images/Image00033.jpg}
	\caption{损伤修复机制}
	\label{fig2-10}
\end{figure}


\subsubsection{干细胞}

干细胞(stem
cell)是一类未充分分化且具有自我复制能力(self-renewing)的多潜能细胞。在一定条件下,它可以分化成多种功能细胞。根据干细胞所处的发育阶段分为胚胎干细胞(embryonic
stem cell,ES细胞)和成体干细胞(somatic stem
cell),近年科学家还在实验室用基因工程方法构建了诱导型多能干细胞。根据干细胞的发育潜能分为三类:全能干细胞(totipotent
stem cell,TSC)、多能干细胞(pluripotent stem
cell)和单能干细胞(unipotent stem
cell)(专能干细胞)。干细胞具有再生各种组织器官和人体的潜在功能。

组织损伤后,干细胞激活,可向特定方向分化、增殖,修复组织缺损。

\subsubsection{生长因子}

细胞受到损伤因素刺激后,可通过释放多种生长因子(growth
factor),刺激同类细胞或同一胚层发育来的细胞增生,促进修复过程。生长因子在细胞移动、收缩和分化中也发挥重要作用。常见的有以下几种:

\paragraph{血管内皮生长因子(vascular endothelial growth factor,VEGF)}
是至今发现的最强的血管通透促进剂,可促进内皮细胞增殖,在胚胎发育、创伤愈合等生理及病理过程中具有明显的促血管增生作用。

\paragraph{纤维母细胞生长因子(fibroblast growth factor,FGF)}
具有广泛的生物学活性,能影响多种细胞(血管内皮细胞、平滑肌细胞、纤维母细胞等)的生长、分化及功能。FGF可使血管内皮细胞分裂并诱导其产生蛋白溶解酶,后者溶解基膜,便于内皮细胞穿越生芽。

\paragraph{血小板源性生长因子(platelet derived growth factor,PDGF)}
主要由黏附于血管损伤处血小板的α颗粒释放,能刺激血管平滑肌细胞、纤维母细胞和胶质细胞等的分裂、增殖,通过刺激胶原合成和胶原酶的活化作用,调节细胞外基质的更新。

\paragraph{表皮生长因子(epidermal growth factor,EGF)}
通过作用于靶细胞膜上的特异性受体而发挥多种生物学效应,是一种强有力的促细胞分裂、分化和增殖的因子,对上皮细胞、纤维母细胞、平滑肌细胞都有促进增殖的作用。

\paragraph{转化生长因子(transforming growth factor,TGF)}
TGF-α可与EGF受体结合,与EGF具有类似作用。TGF-β具有复杂的生物学功能,对纤维母细胞和平滑肌细胞增生的作用依其浓度而异,高浓度可抑制
PDGF受体表达,使其生长受到抑制,低浓度诱导PDGF合成、分泌。

\paragraph{肿瘤坏死因子(tumor necrosis factor,TNF)}
是多功能的多肽,可促进内皮细胞分化,诱导基质产生,也可间接刺激其他细胞产生血管生长因子。在体内可促进内皮细胞形成血管,在体外可刺激培养的内皮细胞形成管样结构。

\subsubsection{细胞外基质及其受体}

人体各种组织均由细胞外基质(extracellular
matrix,ECM)构成支架,它的主要作用是把细胞连接在一起,借以支撑和维持组织的生理结构和功能。ECM能影响细胞的形态、分化、迁移、增殖和生物学功能,在调控胚胎发育、创伤修复及肿瘤浸润转移等方面都起着重要作用。研究表明,尽管不稳定细胞和稳定细胞都具有完全再生能力,但能否重新构建为正常结构尚依赖ECM。

ECM的主要成分如下:

\paragraph{胶原蛋白和弹力蛋白}
胶原蛋白(collagen)是ECM的主要组成成分,几乎分布于所有组织中,为多细胞生物提供细胞外支架。目前发现的胶原类型达18种之多,其中Ⅰ~Ⅳ型含量较多。Ⅰ、Ⅱ、Ⅲ型胶原为纤维性胶原,Ⅰ和Ⅲ型主要分布于间质结缔组织中,Ⅱ型胶原则主要分布于软骨;Ⅳ型胶原为基底膜胶原,在基底膜主要基质蛋白成分中占60%。弹力蛋白(elastin)分子结构与胶原蛋白相似,但分子间交联较少。主要存在于血管、皮肤、韧带、肺等组织中,分子量约70kD,对维持组织的弹性与张力起重要作用。

\paragraph{蛋白多糖}
蛋白多糖(proteoglycans)是ECM的另一重要成分,其结构包括核心蛋白及与其相连接的多糖或多个多糖聚合形成的氨基多糖(glycosaminoglycans)。常见的蛋白多糖有硫酸肝素、硫酸软骨素、硫酸皮肤素、硫酸角质素和透明质酸等,其功能主要是通过介导一系列生物大分子之间的信息传递参与组织的发育和维持正常的生理功能。透明质酸是大分子蛋白多糖复合物的骨架,与调节细胞增殖和迁移有关。

\paragraph{黏附性糖蛋白}
黏附性糖蛋白(adhesive
glycoproteins)既能与其他细胞外基质结合,又能与特异性的细胞表面蛋白结合,将不同的细胞外基质与细胞之间联系起来。纤维连接蛋白(fibronectin)作为一种多功能的黏附性糖蛋白,能使细胞与各种基质成分发生粘连,与细胞黏附、细胞迁移等功能直接相关。层黏连蛋白(laminin)可与细胞表面的特异性受体结合,也可与基质成分如IV型胶原和硫酸肝素结合,还可介导细胞与结缔组织基质黏附。

\paragraph{整合素}
整合素(integrins)是位于细胞膜上的细胞外基质受体,对细胞和细胞外基质的黏附起介导作用,可将来自细胞外基质之信号传入细胞。其特殊类型在白细胞黏附过程中还可诱导细胞与细胞间相互作用。

\subsubsection{抑素与接触抑制}

抑素(chalon)具有组织特异性,似乎任何组织都可以产生一种抑素抑制本身的增殖。如已分化的表皮细胞能分泌表皮抑素,抑制基底细胞增殖。当已分化的表皮细胞丧失时,抑素分泌终止,基底细胞分裂增生,直到增生分化的细胞达到足够数量或抑制达到足够浓度为止。TGF-β虽然对某些间叶细胞增殖起促进作用,但对上皮细胞则是一种抑素。此外干扰素-α、前列腺素E2和肝素在组织培养中对成纤维细胞及平滑肌细胞的增生都有抑素样作用。

皮肤创伤,缺损部周围上皮细胞移动,分裂增生,将创伤面覆盖而相互接触时,或部分切除后的肝脏,当肝细胞增生达到原有大小时,细胞停止生长,不至堆积起来。这种现象称为接触抑制(contact
inhibition)。细胞缝隙连接(可能还有桥粒)也许参与接触抑制的调控。

\begin{center}
	\textbf{知识链接}
\end{center}
\chapterabstract{生物敷料可以与伤口密切贴合,保持愈合环境湿润,减轻疼痛,辅助局部使用药物和内源性分子促进伤口愈合。胶原、透明质酸等材料制备的生物敷料不仅具有止血促凝作用,还可影响生长因子(VEGF、FGF、TGF)分泌,诱导多种细胞增殖分化,有利于伤口愈合。}

{【附】与创伤愈合有关的生长因子}

对单核细胞具有趋化作用:PDGF、FGF、TGF-β

纤维母细胞迁移:PDGF、EGF、FGF、TGF-β、TNF

纤维母细胞增殖:PDGF、CTGF、EGF、FGF、TNF

血管生成:VEGF、FGF

胶原合成:TGF-β、PDGF、TNF

分泌胶原酶:PDGF、FGF、EGF、TNF、TGF-β抑制物

\section*{复习与思考}

{一、名词解释}

修复 再生 纤维性修复 稳定性细胞 永久性细胞 肉芽组织 一期愈合

{二、问答题}

1. 试述肉芽组织的结构及其在修复过程中的作用。

2. 影响细胞再生的因素有哪些?

3. 影响创伤愈合的因素有哪些?

4. 试述骨折愈合的基本过程。

\chapter{超声在休克和循环功能监测及支持中的应用}

\section{前沿学术综述}

超声心动图是目前能够在床旁提供实时有关心脏结构和功能信息的唯一影像工具。多普勒心脏超声技术可以更加详细地评估患者的血流动力学改变,因而更有助于快速明确导致急性循环衰竭的机制与原因。由于可以在很短的时间内准确评估血流动力学状态,心脏超声对于休克或存在循环衰竭的重症患者,无论是早期识别与评估,还是整个诊疗过程中都有理由成为适合的理想的监测工具。另外,随着科学技术和电子技术的快速进步、经食管的多平面探头的出现,使心脏超声的图像质量大幅提高,使一些过去经胸心脏超声很难获得满意图像的患者也可以获得可靠的相关信息。目前许多研究表明,心脏超声在重症患者的应用,可以促使患者的治疗产生有益的改变
\protect\hyperlink{text00009.htmlux5cux23ch1-8}{\textsuperscript{{[}1{]}}}
。同时,值得关注的是,肺部超声、肾脏超声在重症监测的快速发展进一步丰富了超声在休克和循环功能监测及支持中的应用,因此超声作为有前途的重症监测与支持工具在重症医学科的应用中逐渐走向成熟与普及。

\subsubsection{心脏超声在重症医学科中应用的发展与特点}

早期的综合重症医学科,心脏超声检查大多由通过资质认证的心脏专科医生来进行,主要目的是快速准确获得图像,帮助诊断心血管疾病,如心包填塞、急性心肌梗死的并发症、自发的主动脉夹层和创伤性主动脉损伤等。而对于血流动力学的无创评估仅仅是应用二维技术联合多普勒模式来测量每搏输出量和每分心脏输出量。事实上,当时的重症医学科医生对心脏超声的潜力和作用缺乏全面的认识。直到20世纪80年代中期,一些重症医学科医生中的先行者开始拓展应用心脏超声对血流动力学的全面而详尽的评估。首先推荐用于感染性休克和急性呼吸窘迫综合征患者,应用心脏超声替代右心漂浮导管进行血流动力学评估,并且率先开始自己进行心脏超声检查,尤其是可以24小时随时床旁进行重复检查和评估,并且指导治疗。随后由于在循环衰竭诊断与评估应用的扩展、随着监测和测量经验的积累,尤其经食管超声(TEE)准确度的增加,重症患者床旁超声的应用价值逐步得到认识和肯定,有研究表明其对治疗支持的影响和预测病死率有重要作用。

但直到上世纪90年代,重症医学科医生对心脏超声的兴趣才真正开始明显增加,主要原因有:心脏漂浮导管研究出现大量阴性甚至负面结果;与传统有创血流动力学评估手段相比心脏超声无创、实用;大量相关研究文献发表和大量相关重症医学科医生心脏超声培训课程出现使得重症医学科医生的超声应用技能得以明显提高。在这一时期,一些官方组织开始推荐经食管超声作为急性循环衰竭的一线评估手段。

近年来,功能血流动力学评估概念的提出,再次间接推动了心脏超声在重症医学科循环衰竭患者中的应用。越来越多证据显示,超声检查参数可准确评估重症医学科机械通气的感染性休克患者的心功能和液体反应性,而这些参数丰富了重症医学科时刻存在的心功能和液体反应性评估指标,同时大大激发了重症医学科医生对心脏超声的兴趣
\protect\hyperlink{text00009.htmlux5cux23ch2-8}{\textsuperscript{{[}2{]}}}
。

\subsubsection{心脏超声在评估心脏前负荷及容量反应性方面的作用}

众所周知,在重症医学科管理血流动力学不稳定的患者时,最常见的临床行为就是实现以提高心输出量和组织灌注为目的的血管内容量和心脏前负荷的最佳化调节。而在此调节过程中,无论是让患者处于容量不足还是容量过负荷状态均会产生严重的后果,评估患者的容量状态极为重要。所以在有指征给患者输液时,进行容量反应性的评估尤为重要,而心脏超声给我们提供了更多更准确更便捷的选择。

心脏超声能够评估患者的容量状态,是传统有创血流动力学监测评估的有益补充,更有可能更加可信可靠。一般情况下,经胸心脏超声已经可以提供足够可用的信息。当经胸超声图像欠理想时,经食管超声检查可以提供理想图像,用于比经胸心脏超声更准确的评估心内流量、心肺相互作用、上腔静脉的扩张变异度等。

心脏超声对容量状态的评估可采用静态或动态指标,静态指标即单一的测量心脏内径大小和流量快慢;动态指标用来判断液体反应性,包括自主或机械通气时呼吸负荷的变化、被动抬腿试验和容量负荷试验等。其中,动态指标临床使用更多。心肺相互作用的指标如上腔静脉塌陷率、下腔静脉扩张指数、左室射血的呼吸变化率等,用于预测窦性心律、无自主呼吸机械通气患者的容量反应性;被动抬腿试验相当于内源性容量负荷试验,通过超声观察抬腿前后左室射血流速增加情况来预测容量反应性,无论患者自主呼吸或机械通气、任何心律情况下,均可应用。临床治疗中,可动态和静态指标联合应用进行评估。如严重低血容量时评估的超声征象:功能增强但容积很小的左室;自主呼吸时下腔静脉吸气塌陷非常小;机械通气患者呼气末下腔静脉呼吸变化非常小。

评估容量反应性时,必须考虑以下因素:①容量反应性的评估需要测量多个参数,综合分析;②左室或右室内径大小的变化对容量反应性的预测不可靠;③评估容量反应性时,必须考虑自主呼吸与正压通气对采用指标的不同影响,当患者存在心律失常或自主呼吸时,应用心肺相互作用的指标评估容量反应性并不准确,可选择被动抬腿试验;④非心脏超声获得的心肺相互作用评估容量反应性(如脉压呼吸变化率)的假阳性原因(尤其严重右心衰)易于通过心脏超声检查明确。

总之,心脏超声在评估心脏前负荷及容量反应性方面可用、有效且极具前景。

\subsubsection{心脏超声在评估心功能中的作用}

重症患者心功能的改变非常常见,如心功能衰竭或心肌抑制,此时心室收缩、舒张功能的定量分析对于病情监测、指导治疗和判断预后具有十分重要的临床意义。心脏超声通过二维心脏超声、M型心脏超声、利用几何模型的容量测定、辛普森法、组织多普勒技术、Tei指数和三维心脏超声等方法对心脏功能进行评估,无创且便捷。心功能测定包括左(右)心室收缩和舒张功能测定,其中,左心室功能检测在临床病情评估和治疗中最为重要
\protect\hyperlink{text00009.htmlux5cux23ch3-8}{\textsuperscript{{[}3{]}}}
。

射血分数是目前研究最多,且最为临床所接受的心脏功能指标,具有容易获得(甚至有经验的操作者目测的结果与实测结果相差很小,相关系数达0.91)、可重复性好以及能够较早评价全心收缩功能等优点(不同于环周纤维缩短率,在有节段异常时,也经常发生改变)。目前研究表明,射血分数是与预后最相关的心功能指标。射血分数的测量方法很多,其中Simpson最准确,被美国超声学会所推荐。但最大的缺陷在于对心内膜边缘的确认水平要求足够高,两腔像与四腔像要求垂直,而且操作略显繁杂费时。射血分数值作为一个最重要的评价心脏收缩功能指标,也具有明显的局限性,受前后负荷的影响非常明显。前负荷增加通过Frank-Starling机制增加射血分数值,而后负荷增加抑制射血分数值,如在没有血管活性药物支持、仅扩容治疗的感染性休克患者,前负荷稳定或增加,同时血压/外周阻力明显下降都会导致射血分数测量值不能代表心肌的真实收缩功能。另外一个重要的心功能指标是平均环周纤维缩短率,最大优点在于不依赖于前负荷改变,同时,经过心率纠正后的指标心率纠正的平均环周纤维缩短率,由于去除了心率的影响,似乎比射血分数能更好地反映心肌收缩功能。

有研究显示组织多普勒技术测定的心肌收缩速度可以代表全心室功能,尤其可反映二尖瓣环心肌收缩速度。另外,有研究表明,尽管存在对前后负荷的依赖,在肥厚性心肌病和舒张功能不全的患者,运用组织多普勒技术测定的心肌收缩速度指标可以在显性心肌肥厚和心脏收缩功能不全之前即发现渐进的心肌收缩功能受损,同时,这些指标对受心脏前后负荷的影响不大
\protect\hyperlink{text00009.htmlux5cux23ch4-8}{\textsuperscript{{[}4{]}}}
。

综上所述,近年来,在心脏超声多普勒技术领域,评估左心室收缩功能的进展主要集中在两个方向。首先是探索对负荷依赖程度低的指标,即接近心肌内在性能的指标,如左心室等容收缩压力增加速率,不依赖后负荷而对前负荷轻度依赖,同时,已有许多研究表明这些指标有助于预后判断;其次是研究心肌本身的指标,以往的许多指标大多依赖于血容量(腔室的大小)和血流(多普勒流速和压力的变化)进行测量,而随着超声多普勒技术的进步,尤其是组织多普勒的发展,最近的研究则侧重于应用无创技术测量心肌本身或其内在的机能。目前可测得的主要指标包括心肌收缩速度、左室质量、应变和应变率以及与应力的关系等。这些指标对患者预后影响的研究尚少,尤其缺乏大规模研究,仅发现充血性心衰患者心肌收缩速度<5厘米/秒可预测心脏不良事件的发生。

组织多普勒技术测定的Tei指数又称为心肌做功指数,心肌做功指数=(心室等容收缩时间+心室等容舒张时间)/心室射血时间。该指数于1995年由日本学者Tei提出,无创、敏感,能综合反映心室收缩及舒张功能,是可行的评价左室功能的指标,是对常规测定的血流多普勒参数的重要补充。目前尚无公认的正常值。

实时三维心脏超声全面、快速准确地测定左室功能,一直是心脏超声工作者的梦想。有人研究应用这一新的技术,测定正常人和心脏病患者的左室射血分数,并与常规双平面二维改良Simpson's法测定左室射血分数进行对照,证明实时一次心动周期三维超声即能准确、快速测定左室射血分数。实时三维心脏超声可以产生实时三维的心脏图像及左室容积时间曲线,克服了二维超声的限制,在测量心室容积时不需要几何形状的假定,不受心脏几何形态的影响,因而测量的结果更为准确,能全面实时地观察和测量动态心室的整体及局部容积大小、运动及功能状态,从而提高心功能评估的可靠性,是一种无创的新方法。

\subsubsection{心脏超声对外周血管阻力的评估}

心脏超声多普勒技术可以直接测量外周血管阻力,但不易方便和简单使用,因此在临床工作当中,经常根据临床和心脏超声的检查结果进行排除诊断,如在心脏负荷足够同时左右心脏收缩功能均满意的情况下,仍然存在低血压则提示外周血管阻力低。

\subsubsection{心脏超声在特殊情况下的应用}

严重感染和感染性休克是常见病、多发病,与急性心肌梗死发病率相当,甚至高于许多肿瘤的发病率,是住院患者最常见的死亡原因之一,且病死率随着年龄增加而增加,甚至大于急性心肌梗死,达到30%~60%。其中,早期出现心功能异常的患者若表现为低心排,死亡率>80%。另有研究提示,合并出现心血管损害的全身性感染患者,死亡率由20%升至70%~90%。

临床上常见严重感染和感染性休克时,心输出量并不降低或反而增加,但合并心肌功能不全。这种心功能不全多出现于感染性休克早期,往往难以早期发现及处理,造成的危害极大。随着心脏超声在评估左室心脏功能应用的进展,目前已被应用于感染性休克相关的心肌抑制的早期发现与指导支持治疗
\protect\hyperlink{text00009.htmlux5cux23ch5-8}{\textsuperscript{{[}5{]}}}
。目前常用指标有射血分数、环周纤维缩短率、心肌收缩速度等,而应用应变和应变率以及与应力的关系等对于早期发现与感染相关的心肌抑制及指导正性肌力药物应用具有更好的前景。

无论是围手术期还是严重创伤患者,缺血性心脏病非常常见。局部心肌缺血导致局部心肌运动异常。临床实际中,局部心肌缺血的评估最常用到的方法是对二维超声显像室壁运动和室壁增厚率进行目测。与心肌节段的室壁增厚率相比较,二维超声应变成像对心肌缺血的变化更加敏感。急性心肌梗死后可出现多种舒张期充盈异常即左心室舒张功能异常,表现为二尖瓣血流频谱E峰峰值速度减低,A峰峰值速度增高,E/A比值<1,E峰减速时间延长,等容舒张时间延长,肺静脉血流频谱S/D峰值比值增加等。另外,随着彩色多普勒心脏超声在临床的广泛运用,急性心肌梗死后左室舒张功能得到更全面深刻的认识,对临床治疗方案的制定和调整也起到重要作用。心肌应变测量的是心肌各节段的变形,在定量评价心肌各节段的收缩和舒张功能时,心肌应变与心肌的收缩和舒张功能密切相关,因此能准确评估心肌收缩和舒张功能。

急性肺血栓栓塞是临床上一种危重心肺疾病,心脏超声对其病变程度、治疗效果及预后评估有重要作用,已经普遍应用于临床。超声检查急性肺血栓栓塞一般包括心脏超声检查及下肢深静脉检查。尤其对于确诊的急性肺血栓栓塞患者,超声探测到中度、重度右室功能障碍者,其近期及长期病死率均明显升高,而不伴有右室负荷过重的患者,近期预后良好。因此超声能够根据右室功能状态进行危险度分层及预后判断。心脏超声可以动态、无创、重复估测肺动脉压力,因此可以判断治疗效果,可以作为随访追踪的一种快速、简便的检查手段。

\subsubsection{肺部超声在循环监测与支持中的作用}

最近几年来,随着肺部超声的进步与推广,成为能够发现与评估不同肺部与胸腔病变的有力技术。肺部超声常见征象与特点包括:①正常通气,胸膜线下平行排列的A线;②肺间质肺泡综合征,彗星尾征,根据B线的间隔不同分为B7线(B线间隔大约7mm,主要是肺小叶间隔增厚)和B3线(B线间隔3mm);③肺实变征,包括组织样征、碎片征和支气管气象;④胸腔积液,静态征象为四边形征,动态征象为水母征和正弦波征;⑤气胸,肺点消失。

以上是常见肺部病变的超声表现。对于肺水肿患者,肺水含量的评估非常重要,肺部超声获得B线可以早期发现在血气分析改变之前的肺水肿,而且超声具有简单、无创、无放射性和实时性等优点。超声监测导向诊断的难点在于急性心源性肺水肿与ARDS肺水肿的鉴别,最新有研究表明,循环支持过程中,肺部超声的A-优势型表现提示肺动脉嵌顿压<13mmHg的可能性大;而在B-优势型时,提示肺动脉嵌顿压>18mmHg可能性大
\protect\hyperlink{text00009.htmlux5cux23ch6-8}{\textsuperscript{{[}6{]}}}
。

\subsubsection{重症肾脏超声在循环监测及休克支持中的作用}

肾脏是休克时最容易受损或最早受损的器官之一,重症患者病变过程中易并发急性肾损伤。术后患者发生率1%,重症患者达到35%,尤其感染性休克患者发生率在50%以上。因此预测、发现和评估急性肾损伤非常重要。重症肾脏超声能够床旁及时无创监测肾脏改变,能够同时关注和监测肾脏大循环与微循环情况,为休克循环监测和支持提供了新的重要思路。

总之,重症超声包括超声心动图、肺部超声和重症肾脏超声在血流动力学评估,尤其对于心脏功能、容量反应性等血流动力学评估的作用越来越重大;在重症医学科常见的重症疾病如休克的监测与支持等诸多方面都开始发挥举足轻重的作用,已经被众多重症医学科医生所接受和掌握。因此,全世界范围内的重症医学科医师的重症超声培训和认证正在如火如荼地进行。

\section{临床问题}

\subsection{超声评价血流动力学的作用}

\subsubsection{为什么超声是评价重症患者血流动力学的重要方法?}

在重症患者中,血流动力学不稳定(急性或慢性)是很常见的问题。长期低血压可能导致器官缺血、功能紊乱等不良后果。相反,快速的诊断和早期干预可以避免血流动力学的进一步恶化。然而,仅仅依靠临床常规检查尚不足以做出正确的诊疗决策。对于不常见的临床问题,临床疑诊是建立鉴别诊断和灵活应用诊疗技术来做出诊疗决策的关键。超声心动图就是能够在不同疾病的快速诊断中发挥重要作用的技术之一。因此,对于患者血流动力学不稳定的原因和监测,超声心动图能够发挥强大作用,可以用于评估前负荷、后负荷和心肌收缩力。各类研究表明,超声心动图的应用使至少1/4的重症患者的治疗有所改变。

应该强调的是,应用超声心动图来评估重症患者,能快速而可靠地排查像肺栓塞和心包填塞等能引起患者血流动力学不稳定的主要病因,而这些操作可由经过简易超声心动图检查训练的重症医学科医生或者急诊医生完成,并且是血流动力学不稳定重症患者评估的关键一步。

在排除了一些主要病因之后,需评价患者的容量状态和心功能。最重要也是最常使用的评价左心室整体或者局部室壁运动的方法,是多切面的定性评估。这种方法快速而有效,并且与核素扫描结果具有很好的一致性。超声心动图的检查结果不仅能评估局部室壁运动,还能通过估计射血分数来评估左心室整体功能。心室功能的定量评估能提供可测量性更好的、误差更少的评价方法。但需要警惕的是,所有有效的评估方法都既有长处,又有各自局限性。

\subsubsection{如何看待经胸壁超声心动图、经食管超声心动图和手持设备在重症医学科的作用?}

在重症医学科中,经食管超声心动图经常被认为比经胸壁超声心动图更有优势,因为后者常常由于下列原因得到的图像质量欠佳:比如术后患者由于机械通气(呼气末正压>15cm
H\textsubscript{2}
O)无法调整体位、缺乏合作耐心、胸壁水肿以及由于伤口敷料、胸腔引流管、胸腹壁开放而使视野阻断。经胸壁超声心动图在被检患者中的成功率为50%~80%,而经食管超声心动图的成功率高达90%。但近年来,更多研究表明经胸壁超声心动图有助于诊疗的超声切面获得率在86%以上。另外,经胸壁超声心动图的常规实施过程面临很多问题。与经胸壁超声心动图相比,经食管超声心动图耗时更长,对专业知识要求更高,而且经食管置入探针有误入气道而阻塞气道的风险。另外,虽然经食管超声心动图会产生像食管穿孔这样的严重并发症,但其可能性较小,大约只有0.01%。

手持式可移动设备轻巧、简单而且方便,能提供定性评估。手持式设备在经超声引导下胸穿以及中心静脉置管等操作中作用明显。新一代的电池供电的检查设备也已出现,这些设备在血流动力学不稳定的重症医学科患者中的地位和应用在进一步加强。

不管检查形式怎样,检查过程本身必须是完整的,并且跟从业人员在训练中要求的一样全面。如果初期检查因为不同原因有所限制,或者结果存在疑问,要求更加有经验的从业人员及早进行更全面的检查。全面检查就是尽量避免罕见疾病的漏诊。经过反复练习之后,完整的检查过程应该在数分钟内完成。合理的检查程序应该是在体格检查的基础上定位于可疑病变部位或结构。一旦解决了直接问题,接下来应该做更加全面的检查,对于可疑病变部位能够有更加充分的检查时间。目前的指南上有经食管超声心动图和经胸壁超声心动图检查的标准图像,以确保所有结构都是从多角度去查看的,而单个结构能被完整而准确的评估并且根据需要被记录下来。标准切面能保证任何结构不被遗漏,还能为从业人员的相互交流提供有效的媒介。

\subsection{超声在容量及容量反应性监测中的作用}

\subsubsection{什么是容量状态与容量反应性?超声检查在其中有什么作用?}

血管内容量和心脏前负荷的最佳化调节是提高心输出量和改善组织灌注的重要环节,通常是血流动力学支持最早期的临床行为。在此调节过程中,评估患者的容量状态极为重要。因为无论是让患者处于容量不足还是容量过负荷状态均会导致严重的后果。所以在有指征给患者输液时,进行容量反应性的评估尤为重要。

目前对容量治疗有反应定义为给予液体治疗后,心输出量指数或每搏输出量指数较前增加≥15%。心脏对容量治疗有反应的生理机制是基于Frank-Starling机制:当心功能处于心功能曲线上升支时,增加前负荷,则可以显著增加心输出量,改善血流动力学,提高氧输送,从而改善组织灌注;而心功能处于平台期时,提高前负荷的潜能有限,扩容则难以进一步增加心输出量,反而可能带来肺水肿等容量过多的危害。

提出容量反应性近20年来,大量研究力图寻找简单可靠并且敏感快捷的指标或方法来预测,进而指导液体治疗,如何选择和应用这些指标也一直是研究的热点。目前预测容量治疗反应的指标或方法,主要包括传统的静态前负荷参数(前负荷压力指标及前负荷容积指标)的监测、容量负荷试验,以及近来研究较多的经心肺相互作用的动态前负荷参数(收缩压变异度、脉搏压变异度、每搏输出量变异度等)和被动腿抬高试验等。

心脏超声能够评估患者的容量状态和容量反应性,是传统有创血流动力学监测评估的有益补充,更有可能比之更加可信可靠。当经胸超声图像欠理想时,经食管超声可以提供理想图像,用于比经胸心脏超声更准确地评估心内流量、心肺相互作用、上腔静脉的变异度等。当然,一般情况下,经胸心脏超声已经可以提供足够可用的信息。心脏超声对容量状态和容量反应性的评估一般包括静态指标和动态指标,静态指标即单一的测量心脏内径、面积及容积大小和流量的快慢;动态指标,广义包括流量和内径大小对于动态手段的变化(自主或机械通气时呼吸负荷的变化、被动腿抬高试验、容量负荷试验等),狭义即指心肺相互关系引导的动态指标。

\subsubsection{根据临床经常面临的容量和容量反应性问题,超声临床判断评估的流程与思路及评估的指标与方法是什么?}

(1)严重容量不足或输液有明显限制时液体反应性的评估 当患者没有进行容量状态和容量反应性评估的指征时,首先可以快速判断是否存在严重容量不足或输液有明显限制及容量过负荷,此时应用的大多为静态指标。

严重低血容量时,预测容量反应性阳性结果的可能非常大。超声评估指标包括:功能增强但容积很小的左室,左心室舒张末期面积<5.5cm\textsuperscript{2}
/m\textsuperscript{2}
体表面积;在自主呼吸时下腔静脉内径小且吸气塌陷非常明显;在机械通气患者呼气末下腔静脉内径非常小,常见<9mm,并且容易随呼吸变化。

容量过负荷或输液限制明显,预测容量反应性阴性可能很大时的超声评估指标包括:在无心包填塞时上下腔静脉有明显充盈表现(扩张或固定);严重右室功能不全及过负荷(右室大于左室的超声证据);心脏超声估测有很高的左室充盈压,如很高的E/E'值。

类似的这些静态指标在评估容量反应性时,有多种影响因素。所以单纯根据一个静态指标评估容量反应性可靠性很差,但对于评估容量明显缺乏和明显过负荷时,却较为可靠,即尽管不敏感,但特异性很强。

(2)既不是严重容量不足、也不是容量过负荷时容量反应性的评估 当患者既不是严重容量不足、也不是容量过负荷,即容量反应性判断比较困难时,此时包括完全机械通气和自主呼吸两种不同的情况,选择的指标和方法如下。

1)完全机械通气容量反应性的评估:在完全机械通气的无心律失常患者,选择心肺相互作用相关的动态指标可以预测容量反应性,如主动脉流速和左室每搏射血的呼吸变化率以及上腔静脉塌陷率、下腔静脉扩张指数等,并且研究证明同非超声获得的动态指标一样,上述指标均明确优于静态指标。

近年来,随着对心肺相互作用认识的进步,在机械通气的患者,左室每搏输出量的呼吸变化率可以作为容量反应性的指标,但由于床旁左室每搏输出量的测量依然复杂而相对困难,所以一些左室每搏输出量呼吸变化率的替代指标被应用和研究,包括动脉监测的脉压呼吸变化率和脉搏轮廓推导的每搏输出量变化率。当然随着心脏超声在重症医学科的更广泛应用,尤其对于血流动力学不稳定患者评估的应用,一些超声检查可以获得的左室每搏输出量呼吸变化率的替代指标被认识和研究应用。2000年前后,Feissel等应用经食管超声测量主动脉瓣环的主动脉血流速的呼吸变化率判断容量反应性,2005年Monnet和Teboul等应用食管多普勒探头直接测量降主动脉峰流速的呼吸变化率来预测容量反应性,均取得理想结果;在儿童相关的研究中,进一步证明经胸超声获得的主动脉峰流速呼吸变化率在预测液体反应性、评估心脏前负荷储备时优于脉搏压变异度和收缩压变异度。另外,在动物研究(阶梯失血兔子模型)中,无论应用经食管超声测量主动脉流速还是经胸超声测量的主动脉血流速度积分呼吸变化率,均可高度准确预测容量反应性。

须说明的是,主动脉流速的测量无论经食管还是经胸,都存在一定的技术问题。而外周的动脉血管,包括桡动脉、肱动脉和股动脉等,其超声血流图像易于获得,因此,近年来研究显示肱动脉峰值血流速的呼吸变化率可预测患者的容量反应性,其敏感度和特异度都达到了90%以上,不亚于脉搏压变异度等动态指标,尤其优于一些静态指标。当然优点还在于完全无创,同时简单易学,甚至于需要培训的时间很短且不需要经验的积累。

对于非外周动脉流速的测量有限性在于需要减低操作者依赖性和进行可重复性可靠性研究,而对于外周动脉,仅仅需要关注局部肌肉收缩对测量的影响。另外尤其要注意这些指标只适用于没有自主呼吸及心律失常的机械通气患者。

使用具有心内膜自动描记功能的超声诊断仪时,可以用左室每搏射血面积呼吸变化率来预测液体反应性。

尽管大规模的荟萃综述分析认为脉搏压变异度是最理想的判断容量反应性的动态指标,但研究对比的对象是收缩压变异度和每搏输出量变异度。在应用超声进行评估时,由于主动脉流速甚至外周动脉的流速变化早于每搏输出量,因此,未来的研究需进一步明确其优越性。

以往的研究多以机械通气的休克患者为研究对象,最近一个关于自主呼吸志愿者的研究证实,在一些较单纯的情况下,如仅仅低血容量时,在自主呼吸状态下主动脉流速的呼吸变化率也可以预测液体反应性,不过此研究需要进一步验证。

另外,还可以通过判断腔静脉的变异度判断容量反应性,如下腔静脉呼吸扩张率和上腔静脉呼吸塌陷率。有研究表明,感染性休克患者下腔静脉扩张率为18%时,预测液体反应性的敏感性和特异度均在90%以上,而上腔静脉呼吸塌陷率的预测值为36%,预测容量反应性的敏感性和特异度也均在90%以上。但需要关注的是,影响腔静脉变异度的因素除了容量状态外还有右心功能和静脉顺应性。下腔静脉呼吸扩张率提出较早,但直到近年,随着对正压通气对下腔静脉影响认识的进步才被广泛接受和应用;而上腔静脉呼吸塌陷率的认识得益于经食管超声在重症患者中的广泛应用,尤其用于对血流动力学不稳定患者的评估
\protect\hyperlink{text00009.htmlux5cux23ch7-8}{\textsuperscript{{[}7{]}}}
。最近,针对失血性休克、全身性感染、蛛网膜下腔出血的患者,尤其慢性肾衰接受肾脏替代治疗患者的研究,进一步显示出腔静脉变异度的临床意义,但依然没有统一的预测值,仍需扩大研究规模。

2)自主呼吸或存在心律失常时容量反应性评估:对于存在自主呼吸或心律失常患者容量反应性的评估,可选择应用被动腿抬高试验相关的超声指标,相当于内源性的容量负荷试验,被动腿抬高试验产生300~450ml血浆快速输入。有研究表明,可应用超声观察每搏输出量的替代指标如被动腿抬高试验前后左室射血流速和流速积分变化来预测容量反应性,并且已经证明其敏感性和特异度均优于收缩压力和心率等;而在具有心内膜自动描记功能的超声诊断仪时,可以用左室每搏射血面积在被动腿抬高试验前后变化情况来预测液体反应性。

除应用左室射血流速和流速积分变化来预测容量反应性,最新有研究发现对于全身性感染和重症胰腺炎患者,在被动腿抬高试验前后应用外周动脉如股动脉峰值流速的变化与每搏输出量、脉压变化都可以用来预测液体反应性,前后变化分别为8%、10%和9%,同时研究还发现用心率来代表被动腿抬高试验前后自主神经功能时,前后没有变化,使得临床可操作性明显增强,当然除了选择股动脉还可以考虑其他外周动脉,如桡动脉和肱动脉等。

最近的一项包括9个相关研究的被动腿抬高试验荟萃分析认为,被动腿抬高试验相关的心指数和每搏输出量变化优于脉搏压的变化来预测液体反应性,可喜的是,其中6个研究应用了超声技术,入选患者数居多,所以随着未来有关主动脉流速和外周动脉流速的研究的增加,或许会有不同结论产生。

当然,在完全机械通气时和任何心律情况下,无论此时能不能合理应用动态指标,也可选择应用被动腿抬高试验相关的超声指标。

3)选择容量负荷试验进行容量反应性评估:当以上的方法依然不能合理预测容量反应性时,最终在谨慎考虑输液限制情况下,还可以选择容量负荷试验。此时,可选择超声测量每搏输出量、心输出量和左心室舒张末期面积变化以及多普勒测量左室充盈压变化判断容量负荷试验。最近的研究表明,容量负荷试验前后应用外周动脉流速变化如股动脉流速变化同样可以预测容量反应性,应该说除需要承担液体过负荷风险外,在评估容量反应性上完全与被动腿抬高试验接近,甚至于更可靠些
\protect\hyperlink{text00009.htmlux5cux23ch8-8}{\textsuperscript{{[}8{]}}}
。

\subsubsection{超声容量反应性评估时的注意事项是什么?}

在评估容量反应性时,一定要认真考虑以下因素:①液体反应性的评估需要测量多个参数,因为没有任何一个指标是绝对和排他的,临床上应该结合具体临床情况联合应用,最终有助于准确评估容量反应性;②心脏超声获得的心肺相互作用评估容量反应性的动态指标不但有助于评估容量反应性,同时心脏超声易于发现非超声获得的动态指标的假阳性(尤其严重右心衰),但依然需要更多的研究来证明临床价值。

总之,心脏超声在评估前负荷及容量反应性方面可用、有效,且极具前景。在应用心脏超声时,无论评估的流程还是指标的选择均有一定科学内涵,应该在应用时进一步设计合理的临床研究来证实临床有效性,期待能够对死亡率和致残率以及并发症发生率产生深远的影响。

\subsection{左心室功能的超声心动图评估}

\subsubsection{左心室功能评估的要点是什么?}

心室收缩与舒张功能及其随时间变化的评价在重症患者中作用很大。由于超声心动图以二维图像来展示三维结构,所以在诊断或者治疗之前,每个结构至少要得到相互垂直的两个切面的图像。新出现的或者进一步恶化的室壁运动异常可能提示急性心肌缺血或者缺血所致损伤,而像重症感染等多种重症疾病所导致室壁运动异常并非心室局部的功能障碍,而是心室的整体功能异常,因此全心室收缩功能评估十分重要。

心室收缩功能同时依赖于心脏的前负荷和后负荷,所以必须在不同负荷状态下评估收缩功能才能确保得到真实结果。另外还要注意连续评估的重要性,不能仅仅依赖某一次评估的结果得出结论。压力容积关系是不依赖于容量状态的左心室心肌收缩力的评估方法。超声心动图中用来评估整个左心室收缩功能的定性和半定量测量指标有射血分数、缩短分数、面积变化分数、左心室功能评估的Simpson法、二尖瓣环运动、用二尖瓣反流束计算等容收缩压力增加速率、使用标准17-节段模型和应变率来评估局部室壁运动异常。最常用的方法是射血分数。

\subsubsection{左心室收缩功能定性评估的首要问题是什么?}

评价左心室的收缩功能时,首先要明确以下问题:心室充盈如何?心肌有足够的收缩力吗?在冠脉分布的范围内心肌收缩一致吗?

\subsubsection{如何运用左心室标准的17节段分法进行视觉评估左室功能?}

左心室功能评估的形式多种多样,如心脏MRI、超声心动图、核素扫描、血管造影等。为了能统一术语,美国心脏学会达成共识,将左心室分成17个不同的节段。沿心脏长轴左心室分为基底段、中段和心尖段,基底段和中段又各自进一步分为6个节段,尖段分为4个节段,再加上第17节段的心尖帽部。相应的冠脉分布为:左前降支提供心脏的前壁和前间壁前2/3的血供,左回旋支提供左心室侧壁的血供,右冠状动脉提供室间隔后1/3和左心室下壁的血供。室壁运动评分和指数可以用来进行半定量评估。左心室收缩力依赖心脏从基底部到心尖部的运动、室壁的厚度和左心室螺旋挤压和旋转运动。心室壁的切面厚度以及左心室局部心内膜运动幅度对心室壁运动的评估十分重要。室壁运动评分描述如下:

正常(>30%心内膜运动幅度,>50%室壁厚度);

轻度运动功能减退(10%~30%心内膜运动幅度,30%~50%室壁厚度);

严重运动功能减退(<20%心内膜运动幅度,<30%室壁厚度);

运动不能(心内膜运动幅度为零,<10%室壁厚度);

运动障碍(收缩期反常运动)。

室壁运动评分指数是指局部的室壁运动分数,是一种主观评估方法,分数之间没有真正意义的线性关系。缺乏血流灌注的心肌将表现为异常的室壁运动。只有多个切面的图像才能真正反映左心室受损情况和相应冠脉分布情况。仅仅是心内膜运动幅度的改变可能是心肌栓塞造成的,而室壁厚度改变是缺血的确切指征。经过多次室壁厚度的测量可以得出以下结论:沿长轴平面很难获得连续的室壁厚度数据;多角度多平面测量可以减小误差;确定边界、方位和角度值。

\subsubsection{什么是射血分数及测量方法?}

每搏输出量等于舒张末容积与收缩末容积之差。射血分数等于每搏输出量除以舒张末容积。可以在经胸壁超声心动图的左室长轴和短轴不同平面测量,但美国超声心动图学会建议使用修改后的Simpson法,计算两个平面的射血分数然后取平均值。该方法可通过经食管超声心动图的经中段食管切面、四腔切面、二腔切面进行计算。局限性在于测量时要求心内膜边界能清晰显示,而二尖瓣环的钙化通常会干扰心内膜边界的探查;在四腔切面中,因为超声束与心室侧壁平行,所以会出现侧壁信号丢失的情况;左心室内小梁形成也会干扰心内膜边界的探查。在这种情况下,使用造影剂能提高边界成像的清晰度。左心室尖部常因为透视原理而缩小。

\subsubsection{怎样进行左心室收缩功能的超声心动图定量评估?}

(1)心输出量的计算

心输出量=心率×每搏输出量

在重症医学科中,肺动脉导管可以用来测量心输出量。但目前的证据显示,肺动脉导管的使用并没有明显优势,所以超声心动图对心输出量的测定具有重要作用。左右心室的心输出量都可以通过超声心动图来测量。左心室心输出量测量的可重复性和准确性更高:

左心室流出道面积=左心室流出道半径\textsuperscript{2} ×3.14。

心率可以通过心电图测量,或者从一个速度-时间积分到另一个速度-时间积分进行推算。每搏输出量等于左心室流出道面积乘以左心室主动脉瓣收缩期射血速度-时间积分。当血液从左心室射进圆柱体形的主动脉,每搏输出量就可以通过圆柱体血液的高度来计算,而这个高度就是速度-时间积分。圆柱体形的底是左心室流出道,而流出道面积能够很容易进行计算。圆柱体的高,也就是速度-时间积分,是通过经胸壁超声心动图时的心尖五腔切面、经食管超声心动图(TEE)时经胃主动脉瓣切面或者经胃主动脉瓣长轴切面运用脉冲多普勒测量左心室流出道的血流得出。该参数的准确测定基于左心室流出道面积在收缩期恒定不变的基础之上。左心室流出道半径的测量误差将使面积计算的误差放大。为了使误差最小化,图像的灰度要减小,而左心室流出道要尽量大;另一个假设是通过左心室流出道的血流是层流。这个假设通过脉冲多普勒上的窄流速带和平滑的光谱信号来证实。将样本体积的液体流通过两个互相垂直的切面来解释液体流的中心流速和边缘流速相等,以此证实平均流速分布图的存在。需要强调的是,多普勒射束应该与血流平行或者<20°。多普勒信号记录的是与血流平行的拦截角,所以能准确测量血流速度。左心室流出道直径和脉冲多普勒应该在同一解剖位置进行测量以保持脉冲多普勒的空间与即时关系。选择某一个靠近动脉瓣的位置当作常规测量点可以减小误差。因为在不同心率下血流动力学有所不同,因此这些测量应该在同一时间点进行,当在不同时间点评估心输出量时,所有的测量都要重复进行。

(2)不同部位每搏输出量的测量 使用经食管超声心动图时,一般选择左心室流出道作为最主要的测量点,然后就是肺动脉和右心室流出道。经食管超声心动图测量每搏输出量时,可以选择在主动脉瓣瓣叶尖端或者升主动脉。升主动脉直径是从胸骨旁长轴切面测量的,从胸骨上切迹或者心尖部的经胸壁超声心动图切面测出。二尖瓣口每搏输出量也可以通过脉冲多普勒在二尖瓣瓣叶尖端测得。因为二尖瓣的复杂几何特征和大量的假设,一般不选择该处作为心输出量的常规测量点。在心脏的右侧,可以选择三尖瓣或者肺动脉来测量每搏输出量。右心室心输出量也可以测量得出。然而,大的肺动脉直径不是固定的,而是依赖于切面的不同而不同;另外,并非时时都能取到与右心室射出血流平行的多普勒图。

\subsubsection{左心室收缩功能的超声心动图半定量测量方法如何采用?}

(1)测量缩短分数 缩短分数是一种评价左心室整体收缩功能的一维测量方法。经左心室乳头肌短轴的M型超声能测量出该参数的值。M型超声的定格分析用来计算缩短分数。缩短分数=(左心室舒张期内径-左心室收缩期内径)/左心室舒张期内径×100(正常值>25%)。正常值在25%~45%之间。

缩短分数的测量是一种基本的粗糙的左心室整体收缩功能的评估方法,优点是快捷而且可重复性高,M型超声检查可以节约很多时间,而且心内膜边界显示非常清晰。在测量过程中需注意,如果局部心室壁存在异常运动,容易产生误差;一维平面的斜切可能导致长度测量的误差。因此,在这个半定量测量中加入另外维度的测量可以增加整体功能评估的准确性。

(2)测量面积变化分数 面积变化分数是测量左心室收缩功能的二维参数。测量的准确性依赖于获得足够清晰的心内膜边界,边界显示不清晰时进行描记是十分困难和耗时的。面积变化分数可以定量评估射血分数。面积变化分数=(左心室舒张末面积-左心室收缩末面积)/左心室舒张末面积×100%。正常值>50%~75%。

面积变化分数高度依赖后负荷,也一定程度依赖前负荷。其中,经胃乳头肌短轴切面计算的面积变化分数与放射性核素血管造影术测量有很好的相关性。

(3)等容收缩压力增加速率的测量 评价左心室功能指标在射血期很容易得到,但这些指标的负荷依赖性明显影响心室功能的客观和准确评估。等容收缩压力增加速率对心肌收缩能力的变化较为敏感,受前后负荷变化影响较小,对左室心肌收缩力的评估较为准确,可用来反映心肌收缩力的变化。测量方法如下:连续波超声多普勒测定二尖瓣反流的速度,测量从1m/秒增加到3m/秒所需时间。根据简化的伯努利方程(压力=4×速度\textsuperscript{2}
),等容收缩压力增加速率(dP/dt)可以表示为:dP/dt=32/Δt;即运用简化的伯努利方程,速度为3m/秒时,压力为4×3\textsuperscript{2}
=36mmHg;速度为1m/秒时,压力为4×1\textsuperscript{2}
=4mmHg,压力差为32mmHg。用压力差除以速度从1m/秒增加到3m/秒所需的时间Δt,等容收缩压力增加速率即可计算出来。正常值>1200mmHg/秒,小于1000mmHg/秒则为异常。左心室功能良好的状态下,该时间可以大大缩短。值得注意的是,测量该指数时患者必须存在二尖瓣反流。

(4)运用组织多普勒成像评估心室功能 组织多普勒成像是一种量化测量左心室整体和局部功能的手段。组织多普勒显示的二尖瓣环下行速度可以评估左心室的收缩功能。心肌组织速率一般在二尖瓣环的室间隔、侧壁、下壁、前壁、后壁和前间隔部位测量。从上述部位得到的二尖瓣环下行平均峰速度可以衍生出以下计算方程:左心室射血分数=8.2×二尖瓣环平均峰速+3%。

该方程可以评估心内膜边界显示欠佳患者的整体左心室功能,缺点在于不能鉴别真正的心肌运动与心肌被动牵拉运动或者心室的整体位移运动。这些参数能从节段性应变成像模式中获得。

(5)比较少用的左心室收缩功能半定量测量工具

1)压力-容积环 压力0容积环的Y轴代表压力,X轴代表容量,压力-容积环的斜率反映心肌收缩能力,不受心脏前后负荷的影响。左室收缩功能增强压力-容积环向左上移动,反之,收缩功能下降时移向右下。将心室不同前负荷所对应的不同环的收缩末压点相连,即可得到反映收缩末期压力-容积环的变化关系,也被称作弹量。该方法测定需要足够的时间,而且前负荷的改变易于影响患者病情的稳定,因此,不具有实用性,尤其不适用于重症患者。

2)室壁应力和左心室质量 室壁应力是指施加在单位心肌面积上的力,取决于心腔容积、压力和室壁厚度。室壁应力包括圆周、子午或径向三个方面。通常计算收缩末期的圆周及子午室壁应力。将心肌体积乘以特异的心肌密度即可计算出左室心肌质量。超声心动图可以通过评估左室流出道的收缩速度加速度以及心肌收缩的应变率得到收缩末弹量。心肌做功指数(Tei)是另一种心肌收缩功能的评估方法,通过等容收缩期与等容舒张期之和除以射血时间得到,然而心肌做功指数的临床实用性仍有争议。

(6)左心室收缩功能半定量测量新技术

1)运用组织多普勒、应变和应变率评估心功能 多普勒组织成像和斑点追踪成像是新近发明的测量局部心肌功能的重要方法。组织速度信号是一种低速信号,它通过除去室壁过滤,并使用低增益放大,使得心肌组织速率测定成为可能。放置在心肌特定部位获得的脉冲多普勒或定向的M超声都可以用来展示心肌组织速率。当室壁运动异常与标准评估相混淆时,可用组织多普勒来鉴别。多普勒组织成像的常见缺陷包括:只能测量与超声束平行的运动成分;不能鉴别心室平行的位移运动;不能鉴别被邻近组织牵拉的运动与正常收缩运动。应变和应变率可用来测量在超声扫描线上出现的变形。传感器定位十分敏感,比多普勒的角度依赖性更敏感。心肌峰速度、应变率以及应变能识别静息状态以及应激状态下的局部心肌功能异常。斑点追踪成像可避免角度依赖性,能得到更准确的组织速度、应变率和应变力,用于测量两个维度的变形。在静息、应激(应力)、局部缺血等状态下的局部功能是运用多普勒组织成像或者斑点追踪成像进行应变率和应变评估的指征,将其与三维斑点追踪成像技术相结合是评估左心室功能的有力工具。

2)有利于分辨心内膜边界评估心室功能的新技术 心内膜边界的清晰度在左心室功能评估中十分重要。处于不同状态时,如肥胖或者肺气肿的患者,心内膜边界不太清晰。超声心动图造影技术在这些患者中有重要作用。彩色室壁运动技术通过声学定量原理能够将组织和血液区分开来,自动勾勒出心内膜边界,能够动态定量分析左室功能。在有室壁瘤或者其他心室不对称等异常情况下,该方法的有效性需要进行校正。这种情况下,三维超声能够真实反映左心室功能。

\subsection{左心舒张功能评估}

\subsubsection{如何应用跨二尖瓣左心室充盈评估左心舒张功能?}

左心室的舒张功能与收缩功能同等重要,舒张功能正常可防止肺静脉淤血和心源性肺水肿。超声心动图检查可通过测定跨二尖瓣左心室充盈、肺静脉血流模式和二尖瓣环侧壁心肌速度来评估左心室舒张功能。

将脉冲多普勒取样窗放置在二尖瓣瓣叶尖端可以获得舒张早期最大流速E和心房收缩期最大流速A。正常左心室E峰一般大于A峰。左心室肥厚或老年患者,E/A比值<1,反映舒张功能受损。E峰加速度与左心房压力除以τ的比值成正比,其中τ是等容期左心室压力下降的指数时间常数。为了保证每搏输出量,在有进行性舒张功能障碍的患者中存在进行性左心房压力增高的代偿,以将受损的舒张形态逆转到假性正常化。当左心室功能严重受损,在很短的充盈时间内出现左房压的极度上升,表现为经典的减速时间减少和高E/A比值。这些参数都是随着前负荷的变化而改变,单凭这种评估方法不能鉴别舒张功能不全的所有形式,还可能造成一些病例的漏诊。一些特定方法像Valsalva试验等可以帮助鉴别假性正常化的形态和进行性左心室舒张功能障碍。

\subsubsection{如何应用肺静脉血流脉冲多普勒评估左心舒张功能?}

肺静脉血流脉冲多普勒是一种通过评估跨二尖瓣充盈来诊断心室舒张功能障碍的辅助手段。将脉冲多普勒放置在肺静脉入左心房开口的远心端,能得到收缩波S、舒张波D和心房波A。在心房收缩产生的心房逆转波大小和形态最有临床应用价值。跨二尖瓣时间与肺静脉A波时间的差值有助于预测左心室舒张末压。

\subsubsection{如何应用M型彩色多普勒测量血流加速度?}

舒张期通过二尖瓣血流的时空图与左心室舒张有关,而这个时空图就是血流加速度。将彩色多普勒取样窗放置在左心室流入道,再将M型取样线穿过此窗口即可获得血流加速度。将色彩基线调整至最大二尖瓣口流速的30%~40%,然后计算红蓝渐变斜率即可计算血流加速度。与跨二尖瓣口血流充盈评估相比,血流加速度一般不会出现假性正常化,当其<45cm/秒提示左室舒张功能障碍。该方法的主要局限性是可重复性不高。当跨二尖瓣血流充盈和肺静脉脉冲多普勒相结合在左室舒张功能不全的诊断中不明确时,多普勒组织成像在外侧二尖瓣环获得的E峰、A峰以及血流加速度等附加标准有助于鉴别舒张功能障碍的程度。

\subsubsection{如何进行左心室充盈压评估?}

肺动脉导管可以用来测量左心室充盈压。在没有任何远端梗阻情况下的肺小动脉嵌顿压近似于舒张末期左心室压力,在左心室顺应性正常的情况下,该压力可以间接反映左心室舒张末期容积,也就是左心室前负荷。而在高龄或者高血压患者中,左心室肥厚以及左心室顺应性降低较常见,导致舒张末期左心室压力与左心室舒张末期容积关系发生改变。此时,超声心动图检查有助于评估左心室舒张末期压力和舒张功能。常用的指标为左心室的被动跨二尖瓣充盈(E峰)和与之相对应的侧面二尖瓣环移位(E'峰)关系及比值,比值>15,提示左心室舒张末压>15mmHg;比值<8,提示左心室舒张末压<15mmHg。E'速度<5cm/秒则提示心室顺应性减低。

\subsubsection{如何对左心室容积进行半定量评估?}

通过压力测量来评估左心室容量状态是临床常用的方法。然而,对于部分特定的患者,特别是机械通气患者,压力与充盈容积的对应关系并不准确,因此,压力指标不能准确反映患者容量状态。而超声心动图中有很多方法评估左心室容积和压力,既可以单次使用,也可以重复应用以监测患者对补液的反应。因此,在临床的应用逐步得到推广。左心室具有对称性,有两个相对相等的短轴,而长轴从心底指向心尖。长轴方向心尖较圆钝,近心尖侧左心室为半椭圆形,而心底侧为圆柱形,所以在短轴切面呈圆形。因此,在测量和计算左心室容积时,可假设为M型超声或者二维切面时的形状。但使用这些参数来评估正常或者异常形状的左心室时仍需要谨慎分析。

左心室舒张末容积、左心室舒张末表面积、上腔静脉塌陷率、下腔静脉宽度、容量反应性等都可用来评估左心前负荷。低血容量的诊断指标包括舒张末直径<25mm、左心室腔收缩闭塞和左心室舒张末表面积<55cm\textsuperscript{2}
。在经食管超声心动图的经胃乳头肌短轴平面可以比较容易得出这些参数。存在基础心脏疾病或者左心室低顺应性的患者,左心室的压力容积关系都将改变,最适左心室舒张末表面积将比正常人的更大。这就突出了对于既定的左心室舒张末表面积与每搏输出量测量的匹配关系。相对于单次测量结果,连续测量左心室舒张末容积更加可靠,但非常耗时,同时在实践中很难实现。追踪容量状态变化能证实与左心室舒张末表面积测量的相关性,左心室舒张末表面积是通过追踪上述切面的左心室舒张末静态轮廓来计算的。此过程可以通过使用自动声学定量边界监测系统来简化。收缩末与舒张末的容积都应进行检测,随着时间的推移,还可以追踪容量状态的变化。收缩末心室腔闭塞或者叫“乳头肌亲吻征”是低血容量的征象,预测心室收缩末表面积减少的敏感性达100%,但特异性只有30%。

二尖瓣环(E')的组织多普勒成像与二尖瓣口E波血流模式相结合可以预测左心室顺应性和平均舒张压。E/E'比值<8表示心室顺应性良好,>15表示左心室平均充盈压高,顺应性低。中间值的评估还需要结合其他参数,比如肺静脉血液回流和二尖瓣流入减速时间。

\subsection{左心室功能评估的新技术}

\subsubsection{如何利用三维技术进行左心室功能评估?}

实时图像重建能获得左心室图像。当进行三维图像重建时,通过一个固定的传感器在3°或5°标准下可获得一系列的二维图像。平面的数量和二维图像的质量共同决定三维图像的质量。矩阵阵列传感器的发明使得多线图像同时用于重建一组超声数据。但对于左心室,需要将连续心动周期获得的数据组整合起来进行评估。

左心室容量和功能也能通过三维方法来计算。与MRI相比,该方法观察者之间的主观误差少,图像重建的假设成分少,因而能够更加准确评估左心室的前负荷和射血分数。随着图像分析时间的进一步减少以及更多先进科技的出现,三维未来将成为评估重症患者左室容积和功能的最好方法,但该方法也有一定的局限性:三维容积中的线条密度比二维图像低,所以经常需要填描;当图像是从垂直于很多器官的固定传感器得到的,那么结果会是质量欠佳的图像。另外,随着呼吸运动和心律失常会出现结果的伪像。

\subsection{右心功能的评估}

\subsubsection{如何评估右室收缩功能?}

因为右室缺乏特殊的形态,心脏超声很难定量评估右室功能。因此,在正常和疾病状态下,通常仅能对右室形态大小与功能进行定性评估。判断右室扩张程度、室间隔左向偏移及运动情况是定性评估右室功能常用的基本方法。近年来,有研究逐步探索定量评估右心大小及功能的指标和方法,这些指标包括面积变化分数、三尖瓣环位移、组织多普勒三尖瓣环心肌收缩速度和心肌做功指数。最近三维超声技术的发展将进一步有助于临床准确评估右室大小及功能。其他的复杂技术如应变与应变率等目前仅在有经验的实验室作为特殊临床或试验研究应用,尚未应用于临床。

\subsubsection{如何评估右室舒张功能?}

对于右室功能障碍的患者,应测定右室舒张功能。三尖瓣E/A比、E/E'比及右房大小,已被证明均是有效的指标。右室舒张功能的分级如下:三尖瓣E/A比<0.8,提示松弛不良;三尖瓣E/A比0.8~2.1、同时E/E'比>6或肝静脉舒张期流量显著,提示假性充盈;三尖瓣E/A比>2.1、结合减速时间<120毫秒提示限制性充盈。进一步的研究需要针对上述指标的敏感性及特异性进行探讨,并研究分级与患者预后间的关系。

\subsection{超声在感染性休克循环支持中的作用}

\subsubsection{感染性休克的血流动力学特点是什么?}

感染性休克是重症患者转入重症医学科的常见原因之一。感染性休克的分子病理生理学机制复杂,以外周血管阻力降低、有效循环血量减少和组织灌注不足为特征的血流动力学改变是其显著的临床特点,因此超声心动图在感染性休克患者的病情监测和床旁管理中逐步得到应用。感染性休克的病理生理学特点包括低血容量、左室收缩和舒张功能障碍、右室收缩功能障碍及外周血管麻痹。超声心动图使重症医学科医师能识别这些过程,监控其发展,并采取相应的治疗性干预。

\subsubsection{感染性休克的容量特点是什么?}

感染性休克患者的容量特点是有效循环血量不足。表现为绝对或相对低血容量。绝对低血容量是指总循环血量减少,常为感染性休克早期的表现,需要立即纠正,常见的原因包括:非显性丢失,如由于发热、出汗和过度通气经皮肤和呼吸道丢失所致;经胃肠道丢失,如腹泻和呕吐;经第三间隙丢失,如胰腺炎、烧伤、软组织损伤、血管渗漏、低胶体渗透压、腹水、胸水;液体摄入过少,如精神状态改变、身体虚弱、医院内液体复苏不足。

相对低血容量由血液在外周和中心腔室内异常分布所致。相对血容量不足在感染性休克中常见,并可在初步液体复苏后持续存在。这类患者总血容量可能正常,但血容量分布在中心腔室以外。血管扩张是由于外周血管收缩机制障碍和血管扩张机制的异常激活所致。

无论低血容量是绝对、相对还是混合性,导致的后果一致,均表现为组织氧供减少和组织缺氧。液体复苏通过增加静脉回流、前负荷、心输出量和动脉压(收缩压、平均压和脉压)来改善感染性休克的容量状态。识别并纠正低血容量状态是感染性休克治疗的一个重要目标。

\subsubsection{感染性休克时左室收缩功能障碍的特点是什么?}

感染性休克患者常出现心肌收缩障碍。实验和临床研究表明多种因素共同作用导致感染性休克产生心肌功能抑制,如心肌水肿、心肌细胞凋亡、细胞因子作用(尤其是白介素-1、白介素-6和肿瘤坏死因子-α)以及一氧化氮激活。虽然无冠脉灌注和心肌能量代谢异常,但肌钙蛋白水平升高却很常见。

由于传统的左室收缩功能的超声心动图参数受左室前后负荷的影响,因此,超声心动图识别左室收缩功能障碍很难。如心室前负荷降低而血管扩张导致的低血压患者射血分数可以正常。在容量复苏和使用血管加压药物调整合适的前后负荷前提下,再进行超声心动图检查才能真正显示心室收缩功能的改变。而前后负荷的进一步变化又可以改变超声心动图的结果。因此,射血分数正常并不能排除左室功能障碍。临床和实验研究显示,感染性休克发生早期出现可逆的左室功能抑制,表现为左室压力容积曲线右移,射血分数下降,警示临床医生可能需要控制后负荷和给予强心治疗。感染性休克中左室收缩功能障碍的改善与生存率变化的关系仍存在争议。Parker等的研究首先显示两者具有相关性;但Vieillard-Baron等进行的研究没有得出类似的结果。有假说认为感染性休克左室扩张与收缩功能受到抑制有关,是心脏为维持心输出量而做的适应性改变,该假说已被部分超声心动图检查所证实。

\subsubsection{感染性休克时左室舒张功能障碍特点是什么?}

感染性休克常伴有左室舒张功能障碍,并与死亡率增加相关。这主要与肌钙蛋白水平升高、细胞因子活性(肿瘤坏死因子-α、白介素-8、白介素-10)增加有关。舒张功能障碍常与收缩功能障碍同时发生,但约20%的病例单独出现。

\subsubsection{超声心动图在感染性休克管理中的应用特点是什么?}

有效循环血量降低在感染性休克患者中很常见,而早期足够的容量复苏与患者的预后显著相关。因此,临床治疗中不能因等待超声心动图检查而延迟液体复苏。入院前和急诊的临床评估有助于获得初步的信息来决定容量复苏的补液量。入住重症医学科后需要关注的问题是患者是否还需要进一步进行容量复苏、是否需要继续调整血管活性药物的使用。在这种情况下,超声心动图是评估容量状态和心功能的理想工具,有助于识别低血容量、评价左室收缩期和舒张期功能障碍和右室功能障碍。最初的评估结果有助于制定治疗计划,而后续治疗过程中的监测有助于评估治疗效果、疾病变迁并识别新问题的出现
\protect\hyperlink{text00009.htmlux5cux23ch9-8}{\textsuperscript{{[}9{]}}}
。

\subsubsection{超声心动图如何评估感染性休克患者的容量反应性?}

对感染性休克患者进行容量复苏是初始复苏的重要部分,但容量复苏过度则导致相反的后果。利用床旁超声心动图检查可评估容量状态和容量反应性,常选用动态容量指标来进行评价。

下腔静脉直径的呼吸变异是判断容量反应性的有效方法,但要求患者必须有机械通气支持并完全没有自主呼吸。此外,超声心动图显示感染性休克患者小的高动力左室(收缩末左室腔消失)或小的下腔静脉直径(<10mm)提示患者存在容量反应性。

具有高级重症超声心动图检测能力的重症医学医师能通过多种多普勒方法来了解感染性休克患者是否需要进一步容量复苏。对于无自主呼吸、窦性心律的机械通气患者,可用经食管超声心动图测得的上腔静脉直径的呼吸变异测定容量反应性,也可通过多普勒测得的每搏输出量的呼吸变异进行判断。对于有自主呼吸和心律不齐的患者,可采用被动抬腿前后用多普勒测量每搏输出量和心输出量判断容量反应性。

\subsubsection{超声心动图如何评估感染性休克患者的左室收缩功能?}

感染性休克早期,常出现左室收缩功能受损,且通常在感染性休克恢复后7~10天完全恢复。感染性休克患者血流动力学改变呈“高动力”状态的高排低阻表现。对心脏功能非容量依赖性指数的研究显示,即使心输出量和射血分数正常或升高,但患者仍表现为收缩功能损害。超声心动图检查结果易于将高动力的左室收缩误读为左室充盈不足和后负荷过低。进行容量复苏和血管活性药物治疗调整后负荷后,超声心动图检查可以确切显示左室收缩功能受损
\protect\hyperlink{text00009.htmlux5cux23ch10-8}{\textsuperscript{{[}10{]}}}
。

左室收缩功能的评价依赖于射血分数的测定。超声心动图检查可以通过多种方法测定射血分数值。M型超声依赖于左室直径的测量。Teichholz方法测量的技术要求较高,要求在心室中央水平和胸骨旁长轴测量左室的直径,M型探头与左室壁垂直,重症患者往往心脏难以朝向适合测量的方向,加上由机械运动周期导致的平移运动伪影和用直径测量来定义复杂的三维结构所导致的内在的几何假设,M型射血分数测量方法可能不是测量重症患者射血分数的可靠方法。另外,该方法不能用于有室间隔异常的患者,机械通气的重症患者是否有效尚未得到证实。另一种方法是面积测量法。在胸骨旁短轴的乳头肌水平(使用经食管超声心动图)测量舒张末和收缩末左室腔的面积。尽管在理论上该方法优于基于直径的测量方法,面积测量法仍然易受室间隔异常和平移运动伪影的影响。准确测定射血分数可以采用Simpson方法,通过2个直角平面的顶面观来测量左室舒张末和收缩末面积(顶面四腔和顶面二腔视图)。该方法测定费时、需要明确心内肌边界、较高的测量技术(如理想的轴线和避免平移运动伪影)以及高质量的设备。

射血分数测定有助于评价左室收缩功能,但不能反映每搏输出量和心输出量。低灌注高动力的左室可以表现为射血分数正常,而每搏输出量和心输出量可能不足。同样,扩张而收缩功能下降的左室射血分数虽低,每搏输出量和心输出量可能并不降低。因此,临床治疗中往往需要测量每搏输出量和心输出量,这需要使用多普勒进行测定。在经胸壁超声心动图心尖五腔切面或经食管超声心动图胃深部视图测量,多普勒探头的脉冲波置于左室流出道,超声波束与血流方向平行。主动脉收缩期血流速度曲线下面积与每搏输出量成正比。主动脉收缩期血流速度时间积分乘以左室流出道面积即得到每搏输出量和心输出量。射血分数反映左室收缩功能,而每搏输出量和心输出量反映氧输送。感染休克早期的检查可能显示射血分数显著下降。恢复期检查可显示左室功能完全正常,这为患者的临床管理提供了重要信息。如果没有再次检查,患者可能被视为有慢性左心衰,从而进行不恰当的长期治疗。

\subsubsection{超声心动图如何评估感染性休克的左室舒张功能?}

感染性休克患者常出现左室舒张功能异常。舒张功能的测定非常重要,有助于评估左室舒张压和左房压,评价左心室对容量的耐受性,以尽早采取有效的治疗手段防止左室舒张压升高导致肺动脉压升高和肺水肿。一旦发现左室舒张末期压力升高可以及时采取治疗性干预,如限制液体输注和利尿,以保证在改善组织灌注的同时降低肺水肿发生风险。

传统测量方法依赖于多普勒分析负荷依赖性的二尖瓣流入量,也可以通过非负荷依赖性方法测量二尖瓣环组织的纵向运动多普勒速度(E')。另外,多普勒超声心动图检查可通过多种方法评估肺动脉嵌顿压。采用多普勒脉冲在顶面四腔视图上测量跨二尖瓣舒张期流速,E/A>2与肺动脉嵌顿压力>18mmHg显著相关,其阳性预测值为100%;收缩期前向运动速度/收缩期和舒张期速度<45%提示肺动脉嵌顿压力>12mmHg,其阳性预测值为100%;肺静脉反向A波时间大于二尖瓣流入A波时间提示肺动脉嵌顿压力>15mmHg,阳性预测值为83%;二尖瓣环组织多普勒测量二尖瓣E波速度比E'(E/E')>9提示肺动脉嵌顿压力>15mmHg。

\subsubsection{超声心动图如何评估感染性休克的右室功能?}

感染病原菌、毒素、炎症介质、感染性休克并发症等同样可损害右室功能。急性肺损伤、缺氧性肺血管收缩和正压通气都可能增加右室后负荷而导致急性肺心病。超声心动图有助于识别急性肺心病,从而有利于采取措施降低右室后负荷,缓解右室扩张。

\subsubsection{如何利用超声心动图对血管外周阻力进行评估?}

心脏超声多普勒技术可以直接测量外周血管阻力,但不易方便和简单使用,因此在临床工作当中,较少应用超声心动图检查评价外周血管阻力,而常根据临床和心脏超声的检查结果进行除外诊断,如在心脏前负荷充足的同时左右心脏收缩功能均满意的情况下仍然存在低血压,提示外周血管阻力降低。

\subsubsection{超声心动图在感染性休克管理中的临床应用流程是什么?}

低血容量、有效循环血量降低导致组织灌注不足是感染性休克的最主要特点。除立即使用有效抗生素抗感染治疗,早期的治疗应给予足量的容量复苏合并使用血管活性药物以改善组织灌注。该治疗常在重症医学科外已开始执行。患者转入重症医学科后,医师需要进行评估并进一步制定治疗计划。初步超声心动图检查首先有助于排除其他或并存的导致休克的原因,如早期未发现的心包填塞、严重瓣膜疾病、室间隔异常、缺血性心肌病或肺栓塞;其次有助于进行血流动力学评估,以指导进一步容量管理和血管活性药物的调整。

超声心动图检查显示以下特征性的改变时,提示需要继续进行容量复苏:①显示下腔静脉直径小或高动力的左室、收缩末室腔消失;②没有自主呼吸的机械通气患者,下腔静脉直径或每搏输出量随呼吸发生显著的变异;③有自主呼吸的机械通气患者,测量的被动腿抬高试验变异度>12%。

超声心动图检查有助于评价左心功能,指导血管活性药物的使用和调整。感染性休克患者常合并左室收缩功能下降,但并不说明患者一定需要使用血管活性药物。通过超声心动图检查有助于判断患者是否需要应用正性肌力药物。最常用的方法为直接测量每搏输出量和心输出量。超声检查即使显示左室收缩功能降低,但如果每搏输出量和心输出量在正常范围,没有必要使用强心治疗;如果每搏输出量和心输出量降低以至氧供减少,则有使用正性肌力药物的指征。如果无法进行量化的每搏输出量和心输出量测量,需要综合临床表现来决定是否使用正性肌力药物。

超声心电图检查有助于识别患者有无急性肺心病。多种因素可导致感染性休克患者出现急性肺心病。如细菌毒素、炎症介质、不恰当的机械通气治疗等。右室扩张和室间隔运动障碍,对急性肺心病有重要诊断意义。急性肺心病的识别有助于临床医师及时采取有效措施降低右室后负荷。

\subsection{超声心动图与重症相关心肌梗死}

\subsubsection{超声如何早期发现重症相关心肌梗死?}

无论是围手术期还是严重创伤的重症患者,缺血性心脏病常见,心肌局部缺血导致局部心肌运动异常。临床实际中,超声检查评估局部心肌缺血最常用的方法是进行二维超声显像检查,目测室壁运动和室壁增厚率。与心肌节段的室壁增厚率相比,二维超声应变成像对心肌缺血的变化更加敏感。

心肌应变是指心肌各节段的变形,与心肌的收缩和舒张功能密切相关,因此超声检查心肌应变可用于评估心肌收缩和舒张功能。

随着彩色多普勒心脏超声在临床的广泛运用,使急性心肌梗死后心脏功能、包括左室舒张功能异常得到全面深入的认识,对临床治疗方案的制定也起到重要作用。急性心肌梗死后可出现左心室舒张功能异常,表现为二尖瓣血流频谱E峰峰值速度减低,A峰峰值速度增高,E/A比值<1,E峰减速时间延长,等容舒张时间延长,肺静脉血流频谱S/D峰值比值增加等。

\subsection{超声心动图与急性肺动脉栓塞}

\subsubsection{超声心动图如何早期发现急性肺动脉栓塞?}

急性肺血栓栓塞是临床上一种危重的心肺疾病,超声心动图检查对其病变程度、治疗效果及预后评估有重要作用,已经普遍应用于临床。超声检查急性肺血栓栓塞应心脏超声检查及下肢深静脉检查。心脏超声可以从直接征象及间接征象为诊断急性肺血栓栓塞提供重要辅助诊断依据,其中,直接征象包括肺动脉和左右肺动脉主干内血栓;右心内血栓伴右心扩大、肺动脉高压;血栓到达肺动脉以前,可被腔静脉入右房处的Eustachil瓣、三尖瓣或右心耳阻截,如果同时伴有右心室扩大或肺动脉高压,则可以直接诊断急性肺血栓栓塞。

心脏超声检测急性肺血栓栓塞的间接征象包括肺动脉高压及肺源性心脏病征象。具体表现在以下几方面:栓子栓塞肺动脉,受机械、神经反射和体液因素的综合影响,肺血管阻力升高,右心后负荷增大,导致右心系统扩大;右室壁运动幅度减低;室间隔与左室后壁运动不协调,在左室短轴切面,室间隔向左心室膨出,左心室呈“D”字形改变;由于右心扩大,导致三尖瓣瓣环扩大,可引起不同程度三尖瓣反流及肺动脉压力增高,频谱多普勒可以测得三尖瓣反流压差,并可据此估测肺动脉压力;此外,还可见多普勒改变、肺动脉血流流速曲线发生特征性改变,主要表现为加速、减速时间缩短及频谱形态发生改变,如果伴有肺动脉高压,则血流频谱表现为收缩早期突然加速,加速支陡直,峰值流速前移至收缩早期,而后提前减速,呈直角三角形改变,有时可于收缩晚期血流再次加速,出现第二个较低的峰。

心脏超声可通过上述直接征象来直接诊断急性肺血栓栓塞,但临床检查发现直接征象的概率往往较低,主要原因为:当肺栓塞栓子位于肺动脉外周血管时,往往难以检出;新鲜的血栓回声多较低,超声不易识别;而机化的血栓与血管壁融合,也不易区分。间接征象可以提示诊断,更重要的是对具有胸痛、呼吸困难、心悸、气短等症状的患者进行鉴别诊断,主要与冠心病、急性心肌梗死、主动脉夹层、心包积液等疾病鉴别。对于确诊的急性肺血栓栓塞患者,如超声探测到中度、重度右室功能障碍,则其近期及长期病死率明显升高,而不伴有右室负荷过重的患者,近期预后良好。可见,除辅助诊断外,心脏超声检查还能够根据右室功能状态进行疾病危险度分层及预后判断。由于心脏超声可以动态、无创、重复估测肺动脉压力,因此也是疗效判断、随访追踪的一种快速、简便的检查手段。

\subsection{肺部超声在循环监测与支持中的作用}

\subsubsection{常见的肺部超声征象包括哪些?}

最近几年来,随着肺部超声的进步与推广,超声检查成为肺部和胸腔疾病诊疗的重要手段。正常和疾病状态下肺部超声常见的特征性的表现有:①正常通气征象------胸膜线下平行排列的A线;②肺间质肺泡综合征------彗星尾征,根据B线的不同间隔分为B7线(B线间隔大约7mm,主要是肺小叶间隔增厚)和B3线(B线间隔3mm);③肺实变征------组织样征和碎片征,可见支气管气象;④胸腔积液征象------静态征象为四边形征,动态征象包括水母征和正弦波征;⑤气胸征象------平流征,超声诊断气胸的优势是快速、直接。

\subsubsection{如何认识肺部超声对血流动力学性肺水肿的评估作用?}

血流动力学性肺水肿患者通常需要进行肺水含量的评估。肺部超声检查获得的B线提示患者出现肺水肿,该表现往往出现在血气分析改变之前。另外,超声具有简单、无创、无放射性和实时性等特点,可以实时监测肺水肿的改变。例如,随着肺水肿的增加,由肺间质水肿发展为肺泡水肿,肺部超声检查的B线也相应发生变化
\protect\hyperlink{text00009.htmlux5cux23ch11-8}{\textsuperscript{{[}11{]}}}
\textsuperscript{,}
\protect\hyperlink{text00009.htmlux5cux23ch12-8}{\textsuperscript{{[}12{]}}}
。

\subsubsection{如何利用超声监测鉴别急性心源性(血流动力学性)肺水肿与急性呼吸窘迫综合征肺水肿?}

肺部超声监测导向诊断的难点在于鉴别急性心源性(血流动力学性)肺水肿和急性呼吸窘迫综合征肺水肿。最新有研究对比急性呼吸窘迫综合征与急性心源性(血流动力学性)肺水肿超声征象的不同。研究纳入7个征象:肺泡间质综合征、胸膜线异常征象、胸膜滑动征消失、存在未受损伤的区域、肺部实变、胸腔积液和肺搏动征。研究结果表明:由于两种疾病发病的病理生理机制不同,肺部超声表现也不同。心源性肺水肿时,超声肺彗星尾征的绝对数量与血管外肺水含量明显相关,甚至随着肺部含水量的增加从黑肺到黑白肺直至白肺发展;急性呼吸窘迫综合征时,早期CT能发现的所有特点包括肺部及胸腔改变均可由肺部超声检查发现,包括不均匀的含有未受损伤区域的肺部间质综合征、胸膜线异常改变及肺实变和胸腔积液等。可见肺部超声有助于床旁即时鉴别诊断急性呼吸窘迫综合征肺水肿与急性心源性(血流动力学性)肺水肿
\protect\hyperlink{text00009.htmlux5cux23ch13-8}{\textsuperscript{{[}13{]}}}
\textsuperscript{~}
\protect\hyperlink{text00009.htmlux5cux23ch15-8}{\textsuperscript{{[}15{]}}}
。

\subsubsection{肺部超声如何估测肺动脉嵌压?}

在循环支持的过程中,有研究表明,肺超的A-优势型表现提示肺动脉嵌压<13mmHg的可能性大,而在B-优势型时,提示肺动脉嵌压>18mmHg的可能性较大。

\subsection{重症肾脏超声在循环监测及休克支持中的作用}

\subsubsection{肾脏超声在休克循环监测中也具有重要作用吗?}

肾脏是休克时最容易受损或最早受损的器官之一,术后患者发生率达到1%,而在重症患者则达到35%,尤其感染性休克患者发生率在50%以上。因此肾功能的评估和急性肾损伤的早期诊断非常重要
\protect\hyperlink{text00009.htmlux5cux23ch16-8}{\textsuperscript{{[}16{]}}}
。重症肾脏超声能够床旁及时、无创监测肾脏大循环与微循环的改变,为休克循环监测提供新的诊断依据。

\subsubsection{在循环监测及休克支持中如何应用肾脏超声?}

近年来,应用超声多普勒技术监测肾脏阻力指数成为评估肾脏灌注的重要工具。过去的研究表明,肾脏阻力指数与疾病的进展明确相关,建议肾脏阻力指数用于监测肾脏移植后功能不全、尿路梗阻等。近年,由于超声监测肾脏阻力指数无创、简单、可重复性强,成为重症患者首选监测急性肾损伤发生发展的重要工具,尤其有益于调整休克的血流动力学策略。另外,由于超声造影技术的进展,使床旁定量实时监测大血管与微血管血流成为可能,尤其对于休克时肾脏灌注的变化,包括对于治疗干预的变化均有重要的监测价值。

重症超声是重症医学科中指导血流动力学监测和治疗的有效方法,它为重症医学提供了连续动态管理重症患者的重要床旁工具。

\begin{center}\rule{0.5\linewidth}{\linethickness}\end{center}

参考文献

\protect\hyperlink{text00009.htmlux5cux23ch1-8-back}{{[}1{]}} .Morris
C,Bennett S,Burn S,et al.Echocardiography in the intensive care
unit:current position,future directions.JICS,2010,11:90-97.

\protect\hyperlink{text00009.htmlux5cux23ch2-8-back}{{[}2{]}} .Danilo
T,Marcelo L,Tatiana M,et al.Echocardiography for hemodynamic
evaluation in the intensive care unit.Shock.2010,34S(1):59-62.

\protect\hyperlink{text00009.htmlux5cux23ch3-8-back}{{[}3{]}} .Price
S,Nicol E,Gibson DG,et al.Echocardiography in the critically
ill:current and potential roles.Intensive Care Med,2006,32:48-59.

\protect\hyperlink{text00009.htmlux5cux23ch4-8-back}{{[}4{]}} .Gerstle
J,Shahul S,Mahmood F.Echocardiographically derived parameters of
fluid responsiveness.Int Anesthesiol Clin.2010,48(1):37-44.

\protect\hyperlink{text00009.htmlux5cux23ch5-8-back}{{[}5{]}}
.Vieillard-Baron A,Caille V,Charron C,et al.The actual incidence of
global left ventricular hypokinesia in adult septic shock.Crit Care
Med,2008,36:1701-1706.

\protect\hyperlink{text00009.htmlux5cux23ch6-8-back}{{[}6{]}} .Price
S,Via G,Sloth E,et al.World Interactive Network Focused On Critical
UltraSound ECHO - ICU Group:Echocardiography practice training and
accreditation in the intensive care:document for the World Interactive
Network Focusedon Critical Ultrasound(WINFOCUS).Cardiovasc
Ultrasound,2008,6:49.

\protect\hyperlink{text00009.htmlux5cux23ch7-8-back}{{[}7{]}} .Vincent
Caille1,Jean-Bernard Amiel,Cyril Charron,et al.Echocardiography:a
help in the weaning process.Critical Care,2010,14:R120.

\protect\hyperlink{text00009.htmlux5cux23ch8-8-back}{{[}8{]}} .Salem
R,Vallee F,Rusca M,et al.Hemodynamic monitoring by echocardiography
in the ICU:the role of the new echo techniques.Current Opinionin
Critical Care,2008,14(5):561-568.

\protect\hyperlink{text00009.htmlux5cux23ch9-8-back}{{[}9{]}}
.王小亭,刘大为,张宏民,等.扩展的目标导向超声心动图方案对感染性休克患者的影响.中华医学杂志,2011,91(27):1879-1883.

\protect\hyperlink{text00009.htmlux5cux23ch10-8-back}{{[}10{]}}
.王小亭,刘大为.重视心脏多普勒超声在重症医学领域中的应用.中华内科杂志,2011,50(07).

\protect\hyperlink{text00009.htmlux5cux23ch11-8-back}{{[}11{]}}
.Bellani G,Mauri T,Pesenti A.Imaging in acute lung in jury and acute
respiratory distress syndrome.Curr Opin Crit
Care,2012,18(1):29-34.

\protect\hyperlink{text00009.htmlux5cux23ch12-8-back}{{[}12{]}} .Rajan
GR.Ultrasound lung comets:a clinically useful sign in acute
respiratory distress syndrome/acute lunginjury.Crit Care
Med,2007,35(12):2869-2870.

\protect\hyperlink{text00009.htmlux5cux23ch13-8-back}{{[}13{]}}
.Jambrik Z,Gargani L,Adamicza A,et al.B-lines quantify the lung
water content:a lung ultrasound versus lung gravimetry study in acute
lung injury.Ultrasound Med Biol,2010,36(12):2004-2010.

{[}14{]}.Copetti R,Soldati G,Copetti P.Chest sonography:a useful
tool to differentiate acute cardiogenic pulmonary edema from acute
respiratory distress syndrome.Cardiovasc
Ultrasound,2008,29(6):16.

\protect\hyperlink{text00009.htmlux5cux23ch15-8-back}{{[}15{]}}
.王小亭,刘大为.超声监测导向的ARDS诊断与治疗.重症医学年鉴,2012.

\protect\hyperlink{text00009.htmlux5cux23ch16-8-back}{{[}16{]}} .Le
Dorze M,Bouglé A,Deruddre S,et al.Renal Doppler Ultrasound:A New
Tool to Assess Renal Perfusion in Critical
Illness.Shock,2012,37(4):360-365.

\protect\hypertarget{text00010.html}{}{}


\chapter{药物相互作用}

\section{概述}

药物相互作用(Drug-Drug
Interaction,DDI)是指同时或相继使用两种或两种以上药物时,由于药物之间的相互影响而导致其中一种或几种药物作用的强弱、持续时间甚至性质发生不同程度改变的现象。

药物相互作用有广义和狭义之分。广义药物相互作用是指联合用药时所发生的疗效变化。疗效变化虽然有多种多样表现,但结果只有两种可能,即作用加强或作用减弱。从临床角度考虑,作用加强可表现为疗效提高,也可表现为毒性加大;作用减弱可表现为疗效降低,也可表现为毒性减轻。虽然多药联用的情况非常普遍,但药物相互作用常常只在对患者造成有害影响时才引起充分注意。狭义的药物相互作用通常是指两种或两种以上药物同时或相继使用时产生的不良影响,可以是药效降低甚至治疗失败,也可以是毒性增加,这种不良影响是单一药物应用时所没有的。

一个典型的药物相互作用对(interaction
pair)由两个药物组成:药效发生变化的药物称为目标药(object drug或index
drug),引起这种变化的药物称为相互作用药或促发药(interacting
drug或precipitating
drug)。一个药物可以在某一相互作用对中是目标药(如苯妥英钠-西咪替丁),而在另一相互作用对中是相互作用药(如多西环素-苯妥英钠)。

\subsection{按发生机制分类}

\subsubsection{体外药物相互作用}

体外药物相互作用是指在患者用药之前(即药物尚未进入机体以前),药物相互间发生化学或物理性相互作用,使药性发生变化。即一般所称化学配伍禁忌或物理配伍禁忌,故又称之为物理化学性相互作用。

\subsubsection{药动学相互作用}

药物在其吸收、分布、代谢和排泄过程的任一环节发生相互作用,均可影响药物在血浆或其作用靶位的浓度,最终使其药效或不良反应发生相应改变。

\subsubsection{药效学相互作用}

两种或两种以上的药物作用于同一受体或不同受体,产生疗效的协同、相加或拮抗作用,而对药物的血浆或作用靶位的浓度可无明显影响。

应当注意的是,有时药物相互作用的产生可以是几种机制并存。

\subsection{按严重程度分类}

\subsubsection{轻度药物相互作用}

造成的影响临床意义不大,无须改变治疗方案。如对乙酰氨基酚能减弱呋塞米的利尿作用,但并不会显著影响临床疗效,也无须改变剂量。

\subsubsection{中度药物相互作用}

药物联用虽会造成确切的不良后果,但临床上仍会在密切观察下使用。如异烟肼与利福平合用,利福平是肝药酶诱导剂,会促进异烟肼转化为具有肝毒性的代谢物乙酰异烟肼,而利福平本身也有肝功能损害作用,两者合用会增强肝毒性作用,但两药联用对结核杆菌有协同抗菌作用,所以这一联合用药对肝功能正常的结核病患者仍是首选用药方案之一,但在治疗过程中应定期检查肝功能。

\subsubsection{重度药物相互作用}

药物联用会造成严重的毒性反应,需要重新选择药物,或须改变用药剂量及给药方案。如特非那定与许多药物(大环内酯类、咪唑类、H{2}
受体阻断药、口服避孕药等)合用时代谢过程受阻,其原形对心脏毒性较大,可致患者室性心动过速而死亡。骨骼肌松弛药与氨基糖苷类抗生素庆大霉素等合用,可能增强及延长骨骼肌松弛作用,甚至引起呼吸肌麻痹。

此外,按药物相互作用发生的概率大小可分为:肯定、很可能、可能、可疑、不可能等几个等级。这主要是根据已发表的临床研究或体外研究、病例报告、临床前研究等文献结果进行判断。按发生的时间过程,有的药物相互作用可立即发生,如四环素类抗生素与含钙、铝、镁的抗酸药发生络合反应,可使四环素的吸收立即下降。另一些药物相互作用的影响可能需要数小时或几天后才表现出来,如华法林的抗凝作用可被合用的维生素K逐渐减弱。

\section{体外药物相互作用}

体外药物相互作用是指在患者用药之前(即药物尚未进入机体以前),药物相互间发生化学或物理性相互作用,使药性发生变化。即一般所称化学配伍禁忌或物理配伍禁忌。

\subsection{分类}

\subsubsection{可见配伍变化}

包括溶液混浊、产气、沉淀、结晶及变色。可见配伍变化,应在混合后仔细观察,大多数是可以避免的。有些可见配伍变化不是立即发生的,而是在使用过程中逐渐出现的,更应该引起足够重视。如20%磺胺嘧啶钠注射液(pH值为9.5~11)加入10%的葡萄糖注射液(pH值为3.2~5.5)中,由于pH值的改变,可使磺胺嘧啶微结晶析出,这种结晶输入血管可造成栓塞。

\subsubsection{不可见配伍变化}

包括水解反应、效价下降、聚合变化及肉眼不能直接观察到的直径50μm以下的微粒等,潜在的影响药物对人体的安全性和有效性。如在氨基酸注射液中不能加入对酸不稳定的药物,因为该类药物在氨基酸营养液中容易降解;维生素C(pH值为5.8~6.9)与偏碱性的氨茶碱(pH值为9.0~9.5)溶液混合时,外观无变化,但效价降低。

\subsection{常见注射剂配伍变化产生的原因}

\subsubsection{沉淀}
\paragraph{注射液溶媒组成改变}

因改变溶媒的性质而析出沉淀。某些注射剂内含非水溶剂,目的是使药物溶解或制剂稳定,若把这类药物加入水溶液中,由于溶媒性质的改变而析出药物产生沉淀。如氯霉素注射液(含乙醇、甘油等)加入5%葡萄糖注射液或0.9%氯化钠注射液中,可析出氯霉素沉淀。
\paragraph{电解质的盐析作用}

主要是对亲水胶体或蛋白质药物自液体中被脱水或因电解质的影响而凝集析出。如氟罗沙星注射剂与0.9%氯化钠注射液合用可发生盐析作用而出现沉淀。
\paragraph{pH值改变}

pH值发生改变时,药物的溶解性也会发生改变,会导致药物的析出。5%硫喷妥钠10mL加入5%葡萄糖注射液500mL中,由于溶液pH值下降导致产生沉淀。
\paragraph{形成配合物}

如米诺环素与\ce{Ca^2+} 、\ce{Mg^2+} 等金属离子形成难溶性配合物而析出沉淀。

\subsubsection{变色}

出现新的颜色,或原有颜色消失。酚类化合物、水杨酸及其衍生物以及含酚羟基的药物如肾上腺素与铁盐发生配合反应,或受空气氧化,都能产生有色物质。

\subsubsection{产气}

碳酸盐、碳酸氢盐与酸类药物配伍,铵盐与碱类药物配伍,均可产生气体。

\subsubsection{效价下降}

某些药物在水溶液中不稳定,易分解失效,与其他药物合用,可加速分解,致药物活性下降。如氨苄西林在含乳酸根的复方氯化钠注射液中,由于乳酸根可加速氨苄西林的水解,4h效价损失20%。

\subsubsection{聚合反应}

氨苄西林1%({w/v}
)的储备液在放置期间,会发生变色、溶液变黏稠、形成沉淀,这是由于形成聚合物所致。

\subsection{注射剂配伍变化的预测}

根据注射药物的理化性质,将预测符号分为7类。

AI类为水不溶性的酸性物质制成的盐,与pH值较低的注射液配伍时易产生沉淀。如青霉素类、头孢菌素类、苯妥英钠等。

BI类为水不溶性的碱性物质制成的盐,与pH值较高的注射液配伍时易产生沉淀。如红霉素乳糖酸盐、盐酸氯丙嗪、盐酸普鲁卡因等。

AS类为水溶性的酸性物质制成的盐,其本身不因pH值变化而析出沉淀。如维生素C、氨茶碱、葡萄糖酸钙、甲氨蝶呤(MTX)等。

BS类为水溶性的碱性物质制成的盐,其本身不因pH值变化而析出沉淀。如硫酸阿托品、硫酸多巴胺、硫酸庆大霉素、盐酸林可霉素等。

N类为水溶性无机盐或水溶性不成盐的有机物,其本身不因pH值变化而析出沉淀,但可导致AS、BI类药物产生沉淀。如氯化钾、葡萄糖、碳酸氢钠、氯化钠等。

C类为有机溶媒或增溶剂制成不溶性注射液(如氢化可的松),与水溶性注射剂配伍时,常由于溶解度改变而析出沉淀。如氯霉素、维生素K{1}
、地西泮等。

P类为水溶性的具有生理活性的蛋白质(如胰岛素),pH值变化、重金属盐、乙醇等均可影响其活性或使其产生沉淀。如抗利尿激素、透明质酸酶、催产素、肝素等。

\section{药动学方面的相互作用}

药物代谢动力学(pharmacokinetics,PK)简称药动学,是研究药物在体内变化规律的一门学科。药动学的研究内容主要包括:一是药物的体内过程,包括吸收、分布、代谢和排泄;二是药物在体内随时间变化的速率过程。前者主要描述药物在体内变化过程的一般特点;后者主要以数学公式定量地描述药物随时间改变的变化过程。

机体对药物的处理是药物与机体相互作用的一个重要组成部分,药动学过程包括药物在其吸收、分布、代谢和排泄过程的任一环节发生相互作用,均可影响药物在血浆或其作用靶位的浓度,最终使其药效或不良反应发生相应改变。

\subsection{影响药物吸收的相互作用}

药物由给药部位进入血液循环的过程称为吸收。除静脉注射和静脉滴注给药外,其他血管外给药途径都存在吸收过程。临床常用的血管外给药途径可分为消化道给药、注射给药、呼吸道给药及皮肤黏膜给药,口服是最常用的给药途径。药物在胃肠道吸收时相互影响的因素有如下几个方面。

\subsubsection{pH值的影响}

药物在胃肠道的吸收主要通过被动转运。药物的脂溶性愈大、非解离型比值越大,越易吸收。胃肠道的pH值可通过影响药物的溶解度和解离度,进而影响药物的吸收。如酸性药物在酸性环境以及碱性药物在碱性环境下解离度低,非解离型药物占大多数,因而药物脂溶性较高,较易透过生物膜被吸收;反之,酸性药物在碱性环境或碱性药物在酸性环境下解离度高,因而药物脂溶性低,扩散透过生物膜的能力差,吸收减少。药物与能改变胃肠道pH值的其他药物合用,其吸收将会受到影响。如水杨酸类药物在酸性环境下吸收较好,若同时服用抗酸药碳酸氢钠,将减少水杨酸类药物的吸收。

\subsubsection{配合作用与吸附作用的影响}

含有2、3价的阳离子(\ce{Ca^2+} 、\ce{Al^3+} 、\ce{Mg^2+}
等)能与四环素类抗生素、异烟肼、喹诺酮类抗菌药物等形成不溶性或难以吸收的配合物,从而影响药物吸收。如口服的四环素与金属离子(\ce{Ca^2+}
、\ce{Al^3+} 、\ce{Mg^2+} 等)配合,使其吸收减少。

阴离子交换树脂如考来烯胺、考来替泊,对酸性分子如阿司匹林、地高辛、华法林、环孢素、甲状腺素等有很强的亲和力,妨碍了这些药物的吸收。药用炭、白陶土等吸附剂也可使一些与其一同服用的药物吸收减少,如林可霉素与白陶土同服,其血药浓度只有单独服用时的10%。

这些药物相互作用可采用增加给药时间间隔的方法来避免。

\subsubsection{胃肠运动的影响}

大多数口服药物主要在小肠上部吸收,因此改变胃排空和肠蠕动速度的药物能影响目标药物到达小肠吸收部位的时间和在小肠滞留的时间,从而影响目标药物吸收程度和起效时间。

一般而言,胃肠蠕动加快,药物起效快,但在小肠滞留时间短,可能吸收不完全;胃肠蠕动减慢,药物起效慢,吸收可能完全。这在溶解度低和难吸收的药物中表现得比较明显。如地高辛片剂在肠道内溶解度较低,与促进胃肠蠕动的甲氧氯普胺等合用,地高辛的血药浓度可降低约30%,有可能导致治疗失败;而与抑制胃肠蠕动的溴丙胺太林合用,地高辛的血药浓度可提高30%左右,如不调整地高辛剂量,就可能中毒;而口服快速溶解的地高辛溶液或胶囊,则溴丙胺太林对其吸收的影响相对较小。但是,对那些在胃的酸性环境中会被灭活的药物如左旋多巴,抑制胃肠蠕动的药物可增加其在胃黏膜脱羧酶的作用下转化为多巴胺(DA),从而降低其口服生物利用度。

\subsubsection{肠吸收功能的影响}

抗肿瘤药物如环磷酰胺、长春碱以及对氨基水杨酸、新霉素等能破坏肠壁黏膜,引起吸收不良。如环磷酰胺可使合用的地高辛吸收减少,血药浓度降低,疗效下降。

\subsubsection{食物的影响}

一般情况下食物可减少药物的吸收。如利福平、异烟肼等可因进食而吸收缓慢,但对药物吸收总量未有影响。但某些脂溶性药物,如灰黄霉素与高脂肪的食物同服,可明显增加吸收量。

\subsubsection{肠道菌群的影响}

消化道的菌群主要位于大肠内,胃和小肠内数量极少。因此,主要在小肠内吸收的药物较少受到肠道菌群的影响。口服地高辛后,在部分患者的肠道中,地高辛能被肠道菌群大量代谢灭活,如同时服用红霉素等能抑制这些肠道菌群的抗生素,可使地高辛血浆浓度增加一倍。

部分药物结合物经胆汁分泌,在肠道细菌的作用下可水解为有活性的原药而重吸收,形成肠肝循环。抗菌药物通过抑制细菌可抑制这些药物的肠肝循环。如抗生素可抑制口服避孕药中炔雌醇的肠肝循环,导致循环血中雌激素水平下降。

\subsubsection{其他因素的影响}

消化液是某些药物重要的吸收条件。硝酸甘油片舌下含服,需要充分的唾液帮助其崩解和吸收,如同服抗胆碱药,则由于唾液分泌减少而使之降效。

某些药物合并用药可影响胃肠道黏膜内外酶和酶系统,从而影响药物的吸收。如秋水仙碱能抑制肠黏膜中多种酶系统(如蔗糖酶、麦芽糖酶、乳酸酶等),导致维生素B{12}
的吸收不良。

另外,口服以外的给药途径也有可能因相互作用而影响吸收。如应用局麻药时,常加入微量肾上腺素以收缩血管,延缓局麻药的吸收,达到延长局麻药作用时间、减少不良反应的效果。

\subsection{影响药物分布的相互作用}

药物吸收后,通过各种生理屏障经血液转运到组织器官的过程称为分布(distribution)。分布过程中的药物相互作用方式,可表现为相互竞争血浆蛋白结合部位,改变游离型药物的比例,或改变药物在某些组织的分布量,从而影响它们在靶部位的浓度。

\subsubsection{竞争血浆蛋白结合部位}

药物经吸收进入血液循环后,大部分药物或其代谢产物均不同程度地与血浆蛋白发生可逆性结合,称结合型药物;另一部分为游离型药物。

当药物合用时,它们可在蛋白结合部位发生竞争,结果是与蛋白亲和力较强的药物可将另一种亲和力较弱的药物从血浆蛋白结合部位上置换出来,使后一种药物的游离型增多。由于游离型的药物分子才能跨膜转运,产生生物活性,并能被分布、代谢与排泄,因此这种蛋白结合的置换可对被置换药物的药动学和药效学产生一定的影响。

通过体外试验很容易证明,许多药物间均存在这种蛋白结合的置换现象。因此,过去一度认为它是临床上许多药物相互作用的一个重要机制。但近年来,更严谨的研究得出结论:大多数置换性相互作用并不产生严重的临床后果,因为置换使游离型药物增多的同时,相应分布、消除的比例也增加,仅引起血药浓度的短暂波动。

保泰松与华法林的相互作用研究是对蛋白结合置换现象的临床意义进行重新认识的典型例子。保泰松可以增强华法林的抗凝作用而致出血不止。过去一直认为,保泰松将华法林从其血浆蛋白结合部位置换出来,游离型华法林浓度升高导致出血。并据此认为任何非甾体抗炎药(NSAID)均以这种方式增强华法林的抗凝作用。现在的研究认识到,华法林是R和S两种异构体的混合物,S异构体的活性较R强5倍;保泰松除了竞争置换出华法林外,还可抑制S-华法林的代谢(由CYP2C9/18催化)而促进R-华法林代谢(由CYP1A2、CYP3A4催化),这样表面上药物总的半衰期不变,但血浆中活性高的S-华法林的比例增大,因而抗凝作用增强。

药物在蛋白结合部位的置换反应能否产生明显的临床后果,取决于目标药的药理学特性,那些蛋白结合率高、分布容积小、半衰期长和安全范围小的药物被置换下来后,往往发生药物作用的显著增强而导致不良的临床后果。表\ref{tab4-1}列出了一些常见的通过血浆蛋白置换而发生药物相互作用的实例。

\begin{longtable}[]{@{}lll@{}}
    \caption{血浆蛋白置换引起的药物相互作用}
    \label{tab4-1}\\
    \toprule
目标药(被置换药物) & 相互作用药 & 临床后果\tabularnewline
\midrule
甲苯磺丁脲 & 水杨酸、保泰松、磺胺药 & 低血糖\tabularnewline
华法林 & 水杨酸、水合氯醛 & 出血倾向\tabularnewline
MTX & 水杨酸、呋塞米、磺胺药 & 粒细胞缺乏症\tabularnewline
硫喷妥钠 & 磺胺药 & 麻醉时间延长\tabularnewline
卡马西平、苯妥英钠 & 维拉帕米 & 两药毒性增强\tabularnewline
\bottomrule
\end{longtable}

\subsubsection{改变组织分布}
\paragraph{改变组织血流量}

某些作用于心血管系统的药物可通过改变组织血流而影响与其合用药物的组织分布。如去甲肾上腺素减少肝脏血流量,使得利多卡因在肝脏的分布量减少,导致代谢减慢、血药浓度增高;而异丙肾上腺素增加肝脏血流量,增加利多卡因在肝脏中的分布及代谢,使其血药浓度降低。
\paragraph{组织结合位点上的竞争置换}

与药物在血浆蛋白上的置换一样,类似的反应也可发生于组织结合位点上,而且置换下来的游离型药物可返回到血液中,使血药浓度升高。由于组织结合位点的容量一般都很大,通常对血药浓度影响不大,但有时也能产生有临床意义的药效变化。例如奎尼丁能将地高辛从骨骼肌的结合位点上置换下来,可使90%患者的地高辛血药浓度升高约1倍,两药合用时,地高辛用量应减少30%~50%。

\subsection{影响药物代谢的相互作用}

药物在体内发生化学结构的改变称为代谢,或称为生物转化。药物代谢的主要场所是肝脏,肝脏进行药物代谢主要依赖于微粒体中的多种酶系。药物经代谢后可转化为无活性物质;或使原来无药理活性的药物转变为有活性的代谢产物;或将活性药物转化为其他活性物质;或产生有毒物质。影响药物代谢的相互作用占药动学相互作用的40%,是一种具有重要临床意义的药动学相互作用。

\subsubsection{酶诱导}

某些药物能增加肝药酶的合成或提高肝药酶的活性,称之为酶诱导。酶诱导使目标药的代谢加快,一般是导致作用减弱或作用时间缩短。具有酶诱导作用的常见药物如表\ref{tab4-2}所示。如口服抗凝血药双香豆素期间加服苯巴比妥,后者使血中双香豆素的浓度下降,抗凝作用减弱,表现为凝血酶原时间缩短。因此,如果这两类药物合用,必须应用较大剂量才能维持其治疗效应。

\begin{longtable}{ccc}
    \caption{常见的酶诱导及相互作用}
    \label{tab4-2}\\
    \toprule
    药物种类 & 受影响药物 & 相互作用结果\tabularnewline
\midrule
巴比妥类 & 巴比妥类、洋地黄毒苷、类固醇激素& \multirow{4}{3cm}{血药浓度下降、药效减弱或不良反应减轻}\tabularnewline             
保泰松、苯妥英钠 & 口服降血糖药、氢化可的松、茶碱 & ~\tabularnewline
利福霉素 & 口服抗凝药、地高辛、普萘洛尔、美托洛尔等 & ~\tabularnewline
灰黄霉素 & 口服抗凝药 & ~\tabularnewline
\bottomrule
\end{longtable}





需要指出的是,酶诱导促使药物代谢增加,但不一定均导致药物疗效下降,因为有些药物的药效是由其活性代谢物引起的。如环磷酰胺在体外无活性,只有经肝药酶代谢活化生成磷酰胺氮芥,才能与DNA烷化发挥药理作用,抑制肿瘤细胞的生长增殖。另外,如果药物经代谢生成毒性代谢产物,与酶诱导剂合用就可能会导致不良反应增加。如异烟肼产生肝毒性代谢物乙酰异烟肼,若与利福平合用,后者的酶诱导作用将加重异烟肼的肝毒性。

\subsubsection{酶抑制}

一些药物能减少肝药酶的合成或者降低肝药酶的活性,称之为酶抑制。临床上因肝药酶的抑制而引起的药物相互作用较肝药酶诱导所引起的药物相互作用常见。肝药酶被抑制,将使另一药物的代谢减少,因而加强或延长其作用。具有酶抑制作用的常见药物如表\ref{tab4-3}所示。如氯霉素与双香豆素合用,明显加强双香豆素的抗凝血作用,这是由于氯霉素抑制肝药酶,使双香豆素的半衰期延长2~4倍。

\begin{longtable}{ccc}
    \caption{常见的酶抑制及相互作用}
    \label{tab4-3}\\
    \toprule
    药物种类 & 受影响药物 & 相互作用结果\tabularnewline
\midrule
西咪替丁、阿司匹林 & 苯二氮䓬类药物& \multirow{4}*{血药浓度上升、药效增强或出现毒性反应}\tabularnewline
氯霉素、异烟肼 & 苯妥英钠、口服降血糖药 & ~\tabularnewline
别嘌醇 & 口服抗凝药、AZA\footnote{AZA表示硫唑嘌呤(azathioprine)} & ~\tabularnewline
肾上腺皮质激素 & 三环类抗抑郁药、环磷酰胺 & ~\tabularnewline
\bottomrule
\end{longtable}

有些药物在体内通过各自的灭活酶而被代谢,若这些酶被抑制,将加强相应药物的作用。食物中的酪胺在吸收过程中被肠壁和肝脏的单胺氧化酶所灭活,因而不呈现作用。但在服用单胺氧化酶抑制剂期间,若食用酪胺含量高的食物如奶酪、红葡萄酒等,由于肠壁及肝脏的单胺氧化酶已被抑制,被吸收的酪胺不经破坏,大量到达去甲肾上腺素能神经末梢,引起末梢中的去甲肾上腺素大量释放出来,使动脉血压急剧升高,产生高血压危象,危及患者生命。

虽然酶抑制可导致相应目标药自机体的清除减慢,体内药物浓度升高,但酶抑制能否引起有临床意义的药物相互作用取决于多种因素。
\paragraph{目标药的毒性及治疗窗的大小}

药物相互作用能产生临床意义的药物通常其治疗窗很窄,即治疗剂量和中毒剂量之间的范围很小;或其剂量-反应曲线陡峭,药物浓度虽然只有轻微改变,但是其效果差异变化显著。如抗过敏药阿司咪唑具有心脏毒性,与酮康唑、红霉素等酶抑制剂合用时,由于代谢受阻血药浓度显著上升,可出现致死性的心脏毒性。而酮康唑抑制舍曲林的代谢则不会引起严重的心血管不良反应。
\paragraph{是否存在其他代谢途径}

如果目标药可由多种肝药酶催化代谢,当其中一种酶受到抑制时,药物可代偿性经由其他途径消除,药物代谢速率所受影响可不大。但对主要由某一种肝药酶代谢的药物,如果代谢酶受到抑制,则容易产生明显的药物浓度和效应的变化。
\paragraph{与能抑制多种肝药酶的药物合用}

有些药物能抑制多种肝药酶,在临床上容易发生与其他药物的相互作用。如H{2}
受体阻断剂西咪替丁,其结构中的咪唑环可与肝药酶中的血红素部分紧密结合,故能抑制多种肝药酶而影响许多药物在体内的代谢。目前已报道有70多种药物的肝清除率在与西咪替丁合用后,出现不同程度的下降。临床上当药物与西咪替丁合用时,应注意调整剂量,必要时可用雷尼替丁代替西咪替丁。

酶抑制引起的药物相互作用常常导致药物作用的增强及不良反应的发生,但也有例外。如奎尼丁是酶抑制剂,而可待因须经肝药酶代谢生成吗啡产生镇痛作用,两者合用可使可待因的镇痛作用明显减弱,药效降低。

\subsection{影响药物排泄的相互作用}

药物及其代谢产物经机体的排泄器官或分泌器官排出体外的过程称为排泄。大多数影响药物排泄的相互作用发生在肾脏。当一种药物改变肾小管液的pH值、干扰肾小管的主动转运过程或重吸收过程或影响到肾脏的血流量时,就能影响一些其他药物的排泄,尤其对以原形排出的药物影响较大。

\subsubsection{改变尿液pH值}

尿液的pH值通过影响解离型/非解离型药物的比例,改变进入肾小管内药物的重吸收。这主要是因为大多数药物为有机弱电解质,在酸性尿液中,弱酸性药物(pKa为3.0~7.5)大部分以非解离型存在,脂溶性高,易通过肾小管上皮细胞重吸收;而弱碱性药物(pKa为7.5~10)的情况相反,大部分以解离型存在,随尿液排出多。临床上可通过碱化尿液增加弱酸性药物的肾清除率,如苯巴比妥多以原形自肾脏排泄,当过量中毒时,可用碳酸氢钠碱化尿液,减少重吸收,促进苯巴比妥的排泄而解毒。同理,酸化尿液可促进碱性药物的排泄。

但在药物的相互作用中,尿液pH值改变的临床意义甚小,因为除小部分药物直接以原形排出,大多数药物经代谢失活后,最终从肾脏消除;同时能大幅度改变尿液pH值的药物在临床上也很少使用。

\subsubsection{干扰肾小管分泌}

肾小管分泌是一种主动转运过程,要通过肾小管的特殊转运载体,包括酸性药物载体和碱性药物载体。当两种酸性药物合用时(或两种碱性药物合用),可相互竞争酸性(或碱性)载体,竞争力弱的药物,经由肾小管分泌的量减少,肾脏排泄减慢,有可能增强其疗效或毒性。如痛风患者合用丙磺舒和吲哚美辛,两者竞争酸性载体,可使吲哚美辛的分泌减少,排泄减慢,不良反应发生率明显增加。

但是有些药物间的这种竞争可被用于产生有益的治疗目的。如丙磺舒和青霉素竞争肾小管上的酸性转运系统,可延缓青霉素的经肾排泄过程,使其发挥持久的治疗作用。

\subsubsection{改变肾脏血流量}

减少肾脏血流量的药物可妨碍药物的经肾排泄,但这种情况在临床上并不多见。肾脏的血流量部分受到肾组织中扩血管的前列腺素生成量的调控。有报道指出,如果这些前列腺素的合成被吲哚美辛等药物抑制,则锂的肾排泄量会降低,并伴有血清锂水平的升高。这提示合用锂盐和NSAIDs的患者,应密切监测血清锂水平。

\section{药效学方面的相互作用}

药效学方面的药物相互作用是指不同药物通过与疾病相关药物靶点的影响,使一种药物增强或减弱另一种药物的效应或不良反应的现象。相互作用结果可分为药物效应的相加、协同和拮抗。

\subsection{相加或协同作用}

相加作用(addition effect)或协同作用(synergistic
effect)是指作用于疾病相关靶点的两种药物合用的效果等于(相加)或大于(协同)单用效果之和。相加或协同作用是临床用药的主要目的。

\subsubsection{表现为药理作用的增强}

如磺胺甲噁唑(SMZ)和甲氧苄啶(TMP)通过双重阻断机制(SMZ抑制二氢叶酸合成酶,TMP抑制二氢叶酸还原酶),协同阻断敏感菌的四氢叶酸合成,抗菌活性是两药单独等量应用时的数倍至数十倍,甚至呈现杀菌作用,且抗菌谱扩大,并减少细菌耐药性的产生。常将SMZ与TMP按5∶1的比例制成复方磺胺甲噁唑(SMZco)用于临床。另外,临床上常用青霉素和庆大霉素联用抗感染、异烟肼和利福平联用抗结核,这些联用都表现为治疗效应的增强。

\subsubsection{表现为药理作用的相加}

如应用一般治疗剂量的巴比妥类药物或其他具有中枢神经系统抑制作用的药物时,饮用少量酒即可引起昏睡,因为乙醇具有非特异性中枢神经系统的抑制作用,致使药理作用的相加。

\subsubsection{表现为增加药物不良反应的风险}

如治疗帕金森病的抗胆碱药物,与具有抗胆碱作用的其他药物(如氯丙嗪、H{1}
受体阻断药、三环类抗抑郁药)合用时可产生性质协同的相互作用,常可出现过度的抗胆碱能效应,在老年患者甚至可能出现抗胆碱危象。口服广谱抗生素抑制肠道菌群后,可使维生素K合成减少,从而增加香豆素类抗凝药的活性,应适当减少抗凝药的剂量。临床常见的药物相加或协同作用如表\ref{tab4-4}所示。

\begin{longtable}{cc}
    \caption{临床常见的药物相加或协同作用}
    \label{tab4-4}\\
\toprule
\endhead
相互作用药物 & 药理效应\tabularnewline
\midrule
NSAIDs和华法林 & 增加出血的风险\tabularnewline
血管紧张素转换酶抑制剂和氨苯蝶啶 & 增加高血钾的风险\tabularnewline
维拉帕米和β受体拮抗剂 & 心动过缓和停搏\tabularnewline
氨基糖苷类和呋塞米 & 增加耳、肾毒性\tabularnewline
骨骼肌松弛药和氨基糖苷类 & 增加骨骼肌松弛作用\tabularnewline
乙醇与苯二氮䓬类 &
增强镇静作用\tabularnewline
MTX与复方磺胺甲噁唑 & 骨髓巨幼红细胞症\tabularnewline
\bottomrule
\end{longtable}

\subsection{拮抗作用}

拮抗作用是指两种或两种以上药物合用所产生的效应小于其中一种药物单用的效应。在临床上,通常要尽量避免药物治疗作用的相互拮抗。根据作用机制,可将药物的拮抗作用分为两类。

\subsubsection{竞争性拮抗}

两种药物在共同的作用部位或受体上产生了拮抗作用。本类相互拮抗作用可发挥治疗作用,如在治疗虹膜炎时,交替使用毛果芸香碱和阿托品,可防止虹膜粘连;也可产生药理性拮抗作用,在药物中毒时抢救患者的生命。
如用苯二氮䓬
类受体拮抗剂氟马西尼抢救苯二氮䓬
类过量中毒;用α-肾上腺素受体激动剂去甲肾上腺素对抗氯丙嗪过量引起的低血压。

\subsubsection{非竞争性拮抗}

作用物与拮抗物不是作用于同一受体或同一部位,也可出现拮抗作用。如较大剂量的氯丙嗪用于治疗精神分裂症时,因阻断黑质-纹状体通路的多巴胺受体,使中枢乙酰胆碱作用相对增强,可引起锥体外系反应,而苯海索具有中枢抗胆碱作用,可减轻锥体外系反应;氨茶碱可因兴奋中枢而引起失眠,常合用催眠药加以对抗;维生素B{6}
能增加外周多巴脱羧酶活性,加速左旋多巴在外周部位脱羧,减少左旋多巴进入中枢的量,降低左旋多巴的疗效,产生对抗左旋多巴的作用。


\chapter{消化系统}


\section{正常X线解剖}

一、正常X线表现

胃肠道疾病的检查主要应用透视、腹部X线平片以及钡剂造影,显示胃肠道的位置、轮廓、腔的大小、内腔及粘膜皱襞的情况,但对于胃肠道肿瘤的内部结构、胃肠壁的浸润程度和转移等尚有一定困难,还需与其他检查相结合。目前,对于胃肠道疾病的检查,首选当是钡剂造影的检查方法。

1.咽部 是胃肠道的开始部分,它是含气空腔。吞钡造影正位观察,上方正中为会厌,两旁充钡小囊状结构为会厌谷。会厌谷外下方较大的充钡空腔是梨状窝,近似菱形且两侧对称,梨状窝中间的透亮区为喉头,勿误为病变。正常情况下,一次吞咽动作即可将钡剂送入食管,吞钡时梨状窝暂时充满钡剂,但片刻即排入食管。

2.食管 是一个连接下咽部与胃的肌肉管道,起于第6颈椎水平与下咽部相连。食管入口与咽部连接处及膈的食管裂孔处各有一生理狭窄区,为上、下食管括约肌。

食管充盈像:食管吞钡充盈,轮廓光滑整齐,宽度可达2~3cm。正位观察位于中线偏左,胸上段更偏左,管壁柔软,伸缩自如。右前斜位是观察食管的常规位置,在其前缘可见三个压迹,从上至下为主动脉弓压迹、左主支气管压迹、左心房压迹。于主动脉弓压迹与左主支气管压迹之间,食管显示略膨出,注意不要误认为憩室。

食管粘膜像:少量充钡,粘膜皱襞表现为数条纵行、相互平行的纤细条纹状阴影。这些粘膜皱襞通过裂孔时聚拢,经贲门与胃小弯的粘膜皱襞相连续。

透视下观察,正常食管有两种蠕动。第一蠕动为原发性蠕动,系由下咽动作激发,使钡剂迅速下行,数秒钟达胃内。第二蠕动又称继发蠕动波,由食物团对食管壁的压力所引起,始于主动脉弓水平,向下推进。所谓第三蠕动波是食管环状肌的局限性不规则收缩运动,形成波浪状或锯齿状边缘,出现突然,消失迅速,多发于食管下段,常见于老年人和食管贲门失迟缓症者。

另外,当深吸气时膈肌下降,食管裂孔收缩,致使钡剂暂时停顿于膈上方,形成食管下端膈上一小段长4~5cm的一过性扩张,称之膈壶腹,呼气时消失,属于正常现象。

此外,贲门上方3~4cm长的一段食管,是从食管过渡到胃的区域,称之食管前庭段,具有特殊的神经支配和功能。此段是一高压区,有防止胃内容物反流的重要作用。现将原来所定的下食管括约肌与胃食管前庭段统称为下食管括约肌。它的左侧壁与胃底形成一个锐角切迹,称为食管胃角或贲门切迹。

3.胃 一般分为胃底、胃体、胃窦三部分及胃小弯和胃大弯。胃底为贲门水平线以上部分,立位时含气,称胃泡。贲门至胃角(胃体与胃窦小弯拐角处,也称胃角切迹)的一段称胃体。胃角至幽门管斜向右上方走行的一部分,称胃窦。幽门为长约5mm的短管,宽度随括约肌收缩而异,将胃与十二指肠相连。胃轮廓的右缘为胃小弯,左缘是胃大弯。胃的形状与体形、张力及神经系统的功能状态有关,一般可分为4种类型:牛角型(位置、张力均高,呈横位,上宽下窄,胃角不明显,形如牛角。多见于肥胖体形的人);钩型(位置、张力中等,胃角明显,胃的下极大致位于髂嵴水平,形如鱼钩)。瀑布型(胃底大呈囊袋状向后倾,胃泡大,胃体小,张力高。充钡时,钡剂先进入后倾的胃底,充满后再溢入胃体,犹如瀑布)。长钩型(又称为无力型胃,位置、张力均低,胃腔上窄下宽如水袋状,胃下极位于髂嵴水平以下。见于瘦长体形的人)。

胃的轮廓在胃小弯侧及胃窦大弯侧光滑整齐,胃体大弯侧呈锯齿状,系横、斜走行的粘膜皱襞所致。

胃的粘膜皱襞像,可见皱襞间沟内充以钡剂,呈致密的条纹状影。皱襞则显示为条状透亮影。胃小弯侧的皱襞平行整齐,一般可见3~5条。角切迹以后,一部分沿胃小弯走向胃窦,一部分呈扇形分布,斜向大弯。胃体大弯侧的粘膜皱襞为楔形、横行而呈不规则的锯齿状。胃底部粘膜皱襞排列不规则,相互交错呈网状。胃窦部的粘膜皱襞可为纵行、斜行及横行,收缩时为纵行,舒张时以横行为主,排列不规则。

胃的双对比造影显示粘膜皱襞的细微结构即胃小区、胃小沟。正常胃小区为1~3mm大小,呈圆形、椭圆形或多角形大小相似的小隆起,其由于钡剂残留在周围浅细的胃小沟而显示出,呈细网眼状。正常的胃小沟粗细一致,轮廓整齐,密度淡而均匀,宽约1mm以下。

胃的蠕动来源于肌层的波浪状收缩,由胃体上部开始,有节律地向幽门方向推进,波形逐渐加深,一般同时可见2~3个蠕动波。胃窦没有蠕动波,是整体向心性收缩,使胃窦呈一细管状,将钡剂排入十二指肠;之后,胃窦又整体舒张,恢复原来状态。但不是每次胃窦收缩都有钡剂排入十二指肠。胃的排空受胃的张力、蠕动、幽门功能和精神状态等影响,一般于服钡后2~4小时排空。

4.十二指肠 十二指肠全程呈C形,在描述时,可将十二指肠全程称为十二指肠曲。上与幽门连接,下与空肠连接,一般分为球部、降部、水平部和升部。球部呈锥形,两缘对称,尖部指向右后方,底部平整,球底两侧称为隐窝或穹隆,幽门开口于底部中央。球部轮廓光滑整齐,粘膜皱襞为纵行、彼此平行的条纹。降部以下粘膜皱襞的形态与空肠相似,呈羽毛状。球部的运动为整体性收缩,可一次将钡剂排入降部。降、升部的蠕动多呈波浪状向前推进。十二指肠正常时可有逆蠕动。

低张力造影时,十二指肠管径可增加一倍,粘膜皱襞呈横行排列的环状或呈龟背状花纹。降部的外侧缘形成光滑的曲线。内缘中部可见一肩状突起,称为岬部,为乳头所在处,其下的一段较平直。平直段内可见纵行的粘膜皱襞。十二指肠乳头易于显示,位于降部中段的内缘附近,呈圆形或椭圆形透明区,一般直径不超过1.5cm。

5.空肠和回肠 空肠和回肠之间没有明确的分界,但上段空肠与下段回肠的表现大不相同。空肠大部分位于左上中腹,多见于环状皱襞,蠕动活跃,常显示为羽毛状影像,如肠内钡剂少则表现为雪花状影像,回肠肠腔略小,皱襞少而浅,蠕动不活跃,常显示为充盈像,轮廓光滑。肠管内钡剂较少、收缩或加压时可显示粘膜皱襞影像,呈纵行或斜行。末端回肠自盆腔向右上行与盲肠相连。回盲瓣的上下缘呈唇状突起,在充钡的盲肠中形成透明影。小肠的蠕动是推进性运动,空肠蠕动迅速有力,回肠慢而弱。有时可见小肠的分节运动。服钡后2~6小时钡的先端可达盲肠,7~9小时小肠排空。

6.大肠 大肠分盲肠、升结肠、横结肠、降结肠、乙状结肠和直肠,绕行于腹腔四周。升、横结肠转弯处为肝曲,横、降结肠转弯处为脾曲。横结肠和乙状结肠的位置及长度变化较大,其余各段较固定。直肠居于骶骨前缘并与之紧密相连。大肠中直肠壶腹最宽,其次为盲肠,盲肠以下各肠管逐渐变小。但其长度和宽度随肠管充盈状态及张力有所不同。

大肠充钡后,X线主要特征为结肠袋,表现为对称的袋状突出。它们之间由半月襞形成不完全的间隔。结肠袋的数目、大小、深浅因人因时而异,横结肠以上较明显,降结肠以下逐渐变浅,至乙状结肠接近消失,直肠则没有结肠袋。

大肠粘膜皱襞为纵、横、斜三种方向交错结合状表现。盲肠、升结肠、横结肠皱襞密集,以斜行和横行为主,降结肠以下皱襞渐稀且以纵行为主。

大肠的蠕动主要是总体蠕动,右半结肠出现强烈的收缩,呈细条状,将钡剂迅速推向远侧。结肠的充盈和排空时间差异较大,一般服钡后6小时可达肝曲,12小时可达脾曲,24~48小时排空。

阑尾在服钡或钡灌肠时均可能显影,呈长条状,位于盲肠内下方。一般粗细均匀,边缘光滑,易推动。阑尾不显影、充盈不均匀或其中有粪石造成的充盈缺损,不一定是病理性的改变,阑尾排空时间与盲肠相同,但有时可延迟达72小时。

双对比造影时膨胀而充气肠腔的边缘为约1mm宽的光滑而连续线条状影,勾画出结肠的轮廓,结肠袋变浅,粘膜面可显示出与肠管横径平行的无数微细浅沟,称之为无名沟或无名线。它们既可平行又可交叉形成微细的网状结构,从而构成细长的纺锤形小区,与胃小区相似。小区大小为1mm×(3~4)mm。小沟与小区为结肠双对比造影能显示粘膜面的最小单位,为结肠病变早期诊断的基础。

另外,在结肠X线检查时,某些固定部位较经常见到有收缩狭窄区,称为生理性收缩环。狭窄段自数毫米至数厘米长,形态多有改变,粘膜皱襞无异常,一般易与器质性病变相鉴别。但在个别情况下,当形态较固定时,注意与器质性病变鉴别。

二、检查方法及其目的

1.透视和腹部X线平片 主要用于急腹症,如胃肠道穿孔、肠梗阻等。急腹症的X线检查应简单、迅速、准确,以尽量减轻患者痛苦。

(1)腹部仰卧前后位:照片应包括横膈至耻骨联合,为观察腹部解剖构造及病理变化最好的位置。

(2)腹部立位前后位:照片应包括横膈至耻骨联合,可观察:①是否存在液平面。②是否存在气腹。③腹腔内阴影是否随体位变化。④能更细致地观察肠管。⑤了解肠间隔是否增厚。

(3)侧卧位水平投照:方法:①患者采取左侧卧位,X线水平方向投照。照片应包括全腹部,要特别注意右胁腹部、右下胸部应摄于片中。②患者采取右侧卧位,X线水平方向投照。照片也应包括全腹部,但以左胁腹部及左下胸部为重点。此二位置可进一步验证其他位置之所见,对不能站立的患者也可采用此位置投照,以观察是否存在气腹及液平面等。

(4)腹部侧位:患者仰卧,床面为半立位(角度35°~40°),以剑突为中心,X线水平方向投照。此位置检查的主要目的是观察剑突下是否有游离气体存在,及肠腔内是否存在液平面。

(5)后前立位胸片:要求曝光时间短(1/20~1/50秒)。照片目的:①了解是否存在引起急腹症的胸部病变(如下叶肺炎、食管下端穿孔及膈疝等)。②某些腹部疾病可并发异常的胸部X线表现。例如老年人,由于肠系膜血管病变引起的急腹症,其胸部X线检查可发现心脏疾患的证据(如心脏扩大、不正常的房室外形、心力衰竭等),有助于诊断和治疗。③还可查出与急腹症无关的其他疾病,而对手术及术后处理有重要意义。④膈下是否存在游离气体。总之,常规胸部X线检查是诊断急腹症不可缺少的重要步骤。

2.造影检查 消化道造影仍为胃肠道疾病的主要检查方法,造影检查有粘膜法、充盈法、加压法和气钡双重造影法等四种基本方法。粘膜法是用少量钡剂涂布于粘膜表面显示粘膜皱襞的方法,对于病变的早期诊断有重要价值,所摄片称粘膜像。充盈法,胃肠道某一器官或某器官的一部分有较多钡剂充盈,主要显示该部的轮廓,摄片称充盈像,病变的切线位时可见其轮廓异常,较大肿块可显示充盈缺损,较小的肿块可因钡剂掩盖而漏诊。加压法,加压使该部的钡剂减少变薄,有利于较小的隆起性病变的显示,摄片多为某器官的局部点片,称加压像。双重造影法是先后引入一定量的阳性造影剂硫酸钡悬混液和阴性造影剂气体,以显示胃肠道的细微结构,其照片称双重造影像。气体为最常用的阴性造影剂,故又称气钡双重造影,已广泛地用于胃肠道各部位。双重造影分为低张和非低张双重造影,以低张双重造影显示最佳。双重造影技术与纤维内镜的配合已使胃肠道疾病的早期诊断有了突破性进展。

消化道造影检查根据检查部位的不同分成食管造影、上消化道造影、小肠系造影和钡剂灌肠造影。需要指出的是当怀疑消化道穿孔和肠梗阻时,禁用钡餐造影而改用口服有机碘溶液。

\textbf{【X线表现】}
 上方充钡的小囊为会厌谷,下方圆形透亮区为喉头,勿误为占位引起的充盈缺损。喉头两侧为对称的梨状窝。两侧梨状窝汇入中央即为食管开口,即食管第一生理狭窄处(图5-1-1A)。右前斜位食管充盈像,显示食管吞钡充盈,轮廓光滑整齐,其前缘可见三个压迹,从上至下为主动脉弓压迹(为半月弧形,压迹深度随年龄递增)、左主支气管压迹(其与主动脉弓之间食管往往相对膨出为正常表现,不要误认为食管憩室)、左心房压迹(较长而浅,左心房增大,压迹可增宽,甚至食管局部后移)(图5-1-1B)。右前斜位食管粘膜像,管腔内显示2~5条纵行、相互平行的纤细条纹状阴影,即食管粘膜皱襞,其宽度不超过2mm(图5-1-1C)。左前斜位片如图5-1-1D。

\begin{figure}[!htbp]
 \centering
 \includegraphics{./images/Image00233.jpg}
 \captionsetup{justification=centering}
 \caption{食管钡餐造影片}
 \label{fig5-1-1}
  \end{figure} 

\textbf{【X线诊断】}  正常食管钡餐片。

\begin{figure}[!htbp]
 \centering
 \includegraphics{./images/Image00234.jpg}
 \captionsetup{justification=centering}
 \caption{食管第3蠕动波}
 \label{fig5-1-2}
  \end{figure} 

\textbf{【X线表现】}
 所谓第3蠕动波是食管环状肌的局限性不规则收缩运动,形成波浪状或锯齿状边缘,出现突然,消失迅速,多发于食管下段,常见老年人和食管贲门失迟缓症者。

\textbf{【X线诊断】}  食管贲门失迟缓症;食管第3蠕动波。

\begin{figure}[!htbp]
 \centering
 \includegraphics{./images/Image00235.jpg}
 \captionsetup{justification=centering}
 \caption{胃的X线解剖部位划分及命名}
 \label{fig5-1-3}
  \end{figure} 

(1)贲门:食管进入胃的开口处。

(2)胃底:贲门横线以上区域。

(3)贲门区:以贲门为中心,半径约为2.5cm的圆形区域。

(4)胃小弯:胃的右上侧边缘。

(5)胃大弯:胃的左外下侧边缘。

(6)胃角(角切迹):胃小弯转折处。

(7)胃窦:角切迹与胃大弯最低点连线与幽门之间的区域。

(8)胃体:胃窦与胃底之间的区域。

(9)幽门管:胃部通向十二指肠球部的细短管状结构。

\begin{figure}[!htbp]
 \centering
 \includegraphics{./images/Image00236.jpg}
 \captionsetup{justification=centering}
 \caption{胃钡餐造影片}
 \label{fig5-1-4}
  \end{figure} 

\textbf{【X线表现】}
 胃的轮廓在胃小弯侧及胃窦大弯侧光滑整齐,胃体大弯侧呈锯齿状,系横、斜走行的粘膜皱襞所致。

胃的粘膜皱襞像,可见皱襞间沟内充以钡剂,呈致密的条纹状影。皱襞则显示为条状透亮影。胃小弯侧的皱襞平行整齐,一般可见3~5条,平均宽约0.5cm。角切迹以后,一部分沿胃小弯走向胃窦,一部分呈扇形分布,斜向大弯。胃体大弯侧的粘膜皱襞为楔形、横行而呈不规则的锯齿状,宽0.2~0.4cm,大于0.5cm为异常表现。胃底部粘膜皱襞排列不规则,相互交错呈网状。胃窦部的粘膜皱襞可为纵行、斜行及横行,收缩时为纵行,舒张时以横行为主,排列不规则。

\textbf{【X线诊断】}  正常胃的粘膜皱襞。

\begin{figure}[!htbp]
 \centering
 \includegraphics{./images/Image00237.jpg}
 \captionsetup{justification=centering}
 \caption{胃双对比造影片}
 \label{fig5-1-5}
  \end{figure} 

\textbf{【X线表现】}
 胃的双对比造影显示粘膜皱襞的细微结构即胃小区、胃小沟。正常胃小区为1~3mm大小,呈圆形、椭圆形或多角形大小相似的小隆起,其由于钡剂残留在周围浅细的胃小沟而显示出,呈细网眼状。正常的胃小沟粗细一致,轮廓整齐,密度淡而均匀,宽约1mm以下。

\textbf{【X线诊断】}  正常胃小区。

\textbf{【临床经验】}
 应当强调,X线征象的显示情况与检查方法有密切的关系。近年来,由于开展了气钡双重造影,对于龛影形态及胃粘膜皱襞的显示提供了良好的条件。临床工作中,只有把充盈像、粘膜皱襞像及粘膜像结合起来,才能比较确实地反映出龛影的病理形态。在良、恶性溃疡鉴别诊断时,良性胃溃疡多数表现为龛周胃小沟纤细,胃小区多数显示不清,少数显示形态不规则。另有见龛周胃小沟粗细不均,胃小区显示清晰,但形态不规则,呈多样性改变。恶性胃溃疡龛周胃小沟、胃小区破坏,癌组织代替了正常粘膜层,呈多样性改变。如结节样、磨砂玻璃样以及条索状,部分病例在靠近正常粘膜区,胃小区尚可辨认,但胃小沟粗细不均、紊乱、破坏。所以我们认为,龛周胃小区改变呈萎缩型或增生型者为良性溃疡;龛周胃小区呈破坏型代之以结节状、磨砂玻璃状、不规则条状皱襞改变者为恶性溃疡。

\begin{figure}[!htbp]
 \centering
 \includegraphics{./images/Image00238.jpg}
 \captionsetup{justification=centering}
 \caption{上消化道钡餐造影片}
 \label{fig5-1-6}
  \end{figure} 

\textbf{【X线表现】}
 十二指肠全程称十二指肠曲,因其成半环形又称为十二指肠环。一般分为球部、降部、水平部和升部。球部:充盈时呈边缘整齐的三角形,尖部指向右上后方,底部平整,两侧有对称的隐窝,幽门开口于球底中央。球尖顶到降部之间的一小段,X线上称为球后部,其长短不一,一般可达4~5cm,短时几乎不存在。粘膜皱襞可呈纵行,有4~5条,也可呈横行或花纹状,在双重造影时,球部粘膜可呈细网状或小点状,为粘膜绒毛及绒毛间沟充钡所致。球部充盈不全时,其边缘可不规则,为粘膜皱襞所致,易误为异常。因球部及球后部向右后方,所以,右前斜位便于观其全貌,左前斜位便于球部前后壁的显示。降部、水平部、升部:充盈后内外缘对称,因粘膜皱襞的影响,两侧缘呈锯齿状,尤以外缘明显,粘膜皱襞呈环形或羽毛状,收缩时则成纵行。蠕动呈波浪状前进,并可见逆蠕动,不能误为异常。降部宽2~3cm。十二指肠双重造影时,管径可增加一倍,羽毛状粘膜皱襞消失,代之以环形或龟背状花纹,或二者兼有。降部内缘可较平直或略凸,中段可见一肩样突起,称为岬部,其下方较平直,可见纵行皱襞。十二指肠乳头在岬部下方,呈圆形或类圆形,边界清晰,直径一般不超过1.5cm。乳头开口处可存钡,表现为点状,为正常现象。在乳头影上方有时可见一直径数毫米的圆形透亮区,为副乳头。

\textbf{【X线诊断】}  十二指肠正常X线表现。

\begin{figure}[!htbp]
 \centering
 \includegraphics{./images/Image00239.jpg}
 \captionsetup{justification=centering}
 \caption{小肠钡餐造影片}
 \label{fig5-1-7}
  \end{figure} 

\textbf{【X线表现】}
 平片检查,正常成人的小肠内虽有气体,但与食糜混合存在,而不能显示。长期卧床、幼儿及肠紧张的老年人,小肠内有分散的气团,多见于腹中部,为正常表现。另外,患者由卧位改成立位检查时,十二指肠球部可有积气,不能误为异常。造影检查,小肠长度平均为280cm,其长度与体重关系明显,与身长关系不明。空回肠两端较固定,其余部分活动度较大。空肠居于左上腹及中腹部,回肠位于右下腹及盆腔。一般上部肠曲多横行,下部肠曲多纵行。空肠管径较大,为2.5~3cm,回肠管径1.5~2.5cm。空肠粘膜呈细羽毛状,其长短、粗细、形态和方向随肠壁肌张力而变化。收缩时呈纵行状,舒张时呈环形,粘膜面仅有少量钡餐附着时,则呈雪花状。回肠粘膜皱襞则稀疏、低平而不明显,其末端常呈纵行皱襞。在小儿,由于淋巴组织丰富,淋巴集结可呈卵石状,多见于回肠。小肠运动主要为蠕动,表现为节段性充盈与排空。空肠蠕动迅速有力,回肠慢而弱,但分节运动较明显,表现为节律性收缩与舒张。小肠的运动受胃内钡剂排出状况影响,胃蠕动强、排出量大时,小肠的运动也增强。常规口服钡餐造影时,钡剂到达回盲瓣的时间一般为2~6小时,7~9小时钡剂从小肠全部排空。老年人排空时间延缓,可达11小时。如果少于1小时钡剂到达盲肠,为运动增快,超过6小时则为运动过缓。为了便于X线检查的描述,按小肠位置将其分为六组:①十二指肠。②上部空肠,位于左上腹部。③下部空肠,位于左腹部。④上部回肠,位于右中腹部。⑤中间回肠,位于右中下腹部。⑥下部回肠,位于盆腔内。

\textbf{【X线诊断】}  小肠正常X线表现。

\begin{figure}[!htbp]
 \centering
 \includegraphics{./images/Image00240.jpg}
 \captionsetup{justification=centering}
 \caption{结肠钡剂造影片}
 \label{fig5-1-8}
  \end{figure} 

\begin{figure}[!htbp]
 \centering
 \includegraphics{./images/Image00241.jpg}
 \captionsetup{justification=centering}
 \caption{结肠气钡双重造影}
 \label{fig5-1-9}
  \end{figure} 

\textbf{【X线表现】}
 盲肠位于右髂窝内,移动度较大,故位置不固定,可高至肝下或低至盆腔,甚至到左下腹部,但一般移动范围在10cm左右。回盲瓣开口于盲肠后内侧壁,上唇较长约2cm。下唇约0.6cm,瓣口为圆形、椭圆形或呈横裂口。阑尾一般位于盲肠下内侧,钡剂造影显示率为60%,充盈时光滑整齐,活动度大,有时可见粪石形成的充盈缺损,阑尾多与盲肠同时排空或稍延缓。横结肠和乙状结肠的系膜较长,因此,活动范围较大,其余部分位置较固定。直肠壶腹部内径最大,盲肠次之、盲肠向远端逐渐变窄,乙状结肠与直肠移行处最窄,为2~3cm,勿误为病理表现。常规钡剂灌肠时,因生理括约肌的作用,在回盲瓣的对侧、升结肠、横结肠近端和远端、降结肠下部、乙状结肠等部位,可见肠腔局限性狭窄,不能误为异常。

结肠的粘膜皱襞有横、纵、斜三个方向相互交错。盲肠、升结肠及横结肠的粘膜皱襞较显著,降结肠及其远段则稀疏。环肌收缩时粘膜呈纵行皱襞。

直肠没有结肠袋,但直肠壶腹的前壁及侧壁可见半圆襞形成的切迹。直肠后壁与骶骨之间称骶骨前间隙或称直肠后间隙,测量方法是第3~5骶骨前缘到直肠后壁的最短距离,而以第5骶骨处测量较准确。约95%的正常人此间隙小于或等于0.5cm,大于1.5cm时可疑异常,大于2cm者为病理性增大。

双重造影时,结肠的轮廓呈连续、均匀的线条,粗约1mm。其微小皱襞称无名线,此乃结肠的基本解剖单位,切线位表现为微细的刺状突出,深约0.2mm。正面观为0.1~0.2mm,并以0.6~1mm的间距与肠壁垂直分布,或交织呈网状。良好的双重造影片上,无名线的显示率可达90%。在结肠排空像的边缘有时可见深0.5~2mm、粗1mm、以3~5mm间距分布的尖刺影,称边缘锯齿征,或称结肠假溃疡征,是钡剂嵌于结肠Lieberkuhns腺管腺窝所致,出现率为5%~10%。复查时可消失,为正常表现。

\textbf{【X线诊断】}  正常结肠造影片。

\begin{figure}[!htbp]
 \centering
 \includegraphics{./images/Image00242.jpg}
 \captionsetup{justification=centering}
 \caption{经内镜逆行胰胆管造影片}
 \label{fig5-1-10}
  \end{figure} 

\textbf{【X线表现】}
 胆囊大小、形态、位置因人的体质及体位不同而不同,一般分为梨形、圆形和长形三种,最常见为梨形,长7~10cm,宽3~4cm,形态上胆囊可分为底部、体部、漏斗部和颈部。胆管分肝内胆管和肝外胆管两部分,肝内胆管由左、右肝管及其分支组成,肝外胆管由肝总管、胆囊管和胆总管组成。肝总管长3~4cm,宽5~6mm;胆囊管长3~4cm,宽2~3mm;胆总管长7~8cm,宽5~6mm。胆总管穿过十二指肠壁,终止于十二指肠大乳头,构成肝胰壶腹(Oddi)括约肌,宽12mm,长约数毫米,在其上方略为膨大成为肝胰壶腹,胰管汇合于此。

\textbf{【X线诊断】}  胆道系统正常X线表现。

\begin{figure}[!htbp]
 \centering
 \includegraphics{./images/Image00243.jpg}
 \captionsetup{justification=centering}
 \caption{T管造影片}
 \label{fig5-1-11}
  \end{figure} 

\textbf{【X线表现】}
 胰腺管分为主胰管和副胰管,主胰管从十二指肠大乳头开始,多为从右下斜行向左上,或呈横行、乙字形走行于第12胸椎至第2腰椎水平之间。主胰管分为头部、体部和尾部,全长14~18cm;宽:头部4mm,体部3mm,尾部2mm。副胰管于主胰管的头、体交界处与主胰管汇合,大致呈水平走向于十二指肠壁开口于十二指肠小乳头。

\textbf{【X线诊断】}  肝内外胆管及胰腺管X线解剖。

\section{食管病变}

\subsection{食管金属异物}

\begin{figure}[!htbp]
 \centering
 \includegraphics{./images/Image00244.jpg}
 \captionsetup{justification=centering}
 \caption{颈胸部正侧位片}
 \label{fig5-2-1}
  \end{figure} 

\textbf{【病史摘要】}
 男性,3岁。玩耍时不慎将1元硬币吞下,烦躁、哭闹1小时。自述吞咽不适。

\textbf{【X线表现】}  第7颈椎水平见一直径约2.0cm大小圆形不透光异物影。

\textbf{【X线诊断】}  食管入口处金属异物。

\textbf{【评  述】}
 依据患儿有明确的误吞金属异物病史,故常规的透视和颈胸部食管正侧位摄片即可观察到异物的形态,确定异物的位置,不需钡餐造影,诊断一般不会发生困难。需注意的是颈胸部的钙化影和气管内异物有时与食管内不透光异物相似,食管异物在侧位片上,位于气管之后,长形异物与食管纵轴一致;扁平形异物正位呈片状,侧位呈条状,而气管异物恰与此相反。异物最易滞留于食管生理狭窄和压迹处,故应重点观察,尤以食管入口(管径最小)为主,对于滞留于非好发部位的异物,应警惕食管器质性病变的可能性。有时食管内异物在患者的强力吞咽动作下,食管的生理狭窄和压迹处也可以充分扩张,使食管异物通过全食管抵达胃部,甚至肠道,所以在临床工作中,如果怀疑有异物存在,颈胸部X线检查未发现异物时,应进一步检查胃肠道,观察异物是否自行咽下,这是我们要注意的地方。

\subsection{食管透光性异物}

\begin{figure}[!htbp]
 \centering
 \includegraphics{./images/Image00245.jpg}
 \captionsetup{justification=centering}
 \caption{钡棉造影检查}
 \label{fig5-2-2}
  \end{figure} 

\textbf{【病史摘要】}
 男性,50岁。1小时前喝鱼汤时误将鱼刺咽下,感咽部疼痛,吞咽有异物感。

\textbf{【X线表现】}
 钡棉透视示钡棉滞留于食管上段平第6颈椎水平无法下行,未显示异物的形态。

\textbf{【X线诊断】}  食管上段透光异物(鱼刺)。

\textbf{【评  述】}
 对于较小的食管异物或是不透X线的异物时,应行钡棉造影检查,钡棉往往能停挂在异物处,嘱咐患者反复吞咽甚至饮水钡棉仍能停留在原处,称为挂絮征象,可以对细小异物做出诊断,也是对不透X线异物检查的有效方法,小的透光异物,如鱼刺、小骨片等一般常规透视和摄片检查不易发现,简单的钡餐透视亦不能显示。过小、过细的骨和鱼刺或嵌入咽或食管较深外露于粘膜面较小的异物不易显示挂絮征象,异物损伤了咽或食管的粘膜,患者的自觉症状也难以与异物滞留鉴别,拟建议内镜检查。

钡棉造影检查中需要重点注意的是:①当怀疑异物在主动脉弓水平附近而又需钡棉检查才能确定时,此时应以少量多次吞服钡棉为好,如一次性吞服大量钡剂有可能会牵引异物而致使食管穿孔,甚至累及大血管而致大出血,危及患者生命。②如患者吞服异物时间较长,在透视或摄片中见到异物处食管周围软组织肿胀甚至出现气液平面,则提示食管异物处有炎症感染或脓肿形成,此时吞钡检查会出现钡剂外溢现象且不能排空。

\subsection{反流性食管炎}

\begin{figure}[!htbp]
 \centering
 \includegraphics{./images/Image00246.jpg}
 \captionsetup{justification=centering}
 \caption{反流性食管炎}
 \label{fig5-2-3}
  \end{figure} 

\textbf{【病史摘要】}
 男性,55岁。胸骨后及心窝处烧灼感及疼痛,进食尤其是进热食疼痛加剧,卧位或弯腰时加重,有轻度吞咽困难。

\textbf{【X线表现】}
 右前斜卧位片示:食管内大量钡剂反流,中下段粘膜增粗、紊乱,内见小颗粒征,粘膜未见中断破坏。

\textbf{【X线诊断】}  反流性食管炎。

\textbf{【评  述】}
 X线为检查食管炎症重要的方法,造影检查与内镜证实的符合率可达90%以上,尤其对于中晚期病例。检查中应充盈法、粘膜法和低张双对比造影法相结合,还要应用多种体位及增加腹压等措施。早期可仅见功能异常,表现为吞咽激发的原发性蠕动到主动脉弓水平处终止或减弱,胃食管反流致中下段痉挛性狭窄,狭窄段可有蠕动,钡剂通过时可扩张,通过后又重复出现,但形态不固定、与癌性浸润不同。或者粘膜呈颗粒状,颗粒为1~2mm。表浅溃疡则呈小针刺状龛影。在后期,因瘢痕收缩,而致永久性无明显分界的狭窄及短缩,狭窄段一般4~5cm,多数规则、光滑,也可因瘢痕收缩牵引而不规则,呈假憩室状,低张双重造影显示较好。食管短缩者可见牵引性裂孔疝。发现痉挛性狭窄时,应再做双重造影,以显示粘膜改变。反流性食管炎主要应与食管癌鉴别,食管炎时粘膜改变为渐进性,而食管癌有粘膜中断、破坏、融合及管壁僵直等表现,且边界清楚。难于鉴别者需内镜和病理证实。

\subsection{食管结核}

\begin{figure}[!htbp]
 \centering
 \includegraphics{./images/Image00247.jpg}
 \captionsetup{justification=centering}
 \caption{食管结核}
 \label{fig5-2-4}
  \end{figure} 

\textbf{【病史摘要】}
 男性,61岁。因突发大量呕血入院。近半年来感低热,胸骨后疼痛,有时为背痛,多呈持续性,吞咽时加重,体重减轻。

\textbf{【X线表现】}
 食管钡透示:食管中下段管腔狭窄明显,粘膜纹理粗乱不规则,管壁轮廓不规则呈锯齿状,可见小针刺状龛影,管壁僵硬不明显,仍有一定的扩张度,钡剂通过稍受阻。

\textbf{【X线诊断】}  食管中下段结核(溃疡型)。

\textbf{【评  述】}
 食管结核在临床极为少见。患者多以吞咽困难、吞咽痛或胸骨后疼痛为主诉就诊,缺乏典型的结核中毒症状。有的患者以呕血为首发症状,甚至表现为内科治疗无法控制的消化道大出血。食管结核的病理类型可分为3种:①溃疡型。②增殖型。③颗粒型。

食管结核的钡剂造影检查可以发现下列征象:①溃疡型几乎都发生在食管中段,主要表现为食管管腔溃疡,可见龛影,但也并非所有患者都能见到溃疡所形成的龛影这一征象。由于瘢痕收缩及周围组织粘连而使管腔轻度狭窄或正常,粘膜纹理粗乱不规则,管壁轮廓可不规则呈锯齿状,但管壁僵硬不明显,仍有一定的扩张度,钡剂可顺利通过。②增殖型多见于食管中段,其次为下段。X线检查多显示程度不等的管腔狭窄,为侧壁局限性充盈缺损,大小不一,管壁有一定弹性,钡剂通过缓慢,而无梗阻。在充盈缺损附近有软组织肿块影,为增厚的管壁或肿大的淋巴结,病变区域的粘膜纹理可以正常,或变形甚至完全消失。

食管结核主要应与食管癌进行鉴别,主要鉴别点为:①食管结核多发生于青壮年,年龄较轻,低于45岁,女性多见;而恶性肿瘤发病多在50岁以上,男性多见。②食管结核患者多有肺结核病史或结核接触史,胸部X线检查提示肺部有陈旧性结核或有活动性结核病灶。③食管结核临床症状轻,由于结核性食管狭窄引起的吞咽困难进展较缓慢,呈非进行性吞咽困难,与食物性状无关,病程常较短,抗结核药物治疗有效;食管恶性肿瘤引起的吞咽困难及胸痛呈进行性加重,常在短时期内(3个月至半年)出现重度吞咽困难,且一般情况恶化快。其病程较长,常伴消瘦症状。④食管结核皮肤结核菌素试验(PPD皮试)阳性、血清结核抗体阳性。⑤X线钡剂造影检查:食管结核食管腔有充盈缺损和溃疡,或粘膜呈虫蚀样改变,管壁稍僵硬,纵隔淋巴结结核压迫食管所致充盈缺损,多呈弧形,局部粘膜平整,附近有软组织肿块影或病变周围结核钙化影;而食管癌管壁不整、僵硬,粘膜明显破坏,充盈缺损明显且不规则。

\subsection{化学性食管炎}

\begin{figure}[!htbp]
 \centering
 \includegraphics{./images/Image00248.jpg}
 \captionsetup{justification=centering}
 \caption{化学性食管炎}
 \label{fig5-2-5}
  \end{figure} 

\textbf{【病史摘要】}
 男性,20岁。因误服少量烧碱1小时入院,自述吞咽唾液时胸骨后疼痛伴吞咽困难。

\textbf{【X线表现】}
 食管钡透检查示食管上段管壁欠光整,边缘毛糙,食管蠕动较正常减弱。

\textbf{【X线诊断】}  化学性食管炎(烧碱)。

\textbf{【评  述】}
 化学腐蚀剂分为酸性和碱性两类。食管粘膜接触了化学腐蚀剂后,在病理上会产生一系列的变化:在短时间内(数小时至24小时内),食管壁会产生急性炎症反应,导致食管粘膜水肿、渗出、表面糜烂及激惹性的痉挛收缩而出现食管的早期明显狭窄或梗阻,如临床处理及时,在数天后水肿消退且同时伴随组织修补过程,进一步则进入瘢痕形成时期。食管受损的范围及程度与化学腐蚀剂的性质、浓度、剂量及服食速度有关。在急性期,主要表现为食管痉挛性收缩,管腔狭窄,病变以上管腔稍有扩大,病变部食管壁边缘不光滑,呈不规则或串珠状改变,粘膜像可显示粘膜增粗或消失;在恢复期,上述征象会有改善。但在病变后期,由于纤维组织增生及瘢痕形成,食管腔会显示连续性的进一步狭窄或间断性狭窄,边缘尚光整或稍不规则,粘膜消失,狭窄段以上食管扩张。

需要重点注意的是:一般需在临床紧急处理后,待病情稳定,再行食管钡餐检查。如怀疑有食管穿孔的可能性,则要求停用钡剂造影而改用碘油造影。

\subsection{食管静脉曲张}

\begin{figure}[!htbp]
 \centering
 \includegraphics{./images/Image00249.jpg}
 \captionsetup{justification=centering}
 \caption{食管静脉曲张}
 \label{fig5-2-6}
  \end{figure} 

\textbf{【病史摘要】}
 男性,61岁。因突发呕血2小时入院,患肝硬化、脾大6年,腹水征阳性。

\textbf{【X线表现】}
 食管钡透示食管上、中、下段粘膜增粗、紊乱,其间可见串珠状或蚯蚓状充盈缺损,食管管壁边缘凹凸不平呈锯齿状,钡剂通过缓慢。

\textbf{【X线诊断】}  食管静脉曲张(重度)。

\textbf{【评  述】}
 轻度静脉曲张局限于食管下段,粘膜皱襞略增粗,管腔边缘可呈轻微的锯齿状,管壁张力无明显异常,此时如检查方法不当或观察不仔细可漏诊;中度静脉曲张,病变累及中段,粘膜皱襞明显增粗,呈串珠状或蚯蚓状,食管边缘呈明显的锯齿状,管壁张力欠佳,钡剂通过迟缓;重度静脉曲张,病变累及食管上段,甚至膈上全部食管,管腔明显扩张,正常粘膜被大小、形态不一的圆形、环形充盈缺损取代,形成链状,食管轮廓更加不整,但管壁柔软,钡剂通过更加迟缓。钡剂检查时,钡剂不宜过多,以避免对曲张的静脉形成物理性挤压作用;钡剂宜一次吞下,防止多次吞咽产生的气泡伪影,干扰诊断。

食管静脉曲张表现典型,如检查方法得当,诊断并不困难。需鉴别者有:①气泡影:随钡剂吞入的小气泡随检查时间推移而变动位置或消失,静脉曲张形态可变化但持续存在。②食管癌:虽然下段食管癌可呈息肉状改变,但其病变局限,边界分明,管壁僵直,粘膜中断、破坏,都具特征性,而与静脉曲张不同。

\subsection{食管功能性憩室}

\begin{figure}[!htbp]
 \centering
 \includegraphics{./images/Image00250.jpg}
 \captionsetup{justification=centering}
 \caption{食管功能性憩室}
 \label{fig5-2-7}
  \end{figure} 

\textbf{【病史摘要】}  女性,30岁。因咽部不适,吞咽时有异物感。

\textbf{【X线表现】}
 食管钡透示食管上段主动脉弓下见囊袋状突起,食管壁柔软,钡剂下行顺畅。

\textbf{【X线诊断】}  食管功能性憩室。

\textbf{【评  述】}
 食管钡透时,可于主动脉弓压迹与左主支气管压迹之间,食管显示略膨出,注意不要误认为器质性憩室。

\subsection{食管憩室}

\begin{figure}[!htbp]
 \centering
 \includegraphics{./images/Image00251.jpg}
 \captionsetup{justification=centering}
 \caption{食管中段憩室}
 \label{fig5-2-8}
  \end{figure} 

\begin{figure}[!htbp]
 \centering
 \includegraphics{./images/Image00252.jpg}
 \captionsetup{justification=centering}
 \caption{食管憩室}
 \label{fig5-2-9}
  \end{figure} 

\textbf{【病史摘要】}  女性,35岁。胸背部不适,吞咽时有哽噎感半年。

\textbf{【X线表现】}
 右前斜位及左前斜位示食管中段囊性突起影,内有钡剂充盈,体位改变后,钡剂部分流出。

\textbf{【X线诊断】}  食管中段憩室。

\textbf{【评  述】}
 食管憩室是食管管壁的囊袋状突出,根据发生的部位,分为咽食管憩室、食管中段憩室、膈上食管憩室。X线检查对憩室的诊断起决定作用。因绝大多数的憩室起自食管的前壁或右侧壁,因此,左前斜位或右前斜位显示较好;有时需做俯卧位,以便钡剂进入憩室。食管憩室吞钡检查表现为囊袋状突出影,边缘光滑整齐,口部较小或较宽,大小可变,有时可见粘膜皱襞伸入。咽食管憩室较大时第6颈椎前软组织增宽,其内可见液平;因常有滞留物(食物或粘液等),充钡时密度不均或呈分层状,大的憩室可压迫食管使其前移。食管中段憩室,多位于气管分叉部附近的前壁或侧前壁,憩室顶端可呈牵幕状,颈部较宽。憩室伴有炎症时,其边缘不规则,邻近食管可有痉挛。憩室穿孔时,可见造影剂流入纵隔或气管、支气管。需要重点注意的是:食管中段憩室应与主动脉和左主支气管压迹之间的食管膨出相鉴别;膈上食管大憩室应注意与食管裂孔疝鉴别,憩室囊袋状结构影与食管相连,而食管裂孔疝的膈上疝囊则通过裂孔与胃相连。

\subsection{食管颈椎增生压迹}

\begin{figure}[!htbp]
 \centering
 \includegraphics{./images/Image00253.jpg}
 \captionsetup{justification=centering}
 \caption{食管颈椎增生压迹}
 \label{fig5-2-10}
  \end{figure} 

\textbf{【病史摘要】}
 男性,69岁。颈椎部疼痛,伴左侧前臂麻木,伴有吞咽时哽噎感。

\textbf{【X线表现】}
 食管上段钡透侧位片示:食管上段平4、5椎体水平后缘见弧形压迹,食管壁柔软,颈椎生理弧度僵直,第4、5颈椎体前缘见唇样骨赘形成。

\textbf{【X线诊断】}  食管颈椎增生压迹。

\textbf{【评  述】}
 食管为后纵隔的肌性器官,两端固定,中间可以移动。食管外压性改变可以是由于脊柱椎体骨质过度增生对食管后方产生局部压迫。临床上多有原发疾病的症状,伴有不同程度的吞咽困难或吞咽受阻感。

\subsection{贲门失迟缓症}

\begin{figure}[!htbp]
 \centering
 \includegraphics{./images/Image00254.jpg}
 \captionsetup{justification=centering}
 \caption{贲门失迟缓症}
 \label{fig5-2-11}
  \end{figure} 

\textbf{【病史摘要】}
 女性,45岁。间歇性胸骨后疼痛,吞咽困难2~3年,近年来自觉胸闷心慌,吞咽困难呈持续性,有时伴有呕吐。

\textbf{【X线表现】}
 食管显著扩张,管径5cm左右,下段扩张,呈萝卜根状,腔内多量钡剂潴留,中下段食管蠕动消失。狭窄段食管管壁光滑,柔软(图A、B)。

\textbf{【X线诊断】}  贲门失迟缓症(早期)。

\textbf{【评  述】}
 本病病因不明,一般认为是由于迷走神经的退行性变所致。临床一般见于20~40岁,女性较多。病程长,可数月至数年,常见症状为吞咽困难,胸骨后有阻塞感,以进食固体性食物时明显,症状时轻时重,与精神因素有一定关系。进食较快或梗阻严重时可出现呕吐。严重的食管失迟缓症,胸部X线片可因高度扩张的食管,使纵隔增宽,其内常有液平,而不致误为纵隔肿瘤。一般需钡餐造影确诊。早期,食管轻度扩张,以下半部明显。食管正常蠕动减弱或消失,代之以紊乱的环肌收缩,食管下端呈漏斗状进入膈下,狭窄段为1~4cm,边缘光滑整齐。呼气时狭窄段管腔可略增宽,吸气时变窄,因此,钡剂可随呼吸断续通过。狭窄段的粘膜细而平行,有利于与浸润型食管癌鉴别。晚期,食管显著扩张、延长、迂曲,食管的不规则收缩减弱或消失,或在服钡的瞬间看到几个蠕动波。第一口钡剂有时可少量通过狭窄段,之后连续服钡,需达一定量,常到主动脉弓水平或更高,借助钡剂的重力,才可经狭窄段喷射进入胃内。食管下段呈S形弯曲,下端呈鸟嘴状,边缘光滑、对称。深呼吸时膈肌裂孔迟缓,狭窄段可略变宽(图C、D)。这种随膈肌裂孔张缩而出现的变化,说明管壁柔软,有助于与食管癌鉴别。

\subsection{食管裂孔疝}

\begin{figure}[!htbp]
 \centering
 \includegraphics{./images/Image00255.jpg}
 \captionsetup{justification=centering}
 \caption{食管裂孔疝}
 \label{fig5-2-12}
  \end{figure} 

\textbf{【病史摘要】}
 女性,35岁。胸骨后不适、烧灼、疼痛2年余,饱食后平卧位症状加重,疼痛向肩部放射。

\textbf{【X线表现】}
 左侧膈上见大小约3.5cm×4.0cm的疝囊影,内见粗大弯曲粘膜皱襞,下方见较宽粘膜皱襞通过裂孔与胃相连,贲门位于膈上疝囊内。

\textbf{【X线诊断】}  食管裂孔疝。

\textbf{【评  述】}
 腹腔内脏器移位于胸腔,称为膈疝。腹腔内脏器经食管裂孔疝入胸腔者,称食管裂孔疝,约占膈疝的70%。一般将食管裂孔疝分为四型:①可复型食管裂孔疝。②牵引型食管裂孔疝。③食管旁食管裂孔疝。④先天性短食管型裂孔疝。X线检查为食管裂孔疝的可靠的检查方法。

食管裂孔疝常用检查方法是:①仰卧头低足高位大量服钡使胃过度充盈,之后从右前斜位转至左前斜位,或患者仰卧直腿抬高同时腹部加压。②卧位Valsalva呼吸实验。③胃充满后侧立位弯腰,有利于疝囊的显示。

食管裂孔疝的X线表现有:①膈上疝囊:为疝入胸腔的小部分胃构成,除食管旁型裂孔疝外,皆包括胃食管前庭部。疝囊呈圆柱状或漏斗状,疝囊上方可见下食管括约肌形成的收缩区,称A环。②食管胃环:为食管粘膜与胃粘膜交界部,正常位于膈下,不能显示,裂孔疝时,疝入胸腔,变为疝囊两侧对称性、局限性切迹,称B环。③膈上出现胃粘膜:表现为粗大迂曲的皱襞。④胃小区:个别患者双重造影时,裂孔上出现胃小区。⑤鸟嘴征:仰卧位时,钡剂使贲门轻度张开,形似鸟嘴状,常与其他裂孔疝之X线征象并存。⑥孔征:膈上胃囊充气时,轴位投影于胃底,其形态似孔,称孔征。⑦食管旁型食管裂孔疝:贲门仍位于膈下,疝囊在食管左前方,较大时可压迫食管。根据以上表现,裂孔疝诊断不难。在鉴别诊断方面,应注意不可将食管膈壶腹误为裂孔疝,前者为生理性表现,位于膈上4~5cm一段食管,扩大呈椭圆形,粘膜为纵行纤细的食管粘膜,无下食管括约肌收缩环及疝囊表现。膈上憩室,扩大的囊腔与食管有窄颈相连,其下有一段正常食管通过食管裂孔与贲门相连,胃底正常。

\subsection{食管平滑肌瘤}

\begin{figure}[!htbp]
 \centering
 \includegraphics{./images/Image00256.jpg}
 \captionsetup{justification=centering}
 \caption{食管平滑肌瘤}
 \label{fig5-2-13}
  \end{figure} 

\textbf{【病史摘要】}
 女性,45岁。近2年来进食时有哽噎感,无异物感及疼痛,既往体健,无消瘦。

\textbf{【X线表现】}
 食管钡透检查,示食管中下段椭圆形充盈缺损,见环形征,边缘光滑,食管粘膜未见明显中断、破坏,管腔无明显狭窄,管壁柔软无僵硬。

\textbf{【X线诊断】}  食管中下段平滑肌瘤。

\textbf{【评  述】}
 食管平滑肌瘤是最常见的食管良性肿瘤,占食管良性肿瘤的2/3。食管平滑肌瘤起自平滑肌层或粘膜肌层,位于壁内粘膜下,呈膨胀性生长。平滑肌瘤可发生于食管的各段,以中下段多见。钡餐检查最常见的征象为充盈缺损,呈圆形、椭圆形,边界清楚,较多钡剂通过后,肿瘤周围仍有钡剂存留,形成所谓环形征;肿瘤表面粘膜可变宽或展平,少数病例可见溃疡龛影。钡剂通过肿瘤部位时,在正位,钡剂自肿瘤两侧分流,管腔可变宽;切线位时,钡剂偏流而过。食管平滑肌瘤和壁内其他良性肿瘤的X线征象相似,难以鉴别。恶性肿瘤中食管平滑肌肉瘤罕见,可分息肉型及浸润型,前者多表现为不规则的分叶状或表面有大小不等的息肉状充盈缺损,易发生溃疡;后者与食管癌相似。食管平滑肌瘤则极少发生溃疡,其典型表现为规则的圆形或类圆形充盈缺损。平滑肌瘤与食管癌的鉴别,主要是癌瘤不规则,粘膜破坏,及浸润性生长而致管腔狭窄、僵硬等。

\subsection{食管癌(早期)}

\begin{figure}[!htbp]
 \centering
 \includegraphics{./images/Image00257.jpg}
 \captionsetup{justification=centering}
 \caption{食管癌(早期)}
 \label{fig5-2-14}
  \end{figure} 

\textbf{【病史摘要】}  男性,65岁。咽部不适伴胸骨后轻微疼痛6个月余。

\textbf{【X线表现】}
 食管钡透示:食管上段平第6、7胸椎水平局限性粘膜皱襞扭曲、中断。食管管壁边缘毛糙,呈轻度缩窄,食管蠕动较差。

\textbf{【X线诊断】}  食管上段早期癌。

\textbf{【评  述】}
 本例经手术证实为早期食管癌。早期食管癌病变表浅,X线改变比较轻微,由于造影检查时食管粘膜皱襞显示不清,诊断往往困难。所以早期食管癌的X线检查,应重点注意食管的蠕动、管壁的扩张情况,并多轴位的双重造影像及粘膜像结合诊断。早期食管癌的X线表现主要为:①食管粘膜皱襞的改变:粘膜皱襞增粗、迂曲,有1~2条粘膜皱襞中断,边缘毛糙。②形成小溃疡:在紊乱粗糙的粘膜面上出现小溃疡,可单发或多发,大小不等,一般在0.2~0.4cm,局部管壁轻度痉挛。③局限性小充盈缺损:直径多在0.5cm左右,最大不超过2cm,边缘毛糙,局部粘膜紊乱,少数病例在充盈缺损的病灶中有米粒样龛影。④管壁局限性僵硬:少数病例出现局限性舒展度减低,偏侧性管壁僵直。钡剂在此处通过减慢,呈滞留现象,或出现痉挛性收缩波。在粘膜像阴性情况下,这些征象可作为早期食管癌的定位征象。

\subsection{进展期食管癌(浸润型)}

\begin{figure}[!htbp]
 \centering
 \includegraphics{./images/Image00258.jpg}
 \captionsetup{justification=centering}
 \caption{进展期食管癌(浸润型)}
 \label{fig5-2-15}
  \end{figure} 

\textbf{【病史摘要】}
 男性,56岁。进行性吞咽困难5个月余,近1个月来感胸骨后疼痛,只能进流质,并有呕吐症状。

\textbf{【X线表现】}
 食管钡透示:食管中下段管腔环形狭窄,钡剂下行受阻,狭窄段呈漏斗状,管壁僵硬,蠕动消失,狭窄段以上食管扩张明显。

\textbf{【X线诊断】}  进展期食管癌(浸润型)。

\subsection{进展期食管癌(溃疡型)}

\begin{figure}[!htbp]
 \centering
 \includegraphics{./images/Image00259.jpg}
 \captionsetup{justification=centering}
 \caption{进展期食管癌(溃疡型)}
 \label{fig5-2-16}
  \end{figure} 

\textbf{【病史摘要】}
 女性,48岁。进行性吞咽困难3个月,近期进食流质时出现哽噎,消瘦。

\textbf{【X线表现】}
 食管钡透示:食管中下段管腔局限性狭窄,粘膜皱襞中断破坏,并见一较大龛影,与食管纵轴一致,切线位位于食管轮廓之内。

\textbf{【X线诊断】}  进展期食管癌(溃疡型)。

\subsection{进展期食管癌(增生型)}

\begin{figure}[!htbp]
 \centering
 \includegraphics{./images/Image00260.jpg}
 \captionsetup{justification=centering}
 \caption{进展期食管癌(增生型)}
 \label{fig5-2-17}
  \end{figure} 

\textbf{【病史摘要】}
 男性,45岁。进行性吞咽困难3个月余,近期进食固体类食物时下咽困难,流质尚可,既往体健。

\textbf{【X线表现】}
 食管钡透示:食管中上段管腔内见不规则充盈缺损,管腔呈偏心性狭窄,充盈缺损基底部管壁僵硬,蠕动消失。

\textbf{【X线诊断】}  进展期食管癌(增生型)。

\textbf{【评  述】}
 本病一般分为三型:浸润型、溃疡型、增生型。进展期食管癌侵及肌层后,进展加快,X线征象也日益显著,主要表现为:①粘膜皱襞增粗、紊乱、中断、破坏,代之以肿瘤形成的不规则影。②病变区管腔不规则、偏心性狭窄,管壁僵硬,伴有梗阻征,其近端食管扩张。③腔内不规则的充盈缺损,其表面常有破坏形成的龛影。④不规则的龛影,位于食管轮廓之内。上述征象常混合存在。

食管癌的类型不同,X线表现也各具特征:①浸润型:管腔呈环形狭窄,范围局限,一般为3~5cm,严重时呈漏斗状,管壁僵硬,边缘多较光滑,上段食管扩张明显。②增生型:以腔内不规则的充盈缺损及管腔偏心性不规则的狭窄为特征,充盈缺损表面常有不规则的溃疡,为肿瘤坏死所致。③溃疡型:以边界清楚、形态不规则的龛影为特征。龛影多较长,与食管纵轴一致,在切线位多在食管轮廓线内,较深时可超出食管轮廓以外。溃疡周围隆起明显者,可见环堤征。增生型食管癌需注意与良性肿瘤中最多见的平滑肌瘤相鉴别,后者切线位也可表现为管腔内圆形或椭圆形充盈缺损,致食管管腔狭窄,但其边缘一般光滑、肿瘤区粘膜皱襞可消失而周围粘膜皱襞正常,管壁柔软,正位可显示钡剂环绕形成的环形征,管腔可变宽,管腔外可见软组织肿块影是其特征。增生型食管癌,特别是较大者与恶性癌肉瘤X线区分有一定难度,可以作为参考的是癌肉瘤虽然肿瘤较大,与食管癌相比较,患者临床梗阻症状一般较轻,癌肉瘤多发生在食管中下段,管腔外往往可显示有软组织块影,而食管癌最常见发生在食管中上段,管腔外较少形成软组织肿块影,两者表现不同,有时也需结合内镜病理活检方可鉴别。

\subsection{食管平滑肌肉瘤}

\begin{figure}[!htbp]
 \centering
 \includegraphics{./images/Image00261.jpg}
 \captionsetup{justification=centering}
 \caption{食管平滑肌肉瘤}
 \label{fig5-2-18}
  \end{figure} 

\textbf{【病史摘要】}
 男性,62岁。自述吞咽困难3个月,胸骨后感疼痛,既往体健。

\textbf{【X线表现】}
 食管钡透示:食管中下段管腔呈梭形扩张,内见多枚大小不等类圆形充盈缺损,食管壁尚光滑,食管粘膜皱襞消失。

\textbf{【X线诊断】}  食管平滑肌肉瘤。

\textbf{【评  述】}
 本病少见。好发于食管中下段,多呈息肉状突入管腔,少数为浸润性生长,肿瘤常限于粘膜或粘膜下层,个别为环形浸润,很少转移,预后较好。临床表现不具特征性,常因吞咽困难就诊。X线表现典型者为食管中下段腔内息肉状充盈缺损,基底小,可有蒂,局部管腔扩张。少数不典型者,如呈环形浸润性生长者与增生型食管癌难于诊断,需结合内镜病理活检方可鉴别。

\section{胃部病变}

\subsection{胃憩室}

\begin{figure}[!htbp]
 \centering
 \includegraphics{./images/Image00262.jpg}
 \captionsetup{justification=centering}
 \caption{胃底憩室}
 \label{fig5-3-1}
  \end{figure} 

\textbf{【病史摘要】}
 女性,31岁。上腹部不适1周,无嗳气、反酸,无腹胀。体格检查:上腹部无明显压痛,肝、脾未及,心肺阴性。

\textbf{【X线表现】}
 胃钡透示:胃底部见囊袋状突起,边缘光滑整齐,内见钡剂残留,颈部狭窄,胃底粘膜伸入囊袋。

\textbf{【X线诊断】}  胃底憩室。

\textbf{【评  述】}
 本病一般为单发,80%位于贲门下方小弯侧的后壁,其次为幽门前区。憩室呈圆形或囊袋状,颈部狭窄、光滑。缺乏肌层者无收缩力,常因食物残留而致憩室炎。胃周粘连牵拉所致者,口部较宽,很少有食物残留。憩室有完整的粘膜层。胃憩室多无症状。伴发憩室炎时,可有腹痛、腹胀、恶心、呕吐及出血表现。X线表现主要有:①憩室外形光滑,如囊袋状影突出于胃腔之外,但基底部与胃相连,颈部略细。②粘膜像可见胃粘膜皱襞自颈部与憩室相连。③如合并憩室炎,其外形变得不规则,并可见局部胃壁痉挛变形等改变。根据憩室上述X线表现特点,尤其要注意粘膜皱襞形态,易与胃穿透性溃疡鉴别。憩室呈光滑的囊袋状,有正常的粘膜伸入其内,而没有粘膜皱襞纠集等表现;穿透性溃疡无粘膜皱襞伸入其中是与胃憩室区别之要点。

\subsection{胃底静脉曲张}

\begin{figure}[!htbp]
 \centering
 \includegraphics{./images/Image00263.jpg}
 \captionsetup{justification=centering}
 \caption{胃底静脉曲张}
 \label{fig5-3-2}
  \end{figure} 

\textbf{【病史摘要】}
 男性,61岁。患肝硬化多年,突发呕血1天伴黑便。体格检查:肝、脾大,腹水征阳性,腹壁静脉曲张,功能异常。

\textbf{【X线表现】}
 胃钡透示:胃底粘膜皱襞增宽迂曲,呈蚯蚓状,边缘光滑,未见明显中断破坏。

\textbf{【X线诊断】}  胃底静脉曲张。

\textbf{【评  述】}
 本病患者除具有门脉高压症的表现外(如肝脾肿大、脾功能亢进、肝功能异常、腹水、腹壁静脉曲张等),主要表现为呕血及黑便。X线检查具有安全、简便、准确的特点,易为患者接受。胃底静脉曲张常与食管静脉曲张并存,但也可单独存在,双对比造影可提高其显示率。胃底静脉曲张一般可分为两型:①泡沫型:胃贲门区及胃底粘膜呈葡萄状或息肉状透亮区,直径1~2cm,周围见薄层钡剂环绕,形如泡沫状。②肿块型:胃贲门区及胃底呈分叶状或团块状边缘光滑的充盈缺损。除上述表现外,胃底静脉曲张时,胃底与膈肌间距可增大,胃贲门角增大。因多伴有脾肿大,可造成胃的压迫性移位。胃底静脉曲张主要应与胃癌鉴别,静脉曲张形成的肿块影边缘光滑锐利,胃壁柔软(可借助气钡双重造影、呼吸动作或心脏搏动观察),贲门及腹段食管不被侵犯,结合临床病史,可以鉴别,个别病例可借助内镜检查或选择性血管造影帮助鉴别。

\subsection{胃内异物}

\begin{figure}[!htbp]
 \centering
 \includegraphics{./images/Image00264.jpg}
 \captionsetup{justification=centering}
 \caption{胃内异物(胃柿石)}
 \label{fig5-3-3}
  \end{figure} 

\textbf{【病史摘要】}
 女性,35岁。上腹部不适伴疼痛2个月余,可自行缓解,近期有大量进食柿子史。

\textbf{【X线表现】}
 胃钡透示:胃窦部见类圆形充盈缺损影,大小4cm×3.5cm左右,边缘稍毛糙,表面见凹凸不平的不规则钡斑,体位改变后,充盈缺损位置改变。

\textbf{【X线诊断】}  胃内异物(胃柿石)。

\textbf{【评  述】}
 柿子、豆类、毛发、绒线及粘液性物质进入胃内,因机械作用而相互缠绕成团,形成胃石。在产柿地区,胃柿石是最常见的胃石。因进食大量不成熟柿子后,与胃酸起作用,凝结成块,而成胃石。X线表现胃内可见类圆形充盈缺损影,体积可以很大,也可分成数块,表面凹凸不平,呈现不规则的钡斑。充盈缺损可在胃内移动,压迫或变动体位,其位置有变化。此外,周围胃壁柔软,蠕动正常,这些特点可与胃肿瘤相鉴别。特别注意的是要结合临床病史做出最后诊断。

\subsection{幽门肌肥厚症}

\begin{figure}[!htbp]
 \centering
 \includegraphics{./images/Image00265.jpg}
 \captionsetup{justification=centering}
 \caption{幽门肌肥厚}
 \label{fig5-3-4}
  \end{figure} 

\textbf{【病史摘要】}
 女性,40岁。右上腹部饱胀数月,无明显疼痛,时有恶心、呕吐。体格检查:中等体质,腹软,肝、脾未及,右上腹无明显压痛,心、肺阴性。

\textbf{【X线表现】}
 胃钡透示:胃幽门管细而长,其纵行粘膜皱襞显示呈双轨征,十二指肠球基底部形成蘑菇征。

\textbf{【X线诊断】}  幽门肌肥厚。

\textbf{【评  述】}
 本病是胃幽门环肌高度肥厚所致。多见于成年人。腹痛、腹胀、恶心、呕吐等为常见症状。幽门肌肥厚X线表现主要有:①钡剂通过狭窄的幽门管,幽门管显示狭窄而延长,呈线条状。②由于肥大的幽门肌终止于十二指肠球基底部,造成肥大的环形肌肉压迫球部基底部形成蘑菇征。③钡剂进入狭窄的幽门管,当充盈不全时似一个细长的鸟嘴突向十二指肠球部。④狭窄之幽门管纵行的粘膜皱襞显影形成双轨征。X线诊断应与幽门痉挛鉴别,但幽门痉挛无以上X线征象,确诊并不困难。

\subsection{胃息肉}

\begin{figure}[!htbp]
 \centering
 \includegraphics{./images/Image00266.jpg}
 \captionsetup{justification=centering}
 \caption{胃息肉}
 \label{fig5-3-5}
  \end{figure} 

\textbf{【病史摘要】}
 女性,43岁。上腹部疼痛不适,无食欲减退、嗳气、反酸。体格检查:腹软,上腹部轻压痛,肝、脾未及,心、肺阴性。

\textbf{【X线表现】}
 胃钡透示:胃窦部圆形充盈缺损,边缘整齐锐利,表面光滑,周围胃壁柔软,无僵硬,蠕动正常。

\textbf{【X线诊断】}  胃息肉(胃窦部)。

\textbf{【评  述】}
 本病常为单发,也可多发,多发者称胃息肉病。典型X线表现呈圆形或类圆形充盈缺损突入于胃腔,有蒂或无蒂,直径一般小于2cm。多见于胃窦及胃体下部,幽门前区带蒂息肉可脱入十二指肠内。息肉表面光滑整齐。组织学检查可分为腺瘤性息肉及增生性息肉两类。腺瘤性息肉多位于胃窦部,常伴萎缩性胃炎,可分为腺瘤及乳头状瘤,后者可呈菜花状。增生性息肉是在慢性胃炎基础上发生的,很少超过1cm。腺瘤性息肉被认为是癌前期病变,可与胃癌同时存在。临床一般多无症状,少数可有上腹部不适疼痛。带蒂者可随蠕动或压迫而移位。在幽门前区者突入十二指肠时,表现为十二指肠球部的充盈缺损,充盈加压检查或双重造影法可显示其带蒂。乳头状腺瘤可不规则。息肉应与息肉样癌鉴别,息肉样癌的充盈缺损一般多大于2cm,形态不规则,表面粗糙,肿瘤与胃壁交界欠清,一般无蒂。

\subsection{胃窦炎}

\begin{figure}[!htbp]
 \centering
 \includegraphics{./images/Image00267.jpg}
 \captionsetup{justification=centering}
 \caption{胃窦炎伴幽门痉挛}
 \label{fig5-3-6}
  \end{figure} 

\textbf{【病史摘要】}
 女性,40岁。左上腹部不适3个月余,食欲减退,时有恶心、呕吐。体格检查:腹软,肝、脾未及,上腹压痛,心、肺阴性。

\textbf{【X线表现】}
 胃窦部粘膜皱襞增粗、紊乱,幽门管痉挛,钡剂通过幽门管稍受阻,胃窦壁轮廓见锯齿状影,胃窦壁柔软,蠕动增强。

\textbf{【X线诊断】}  胃窦炎伴幽门痉挛。

\textbf{【评  述】}
 胃窦炎为局限于胃窦部的慢性炎症,可为浅表性或萎缩性,十分常见。轻症无阳性X线表现。常见的异常征象有:胃窦部粘膜皱襞增粗、紊乱。正常的胃窦部粘膜皱襞较体部细小,胃窦炎时可增大。紊乱的粘膜皱襞,即使在半收缩状态也呈横行,使窦部轮廓呈光滑的锯齿状。增粗的粘膜皱襞可呈息肉状,并随蠕动、舒缩或压迫而变形。肌层受累者,窦部呈向心性狭窄,但仍可见呈纵行的粘膜皱襞。窦部因环形及纵行肌的收缩与增厚而变短、变窄,其界线呈渐进性或较清楚。肌层的痉挛或增厚可致幽门前区小弯侧呈弧形压迫。幽门管可变窄并伸长。粘膜下层的增厚,使粘膜活动度增加,易形成粘膜脱垂。胃小沟及胃小区增宽、增大。窦部痉挛及分泌功能增强为常见的功能异常。胃窦炎常致窦部狭窄而应与胃窦癌鉴别。胃窦炎的狭窄,形态可变、粘膜皱襞存在、轮廓也较整齐,而胃窦癌表现为胃窦狭窄壁僵硬,与正常胃段分界陡峭呈截断征象,粘膜皱襞破坏,典型者可呈肩胛征。

\subsection{慢性胃炎}

\begin{figure}[!htbp]
 \centering
 \includegraphics{./images/Image00268.jpg}
 \captionsetup{justification=centering}
 \caption{慢性胃炎}
 \label{fig5-3-7}
  \end{figure} 

\textbf{【病史摘要】}
 男性,65岁。上腹部疼痛不适,嗳气、反酸、餐后饱胀数月。体格检查:腹软,肝、脾未及,右上腹压痛,心、肺阴性。

\textbf{【X线表现】}
 胃钡透示:胃体、胃窦粘膜皱襞增粗、肥厚、紊乱,部分呈弯曲、交叉状,胃体、胃窦处胃壁毛糙,压迫后胃壁柔软,胃蠕动正常。

\textbf{【X线诊断】}  慢性胃炎。

\textbf{【评  述】}
 本病为成人常见病,病因尚未完全明确。病理上慢性胃炎可分为慢性浅表性胃炎和慢性萎缩性胃炎及慢性肥厚性胃炎。慢性胃炎时粘膜充血、水肿、炎性细胞浸润及纤维结缔组织增生。轻微者肉眼难以发现,较重者粘膜皱襞增粗、迂回呈脑回状;部分萎缩性胃炎粘膜层萎缩变薄,皱襞细小。慢性胃炎病程较长,可长期反复发作。一般临床表现为食欲不振、腹痛、腹胀、恶心、呕吐、嗳气等。萎缩性胃炎可有贫血、营养不良、腹泻等表现。慢性胃炎的X线表现主要为粘膜皱襞增粗、迂曲、走行异常、失去与小弯平行的特点,体部及窦部粘膜皱襞超过0.5cm,甚至超过1cm;充盈像,因粘膜皱襞增粗、迂曲而使小弯侧凹凸不平,但形态不变,蠕动正常,而不致误为肿瘤。除上述外,还可见分泌功能增强及蠕动增强等变化。部分萎缩性胃炎粘膜皱襞纤细、稀少,服钡或双重造影的气体稍多,胃呈轻度扩张时,皱襞即可变平,甚至大弯侧也可变得光滑。胃小区增大,多数大于3mm,而且粗糙不规则。慢性胃炎常与溃疡并存,而有相应X线征。

\subsection{腐蚀性胃、十二指肠炎}

\begin{figure}[!htbp]
 \centering
 \includegraphics{./images/Image00269.jpg}
 \captionsetup{justification=centering}
 \caption{腐蚀性胃、十二指肠炎}
 \label{fig5-3-8}
  \end{figure} 

\textbf{【病史摘要】}
 女性,35岁。因进食时吞咽困难、呕吐频繁伴胸痛入院,数月前有硫酸误服致上消化道灼伤史。

\textbf{【X线表现】}
 胃钡透示:胃、十二指肠高度狭窄、壁僵硬,粘膜皱襞消失,部分边缘可见针尖样突出的小溃疡。

\textbf{【X线诊断】}  腐蚀性胃、十二指肠炎。

\textbf{【评  述】}
 吞服酸性腐蚀剂类物质易损伤食管、胃、十二指肠,若腐蚀剂浓度高、量大、接触时间长,可引起食管、胃、十二指肠以及空肠的烧灼性炎症。病理改变主要为粘膜及粘膜下层坏死。溃疡形成,晚期纤维瘢痕形成导致不同程度各种各样的狭窄。X线表现早期改变为胃粘膜皱襞粗大、水肿,可有溃疡龛影。胃蠕动消失。晚期可见胃腔狭窄呈漏斗状,胃壁边缘不规则如锯齿状,胃幽门瘢痕性狭窄。由于有患者误服腐蚀剂病史,故诊断一般不难。

\subsection{胃粘膜脱垂}

\begin{figure}[!htbp]
 \centering
 \includegraphics{./images/Image00270.jpg}
 \captionsetup{justification=centering}
 \caption{胃粘膜脱垂}
 \label{fig5-3-9}
  \end{figure} 

\textbf{【病史摘要】}
 男性,45岁。因上腹部不适,嗳气、反酸入院。体格检查:腹软,肝、脾未及,右上腹压痛,心、肺阴性。

\textbf{【X线表现】}
 胃钡透示:幽门管变宽,内见条状平行胃粘膜皱襞,十二指肠球部呈伞状,基底部见类圆形充盈缺损影。

\textbf{【X线诊断】}  胃粘膜脱垂。

\textbf{【评  述】}
 胃粘膜进入十二指肠称为胃粘膜脱垂,常为可复性。常见症状为上腹部疼痛,可随体位改变而缓解。X线表现随脱垂的粘膜数量及程度而异,一般可见幽门管增宽,其内可见条形皱襞。十二指肠球内见圆形或椭圆形充盈缺损,位于正中或呈偏侧性,随窦部的加压或体位的改变而时隐时现。球底一般呈伞状。诊断时应与幽门前区带蒂肿瘤脱入十二指肠相鉴别。后者形态、大小固定,不随压迫变形,回纳后,幽门前区仍可见之。钡餐检查时,当怀疑有胃粘膜脱垂可能时,需要重点注意的是:①应充分利用立位检查或腹部加压检查。②尽可能使球部纵轴走行方向与X线方向垂直。③胃窦处于舒张状态时摄片。④诊断胃粘膜脱垂时,必须肯定球底部之阴影为胃粘膜皱襞,除外体位不当造成的假象。

\subsection{胃溃疡}

\begin{figure}[!htbp]
 \centering
 \includegraphics{./images/Image00271.jpg}
 \captionsetup{justification=centering}
 \caption{胃角溃疡}
 \label{fig5-3-10}
  \end{figure} 

\textbf{【病史摘要】}
 男性,55岁。上腹部不适数年,进餐后可缓解,近一周上腹部疼痛加重,具有周期性及节律性,伴恶心、呕吐、嗳气、反酸。体格检查:腹软,肝、脾未及,右上腹压痛明显,心、肺阴性。

\textbf{【X线表现】}
 胃钡透示:胃角处见一突出于胃腔外的乳头状影,基底部见狭颈征,龛周粘膜皱襞纠集。

\textbf{【X线诊断】}  胃角溃疡。

\textbf{【评  述】}
 本例经胃镜检查病理证实为胃体小弯侧溃疡。上消化道钡餐造影显示胃角处见一突出于胃腔外的乳头状影,基底部见狭颈征,龛周粘膜皱襞纠集,符合良性胃溃疡的X线表现,故诊断不难。需要鉴别的是胃小弯侧恶性溃疡,后者以壁龛及邻近胃壁变化为主要表现。龛影多数较浅而大,形态多不规则,具有特征性的为口部指压迹征和裂隙征,与良性溃疡平坦的口部出现的狭颈征、项圈征对比分明。

\subsection{幽门管溃疡}

\begin{figure}[!htbp]
 \centering
 \includegraphics{./images/Image00272.jpg}
 \captionsetup{justification=centering}
 \caption{幽门管溃疡}
 \label{fig5-3-11}
  \end{figure} 

\textbf{【病史摘要】}
 男性,45岁。上腹部疼痛伴嗳气、反酸2个月,近期疼痛加重伴呕吐。体格检查:腹软,肝、脾未及,右上腹压痛明显,心、肺阴性。

\textbf{【X线表现】}
 上消化道钡餐造影示:幽门管区见突出于胃腔外的三角形龛影,底部狭窄,呈项圈征。

\textbf{【X线诊断】}  幽门管溃疡。

\textbf{【评  述】}
 本病为常见病,发病机制不甚明了,好发年龄为20~50岁。胃溃疡常单发,多在小弯及胃角处,其次为胃窦部,其他部位少见。病理改变主要为胃壁溃烂缺损,形成壁龛。溃疡先从粘膜开始并逐渐侵及粘膜下层,常深达肌层。X线检查是胃溃疡的重要检查方法,尤其是气钡双重造影,可显示小而表浅的溃疡。溃疡病的X线表现,可分为直接与间接征象,前者为X线诊断的主要依据。

1.直接征象 龛影为溃疡充钡后在X线上的反映,是溃疡的直接征象。在正位像上,呈圆形或类圆形影;如果溃疡内的钡剂较少,仅四周壁附薄层钡剂,则呈环形,即所谓环形龛影。在切线位上,龛影突出于胃轮廓之外,多呈乳头状,或为半圆形及锥形。边缘光滑整齐,底部平整。在切线位上还可显示:①粘膜线:为溃疡口部宽1~2mm的透光线影,见于口部的上缘、下缘或横贯整个口部。②狭颈征:龛影口部明显狭小,使龛影犹如一个狭长的颈。③项圈征:龛影口部的透明带,宽0.5~1cm,犹如一项圈。粘膜线、狭颈征、项圈征皆为溃疡周围炎性水肿所致。④粘膜纠集,溃疡周围的粘膜皱襞,因瘢痕收缩向壁龛均匀性纠集,直达龛影,呈星芒状。

2.间接征象 下述X线表现常见于胃溃疡,也可因胃癌所致,不具有特异性,但在综合分析时有一定价值。①胃小弯短缩:是小弯侧溃疡纤维组织增生,牵拉幽门及贲门靠近。②胃大弯侧指状切迹:胃小弯侧溃疡,因环形肌痉挛性收缩,在溃疡的对侧可见一指状切迹,立位时明显。③幽门梗阻及狭窄:幽门及其邻近部的溃疡可致幽门持久性痉挛,或因瘢痕形成而使幽门梗阻。X线可见空腹胃潴留液增多,幽门管狭小,钡剂通过困难。④胃液分泌增多:在无幽门梗阻的情况下,出现少至中等量的胃内空腹潴留液,使钡剂不易附着于胃壁而难以显示粘膜皱襞。⑤胃蠕动的变化:蠕动增强或减弱,张力增高或降低,排空加速或延缓。⑥局限性压痛:龛影部位常有明显的局限性压痛。

\subsection{穿透性溃疡}

\begin{figure}[!htbp]
 \centering
 \includegraphics{./images/Image00273.jpg}
 \captionsetup{justification=centering}
 \caption{胃角穿透性溃疡}
 \label{fig5-3-12}
  \end{figure} 

\textbf{【病史摘要】}
 男性,48岁。胃溃疡病史3年,近一周来上腹部疼痛加剧伴恶心、呕吐。体格检查:贫血貌,上腹部拒按,压痛明显,心、肺阴性。

\textbf{【X线表现】}
 上消化道钡餐检查示:胃小弯侧腔外见一1.8cm×2.0cm大小囊袋状影,轮廓尚光整,颈部狭长,狭颈征明显。

\textbf{【X线诊断】}  胃角穿透性溃疡。

\textbf{【评  述】}
 本例患者经手术证实为穿透性溃疡。穿透性溃疡为胃溃疡的特殊类型,其特点为龛影大而深,其深度与大小均超过1.0cm,形如囊袋状,狭颈征显著。需注意此征象与较大的胃憩室相鉴别:胃憩室发生部位以胃底贲门区后壁为多见,憩室内可显示胃粘膜皱襞影;穿透性溃疡X线表现如周围较广泛的水肿带以及粘膜皱襞向溃疡口部纠集征象与胃憩室不同。胃溃疡根据以上典型表现,诊断一般不难。但有时因瘢痕组织的不规则增生或溃疡比较扁平者易与恶性溃疡混淆。良性溃疡和恶性溃疡的鉴别诊断,应从龛影的形态、溃疡的位置、溃疡的口部、周围粘膜皱襞的情况、邻近胃壁的柔软与蠕动等多方面综合分析,详见下表。

胃良、恶性溃疡的X线鉴别诊断

\includegraphics{./images/Image00274.jpg}

\subsection{胃平滑肌瘤}

\begin{figure}[!htbp]
 \centering
 \includegraphics{./images/Image00275.jpg}
 \captionsetup{justification=centering}
 \caption{胃平滑肌瘤}
 \label{fig5-3-13}
  \end{figure} 

\textbf{【病史摘要】}
 女性,42岁。吞咽困难半年余,无明显疼痛,无消瘦。体格检查:腹软,肝、脾未及,腹部无压痛,心、肺阴性。

\textbf{【X线表现】}
 上消化道钡餐造影示:胃贲门下部见一类圆形充盈缺损,边缘光滑清晰,中央部可见小钡斑。

\textbf{【X线诊断】}  胃平滑肌瘤。

\textbf{【评  述】}
 本例患者经手术治疗病理证实为胃平滑肌瘤。胃平滑肌瘤来源于中胚层组织,大多在5cm以下,可分为胃内型、胃壁型、胃外型。X线表现主要有:①胃内隆起性病变:正面呈圆形、椭圆形,位于大、小弯者显示其侧面像为半圆形。充盈像时呈边缘光滑的充盈缺损。②粘膜皱襞:肿瘤表面被附粘膜,可见粘膜皱襞通过肿物征象,粘膜被抬起形成桥形皱襞,或被推开形成粘膜皱襞的躲避、迂回征象。肿瘤较大时,皱襞受压变薄、变平以致消失。③中心凹陷:肿瘤表面,尤其顶部常形成小凹陷,造影时出现小钡斑,即所谓的中心性凹陷。④肿瘤触诊:平滑肌瘤较硬,压之无变形。⑤钙化:平滑肌瘤可发生钙化,X线检查可见钙化斑。⑥周围改变:平滑肌瘤对周围无浸润,胃轮廓无僵硬。⑦恶性变:平滑肌瘤可恶变成为肉瘤。一般肿瘤较大、形态不规则、中心溃疡大而深又不规则时,应考虑恶变的可能性。

\subsection{胃淋巴瘤}

\begin{figure}[!htbp]
 \centering
 \includegraphics{./images/Image00276.jpg}
 \captionsetup{justification=centering}
 \caption{胃淋巴瘤}
 \label{fig5-3-14}
  \end{figure} 

\textbf{【病史摘要】}
 女性,40岁。上腹部不适3个月,食欲不振、消瘦,近期出现低热。体格检查:腹软,肝、脾未及,腹部无压痛,心、肺阴性,右侧锁骨上窝触及一类圆形肿块。

\textbf{【X线表现】}
 上消化道钡餐造影示:胃底大弯侧见不规则充盈缺损影,粘膜皱襞不规则增粗,胃壁柔韧度减弱,胃蠕动及收缩存在。CT检查示:胃底部胃壁局限性增厚,增强后均匀中度强化,壁柔软。

\textbf{【X线诊断】}  胃淋巴瘤。

\textbf{【评  述】}
 本例患者经手术治疗病理证实为胃淋巴瘤。胃肠道是淋巴结外淋巴瘤的最多见部位,最好累及胃。胃淋巴瘤可以是全身淋巴瘤的一部分,也可以是唯一的原发部位,多见于非霍奇金淋巴瘤。按形态学分类为:肿块型、溃疡型、浸润型和结节型。好发部位是胃体小弯侧和后壁。临床表现有上腹部疼痛、消瘦及上腹部肿块,可伴有全身淋巴瘤的其他表现。胃淋巴瘤X线常表现为局限性或广泛浸润性表现,前者为粘膜皱襞不规则、粗大,胃壁柔韧度消失,位于胃窦时呈漏斗状狭窄;后者为巨大粘膜皱襞的改变,排列紊乱,胃腔缩窄或变形,但其缩窄与变形程度不及浸润型胃癌。胃淋巴瘤缺乏特征性的X线表现,因此常不易与胃癌及其他肿瘤鉴别。但如下特征有助于本病的诊断:病变虽然广泛,但胃蠕动与收缩仍然存在,胃部病灶明显但临床一般情况较好,胃粘膜较广泛增粗,形态比较固定,临床有其他部位淋巴瘤的表现。

\subsection{早期胃癌(Ⅰ型)}

\begin{figure}[!htbp]
 \centering
 \includegraphics{./images/Image00277.jpg}
 \captionsetup{justification=centering}
 \caption{早期胃癌(Ⅰ型)}
 \label{fig5-3-15}
  \end{figure} 

\textbf{【病史摘要】}
 男性,61岁。上腹部不适、食欲不振、嗳气、反酸近1年,无黑便,无呕吐。体格检查:腹软,肝、脾未及,上腹部轻压痛,心、肺阴性。

\textbf{【X线表现】}
 上消化道钡餐造影示:胃窦部见一椭圆形充盈缺损影,边缘尚光整,周围粘膜皱襞中断破坏,基底部较宽,表面尚平坦,未见明显糜烂点,局部胃壁稍僵硬。

\textbf{【X线诊断】}  早期胃癌(Ⅰ型)。

\textbf{【评  述】}
 本例患者经手术病理证实结果为胃窦部早期胃腺癌,未侵犯胃粘膜肌层,未见明显转移灶。患者X线表现示胃窦部隆起型病变,轮廓尚光整,边缘稍粗糙,周围粘膜皱襞见中断破坏,局部胃壁稍僵硬,故诊断为早期胃癌(Ⅰ型)。Ⅰ型早期胃癌即表面隆起型,为肿瘤向胃腔内突出高度超过周围粘膜的5mm。早期胃癌发展缓慢,短者1~2年,长着可10余年无明显变化。但一般隆起型发展较快,而溃疡型发展较慢。胃癌向深层侵犯较快,而在粘膜内浸润较慢。隆起型早期胃癌,大小不一,直径多大于2cm,圆形、类圆形或不规则形,边界清楚,基底部较宽,极个别者可有蒂。肿瘤表面粗糙,常伴出血及糜烂。值得注意的是,由于早期胃癌病变范围较小,故X线检查可发现其存在,但最终诊断需要密切结合内镜与活检结果方能明确。

\subsection{早期胃癌(Ⅱa型)}

\begin{figure}[!htbp]
 \centering
 \includegraphics{./images/Image00278.jpg}
 \captionsetup{justification=centering}
 \caption{胃窦部早期胃癌(Ⅱa型)}
 \label{fig5-3-16}
  \end{figure} 

\textbf{【病史摘要】}
 女性,41岁。上腹部不适伴嗳气、反酸、食欲减退1个月。体格检查:腹软,上腹部轻度压痛,未扪及包块,心、肺阴性。

\textbf{【X线表现】}
 上消化道钡餐造影示:胃窦部见一形态不规则的平盘状充盈缺损影,表面凹凸不平,见小钡斑(箭头)。

\textbf{【X线诊断】}  胃窦部早期胃癌(Ⅱa型)。

\textbf{【评  述】}
 本例经手术病理证实为胃窦部早期胃癌Ⅱa型。胃双对比造影可显示粘膜面的细微结构而对早期胃癌的诊断具有重要价值。早期胃癌的X线表现主要为:①Ⅰ型(隆起型):肿瘤与周围粘膜有明显的分界,形态多不规则,呈息肉状、分叶状、菜花状等。表面不光滑,因有表层坏死形成粘膜缺损,双对比造影可见不规则钡斑。其基底部与正常粘膜分界清楚,侧面观可为广基型、无蒂型、有蒂型,有蒂者肿瘤在2cm以上。②Ⅱa型(表面隆起型):肿瘤形态不规则,呈平坦的息肉状、花坛状、平盘状等。表面有不规则凹凸而显示为不规则钡斑,基底部多为广基型。③Ⅱb型(表面平坦型):双对比造影主要表现为胃小区的细微变化,如胃小区粗大、紊乱,呈不规则之颗粒状形态。④Ⅱc型(表面凹陷型):肿瘤表现为形态不规则之表浅溃疡,呈楔形、星芒状等,边缘清楚、锐利,病变一般较小,病变周围伴有粘膜皱襞纠集现象,其粘膜皱襞尖端有明显的病理变形,如杵状增粗、笔尖样变细、阶梯状变薄、皱襞融合等。⑤Ⅲ型(凹陷型):为一深溃疡,其深溃疡本身不是癌,只于溃疡口边缘有癌浸润。X线表现其深溃疡形态很像良性溃疡,难以鉴别,只于溃疡口边缘显示轻微毛糙不平为其特征。由于早期胃癌的病变范围较小,因而X线双重造影检查的重点在于发现它的存在,最后的诊断需要密切结合内镜与活检方能明确。

\subsection{早期胃癌(Ⅱc型)}

\begin{figure}[!htbp]
 \centering
 \includegraphics{./images/Image00279.jpg}
 \captionsetup{justification=centering}
 \caption{胃窦小弯侧早期胃癌(Ⅱc型)}
 \label{fig5-3-17}
  \end{figure} 

\textbf{【病史摘要】}
 男性,45岁。上腹部不适、疼痛3个月余,近1个月疼痛加重。体格检查:腹软,剑突下压痛,未扪及包块,心、肺阴性。

\textbf{【X线表现】}
 上消化道钡餐造影示:胃窦小弯侧见一小不规则钡斑,表面凹凸不平,周围粘膜皱襞纠集。

\textbf{【X线诊断】}  胃窦小弯侧早期胃癌(Ⅱc型)。

\textbf{【评  述】}
 本例经手术病理证实为胃窦部小弯侧粘膜下癌,未突破粘膜肌层。胃癌是我国最常见的恶性肿瘤之一,好发年龄为40~60岁,可发生在胃的任何部位,但以胃窦、胃小弯及贲门区常见。目前,国内外均采用日本内镜学会提出的早期胃癌的定义及分型。早期胃癌是指癌限于粘膜及粘膜下层,而不论其大小或有无转移。依据肉眼形态分为三个基本型与三个亚型:Ⅰ型,隆起型,癌肿隆起高度大于5mm,呈息肉状。Ⅱ型,浅表型,癌灶比较平坦,不形成明显隆起或凹陷。本型根据其癌灶凹凸程度不同又分三个亚型:Ⅱa型,浅表隆起型,癌灶隆起高度小于5mm。Ⅱb型,浅表平坦型,与周围粘膜几乎同高,无隆起或凹陷。Ⅱc型,浅表凹陷型,癌灶凹陷深度小于5mm。Ⅲ型,凹陷型,癌灶深度大于5mm,形成溃疡,癌组织不超过粘膜下层。除上述三型外,尚有混合型。根据胃窦小弯侧不规则形表浅凹陷形成边缘粗糙的钡斑,其周围粘膜皱襞纠集呈杵状增粗和融合,拟诊断为早期胃癌,浅表凹陷型(Ⅱc型)。需要鉴别的是良性溃疡病变,其钡斑密度均匀、边缘光整,多呈圆形、椭圆形,溃疡周围粘膜皱襞纠集一般比较均匀规则,呈自远而近逐渐变细,与癌可形成鲜明的对照。

\subsection{早期胃癌(Ⅱa+Ⅱc型)}

\begin{figure}[!htbp]
 \centering
 \includegraphics{./images/Image00280.jpg}
 \captionsetup{justification=centering}
 \caption{胃体部早期胃癌(Ⅱa+Ⅱc型)}
 \label{fig5-3-18}
  \end{figure} 

\textbf{【病史摘要】}
 女性,45岁。上腹部疼痛不适伴嗳气、反酸5个月,近1个月疼痛加重,食欲减退。体格检查:腹软,上腹部压痛,未扪及包块,心、肺阴性。

\textbf{【X线表现】}
 上消化道钡餐造影示:胃体上部见隆起型小充盈缺损影,边缘欠光整,表面凹凸不平,见不规则钡斑影,周围粘膜皱襞中断破坏,胃小弯侧上段胃壁僵硬。

\textbf{【X线诊断】}  胃体部早期胃癌(Ⅱa+Ⅱc型)。

\textbf{【评  述】}
 本例经手术病理证实为胃体上部早期胃腺癌(Ⅱa+Ⅱc型),粘膜肌层未侵犯,周围淋巴结未见转移。早期胃癌除上述三个基本型及亚型外,病灶若具有两种形态者,称之为混合型,一般表述时将占优势的一型记录在前,如本例为Ⅱa+Ⅱc型,表示隆起型病灶的中央存在糜烂的深凹陷。

\subsection{胃癌(息肉型)}

\begin{figure}[!htbp]
 \centering
 \includegraphics{./images/Image00281.jpg}
 \captionsetup{justification=centering}
 \caption{胃窦部胃癌(息肉型)}
 \label{fig5-3-19}
  \end{figure} 

\textbf{【病史摘要】}
 男性,52岁。上腹部疼痛2年,无节律性,近期疼痛加重。体格检查:上腹部压痛,未扪及明显包块,腹软,肝、脾未及,心、肺阴性。

\textbf{【X线表现】}
 上消化道钡餐造影示:胃窦部近小弯侧见一充盈缺损,呈分叶状,轮廓欠光整,周围粘膜皱襞中断破坏,局部胃壁较僵硬,十二指肠水平段见囊袋状影。

\textbf{【X线诊断】}  胃窦部胃癌(息肉型);十二指肠水平段憩室。

\textbf{【评  述】}
 本例患者经手术病理证实为胃体近小弯侧胃腺癌,侵犯胃粘膜肌层。息肉型胃癌为常见病,好发于胃窦部,其次是胃底。早期癌肿突向胃腔,高约5mm,轮廓大多不规则,可广基底或呈狭蒂。中后期,癌肿进一步增大,表面高低不平如菜花样,与胃壁边界明确。临床多见于40岁以上男性,早期无症状,或类似溃疡病的症状。中后期,症状加剧,有中上腹痛,上消化道出血,扪及肿块,以及癌肿所在部位所产生的一些继发症状,如梗阻、呕吐等。

X线特点主要为:早期隆起型胃癌在适当加压或双重造影检查时,可见小的轮廓不规则的充盈缺损。至中晚期,一般钡餐检查,即可显示出轮廓不光整的充盈缺损,基底广、边界明确、直径3~4cm以上。缺损区邻近粘膜纹中断、破坏,胃壁僵硬。早期隆起型胃癌主要应与胃内良性肿瘤相鉴别,胃癌充盈缺损边缘不光整,粘膜破坏,胃壁僵硬。中晚期胃癌应与胃内其他恶性肿瘤相鉴别。早期隆起型胃癌如果病灶较小,常规钡餐检查容易漏诊,应注意适当加压方能显示出病灶。内镜检查有利于发现早期病灶,并能提供病理依据,便于明确诊断。

\subsection{贲门癌}

\begin{figure}[!htbp]
 \centering
 \includegraphics{./images/Image00282.jpg}
 \captionsetup{justification=centering}
 \caption{贲门癌}
 \label{fig5-3-20}
  \end{figure} 

\textbf{【病史摘要】}
 男性,56岁。上腹部疼痛、吞咽不适伴嗳气、反酸半年余,近1个月进食干性食物时吞咽困难加重伴呕吐。体格检查:腹软,上腹部轻度压痛,未扪及包块,心、肺阴性。

\textbf{【X线表现】}
 上消化道钡餐造影示:胃底贲门区见类圆形充盈缺损影,边缘毛糙,轻度分叶,周围粘膜皱襞破坏,胃体小弯侧上方胃壁僵硬。

\textbf{【X线诊断】}  贲门癌侵及胃体小弯上段(进展期)。

\textbf{【评  述】}
 本例患者经手术病理证实为胃底贲门腺癌,侵及胃体小弯侧上段。贲门癌为源于贲门中心周围2.0~2.5cm以内的胃癌。由于其位置比较特殊而易漏诊,主要原因为贲门区位于肋弓内不能触及肿块,贲门胃底部粘膜皱襞粗大使较小的病变难以识别。因此,检查贲门癌时应采用气钡双重造影,产气量越大越可形成良好的对比。一般先于立位观察,再采用仰卧、俯卧及左右斜位观察以免漏诊。

贲门癌的典型X线征象为:①贲门区肿物,可位于贲门开口上方或下方。②钡剂通过贲门时受阻,或在肿瘤之上绕过形成钡剂分流现象,有时呈喷射状入胃。③胃底增厚,呈多个弧形影,胃底与膈面距离加大(>1.5cm有诊断价值)。④贲门下方之胃小弯胃壁僵硬。⑤可合并龛影及出现环堤征。⑥食管下段受侵犯,出现狭窄、僵硬、变形等。

需要鉴别的是贲门失迟缓症,后者X线表现的食管下端狭窄对称、边缘光滑、壁柔软,管腔大小可变,腔内可见细而平行的粘膜皱襞,特别是无贲门癌胃泡内组织块影是鉴别要点。而发生于胃底的平滑肌瘤和平滑肌肉瘤,胃泡内也可见轮廓光整或分叶状软组织肿块影,但两者X线不仅有腔内的软组织肿块影,而且向胃腔外生长,还有胃壁改变,腔外较大肿块可推压邻近器官,再者平滑肌瘤与平滑肌肉瘤很少有侵犯食管下端,这些都与贲门癌不同。贲门区解剖结构特殊,发生于此的溃疡、静脉曲张及其他良恶性肿瘤X线表现与贲门癌有时差异也不显著,需密切结合临床病史,必要时做胃镜协助诊断。

\subsection{胃窦癌}

\begin{figure}[!htbp]
 \centering
 \includegraphics{./images/Image00283.jpg}
 \captionsetup{justification=centering}
 \caption{胃窦癌}
 \label{fig5-3-21}
  \end{figure} 

\textbf{【病史摘要】}
 女性,50岁。上腹部饱胀、疼痛2年,近2个月来疼痛加重,伴恶心、呕吐。体格检查:消瘦贫血貌,上腹部压痛并触及固定包块,心、肺阴性。

\textbf{【X线表现】}
 上消化道钡餐造影示:胃窦部狭窄,胃窦大弯侧见不规则充盈缺损,呈现肩胛征,胃窦部粘膜皱襞破坏紊乱,胃壁僵硬蠕动消失,胃内粘液潴留较多。

\textbf{【X线诊断】}  胃窦癌(进展期)。

\textbf{【评  述】}
 本例患者经手术病理证实为胃窦癌。胃窦部为胃癌另一好发部位,易发生浸润型胃癌,极易引起胃窦狭窄,狭窄的胃窦呈漏斗状或山峰状,出现肩胛征或袖口征,前者指狭窄的胃窦与其近端舒张的胃壁相连处呈肩胛状,后者则表现为狭窄近端随蠕动推进套在僵硬段上呈袖口状。此外,胃窦癌易于侵犯幽门而形成幽门梗阻,致胃排空延迟、胃残留物及滞留液增多,故必须做好检查前准备,清除和减少胃内滞留物。通常采用延长禁食时间和插胃管洗胃等方法,尚可使用辅助药物或针刺来改变胃窦张力和蠕动,以利于清晰显示狭窄段情况。胃窦癌须注意与胃窦炎或溃疡引起的良性狭窄相鉴别。鉴别的要点为良性狭窄病变段与正常胃分界呈渐进性,狭窄形态可变,可以收缩与扩张,粘膜皱襞存在、排列不整齐,其与胃窦癌形成的胃窦狭窄X线征象截然不同,鉴别不难。

\subsection{溃疡型胃癌}

\begin{figure}[!htbp]
 \centering
 \includegraphics{./images/Image00284.jpg}
 \captionsetup{justification=centering}
 \caption{胃小弯侧溃疡型胃癌}
 \label{fig5-3-22}
  \end{figure} 

\textbf{【病史摘要】}
 男性,65岁。上腹部疼痛不适2年,近1个月来疼痛加剧,消瘦明显。体格检查:消耗面容,上腹部可触及固定硬质包块,压痛明显,心、肺阴性。

\textbf{【X线表现】}
 上消化道钡餐造影示:胃角处胃腔内见一不规则龛影,龛影周围显示有不规则透亮环堤,其内可见指压迹征和裂隙征,局部胃壁僵硬,蠕动消失。

\textbf{【X线诊断】}  胃小弯侧溃疡型胃癌(进展期)。

\textbf{【评  述】}
 本例患者经手术后病理证实为胃小弯侧腺癌。溃疡型胃癌是进展期胃癌中的最多见类型,其X线表现以壁龛及邻近胃壁变化为主要表现。龛影多数较浅而大,形态多不规则,具有特征性的为口部指压迹征和裂隙征,与良性溃疡平坦的口部出现的狭颈征、项圈征对比分明。切线位显示龛影位于胃轮廓线以内或与之相平。龛影周围一圈不规则充盈缺损为环堤,环堤大小不一,高低不平,与正常胃壁界限清楚,其病理基础为癌肿破溃后留下的一圈隆起的边缘。若龛影骑跨于胃小弯前后壁,与周围的半弧形环堤构成了半月综合征。邻近粘膜皱襞亦可有聚拢表现,但近环堤处粘膜中断且有指压迹改变。

\subsection{浸润型胃癌}

\begin{figure}[!htbp]
 \centering
 \includegraphics{./images/Image00285.jpg}
 \captionsetup{justification=centering}
 \caption{浸润型胃癌}
 \label{fig5-3-23}
  \end{figure} 

\textbf{【病史摘要】}
 女性,48岁。上腹部疼痛2年,嗳气、反酸,近1个月来疼痛加剧,出现黑便,食欲减退。体格检查:消瘦,肝、脾未及,腹壁紧张,全腹压痛,心、肺阴性。

\textbf{【X线表现】}
 上消化道钡餐造影示:全胃胃腔缩小,胃壁僵硬,无蠕动,粘膜皱襞增粗、紊乱,呈脑回状改变,形态固定不变。

\textbf{【X线诊断】}  广泛浸润型胃癌(皮革胃)。

\textbf{【评  述】}
 本例患者经手术病理证实为高度恶性晚期进展期胃癌,X线表现呈皮革状胃,癌肿全胃广泛浸润。浸润型胃癌根据癌肿浸润范围不同,X线表现可分为局限浸润型和弥漫浸润型。局限型可发生于胃的任何部位,以胃窦部多见。X线表现为:癌肿浸润胃壁全周或半周时,胃腔显示为局限性、固定性狭窄与僵硬,严重时可呈管状、漏斗状狭窄。胃壁局限性僵硬、蠕动消失。胃粘膜皱襞增粗、紊乱,部分呈脑回状,形态固定不变。广泛型为癌肿浸润胃大部或全部,胃腔明显缩小,粘膜皱襞平坦、消失,胃壁僵硬、蠕动消失,犹如皮革囊状,称皮革胃。因幽门受侵而失去正常功能,由于钡剂的重力作用,可见造影时幽门处于开放状态,有造影剂源源不断地进入十二指肠。浸润型胃癌有时应和胃淋巴瘤相鉴别。胃淋巴瘤主要为粘膜下浸润性生长,肌层较少受浸润,且又无明显的纤维组织增生,故胃壁虽可增厚,但胃腔缩小一般不明显,胃壁僵硬也不显著,往往可见有蠕动波为鉴别之要点。

\subsection{残胃癌}

\begin{figure}[!htbp]
 \centering
 \includegraphics{./images/Image00286.jpg}
 \captionsetup{justification=centering}
 \caption{残胃癌}
 \label{fig5-3-24}
  \end{figure} 

\textbf{【病史摘要】}
 男性,65岁。5年前因溃疡型胃癌行胃大部分切除术、Billroth
Ⅰ式吻合术,近期上腹部疼痛加重伴嗳气、反酸,并有恶心、呕吐。体格检查:消瘦,皮肤、巩膜无黄染,左上腹部压痛明显,肝、脾无增大,心、肺阴性。

\textbf{【X线表现】}
 上消化道钡餐造影示:残胃吻合口下部小弯侧见不规则龛影,龛影周围显示有不规则透亮环堤,其内可见指压迹征,局部胃壁僵硬,蠕动消失。

\textbf{【X线诊断】}  残胃癌。

\textbf{【评  述】}
 本例患者经手术病理证实为残胃癌。残胃癌的诊断标准不一,国外多主张因良性疾患行胃部分切除术、胃肠吻合术后3年以上,残胃生癌者为残胃癌。我国主张良性胃疾患胃部分切除术后3年以上,胃癌行部分切除术后5年以上,残胃生癌者称残胃癌。国内外材料一致认为,胃空肠吻合术后残胃癌发生率高于胃十二指肠吻合术者。残胃癌病因不明,可能与碱性肠液刺激及术后引起吻合口慢性刺激有关。早期残胃癌临床症状不具特征性。中晚期残胃癌常见症状是中上腹疼痛、食欲减退和出血。残胃癌好发于胃残端部,其次为贲门区和大小弯前后壁交界部。因术后粘连及变形,残胃癌的X线诊断困难。常见的表现为吻合口狭窄、排空迟缓、残胃扩张;胃腔狭窄变形,胃壁僵直丧失舒缩功能;粘膜破坏;充盈缺损或不规则龛影等。残胃癌应与炎症性粘膜肿胀、缝线引起的异物反应或肉芽肿等鉴别,手术缝合时,可使小弯侧结节状改变也应注意,鉴别困难时,应结合纤维胃镜检查及组织学检查。

\section{十二指肠病变}

\subsection{十二指肠球部溃疡}

\begin{figure}[!htbp]
 \centering
 \includegraphics{./images/Image00287.jpg}
 \captionsetup{justification=centering}
 \caption{十二指肠球部溃疡}
 \label{fig5-4-1}
  \end{figure} 

\textbf{【病史摘要】}
 男性,55岁。上腹部节律性疼痛伴反酸2年,餐后疼痛缓解。体格检查:上腹部剑突下压痛,肝、脾无肿大,心、肺阴性。

\textbf{【X线表现】}
 上消化道钡餐造影示:十二指肠球基底部见一龛影,黄豆大小,边缘光整,周围粘膜纠集,球部变形。

\textbf{【X线诊断】}  十二指肠球部溃疡。

\textbf{【评  述】}
 本例经胃镜证实为十二指肠球部溃疡。十二指肠溃疡为常见病,其发生率高于胃溃疡。十二指肠溃疡好发于青壮年,40岁以下占80%。男性多于女性。十二指肠溃疡病因复杂,尚未完全阐明。十二指肠溃疡85%发生于球部,其次在球后部。发生于球部者,前壁较多,占50%,其次为后壁及球部的大小弯。单发为主,也可多发。临床主要征象为周期性、节律性右上腹痛,多在餐后3~4小时发生,进餐后可缓解。

X线表现主要有:①球部龛影,为球部溃疡的直接征象,正面观呈圆形或椭圆形,少数呈线状,需双重造影显示。充盈加压时,溃疡周围的水肿、增生表现为外缘模糊的透光带。切线位上龛影突出于轮廓线以外,呈锥形或乳头状,以充盈像显示较好。②球部变形,多由溃疡所致,少数可因胆系或胰腺等邻近脏器疾病所致,因此发现球部变形时需除外其他原因后方可诊断溃疡。球部呈二叶状、山字形、花瓣状畸形为瘢痕收缩的结果,球部的大、小弯侧可见袋状突出,称假性憩室,也因瘢痕收缩所致。球部整体性痉挛及严重的瘢痕收缩皆可致明显缩窄,此时常伴幽门梗阻,平滑肌松弛剂的应用有助于痉挛与瘢痕挛缩的鉴别。③激惹征,表现为钡剂迅速经过球部而不能满意充盈,为炎症刺激所致。

\subsection{十二指肠复合性溃疡}

\begin{figure}[!htbp]
 \centering
 \includegraphics{./images/Image00288.jpg}
 \captionsetup{justification=centering}
 \caption{十二指肠复合性溃疡}
 \label{fig5-4-2}
  \end{figure} 

\textbf{【病史摘要】}
 男性,35岁。上腹部不适、嗳气、反酸1年,近日夜间疼痛加重,饥饿时加重,进食后缓解。体格检查:上腹部剑突下压痛,肝、脾无增大,心、肺阴性。

\textbf{【X线表现】}
 上消化道钡餐造影示:十二指肠球部及球后部见大小不等龛影,球部变形呈二叶形,十二指肠上曲狭窄,钡剂下行受阻。

\textbf{【X线诊断】}  十二指肠复合性溃疡。

\textbf{【评  述】}
 本例患者经胃镜检查证实为十二指肠复合性溃疡,即十二指肠球部及球后部溃疡。十二指肠球后部是指球部与降部之间的肠管。该部溃疡以龛影为主,可合并局限性偏心性狭窄,十二指肠激惹征较为明显,局部压痛可同时存在。由于球部的重叠,十二指肠球后部溃疡较难显示。检查的方法包括以下三方面:①常规钡餐,利用各种体位及结合加压来显示病变,可以较好地了解球部充盈及排空情况以及十二指肠蠕动状况。②低张气钡造影,能更清晰地显示细微结构及病变情况。③内镜应用,钡餐结合内镜所见来提高诊断率已经成为一种必不可少的手段。

\subsection{肠系膜上动脉压迫综合征}

\begin{figure}[!htbp]
 \centering
 \includegraphics{./images/Image00289.jpg}
 \captionsetup{justification=centering}
 \caption{肠系膜上动脉压迫综合征}
 \label{fig5-4-3}
  \end{figure} 

\textbf{【病史摘要】}
 女性,35岁。进食后上腹部饱胀恶心、呕吐,俯卧位后症状缓解。体格检查:瘦长体形,肝、脾未及,腹部无明显压痛,心、肺阴性。

\textbf{【X线表现】}
 上消化道钡餐造影示:无力型胃,胃角位于髂嵴连线下方2.5cm左右,蠕动缓慢。十二指肠水平段钡剂受阻,水平段以上肠管扩张,蠕动亢进,并见逆蠕动发生,受阻处十二指肠见管状压迹。

\textbf{【X线诊断】}  肠系膜上动脉压迫综合征;胃下垂。

\textbf{【评  述】}
 本例患者经腹部CT扫描及CTA检查,示肠系膜上动脉自腹主动脉分出后,夹角过小,压迫十二指肠水平段,引起十二指肠郁积,故确诊为肠系膜上动脉压迫综合征。肠系膜上动脉压迫综合征多见于中年体弱和瘦长体形者,女性多于男性。其主要原因为肠系膜上动脉根部紧张度增强或先天性原因使肠系膜上动脉与腹主动脉间夹角变小,引起十二指肠水平段受压,使受压部以上肠管扩张而出现郁积。临床上主要表现为食后上腹部饱胀、恶心、呕吐,且呕吐物中带有胆汁,俯卧位时症状缓解或消失。X线表现主要是立位检查时钡剂通过十二指肠水平段受阻,十二指肠降段以上肠管扩张,蠕动亢进,可见钡剂如钟摆样来回运动,水平段受压处有一光滑整齐的纵形压迹,称为笔杆状压迹,使肠管紧贴脊柱,粘膜变平,当患者俯卧位时,该压迹可消失。诊断本病时应谨慎,因正常瘦长体形的人也可出现十二指肠水平段钡剂暂时性停留和少量逆蠕动,但无肠管扩张及胃排空延迟。此外,还需与器质性病变所致的梗阻相鉴别,若梗阻端形态显示良好,鉴别应无困难。

\subsection{十二指肠憩室}

\begin{figure}[!htbp]
 \centering
 \includegraphics{./images/Image00290.jpg}
 \captionsetup{justification=centering}
 \caption{十二指肠降部憩室}
 \label{fig5-4-4}
  \end{figure} 

\textbf{【病史摘要】}
 男性,45岁。上腹部疼痛不适月余。体格检查:上腹部剑突下压痛,肝、脾无增大,心、肺阴性。

\textbf{【X线表现】}
 上消化道钡餐造影示:胃窦部粘膜皱襞增粗、紊乱,胃壁柔软,十二指肠降部见一囊袋状影突出于肠壁,内见钡剂充盈,可见十二指肠粘膜纹理伸入其中。

\textbf{【X线诊断】}  胃窦炎;十二指肠降部憩室。

\textbf{【评  述】}
 本病比较常见,大多数患者无明显症状,多见于中老年人。发生部位多位于降段内后壁,其次为十二指肠水平段。若合并憩室炎可引起糜烂、溃疡和出血,壶腹部附近憩室尚可引起胆管炎或胰腺炎等。十二指肠憩室发生的原因可能与肠壁生长发育过程中的局部缺陷与薄弱有关,随着年龄增长而加剧退变,在肠内压异常增加或肠肌收缩不协调时,薄弱点向腔外突出而形成憩室。十二指肠憩室X线表现主要是充钡后憩室呈圆形、椭圆形或三角形囊袋状突出物,轮廓光整,颈部较狭窄,并可见十二指肠粘膜皱襞伸入其中。憩室大小不一,较大者立位可见囊内气、液、钡分层现象,较小者可呈短管状,一般钡透不易发现,需行低张气钡双重造影才不至于漏诊。憩室轮廓不规则、压痛、邻近十二指肠有肠激惹征象者应考虑合并憩室炎。此外,憩室尚需与溃疡鉴别,后者常伴有狭窄痉挛,龛影内无粘膜皱襞伸入。

\subsection{十二指肠腺瘤}

\begin{figure}[!htbp]
 \centering
 \includegraphics{./images/Image00291.jpg}
 \captionsetup{justification=centering}
 \caption{十二指肠腺瘤}
 \label{fig5-4-5}
  \end{figure} 

\textbf{【病史摘要】}
 男性,45岁。上腹部疼痛不适,有嗳气、反酸。体格检查:上腹部剑突下压痛,肝、脾无增大,心、肺阴性。

\textbf{【X线表现】}
 上消化道钡餐造影示:十二指肠降部下段外侧可见分叶状充盈缺损(箭头),其基底部与肠壁形成切迹,肠壁略凹陷,周围粘膜皱襞正常,未见明显中断、破坏。

\textbf{【X线诊断】}  十二指肠降部腺瘤。

\textbf{【评  述】}
 本例经手术病理证实为十二指肠降部腺瘤。十二指肠良性肿瘤约占小肠良性肿瘤的20%。以腺瘤、平滑肌瘤、脂肪瘤多见。发生部位以球部最多,占50%以上,降部次之,升部最少。肿瘤多向肠腔内呈息肉状生长,少数向肠腔外生长。临床上多见于老年人,因肿瘤多较小而少有症状。食欲不振、恶心、上腹部疼痛及出血为常见症状。十二指肠腺瘤与消化道其他部位的腺瘤相似,X线表现为圆形、椭圆形或分叶状充盈缺损,边缘光滑,局部肠壁柔软,粘膜皱襞无破坏,一般以单发为多见,少数可多发,可带蒂,此时可见肿瘤随肠蠕动而移动。腺瘤多发时要与布氏腺增生鉴别。布氏腺增生比较罕见,多发生在球部,亦可延及降部。病因不明,通常认为是一种炎症。病理上有多发型和单发型两种,前者为广泛结节状粘膜增生,后者与单发腺瘤相似,可带蒂。X线表现为十二指肠球部粘膜紊乱,皱襞增粗,其中可见多枚黄豆或绿豆大小的充盈缺损,形态固定。单发者为单个充盈缺损,与腺瘤无法鉴别,通常十二指肠没有激惹和变形。

一般临床工作中,如发现十二指肠单发带蒂肿瘤应首先考虑腺瘤的诊断,而十二指肠多发结节状充盈缺损则应首先考虑布氏腺增生的诊断。最终诊断要结合内镜或手术病理诊断。

\subsection{十二指肠平滑肌瘤}

\begin{figure}[!htbp]
 \centering
 \includegraphics{./images/Image00292.jpg}
 \captionsetup{justification=centering}
 \caption{十二指肠平滑肌瘤}
 \label{fig5-4-6}
  \end{figure} 

\textbf{【病史摘要】}
 男性,35岁。上腹部疼痛不适月余,无嗳气、反酸,无恶心、呕吐。体格检查:上腹部无压痛,肝、脾无增大,心、肺阴性。

\textbf{【X线表现】}
 上消化道钡餐造影示:十二指肠下曲见类圆形充盈缺损影,边缘光整,内见小钡斑影,十二指肠腔未见狭窄,肠壁未见僵硬,蠕动正常。

\textbf{【X线诊断】}  十二指肠平滑肌瘤。

\textbf{【评  述】}
 本例患者经手术治疗病理证实为十二指肠平滑肌瘤。本例患者十二指肠病变X线征象符合良性肿瘤的表现,发生于十二指肠的良性肿瘤较少见,以平滑肌瘤及腺瘤多见。平滑肌瘤来源于中胚层组织。

X线表现主要为:①小肠局限性肿物,瘤体一般<5cm。②肿物呈球形或分叶状,周界规则,切线位上呈半圆形充盈缺损。③向腔内生长者,肿瘤一般体积都较大,无蒂,较固定,活动度差,局部管腔狭窄,可致肠梗阻;腔外生长者多无临床症状;如同时向腔内及腔外生长,尚可见肠管受压甚至移位。④瘤体中心因血供缺乏,往往容易发生坏死,出现龛影或表面糜烂,X线表现为钡斑。需要指出的是,平滑肌瘤和平滑肌肉瘤皆为粘膜下肿瘤,均具有粘膜下肿瘤的特征,故两者X线表现有时很相似,鉴别有一定难度。但平滑肌肉瘤瘤体体积常大于5cm,形态不规则,表面常凹凸不平,并且常较早出现肝脏、淋巴结转移,这有利于两者的鉴别诊断。

\subsection{十二指肠腺癌}

\begin{figure}[!htbp]
 \centering
 \includegraphics{./images/Image00293.jpg}
 \captionsetup{justification=centering}
 \caption{十二指肠降部腺癌}
 \label{fig5-4-7}
  \end{figure} 

\textbf{【病史摘要】}
 男性,65岁。上腹部疼痛不适伴呕吐3个月余,近期有黑便。体格检查:腹软,中上腹部压痛,肠肠鸣音正常,肝、脾未及,心、肺阴性。

\textbf{【X线表现】}
 上消化道钡餐造影示:十二指肠降部管腔明显环形狭窄,粘膜破坏,管壁僵硬,蠕动消失,近端肠管扩张。

\textbf{【X线诊断】}  十二指肠降部腺癌。

\textbf{【评  述】}
 本例经手术治疗病理证实为十二指肠降部粘液腺癌。十二指肠腺癌占小肠腺癌的40%~50%,好发于60~70岁,男女之比约为1.2∶1。按癌瘤发生的部位可分为乳头上部癌、乳头周围癌和乳头下部癌,其中以乳头周围癌最多见,约占65%,乳头上部癌约占20%,乳头下部癌约占15%。按肿瘤的大体形态可分为息肉型、浸润溃疡型、缩窄型和弥漫型。临床表现与肿瘤的类型及部位有关。主要症状有:上腹部隐痛、烧灼样痛或钝痛:酷似十二指肠溃疡,但进食及制酸药均不能缓解疼痛。黄疸:乳头周围癌75%~80%可发生黄疸。肠梗阻:息肉型或缩窄型癌容易导致肠腔狭窄或堵塞,导致部分或完全性十二指肠梗阻;乳头上部癌导致的完全性肠梗阻,呕吐物内不含胆汁,易被误诊为幽门梗阻。出血:十二指肠癌患者的大便隐血试验阳性者占60%~80%,出血明显者可有黑便,大出血时可发生呕血。腹块:右上腹出现肿块者占10%~25%。

根据肿瘤的X线表现可分为息肉型、溃疡型及浸润型:①息肉型:表现为息肉样隆起病变,形态不规则呈分叶状,粘膜破坏消失。肠腔可呈扩张状,钡剂分流,如果肿块较大可填塞十二指肠,钡剂受阻,近端肠腔扩张。同时也可伴有溃疡,肠壁僵硬等。②溃疡型:表现为粘膜破坏,出现不规则的腔内龛影,或部分腔内部分位于腔外。溃疡口部可有环堤、裂隙征及指压痕等恶性溃疡的征象。同时也可伴有局部肠壁僵硬,出现不规则的隆起性改变。③浸润型:X线表现为肠壁受到肿瘤浸润而僵硬,蠕动消失,肠腔狭窄,近端肠腔扩张,粘膜破坏,可伴有溃疡及不规则隆起性病变。本例患者为发生在十二指肠乳头上部的浸润型腺癌,X线表现比较明确,手术病理予以证实。但十二指肠癌如发生在乳头区则需与胰头癌相鉴别。十二指肠癌可推移相对正常的胰头或钓突结构向前内侧移位,肿块密度不均匀伴溃疡形成,十二指肠内外侧壁都呈不规则增厚和肠腔狭窄等有助于与胰头癌鉴别。但当肿瘤侵犯胰头时,两者的鉴别极为困难。

\subsection{十二指肠平滑肌肉瘤}

\begin{figure}[!htbp]
 \centering
 \includegraphics{./images/Image00294.jpg}
 \captionsetup{justification=centering}
 \caption{十二指肠平滑肌肉瘤}
 \label{fig5-4-8}
  \end{figure} 

\textbf{【病史摘要】}
 男性,55岁。上腹部疼痛半年余,无嗳气、反酸,无明显节律性。近期疼痛突然加剧,伴恶心、呕吐。体格检查:腹部拒按,上腹部触及包块,肝、脾未及,心、肺阴性。

\textbf{【X线表现】}
 上消化道钡餐造影示:十二指肠肠曲扩大,上曲内缘呈弧形压迹,肠腔伴有狭窄,其边缘皱襞有不规则破坏,并可见一不规则线状钡影呈水平状伸向十二指肠肠曲内,其上方见一小憩室。

\textbf{【X线诊断】}  十二指肠降部平滑肌肉瘤,肿瘤液化坏死与肠腔相通。

\textbf{【评  述】}
 本例患者经手术治疗病理证实为十二指肠降部平滑肌肉瘤,肿瘤液化坏死与肠腔相同。平滑肌肉瘤发生于十二指肠较少见。其病理变化与肿瘤生长方式有关,如肿瘤向肠腔内生长,呈半球状突入肠腔,可略有分叶,广基底,粘膜面糜烂或呈不规则溃疡,如肿瘤向肠腔外生长,则压迫十二指肠移位。临床可扪及腹块,质硬;可伴上消化道出血。

X线表现主要有:①腔内充盈缺损,略带有分叶改变,局部粘膜纹消失,可伴有不规则龛影,肠腔扩张,钡流改道。②肠腔被压迫移位,导致十二指肠肠曲变形,其形态改变依据肿瘤的部位和大小而定。上述两方面变化,有时混合存在。平滑肌瘤和平滑肌肉瘤皆为粘膜下肿瘤,均具有粘膜下肿瘤的特征,故两者X线表现有时很相似,鉴别有一定难度。但本例患者瘤体体积较大,形态不规则,表面凹凸不平,肿瘤液化坏死,并出现与肠腔相通的窦道,肠壁较僵硬,局部蠕动消失,与平滑肌瘤的表现相异,故诊断为平滑肌肉瘤。

\subsection{十二指肠淋巴瘤}

\begin{figure}[!htbp]
 \centering
 \includegraphics{./images/Image00295.jpg}
 \captionsetup{justification=centering}
 \caption{十二指肠淋巴瘤}
 \label{fig5-4-9}
  \end{figure} 

\textbf{【病史摘要】}
 男性,40岁。上腹部疼痛不适伴低热,无嗳气、反酸,偶有呕吐。体格检查:上中腹部压痛,中腹部似触及包块,肝、脾未及,心、肺阴性。

\textbf{【X线表现】}
 上消化道钡餐造影示:十二指肠降部中上段(箭头)肠腔狭窄,见不规则充盈缺损及小龛影,粘膜中断,肠壁略显僵硬。

\textbf{【X线诊断】}  十二指肠降部淋巴瘤。

\textbf{【评  述】}
 本例患者经内镜检查并经病理证实为十二指肠降部非霍奇金淋巴瘤。原发性十二指肠恶性淋巴瘤(primary
malignant lymphoma of
duodenum),是指原发于十二指肠肠壁淋巴组织的恶性肿瘤,原发性十二指肠恶性淋巴瘤好发于40岁左右,较其他恶性肿瘤发病年龄轻,男女发病之比为1:1~3:1。该病的临床表现无特异性,可因肿瘤的类型和部位而异,主要表现为上腹痛、腹块、弛张热等。病理巨检表现有浸润型、息肉型、溃疡型,可混合存在。

X线平片检查有时可显示十二指肠梗阻的X线表现,或软组织块影。胃肠道钡餐双重对比造影对十二指肠肿瘤的诊断准确率达42%~75%,其影像表现有:①十二指肠粘膜皱襞变形、破坏、消失,肠壁稍僵硬。②肠壁充盈缺损、龛影或环状狭窄。③肠管可有局限性囊样扩张,呈动脉瘤样改变。④肠壁增厚,肠管变小,呈多发性结节状狭窄。十二指肠低张造影,更有利于观察粘膜皱襞的细微改变,使其诊断准确率提高到93%左右。肠穿孔是本病的主要并发症,有15%~20%的十二指肠恶性淋巴瘤患者会发生肠穿孔,比其他恶性肿瘤发生率高。此多为肿瘤侵犯肠壁发生溃疡、肠坏死,或肿瘤继发感染而引致。本例患者十二指肠降部肠腔狭窄,肠壁见充盈缺损,内见小龛影,周围粘膜皱襞破坏,肠壁略显僵硬,故符合十二指肠淋巴瘤的诊断,最终需经内镜检查或手术行活检以获病理确诊。

\subsection{十二指肠类癌}

\begin{figure}[!htbp]
 \centering
 \includegraphics{./images/Image00296.jpg}
 \captionsetup{justification=centering}
 \caption{十二指肠类癌}
 \label{fig5-4-10}
  \end{figure} 

\textbf{【病史摘要】}
 男性,39岁。上腹部疼痛不适3个月余,无恶心、呕吐。体格检查:腹软,中腹部压痛,未触及包块,肝、脾未及,心、肺阴性。

\textbf{【X线表现】}
 上消化道钡餐造影示:十二指肠降部及水平部见不规则充盈缺损,局部粘膜破坏,肠壁稍僵硬,肠腔稍窄。

\textbf{【X线诊断】}  十二指肠占位:淋巴瘤?腺癌?类癌?

\textbf{【评  述】}
 本例患者经手术病理证实为十二指肠降部恶性神经内分泌癌,即类癌。十二指肠类癌是很特殊的一种类癌,好发部位依次为十二指肠第二段、第一段、第三段。年龄22~84岁,平均55岁。男女发病率差别不大。常合并Von
Recklinghausen's病、Zollinger-Ellison综合征和多发性内分泌肿瘤(MEN)。十二指肠和壶腹部还可发生杯状细胞类癌(腺类癌)和小细胞神经内分泌癌。杯状细胞类癌又称腺类癌或粘液类癌。主要病理改变为息肉状病变,大小不等,单发或多发,小的表现为粘膜下结节,大的则明显突向腔内,以致肠腔阻塞。局部常伴有腔外肿块。可合并有其他小肠、结肠或肺、气管类癌。临床症状无特征性,可扪及腹块。

类癌瘤较小时常被漏诊,发展到一定大小后,X线即可表现为:①病变范围内大小不等的结节状透亮区,或较大的充盈缺损,缺损区伴有肠腔外肿块,为本病重要表现之一。②病变区粘膜纹粗大,可伴有肠腔狭窄。③病变段肠曲固定,移动度消失。④病变可多发,同时可见于胃肠道的其他部分,甚至肺、支气管,亦有类癌瘤存在。类癌的鉴别诊断极为困难,因其表现多样化,仅凭影像学表现很难与其他肠道的良恶性肿瘤鉴别。以往主要依赖消化道钡剂造影,病变检出的阳性率较低。但CT广泛使用后,尤其是多层螺旋CT的使用大大提高了肿瘤的发现比例。CT不但可发现原发病灶,还可显示肿瘤对邻近组织的侵犯情况,观察肝脏的转移灶,肠系膜的侵犯,后腹膜及邻近淋巴结的转移。

\section{小肠病变}

\subsection{空肠憩室}

\begin{figure}[!htbp]
 \centering
 \includegraphics{./images/Image00297.jpg}
 \captionsetup{justification=centering}
 \caption{空肠憩室}
 \label{fig5-5-1}
  \end{figure} 

\textbf{【病史摘要】}
 男性,45岁。上腹部疼痛半年余伴腹胀。体格检查:腹软,腹部无明显压痛,未扪及包块,肝、脾未及,心、肺阴性。

\textbf{【X线表现】}
 全消化道钡餐造影示:空肠见一卵圆形袋状阴影,边缘整齐光滑,以宽窄不等的开口通向肠腔,内见钡剂进出。

\textbf{【X线诊断】}  空肠憩室。

\textbf{【评  述】}
 憩室是由于钡剂经过胃肠道管壁的薄弱区向外膨出形成的囊袋状影像,或是由于管腔外邻近组织病变的粘连、牵拉造成管壁全层向外突出的囊袋状影像,其内及附近的粘膜皱襞形态正常,称之为憩室。小肠憩室好发于上段空肠,少数在回肠。正常空肠上段的终末血管粗大,肠系膜缘血管进入处的肠壁结构较薄弱,容易成为憩室的好发部位。憩室可为单发,多为多发性,多个憩室集中于某段空肠。多发性憩室数目由2~40个不等;直径由数毫米到数厘米。憩室均沿小肠系膜侧肠壁终末血管区分布,形状呈圆形或卵圆形的袋状结构向肠壁外膨出,并以宽径或窄径基底部向肠腔开口。

小肠气钡双重造影检查憩室的X线表现主要有:显影的憩室在小肠系膜侧呈圆形或卵圆形袋状阴影,边缘整齐光滑,以宽窄不等的开口通向肠腔。较大的憩室腔内可显示气体、液体和钡剂的3层平面,如遇开口宽大的憩室可见造影剂在憩室和肠腔之间自由进出,此为本症特有的X线造影表现。小肠憩室发生憩室粘膜出血、憩室穿孔、气腹和小肠壁气囊肿或肠梗阻时,应与消化性溃疡出血及穿孔、机械性肠梗阻等相鉴别。

\subsection{小肠蛔虫症}

\begin{figure}[!htbp]
 \centering
 \includegraphics{./images/Image00298.jpg}
 \captionsetup{justification=centering}
 \caption{小肠蛔虫症}
 \label{fig5-5-2}
  \end{figure} 

\textbf{【病史摘要】}
 男性,35岁。中腹部疼痛不适1周。体格检查:腹软,腹部无压痛,肝、脾未及,心、肺阴性。

\textbf{【X线表现】}
 全消化道钡餐检查示:回肠内可见边缘光滑之细长条状弯曲的充盈缺损影,中央可见细线状钡影,周围粘膜皱襞正常。

\textbf{【X线诊断】}  小肠蛔虫症。

\textbf{【评  述】}
 本病相对少见。根据小肠肠腔内边缘光滑之细长条状弯曲的充盈缺损影,特别是其中央可见与充盈缺损纵轴相一致的细线状钡影,为钡剂进入虫体腔内所致,小肠蛔虫症的诊断可以确定。而小肠腔内的各类占位性及其他病变均不能表现出以上的X线形态特征。需要注意的是小肠蛔虫的X线检查应仔细,常常需要加压观察,尤其是位置隐蔽、虫体较小时容易漏诊。

\subsection{小肠Crohn病}

\begin{figure}[!htbp]
 \centering
 \includegraphics{./images/Image00299.jpg}
 \captionsetup{justification=centering}
 \caption{小肠Crohn病}
 \label{fig5-5-3}
  \end{figure} 

\textbf{【病史摘要】}
 男性,35岁。下腹部疼痛伴腹泻1年余,时有发热,近1个月疼痛加剧,食欲减退。体格检查:右下腹部压痛,未扪及包块,肝、脾未及,心、肺阴性。

\textbf{【X线表现】}
 全消化道钡餐造影示:回肠末端边缘不整,管壁略僵硬,边缘呈锯齿状改变,粘膜紊乱,内见卵石样或息肉样充盈缺损影。

\textbf{【X线诊断】}  小肠Crohn病。

\textbf{【评  述】}
 小肠Crohn病,又称克罗恩病、局限性肠炎、肉芽肿性肠炎。1932年由Crohn和Oppenheimer最早描述。病因不明。发病年龄呈双峰特征:15~30岁和55~80岁高发,女性比男性发病率高20%~30%。临床症状多样化,如腹痛、腹泻、便秘、肠梗阻、便血、低热、消瘦、贫血、胃肠外症状等。本病从口至肛门的全胃肠道的任何部位均可受累,病变呈跳跃式或节段性分布。小肠和结肠同时受累最为常见,占40%~60%;限于小肠,主要是末端回肠发病的占30%~40%。病理改变主要为:特征性肠系膜侧纵行线状溃疡;在纵横交错的溃疡之间出现粘膜隆起,形成卵石征;纤维化致肠壁增厚,肠腔狭窄;瘘管形成;周围淋巴结肿大。

X线表现主要为:早期为肠粘膜纹理增粗,甚至有卵石样充盈缺损,或锯齿状或尖刺状龛影,病变段肠管形态固定,蠕动不明显,肠间距增宽。后期则有不规则的线样狭窄,范围不一,多为1~2cm或更长,间断发病,可合并肠粘连或肠梗阻表现。

此病主要与小肠结核鉴别,两者的X线表现非常相似,有时区别十分困难。肠结核常伴有回盲瓣病变,因结核病变使回盲瓣变形、开放,造影剂自由通过,而Crohn病使回盲部形成狭窄,可助鉴别。

\subsection{小肠结核}

\begin{figure}[!htbp]
 \centering
 \includegraphics{./images/Image00300.jpg}
 \captionsetup{justification=centering}
 \caption{小肠结核}
 \label{fig5-5-4}
  \end{figure} 

\textbf{【病史摘要】}
 男性,45岁。右下腹疼痛、恶心、呕吐伴食欲减退半年余,近半个月出现腹泻伴发热。2年前有肺结核病史,经治疗呼吸道症状消失。体格检查:右下腹压痛,腹肌紧张,未扪及包块,肝、脾未及,心、肺阴性。

\textbf{【X线表现】}
 全消化道钡餐造影示:末端回肠狭窄伴瘘管形成,盲肠狭窄,盲肠及回肠末端上移靠拢形成一字征。

\textbf{【X线诊断】}  回盲部肠结核(溃疡型)。

\textbf{【评  述】}
 肠结核好发于回盲部,但也见于十二指肠、空肠和回肠。肠结核分为溃疡型和增殖型两型,溃疡型多见,也见两型同时存在。早期是肠壁集合淋巴结与Peyer氏淋巴丛肿胀,以后融合成干酪性病灶、粘膜破溃,形成与长轴垂直的溃疡;病变严重者,愈合后形成大量瘢痕组织引起肠腔环形狭窄。也有些病例在结核初期,就有肠壁粘膜下层的结核性肉芽组织增生与纤维化,从而粘膜面产生许多大小不一的隆起性结节,肠壁变硬,早期就有肠腔狭窄。本病常见的症状为腹痛、腹泻,或腹泻、便秘交替出现。右下腹块与不全性梗阻症状与体征。

X线特点主要有:①早期表现为受累肠曲有激惹现象,回肠末端可以始终不充盈,或呈细线状。②溃疡形成时可见肠管边缘呈锯齿状,或呈斑点状龛影。③增生显著者,则表现为回盲部粘膜增粗,犹如多发性、大小不一的息肉样充盈缺损,甚至类似于肿瘤样表现。④愈合后常遗有环形肠腔狭窄与狭窄上肠曲扩张。

本例患者回盲部X线表现结合患者有肺结核病史,溃疡型肠结核诊断明确。肠结核需与肿瘤、克罗恩病鉴别,增生型肠结核的病变多为移行性,多发性小息肉样充盈缺损,粘膜增粗、紊乱,激惹征,回盲瓣受累机会高,而与肿瘤不同。肠结核与克罗恩病鉴别困难,而克罗恩病常见的纵行溃疡以及对侧假性憩室样囊袋状膨出和周围卵石样充盈缺损、偶见瘘管形成与溃疡型肠结核表现不同。

\subsection{小肠腺瘤}

\begin{figure}[!htbp]
 \centering
 \includegraphics{./images/Image00301.jpg}
 \captionsetup{justification=centering}
 \caption{小肠腺瘤}
 \label{fig5-5-5}
  \end{figure} 

\textbf{【病史摘要】}
 男性,45岁。上腹部不适3个月,无嗳气、反酸,无恶心、呕吐。体格检查:腹软,无压痛,肝、脾未及,心、肺阴性。

\textbf{【X线表现】}
 全消化道钡餐造影示:空肠内见一椭圆形充盈缺损影,边缘光整,基底部见带蒂,周围粘膜皱襞未见异常,肠蠕动正常。

\textbf{【X线诊断】}  空肠占位,考虑腺瘤可能性大。

\textbf{【评  述】}
 本例患者经手术治疗病理证实为空肠息肉状腺瘤。小肠腺瘤是发生于小肠粘膜上皮或肠腺体上皮的良性肿瘤,体积小、带蒂,呈息肉样生长,故又称肠息肉。小肠腺瘤多发生于十二指肠和回肠,空肠较少。一般来自肠粘膜上皮或腺上皮,多向肠腔内突出,表面覆盖粘膜和粘膜下组织。

根据组织学结构小肠腺瘤有3种类型:①管状腺瘤,亦称腺瘤样息肉或息肉状腺瘤,以发生于十二指肠最多,多是单发,也可多发,此种腺瘤呈息肉状,大多有蒂。②绒毛状腺瘤,亦称乳头状腺瘤。较管状腺瘤少见,最多发生于十二指肠内,体积较管状腺瘤大。③混合性腺瘤。小肠容受性好,内容物常为液体,而且腺瘤一般生长较慢,故小肠腺瘤可在较长时间内无症状。小肠腺瘤X线表现多为腔内的圆形充盈缺损,大小不一,轮廓光整、边缘光滑,如有带蒂,则可以移动。扪之柔软,易变形。本例患者钡餐X线表现为空肠内椭圆形充盈缺损,边缘光整,基底部见带蒂,周围粘膜皱襞未见异常,肠蠕动正常,故考虑小肠腺瘤可能性大。本病需与增生型肠结核及小肠癌鉴别。增生型肠结核X线钡剂检查表现为回盲部粘膜增粗,犹如多发性、大小不一的息肉样充盈缺损,盲肠收缩上移,回肠末端与其靠拢形成的一字征为其特点,小肠腺瘤不具此征;小肠癌好发于十二指肠、空肠与回肠下段,多呈环形生长,X线表现可显示局限性的不规则环形狭窄及狭窄前扩张,局部粘膜纹理破坏与不规则的结节样充盈缺损,很少见有龛影,局部肠壁僵硬,可扪及肿块,此与小肠腺瘤X线表现不同。

\subsection{小肠平滑肌瘤}

\begin{figure}[!htbp]
 \centering
 \includegraphics{./images/Image00302.jpg}
 \captionsetup{justification=centering}
 \caption{空肠平滑肌瘤}
 \label{fig5-5-6}
  \end{figure} 

\textbf{【病史摘要】}
 女性,35岁。因右下腹痛伴腹胀半年余,时有恶心、呕吐。体格检查:右下腹可扪及鸡蛋大小包块,可移动,肝、脾未及,心、肺阴性。

\textbf{【X线表现】}
 全消化道钡餐造影示:空肠内见一分叶状软组织肿块,局部肠壁凹陷。空肠肠襻折曲成角,下方见光滑弧形压迹,粘膜皱襞未见异常。

\textbf{【X线诊断】}  空肠占位,空肠平滑肌瘤可能性大,小肠腺癌待排。

\textbf{【评  述】}
 本例患者经手术治疗病理证实为空肠平滑肌瘤。小肠平滑肌瘤是最常见的小肠良性肿瘤,源自小肠固有肌层,少数来自粘膜肌层,为一肠壁间肿瘤,在小肠良性肿瘤中其发病率居第二位,仅次于腺瘤。在小肠各段的分布中以空肠为最多,回肠次之,肿瘤多为单发,大小不一,常为圆形或椭圆形,有时呈分叶状或结节状。根据肿瘤在肠壁间的部位及其生长方式,可分为四种类型:腔内型、壁内型、腔外型、腔内外型。主要临床表现为消化道出血、腹痛、腹块、肠梗阻,及并发内瘘。

X线表现主要为:①边界清楚的圆形或结节样肿块。②脐样或牛眼样龛影。③肠管3字征。④粘膜部分消失、部分呈弧形或横形展开。⑤局部钡剂不同程度受阻;局部肠腔狭窄;肠管或周围器官受压移位;近端肠腔不同程度扩张。

小肠平滑肌瘤需与小肠腺癌鉴别。小肠腺癌X线表现为肠腔内不规则的分叶状或菜花状充盈缺损伴溃疡形成,周围粘膜中断、破坏,这些都与小肠平滑肌瘤X线表现有区别。本例患者空肠占位基底部较宽,虽肿块呈分叶状,但未见溃疡龛影,周围粘膜皱襞未见中断破坏,故首先考虑空肠平滑肌瘤可能性大。

\subsection{小肠淋巴瘤}

\begin{figure}[!htbp]
 \centering
 \includegraphics{./images/Image00303.jpg}
 \captionsetup{justification=centering}
 \caption{小肠淋巴瘤}
 \label{fig5-5-7}
  \end{figure} 

\textbf{【病史摘要】}
 女性,35岁。右下腹疼痛伴弛张热6个月余,近期恶心、呕吐、腹胀。体格检查:右下腹压痛,未触及包块,肝、脾未及,心、肺阴性。

\textbf{【X线表现】}
 全消化道钡餐造影示:局部回肠肠腔瘤样扩张,边缘凹凸不平,肠腔内见较大不规则充盈缺损,周围粘膜皱襞破坏、消失,肠壁稍僵硬。

\textbf{【X线诊断】}  回肠占位,回肠淋巴瘤可能性大,回肠腺癌待排。

\textbf{【评  述】}
 本例患者经手术治疗病理证实为小肠淋巴瘤,非霍奇金型。小肠淋巴瘤起源于粘膜下层淋巴组织,病变沿肠壁向纵深方向发展。向外侵及浆膜层、肠系膜及其淋巴结,向内浸润粘膜,使之变平、僵硬。肠腔可以狭窄,也可以因为肌间神经丛受损而发生麻痹性扩张,病变肠区范围可以较肿瘤为广,而且界限不明确。临床表现主要有:腹痛伴有恶心、呕吐,腹块,腹泻、腹胀。钡餐造影主要表现为:病变广泛,小肠正常粘膜皱襞大部分或全部消失,肠腔内可见到无数小的息肉样充盈缺损,肠腔宽窄不一,沿肠壁可见到锯齿状切迹。

小肠淋巴瘤X线表现无明显特征性,需与克罗恩病、肠结核以及小肠癌相鉴别。克罗恩病可有节段性狭窄、卵石征或假息肉的征象,有时难与恶性淋巴瘤相鉴别。但克罗恩病一般病史较长,可有腹部肿块,往往因局部炎症穿孔形成内瘘,钡剂检查可见内瘘病变,节段性狭窄较光滑,近段扩张较明显,线性溃疡靠肠系膜侧,并有粘膜集中,肠襻可聚拢,呈车轮样改变。小肠恶性淋巴瘤一般无内瘘形成,临床表现重,X线下狭窄段不呈节段性分布,边缘不光滑,结节大小不一,溃疡和空腔较大而不规则。增殖型小肠结核X线表现为单发或多发的局限性肠腔狭窄,边缘较恶性淋巴瘤光滑,近端扩张亦较明显;溃疡型小肠结核龛影一般与肠管纵轴垂直,恶性淋巴瘤的溃疡部位不定,龛影较大而不规则。小肠癌病变往往局限,很少能触及包块,即使有亦是较小的局限的包块,X线钡餐检查仅为一处局限性肠管狭窄、粘膜破坏,这与小肠淋巴瘤范围较广不同。

\subsection{小肠腺癌}

\begin{figure}[!htbp]
 \centering
 \includegraphics{./images/Image00304.jpg}
 \captionsetup{justification=centering}
 \caption{小肠腺癌}
 \label{fig5-5-8}
  \end{figure} 

\textbf{【病史摘要】}
 女性,71岁。右下腹痛、进行性消瘦1年余,近期有黑便。体格检查:消瘦,右下腹触及鸡蛋大小包块,质硬,无移动,压痛明显,肝、脾未及,心、肺阴性。

\textbf{【X线表现】}
 全消化道钡餐造影示:空肠近端不规则充盈缺损,周围粘膜皱襞破坏消失,管壁僵硬,蠕动消失。空肠近端管腔不规则狭窄,狭窄端以上肠管明显扩张。

\textbf{【X线诊断】}  空肠近端占位,小肠腺癌。

\textbf{【评  述】}
 本例患者经手术治疗病理证实为空肠腺癌,侵及浆膜层。小肠恶性肿瘤主要为腺癌,多见于回肠,其次为空肠。可分为息肉型、溃疡型、弥漫型、溃疡浸润型四型。临床表现主要为腹部肿块、腹痛、肠梗阻、消瘦、消化道出血。

X线钡餐造影主要表现为:①肿块型腺癌,肠腔内见不规则的分叶状或菜花状充盈缺损,并常可引起套叠,若有溃疡形成,则显示不规则腔内龛影。②浸润狭窄型腺癌,肠腔呈环形向心性狭窄,狭窄段的近、远侧两端有病变突出于肠腔内,使病变段肠腔呈苹果核样形态,核心则为癌溃疡。③病变近侧的肠腔常有不同程度的扩张,有时在病变的一端或两端可出现反压迹征,这是由于病变区肠管与其上下的正常肠管截然分界,钡剂不能通过病变区,此时蠕动频繁增强的正常肠管覆盖在肿块上而造成。④病变部位粘膜皱襞破坏消失,管壁僵硬,蠕动消失。本例患者空肠近端X线表现符合肿块型腺癌诊断,其与小肠良性肿瘤、平滑肌瘤、腺瘤等疾病形成的边界光滑整齐的充盈缺损表现不同。

X线诊断小肠腺癌需注意与淋巴瘤和平滑肌肉瘤相鉴别,淋巴瘤一般侵犯范围较广,肿瘤沿肠壁侵犯,也可侵犯肠系膜,系膜肿大淋巴结侵犯、压迫肠管形成狭窄,但不易引起梗阻,部分淋巴肉瘤局部肠管不狭窄反而扩张。而平滑肌肉瘤则生长迅速,一般瘤体较大,常伴有巨大溃疡,肿瘤呈肠外生长,附近肠曲受压推移,但也不易形成梗阻。

\subsection{小肠类癌}

\begin{figure}[!htbp]
 \centering
 \includegraphics{./images/Image00305.jpg}
 \captionsetup{justification=centering}
 \caption{小肠类癌}
 \label{fig5-5-9}
  \end{figure} 

\textbf{【病史摘要】}
 女性,65岁。左下腹疼痛不适,无恶心、呕吐,近期出现腹泻伴皮肤潮红。体格检查:下腹部压痛,触及包块,质稍硬,肝、脾未及,心、肺阴性,尿液检查:5-羟吲哚醋酸增高。

\textbf{【X线表现】}
 全消化道钡餐造影示:局部回肠狭窄,呈息肉样充盈缺损,周围粘膜皱襞破坏、消失。

\textbf{【X线诊断】}  回肠占位性病变,腺癌可能,类癌待排。

\textbf{【评  述】}
 本例患者经手术治疗病理证实为回肠类癌。小肠类癌来源于肠壁腺泡的细胞,是一种能产生小分子多肽类或肽类激素的肿瘤。小肠类癌以回肠多见,其在粘膜下生长,多为1~3cm的粘膜下结节,呈广基息肉状。传统的观念认为类癌属于低度恶性肿瘤。可将类癌分为三类:①典型的类癌。②不典型类癌。③低分化神经内分泌癌(小细胞癌)。常见的症状为皮肤潮红、腹泻、喘息、右心瓣膜病、糙皮病等症状。

X线钡剂造影主要表现为:由于小肠类癌系粘膜下肿瘤,当肿瘤较小时,X线钡剂造影不易发现。肿瘤较大长入肠腔或浸润肠壁引起肠管狭窄时,可显示肠腔内息肉样充盈缺损或出现肠套叠征象,病变增大侵及肠系膜则可显示肠外肿块推移邻近肠襻,肠系膜的牵拉使肠襻呈辐辏状排列,肠壁扭曲、肠腔狭窄,甚至梗阻,严重者可引起肠系膜上动脉闭锁,而导致小肠缺血坏死。小肠类癌X线表现无特异性,故与小肠腺癌鉴别诊断困难,因此X线诊断该疾病时必须密切结合临床症状和实验室检查。本例患者X线表现为回肠息肉样充盈缺损,周围粘膜皱襞中断、破坏,结合患者临床上出现皮肤潮红、腹痛、腹泻等类癌综合征的表现,尿液检查5-羟吲哚醋酸增高,故应该考虑小肠类癌诊断的可能性。

\subsection{转移性小肠肿瘤}

\begin{figure}[!htbp]
 \centering
 \includegraphics{./images/Image00306.jpg}
 \captionsetup{justification=centering}
 \caption{回肠转移性肿瘤}
 \label{fig5-5-10}
  \end{figure} 

\textbf{【病史摘要】}
 男性,67岁。结肠癌术后2年,近1个月来右下腹疼痛不适,无节律性,时有腹胀伴恶心、呕吐。体格检查:右侧腹部见手术瘢痕,右下腹压痛,扪及包块,质硬,无移动,肝、脾未及,心、肺阴性。

\textbf{【X线表现】}
 全消化道钡餐造影示:末端回肠可见腔内不规则充盈缺损,中央部见不规则龛影,肠粘膜皱襞破坏。

\textbf{【X线诊断】}  回肠末段转移性肿瘤。

\textbf{【评  述】}
 本例患者经手术治疗病理证实为回肠末端腺癌(转移性)。转移性小肠肿瘤临床罕见,常发生于恶性肿瘤晚期或广泛转移者,尤其是来源于其他消化道恶性肿瘤者。转移灶多见于回肠,尤其是末端回肠,其次为空肠,十二指肠较少见。可单发也可多发,而鳞癌两者均可见到。组织学分类以腺癌及鳞癌居多,其次为恶性黑色素瘤。恶性肿瘤可通过血行、淋巴、腹腔内种植侵犯小肠,尤以血行和腹腔内种植更常见。

小肠气钡双对比造影检查对检出小肠转移瘤有较重要价值,具体表现可有:①局限性向心性狭窄,粘膜破坏,皱襞消失,肠壁光滑僵硬。②孤立性隆起性病变,充盈缺损。③溃疡形成,不规则较大龛影,常伴有轻度狭窄和结节样病变。④瘘管形成,钡剂外溢。⑤冰冻征,见于广泛的腹腔转移和恶性弥漫性腹膜间皮瘤。⑥多发性结节样肠壁压迹。可见有肠梗阻征象,偶有气腹。由于患者都有明确的恶性肿瘤病史,故结合其X线表现诊断一般较明确。

\section{结肠病变}

\subsection{结肠多发性憩室}

\begin{figure}[!htbp]
 \centering
 \includegraphics{./images/Image00307.jpg}
 \captionsetup{justification=centering}
 \caption{结肠多发性憩室}
 \label{fig5-6-1}
  \end{figure} 

\textbf{【病史摘要】}
 男性,66岁。大便习性改变伴腹泻近1个月。体格检查:腹软,无明显压痛,未扪及包块,肝、脾未及,心、肺阴性。

\textbf{【X线表现】}
 全消化道钡餐造影示:盲肠、升结肠、横结肠、降结肠及乙状结肠见多发大小不等乳头状的囊袋状影,阴影凸向肠壁的腔壁线之外,以降结肠段明显,各段结肠未见明显狭窄及其他异常。

\textbf{【X线诊断】}  结肠多发性憩室。

\textbf{【评  述】}
 结肠憩室国外常见,国内少见,好发于40岁以上,男性多于女性。可见于结肠各部分,而乙状结肠、降结肠最多见。结肠憩室一般无明显症状,或仅有轻微不适、便秘等。结肠憩室以钡剂灌肠造影检查较好,尤其是低张双重造影更有利于憩室的显示。常见表现为:突出于肠腔之外的圆形或类圆形阴影,位于结肠袋的顶端,大小不一,口部常较细小,其表现与憩室的大小、充盈状况及粪便多少等因素相关,如憩室内完全为钡剂充盈,则呈圆形或类圆形影;其内有粪便,钡剂涂布于粪团周围的粘膜上,造影表现为环状;憩室完全由粪便充填,钡剂只能充盈于憩室颈部,表现为柱状、杯口状等。憩室的正面观在充盈像上难于发现,需排除钡剂后或双重造影显示。结肠憩室需与溃疡性结肠炎鉴别,后者由于浅小溃疡使结肠壁显示多发的细小毛刺状突出,较大溃疡,结肠壁可见揿扣状壁龛,肠腔表面显示颗粒状粘膜,结肠腔壁线粗糙不光整,病史较长者往往结肠袋消失,管腔变窄。

\subsection{先天性巨结肠}

\begin{figure}[!htbp]
 \centering
 \includegraphics{./images/Image00308.jpg}
 \captionsetup{justification=centering}
 \caption{先天性巨结肠}
 \label{fig5-6-2}
  \end{figure} 

\textbf{【病史摘要】}
 女性,21岁。自幼诊断为先天性巨结肠,近期腹胀、便秘加重。体格检查:发育尚正常,腹部无明显压痛,未及包块,肝、脾未及,心、肺阴性。

\textbf{【X线表现】}
 气钡双重造影示,乙状结肠中段肠腔狭窄,近段结肠扩张明显,可见横向平行的粗大的粘膜皱襞,钡剂下行困难。

\textbf{【X线诊断】}  先天性巨结肠。

\textbf{【评  述】}
 本病是由于直肠或结肠远端的肠管持续痉挛,粪便淤滞在近端结肠,使结肠肥厚、扩张,是小儿常见的先天性肠道畸形。主要临床表现为顽固性便秘、腹胀、营养不良、发育迟缓等。钡剂灌肠的目的在于显示狭窄段及狭窄-扩张移行段结肠,不必充满整个结肠。侧位和前后位照片中可见到典型的痉挛肠段和扩张肠段,排钡功能差,24小时后仍有钡剂存留,若不及时灌肠洗出钡剂,可形成钡石,合并肠炎时扩张肠段肠壁呈锯齿状表现。新生儿先天性巨结肠要与其他原因引起的肠梗阻如结肠闭锁、胎便性便秘、新生儿腹膜炎等鉴别。较大的婴幼儿、儿童应与直肠肛门狭窄、管腔内外肿瘤压迫引起的继发性巨结肠、结肠无力(如甲状腺功能低下患儿引起的便秘)、习惯性便秘以及儿童特发性巨结肠(多在2岁以后突然发病,为内括约肌功能失调)等相鉴别。并发小肠结肠炎时与病毒、细菌性肠炎或败血症肠麻痹鉴别。对短段型先天性巨结肠,尤其是超短段型先天性巨结肠,难与特发性巨结肠鉴别。

\subsection{溃疡性结肠炎}

\begin{figure}[!htbp]
 \centering
 \includegraphics{./images/Image00309.jpg}
 \captionsetup{justification=centering}
 \caption{溃疡性结肠炎}
 \label{fig5-6-3}
  \end{figure} 

\textbf{【病史摘要】}
 女性,35岁。左下腹部疼痛,粘液血便近半年,腹泻、腹胀,食欲减退。体格检查:消瘦,腹部稍膨隆,左下腹压痛明显,肝、脾未及,心、肺阴性。

\textbf{【X线表现】}
 钡剂灌肠检查示:降结肠及横结肠脾曲段结肠袋消失,肠壁粗糙,边缘见多发锯齿状突起,粘膜面网状结构消失而见大小不等的点状致密影。

\textbf{【X线诊断】}  溃疡性结肠炎。

\textbf{【评  述】}
 本例患者钡剂灌肠检查X线征象典型,结肠镜所见证实为溃疡性结肠炎。溃疡性结肠炎原因不明,常发生于青壮年。本病首先侵犯直肠,继而沿长轴向上发展,逐一波及乙状结肠、降结肠、横结肠,甚至全部结肠,但仍以左半结肠为主。病变主要在粘膜与粘膜下层,溃疡很浅,底在肌层,可以自行愈合,溃疡与溃疡之间的肠粘膜面,可由于大量增生而形成许多炎症性息肉。病变愈合后,粘膜下层的纤维组织增生,可使肠腔普遍性变窄,肠管缩短,而呈光滑的直筒状外观。临床上有发作与缓解交替出现的肠炎症状,病程较长。钡剂灌肠检查可见从直肠开始就有刺激性痉挛收缩,左半结肠肠袋变浅,边缘可有许多尖刺状突起,而呈锯齿状。肠粘膜息肉样增生可表现为许多赤豆大小的充盈缺损。上述X线表现,以粘膜像或双对比造影像显示为佳。晚期纤维化之肠管,可呈铅管样结肠。溃疡性结肠炎主要与结肠克罗恩病及肠结核鉴别。结肠克罗恩病好发于右半结肠,病变呈跳跃式,且往往累及末端回肠。结肠结核病变可呈连续性,但往往大多数为末端回肠、盲肠、升结肠受累,发生于结肠其他部位者少见,与溃疡性结肠炎不同。

\subsection{结肠息肉}

\begin{figure}[!htbp]
 \centering
 \includegraphics{./images/Image00310.jpg}
 \captionsetup{justification=centering}
 \caption{结肠多发息肉}
 \label{fig5-6-4}
  \end{figure} 

\textbf{【病史摘要】}
 男性,35岁。下腹部疼痛伴便血半个月,食欲减退。体格检查:腹软,腹部无明显压痛,未扪及包块,肝、脾未及,心、肺阴性。

\textbf{【X线表现】}
 气钡双重造影示:直肠内见多发轮廓光整的充盈缺损,基底部位于肠壁,肠腔壁柔软,光滑整齐。

\textbf{【X线诊断】}  直肠及乙状结肠多发息肉。

\textbf{【评  述】}
 本例患者经结肠镜检查病理证实为直肠及乙状结肠腺瘤样息肉。凡从粘膜表面突出到肠腔的息肉状病变,在未确定病理性质前均称为息肉,按病理可分为:腺瘤样息肉,炎性息肉,错构瘤型息肉。结肠息肉多见于40岁以上成人,男性稍多。大部分病例并无引人注意的症状。仅在体格检查或尸体解剖时偶然发现,部分病例可以具有如便血、粪便改变、腹痛及息肉脱垂等症状。适当的检查方法对提高诊断效率,特别是较小的息肉诊断最为关键,理想的检查是要获得良好的粘膜像与气钡双重造影,结合多轴面透视观察,适当加压,才能充分显示病变。

钡灌肠检查表现主要为:肠腔内轮廓光整的充盈缺损,多发性息肉表现为多个大小不等充盈缺损,带蒂的息肉可显示其长蒂,有一定的活动度。而息肉病表现为直肠、乙状结肠及结肠其他部位有大大小小的充盈缺损,在粘膜相上出现无数轮廓光整葡萄状的块影,充满肠腔。结肠息肉有时应注意与肠腔内气泡和粪块相鉴别,粪块和气泡转换体位时形态、位置均会有改变,尤其重复检查对鉴别帮助最大,因为粪块和气泡不会在多次检查中位于同一部位。结肠息肉来源于粘膜上皮,不累及肌层,故局部肠壁及结肠袋一般正常,此与溃疡性结肠炎形成的假性息肉所致的肠壁及结肠袋的改变不同,应注意区别。息肉的恶变,文献报道,息肉大小在良、恶性鉴别上有肯定意义,息肉直径大于2cm者恶变概率在50%,小于5mm者恶变概率不到0.1%。带蒂息肉恶变概率较小,大于1cm的息肉,基底部出现不规则凹陷和回缩可考虑为恶变征象。

\subsection{回盲型肠套叠}

\begin{figure}[!htbp]
 \centering
 \includegraphics{./images/Image00311.jpg}
 \captionsetup{justification=centering}
 \caption{回盲部肠套叠}
 \label{fig5-6-5}
  \end{figure} 

\textbf{【病史摘要】}
 男性,6岁。因腹胀、恶心、呕吐伴肛门停止排气1天入院。体格检查:右下腹痛,拒按,触及腊肠状腹块,肝、脾未及,心、肺阴性。大便隐血(++)。

\textbf{【X线表现】}
 钡剂灌肠示结肠肝曲处见钡剂受阻,呈杯口样充盈缺损,其内可见弹簧状纹理。灌注空气,示钡剂进入升结肠、盲肠及回肠末端,肠套叠复位。

\textbf{【X线诊断】}  回盲部肠套叠。

\textbf{【评  述】}
 本例患者经钡剂灌肠检查及空气灌注整复,确诊为回盲型肠套叠。肠套叠是指一段肠管套入与其相连的肠腔内,并导致肠内容物通过障碍。有原发性和继发性两类。原发性肠套叠多发生于婴幼儿,继发性肠套叠则多见于成人。成人肠套叠多发生在回盲部,且继发于肿瘤、息肉等。肠套叠可发生在小肠或大肠的任何部位,按套入肠的顶端和外鞘、颈部肠段的不同分为5型:小肠型,回盲型,回结型,结肠型,空肠胃套叠。上述类型中,回盲型肠套叠发病率最高。回盲型肠套叠系套入部位于盲肠内,造成充盈缺损而导致盲肠变形。急性肠套叠临床表现主要为急性肠梗阻症状、便血,并可扪及腊肠状腹块,慢性肠套叠表现为慢性不全性梗阻,同时伴有便血、腹块。本病常用空气或钡剂灌肠法检查,在不全性梗阻的病例中可使用口服法检查,但应特别慎重,否则有可能加重梗阻而使症状加重。

X线影像主要为:①钡剂在套叠部,先入套入部,或称套叠中央管,其表现较具特征,即该套入部肠腔明显变窄,由于该套入部充盈钡剂程度不同,表现各异,充盈多时,可见皱襞呈纵形平整的条索,充盈不足时,仅呈窄细的线形,远端肠扩大,呈杯口状或螺旋状环绕套入中央管。②由于成人慢性肠套叠以回盲结型多见,回肠末端及其系膜被卷入升结肠内,受系膜的牵拉,使整个套叠部向内下移位,遇有局部痉挛、激惹等使上述套叠结构显示不清时,如果见升结肠、肝曲有向内、下移位现象,应考虑回盲部套叠所致,并排除回盲部结核和肿瘤。③钡剂通过套叠部时间延长呈半梗阻状态。④套叠头部常呈分叶状,钡剂仅从其中之一通过至远端结肠。肠套叠X线征象典型,诊断一般不难,但引起套叠的肠壁实质性占位有时确诊并不容易,尤其是肿瘤较大,加之肠套叠套鞘与套入部形成密集的弹簧状及发状粘膜皱襞的遮盖,肿瘤形态不易观察,检查及诊断应多时相、多体位、密切结合临床间断观察。

\subsection{阑尾周围脓肿}

\begin{figure}[!htbp]
 \centering
 \includegraphics{./images/Image00312.jpg}
 \captionsetup{justification=centering}
 \caption{阑尾周围脓肿}
 \label{fig5-6-6}
  \end{figure} 

\textbf{【病史摘要】}
 男性,35岁。转移性右下腹疼痛1周,伴发热、恶心、呕吐。体格检查:右下腹压痛明显,扪及质软包块,肝、脾未及,心、肺阴性。血常规血白细胞计数12×10\textsuperscript{9}
/L,中性粒细胞比例85%。

\textbf{【X线表现】}
 钡剂灌肠检查示:盲肠下端管腔狭窄,边缘不整齐,见弧形压迹影,钡剂通过有激惹征象,周围粘膜皱襞未见明显中断、破坏。

\textbf{【X线诊断】}  阑尾区占位性病变,考虑阑尾周围脓肿可能性大。

\textbf{【评  述】}
 本例患者经手术治疗,术后病理证实为阑尾脓肿。急性阑尾炎化脓坏疽或穿孔,如果此过程进展较慢,大网膜可移至右下腹部将阑尾包裹、粘连形成炎性肿块或阑尾周围脓肿。细菌感染和阑尾腔的阻塞是阑尾炎发病的两个主要因素。由早期炎症加重而致,或由于阑尾管腔梗阻,内压增高,远端血运严重受阻,感染形成和蔓延迅速,以致数小时内即成化脓性甚至蜂窝织炎性感染。阑尾肿胀显著,浆膜面高度充血并有较多脓性渗出物,部分或全部为大网膜所包裹。临床表现:患者多有右下腹疼痛,或者转移性右下腹疼痛病史,可有发热、恶心、呕吐等表现。亦可有轻微腹泻等表现。少数患者可因大网膜压迫肠管,造成不全肠梗阻症状。钡灌肠能很好地观察结肠及回盲部的充盈情况和粘膜有无异常,为首选方法。钡剂造影检查可见右下腹包块与肠管粘连,不能分开;盲肠变形,边缘不规则,但粘膜皱襞无破坏,局部有压痛;盲肠有激惹征象,钡剂通过快,盲肠也可处于痉挛状态;盲肠局部可出现压迹,末端回肠可同时向上推移。若脓肿与盲肠相通,可使之显影,显示为肠道外不规则窦腔。根据上述阑尾脓肿的X线特点,结合临床,多数诊断当无困难,但少数病例由于临床表现复杂,需与下列回盲部病变鉴别:包括回盲部良、恶性肿瘤及炎性病变,有些表现与脓肿相似,但均有相应的临床及X线特点可资鉴别。如结肠癌时的肠腔狭窄、充盈缺损,形态恒定,管壁僵硬,粘膜破坏,无弧形压迹,能触及肠腔内包块,临床可有粘液血便等。炎性病变可见肠腔狭窄、短缩,牵拉移位及激惹等,且有弧形压迹及包块,与阑尾周围脓肿表现不同。

\subsection{结肠癌}

\begin{figure}[!htbp]
 \centering
 \includegraphics{./images/Image00313.jpg}
 \captionsetup{justification=centering}
 \caption{横结肠浸润性结肠癌}
 \label{fig5-6-7}
  \end{figure} 

\textbf{【病史摘要】}
 男性,55岁。腹痛、腹胀、便秘2个月余,粘液脓血便。体格检查:消瘦,左上腹扪及包块,质硬,肝、脾未及,心、肺阴性。大便常规隐血(++)。

\textbf{【X线表现】}
 气钡双重造影示:横结肠管腔向心性狭窄,粘膜皱襞中断、破坏,病变与正常肠壁分界清楚。

\textbf{【X线诊断】}  横结肠浸润性结肠癌(进展期)。

\textbf{【评  述】}
 本例经手术治疗病理证实为横结肠腺癌。结肠癌是发生于结肠部位的常见的消化道恶性肿瘤。好发部位为直肠及直肠与乙状结肠交界处,以40~50岁年龄组发病率最高。浸润性结肠癌以向肠壁各层呈浸润生长为特点。病灶处肠壁增厚,表面粘膜皱襞增粗、不规则或消失变平。早期多无溃疡,后期可出现浅表溃疡。如肿瘤累及肠管全周,可因肠壁环状增厚及伴随的纤维组织增生使肠管狭窄,即所谓的环状缩窄型,此时在浆膜局部可见到缩窄环。切面肿瘤边界不清,肠壁因肿瘤细胞浸润而增厚。临床常见症状为排便习惯改变,血性便及肠梗阻。肠梗阻可表现为突然发作的急性完全性梗阻,但多数为慢性不完全性梗阻,腹胀很明显,大便变细形似铅笔,症状进行性加重最终发展为完全性梗阻。钡剂灌肠检查可见癌肿部位的肠壁僵硬,扩张性差,蠕动至病灶处减弱或消失,结肠袋形态不规则或消失,肠腔狭窄,粘膜皱襞紊乱、破坏或消失、充盈缺损等。结肠进展期各型癌肿X线征象均较明确,诊断不难。

结肠癌有时需注意和结肠其他少见肿瘤的鉴别,平滑肌瘤以累及直肠为多,X线表现为粘膜下肿瘤的特点,其大部分位于肠腔外是其特征。淋巴瘤少见,发生多位于盲肠或直肠,常常累及末端回肠,环状浸润范围较长,可表现为向心性狭窄,但很少出现梗阻,弥漫型可累及长段或全结肠与结肠癌不同。类癌大多发生于直肠,其次为盲肠。X线表现为不规则伞状充盈缺损,另一种为不规则环状狭窄,与结肠癌不易鉴别。发生在直肠的类癌,直肠镜检查优于X线检查。

\section{急腹症}

\subsection{胃穿孔}

\begin{figure}[!htbp]
 \centering
 \includegraphics{./images/Image00314.jpg}
 \captionsetup{justification=centering}
 \caption{消化道穿孔}
 \label{fig5-7-1}
  \end{figure} 

\textbf{【病史摘要】}
 男性,45岁。进食后突发持续性剧烈腹痛1小时,伴恶心、呕吐,既往有胃溃疡病史。体格检查:全腹压痛、反跳痛与肌紧张,肠鸣音减弱,体温38.2℃,血常规白细胞计数11.7×10\textsuperscript{9}
/L,中性粒细胞比例85%。

\textbf{【X线表现】}
 立位腹部平片示两侧膈下见新月形游离气体影;左侧卧位片示右侧胸腹壁下见半月形游离气体影。

\textbf{【X线诊断】}  消化道穿孔,结合胃溃疡病史,诊断胃溃疡急性穿孔。

\textbf{【评  述】}
 本例患者经手术治疗证实为胃体小弯侧溃疡穿孔。胃肠腔外气体的来源最多见于消化道穿孔,气体游离于腹腔内,其次是腹腔内产气细菌性脓肿及外科腹部手术或外伤后空气进入腹腔。膈下游离气体是胃肠道穿孔的最重要的X线表现。立位X线检查时,腹腔内游离气体上升于膈下,呈镰刀样或半月形透明阴影。右侧比左侧多见,但亦有单独出现在左侧的,此时要注意与结肠脾曲及单纯胃泡相区别。若有疑问可进一步做左侧卧位水平X线投照及半立位侧水平X线片检查,观察肝脏上方及剑突下有无游离气体。胃后壁溃疡穿孔时,气体进入小网膜囊,于上腹部或左上腹部存在透明气影,它不随体位变化而移动。

膈下游离气体应与假性气腹相鉴别,以免误诊。可出现假性气腹的有:①横膈下脂肪垫,肥胖患者在透视或平片X线检查时,于膈下有时可见条状或带状不规则透亮阴影,很似膈下游离气体。但透亮度一般比气腹低,变换体位时,此透亮影固定不变,无移动性。②膈下脓疡,于膈下可见一局限性包裹性气影,可有液平面。此外尚有患侧膈肌升高、运动减弱或消失,有胸膜反应或胸腔积液等X线征象。③间位结肠或间位小肠,仔细观察可见到结肠袋间隔或小肠的环形皱襞阴影,可与膈下游离气体相鉴别。④两侧弥漫性阻塞性肺气肿或下肺野局限性肺气肿时,其气肿的肺组织投影于膈下区域,有时很像膈下游离气体,要注意鉴别。⑤内脏转位,内脏反位患者,胃泡气影位于右侧膈下,同时可见到其他脏器的反位现象,区别并不困难。气腹除见于上述原因外,还可见于下列原因:人工气腹、腹腔穿刺后、输卵管通气术后、阴道冲洗后、肠壁气囊肿破裂等,故诊断消化道穿孔需密切结合临床资料,综合分析诊断。

\subsection{小肠机械性肠梗阻}

\begin{figure}[!htbp]
 \centering
 \includegraphics{./images/Image00315.jpg}
 \captionsetup{justification=centering}
 \caption{小肠机械性肠梗阻}
 \label{fig5-7-2}
  \end{figure} 

\textbf{【病史摘要】}
 女性,65岁。阵发性腹痛3天并逐渐加重,伴恶心、呕吐、肛门停止排便排气1天。体格检查:腹部膨隆,脐周压痛,肠鸣音亢进,可闻及气过水声。

\textbf{【X线表现】}
 立位腹部平片示:肠腔气体郁积,见多发宽窄不等、阶梯状排列的气液平面。

\textbf{【X线诊断】}  低位单纯性小肠机械性梗阻。

\textbf{【评  述】}
 肠梗阻为常见的急腹症,X线检查是诊断的可靠方法之一。本例患者立位腹部平片梗阻征象明确,诊断成立。小肠高位机械性肠梗阻时,梗阻近端之肠管内大量液体滞留,而气体多反流入胃内,故X线征象不多,平片诊断常很困难。最好采用口服有机碘溶液造影检查。低位性小肠梗阻时,梗阻近端的小肠积气扩张,小肠呈线团状或鱼骨状粘膜皱襞形态,主要见于空肠。回肠环形粘膜皱襞较少,特别远端回肠更少。当肠管明显扩张时,回肠粘膜皱襞可完全消失。梗阻远端肠曲收缩,结肠内很少或没有气体存留。于立位照片时,腹部可见多个呈阶梯状液平面,似倒U形。透视可见液平面上下不规则地移动。如机械性肠梗阻持续时间长,可继发肠麻痹(反射性肠淤张)。此时前者的征象可以完全被掩盖,对诊断及识别病变的真正过程造成一定困难,需全面检查并结合临床仔细分析才能得出正确结论。

\subsection{麻痹性肠梗阻}

\begin{figure}[!htbp]
 \centering
 \includegraphics{./images/Image00316.jpg}
 \captionsetup{justification=centering}
 \caption{小肠麻痹性肠梗阻}
 \label{fig5-7-3}
  \end{figure} 

\textbf{【病史摘要】}
 女性,56岁。1周前因胃癌行毕Ⅰ式胃大部分切除术,现自述腹胀、肛门无排便排气。体格检查:腹部见手术吻合钉影,腹部膨隆,压痛,未及包块,肝、脾未及,心、肺阴性。

\textbf{【X线表现】}
 立位腹部平片示:残胃、小肠、结肠均积气,肠腔扩张不明显,可见多发小液平。

\textbf{【X线诊断】}  小肠麻痹性肠梗阻。

\textbf{【评  述】}
 本例患者有近期胃大部分切除术史,现临床出现肠梗阻症状,结合立位腹部平片表现,麻痹性肠梗阻诊断不难。麻痹性肠梗阻常见于腹部手术后、腹部炎症、腹膜炎、胸腹部外伤及感染等。临床症状表现为疼痛、呕吐、腹胀、肛门停止排便排气、腹软、肠鸣音减弱或消失。麻痹性肠梗阻由于没有肠管的器质性狭窄,而是肠管处于麻痹状态,引起肠内容物的通过和吸收障碍。其X线特点为胃、大肠、小肠呈均等的积气扩张,并有液平面,液平面较宽,但小于机械性小肠梗阻。多次复查肠管形态改变不明显。如果不合并有腹膜炎,则扩张的肠曲相互靠近,肠间隙正常。如果同时合并腹腔内感染,则肠间隙可增宽,腹脂线模糊。

\subsection{小肠绞窄性肠梗阻}

\begin{figure}[!htbp]
 \centering
 \includegraphics{./images/Image00317.jpg}
 \captionsetup{justification=centering}
 \caption{小肠绞窄性肠梗阻}
 \label{fig5-7-4}
  \end{figure} 

\textbf{【病史摘要】}
 男性,48岁。突然出现腹部剧痛伴恶心、呕吐、肛门停止排便排气1天。体格检查:腹部膨隆,有压痛,可见肠形,听诊肠鸣音亢进,有气过水声,血压110/70mmHg。

\textbf{【X线表现】}
 立位腹部平片示:小肠积气,扩张明显,中腹部见多个跨度卷曲肠襻,呈花瓣型。

\textbf{【X线诊断】}  小肠绞窄性肠梗阻。

\textbf{【评  述】}
 绞窄性肠梗阻是由于肠系膜血管发生狭窄,致使血循环发生障碍,引起小肠坏死。常见的原因是小肠扭转、粘连带压迫和内疝等。肠系膜过长、肠管功能紊乱以及肠内容物增加均易造成小肠扭转。绞窄性肠梗阻时,早期即出现严重的临床症状:休克、呕吐、便血、脉快而弱等。

X线改变主要有:①假肿瘤征:充满液体的嵌闭肠曲呈圆形肿块,边缘清楚,不可活动,多位于下腹部,可压迫周围肠曲或膀胱引起移位。立位时可见液平面,但若是完全性绞窄性梗阻,则绞窄肠曲内多无气体。②咖啡豆状征:这是较具特征性的改变;当肠系膜绞窄时,系膜因痉挛水肿而挛缩变短,于是以肠系膜为中心,牵拉闭襻梗阻肠曲的两端使之纠集变位,出现各种排列状态,如C字形、8字形、花瓣征、香蕉征等。③阻塞近端肠管大量积液扩张并有液平面。因肠管麻痹,气体多反流至胃,形成小肠内气体较少,液平面较长,其上气柱低而扁。且活动度低。④空回肠换位征。本例患者出现典型的花瓣征,结合临床症状,诊断小肠绞窄性肠梗阻。绞窄性肠梗阻的诊断非常重要,因为明确绞窄性肠梗阻诊断后,外科需立即急诊手术治疗,否则病死率极高。因此,当已确定小肠梗阻时,还必须检查分析是否有绞窄性肠梗阻可能,并结合临床症状、体征和发病过程,再排除与其相似的疾病,可做出初步诊断。

\subsection{乙状结肠扭转}

\begin{figure}[!htbp]
 \centering
 \includegraphics{./images/Image00318.jpg}
 \captionsetup{justification=centering}
 \caption{乙状结肠扭转}
 \label{fig5-7-5}
  \end{figure} 

\textbf{【病史摘要】}
 男性,55岁。突发腹部持续性剧烈疼痛半天,肛门停止排气、排便。有右股骨颈骨折内固定手术史。体格检查:腹部膨隆,左下腹压痛明显,肝、脾未及,心、肺阴性。

\textbf{【X线表现】}
 钡剂灌肠检查示:乙状结肠中段阻塞,近端狭窄呈鸟嘴状,鸟嘴尖端指向左侧,远端直肠、乙状结肠扩张明显。

\textbf{【X线诊断】}  乙状结肠扭转。

\textbf{【评  述】}
 本例患者经手术治疗明确诊断为乙状结肠扭转。乙状结肠较长,而乙状结肠系膜附着处又短窄,近侧和远侧两侧肠管接近,肠襻活动度大,这是容易发生扭转的解剖基础。乙状结肠扭转可以呈顺时针或逆时针方向。扭转对肠管血循环的影响程度,主要决定于扭转的多少和松紧程度,如扭转180°时,肠系膜血循环可无绞窄,仅位于乙状结肠壁后面的直肠受压而出现单纯性肠梗阻。扭转超过360°时,必将造成绞窄性闭襻性肠梗阻。X线检查腹部平片可见腹部偏左明显充气的巨大孤立肠襻自盆腔达中上腹部,甚至可达膈下,占据腹腔大部形成所谓弯曲管征。在巨大乙状结肠肠襻内,常可看到两个处于不同平面的液气面。左、右半结肠及小肠有不同程度的胀气。钡剂灌肠造影可见钡剂在直肠与乙状结肠交界处受阻,钡柱尖端呈锥形或鸟嘴形,且灌肠之容量往往不及500ml(正常可灌入2000ml以上),并向外流出,即可证明在乙状结肠处有梗阻。此项检查仅适用于一般情况较好的早期扭转病例,当有腹膜刺激征或腹部压痛明显者禁忌钡灌肠检查,否则有发生肠穿孔的危险。

\section{胆道疾病}

\subsection{先天性胆总管囊肿}

\begin{figure}[!htbp]
 \centering
 \includegraphics{./images/Image00319.jpg}
 \captionsetup{justification=centering}
 \caption{先天性胆总管囊肿}
 \label{fig5-8-1}
  \end{figure} 

\textbf{【病史摘要】}
 男性,9岁。反复发作性右上腹痛伴皮肤发黄、瘙痒,近期疼痛加剧并伴发热。体格检查:皮肤、巩膜黄染,腹软,右上腹可触及一鸡蛋大小肿块,质软,可推动,轻度压痛,心、肺阴性。

\textbf{【X线表现】}
 经皮肝穿胆管造影示:胆总管扩张明显,呈囊状,壁光滑,其扩张段直径与胆道其余部分失去比例关系。

\textbf{【X线诊断】}  先天性胆总管囊肿。

\textbf{【评  述】}
 本例患者经手术证实为先天性胆总管囊肿。先天性胆总管囊肿又称为胆总管囊性扩张,病因尚不明确,多见于女性、儿童。胆总管囊性扩张范围不一定,可涉及胆总管某一部分或全部,囊肿大小不等,多位于胆总管中段。胆总管囊肿的典型三联症是腹痛、黄疸和腹部包块,其中以腹部疼痛最为明显。口服或静脉胆管造影多不显影,囊肿穿刺造影虽可显示囊肿大小和位置,但有一定的危险性,即有可能继发胆汁性腹膜炎。内镜逆行胰胆管造影(ERCP)则能直接显示整个胆道系统,尤其是对了解胰管与胆管的关系,肝内胆管有无结石和狭窄等提供直接证据。先天性胆总管囊肿,有时需注意与胆总管下端肿瘤或结石引起的肝内外胆管扩张相鉴别,前者胆总管呈球形或梭形局限性扩张,肝内胆管扩张但并不广泛,以靠近肝门近端周围肝管扩张为其特点;后者形成梗阻所致梗阻以上肝内外胆管扩张广泛而均匀,胆总管下端可见圆形或类圆形低密度充盈缺损结石影为其特点。

\subsection{先天性胆囊畸形}

\begin{figure}[!htbp]
 \centering
 \includegraphics{./images/Image00320.jpg}
 \captionsetup{justification=centering}
 \caption{先天性胆囊畸形}
 \label{fig5-8-2}
  \end{figure} 

\textbf{【病史摘要】}
 男性,42岁。右上腹痛1年,向肩背部放射,疼痛时伴恶心、呕吐。体格检查:腹软,右上腹轻压痛,未扪及包块,皮肤、巩膜无黄染,心、肺阴性。

\textbf{【X线表现】}
 胆囊造影检查示:胆囊呈葫芦形,上部见囊壁局部缩窄,胆囊壁光整,肝内胆管及肝总管、胆总管未见充盈缺损及扩张,造影剂经十二指肠弥散通畅(图A)。

\textbf{【X线诊断】}  胆囊炎;先天性胆囊畸形(葫芦形)。

\textbf{【评  述】}
 一般的胆囊先天性异常无临床症状且不影响生理功能,仅在影像检查时偶然发现。如葫芦形胆囊(图A);错位胆囊,包括肝内胆囊(图B)及左位胆囊(图D);巨胆囊(图C)等。此类异常的诊断有赖于对影像的正确认识,最好有两种以上的检查印证,如B超、胆囊造影、MRCP等。其诊断的意义在于与病理状态的鉴别及了解有无合并其他病变。

\subsection{胆道蛔虫症}

\begin{figure}[!htbp]
 \centering
 \includegraphics{./images/Image00321.jpg}
 \captionsetup{justification=centering}
 \caption{胆道蛔虫症}
 \label{fig5-8-3}
  \end{figure} 

\textbf{【病史摘要】}
 男性,23岁。右上腹部剧痛伴恶心、呕吐。体格检查:右上腹压痛明显,肝、脾未及,心、肺阴性。

\textbf{【X线表现】}
 胆总管及右肝管内见一边缘光整、稍呈弯曲的条状透亮阴影,右肝管稍扩张,胆总管未见扩张。

\textbf{【X线诊断】}  胆道蛔虫症。

\textbf{【评  述】}
 依据胆管内显示边缘平滑并呈弯曲的条状透亮阴影,形状与蛔虫相似,两侧的造影剂呈现出双轨征,胆道蛔虫症可以确诊。胆道蛔虫症是肠蛔虫病的常见并发症,也是常见的急腹症。X线检查以胃肠道钡剂造影和直接胆管造影为主。平片检查价值有限,静脉胆道造影作用也有限。超声能清楚显示进入胆管的蛔虫,并能在超声导向下做取虫治疗。胆道造影检查常表现为胆管内发辫状或长圆柱状充盈缺损,为蛔虫的直接征象。充盈缺损影纵轴与胆管方向一致,多为一条,也有数目较多者。蛔虫不仅位于肝外胆管,也可伸入肝内胆管。蛔虫死亡解体后的残体以及所形成的结石也形成充盈缺损,形态各异,注意同单纯胆石鉴别。除充盈缺损外,还可显示胆管扩张性改变。

\subsection{胆总管结石}

\begin{figure}[!htbp]
 \centering
 \includegraphics{./images/Image00322.jpg}
 \captionsetup{justification=centering}
 \caption{胆道多发性结石}
 \label{fig5-8-4}
  \end{figure} 

\textbf{【病史摘要】}
 男性,55岁。胆囊结石行胆囊切除术后,反复发作性右上腹胀痛、不适近3年,无明显发热、恶心、呕吐,近1周巩膜、皮肤黄染,伴有呕吐,食欲减退。体格检查:皮肤、巩膜黄染,上腹部压痛,未扪及包块,肝、脾无增大,心、肺阴性。

\textbf{【X线表现】}
 胆总管内见两枚大小不一的充盈缺损,胆总管及肝内胆管均显示扩张(图A),胆囊未见。

\textbf{【X线诊断】}  胆总管多发结石。

\textbf{【评  述】}
 本例患者经十二指肠大乳头括约肌切开网篮取出两枚完整结石。胆管结石常见。结石可位于肝胆管的任何部位,如胆总管下端(图B);肝内胆管(图C);肝总管(图D),以胆总管结石为多见。胆总管结石大多由胆色素及胆固醇组成,一般含钙盐较少,通常透X线,故胆区平片观察结石价值不大。B超检查可发现胆管内结石及胆管扩张影像,故胆管结石一般首选B超检查,必要时可加行ERCP或PTC。

PTC的X线特征有:①肝总管或左右肝管处有环形狭窄,狭窄近端胆管扩张,其中可见结石阴影。②左右肝管或肝内某部分胆管不显影。③左右叶肝内胆管呈不对称性、局限性、纺锤状或哑铃状扩张。ERCP可选择胆管造影,对肝内胆管结石具有较高的诊断价值,可清晰显示肝内胆管结石,确定结石的部位、大小、数量,肝内胆管的狭窄或远端扩张。CT扫描对于肝内胆管结石的诊断意义较大。胆总管结石由于较大而容易被发现,而胰腺钓突内结石则较小,尤其是含钙量少时只表现为小致密点,因为CT密度分辨率较高,则可显示。胆总管扩张时,胆总管的横断面呈边界清楚的圆形或椭圆形低密度影,自上而下逐渐变小。核磁共振胰胆管造影(MRCP)是不同于ERCP的全新的检查方法,属无创性检查,不需要做十二指肠镜即可诊断肝内、外胆管结石。对肝内胆管结石有较大诊断价值,但价格较贵,不易普及。总之,B超、ERCP、胆道镜等方法诊断价值较大,简便易行,是诊断肝内胆管结石的首选方法。尤其是ERCP和胆道镜,对肝内胆管结石诊断的准确性高于B超。在B超检查发现肝内胆管结石后,应常规进行上述方法的检查。

胆管结石需与胆管肿瘤鉴别。胆管良性肿瘤极为少见。多见的胆管癌阻塞端常有破坏、狭窄、僵直及不规则充盈缺损。胆管结石的阻塞端多为圆形充盈缺损,典型者则显示杯口状充盈缺损是其特征,无破坏、狭窄及僵直改变。胆管癌扩张的肝内胆管往往呈软藤状,而结石扩张的肝内胆管则显示枯枝状,两者表现不同。

\subsection{慢性胆囊炎、胆结石}

\includegraphics{./images/Image00323.jpg}

\begin{figure}[!htbp]
 \centering
 \includegraphics{./images/Image00324.jpg}
 \captionsetup{justification=centering}
 \caption{胆囊结石}
 \label{fig5-8-5}
  \end{figure} 

\textbf{【病史摘要】}
 女性,35岁。右上腹疼痛近1年,伴发作时恶心、呕吐。体格检查:腹软,右上腹压痛,未扪及包块,皮肤、巩膜无黄染,心、肺阴性。

\textbf{【X线表现】}
 腹部平片示:胆囊内见多发大小不等结节样充盈缺损,边缘密度较高,中央密度较低,边缘光整(图A)。

\textbf{【X线诊断】}  胆囊结石。

\textbf{【评  述】}
 胆囊炎、胆结石临床常见。结石可发生在胆囊任何部位,如胆囊底部(图B);胆囊底、体颈部(图C);胆囊管部(图D)。胆囊阳性结石为10%~20%,结石密度不均匀,可为年轮状致密影,中央透光而周围呈不同厚度的环形影,形如石榴子状。急性胆囊炎依据患者的病史及症状、实验室检查,即可做出诊断。X线检查对其诊断有一定限度。慢性胆囊炎由于胆囊壁增厚、瘢痕收缩以及周围组织粘连,并经常与胆囊结石并发,故X线征象典型。阳性胆囊结石平片诊断中需与肾结石、肾上腺钙化、肠系膜淋巴钙化、胰腺结石、肝包囊虫钙化等进行鉴别,一般通过改变体位不难区别。阴性结石在造影中形成充盈缺损,这需与胆囊良性肿瘤、胆固醇息肉、胆囊腺肌增生症、胆囊癌以及结肠内气体重叠干扰等进行鉴别。胆囊癌为胆囊腔内分叶状不规则充盈缺损,囊壁僵硬,有内陷,轮廓不光整,特别是患者的临床表现以及胆囊区可触及的质硬肿块为鉴别的重要参考,而胆石在不同体位时可以移动,是重要鉴别点。

\subsection{胆管癌}

\begin{figure}[!htbp]
 \centering
 \includegraphics{./images/Image00325.jpg}
 \captionsetup{justification=centering}
 \caption{胆管癌}
 \label{fig5-8-6}
  \end{figure} 

\textbf{【病史摘要】}
 男性,65岁。因胆囊炎、胆结石行胆囊切除术后5年,近1个月来右上腹部疼痛伴黄疸。体格检查:体瘦,皮肤、巩膜黄染,腹壁紧张,右上腹压痛,未扪及包块,肝、脾未及,心、肺阴性。

\textbf{【X线表现】}
 ERCP造影检查示:肝总管不规则狭窄、扭曲,梗阻近侧端胆管正常,肝内胆管扩张(图A)。

\textbf{【X线诊断】}  肝总管胆管癌。

\textbf{【评  述】}
 胆管癌可发生于胆管的各个部位,如胆总管下段(图B)。近50%肝外阻塞的患者是由非结石性病因引起的,其中以恶性肿瘤最多见。这些恶性肿瘤大多数发生于远端胆总管所在的胰头部,少数发生于壶腹部、胆管、胆囊和肝内。由转移性肿瘤和淋巴结阻塞胆管的现象极为少见。发生在胆管的一些良性乳头状瘤或绒毛状腺瘤也可阻塞胆管。早期肿瘤较小时,多无临床症状。随着胆管阻塞的症状和体征进行性加重,可见黄疸、不同程度的腹部不适、厌食、体重下降、皮肤瘙痒、腹部可触及包块或胆囊等,但寒战、高热少见。

X线所见:早期多为偏侧性充盈缺损而造成胆管狭窄,其范围多在1cm以下,边缘光滑者应考虑为良性肿瘤,边缘不规则者多为癌,同时伴有狭窄上端胆管扩张;晚期则胆管不显影。

胆管肿瘤需与胆管结石鉴别。胆管良性肿瘤极为少见。多见的胆管癌阻塞端常有破坏、狭窄、僵直及不规则充盈缺损。胆管结石的阻塞端多为圆形充盈缺损,典型者则显示杯口状充盈缺损是其特征,无破坏、狭窄及僵直改变。胆管癌扩张的肝内胆管往往呈软藤状,而结石扩张的肝内胆管则显示枯枝状,两者表现不同。

结节型胆管癌影像学有时需与胆管良性肿瘤如乳头状腺瘤相鉴别,后者少见,其在胆管内可形成广基底或带蒂的充盈缺损,轮廓光整,胆管壁光滑无内陷。而浸润型胆管癌所致胆管不规则狭窄,管壁粗糙僵硬与硬化型胆管炎累及范围较长、管腔狭窄、管壁光滑的影像也不同。

\section{胰腺病变}

\subsection{慢性胰腺炎}

\begin{figure}[!htbp]
 \centering
 \includegraphics{./images/Image00326.jpg}
 \captionsetup{justification=centering}
 \caption{慢性胰腺炎}
 \label{fig5-9-1}
  \end{figure} 

\textbf{【病史摘要】}
 男性,55岁。反复发作性上腹部痛2年,伴恶心、呕吐,诊断为慢性胰腺炎,行药物及饮食治疗,近1周出现上腹部疼痛加剧,伴腹泻、恶心、呕吐、食欲减退。体格检查:上腹部压痛明显,未扪及包块,心、肺阴性。实验室检查:血清淀粉酶650U。

\textbf{【X线表现】}
 主胰管扩张、迂曲,主胰管周围胰管分支扩张、粗细不均(箭头),胰管内未见明显充盈缺损。

\textbf{【X线诊断】}  慢性胰腺炎。

\textbf{【评  述】}
 慢性胰腺炎病因是多方面的。70%~80%的病例与长期酗酒有关。乙醇作用可减少胰液的分泌,使胰液中的蛋白质成分增加,在小胰管中沉积,引起填塞、慢性炎症和钙化。本例患者内镜逆行胰胆管造影检查见主胰管扩张改变,结合临床表现及实验室检查,慢性胰腺炎诊断可以确诊。慢性胰腺炎部分患者X线平片在胰腺区可见不规则斑点状钙化阴影。内镜逆行胰胆管造影(ERCP)主要表现为主胰管及其分支规则、均匀性扩张,也可表现为扩张与狭窄交替,扭曲呈串珠状改变。

主胰管内形成结石或囊肿以及纤维化,可出现杯口及截断性梗阻,往往需与胰腺癌加以鉴别,后者主胰管因癌症浸润截然中断,阻塞部断端锐利,并伴有不规则僵硬征象,邻近胰管小分支消失。而前者阻塞部断端表现圆钝而光滑,周围可见胰管小分支显影。主胰管狭窄胰腺癌可呈直线状或尖端变细如鼠尾状明显僵直。胰腺炎狭窄可为局限性或为范围较长,但边缘一般比较光滑,狭窄部与扩张部移行性过度,周围胰管分支可见,两者征象表现不同应能鉴别。但实践中胰腺炎特别是局限性胰腺炎与胰腺癌的鉴别仍有许多困难,临床工作中应采用影像学综合检查明确诊断。近年来由于CT、MRI检查的运用,诊断符合率明显增加。

\subsection{胰头癌}

\begin{figure}[!htbp]
 \centering
 \includegraphics{./images/Image00327.jpg}
 \captionsetup{justification=centering}
 \caption{胰头癌}
 \label{fig5-9-2}
  \end{figure} 

\textbf{【病史摘要】}
 女性,55岁。上腹部疼痛不适伴黄疸近2个月,恶心、呕吐,食欲减退。体格检查:体瘦,皮肤、巩膜黄染,上腹部压痛,未扪及包块,肝、脾肋下未及,心、肺阴性。

\textbf{【X线表现】}
 上消化道钡餐造影示:胃窦、十二指肠球部大弯侧、十二指肠肠曲内侧缘粘膜皱襞紊乱、固定,呈锯齿状改变。

\textbf{【X线诊断】}  胰头癌侵犯十二指肠、胃窦。

\textbf{【评  述】}
 本例患者经手术治疗病理证实为胰头癌,侵及十二指肠及胃窦。胰腺癌是胰腺最常见的肿瘤。多发生于40岁以上的中老年人。临床表现主要为腹部胀痛不适、胃纳减退、体重减轻、黄疸和腰背部疼痛。胰腺癌发生于胰头部最多,占60%~70%;胰体癌其次,胰尾癌更次之。胰头癌因常常早期侵犯胆总管下端、引起梗阻性黄疸而发现较早。

X线平片检查不能显示胰腺,故没有价值。胃肠道钡餐造影检查在胰头癌肿块较大、侵犯十二指肠时做低张十二指肠钡剂造影检查,可见十二指肠内缘反3字形压迹,并有内缘肠粘膜破坏。胰体、胰尾癌进展期可侵犯十二指肠水平段,致局限性肠管狭窄、僵硬、粘膜破坏、钡剂通过受阻。CT因其无创、分辨率高,是首选的检查方法。其主要表现为胰腺局部增大,肿块形成,增强扫描时肿块密度增加不明显,胰管扩张,胰头癌时胆总管扩张呈所谓双管征。MRI检查除能横断位成像外,还能做MRCP检查,有其独特的价值。内镜逆行胰胆管造影是显示胰管影像的可靠方法,故诊断胰头癌相当有价值,胰头癌内镜逆行胰胆管造影典型改变以主胰管的狭窄阻塞;断端僵直、锐利;边缘受压、远端扩张为主要表现。

胰头癌侵犯胆总管特别是胆总管下端需注意与肝胰壶腹癌鉴别。肝胰壶腹癌多见息肉型,乳头状息肉突入胆管腔内或侵犯一侧管壁使管腔变形与胰头癌围管浸润的特点,X线征象不同,有时两者表现也极为相似,影像学检查难以区分,必要时应做内镜检查及临床综合分析判断。

\subsection{壶腹癌}

\begin{figure}[!htbp]
 \centering
 \includegraphics{./images/Image00328.jpg}
 \captionsetup{justification=centering}
 \caption{壶腹癌}
 \label{fig5-9-3}
  \end{figure} 

\textbf{【病史摘要】}
 男性,61岁。上腹部不适1年,近1周皮肤黄染,恶心、呕吐、食欲减退。体格检查:上腹部轻压痛,未扪及明显包块,肝右侧肋下触及,脾脏不大,心、肺阴性。

\textbf{【X线表现】}
 上消化道钡餐造影示:十二指肠下曲乳头部可见2cm×2cm大小的隆起性病变,边缘比较光滑,内侧缘肠壁稍显僵直。

\textbf{【X线诊断】}  壶腹癌(肿瘤型)。

\textbf{【评  述】}
 本例患者经手术治疗,病理证实为壶腹部低分化腺癌,侵及肠壁全层。壶腹癌为腺癌,位置在胰管与胆总管接合处,称为乏特氏乳头(papilla
of vater)。其发生率占所有胆管癌的8%,占所有壶腹周围癌的10%。

低张十二指肠钡餐造影X线表现为:①直接征象:肿瘤型壶腹癌壶腹的正常形态消失,代之以局限性不规则的充盈缺损,当浸润粘膜时,可见周围横行皱襞中断、破坏。溃疡型壶腹癌可见壶腹部边缘不规则的局限性充盈缺损,其内可见形态不规则的溃疡形成,多伴有粘膜皱襞的中断、破坏。②间接征象:可见因胆总管扩张或胆囊扩大,在十二指肠上部可见光滑的压迹。内镜逆行胰胆管造影(ERCP)对于壶腹部癌有明显诊断作用。其主要表现为胆总管末端可见局限性不规则充盈缺损,与低张十二指肠造影所见的充盈缺损部位一致,其上部胆总管扩张。

\protect\hypertarget{text00011.html}{}{}


\chapter{特殊人群用药}

\section{妊娠期和哺乳期妇女用药}

\subsection{妊娠期临床用药}

妊娠期由于母体变化、胎儿胎盘的存在及激素的影响,药物代谢和转运与非妊娠时期有很大差别。在全妊娠过程中,母体、胎盘、胎儿三者相互关联组成一个生物学、药物代谢动力学的组合单位。除极少数药物(例如胰岛素、肝素)不通过胎盘到胎儿,大多数药物均能通过胎盘进入胎儿体内。因此,孕妇用药必须了解药物的药代动力学,了解药物经胎盘到胎儿体内对胎儿及新生儿的药理作用,选择安全有效的药物,适时、适量地用药。

\subsubsection{药物在胎盘的转运与代谢}

胎盘由羊膜、属于子体部分的绒毛膜和属于母体部分的底蜕膜构成,将母血与胎儿血分开,称“胎盘屏障”。胎盘通透性与一般的血管生物膜相似,相当多的药物能够通过“胎盘屏障”进入胎儿体内。感染、缺氧常能破坏“胎盘屏障”,能使正常情况下不易通过“胎盘屏障”的抗生素容易通过。
\paragraph{影响药物通过胎盘的因素}

药物多以被动转运方式经胎盘转运,其速度受以下因素影响。①药物脂溶性高低。脂溶性药物,如安替比林及硫喷妥钠,能很快地以扩散方式通过胎盘。②药物分子的大小。较小分子量药物比大分子量药物扩散速度快。③药物离子化程度。④与蛋白结合能力。药物与蛋白质结合能力的高低与通过胎盘的药量成反比。⑤胎盘血流量。合并先兆子痫、糖尿病等全身性疾病的孕妇,麻醉或脐带受压迫时引起子宫胎盘血流量的改变,也可以使胎盘输送功能受到不同程度的影响,减缓药物转运。
\paragraph{药物在胎盘的代谢}

有些药物需要在胎盘经过代谢转化,才能成为容易输送的物质。胎盘有无数有活力的酶系统,具有生物合成及降解药物的功能。有些药物通过胎盘代谢降低活性,有些药物则增加活性。如天然或人工合成的肾上腺皮质激素,皮质醇及泼尼松通过胎盘转化为失活的11-酮衍化物;地塞米松通过胎盘则不需要经过代谢就能进入胎儿体内。因此,为了治疗孕妇疾病可用泼尼松,治疗胎儿疾病宜应用地塞米松。胎盘能代谢的仅限于几类酶所作用的物质,主要承担甾体类及多环碳氢化合物的代谢。

\subsubsection{母体药代动力学}
\paragraph{药物吸收}

妊娠期因孕激素影响胃肠系统的张力及活动力减弱,胃酸分泌减少,使口服药物的吸收延缓,达峰时间延长。但难溶性药物(如地高辛)因药物通过肠道的时间延长而生物利用度提高。

妊娠妇女由于肺潮气量和每分通气量明显增加,心排出量和肺血流量也增加,可使呼吸道吸入给药经肺泡摄取的药量增加。在妊娠妇女吸入麻醉时麻醉药的剂量通常应减少。
\paragraph{药物分布作用}

药物吸收后进入较非孕期增多的血浆、体液及脂肪组织中,使药物的分布容积增大,血药浓度低于非妊娠期。
\paragraph{药物与蛋白结合}

妊娠期虽然生成白蛋白的速度加快,但因血浆容积增加,形成生理性血浆蛋白低下。同时妊娠期很多蛋白结合部位被内泌素等物质所占据,所以使妊娠期药物蛋白结合能力下降,游离药物增多,药效和不良反应增强。
\paragraph{肝的代谢作用}

肝微粒体酶降解的药物可能减少,妊娠期高雌激素水平使胆汁在肝脏郁积,药物从胆汁排出减慢,从而使药物在肝脏清除减慢。
\paragraph{药物排出}

肾血流量及肾小球滤过率均增加,肾排泄药物或代谢产物加快,使主要以原形从尿中排出的药物消除加快,血药浓度不同程度降低(妊高征除外)。晚期妊娠期仰卧位时肾血流量减少而使由肾排出的药物作用延长,孕妇可采用侧卧位以促进药物的消除。

\subsubsection{药物在胎儿体内的转运与代谢}

胎儿体内的药物大部分经胎盘转运而来,也有少量药物经羊膜转运进入羊水中,而被胎儿吞饮经胃肠道吸收,或直接经皮肤吸收。
\paragraph{肝脏中的代谢}

药物通过胎盘经脐静脉进入胎儿血循环中。胎儿肝脏中酶的水平为成年人的30%~50%,胎儿对药物代谢能力较成年人低,所以胎儿体内药物浓度较母体高。因胎儿肝细胞缺乏催化葡萄糖醛酸苷类生成的酶,对药物解毒能力很差,如巴比妥、水杨酸类和激素等,易在胎儿体内达到毒性浓度。
\paragraph{肝外的代谢}

胎儿肝脏以外的代谢部位为肾上腺,胎儿肾上腺有很高活性的细胞色素P-450,在胎儿肾上腺内代谢的酶作用物质可能与肝脏是相同的。
\paragraph{排泄}

胎儿的肾小球滤过率甚低,肾排泄药物功能极差。许多药物在胎儿体内排泄缓慢,容易造成蓄积,如氯霉素、四环素等药物在胎儿体内排泄速度较母体明显减慢。胎儿进行药物消除的主要方式是将药物或其代谢物经胎盘返运回母体,由母体消除。

\subsubsection{胎儿治疗学}

胎儿治疗学指妊娠期孕妇用药,其目的不为治疗孕妇,而是为了给胎儿用药。胎儿治疗学所选用药物应注意其药代动力学,必须是经胎盘转运到胎儿,未经胎盘代谢,保持原有药效作用。已证实有效的治疗药物,如预计要早产的孕妇,妊娠期用肾上腺皮质类固醇促使胎儿肺提前成熟,选用肾上腺皮质类固醇时用地塞米松而不用泼尼松。

\subsubsection{妊娠期合理用药的条件}

鉴于许多药物可以通过胎盘,故在用药前应考虑以下几点。

(1)采用疗效肯定、不良反应小且对于药物代谢有清楚说明的药物,避免使用尚难确定有无不良影响的新药。

(2)已证明药物对灵长目动物胚胎是无害的。但没有任何一种药物对胎儿的发育是绝对安全的。

(3)用药时需清楚地了解妊娠周期。因为很难确定何时是胚胎器官形成的最终时刻,所以用药最好能在妊娠足4个月以后开始,在怀孕的前3个月内应避免应用任何药物。

(4)用药需有明确指征。用可能对胎儿有影响的药物时,要权衡利弊后给药,只有药物对母亲的益处多于对胎儿的危险时才考虑在孕期用药。

\subsubsection{FDA颁布的药物对妊娠的危害等级标准}

(1)A级:在有对照组的研究中,妊娠3个月的妇女未见到对胎儿有危害的迹象(并也没有在其后的6个月有危害性的证据),可能对胎儿的影响甚微。

(2)B级:在动物繁殖性研究中(并未进行孕妇的对照研究),未见到对胎儿的不良影响。在动物繁殖性研究中发现有不良反应,但这些不良反应并未在妊娠3个月的妇女得到证实(也没有对其后6个月的危害性证据)。

(3)C级:动物研究证明对胎儿有危害性(致畸或杀死胚胎),但并未对对照组妇女进行研究,或没有对妇女和动物平行地进行研究。本类药物只有在权衡了对孕妇的好处大于对胎儿的危害后方可应用。

(4)D级:对胎儿的危害性有明确的证据,尽管有危害性,但孕妇用药后有绝对好处。例如孕妇受到死亡的威胁或患有严重疾病,应用其他药物虽然安全但无效,因此需要用此类药物。

(5)X级:在动物或人的研究中均表明它可造成胎儿异常,或根据经验认为对人或动物是有危害性的,给孕妇应用这类药显然无益。本类药物禁用于妊娠或即将妊娠的患者。

\subsection{哺乳期临床用药}

大部分药物均能从乳汁排出并能测出药物浓度,一般药物由乳汁排出的浓度低,不超过母体1d内药量的1%。如果哺乳期需要用药,而且是一种比较安全的药,应在婴儿哺乳后(即下次哺乳前3~4h)用药。个别药物在乳汁中可达到较高浓度,如甲硝唑、异烟肼、红霉素及磺胺类等药物,它们在乳汁中的浓度可达到乳母血药浓度的50%。有时也可利用药物进入乳汁来治疗乳儿疾病。如用苯海拉明治疗婴儿皮肤过敏性疾患时,可让母亲服用常用量(25~50mg),通过哺乳,乳儿可获得治疗量的药物。

\section{老年人用药}

老年人由于年龄的增长,其生理功能处于逐渐衰退的状况,肌体对于药物的吸收、生物转化和排泄功能等各项指标都在下降,对药物处置能力及药物的反应性相应降低。在用药过程中由于多种疾病的存在使药物的体内过程复杂化,且多种疾病的并存往往需要同时使用多种药物治疗,由此产生的药物相互作用不仅影响老年人的药物治疗效果,同时药物不良反应所带来的用药风险性也随之增加。

\subsection{老年人药代动力学改变}

\subsubsection{吸收}

老年人胃肠吸收功能减退,药物吸收减少。但由于胃肠蠕动减慢,药物在胃肠中停留时间及与肠道吸收表面接触时间均延长,故对大多数药物(被动转运吸收的药物)总吸收的影响不明显,老年人和成年人相比无明显差异。但对靠主动转运来吸收的药物(如铁、木糖、钙以及维生素B{1}
、B{2} 、B{12}
、C等),由于老年人吸收这些药物所需的酶和糖蛋白等载体分泌减少,故吸收机能减弱。由于药物在胃肠内滞留时间延长,对胃肠道刺激增加,胃肠道反应增加。

\subsubsection{分布}

老年人机体组成成分发生改变,细胞内液减少,身体总水量减少,脂肪组织增加。故水溶性药物分布容积减少,血药浓度增加,如吗啡、乙醇、水杨酸盐、青霉素等;脂溶性药物分布容积增大,作用持续较久,半衰期延长,易在体内蓄积中毒。如老年人使用利多卡因时毒性反应明显增加,70岁以上者发病率为80%。

老年人血浆白蛋白含量减少,病情严重或极度虚弱的老年人下降尤为明显。应用血浆白蛋白结合率高的药物时,血中游离型药物浓度增大,易出现不良反应。如华法林,老年人用成人剂量时不良反应大,有引起出血的危险。此类药物还有普萘洛尔、苯妥英钠、安定、保泰松、地高辛和水杨酸盐,用时应注意减量。

\subsubsection{代谢}

老年人肝细胞减少,肝微粒酶的活性降低,肝血流量减少,使代谢能力下降,药物代谢减慢,造成药物蓄积,引起不良反应。对肝清除率高、首过效应明显的药物影响尤为显著。如老年人服用利多卡因、咖啡因、氨基比林、普萘洛尔等,要注意减少用量,或延长服药的间隔时间。

\subsubsection{排泄}

老龄所致的最大药代动力学改变在于药物的排泄,是老年人发生药物中毒反应的最重要因素。人的年龄达到40岁后,肾小球滤过率和肾小管排泄能力按每年1%的速度降低。因此,老年人药物清除率降低,即使无肾脏疾病,使用主要经肾排泄的药物时,易在体内蓄积而造成中毒。如地高辛、氨基糖苷类抗生素、青霉素G、苯巴比妥、西咪替丁及磺酰脲类降糖药等,都可因肾功能减退而排泄减少,半衰期显著延长,并有蓄积中毒的危险,均应相应减少用量或延长给药间隔时间。

\subsection{与增龄相关的系统改变}

\subsubsection{药物的相互作用}

老年人多病,常多药并用,药物的相互作用不仅影响老年人的药物治疗效果,同时使不良反应发生率上升,用药风险性随之增加。

\subsubsection{疾病因素}

老年人某些疾病的发病率亦迅速增加,使药物在体内过程复杂化。

(1)神经系统:衰老时中枢神经有某种病理变化的缓慢发展,对用药的影响表现在:①因记忆力差,引起服药的差错增多,对需要有稳态血药浓度的药物易因漏服而出现症状或因过量而出现不良反应;②应慎用对神经有毒性的药物,以防毒性叠加;③对药物的反应性有变化,如服用地西泮有引起脑功能失调的报道。

(2)心血管系统:老年人应激时调节最大心律的能力下降;平均收缩压较高,对血压调节功能降低,易出现体位性低血压。应慎用降压药和利尿药,避免引起体位性低血压;应注意控制甲状腺功能亢进、感染(特别是肺部)等疾病和输液用量,避免加重充血性心力衰竭。

(3)肾脏:主要由肾脏消除的药物应调整剂量,同时应关注体液和电解质平衡的紊乱。

(4)消化系统:肝脏清除药物减慢,主要经肝脏消除的药物必要时需调整剂量;慎用易引起便秘的药物。

(5)血液系统:造血组织的总量有所减少,但血液成分的变化不明显。对用药的影响主要为慎用有骨髓抑制不良反应的药物。

\subsection{老年人在临床治疗中需特别注意的常用药物}

\subsubsection{抗菌药物}

(1)青霉素类:主要经肾清除,老年人肾功能减退引起其消除半衰期延长,血药浓度增高,易出现神经精神症状,如幻觉、抽搐、昏睡、知觉障碍等。当控制感染需较大剂量青霉素类时,必须减少剂量或延长给药间隔时间。肌酐清除率可以作为其可靠的衡量指标。老年人处理电解质平衡的能力低,要注意避免处方含钠青霉素类而致钠过多,而处方羧苄西林或替卡西林时应注意有无血钾过低。

(2)头孢菌素类:所有头孢菌素都会抑制肠道菌群产生维生素K,因此具有潜在的致出血作用。服用阿司匹林、华法林等抗凝药物的老年人在给予头孢菌素类药物抗感染时,尤其需密切监测凝血酶原时间的变化,以免发生出血等严重不良反应。

(3)氨基糖苷类:均有不可逆的耳毒性和不同程度的肾毒性,肌酐清除率降低,使药物排泄受到一定限制;对耳毒药物更为敏感,更易发生上述毒性反应。65岁以上老年人应慎用此类药物,临床上对确需使用氨基糖苷类药物的老年患者应考虑采用每日1次的给药方案,以减小其耳肾毒性。同时注意避免与呋塞米、依他尼酸、顺铂等其他耳、肾毒性药物联合应用。

(4)喹诺酮类:该类药物具有脂溶性,脑脊髓中浓度高,并抑制脑内抑制性递质γ-氨基丁酸与其受体的结合,从而增加中枢神经系统的兴奋性。老年人存在不同程度的脑萎缩或脑动脉硬化,且肾清除药物的能力降低,因此老年人静脉滴注喹诺酮类药物,引起精神紊乱或中枢神经系统兴奋等不良反应的发生率较年轻人高。

\subsubsection{地高辛}

地高辛是临床上治疗充血性心力衰竭的常用药物,但治疗窗窄,中毒反应严重。地高辛中毒的发生率随年龄增加而增高,因此,老年人使用地高辛时,需监测地高辛血药浓度,且老年人的地高辛血药浓度的治疗范围可适当降低(<2.0ng/mL)。

\subsubsection{镇静催眠药}

老年人感觉较为迟钝,智力反应减低,应用镇静催眠药更易发生不良反应。老年人使用巴比妥类药物会发生兴奋激动,不宜常规应用。老年人对地西泮的中枢抑制作用比年轻人更敏感,应用时需谨慎,给药的时间间隔要加长。

\subsubsection{氨茶碱}

氨茶碱是慢性支气管炎和心源性哮喘患者的常用药,被肝脏的混合功能酶代谢。老年人肝功能都有不同程度的降低,半衰期因此延长。所以老年人服用氨茶碱后容易出现氨茶碱中毒,表现出烦躁、呕吐、忧郁、记忆力减退、定向力差、心律失常、血压急剧下降等现象乃至死亡。静脉注射速度过快或浓度太高可引起心悸、惊厥等严重反应。因此,对于急性心肌梗死、低血压、甲状腺功能亢进的患者禁用。老年人应用氨茶碱一定要慎重,开始用药要小剂量试用,询问氨茶碱的用药史。一旦发现有胃部不适或兴奋失眠,可用安定、复方氢氧化铝等药物来对抗或停药。氨茶碱主要通过肝药酶CYP1A2代谢,当与CYP1A2酶抑制剂(如环丙沙星等喹诺酮类抗菌药物)联合用药时,适当减少茶碱给药剂量或调整给药间隔,并密切监测茶碱血药浓度,以避免茶碱血药浓度过高而引起不良反应。

\subsubsection{HMG-CoA还原酶抑制剂(他汀类)}

他汀类药物是目前最强有力的调脂药物,肌病和横纹肌溶解是他汀类的最严重不良反应。他汀类通过CYP3A4药酶代谢(普伐他汀类除外),如果联用CYP3A4抑制剂,如大环内酯类的红霉素或克拉霉素、唑类抗真菌药(伊曲康唑、氟康唑、酮康唑等)、贝特类调脂药,可潜在地引起肌病和随后的横纹肌溶解。应控制剂量,对高龄老人慎用或减量,尽量避免与CYP3A4酶抑制剂或贝特类药物联用,如必须与CYP3A4酶抑制剂合用可选择普伐他汀。用药期间定期检测肝肾功能及血清肌酸激酶,他汀类药物也是可以安全使用的。

\subsection{老年人临床合理用药原则}

\subsubsection{不应当随意加服药物}

如因情绪不稳定、过度紧张、过度疲劳、睡前用脑过度而影响睡眠,出现失眠的情况,可以通过改变生活方式、心理慰藉来改善睡眠障碍,不应滥用安眠药。

\subsubsection{减少用药数量和剂量}

《中国药典》规定60岁以上老年人用药剂量为成年人的3/4,中枢神经系统抑制药应当以成年剂量的1/2或1/3作为起始剂量。

\subsubsection{加强宣传教育}

加强对老年人合理用药的宣传教育,应告知其按医师嘱咐合理用药。

\subsubsection{必要的血药浓度监测指导用药}

一些安全范围很窄的药物,应当做血药浓度监测,以调整用药剂量或更换药物治疗,并做到给药方案的个体化,如抗心律失常药普鲁卡因胺。

\subsubsection{合理使用抗生素}

老年人抗生素应用频率很高,但由于老年人肾功能呈生理性退行性改变,药物排出减少,血药浓度易在体内增高,易产生不良反应,一般用正常治疗剂量的2/3~1/2为宜,包括头孢菌素、青霉素等β-内酰胺类,应尽量减少用毒性大的抗菌药物如万古霉素及氨基糖苷类等品种。

\subsubsection{传统医药的应用}

老年人常服用补虚扶正中成药,但也不应太过或随意服用,一般提倡应用调补药品。但所用补药剂量不应过大。老年人用中药也宜从小剂量开始,因人、因时、因地不同而辨证论治。

\subsubsection{中西药的相互作用问题}

临床配合应用中西药物的现象很多,但对其相互作用研究不多。如在中药汤药中,习惯用甘草调和诸药,如果患者同时用呋塞米等利尿药,血钾浓度可能下降;并用降糖药者,其效用可能减低。

\section{儿科用药}

从胎儿到青春期(14岁)为儿科范围,药物治疗是儿科防病治病的主要手段。因小儿正处于生长发育的重要时期,所以用药时须特别注意。小儿时期具体包括新生儿期(出生至生后28d)、婴儿期、幼儿期、学龄前期、学龄期和少年期,应注意不同年龄分期,结合儿童的具体情况,如营养状况、体质等,根据药物的性质、用药方式作调整,才能取得满意效果。给药途径取决于病情的轻重缓急、用药目的及药物本身性质。一般情况下,有小儿剂型的药物不使用成人剂量分成几等份后服用。

\subsection{小儿药代动力学特点}

小儿时期,其器官和组织均处在不断发育和成熟的过程,新生儿期尤其是一个特殊阶段。为保证用药安全,应根据小儿生理的特征及药物在体内的药动学和药效学特点合理选择药物。

\subsubsection{吸收}

新生儿胃排空时间长,通过胃肠道吸收的药物比成人慢,肠壁相对长而薄,通透性高,可使一些药物的吸收增加。各种消化液分泌量少,胃液及酶的浓度小,消化能力弱。胃酸pH值较高,对遇酸不稳定的青霉素分解少,吸收好;对弱酸性药物,由于在胃液中解离增加,吸收少。

\subsubsection{分布}

小儿的体液总量和细胞外液量较成人比例高,可影响药物的分布。新生儿体脂含量低,脂溶性药物不能充分与之结合,因而分布容积小,游离药物浓度高,易出现中毒。同时,新生儿脑组织占身体比重较大,血脑屏障发育不完善,使脂溶性药物易进入大脑,出现神经系统反应。新生儿药物血浆蛋白的结合力低于成年人,营养不良和低蛋白血症的新生儿更低,对某些药物的敏感性增加。因此,在成人被认为是安全的、很低的血浆药物浓度,对新生儿可能引起不良反应。

\subsubsection{代谢}

新生儿肝容积与体重的比例较成人大,部分酶类较成人多,使一些完全在肝脏代谢的药物血浆半衰期缩短。同时,新生儿肝内混合功能氧化酶和化合酶代谢药物的活性比成人低得多,又使很多药物代谢缓慢,血浆半衰期延长。除了代谢程度,新生儿药物代谢与成人相比还有本质上的差别,如新生儿使用茶碱,将有一部分代谢为咖啡因,还需考虑咖啡因的药理作用。所以,对新生儿用药时,应考虑品种和剂量的选择,以防药物蓄积中毒。

\subsubsection{排泄}

新生儿的肾小球滤过功能和肾小管的分泌功能均不足,对主要在肾脏排泄的药物清除慢,引起中毒的危险,药物剂量需进行调整。

\subsection{小儿用药的特殊反应}

\subsubsection{药物敏感性和耐受性改变}

新生儿对酸、碱和水电解质平衡的调节能力差,过量水杨酸类药物可致酸中毒,利尿剂可致缺钠或缺钾,氯丙嗪易引起麻痹性肠梗死,氯霉素致灰婴综合征和再生障碍性贫血,长期使用皮质激素易引起胰腺炎等。

儿童对铁盐耐受性很差,成年人可耐受50g之多,而婴儿口服1g即可引起严重中毒反应,2g以上可致死,原因是可溶性铁盐引起婴幼儿肠道黏膜的损伤,甚至严重呕吐、腹泻、胃肠出血导致失水、休克。

小儿应用解热镇痛药后,可因体温骤降、出汗引起虚脱。应注意的是,阿司匹林、吲哚美辛可收缩血管,使新生儿动脉导管迅速关闭,致肺动脉高压,使新生儿病死率增加。解热镇痛药之间有交叉过敏反应,如对阿司匹林过敏,应用吲哚美辛、萘普生等也可能过敏。所以,在用药的过程中,要密切观察,防止少数患儿因过敏致死。

儿童因苯巴比妥过敏反应较多,故很少用。使用苯妥英钠常见的不良反应是癫痫
发作频率增加,如果此时未检测血药浓度,则往往认为是剂量不足,再增加剂量则症状更显著,所以使用也相对较少。通常用丙戊酸钠,其不良反应发生率较低,但有肝毒性,2岁以下儿童在合用其他抗癫痫
药时较易发生,用药期间应注意监测肝功能。

\subsubsection{溶血反应}

主要发生在红细胞G-6-PD缺乏的新生儿,使用维生素K、磺胺类、噻嗪类、利尿类、萘啶类、呋喃唑酮等有氧化性的药物时,可使红细胞膜发生破裂,引起溶血。

已知在妊娠后期,临床应用容易引起新生儿溶血或黄疸的药物包括较大剂量的解热镇痛药,如非那西丁、阿司匹林、氨基比林、安替比林、辛克芬;抗疟疾药,如奎宁、伯氨喹等;抗微生物药物,如头孢菌素类、青霉素、新生霉素、金霉素、氯霉素、四环素;中枢神经系统抑制剂,如吩噻嗪类、地西泮、苯巴比妥、苯妥英、乙醇、氯仿、水合氯醛;洋地黄毒苷类;性激素类等。其中,可引起免疫性溶血的药物包括青霉素、头孢菌素、磺胺药、异烟肼、奎宁、甲基多巴、非那西丁等。

\subsubsection{核黄疸}

新生儿本来就有黄疸的因素,在使用一些与胆红素竞争血浆蛋白的药物时,血中游离的胆红素升高,进入脑内与基底核结合导致胆红素脑病或核黄疸。维生素K{3}
、磺胺类、新生霉素都能影响胆红素代谢,加重新生儿黄疸,故慎用。

\subsubsection{神经系统反应}

新生儿血脑屏障发育不成熟,药物易透过血脑屏障直接作用于脆弱的中枢神经系统,引起神经系统反应。如阿片类药物易引起呼吸抑制;抗组胺药、苯丙胺、氨茶碱、阿托品可致昏迷及惊厥;皮质激素易引起手足抽搐;氨基糖苷类抗生素易引起听神经损伤等。

\subsubsection{牙色素沉着}

四环素、多西环素、米诺环素等可沉积于骨组织和牙齿,引起永久性色素沉着,如牙齿发黄,四环素还可抑制骨的生长发育。故妊娠4个月后,哺乳期妇女和8岁以下儿童除眼科局部用药外,不得应用四环素。
\chapter{呼吸系统疾病}

\chapterabstract{本章主要介绍呼吸系统常见疾病(慢性阻塞性肺疾病、肺炎、硅肺、慢性肺源性心脏病、鼻咽癌、肺癌)的病理形态特点和临床病理联系,对这些常见疾病的病因及发病机制也进行了简单的介绍。要求掌握慢性支气管炎、肺气肿、肺心病的病理变化;大叶性肺炎的病理变化、病变性质及并发症;小叶性肺炎的病变性质及病变特征;肺癌分型及病理变化。熟悉大叶性肺炎的主要临床病理联系;小叶性肺炎的病因、发病机制及临床病理联系;了解慢性支气管炎、肺气肿、肺心病、大叶性肺炎的病因及发病机制;病毒性肺炎(包括SARS)和支原体性肺炎的病因、发病机制、病理变化及临床病理联系;了解硅肺的发病机制及病理变化。}

呼吸系统是通气和换气的器官。终末细支气管以上为气体传导部分,呼吸性细支气管、肺泡管、肺泡囊为换气部分。传导性气道管壁被覆纤毛柱状上皮。肺泡由I型和Ⅱ型肺泡上皮细胞覆盖。黏液-纤毛排送系统是呼吸道的主要防御功能之一。肺泡巨噬细胞又称为尘细胞,是肺内重要的防御细胞,不仅具有吞噬功能,还可摄取和处理抗原、增强淋巴细胞的免疫活性。此外,呼吸道分泌物中的溶菌酶、补体系统、干扰素和分泌型IgA等也具有增强局部免疫力的作用。

呼吸系统与外界相通,极易受外界环境中有害物质的作用诱发疾病。肺又是全身血液循环必经之处,因此许多疾病常可以并发肺部病变。一些自身免疫或代谢性全身疾病,如系统性红斑狼疮、类风湿关节炎等都可累及肺部,因而呼吸系统疾病比较常见。由于大气污染、吸烟、人口老龄化及其他因素使慢性阻塞性肺疾病、肺癌、肺部弥散性间质纤维化、慢性肺源性心脏病等的发病率、死亡率日趋增多。

\section{慢性阻塞性肺疾病}

慢性阻塞性肺疾病(chronic obstructive pulmonary
disease,COPD)是一组由多种原因引起的以持续气流受限为特征的慢性阻塞性气道疾病的总称,以呼气性呼吸困难为特征,病人常因肺功能不全、肺动脉高压等死亡。属于这组疾病的有慢性支气管炎、肺气肿、支气管扩张症和支气管哮喘等疾病,以华北及东北地区多见。因篇幅有限,在此只介绍前三种疾病。

\subsection{慢性支气管炎}

慢性支气管炎(chronic
bronchitis)是气管、支气管黏膜及周围组织的慢性非特异性炎症。临床以咳嗽、咳痰、喘息为主要症状,每年发病持续三个月,连续两年或两年以上。以老年男性多见,冬春季节高发。病情缓慢发展,易并发阻塞性肺气肿,肺动脉高压和肺源性心脏病。是严重威胁人体健康的常见疾病。

\subsubsection{病因及发病机制}

\paragraph{吸烟}
吸烟为发病的主要因素。香烟中的焦油、尼古丁和氰氢酸等可损伤呼吸道黏膜上皮细胞,导致气道净化功能下降并能刺激黏膜下感受器,使副交感神经功能亢进,引起支气管平滑肌收缩,导致气道阻力增加以及腺体分泌增多。杯状细胞增生、支气管黏膜充血水肿、黏液积聚容易诱发感染。此外,香烟烟雾还可使毒性氧自由基产生增多,诱导中性粒细胞释放各类蛋白水解酶,破坏肺弹力纤维,诱发肺气肿的发生。研究表明,吸烟者慢性支气管炎的患病率较不吸烟者高2~8倍,烟龄越长,烟量越大,患病率亦越高。

\paragraph{空气污染}
空气中二氧化硫、二氧化氮、氯气及臭氧等对气道黏膜上皮均有刺激和细胞毒作用。二氧化硅、煤尘、蔗尘、棉屑等损伤支气管黏膜,使肺清除功能遭受损害,引起细菌感染。

\paragraph{感染因素}
感染是慢性支气管炎发生和发展的重要因素之一。细菌、病毒和支原体感染为本病急性发作的主要原因。病毒感染以流感病毒、鼻病毒、腺病毒和呼吸道合胞病毒常见;细菌感染以肺炎链球菌、流感嗜血杆菌及葡萄球菌多见,常继发于病毒或支原体感染。

\paragraph{过敏因素}
喘息型慢性支气管炎患者多有过敏史,痰液中嗜酸性粒细胞数量和组胺含量和血中IgE具有增多的趋向。部分患者血清中类风湿因子阳性以及T淋巴细胞亚群分布异常等,提示过敏因素与本病的发生有关。

\paragraph{其他}
慢性支气管炎急性发作在冬季较多见。寒冷空气可刺激腺体分泌黏液增加和纤毛运动,减弱、削弱气道的防御功能,还可引起支气管平滑肌痉挛、黏膜血管收缩、局部血循环障碍。大多患者具有自主神经功能失调的现象,部分患者副交感神经功能亢进,气道反应性较正常人增高。老年人肾上腺皮质功能减退、细胞免疫功能受损、溶菌酶活性降低、营养低下、维生素A、维生素C不足等均可使气道黏膜血管通透性增加和上皮修复功能减退。遗传因素是否与慢性支气管炎的发病有关迄今尚未证实。

\subsubsection{病理变化}

\paragraph{呼吸道上皮的损伤与修复}
由于炎性渗出和黏液分泌增多,使黏膜上皮的纤毛因负荷过重而发生粘连、倒伏,甚至脱失。上皮细胞变性、坏死。病变严重或持续过久,可发生鳞状上皮化生(图\ref{fig7-1})。

\begin{figure}[!htbp]
 \centering
 \includegraphics{./images/Image00110.jpg}
 \captionsetup{justification=centering}
 \caption{慢性支气管炎(HE染色,中倍)\\ {\small 支气管黏膜上皮变性、坏死,鳞状上皮化生}}
\label{fig7-1}
  \end{figure}

\paragraph{呼吸道腺体的病变}
为支气管炎的形态学特征。表现为黏膜上皮层内杯状细胞增多;黏液腺泡增生、肥大;浆液腺泡黏液化,黏液分泌增多。

\paragraph{管壁组织的损害}
急性发作时,黏膜层及黏膜下层充血、水肿,淋巴细胞、浆细胞及中性粒细胞浸润。炎症反复发作可破坏平滑肌、弹力纤维和软骨。支气管黏膜发生溃疡,管壁结缔组织增生,管腔狭窄,管腔内黏液栓潴留致气道阻塞。局部管壁塌陷、扭曲、变形。

\subsubsection{病理临床联系}

慢支患者缓慢起病,病程长,反复急性发作而病情加重。主要症状为咳嗽、咳痰,或伴有喘息。急性加重系指咳嗽、咳痰、喘息等症状突然加重,主要原因是呼吸道感染。由于黏液的分泌增多,痰液和炎症刺激支气管黏膜而引起咳嗽、咳痰。痰呈白色泡沫状,并发感染时可呈脓性。肺部听诊可闻及干湿性啰音。由于支气管黏膜肿胀、痰液阻塞和平滑肌痉挛,可出现哮喘样发作。

\subsubsection{并发症}

慢性支气管炎反复发作,病变逐渐加重。炎症向肺泡及支气管壁周围扩展,导致细支气管周围炎,还可发生纤维闭塞性支气管炎,以后引起阻塞性肺气肿、支气管扩张症,最终导致肺源性心脏病。长期炎症刺激可引起气管和支气管黏膜上皮发生鳞状上皮化生,进而发生不典型增生,最终恶变为鳞状细胞癌。

\subsection{支气管扩张症}

支气管扩张症(bronchiectasis)是由于支气管及其周围肺组织慢性化脓性炎症和纤维化,使支气管壁的肌肉和弹性组织破坏,导致支气管变形及持久扩张。典型的症状有慢性咳嗽、咳大量脓痰和反复咯血。主要致病因素为支气管感染、阻塞和牵拉,部分有先天遗传因素。患者多有麻疹、百日咳或支气管肺炎等病史。随着人民生活的改善,麻疹、百日咳疫苗的预防接种,以及抗生素的临床应用,本病的发病率大为减少。

\subsubsection{病因}

\paragraph{感染}
感染是引起支气管扩张的最常见原因。儿童时期麻疹、百日咳、流行性感冒(某些腺病毒感染)或严重的肺部感染如肺炎克雷白杆菌、葡萄球菌、流感病毒、真菌、分枝杆菌以及支原体感染,使支气管各层组织尤其是平滑肌纤维和弹性纤维遭到破坏,黏液纤毛清除功能降低,削弱了管壁的支撑作用,可继发支气管扩张。

\paragraph{支气管阻塞}
支气管由于受肿瘤、肿大淋巴结的压迫,或因腔内异物而发生不完全阻塞,使阻塞处以下的支气管腔内压力不断增大,管壁受损、管腔扩张。支气管完全阻塞时,其所属肺泡腔内空气被吸收而萎缩,该部位支气管壁受胸腔负压的牵拉而扩张。

\paragraph{先天性和遗传性疾病}
先天性较少见,是由于先天性支气管发育不良,存在先天性缺陷或遗传性疾病,使肺的外周不能进一步发育,导致已发育支气管扩张,如支气管软骨发育不全(Williams-Camplen综合征)。有的病人支气管扩张在出生后发生,但也有先天异常因素存在,如Kartagener综合征,患者除支气管扩张外可伴有内脏异位和胰腺囊性纤维化病变。支气管扩张症也可见于Young综合征,特征为阻塞性精子缺乏,慢性鼻窦炎,反复肺部感染和支气管扩张。部分支气管扩张病人显示免疫球蛋白缺陷,易于反复细菌感染。

\subsubsection{病理变化}

肉眼观:支气管扩张多发生于肺段及段以下支气管(Ⅲ~Ⅳ级支气管及细支气管)。常见于一个肺段,也可在双侧多个肺段发生。左肺较右肺多见,特别见于左肺下叶。扩张部支气管腔明显扩大,形态可分为圆柱状、囊状两种,亦常混合存在。柱状扩张者管壁破坏较轻,随着病变发展,破坏严重,乃出现囊状扩张(图\ref{fig7-2})。严重者肺组织呈蜂窝状。扩张的管腔内充满黄绿色黏稠脓性或血性渗出物。管壁黏膜萎缩或增生、肥厚,形成纵行皱襞。

\begin{figure}[!htbp]
 \centering
 \includegraphics{./images/Image00111.jpg}
 \captionsetup{justification=centering}
 \caption{支气管扩张症\\ {\small 支气管呈圆柱状扩张,直达胸膜}}
\label{fig7-2}
  \end{figure}

镜下观:支气管呈慢性化脓性炎症,并伴有不同程度的组织破坏及管壁纤维化、瘢痕化。支气管扩张症易发生反复感染,其炎症可蔓延到邻近的肺实质,引起不同程度的肺炎、小脓肿或肺小叶不张以及慢性支气管炎的病变,久之可形成肺纤维化和阻塞性肺气肿,上述变化又加重支气管扩张。

\subsubsection{临床病理联系}

支气管扩张症的典型症状为慢性咳嗽、大量脓痰和反复咯血。咳嗽、咳痰主要是慢性炎症的刺激、黏液分泌增多及继发化脓菌感染所致。咳嗽和痰量与体位改变有关,尤其是清晨起床可咳出大量脓性痰,若有厌氧菌感染,则有臭味。咯血为小血管炎性破坏及咳嗽所致。有些患者仅表现为反复咯血,平时无咳嗽、脓痰等呼吸道症状,临床上称为“干性支气管扩张”。并发胸膜炎时,可出现胸痛。慢性重症支气管扩张症,肺组织广泛纤维化,病人可出现气急、发绀、杵状指(趾)。

\subsubsection{并发症}

支气管扩张症常见的并发症有肺脓肿、脓胸、脓气胸等。病灶内细菌经血道播散可到达远处器官,最常见的是引起脑膜炎、脑脓肿;由于抗菌药物的运用,此种情况已较少见。严重的支气管扩张致肺组织广泛纤维化,破坏肺血管床或形成支气管动脉与肺动脉分支吻合,则可导致肺动脉高压,引起肺心病。此外,在鳞状上皮化生的基础上可发生鳞状细胞癌。

\subsection{慢性阻塞性肺气肿}

慢性阻塞性肺气肿(chronic obstructive pulmonary
emphysema)是由于慢性支气管炎等引起呼吸性细支气管以远的末梢肺组织因残气量增多而呈持久性扩张,并伴有肺泡间隔破坏,以致肺组织弹性减弱,容积增大的阻塞性肺病。

\subsubsection{病因和发病机制}

肺气肿是支气管和肺疾病常见的并发症,与吸烟、空气污染、小气道感染、尘肺等关系密切,尤其是慢性阻塞性细支气管炎是引起肺气肿的重要原因。发病机制与下列因素有关:

\paragraph{阻塞性通气障碍}
慢性细支气管炎时,由于小气道的狭窄、阻塞或塌陷,导致阻塞性通气障碍,使肺泡内残气量增多。细支气管周围炎症使肺泡壁破坏、弹性减弱,末梢肺组织残气量不断增多而发生扩张,肺泡孔扩大,肺泡间隔断裂,扩张的肺泡互相融合形成气肿囊腔。此外,炎症损伤细小支气管壁软骨,细支气闭塞时,吸入的空气可经存在于细支气管和肺泡之间的Lambert孔进入闭塞远端的肺泡内(即肺泡侧流通气),而呼气时,Lambert孔闭合,空气不能排出,导致肺泡内储气量增多、肺泡内压增高。

\paragraph{弹性蛋白酶增多、活性增高}
与肺气肿发生有关的内源性蛋白酶主要是中性粒细胞和单核细胞释放的弹性蛋白酶。慢性支气管炎伴有肺感染尤其是吸烟者,肺组织内渗出的中性粒细胞和单核细胞较多,可释放多量弹性蛋白酶。此酶能降解肺组织中的弹性硬蛋白、结缔组织基质中的胶原和蛋白多糖,破坏肺泡壁结构。

\paragraph{遗传性α{1}-抗胰蛋白酶(α{1}-AT)缺乏}  α{1}
-抗胰蛋白酶由肝细胞产生,是一种分子量为45 000~56
000的糖蛋白,它能抑制蛋白酶、弹性蛋白酶、胶原酶等多种水解酶的活性。该酶缺失则增强了弹性蛋白酶的损伤作用。遗传性α{1}
-抗胰蛋白酶缺乏是引起原发性肺气肿的原因,α{1}
-抗胰蛋白酶缺乏的家族,肺气肿的发病率比一般人高15倍,主要是全小叶型肺气肿。

\subsubsection{病理变化}

\paragraph{肉眼观}
肺显著膨大,边缘钝圆,色泽灰白,表面常可见肋骨压痕,肺组织柔软而弹性差,指压后的压痕不易消退,触之捻发音增强。表面可见多个大小不一的大泡(图\ref{fig7-3})。

\begin{figure}[!htbp]
 \centering
 \includegraphics{./images/Image00112.jpg}
 \captionsetup{justification=centering}
 \caption{肺气肿}
 \label{fig7-3}
  \end{figure} 

\paragraph{镜下观}
肺泡扩张,间隔变窄,肺泡孔扩大,肺泡间隔断裂,扩张的肺泡融合成较大的囊腔。肺毛细血管床明显减少,肺小动脉内膜呈纤维性增厚。小支气管和细支气管可见慢性炎症。根据扩张部位又可分为小叶中央型、小叶周围型和全小叶型肺气肿(图\ref{fig7-4})。

(1)小叶中央型肺气肿:是临床最常见的一型。病变累及肺小叶的中央部分,呼吸性细支气管病变最明显,呈囊状扩张。而肺泡管、肺泡囊变化则不明显。

(2)全小叶型肺气肿:病变累及肺小叶的各个部位,从终末呼吸细支气管直至肺小叶和肺泡均呈弥漫性扩张,遍布于肺小叶内。如果肺泡间隔破坏较严重,气肿囊腔可融合成直径超过1
cm的大囊泡,形成大泡性肺气肿。

(3)小叶周围型肺气肿:病变主要累及肺小叶远端部位的肺泡囊,而近端部位的呼吸性细支气管和肺泡管基本正常。常合并有小叶中央型和全小叶型肺气肿。

肺气肿的气肿囊泡为扩张的呼吸性细支气管,在近端囊壁上常可见呼吸上皮(柱状或低柱状上皮)及平滑肌束的残迹。全小叶型肺气肿的气肿囊泡主要是扩张变圆的肺泡管和肺泡囊,有时还可见到囊泡壁上残留的平滑肌束片断,在较大的气肿囊腔内有时还可见含有小血管的悬梁。

\begin{figure}[!htbp]
 \centering
 \includegraphics{./images/Image00113.jpg}
 \captionsetup{justification=centering}
 \caption{慢性阻塞性肺气肿类型模式图}
 \label{fig7-4}
  \end{figure} 

\subsubsection{临床病理联系及转归}

肺气肿患者的主要症状是气短,轻者仅在体力劳动时发生,随着气肿程度加重,气短逐渐明显,甚至休息时也出现呼吸困难,并常感胸闷。每当合并呼吸道感染时,症状加重,并可出现缺氧、酸中毒等一系列症状。患者胸廓前后径增大,呈桶状胸。胸廓呼吸运动减弱。叩诊呈过清音,心浊音界缩小或消失,肝浊音界下降。语音震颤减弱。听诊时呼吸音减弱,呼气延长,用力呼吸时两肺底部可闻及湿啰音和散在的干啰音。剑突下心音增强,肺动脉瓣第二音亢进。

肺气肿严重时可引起肺源性心脏病及衰竭。肺大泡破裂后引起自发性气胸,并可导致大面积肺萎陷。由于外呼吸功能严重障碍,导致呼吸衰竭及肺性脑病。呼吸衰竭时发生的低氧血症和高碳酸血症会引起各系统的代谢功能严重紊乱。中枢神经系统对缺氧最为敏感,随着缺氧程度的加重,可出现一系列中枢神经系统功能障碍,由开始的大脑皮层兴奋性增高而后转入抑制状态。病人表现由烦躁不安、视力和智力的轻度减退,逐渐发展为定向和记忆障碍,精神错乱、嗜睡、惊厥以至意识丧失。

\section{肺源性心脏病}

肺源性心脏病(cor
pulmonale,)简称肺心病,主要由于支气管肺组织、胸廓或肺动脉血管病变所致肺循环阻力增加,肺动脉高压,导致右心室肥厚、扩张而引起的心脏病。根据起病缓急和病程长短,可分为急性和慢性两类。临床上后者多见,除原有肺、胸疾病的各种症状和体征外,主要是逐步出现肺、心功能衰竭以及其他器官损害的征象。

\subsection{病因}

\paragraph{支气管、肺疾病}
以慢支并发阻塞性肺气肿最为多见,占80%~90%,其次为支气管哮喘、支气管扩张、重症肺结核、尘肺、慢性弥漫性肺间质纤维化、结节病、过敏性肺泡炎、嗜酸性肉芽肿等。

\paragraph{胸廓运动障碍性疾病}
较少见,严重的脊椎后、侧凸,脊椎结核,类风湿关节炎,胸膜广泛粘连及胸廓形成术后造成的严重胸廓或脊椎畸形,以及神经肌肉疾患如脊髓灰质炎,可引起胸廓活动受限、肺受压、支气管扭曲或变形,导致肺功能受限,气道引流不畅,肺部反复感染,并发肺气肿,或纤维化、缺氧、肺血管收缩、狭窄,使阻力增加,肺动脉高压,发展成肺心病。

\paragraph{肺血管疾病}
甚少见。累及肺动脉的过敏性肉芽肿病,广泛或反复发生的多发性肺小动脉栓塞及肺小动脉炎,以及原因不明的原发性肺动脉高压症,均可使肺小动脉狭窄、阻塞,引起肺动脉血管阻力增加、肺动脉高压和右心室负荷加重,发展成肺心病。

\subsection{发病机制}

上述任何因素引起肺心病的关键环节都是肺动脉高压,其发病机制如下:

\paragraph{肺毛细血管床显著减少}
慢性肺气肿或肺广泛纤维化,使肺泡壁毛细血管受压、扭曲变形,甚至管腔狭窄或闭锁,肺毛细血管床总横断面积减少,从而肺循环阻力增加,肺动脉压升高。

\paragraph{肺内血管分流}
在慢性肺部疾病时,肺泡壁毛细血管受压闭塞,或因肺组织广泛纤维化,肺循环的正常途径受阻,使肺动脉和支气管动脉之间的吻合支开放,压力高的支气管动脉血流入压力低的肺动脉系统,引起肺动脉压升高。

\paragraph{肺通气、换气功能障碍}
严重的慢性肺部疾病可引起肺通气、换气功能障碍,导致缺氧、高碳酸血症和呼吸性酸中毒,使小动脉痉挛收缩,并刺激血管平滑肌细胞,使之增生、肥大,血管壁增厚,引起肺循环阻力增大,肺动脉压升高。慢性缺氧还可产生继发性红细胞增多、血液黏稠度增加,血流阻力随之增高。缺氧还引起肾动脉收缩,肾血流量减少,醛固酮分泌增加,从而引起水、钠潴留,血容量增多,更使肺动脉压升高。临床上缺氧和高碳酸血症得到纠正后,肺动脉压可明显降低,部分病人甚至可恢复到正常范围。

\subsection{病理变化}

肺心病是多种慢性肺部疾病的晚期并发症,形成肺血管改变和右心室肥厚扩张。因此,肺心病的病理变化包括晚期肺部疾病、肺血管和心脏三种病变。肺部疾病详见有关章节,此处仅叙述肺血管和心脏病变。

\subsubsection{肺血管病变}

肺小动脉及其分支的病变在肺动脉高压的形成中起着重要作用,表现为:①肺小动脉中膜平滑肌增生肥大,细胞外基质合成增多,致肺小动脉中膜肥厚,使肺小动脉管壁增厚、变硬,管腔狭窄,肺循环阻力增加。②无肌细动脉肌化:持续缺氧可以刺激肺泡毛细血管前的无肌细动脉管壁平滑肌增生。③肺小、细动脉内膜下胶原纤维增生,并出现纵行肌束。上述改变使肺小、细动脉管壁增厚,管腔狭窄(图\ref{fig7-5})。④肺小动脉炎:若肺部炎症累及邻近的肺小动脉,引起血管的急、慢性炎,致使病变处血管管壁增厚、管腔狭窄或纤维化。⑤肺泡壁毛细血管床数量显著减少。

\begin{figure}[!htbp]
 \centering
 \includegraphics{./images/Image00114.jpg}
 \captionsetup{justification=centering}
 \caption{慢性肺源性心脏病肺小动脉硬化\\ {\small 镜下见肺细动脉管壁增厚,管腔狭窄(HE染色,中倍)}}
\label{fig7-5}
  \end{figure}

\subsubsection{心脏病变}

肉眼观:心脏体积明显增大,重量增加,平均为326 g,最重者可达785
g。右心室壁显著肥厚,后期心腔扩张。心尖钝圆,肺动脉圆锥隆起(图\ref{fig7-6}A),肺动脉瓣下2
cm处右心室肌壁厚度超过0.5 cm。

镜下观:右心室心肌纤维肥大(图\ref{fig7-6}B),可见肌浆溶解、变性、坏死,间质水肿和结缔组织增生。

\begin{figure}[!htbp]
 \centering
 \includegraphics{./images/Image00115.jpg}
 \captionsetup{justification=centering}
 \caption{慢性肺源性心脏病}
 \label{fig7-6}
  \end{figure} 

\subsection{临床病理联系}

肺心病进展缓慢,开始主要表现为原来肺部疾病的症状。随着病变加重,肺动脉压升高,右心负荷增加,患者出现心慌、气急、发绀及下肢水肿、肝肿大等右心力衰竭的症状和体征。重症肺心病,由于呼吸功能衰竭所致缺氧、二氧化碳潴留可引起肺性脑病,患者表现为头痛、烦躁、精神错乱、意识不清和昏迷等,是肺心病患者重要致死原因。

\section{肺炎}

肺炎(pneumonia)通常是指肺的急性渗出性炎症,为呼吸系统的多发病、常见病。在我国各种致死病因中,肺炎占第5位。肺炎按病因可分为感染性肺炎,如细菌性肺炎、病毒性肺炎、支原体肺炎、真菌性肺炎及其他病原体包括立克次体、肺炎衣原体、寄生虫(如弓形体、卡氏肺孢子虫、肺包虫、肺吸虫)等引起的肺炎等。理化因素引起的肺炎,如放射性、吸入性、类脂性肺炎以及变态反应性(如过敏性和风湿性)肺炎等。由于致病因子和机体反应性的不同,炎症发生的部位、累及范围和病变性质也往往不同。炎症发生于肺泡内者称肺泡性肺炎(大多数肺炎为肺泡性),累及肺间质者称间质性肺炎。病变范围以肺小叶为单位者称小叶性肺炎,累及肺段者称节段性肺炎,波及整个或多个大叶者称大叶性肺炎。按病变性质可分为浆液性、纤维素性、化脓性、出血性、干酪性、肉芽肿性或机化性肺炎等不同类型。

\subsection{大叶性肺炎}

大叶性肺炎(lobar
pneumonia)主要是由肺炎链球菌感染引起的肺组织的急性纤维素性渗出性炎症。病变起始于肺泡,并迅速扩展至整个或多个大叶。多见于青壮年,好发于冬、春季节。临床表现为骤然起病、寒战高热、胸痛、咳嗽、咳铁锈色痰、呼吸困难,并有肺实变体征及白细胞增高等。典型病程7~10天。

\subsubsection{病因和发病机制}

95%以上的大叶性肺炎由肺炎链球菌引起,尤以Ⅲ型者毒力最强。此外,肺炎杆菌、金黄色葡萄球菌、溶血性链球菌、流感嗜血杆菌也可引起。本病主要经呼吸道感染,受寒、疲劳、醉酒、感冒、麻醉、糖尿病、肝肾疾病等均为其诱因。此时,呼吸道防御功能被削弱,机体抵抗力降低,易发生细菌感染。细菌侵入肺泡后在其中繁殖,引起肺泡壁水肿,继而白细胞渗出和红细胞漏出,特别是形成的浆液性渗出物有利于细菌繁殖,并使细菌通过肺泡间孔(Cohn孔)或呼吸性细支气管迅速向邻近肺组织蔓延,从而波及肺段或整个大叶。在肺大叶之间的蔓延则系带菌渗出液经叶支气管播散所致。

\subsubsection{病理变化及与临床联系}

病变一般多见于左肺下叶,也可同时或先后发生于两个以上肺叶。由于毛细血管通透性增高,大量纤维蛋白原渗出于肺泡,使肺组织大面积广泛实变。按自然病程可分为四期。

\paragraph{充血水肿期}
发病后1~2天,肺叶充血、水肿,暗红色,切开时有血性浆液自切面流出。镜下观:肺泡壁毛细血管扩张充血,肺泡腔内有大量浆液性渗出物,混有少数红细胞、中性粒细胞和巨噬细胞,并含有大量细菌(图\ref{fig7-7})。

临床上出现高热、寒战、白细胞增多等毒血症症状,听诊可闻及湿性啰音,X线检查病变处呈现淡薄、均匀的阴影,渗出物中可检出大量肺炎双球菌。

\begin{figure}[!htbp]
 \centering
 \includegraphics{./images/Image00116.jpg}
 \captionsetup{justification=centering}
 \caption{大叶性肺炎(充血水肿期)(HE染色,中倍)\\ {\small 肺泡壁毛细血管高度充血水肿,肺泡腔内充满浆液,其中混有少量红细胞和白细胞}}
\label{fig7-7}
  \end{figure}

\paragraph{红色肝样变期}
1~2天后,即有大量纤维蛋白原渗出。肉眼观:肺叶肿大,颜色暗红,质实如肝,切面颗粒状,为充塞于肺泡腔内的纤维素性渗出物突出于切面所致。病变肺叶的胸膜面常有纤维素性渗出物覆盖。镜下观:肺泡壁毛细血管显著充血,肺泡腔内充满混有红细胞、中性粒细胞、巨噬细胞的纤维素性渗出物,纤维素可穿过肺泡间孔与相邻肺泡中的纤维素相互连接成网状,有利于中性粒细胞和巨噬细胞的吞噬作用,防止细菌扩散(图\ref{fig7-8})。

\begin{figure}[!htbp]
 \centering
 \includegraphics{./images/Image00117.jpg}
 \captionsetup{justification=centering}
 \caption{大叶性肺炎红色肝样变期(HE染色,高倍)\\ {\small 肺泡壁毛细血管高度扩张、充血,肺泡腔中含大量纤维素、红细胞及少量白细胞}}
\label{fig7-8}
  \end{figure}

临床上,病人高热稽留不退,呼吸急促。由于红细胞破坏与崩解,被巨噬细胞吞噬,形成含铁血黄素,经痰液排出,使痰呈铁锈色。由于病变累及胸膜,病人常感胸痛。若病变范围较大,实变区的大量静脉血未能氧合便流入左心,引起血氧分压和氧饱和度降低,病人可出现发绀、呼吸困难等缺氧表现。胸部叩诊呈浊音,听诊闻及管性呼吸音和胸膜摩擦音,X线检查见大片致密阴影。渗出物中仍可检出肺炎双球菌。

\paragraph{灰色肝样变期}
在发病4~6天后,肺泡腔内纤维素性渗出物及中性粒细胞继续增加。肉眼观:病变肺叶质实如肝,明显肿胀,重量增加,呈灰白色。如血管损伤较重、出血较多,外观可呈红色。胸膜面仍有纤维素性渗出物覆盖。镜下观:肺泡腔内充满大量纤维素及中性粒细胞,红细胞大部分破坏溶解,被咳出或被吸收。由于肺泡腔内渗出物的压力,肺泡壁毛细血管受压而处于贫血状态(图\ref{fig7-9})。

\begin{figure}[!htbp]
 \centering
 \includegraphics{./images/Image00118.jpg}
 \captionsetup{justification=centering}
 \caption{大叶性肺炎灰色肝样变期}
 \label{fig7-9}
  \end{figure} 

临床上,痰呈脓性,因肺泡壁毛细血管受压,流经病变区的血流量减少,肺静脉血氧合不足的情况反而减轻,故缺氧状况有所改善。听诊及X线检查所见与红色肝样变期表现基本相同。由于特异性抗体的产生或吞噬作用的加强,此期肺炎双球菌大多已被消灭,故不易检出。

\paragraph{溶解消散期}
经5~10天,炎症消退。肉眼观:肺叶质地变软,色转灰红,切面颗粒状外观消失。细菌被吞噬细胞吞噬清除,渗出物被溶解,或经淋巴管吸收或被咳出。肺泡腔逐渐排空,重新充气。大叶性肺炎时,肺组织常无坏死,肺泡壁结构也未遭破坏,愈复后,肺组织可完全恢复其正常结构和功能。胸膜渗出物可完全吸收,否则可遗留胸膜增厚或粘连(图\ref{fig7-10})。

临床上病人体温下降,症状消退,由于渗出物的液化排出,肺部又可闻及湿性啰音。X线检查见病变处阴影密度减低,透亮度逐渐增加。2~3周后肺实变体征完全消失。

大叶性肺炎的一个重要特点是在整个病程中,肺泡壁结构通常未遭破坏,愈合后肺组织可完全恢复正常结构和功能。支气管的炎症病变轻微,仅有充血、点状出血和小支气管黏膜上皮脱落等,晚期可完全恢复正常。

\begin{figure}[!htbp]
 \centering
 \includegraphics{./images/Image00119.jpg}
 \captionsetup{justification=centering}
 \caption{大叶性肺炎消散期\\ {\small 肺泡腔内的渗出物逐渐溶解、吸收,肺泡壁部分毛细血管重新开放(HE染色,低倍)}}
\label{fig7-10}
  \end{figure}

\subsubsection{结局和并发症}

不伴有并发症的大叶性肺炎经过一般治疗,病人在发病后7~10天,体温下降,症状好转,趋向痊愈。需要指出的是,病变的发展是一个连续过程,因而上述分期不是绝对的,特别是抗生素广泛用于临床以来,上述典型病程已很少见到。抗生素的及时应用能减轻病情、缩短病程、提早康复。少数病例可出现以下并发症:

\paragraph{感染性休克}
是最严重的并发症。病人出现高热,血压下降,四肢厥冷,多汗,口唇青紫等休克症状,称休克型或中毒型肺炎。如果抢救不及时,病死率较高。

\paragraph{肺肉质变}
少数病例肺泡腔内渗出物未被及时溶解、清除,由肺泡壁增生的肉芽组织替代,发生机化,使局部肺组织形成肉样组织,称肺肉质变。

\paragraph{败血症}
严重感染时,细菌侵入血液繁殖,形成败血症,可引起心内膜炎、脑膜炎及关节炎等。

\paragraph{肺脓肿、脓胸}
由于抗生素的早期应用,临床已少见。

\subsection{小叶性肺炎}

小叶性肺炎(lobular
pneumonia)主要由化脓菌感染引起,病变起始于细支气管,并以细支气管为中心、向周围或末梢肺组织发展,形成以肺小叶为单位、呈灶状散布的肺化脓性炎。因其病变以支气管为中心故又称支气管肺炎(bronchopneumonia)。主要发生于小儿、年老体弱者或久病卧床者,冬、春季节发病率增高。

\subsubsection{病因和发病机制}

小叶性肺炎主要由细菌感染引起,最常见的细菌为致病力较弱的肺炎球菌,其次为葡萄球菌、链球菌、肺炎球菌、流感嗜血杆菌、铜绿假单胞菌和大肠埃希菌等。这些细菌通常是口腔或上呼吸道内致病力较弱的常驻寄生菌,往往在某些诱因影响下,如患传染病、营养不良、恶病质、慢性心力衰竭、昏迷、麻醉、手术后等,使机体抵抗力下降,呼吸系统的防御功能受损,细菌得以入侵、繁殖,发挥致病作用,引起支气管肺炎。因此,支气管肺炎常是某些疾病的并发症,如麻疹后肺炎、手术后肺炎、吸入性肺炎、坠积性肺炎等。有时成为病人的直接死亡原因,故又有临终性肺炎之称。

\subsubsection{病理变化}

以细支气管为中心的肺组织化脓性炎是小叶性肺炎的特征。

肉眼观:肺组织内散布一些以细支气管为中心的化脓性炎症病灶,常散布于两肺各叶,尤以背侧和下叶病灶较多。病灶大小不等,直径多在1
cm左右(相当于肺小叶范围),形状不规则,色暗红或带黄色(图\ref{fig7-11}A)。严重者,病灶互相融合甚或累及全叶,形成融合性支气管肺炎。

镜下观:病灶中支气管、细支气管及其周围的肺泡腔内充满脓性渗出物,纤维蛋白一般较少(图\ref{fig7-11}B)。病灶周围肺组织充血,可有浆液渗出、肺泡过度扩张等变化。由于病变发展阶段的不同,各病灶的病变表现和严重程度亦不一致。有些病灶完全化脓,支气管和肺组织遭破坏,而另一些病灶内则仅可见浆液性渗出,有的还停留于细支气管及其周围炎阶段。

\begin{figure}[!htbp]
 \centering
 \includegraphics{./images/Image00120.jpg}
 \captionsetup{justification=centering}
 \caption{小叶性肺炎}
 \label{fig7-11}
  \end{figure} 

\subsubsection{临床病理联系}

因小叶性肺炎多为其他疾病的并发症,其临床症状常为原发性疾病所掩盖。由于支气管黏膜的炎症刺激而引起咳嗽,痰呈黏液脓性。因病变常呈灶性散布,肺实变体征一般不明显。病变区细支气管和肺泡内含有渗出物,听诊可闻湿啰音。X线检查可见肺野内散在不规则小片状或斑点状模糊阴影。

\subsubsection{结局和并发症}

小叶性肺炎若治疗及时,多数病例预后良好。如病人为年老、体弱、婴幼儿或作为其他疾病的并发症,则预后较差。常见的并发症有肺脓肿、脓胸、支气管扩张症。严重的小叶性肺炎,病变范围广泛者可并发呼吸功能及心功能不全。

\subsection{间质性肺炎}

间质性肺炎(interstitial
pneumonia)指发生于肺间质的炎症,以淋巴细胞、巨噬细胞浸润为特征。肺间质包括肺泡壁、肺小叶间隔及细支气管周围组织。间质性肺炎的病变及临床症状与大叶性肺炎、小叶性肺炎均不相同,主要由肺炎支原体和病毒引起,其肺部的病理变化大致相似。

其基本病理变化为:

肉眼观:病变常位于一侧肺叶,偶有累及两肺者,以肺下叶较多见。病灶多呈斑片状,红黄色。镜下观:肺泡壁增厚,充血,水肿,常有多量淋巴细胞、巨噬细胞浸润,偶见浆细胞。肺泡腔内渗出物不明显,仅见少量浆液及少数巨噬细胞(图\ref{fig7-12})。

\begin{figure}[!htbp]
 \centering
 \includegraphics{./images/Image00121.jpg}
 \captionsetup{justification=centering}
 \caption{间质性肺炎(HE染色,中倍)\\ {\small 肺间质内有多量巨噬细胞和淋巴细胞浸润,肺泡腔内渗出物少}}
\label{fig7-12}
  \end{figure}

因病原不同,本病病变又各具特点,现分述如下:

\subsubsection{支原体肺炎}

支原体肺炎(mycoplasmal
pneumonia)是由肺炎支原体引起的一种间质性肺炎,发病率占各种类型肺炎的5%~10%。支原体存在于病人口鼻分泌物中,经飞沫传播,引起散发性呼吸道感染或者小流行。

肺炎支原体侵入呼吸道后在支气管黏膜上皮表面繁殖,使纤毛肿胀,活动减弱甚至消失,在免疫功能下降时,引起局部炎症。支原体肺炎多发生于20岁以下的青少年,50岁以上的成人由于隐性感染获得一定免疫力,因而其发病率随年龄增长而降低。

\paragraph{病理变化}
肺炎支原体侵犯整个呼吸道黏膜,引起气管炎、支气管炎和肺炎,甚至全呼吸道炎。肺部病变以下叶多见。肉眼观:病灶无明显实变,肺呈暗红色。切面肺普遍充血、水肿和不同程度的出血。镜下观:呈间质性肺炎改变。

\paragraph{临床病理联系}
本病一般起病较急,多有发热、乏力、咽痛、咳嗽等。由于支气管受炎症刺激,病人突出的症状为剧烈的咳嗽,由于肺泡腔内渗出物不多,痰量少,故常为干咳。肺实变体征不明显。X线检查示肺部病变多样化,可显示肺纹理增加、网织状阴影或斑点片状模糊阴影。

\subsubsection{病毒性肺炎}

引起病毒性肺炎(viral
pneumonia)的病毒种类较多,在成人多为流感病毒,在儿童及幼儿多为呼吸道合胞病毒,其他诸如腺病毒、麻疹病毒、巨细胞病毒等亦可致病。本病主要经呼吸道飞沫传播,在机体免疫力低下时引起肺部病变,少数则是病毒血症的结果。一般为散发性,偶可引起流行。

\paragraph{病理变化}
病毒性肺炎的病变常不一致,除上述典型的间质性肺炎外,还可出现下列病变:在严重病例,肺泡亦受累,肺泡腔内炎性渗出物增多,除浆液外,尚有纤维素、红细胞及巨噬细胞。某些病例渗出现象明显,渗出物浓缩并受空气挤压,在肺泡表面形成红染的膜状物,称为透明膜(图\ref{fig7-13}),这种改变可见于流感病毒、麻疹病毒及腺病毒引起的肺炎。有些病毒性肺炎可见支气管上皮、肺泡壁上皮细胞增生,并有多核细胞形成,在增生的上皮和巨噬细胞内可查见病毒包涵体,具有诊断意义。病毒包涵体常呈球形,约红细胞大小,呈嗜酸性染色,均质或细颗粒状,周围常有清晰的透明晕。包涵体可位于细胞核内(如腺病毒)或胞浆中(如呼吸道合胞病毒)或两者均有(如麻疹病毒)。严重的病例还可继发细菌感染,表现为间质性支气管肺炎。

\begin{figure}[!htbp]
 \centering
 \includegraphics{./images/Image00122.jpg}
 \captionsetup{justification=centering}
 \caption{病毒性肺炎(HE染色,中倍)\\ {\small 肺泡壁充血,巨噬细胞和淋巴细胞浸润,肺泡腔内渗出物形成透明膜}}
\label{fig7-13}
  \end{figure}

\paragraph{临床病理联系}
病毒血症可引起发热及全身中毒症状。由于支气管、细支气管炎症刺激可引起剧烈咳嗽。若肺泡腔内渗出物少,肺部啰音及实变体征不明显。严重病例或继发细菌感染时,肺部出现实变体征,伴有严重的全身中毒和缺氧症状,甚至导致心、肺功能不全,预后不良。

{【附】SARS}

严重急性呼吸道综合征(severe acute respiratory
syndrome,SARS)是由冠状病毒(SARS病毒)引起的一种新的呼吸系统传染性疾病。中国广东省首先发现,最早的病例出现在2002年11月中旬。目前已有多个国家报告发现了SARS病例。本病主要通过近距离空气飞沫和密切接触传播,临床主要表现为肺炎,有比较强的传染力。人群普遍易感,医护人员是本病的高危人群。潜伏期为2~12天,通常为4~5天。传染性主要在急性期(发病早期),尤以刚发病时最强。当病人被隔离及采取抗病毒、提高机体免疫力等治疗措施后,机体开始识别病毒并出现针对SARS的特异性免疫反应来抵抗和中和病毒。随着疾病的康复,SARS病毒逐渐被机体所清除,其传染性也随之消失。SARS起病急,以发热为首发症状,体温一般超过38℃,偶有畏寒;可伴有头痛、关节酸痛、肌肉酸痛、乏力、腹泻;常无上呼吸道其他症状;可有咳嗽,多为干咳、少痰,偶有血丝痰;可有胸闷,严重者出现呼吸加速,气促,或明显呼吸窘迫。肺部体征不明显,部分病人可闻少许湿啰音,或有肺实变体征。实验室检查发现:外周血白细胞计数一般不升高,或降低;常有淋巴细胞计数减少。胸部X线检查为肺部有不同程度的片状、斑片状浸润性阴影或呈网状改变,部分病人进展迅速,呈大片状阴影,常为双侧改变,阴影吸收消散较慢。肺部阴影与症状体征可不一致等。

一、病理变化

SARS死亡病例尸检显示该病以肺和免疫系统的病变最为突出,心、肝、肾、肾上腺等实质性器官也不同程度受累。

1.
肺部病变 肉眼观:双肺呈斑块状实变,严重者双肺完全性实变;表面暗红色,切面可见肺出血灶及出血性梗死灶。镜下观:以弥漫性肺泡损伤为主,肺组织重度充血、出血和肺水肿,肺泡腔内充满大量脱落和增生的肺泡上皮细胞及渗出的单核细胞、淋巴细胞和浆细胞。部分肺泡上皮细胞胞质内可见典型的病毒包涵体,电镜证实为病毒颗粒。肺泡腔内可见广泛透明膜形成,部分病例肺泡腔内渗出物出现机化,呈肾小球样机化性肺炎改变。肺小血管呈血管炎改变,部分管壁可见纤维素样坏死伴血栓形成,微血管内可见纤维索性血栓。

2.
脾和淋巴结病变 脾体积略缩小,质软。镜下见脾小体高度萎缩,脾动脉周围淋巴鞘内淋巴细胞减少,红髓内淋巴细胞稀疏,白髓和被膜下淋巴组织大片灶状出血坏死。肺门淋巴结及腹腔淋巴结固有结构消失,皮髓质分界不清,皮质区淋巴细胞数量明显减少,常见淋巴组织呈灶状坏死。心、肝、肾、肾上腺等器官均有不同程度变性、坏死和出血等改变。

二、结局及并发症

从目前掌握的SARS的传染过程来看,SARS病人的传染性主要在急性期(发病早期),尤以刚发病时为强。随着疾病的康复,SARS病毒逐渐被机体所清除,其传染性也随之消失。所以,SARS病人康复出院后是不会传染他人的。

不足5%的严重病例可因呼吸衰竭而死亡,其并发症及后遗症有待进一步观察确定。

\section{硅肺}

在职业活动中,因长期吸入有害粉尘,引起以肺广泛纤维化为主要病变的疾病,统称尘肺(pneumoconiosis)。尘肺是我国一种法定职业病。硅肺(silicosis)又称矽肺,是尘肺中最常见的类型,也是危害最严重的一种职业病。是人体在生产环境中长期吸入大量含游离二氧化硅(SiO{2}
)粉尘微粒所引起的以肺纤维化为主要病变的全身性疾病。该病发展缓慢,即使在脱离硅尘作业后,病变仍然继续发展。病人多在接触硅尘10~15年后才发病。若因吸入高浓度、高游离二氧化硅含量的硅尘,经1~2年后发病者,称速发型硅肺。硅肺的早期即有肺功能损害,但因肺的代偿能力很强,病人往往无症状;随着病变的发展,尤其是合并肺结核和肺心病时,则逐渐出现不同程度的呼吸和心功能障碍。

\subsection{病因和发病机制}

游离二氧化硅是硅肺的致病因子。硅肺的发生、发展与硅尘中游离二氧化硅的含量,生产环境中硅尘的浓度、分散度,从事硅尘作业的工龄及机体防御功能等因素有关。一般来说,直径大于5
μm的硅尘往往被阻留在上呼吸道,并可被呼吸道的防御装置清除。直径小于5
μm的硅尘才能被吸入肺泡,并进入肺泡间隔,引起病变,其中尤以1~2
μm的硅尘微粒引起的病变最为严重。

吸入肺泡内的硅尘微粒被肺巨噬细胞吞噬,沿肺淋巴流经细支气管周围、小血管周围、小叶间隔和胸膜再到达肺门淋巴结。当淋巴道阻塞后,硅尘沉积于肺间质内引起硅肺病变。若局部沉积的硅尘量多,引起肺巨噬细胞局灶性聚积,可导致硅结节形成;若硅尘散在分布,则引起弥漫性肺间质纤维化。

硅肺的发病机制尚未完全阐明。一般认为,游离二氧化硅颗粒进入肺泡后,被聚集在肺淋巴管起始部位的肺巨噬细胞所吞噬,游离二氧化硅对巨噬细胞有极强的毒性作用,可致其自溶死亡,二氧化硅被吞噬后,被包裹在吞噬细胞溶酶体中,由于石英表面的羟基和巨噬细胞溶酶体膜脂蛋白结构上的氢原子受体(氧、氮及硫原子)间形成氢键,引起细胞膜的改变和通透性的变化,导致巨噬细胞溶酶体崩解,并释放出酸性水解酶进入细胞内,继而导致巨噬细胞死亡,并再次将石英粒子释放,形成恶性循环,造成更多的细胞受损,受损的巨噬细胞释放出非脂类“致纤维化因子”,刺激成纤维细胞,合成胶原纤维增多,形成以胶原纤维为中心的病灶结节-硅结节,硅结节向全肺扩展并相互融合,造成双肺弥漫性损害。纤维化不仅局限于肺内,也存在于巨噬细胞所迁移到的淋巴结内。在许多硅肺病人中已发现血清γ-球蛋白水平增高,自身抗体的存在,以及在硅肺病变中存在γ-球蛋白,故认为硅肺发生与免疫发病有关。

\subsection{病理变化}

硅肺的基本病理变化是肺组织内硅结节形成和弥漫性间质纤维化。硅结节是硅肺的特征性病变,结节境界清楚,直径2~5
mm,呈圆形或椭圆形,灰白色,质硬,触之有砂样感。随着病变的发展,结节可融合成团块状,在团块的中央,由于缺血、缺氧而发生坏死、液化,形成硅肺性空洞。硅结节的形成过程大致分为三个阶段:①细胞性结节,由吞噬硅尘的巨噬细胞局灶性聚积而成,巨噬细胞间有网状纤维,这是早期的硅结节(图\ref{fig7-14}A);②纤维性结节,由成纤维细胞、纤维细胞和胶原纤维构成(图\ref{fig7-14}B);③玻璃样结节,玻璃样变从结节中央开始,逐渐向周围发展,往往在发生玻璃样变的结节周围又有新的纤维组织包绕。

镜下,典型的硅结节是由呈同心圆状或旋涡状排列的、已发生玻璃样变的胶原纤维构成。结节中央往往可见内膜增厚的血管。用偏光显微镜观察,可以发现沉积在硅结节和肺组织内呈双屈光性的硅尘微粒。除硅结节外,肺内还有不同程度的弥漫性间质纤维化,范围可达全肺2/3以上。此外,胸膜也因纤维组织弥漫增生而广泛增厚,甚至可厚达1~2
cm。肺门淋巴结内也有硅结节形成和弥漫性纤维化及钙化,淋巴结因而肿大、变硬。

\begin{figure}[!htbp]
 \centering
 \includegraphics{./images/Image00123.jpg}
 \captionsetup{justification=centering}
 \caption{硅结节}
 \label{fig7-14}
  \end{figure} 

\subsection{硅肺的分期}

根据肺内硅结节的数量、分布范围和直径大小,可将硅肺分为三期。

Ⅰ期硅肺:硅结节主要局限在淋巴系统。肺组织中硅结节数量较少,直径一般在1~3
mm,主要分布在两肺中、下叶近肺门处。X线检查,肺野内可见一定数量的类圆形或不规则小阴影,其分布范围不少于两个肺区。此时,肺的重量、体积和硬度无明显改变。胸膜上可有硅结节形成,但胸膜增厚不明显。

Ⅱ期硅肺:硅结节数量增多、体积增大,可散于全肺,但仍以肺门周围中、下肺叶较密集,总的病变范围不超过全肺;X线表现为肺野内有较多量直径不超过1
cm的小阴影,分布范围不少于四个肺区。此时,肺的重量、体积和硬度均有增加,胸膜也增厚。

Ⅲ期硅肺:硅结节密集融合成块。X线表现有大阴影出现,其长径不小于2
cm,宽径不小于7
cm。此时,肺的重量和硬度明显增加。解剖取出新鲜肺标本可竖立不倒,切开时阻力甚大,并有砂粒感。浮沉试验,全肺入水下沉。团块状结节的中央可有硅肺空洞形成。结节之间的肺组织常有明显的灶周肺气肿,有时肺表面还可见到肺大泡。

\subsection{硅肺的常见并发症}

\paragraph{硅肺结核病}
硅肺合并结核病时称为硅肺结核病。Ⅲ期硅肺的合并率达60%~70%。硅肺病人易合并肺结核可能是因游离二氧化硅对巨噬细胞的毒性损害以及肺间质弥漫性纤维化,导致肺的血液循环和淋巴循环障碍,从而降低了肺组织对结核杆菌的防御能力的缘故。硅肺结核病时,硅肺病变和结核病变可分开存在,也可混合存在。硅肺结核病的病变比单纯硅肺和单纯肺结核的病变发展更快,累及范围更广,更易形成空洞。硅肺结核性空洞的特点是数目多、直径大、空洞壁极不规则。较大的血管易被侵蚀,可导致病人大咯血死亡。

\paragraph{肺感染}
由于硅肺病人抵抗力低,又有慢性阻塞性肺疾病,小气道引流不畅,故易继发细菌或病毒感染。尤其在有弥漫性肺气肿的情况下,肺感染可诱发呼吸衰竭而致死。

\paragraph{慢性肺源性心脏病}
有60%~75%的硅肺病人并发肺心病。这是因为肺间质弥漫性纤维化,肺毛细血管床减少,肺循环阻力增加。同时,硅结节内小血管常因闭塞性血管内膜炎,管壁纤维化,使管腔狭窄乃至闭塞,血管也扭曲、变形,尤以肺小动脉的损害更为明显,加之因呼吸功能障碍造成的缺氧,引起肺小动脉痉挛,均可导致肺循环阻力增加、肺动脉高压和右心室肌壁肥厚,心腔扩张。重症病人可因右心衰竭而死亡。

\paragraph{肺气肿和自发性}
气胸晚期硅肺病人常有不同程度的弥漫性肺气肿,主要是阻塞性肺气肿,有时在脏层胸膜下还可出现肺大泡。气肿囊腔破裂引起自发性气胸。

\section{呼吸系统常见恶性肿瘤}

\subsection{鼻咽癌}

鼻咽癌(nasopharyngeal
carcinoma,NPC)是发生于鼻咽部上皮组织的恶性肿瘤,在我国较为常见。尤其多见于广东、广西、福建等南方地区,有明显的地区多发性。男性患者为女性的2倍,患者多在40~50岁。

\subsubsection{病因及发病机制}

鼻咽癌的病因迄今尚未完全阐明,可能与以下因素相关。

\paragraph{EB病毒}
资料显示100%鼻咽癌患者中有EB病毒的基因组,癌细胞内存在EBV-DNA及EB核抗原(EBNA),患者血清内还有高效价的抗EB病毒抗体,但EB是鼻咽癌的致病启动因素还是其他致癌物质的辅助作用因素尚需进一步研究。

\paragraph{环境致癌物质}
某些环境化学致癌物如亚硝胺、多环芳烃类化合物、微量元素镍等可能与鼻咽癌的发生有关。

\paragraph{遗传因素}
鼻咽癌发病有明显的地区性差异,高发区居民移居他地或国外,其后裔的发病率仍远远高于当地居民。部分鼻咽癌患者还有家族发病倾向,因此在其发病中可能有遗传性易感因素。

\subsubsection{病理变化}

鼻咽癌最常发生于鼻咽顶部,其次为侧壁及咽隐窝。有时还可同时在顶部及侧壁发生。

肉眼观:鼻咽癌呈结节型、菜花型、浸润型及溃疡型四种形态,其中以结节型最常见。早期局部黏膜仅显粗糙、增厚或稍稍隆起,临床检查时易被忽略。有时原发部位未发现肿瘤时已发生颈部淋巴结转移。

绝大多数鼻咽癌起源于鼻咽黏膜柱状上皮的储备细胞,该细胞是一种原始多能性的细胞,既可向柱状上皮方向分化,又可向鳞状上皮方向分化。因此,鼻咽癌的组织学分类较为复杂,迄今还没有统一的鼻咽癌病理学分类。一般来说,可分为两类。

\paragraph{鳞状细胞癌}
按细胞分化程度,可分为角化型和非角化型。

角化型鳞状细胞癌极少见,主要发生于老年患者,此型较少见,一般认为其发生与EB病毒无关。非角化型鳞状细胞癌最为多见,癌细胞呈多角形、卵圆形或梭形,无细胞角化现象,其发生与EB病毒感染关系密切。非角化型癌还可进一步分为分化型和未分化型。分化型即低分化鳞状细胞癌。未分化型又可分为分化极差的鳞状细胞癌和泡状核细胞癌。

\paragraph{腺癌}
高分化腺癌少见,癌细胞排列成腺泡状或管状。低分化腺癌癌细胞呈不规则条索状或片状排列,有时可见到腺腔结构或围成腺腔的倾向。

在鼻咽癌的组织学分型中,非角化型鳞状细胞癌最为常见,其次为未分化型的泡状核细胞癌。低分化腺癌较少,高分化鳞状细胞癌及腺癌最少。

\subsubsection{扩散及转移}

\paragraph{直接蔓延}
鼻咽部解剖解构复杂,肿瘤向上可侵犯颅内,向下扩展到达口咽,向下后方则侵犯梨状隐窝、会厌及喉腔上部,向外侧扩展可侵犯耳咽管至中耳,向后扩展则穿过鼻咽后壁侵犯上段颈椎,向前扩展则进入鼻腔甚至侵入眼眶。

\paragraph{淋巴道转移}
鼻咽黏膜固有层有丰富的淋巴管,故本癌可早期经淋巴道转移。颈淋巴结转移常为同侧,其次为双侧,极少只呈对侧转移。

\paragraph{血道转移}
常转移至肝、肺、骨,其次是肾、肾上腺及胰腺等处。

\subsubsection{临床病理联系}

鼻咽癌临床表现多样,常有鼻衄、鼻塞、耳鸣、听力减退、颈部肿块、复视及偏头痛等症状。当症状明显时多已进入进展期或晚期,治愈率极低,故早期诊断极为重要。

\begin{framed}
{案例7-1}

{【病例摘要】}

患者,男,65岁,因咳嗽、咳痰、痰中带血3个月入院。患者三月前开始出现刺激性咳嗽,自服止咳药未好转,痰中可出现血丝,近一月来症状加重。自发病以来患者体重下降8
kg。既往40余年吸烟史,平均每日1.5包,无酗酒史。入院后体检呈消瘦貌,神萎,血常规示中度贫血。胸片示肺门处一3cm×4cm占位影,怀疑支气管肺癌。

{【问题】}

(1)该患者怀疑支气管肺癌的依据为哪些?

(2)后行支气管镜检查,确诊为小细胞肺癌,试描述肿瘤镜下的组织学特征。
\end{framed}

\subsection{肺癌}

肺癌(lung
carcinoma)又称支气管肺癌,是最常见的恶性肿瘤之一。每年全世界有超过130万人被确诊患有肺癌,超过110万人死于肺癌。我国肺癌的发病率在20世纪70年代至90年代上升一倍多之后,近10年里继续呈明显上升趋势,目前肺癌已成为我国危害最大的癌症。肺癌多发生于45岁以上的中老年人,在55~75岁患病率最高,男女性别比例为2∶1。近年来,由于女性吸烟者的不断增加,女性比例相应上升。

\subsubsection{病因与发病机制}

肺癌的病因较复杂,其发生与下列因素有关。

\paragraph{吸烟}
吸烟是肺癌发生的重要危险因素,大约有3/4的肺癌患者有重度吸烟史。吸纸烟者肺癌的死亡率比不吸烟者高10~13倍。吸烟的量越多、吸烟的时间越长、开始吸烟的年龄越早,肺癌的死亡率越高。戒烟后则随戒烟时间的延长,肺癌发生率逐渐降低。卷烟燃烧的烟雾中含有超过1
200种化学物质,其中多环芳烃、3,4-苯并芘、放射性元素及砷等多种物质均具有致癌作用。3,4-苯并芘等多环芳烃碳氢化合物在人体内的芳烃羟化酶(AHH)的作用下转变为环氧化物而成为终致癌物,可导致细胞基因突变。由于不同人体内AHH的酶活性不同,因此吸烟致癌存在着个体差异。

\paragraph{物理化学致癌因子}
目前比较公认的致癌因子有烟草燃烧的产物、石棉、砷、铬、镍、铍、煤焦油、沥青、烟尘、芥子气、二氯甲醚等。如果长期接触这些物质,可以诱发肺癌。我国云南锡矿工人的肺癌发生率高达435.44/10万,井下作业较地面作业工人肺癌发病率高20倍。可能与工作中长期接触化学致癌物质和放射性物质有关。

\paragraph{大气污染}
煤、汽油、柴油等燃烧后的废气或烟尘、行驶机动车的排气均可造成空气污染。被污染的空气中含有3,4-苯并芘、二乙基亚硝胺和砷等致癌物。调查表明,工业发达国家肺癌发病率比工业落后国家高、城市比农村高、大城市比中小城市高。

\paragraph{基因改变}
各种致癌因素可引起细胞的基因变化而导致细胞发生癌变。目前已知在肺癌中有多种癌基因的突变或肿瘤抑制基因的失活,其中KRAS、c-Myc、P53、Rb和bcl-2基因都是研究的热点。有关遗传或基因因素在肺癌发生过程中的作用,有待于进一步研究探索。

\subsubsection{病理变化}

\paragraph{大体类型}
根据肿瘤的发生部位可把肺癌分为三种类型:中央型、周围型和弥漫型,与临床X线的肺癌分型相一致。

(1)中央型:此型最常见,多起源于主支气管或叶支气管等大支气管,肿瘤位于肺门部,常破坏支气管壁向周围肺组织浸润、扩展。晚期形成巨块,常包绕癌变的支气管(图\ref{fig7-15}A)。

(2)周围型:此型发生率仅次于中央型,多起源于肺段以下的末梢支气管或肺泡。常在靠近胸膜的肺周边部形成孤立的癌结节。肉眼形态多为结节型(图\ref{fig7-15}B)。

(3)弥漫型:此型少见。肉眼观察:多数呈播散性的粟粒性结节,弥漫侵犯部分肺大叶或全肺叶,似肺炎或播散性肺结核(图\ref{fig7-15}C)。

\begin{figure}[!htbp]
 \centering
 \includegraphics{./images/Image00124.jpg}
 \captionsetup{justification=centering}
 \caption{肺癌大体形态}
 \label{fig7-15}
  \end{figure} 

\paragraph{组织学类型}
世界卫生组织(WHO)最新分类中把肺癌分为鳞状细胞癌、腺癌、大细胞癌、腺鳞癌、神经内分泌癌、肉瘤样癌、其他类型癌和唾液腺来源的癌等8种类型。不同组织学类型在临床表现、治疗手段的选择及预后上均不相同。

(1)鳞状细胞癌:是肺癌最常见类型之一,绝大多数为中老年患者,多有吸烟史。多来自段以上或主支气管,肉眼属中央型,纤支镜检查易被发现,痰脱落细胞学检查阳性率高。高分化鳞癌多有角化珠形成,低分化鳞癌仅有少量细胞角化。

(2)腺癌:也是原发性肺癌中最常见类型之一,且近年来发病率有不断上升的趋势。肺腺癌多数为周围型,女性患者较多,患者不吸烟但多有被动吸烟史。腺癌常位于肺周边部呈孤立结节,边界清楚,常累及胸膜。高分化癌可见癌组织形成腺管或乳头,并有黏液分泌。

(3)神经内分泌细胞癌:主要包含小细胞癌、大细胞神经内分泌癌和类癌。小细胞癌为仅低于肺鳞癌及腺癌的相对常见的一型肺癌。其发生率占原发性肺癌的15%~20%。发病年龄较鳞癌低,好发于中年男性,与吸烟及职业性接触有一定关系。肿瘤恶性度极高,生长迅速。多有早期转移,一般不适合手术切除,但对化疗及放疗敏感。本型癌细胞小呈短梭形(燕麦型,图\ref{fig7-16})或小圆形(淋巴细胞样),核浓染,胞浆稀少形似裸核。癌细胞常密集成群,有时围绕小血管排列成假菊形团样结构。电镜下可见一部分癌细胞胞浆含有神经分泌颗粒,现认为该肿瘤起源于APUD系统,可伴有异位内分泌综合征。

\begin{figure}[!htbp]
 \centering
 \includegraphics{./images/Image00125.jpg}
 \captionsetup{justification=centering}
 \caption{燕麦细胞癌(HE染色,高倍)\\ {\small 癌细胞小呈短梭形或小圆形,常密集成群,围绕小血管排列成假菊形团样结构}}
\label{fig7-16}
  \end{figure}

(4)大细胞癌:肺大细胞癌属于未分化癌,恶性度高,癌生长迅速,早期发生转移。

(5)腺鳞癌:此型肺癌含有腺癌细胞及鳞癌细胞两种成分,属于混合性癌。现认为此型肺癌发生自支气管上皮的具多向分化潜能的干细胞。

(6)肉瘤样癌:为近年WHO新列出的一种肺癌分类,癌分化不成熟,恶性度高,有多形性、梭形细胞性、巨细胞癌及癌肉瘤等多种亚型。

\subsubsection{扩散与转移}

\paragraph{直接蔓延}
中央型肺癌常直接侵及肿瘤周围组织如纵隔、心包及周围血管,或沿支气管向同侧甚至对侧肺组织蔓延。周围型肺癌可直接侵犯胸膜,在胸壁生长。

\paragraph{转移}
肺癌沿淋巴道转移时首先转移至肺门淋巴结,再扩散至纵隔、锁骨上、腋窝和颈部淋巴结。周围型肺癌的癌细胞可到达胸膜下淋巴丛,引起胸膜腔的血性渗出液。血行转移常见于肝、脑、肾上腺、骨及肾等处。

\subsubsection{临床病理联系}

肺癌早期因症状不明显易被忽视。患者可有咳嗽、痰中带血丝及胸痛等症状。肿瘤压迫或阻塞支气管可引起远端肺组织的化脓性炎、脓肿形成。癌组织侵及胸膜引起癌性胸膜炎、积液。侵犯纵隔内压迫上腔静脉引起面颈部水肿及颈、胸部静脉曲张(上腔静脉综合征)。肺尖部肺癌易侵犯交感神经引起病侧眼睑下垂、瞳孔缩小和胸壁皮肤无汗等交感神经麻痹综合征(Horner综合征)。有异位内分泌作用的肺癌,尤其是小细胞肺癌可因5-羟色胺分泌过多而引起类癌综合征,表现为支气管哮喘、心动过速、水样腹泻和皮肤潮红等。

\begin{center}
    \textbf{知识链接}
\end{center}
\chapterabstract{肺癌生物治疗是一种利用细胞生物学与分子生物学手段调节机体免疫系统功能或肿瘤生长,从而达到抑瘤目的的治疗方法,是继手术、放疗、化疗模式之后新兴的治疗手段,它具有常规治疗方法无可比拟的优势,并显示出良好的临床应用前景。具体治疗包括树突状细胞疫苗、相关肿瘤抗原疫苗、肿瘤细胞疫苗、过继性细胞免疫治疗和分子靶向治疗等。肺癌的生物治疗不仅开辟了肺癌全新治疗模式,同时也丰富了肿瘤生物治疗范围。}

\section*{复习与思考}

{一、名词解释}

COPD慢性阻塞性肺气肿 小叶中央型肺气肿 全小叶型肺气肿 肺心病 大叶性肺炎 小叶性肺炎 肺肉质变 硅结节 燕麦细胞癌

{二、问答题}

1. 试述慢性支气管炎的病变特点。

2. 试述支气管扩张症的发病机制。

3. 试述慢性肺源性心脏病的病变及临床病理联系。

4. 试述大叶性、小叶性、间质性肺炎的病变特点及大、小叶性肺炎的鉴别点。

5. 硅肺的病理特点是什么?硅结节是如何形成的?

6. 鼻咽癌的主要组织学类型及其扩散途径是什么?

7. 肺癌的常见病理类型有哪些?

8. 右肺上叶有一直径约1.5
cm的球形病灶,试考虑有哪些病变的可能及其病理特点。

{三、临床病理分析}

病史:患者男性,64岁。慢性咳嗽、咳痰28年,痰通常为白色泡沫样,有时发热伴脓痰。近五年来爬坡即感气急,近两年来稍活动即感气急,时有心悸,面部与下肢水肿。入院前一周开始发热,近三日来高达39℃,气急加重,指唇出现青紫,下肢水肿。

既往史:吸烟30年,日吸一包以上。

体检:体温38.6℃,脉搏100次/分,呼吸24次/分,血压正常,神志清但迟钝。口唇轻度青紫,下肢轻度水肿,颈静脉稍充盈,胸廓呈桶状。腹部略膨胀,肝剑突下二横指,质中,轻度压痛。扣诊,心界扣不出。听诊:两肺可闻及广泛湿啰音,肺动脉瓣第二音亢进。X线胸片,两肺透明度增高,肺纹理增强,两肺下叶有小片状模糊炎性阴影,横膈低平,心影扩大,肺动脉圆锥突起。

讨论题:

1. 试分析患者可能发生的疾病。

2. 试叙述疾病的发生发展过程。

3. 试阐明疾病的病理变化。
\chapter{医院获得性肺炎和呼吸机相关性肺炎}

\section{前沿学术综述}

医院获得性肺炎(hospital-acquired
pneumonia,HAP)是患者住院期间发生的肺实质感染,指入院时既不存在,也不处于潜伏期的感染。根据美国国家医院获得性感染监测系统(national
nosocomial infection surveillance
system,NNIS)的资料,肺部感染已经超过泌尿系感染,成为最常见的医院获得性感染,医院获得性肺炎占所有医院获得性感染的27%。在美国,每年约有25万住院患者发生医院获得性肺炎,直接或间接导致3万名患者死亡,我国该病的患病率为1.3%~3.4%。重症医学科(ICU)医院获得性肺炎的患病率与其他住院患者相比增加10~20倍。医院获得性肺炎不仅使患者住院时间平均延长4~9天,也是导致医院获得性感染患者的主要死因。预防和控制医院获得性肺炎具有重要意义。

呼吸机相关性肺炎(ventilator-associated
pneumonia,VAP)是医院获得性肺炎的一种特殊类型,指机械通气48小时后发生的肺炎。根据患者人群不同,呼吸机相关性肺炎患病率在6%~52%不等。并发呼吸机相关性肺炎的患者重症医学科住院时间和总住院时间明显延长,住院费用明显增加。呼吸机相关性肺炎是患者病死率增高的独立危险因素,粗病死率达到30%~70%。

医院获得性肺炎可以由多种微生物病原体引起,常见的病原体包括需氧革兰阴性杆菌,如铜绿假单胞菌、大肠埃希菌、肺炎克雷伯菌和不动杆菌等;革兰阳性球菌,如葡萄球菌,特别是耐甲氧西林的金黄色葡萄球菌(methicillin-resistant
staphylococcus
aureus,MRSA)的感染逐年增多。此外,厌氧菌感染所致医院获得性肺炎在机械通气患者中较罕见。

导致呼吸机相关性肺炎的常见细菌包括革兰阴性杆菌和革兰阳性球菌,如铜绿假单胞菌、肺炎克雷伯杆菌、不动杆菌属、金黄色葡萄球菌(特别是MRSA)等,而且多药耐药(multidrug-resistant,MDR)细菌的比例在明显增高。近年来真菌感染也呈上升趋势。

\subsubsection{呼吸机相关性肺炎的诊断}

诊断呼吸机相关性肺炎基于两个方面:一是依据病史(机械通气48小时以上,有危险因素)、体格检查和X线胸片判断是否存在肺炎;二是明确感染的病原微生物。

目前诊断呼吸机相关性肺炎的金标准仍然是组织病理学有炎症反应和肺活组织培养微生物阳性,但此标准临床难以实现。临床诊断标准为X线胸片出现新的浸润阴影或原有浸润阴影扩大,同时具有下列三项中的两项或两项以上:①体温>38℃;②白细胞计数增高或降低;③脓性痰。此诊断标准的敏感性为69%,特异性为75%。临床肺部感染评分(clinical
pulmonary infection
score)有助于呼吸机相关性肺炎进行量化的诊断,主要从体温、血白细胞计数、痰液性状、X线胸片、氧合指数和半定量培养结果诊断呼吸机相关性肺炎,总分12分,一般以临床肺部感染评分>6分作为诊断标准,与金标准相比其敏感性为77%,特异性为42%,而简化的临床肺部感染评分更便于临床评估。2005年美国胸科协会(American
Thoracic Society,ATS)和美国感染病协会(Infectious Diseases Society of
America,IDSA)在医院获得性肺炎和呼吸机相关性肺炎指南中首次明确提出,临床肺部感染评分可以用于协助肺部感染的诊断和指导抗生素的调整
\protect\hyperlink{text00014.htmlux5cux23ch1-13}{\textsuperscript{{[}1{]}}}
。

微生物学诊断是指对下呼吸道分泌物进行定量培养,确定诊断阈值,超过阈值,可考虑诊断呼吸机相关性肺炎,低于阈值一般认为是定植或污染。其目的是判断何种微生物为致病菌,以及是否开始抗菌药物治疗、选择何种抗菌药物。保护性毛刷(protected
specimen brush)分泌物定量培养以>10\textsuperscript{3}
cfu/ml为诊断标准,支气管肺泡灌洗(bronchoalveolar
lavage)液定量培养以>10\textsuperscript{4}
cfu/ml为诊断标准,气管抽吸分泌物培养以>10\textsuperscript{6}
cfu/ml为诊断标准。特别强调在抗菌药物治疗前应留取标本,但不能因为需要留取标本或等待结果而延误抗菌药物治疗。

近年来部分研究显示,支气管肺泡灌洗液中的髓样细胞表达的可溶性触发受体I(sTREM-I)浓度的检测,也可作为早期诊断呼吸机相关性肺炎的手段,血浆降钙素原、支气管肺泡灌洗液中内毒素浓度检测,也可作为呼吸机相关性肺炎的筛选方法,阳性结果意味着需要更多的细菌学检测和及时经验性抗菌药物治疗,以及定期评价临床效果。

\subsubsection{呼吸机相关性肺炎的治疗}

重症感染及感染性休克导致患者病死率居高不下,一直在50%左右,为此2004年11个国际性组织在巴塞罗那联合推出了重症感染和感染性休克治疗指南,力争在全世界范围内通过教育和指南的推广,使重症感染和感染性休克的病死率在5年之内降低25%
\protect\hyperlink{text00014.htmlux5cux23ch2-13}{\textsuperscript{{[}2{]}}}
。而呼吸机相关性肺炎是目前重症医学科最常见的感染之一,早期合理的抗生素治疗和积极预防呼吸机相关性肺炎的发生非常重要。2004年加拿大危重病学会和加拿大危重病临床试验组联合组成专家委员会,以及2005年美国胸科医师协会和美国感染病协会先后制定了基于循证医学证据的医院获得性肺炎和呼吸机相关性肺炎方面的指南,对于医院获得性肺炎和呼吸机相关性肺炎的诊断和规范化治疗提出了许多建议
\protect\hyperlink{text00014.htmlux5cux23ch1-13}{\textsuperscript{{[}1{]}}}
、
\protect\hyperlink{text00014.htmlux5cux23ch3-13}{\textsuperscript{{[}3{]}}}
、
\protect\hyperlink{text00014.htmlux5cux23ch4-13}{\textsuperscript{{[}4{]}}}
。呼吸机相关性肺炎的治疗原则主要是根据病原菌是否存在多药耐药(MDR)危险性和肺炎发生的时间,结合本单位具体情况,早期及时应用合适、足量的抗菌药物,并根据微生物学涂片结果、培养和患者的临床治疗反应做相应调整。

近年来,在医院获得性肺炎和呼吸机相关性肺炎防治上,强调针对感染危险因素应用多种非抗生素抗感染策略集束化预防。根据循证医学证据,强烈推荐将半卧位、洗手、持续声门下吸引应用于机械通气患者呼吸机相关性肺炎的预防
\protect\hyperlink{text00014.htmlux5cux23ch5-13}{\textsuperscript{{[}5{]}}}
。近来研究显示早期经皮胃造瘘和洗必泰口腔护理也可以降低呼吸机相关性肺炎的发生率
\protect\hyperlink{text00014.htmlux5cux23ch6-13}{\textsuperscript{{[}6{]}}}
\textsuperscript{,}
\protect\hyperlink{text00014.htmlux5cux23ch7-13}{\textsuperscript{{[}7{]}}}
。鉴于热湿交换器和封闭式吸痰在对于降低呼吸机相关性肺炎发生率方面无明确的临床循证医学证据支持
\protect\hyperlink{text00014.htmlux5cux23ch8-13}{\textsuperscript{{[}8{]}}}
\textsuperscript{~}
\protect\hyperlink{text00014.htmlux5cux23ch10-13}{\textsuperscript{{[}10{]}}}
,目前并不推荐常规应用热湿交换器和封闭式吸痰。

\section{临床问题}

\subsection{流行病学与发病机制}

\subsubsection{何谓医院获得性肺炎?}

医院获得性肺炎又称医院内肺炎(nosocomial
pneumonia),指患者入院48小时后发生的肺炎,入院时既不存在、也不处于感染潜伏期。2005年美国胸科协会和美国感染病协会制定的医院获得性肺炎指南中,医院获得性肺炎也包括了两种特殊的类型:呼吸机相关性肺炎和卫生保健相关性肺炎(healthcare-associated
pneumonia),后者包括下列肺炎病人:①最近90天内住院超过2天以上者;②居住在护理之家或长期护理机构者;③接受透析治疗者;④接受家庭输液治疗(抗生素)者;⑤接受过伤口处理者;⑥家庭成员携带多药耐药菌者。

目前认为,在医院获得性感染(包括医院获得性肺炎、血源性感染、泌尿系感染、外科伤口感染、导管相关性感染等)所致的死亡中,医院获得性肺炎是主要死因。

\subsubsection{医院获得性肺炎病原学如何分布?}

医院获得性肺炎可以由多种微生物病原体引起,医院获得性肺炎常见的病原体包括需氧革兰阴性杆菌,如铜绿假单胞菌、大肠埃希菌、肺炎克雷伯菌和不动杆菌等;革兰阳性球菌,如葡萄球菌,特别是耐甲氧西林的金黄色葡萄球菌的感染逐年增多,有研究显示,美国重症医学科中葡萄球菌所致的感染,耐甲氧西林的金黄色葡萄球菌超过50%。在患者免疫力正常的情况下较少发生病毒和真菌的感染。在免疫功能低下的人群,例如器官移植后患者、艾滋病毒感染者以及糖尿病、中性粒细胞减少、潜在肺病和终末期肾脏疾病患者容易发生病毒与真菌感染。此外,厌氧菌感染所致医院获得性肺炎在机械通气患者中较罕见。我国医院内病原菌耐药监测(nosocomial
pathogen resistance
surveillance,NPRS)在1994~2002年的8年间,共分离到12821株革兰阴性杆菌,最常见的是铜绿假单胞菌、大肠埃希菌、克雷伯菌属、不动杆菌属、肠杆菌属、嗜麦芽窄食单胞菌,呼吸道标本中最常见的是铜绿假单胞菌、肺炎克雷伯菌和鲍曼不动杆菌。

早发性(指入院后48小时到5天内发生的)医院获得性肺炎多是由敏感菌,如肺炎链球菌、流感嗜血杆菌、甲氧西林敏感金黄色葡萄球菌(methicillin-sensitive
staphylococcus
aureus,MSSA)和敏感的肠道革兰阴性杆菌(如大肠杆菌、肺炎克雷伯杆菌、变形杆菌和黏质沙雷杆菌)引起的感染。

晚发性(指入院时间≥5天)医院获得性肺炎则很可能是多药耐药(MDR)细菌所致,包括铜绿假单胞菌、产超广谱β-内酰胺酶(extended
broad-spectrum
β-lactamase,ESBL)的肺炎克雷伯杆菌和鲍曼不动杆菌、耐药肠道细菌属、嗜麦芽窄食假单胞菌,以及耐甲氧西林的金黄色葡萄球菌等,免疫抑制患者还需考虑嗜肺军团菌感染可能。

当然,早发性医院获得性肺炎患者如果近期接受过抗生素治疗或在健康护理院住院,也存在多药耐药病原菌定植和感染的危险性,患者病死率明显增高。多药耐药病原菌的流行情况依患者的基础疾病状态、所在地区和医院以及重症医学科的种类而有所不同。

\subsubsection{多药耐药病原菌引起医院获得性肺炎的危险因素有哪些?}

引起医院获得性肺炎的病原菌是否为多药耐药细菌,需考虑是否存在以下危险因素:①先前90天内接受过抗菌药物治疗;②本次住院5天以上;③社区或医院特殊病房中存在高发细菌耐药;④存在卫生保健相关性肺炎危险因素------最近90天内住院2天以上、居住在护理之家或扩大护理机构、家庭静脉治疗(包括抗菌药物)、30天内进行过慢性透析治疗、家庭伤口护理;家庭成员携带多药耐药菌;⑤存在免疫抑制性疾病和(或)在使用免疫抑制剂治疗。

\subsubsection{常见的多药耐药革兰阴性杆菌耐药现状如何?}

常见的多药耐药革兰阴性杆菌耐药主要有铜绿假单胞菌、不动杆菌属(鲍曼不动杆菌等)、产超广谱β-内酰胺酶的革兰阴性菌、产诱导酶细菌、嗜麦芽窄食单胞菌等。

(1)铜绿假单胞菌 是医院获得性肺炎中最常见的多药耐药病原菌。国内外资料表明,铜绿假单胞菌对β-内酰胺类、碳青霉烯类、氨基糖苷类、氟喹诺酮类耐药率呈增加趋势。我国医院内病原菌耐药监测显示,1994~2002年间,铜绿假单胞菌对亚胺培南和头孢他啶的敏感率已分别从1994年的96%、92%降至2001年的75%、79%,目前敏感率较高为阿米卡星和哌拉西林/三唑巴坦(分别为83%、81%),但阿米卡星由于耳、肾毒性造成应用范围受限。2005年来自日本60个医学中心分离的细菌菌株耐药监测显示,铜绿假单胞菌对头孢哌酮/舒巴坦的耐药率为12.5%,对泰能的耐药率高达30.8%。铜绿假单胞菌耐药机制复杂,有研究表明,外膜孔蛋白D(outer
membrane porin channel
D,OprD)的表达减少或丢失是导致铜绿假单胞菌对亚胺培南耐药的重要机制之一。

(2)不动杆菌属(鲍曼不动杆菌等) 对大多数抗菌药物,包括氨基糖苷类、喹诺酮类和广谱的β内酰胺类抗菌药物都耐药。不同菌种对抗菌药物的敏感性不同,其中鲍曼不动杆菌耐药最为严重,容易获得优势生长,其特点是定植菌多于感染菌,在健康人群中其定植率>40%,而在住院患者中定植率高达75%。亚胺培南对不动杆菌活性最高,超过85%的菌株对碳青酶烯类敏感,但由于IMP型或OXA型碳青霉烯酶的产生使细菌耐药性增加。在2005年美国胸科协会和美国感染病协会的医院获得性肺炎指南中,推荐用多黏菌素E治疗碳青霉烯耐药的不动杆菌感染。

(3)产超广谱β-内酰胺酶的革兰阴性菌 肺炎克雷伯菌和大肠埃希菌是常见的产超广谱β-内酰胺酶的细菌。质粒介导的超广谱β-内酰胺酶是由于β内酰胺酶的1~4个氨基酸突变而造成的,能够水解β内酰胺类抗生素和氨曲南,造成细菌对上述药物耐药,同时对其他大多数抗菌药物如氟喹诺酮类和氨基糖苷类耐药。超广谱β-内酰胺酶可以被克拉维酸、舒巴坦和三唑巴坦所抑制。我国流行的超广谱β-内酰胺酶亚型以CTX-M型为主,而在欧美各国以TEM型常见。

(4)产诱导酶细菌 诱导酶是染色体介导的头孢菌素酶,称Ⅰ类酶,又称诱导酶。阴沟肠杆菌、弗劳地枸橼酸杆菌、粘质沙雷菌和铜绿假单胞菌中可分离到此类酶。诱导酶不能被酶抑制剂所抑制,因此产诱导酶的细菌对所有的β内酰胺类/酶抑制剂复合制剂及头霉素均耐药,并且常常对氨基糖苷类和喹诺酮类耐药。治疗药物仅能选择碳青霉烯类和第四代头孢菌素。

(5)嗜麦芽窄食单胞菌 该菌是一种非发酵菌。我国医院内病原菌耐药监测显示,1994~2002年间分离的12821株革兰阴性菌中,嗜麦芽窄食单胞菌居全部标本的第7位,在呼吸道标本中居第6位。该菌天然产生金属酶,可以水解碳青霉烯类,故其对亚胺培南和许多β内酰胺类抗生素天然耐药。临床上复方磺胺嘧啶、米诺环素、左氧氟沙星和替卡西林/克拉维酸有时可以作为药物治疗的选择之一。

\subsubsection{如何评价多药耐药革兰阳性球菌的耐药现状?}

近年来多药耐药革兰阳性球菌逐渐增多,常见的病原菌及耐药现状如下。

(1)耐青霉素肺炎链球菌(penicillin-resistant streptococcus
pneumoniae,PRSP) 1965年美国首次报道肺炎链球菌对青霉素耐药,近年来耐药菌株逐渐增多,国外达30%~50%,国内王辉等报告显示约22.7%,对头孢呋新、大剂量阿莫西林、阿莫西林/棒酸、头孢噻肟、头孢曲松或氟喹诺酮尚敏感,但对红霉素、克林霉素、磺胺等药物多耐药。对以上药物都耐药时只能使用万古霉素或替考拉宁治疗。

(2)耐甲氧西林金黄色葡萄球菌及表皮葡萄球菌 该耐药菌已出现40余年,近年来直线上升。美国1975年耐药率仅2.4%,20世纪末达25%,近年来已达50%。在我国情况更为严重,上世纪末已达50%,本世纪以来不少三甲医院已达80%以上。耐甲氧西林的金黄色葡萄球菌已经成为院内感染的主要细菌之一,它们对除万古霉素和替考拉宁以外的常用抗生素均耐药。更可怕的是,国外出现了对万古霉素中介、甚至耐药的金黄色葡萄球菌,前者多选用万古霉素加利福平、阿米卡星等,耐万古霉素的金黄色葡萄球菌只有依靠更新的抗生素,如利奈唑胺(linezolid)、奎奴普丁/达福普汀(quinupristin/dalfopristin)。所幸国内尚未发现耐万古霉素的耐甲氧西林的金黄色葡萄球菌菌株。

(3)耐药肠球菌 包括粪肠球菌、屎肠球菌。耐药肠球菌不但对一般抗球菌药及广谱抗生素耐药,而且对特异性、很少耐药的万古霉素也耐药,即耐万古霉素肠球菌,1988年英国首次发现。国外报道发生率为7.9%~20%,国内10%~12.5%。治疗上根据药敏选用敏感抗生素,如替考拉宁或氨基糖苷类,体外研究显示新药利奈唑胺、达福普丁/奎奴普汀可用于耐万古霉素的肠球菌治疗。

\subsubsection{有哪些危险因素可以导致医院获得性肺炎的发生?}

医院获得性肺炎的危险因素可分为患者相关因素、感染控制相关因素和治疗相关因素等3类。

(1)患者相关因素 患者相关因素是医院获得性肺炎发生的基本条件。严重的急或慢性疾病、昏迷、营养不良、长期住院、低血压、代谢性酸中毒、吸烟以及同时患有多种疾病(包括中枢神经系统功能障碍、慢性阻塞性肺病、糖尿病、酗酒、氮质血症和呼吸功能衰竭等),均可通过影响患者的自身防御功能,导致细菌定植和感染。由于老年人多数患有多种疾病,因而发生肺炎的危险性增加,此外衰老导致的免疫功能下降也起一定的作用。

(2)感染控制相关因素 致病菌暴露或致病菌定植是医院获得性肺炎发生的必要条件。住院患者通常有很多机会接触大量细菌。细菌可通过医务人员的手在不同患者间传播。常见的原因包括接触不同患者之间不洗手或不更换手套,或使用污染的呼吸治疗设备等。胃肠道通常被认为是肠道革兰阴性杆菌的储存库,误吸可导致大量致病菌进入呼吸道导致感染。

(3)治疗相关因素 治疗相关的因素也参与了医院获得性肺炎的发生。镇静药抑制中枢神经系统功能,增加误吸发生率;糖皮质激素和细胞毒药物影响机体防御机制;长时间外科手术,特别是胸腹联合手术,改变呼吸道纤毛功能和细胞免疫,增加口咽部细菌的定植和肺炎的发生率。此外,长期不适当的抗生素治疗、制酸药的应用等治疗措施,也能增加患者接触大量细菌的机会。接受呼吸治疗患者的医院获得性肺炎发生率(7.7%),明显高于不接受呼吸治疗者(0.3%)。气管插管、气管切开或机械通气的患者发生肺部感染的危险性增加4~66倍。气管插管能影响下呼吸道的纤毛运动,妨碍分泌物的排出,同时破坏上皮细胞表面,使得细菌容易与下呼吸道表面结合。

按内源性和外源性因素分类,医院获得性肺炎发生的主要危险因素见表\ref{tab8-1}。

\begin{table}[htbp]
\centering
\caption{医院获得性肺炎的内源性和外源性危险因素}
\label{tab8-1}
\includegraphics{./images/Image00059.jpg}
\end{table}

\subsubsection{医院获得性肺炎的发病机制是什么?}

医院获得性肺炎的主要发病机制包括口咽部微生物和(或)含微生物的胃内容物的误吸、吸入含有细菌的微粒或远处感染灶的血行播散。

(1)误吸口咽部微生物和(或)含微生物的胃内容物的误吸是医院获得性肺炎最重要的致病因素。

大多数细菌性肺炎,无论是否为医院获得性,其致病菌多为口咽部的细菌。约有10%的健康人口咽部有革兰阴性杆菌的定植。而住院和应激状态下细菌定植可显著增加。30%~40%的普通患者入院后48小时内即有细菌的定植,而危重患者则达70%~75%。革兰阴性杆菌在口咽部或气管支气管的定植是通过与宿主的上皮细胞粘附开始的。许多因素可以影响粘附,如细菌因素(鞭毛、纤毛、荚膜或产生弹力酶等)、宿主细胞因素(表面蛋白和多糖)以及环境因素(pH值和呼吸道分泌物中的粘蛋白)。尽管确切的作用机制尚不明确,但已有研究表明某些物质如纤维连接素能抑制革兰阴性杆菌与宿主细胞的粘附。相反,营养不良、危重病或术后等情况,均可促进细菌粘附,导致细菌定植明显增加。口咽部定植的细菌误吸是医院获得性肺炎发病的必要条件。Huxley等人用同位素示踪法发现,45%的正常人在熟睡时存在误吸。而那些吞咽困难、神志不清、气管插管和(或)机械通气、胃肠道疾患和术后的患者,则更容易发生误吸(70%)。所以,对危重患者而言,误吸是普遍存在的现象,不同的是误吸的量或程度的差异。即使带有套囊的气管切开管也不能防止误吸,研究显示,低容量高压气囊和高容量低压气囊分别有80%和15%的患者发生误吸。口咽部定植的细菌发生误吸后,由于肺部防御机制的障碍(如糖皮质激素、抗生素、氮质血症、酸中毒、经气管吸痰、酗酒或糖尿病等)将引起医院获得性肺炎。

对于机械通气患者,胃是口咽部革兰阴性定植菌的主要来源。健康人胃内pH值低于2,基本处于无菌状态。但当胃内pH值高于4时,微生物即在胃内大量繁殖,在高龄、胃酸缺乏、肠梗阻或上消化道疾患,以及接受胃肠营养、抗酸药或H\textsubscript{2}
受体拮抗剂治疗的患者尤为常见。研究表明,胃内pH值为1.0时,胃液中无细菌生长;而当pH值增加到6.0时,胃液内菌落增至10\textsuperscript{7}
cfu/ml以上。应用西米替丁治疗已被证实是医院获得性肺炎的危险因素之一。因此,应激性溃疡的预防也有其副作用。硫醣铝在保护胃黏膜的同时并不降低胃内pH值。有研究表明,与抗酸药或H\textsubscript{2}
受体拮抗剂相比,硫醣铝能够减少医院获得性肺炎的发生。应用局部或胃肠道不吸收抗生素进行选择性胃肠道去污染(selective
digestive
decontamination,SDD),有可能减少需氧革兰阴性杆菌引起的呼吸道感染,但现有资料不足以推荐选择性胃肠道去污染常规应用于所有患者。

(2)吸入含有细菌的微粒 细菌进入下呼吸道的另一种方式是通过吸入被呼吸治疗或麻醉设备污染的空气。呼吸机雾化装置能通过超声雾化作用产生大量<4μm的微粒,一旦受到污染,其产生的微粒可含有高浓度的细菌,从而进入下呼吸道深部。

(3)远处感染灶的血行播散 细菌性肺炎也可能是远处感染灶通过血行播散所致,这种情况较为少见。动物实验显示,细菌可从胃肠道经由上皮黏膜进入肠系膜淋巴结,最终至肺(细菌移位)。胃肠道细菌移位导致医院获得性肺炎可能发生于免疫抑制、肿瘤和烧伤等患者。

\subsection{临床诊断}

\subsubsection{呼吸机相关性肺炎的诊断标准是什么?}

目前呼吸机相关性肺炎的诊断标准尚不统一。常用的美国国家医院感染监测系统关于呼吸机相关性肺炎的诊断标准包含X线胸片、临床和微生物几个方面(表\ref{tab8-2})。

\begin{table}[htbp]
\centering
\caption{美国国家医院感染监测系统关于呼吸机相关性肺炎的诊断标准}
\label{tab8-2}
\includegraphics{./images/Image00060.jpg}
\end{table}

\subsubsection{呼吸机相关性肺炎的临床诊断标准应用时有哪些注意事项?}

常用的呼吸机相关性肺炎临床诊断标准包括X线胸片上新出现浸润阴影或原有浸润阴影扩大,同时具有下列三项中的两项及以上:①体温>38℃;②白细胞计数增高或降低;③脓性痰。临床操作比较简便,但在具体实践中因无统一的标准和主观差异导致诊断的敏感性和特异性差异很大。

诊断标准强调X线胸片和临床的表现,但二者均不特异。根据体温、白细胞计数和痰的性质很难区分肺部感染和化脓性气管支气管炎。在机械通气的患者,由于急性呼吸窘迫综合征(ARDS)和其他弥漫性肺损伤,临床表现更缺乏特异性。研究表明,肺炎在ARDS患者的急性期非常普遍却常常不被认识。另外,危重患者肺部出现浸润影应注意同肺不张、ARDS、肺栓塞、氧中毒及心力衰竭等进行鉴别。

此外,没有任何临床表现的患者不代表没有肺炎。尸检研究常常发现没有肺炎临床表现的患者存在肺炎而这部分患者并未接受抗菌药物治疗,这提示临床的主观印象可能并不准确,呼吸机相关性肺炎的临床诊断标准(X线胸片异常结合临床表现)可以进行呼吸机相关性肺炎的初筛,但是由于特异性较差,需要采用其他方法(如下呼吸道分泌物的涂片、培养等确定致病菌)和临床肺部感染评分等协助诊断。

\subsubsection{呼吸机相关性肺炎有哪些病原学诊断方法?}

呼吸机相关性肺炎病原学诊断方法包括:①气管内吸引;②经纤维支气管镜方法采样,如支气管肺泡灌洗、保护性毛刷;③血培养和胸腔积液培养;④经纤维支气管镜肺活检和开胸肺活检;⑤尸检;⑥其他,如盲法保护性毛刷、盲法支气管肺泡灌洗等
\protect\hyperlink{text00014.htmlux5cux23ch11-13}{\textsuperscript{{[}11{]}}}
。尤以前三种方法临床上常用。

对于机械通气患者,利用气管内吸引留取标本进行涂片和培养,操作简单,在床旁即可操作,无需复杂的培训。若每个低倍视野下的多形核白细胞不少于25个,上皮细胞不多于10个,尤其当镜下发现大量形态一致的致病菌时,提示下呼吸道存在细菌感染。涂片的结果往往可以为临床更早提供病原学参考。气道内吸引标本培养>10\textsuperscript{6}
cfu/ml,则诊断呼吸机相关性肺炎的敏感性为38%~91%,特异性为59%~92%。但对于感染、定植和污染的鉴别有时仍很困难。

通过支气管镜进行支气管肺泡灌洗和保护性毛刷检查,可直接从下呼吸道取材,而不易被上呼吸道或口腔分泌物污染。当支气管肺泡灌洗液培养结果>10\textsuperscript{4}
cfu/ml、保护性毛刷标本培养结果>10\textsuperscript{3}
cfu/ml时可诊断为呼吸机相关性肺炎。经支气管镜采样诊断有较高的特异性,但在近期使用或更换过抗生素的情况下有可能出现假阴性结果。

血培养对诊断和预后评价有一定价值,建议同时进行检查,但阳性率仅6%。选择合适的采血时间、足够的血量和次数可能有助于提高阳性率。

\subsubsection{保护性毛刷对诊断呼吸机相关性肺炎的意义如何?}

1979年保护性毛刷开始在临床应用,最初用于肺炎诊断,也可用于呼吸机相关性肺炎的诊断。通常采用10\textsuperscript{3}
cfu/ml作为临界值(大致相当于感染部位的密度为10\textsuperscript{6}
cfu/ml)来区分下呼吸道感染与口咽部或气管细菌定植。保护性毛刷的敏感性为40%~100%(多数报道在60%~90%),特异性50%~80%。

然而,保护性毛刷也存在下列问题:①口咽部细菌的污染导致假阳性结果;②应用抗菌药的患者细菌数可低于10\textsuperscript{3}
cfu/ml;③支气管炎患者因细菌负荷较少,保护性毛刷很少高于10\textsuperscript{3}
cfu/ml;④对于保护性毛刷的重复性研究显示,在25%的致病菌和多至40%的患者中,多次保护性毛刷的结果并不一致;⑤培养结果需在24~48小时后才能得到;⑥对临界值的确定仍存在疑问。

相当多的研究对保护性毛刷和其他有创诊断方法(多为支气管肺泡灌洗)的准确性进行了比较,支气管肺泡灌洗敏感性较高,而保护性毛刷特异性较高,但总体结果并未发现其中哪种方法更为优越。

总之,在诊断呼吸机相关性肺炎方面,保护性毛刷的特异性超过其敏感性,有条件或诊断困难时可以使用。

\subsubsection{如何评价支气管肺泡灌洗对诊断呼吸机相关性肺炎的意义?}

支气管肺泡灌洗是指在纤维支气管镜直接插至下呼吸道采集炎症部位标本,以及对支气管以下肺段或亚肺段水平反复以无菌生理盐水灌洗、回收,并对其进行一系列检测和分析。

1988年以后,支气管肺泡灌洗开始用于呼吸机相关性肺炎的诊断。与保护性肺毛刷相比,支气管肺泡灌洗更为简便和安全,而且能够对更大范围的肺组织留取标本,目前多以10\textsuperscript{4}
cfu/ml作为诊断呼吸机相关性肺炎的临界值。另外,Johanson等人根据狒狒的支气管肺泡灌洗标本培养结果中的细菌菌落计数,定义细菌指数(bacterial
index,BI),并发现支气管肺泡灌洗的BI值与肺组织培养后菌落计数结果呈正相关,从而提出以细菌指数>5.0区分肺部感染和细菌定植。对支气管肺泡灌洗标本进行定量培养并计算含有致病菌的细胞百分比(以25%以上的细胞中含有细菌作为诊断分界线),发现支气管肺泡灌洗的诊断敏感性显著高于保护性毛刷。可见,支气管肺泡灌洗能够在定量和定性两个方面反映肺内细菌感染的情况。

支气管肺泡灌洗诊断呼吸机相关性肺炎的敏感性差异很大,平均为(73±18)%,特异性平均为(82±19)%。造成差异的原因包括既往使用抗菌药、研究人群以及采取的参照诊断标准等。值得注意的是,计算的敏感性与定量培养的临界值密切相关。另外,支气管肺泡灌洗的临床应用仍存在一些问题,如诊断标准尚未完全统一,操作步骤尚未标准化,与保护性毛刷结果的一致性较差等,尚有待于进一步研究解决。

近来主张将支气管肺泡灌洗与保护性毛刷相结合,从而使诊断的敏感性和特异性显著提高。

\subsubsection{临床肺部感染评分在呼吸机相关性肺炎诊断中的意义如何?}

临床肺部感染评分是一项综合了临床、影像学和微生物学标准等来评估感染严重程度,协助指导抗菌药物调整的评分系统,对诊断、治疗和评价肺炎患者有一定的意义。

临床肺部感染评分指标包括体温、白细胞计数、气管分泌物、氧合情况、X线胸片和气管吸取物培养,最高评分为12分(表\ref{tab8-3})\footnote{*总分为12分,机械通气情况下临床肺部感染评分>6分提示存在呼吸机相关性肺炎。}。接受机械通气的重症医学科患者临床肺部感染评分>6分即可被诊断为呼吸机相关性肺炎,其与支气管肺泡灌洗诊断有较好的相关性,有研究表明其相关性为0.84。鉴于临床肺部感染评分标准中气道分泌物半定量分析在临床实际工作中有时应用困难,Luma
\protect\hyperlink{text00014.htmlux5cux23ch12-13}{\textsuperscript{{[}12{]}}}
在研究中提到简化的临床肺部感染评分,更便于临床操作,其指标包括体温、血白细胞、气道分泌物、氧合指数和X线胸片,共计10分,≥5分可诊断呼吸机相关性肺炎,联合痰涂片可提高诊断的准确性(表\ref{tab8-4})。\footnote{*
总分为10分,机械通气情况下临床肺部感染评分≥5分提示存在呼吸机相关性肺炎。}

\begin{table}[htbp]
\centering
\caption{诊断呼吸机相关性肺炎的临床肺部感染评分标准\textsuperscript{*}}
\label{tab8-3}
\includegraphics{./images/Image00061.jpg}
\end{table}

\begin{table}[htbp]
\centering
\caption{诊断呼吸机相关性肺炎的简化临床肺部感染评分标准\textsuperscript{*}}
\label{tab8-4}
\includegraphics{./images/Image00062.jpg}
\includegraphics{./images/Image00063.jpg}
\end{table}

持续评价临床肺部感染评分可评估呼吸机相关性肺炎患者的临床转归。一项对427例接受机械通气>72小时患者的研究表明,与发病前3天相比,所有呼吸机相关性肺炎患者发病时的临床肺部感染评分均显著升高。存活组患者呼吸机相关性肺炎发病前3天及发病时的临床肺部感染评分与死亡组相似,但在发病后的第3、5及7天,存活组的临床肺部感染评分显著低于死亡组。氧合指数与转归的相关性最好,在治疗3天后,适当抗生素治疗组的氧合指数>250mmHg,不适当治疗组则持续下降。不过,在近期的研究中,分别计算第1、3天的临床肺部感染评分变化,与呼吸机相关性肺炎患者的病死率并无相关性
\protect\hyperlink{text00014.htmlux5cux23ch13-13}{\textsuperscript{{[}13{]}}}
。

在2005年美国胸科协会(ATS)和美国感染病协会(IDSA)指南中建议应用临床肺部感染评分作为提高临床诊断特异性的工具,协助呼吸机相关性肺炎的诊断和指导抗生素的调整,在保证疗效的前提下尽量缩短抗菌药物的疗程。

\subsubsection{如何评价髓样细胞表达的可溶性触发受体I在呼吸机相关性肺炎诊断中的意义?}

髓样细胞表达的可溶性触发受体I属于免疫球蛋白家族,在中性粒细胞、成熟单核细胞和巨噬细胞表面表达。其能够协同Toll样受体增强微生物感染所致的炎症反应,但对于非感染性免疫复合物所致的炎症反应无促进作用。有研究显示,通过检测支气管肺泡灌洗(BAL)液中的髓样细胞表达的可溶性触发受体I来诊断细菌性呼吸机相关性肺炎,比临床肺部感染评分和降钙素原更准确。应用髓样细胞表达的可溶性触发受体I判断是否存在肺炎的受试者工作特征曲线下面积达0.93,多因素逻辑回归分析提示髓样细胞表达的可溶性触发受体I是判断是否存在肺炎的最重要独立因素,比值比高达41.5。以肺泡灌洗液中髓样细胞表达的可溶性触发受体I浓度200pg/ml作为诊断标准,其灵敏度74%,特异度88%。

因此,检测肺泡灌洗液中的髓样细胞表达的可溶性触发受体I浓度可作为诊断肺炎尤其是诊断呼吸机相关性肺炎的重要手段之一。

\subsubsection{降钙素原是否可以用于呼吸机相关性肺炎的诊断?}

降钙素原是由116个氨基酸组成的无活性的降钙素前体,由甲状腺C细胞合成。健康成人血液中浓度极低,低于0.1mg/L。降钙素原在全身感染中的病理生理学作用机制尚不清楚。细菌感染患者降钙素原血浆浓度增高,可能是细菌毒素直接作用结果,也可能是致炎因子介导的间接反应。实验研究认为降钙素原可能是全身感染导致炎症因子产生过程中的中间产物。目前认为血浆降钙素原浓度高于0.25mg/L,可作为诊断呼吸机相关性肺炎和开始抗菌药物治疗的辅助指标。

\subsubsection{如何诊断重症医院获得性肺炎?}

除确定医院获得性肺炎诊断外,认识其严重程度对于经验性选择抗菌药物,进行支持性治疗以及估计患者预后也非常重要。美国胸科协会提出了重症医院获得性肺炎的诊断标准(表\ref{tab8-5})。

\begin{table}[htbp]
\centering
\caption{重症医院获得性肺炎的诊断标准}
\label{tab8-5}
\includegraphics{./images/Image00064.jpg}
\end{table}

\subsection{治疗与预防}

\subsubsection{什么叫合适抗生素治疗?}

2005年美国胸科协会和美国感染病协会制定的医院获得性肺炎和呼吸机相关性肺炎防治指南中,对合适(adequate)抗生素治疗做出了新的定义。对于明确的感染,在进行抗感染治疗时,适当治疗应包括以下4个方面:①选择正确抗生素,即病原菌敏感的抗生素;②使用最佳的抗生素剂量和疗程;③给药途径正确,确保药物渗透到感染部位;④必要时联合用药。只有同时满足上述4个条件,相应的抗生素治疗才是合适的治疗。

\subsubsection{经验性抗菌药物治疗与呼吸机相关性肺炎患者的预后有何关系?}

临床研究表明,早期正确的抗菌药物治疗能够使呼吸机相关性肺炎患者的病死率下降至少一半。此外,有资料显示,正确抗菌药物治疗是否及时也影响呼吸机相关性肺炎患者的预后。早期(进行纤维支气管镜检查前)即接受正确抗菌药物治疗的呼吸机相关性肺炎患者的病死率最低,对于那些使用了错误的经验性治疗的患者,即使后期根据微生物学资料对药物进行调整,也不能降低患者的病死率(分别为71%、70%)。Iregui等研究提示,即使在达到呼吸机相关性肺炎诊断标准后仅延迟合理应用抗菌药物16小时,病死率仍然增加40%。

由于呼吸机相关性肺炎的诊断非常困难,因此在临床高度怀疑呼吸机相关性肺炎时,立即开始正确的经验性抗菌药物治疗就显得非常关键。此外,经验性选择抗菌药物时,需要考虑到患者的基础情况、宿主因素(疾病的严重程度和并发症)、住院时间、既往抗菌药物应用情况、医院或重症医学科中细菌耐药现状等诸多因素,必要时通过联合用药以力求覆盖所有可能的病原体。

因此,及时正确应用抗菌药物是治疗呼吸机相关性肺炎的基石,初始经验性抗菌药物的选择非常重要,因其可以影响患者预后。

\subsubsection{治疗呼吸机相关性肺炎应如何经验性选择抗菌药物?}

2005年美国胸科协会/美国感染病协会制定的医院获得性肺炎和呼吸机相关性肺炎防治指南中,主要根据发病时间的早晚和是否存在多药耐药危险因素决定初始经验性抗菌药物的选择(表\ref{tab8-6}~表\ref{tab8-8})。

\begin{table}[htbp]
{\centering
\caption{已知危险因素且无多药耐药的早发性医院获得性肺炎和呼吸机相关性肺炎患者的初始经验性抗菌药物治疗}
\label{tab8-6}
\includegraphics{./images/Image00065.jpg}}

\footnotesize 
* 参照表\ref{tab8-8}选择合适的初始剂量。

**
青霉素耐药的肺炎链球菌和多药耐药的肺炎链球菌在不断增加;左氧氟沙星和莫西沙星优于环丙沙星,其他新型喹诺酮如加替沙星的地位尚不明确。
\end{table}



\begin{table}[htbp]
{\centering
\caption{存在多药耐药危险因素的晚发性重症医院获得性肺炎或呼吸机相关性肺炎患者的初始经验性抗菌药物治疗}
\label{tab8-7}
\includegraphics{./images/Image00066.jpg}
\includegraphics{./images/Image00067.jpg}
}

\footnotesize
*
参照表\ref{tab8-8}选择适当的初始剂量,并根据微生物学结果和临床治疗反应及时调整初期的抗菌药物。

**
如果超广谱β内酰胺酶阳性,且考虑可能是肺炎克雷伯杆菌或不动杆菌感染,碳青霉烯是个可信赖的选择;如果考虑存在嗜肺军团菌感染可能,联合使用一种大环内酯类(如阿奇霉素)或氟喹诺酮类药物(如环丙沙星、左氧氟沙星)治疗优于使用氨基糖苷类。

***
如果存在耐甲氧西林金黄色葡萄球菌危险因素或耐甲氧西林金黄色葡萄球菌在当地有很高的发病率,应联合使用。
\end{table}





\begin{table}[htbp]
\centering
\caption{晚发性或多药耐药病原菌引起的医院获得性肺炎、呼吸机相关性肺炎、卫生保健相关性肺炎的初始经验性抗菌药物的成人静脉给药剂量}
\label{tab8-8}
\includegraphics{./images/Image00068.jpg}
\end{table}

* 推荐的剂量是基于正常的肝肾功能。

**
庆大霉素和妥布霉素的谷浓度应低于1μg/ml,阿米卡星的谷浓度应低于4~5μg/ml。

*** 万古霉素的谷浓度在15~20μg/ml。

\subsubsection{如何评价抗菌药物的联合用药在呼吸机相关性肺炎治疗中的意义?}

与单一用药相比,联合用药具有以下优点:①防止耐药细菌的产生;②药物之间可能具有协同和相加作用。

通常,当治疗耐药细菌(如铜绿假单胞菌、多药耐药的不动杆菌、肠杆菌和沙雷菌)引起的严重感染时,需要联合用药。常见的联合用药包括:①氨基糖苷类和β内酰胺类;②氨基糖苷类和喹诺酮类;③喹诺酮类和β内酰胺类。但实际上目前支持联合用药的研究很少。抗菌药物治疗假单胞菌感染的协同作用仅表现在体外试验中,临床治疗也仅能改善中性粒细胞减少和菌血症患者的预后,而这在呼吸机相关性肺炎中并不常见。一项大样本荟萃分析(选择前瞻性随机研究)比较联合β内酰胺类和氨基糖苷类与单用β内酰胺类治疗严重全身感染患者(7586例患者中1200例呼吸机相关性肺炎),结果显示联合用药治疗铜绿假单胞菌感染无任何优势,进一步分析表明联合用药同样不能预防耐药发生,反而使肾毒性明显增加。由此可见对明确的致病菌,联合用药的利弊尚需更多的临床资料去进一步证实。

联合用药目前更多用于初始经验性选择抗菌药物时,为覆盖多药耐药病原菌和可能的混合感染菌的需要。由于联合用药价格昂贵,且让患者暴露于许多不必要的抗菌药物中,有可能导致多药耐药病原菌的发生和不良的预后,因此应尽早明确病原菌,并结合治疗的反应,如有可能应尽早停用不必要的药物(如对治疗临床反应好的患者氨基糖苷类治疗5~7天后停用),或改为单药治疗。对于无多药耐药危险性的、早发性的呼吸机相关性肺炎应首选合适的抗菌药物单一治疗。

\subsubsection{治疗鲍曼不动杆菌感染的常用抗菌药物有哪些?}

(1)舒巴坦及含舒巴坦的β内酰胺类抗生素的复合制剂 舒巴坦对不动杆菌属细菌具抗菌作用,故含舒巴坦的复合制剂对不动杆菌具良好的抗菌活性。对于一般感染,舒巴坦的常用剂量不超过4.0g/天,而多重耐药菌所致感染,国外推荐剂量可增加至6.0g/天,分3~4次给药。肾功能减退患者须调整给药剂量。

(2)碳青霉烯类抗生素 可用于敏感菌所致的各类感染,或与其他药物联合治疗多重耐药菌所致感染。亚胺培南和美罗培南的剂量常需1.0g,每8小时一次或1.0g,每6小时一次,静脉滴注。药代动力学研究显示,对于一些敏感性下降的菌株(最低抑菌浓度4~16mg/L),通过增加给药次数、加大给药剂量、延长静脉滴注时间(如每次静滴时间延长至2~3小时),可使血药浓度高于最低抑菌浓度的时间延长,部分感染病例有效,但目前尚缺乏大规模临床研究。

(3)多黏菌素类抗生素 分为多黏菌素B及多黏菌素E(colistin,粘菌素),临床应用的多为多黏菌素E。国际上推荐多黏菌素E的剂量为每天2.5~5mg/kg或每天200万~400万U(100万U相当于多黏菌素E甲磺酸盐80mg),分2~4次静脉滴注。该类药物的肾毒性及神经系统不良反应发生率高,对于老年人、肾功能不全患者特别需要注意肾功能的监测。另外,多黏菌素E存在明显的异质性耐药,常需联合应用其他抗菌药物。

(4)替加环素(tigecycline) 为甘氨酰环素类抗菌药物,甘氨酰环素类为四环素类抗菌药物米诺环素的衍生物。早期研究发现其对全球分离的碳青霉烯类抗生素耐药鲍曼不动杆菌的最低抑菌浓度\textsubscript{90}
为2mg/L。近期各地报告的敏感性差异大,耐药菌株呈增加趋势,常需根据药敏结果选用。由于其组织分布广泛,血药浓度、脑脊液浓度低,常需与其他抗菌药物联合应用。美国FDA批准该药的适应证为复杂性腹腔及皮肤软组织感染、社区获得性肺炎。常用给药方案为首剂100mg,之后50mg每12小时一次静脉滴注。主要不良反应为胃肠道反应。

(5)四环素类抗菌药物 美国FDA批准米诺环素针剂用于敏感鲍曼不动杆菌感染的治疗,给药方案为米诺环素100mg,每12小时一次静脉滴注,但临床研究不多。国内目前无米诺环素针剂,可使用口服片剂或多西环素针剂(100mg,每12小时一次)与其他抗菌药物联合治疗鲍曼不动杆菌感染。

(6)氨基糖苷类抗生素 这类药物多与其他抗菌药物联合治疗敏感鲍曼不动杆菌感染。国外推荐剂量阿米卡星或异帕米星每天15~20mg/kg,国内常用0.6g每天一次静脉滴注给药,对于严重感染且肾功能正常者,可加量至0.8g/天给药。用药期间应监测肾功能及尿常规,最好监测血药浓度。

(7)其他 对鲍曼不动杆菌具抗菌活性的其他抗菌药物尚有喹诺酮类抗菌药物、第三及第四代头孢菌素如头孢他啶、头孢吡肟,其他β内酰胺酶抑制剂的复合制剂如哌拉西林/他唑巴坦,但耐药率高,达64.1%~68.3%,故应根据药敏结果选用。体外及动物体内研究显示,利福平与其他抗菌药联合对不动杆菌有协同杀菌作用,因其为治疗结核病的主要药物之一,不推荐常规用于鲍曼不动杆菌感染的治疗。

\subsubsection{如何合理选择抗菌药物治疗鲍曼不动杆菌感染?}

(1)非多重耐药鲍曼不动杆菌感染 可根据药敏结果选用β内酰胺类抗生素等抗菌药物。

(2)多重耐药鲍曼不动杆菌感染 根据药敏选用头孢哌酮/舒巴坦、氨苄西林/舒巴坦或碳青霉烯类抗生素,可联合应用氨基糖苷类抗生素或氟喹诺酮类抗菌药物等.

(3)广泛耐药鲍曼不动杆菌感染 常采用两药联合方案,甚至三药联合方案。两药联合用药方案有------①以舒巴坦或含舒巴坦的复合制剂为基础的联合,联合以下一种:米诺环素(或多西环素)、多黏菌素E、氨基糖苷类抗生素、碳青霉烯类抗生素等;②以多黏菌素E为基础的联合,联合以下一种:含舒巴坦的复合制剂(或舒巴坦)、碳青霉烯类抗生素;③以替加环素为基础的联合,联合以下一种:含舒巴坦的复合制剂(或舒巴坦)、碳青霉烯类抗生素、多黏菌素E、喹诺酮类抗菌药物、氨基糖苷类抗生素。三药联合方案有:含舒巴坦的复合制剂(或舒巴坦)+多西环素+碳青霉烯类抗生素、亚胺培南+利福平+多黏菌素或妥布霉素等。

(4)全耐药鲍曼不动杆菌感染 常需通过联合药敏试验筛选有效的抗菌药物联合治疗方案。研究发现,鲍曼不动杆菌易对多黏菌素产生异质性耐药,但异质性耐药菌株可部分恢复对其他抗菌药物的敏感性,因此多黏菌素联合β内酰胺类抗生素或替加环素是可供选择的方案,但尚缺少大规模临床研究。也可结合抗菌药物药代动力学参数要求,尝试通过增加给药剂量、增加给药次数、延长给药时间等方法设计给药方案。

\subsubsection{控制导管生物被膜对防治呼吸机相关性肺炎有何意义?}

细菌生物被膜(biofilm,BF)指细菌粘附于固体或有机腔道表面,形成微菌落,并分泌多糖蛋白复合物将自身包裹其中而形成的膜状物。目前认为细菌BF是导致某些慢性感染反复发作难以治愈的重要原因。其机制包括:①阻滞抗菌药物的渗透;②吸附抗菌药物灭活酶,促进抗菌药物水解;③被膜下细菌代谢低下,呈“亚冬眠状态”,对抗菌药物敏感性低;④阻滞机体免疫系统对细菌的清除,产生免疫逃逸现象,减弱机体免疫力与抗菌药物的协同杀菌作用。临床上容易形成生物被膜的致病菌主要有铜绿假单胞菌、金黄色葡萄球菌、表皮葡萄球菌、大肠埃希菌等。

有研究表明,气管导管表面的细菌生物被膜是导致呼吸机相关性肺炎发生和病情反复的重要原因之一,临床应创造条件尽早拔除气管导管,以减少导管内外细菌生物被膜的形成。国外有人从事抗定植材料的研究,但目前由这种材料制成的导管尚未面世。有报道氟喹诺酮类和大环内酯类(14-元环或者5-元环)抗菌药物可抑制细菌生物被膜的形成,并破坏已形成的细菌生物被膜。

\subsubsection{呼吸机相关性肺炎的非抗生素防治措施有哪些?}

呼吸机相关性肺炎的发生增加患者的住院时间、费用,病死率也明显增加。在抗菌药物抗感染的同时,我们更需关注非抗生素的抗感染防治措施,这有时比抗菌药物更为重要,尤其在病原体为多药耐药或泛耐药的时候。而且实施以下防治策略时应建议采取综合性、集束化(bundle)的非抗生素策略,其主要措施有------①一般性措施,包括手部清洁,戴手套和穿隔离衣,洗必泰口腔护理;②与消化道相关控制策略包括合适的应激性溃疡预防,避免胃过度扩张,避免长时间留置经鼻胃管,应用较细的营养管路,早期的胃造瘘和应用空肠营养;③与患者体位相关策略:保持半卧位(30°~45°),应用动力翻身床,不常规推荐俯卧位;④与人工气道相关策略:避免经鼻气管插管,维持合适的气囊压力(20cm
H\textsubscript{2} O以上,一般25~35cm H\textsubscript{2}
O)和持续声门下吸引;⑤机械通气相关策略:定期的呼吸机设备的清洁,避免不必要频繁更换呼吸机管路,避免过度镇静,每日间断唤醒,减少机械通气时间和尽早脱机;⑥其他措施:合适的血糖控制,控制在8.3mmol/L以下,限制制酸药的使用,增强患者免疫功能。

\subsubsection{如何评价洗必泰口腔护理对于呼吸机相关性肺炎防治的意义?}

在口腔和牙菌斑积聚的细菌进入下呼吸道是导致呼吸机相关性肺炎的重要原因。洗必泰溶液可以控制牙菌斑上细菌生长。临床研究表明,与常规口腔护理相比,对心脏外科术后患者用洗必泰漱口,实施口咽部去污染,可使医院获得性感染的发生率从13.3%降低到4.6%(\emph{P}
<0.01),治疗性抗生素应用也明显降低(23.3%对比13.3%,\emph{P}
<0.05)。更值得注意的是,洗必泰漱口去污染组患者的病死率为1.16%,明显低于常规口腔护理组(5.56%,\emph{P}
<0.05)
\protect\hyperlink{text00014.htmlux5cux23ch7-13}{\textsuperscript{{[}7{]}}}
。另外,实施洗必泰口咽部去污染对抗生素耐药致病菌的局部定植也有明显预防作用。

洗必泰口腔护理简单可行、费用不高,对高危患者应常规使用。

\subsubsection{机械通气患者如何进行应激性溃疡的预防?}

一般认为,危重患者特别是机械通气患者,是上消化道出血的危险人群,应用抑酸药(H\textsubscript{2}
受体阻滞剂或质子泵抑制剂)提高胃液pH值成为预防上消化道出血的常用措施。但胃是口咽部革兰阴性定植菌的主要来源。健康人胃内pH低于2,基本处于无菌状态。但当胃内pH高于4时,微生物即在胃内大量繁殖。研究表明,当pH增加到6.0时,胃液内菌落可增至10\textsuperscript{7}
cfu/ml以上,成为细菌侵入下呼吸道的潜在感染源。因此,避免使用抑酸药,避免胃液pH值的升高,将有助于预防呼吸机相关性肺炎。而硫糖铝口服或鼻饲后,对胃黏膜具有保护作用,能够预防上消化道出血的发生,同时硫糖铝对胃液pH值无明显影响,已有研究显示,与H\textsubscript{2}
受体阻滞剂相比可以降低呼吸机相关性肺炎的发生率。

近年多中心随机双盲对照研究(\emph{n}
=1200)显示,危重病患者应用H\textsubscript{2}
受体拮抗剂雷尼替丁预防应激性溃疡,危及生命的上消化道出血发生率为1.7%;胃黏膜保护剂硫糖铝组发生率为3.8%,H\textsubscript{2}
受体拮抗剂明显优于硫糖铝,但两组呼吸机相关性肺炎发生率分别为19.1%和16.2%,无统计学差异。

因此,对于出血倾向小的患者可建议常规应用硫糖铝进行应激性溃疡的预防;当存在危及生命的上消化道出血风险时则不推荐单独应用硫糖铝,可使用抑酸药预防应激性溃疡。

\subsubsection{如何评价热湿交换器与加温湿化器在呼吸机相关性肺炎防治中的意义?}

人工气道的湿化非常重要,目前临床上常用的湿化装置有加温湿化器(heated
humidifier)和热湿交换器(heated moisture exchanger)。Dedek等
\protect\hyperlink{text00014.htmlux5cux23ch3-13}{\textsuperscript{{[}3{]}}}
在基于循证医学的呼吸机相关性肺炎预防指南中指出:对于无禁忌证(如咯血或高分钟通气量通气)的患者推荐使用热湿交换器进行湿化,并建议每周更换热湿交换器。而2005年美国胸科协会和美国感染病协会关于院内获得性肺炎和呼吸机相关性肺炎指南中建议的不同,即被动式加湿器或热湿交换器能减少呼吸机管路的细菌定植,但并未减少呼吸机相关性肺炎的发生率,因此不能将其作为肺炎的预防措施。近期研究也显示热湿交换器并不降低呼吸机相关性肺炎的发生率,且在Lorente等
\protect\hyperlink{text00014.htmlux5cux23ch10-13}{\textsuperscript{{[}10{]}}}
对104例机械通气患者随机对照研究中发现,机械通气5天以上,患者使用加温湿化器的呼吸机相关性肺炎发生率低于热湿交换器,回归分析显示热湿交换器是呼吸机相关性肺炎的独立危险因素。

因此,目前不建议使用热湿交换器防治呼吸机相关性肺炎,但鉴于其使用方便,在无禁忌证的情况下可用于短期机械通气患者的湿化。

\subsubsection{如何评价无创通气在呼吸机相关性肺炎防治中的意义?}

无创通气可以避免气管插管和气管切开引起的并发症,保留了上呼吸道的防御能力。对于部分合并免疫抑制的急性肺损伤和急性呼吸窘迫综合征患者,使用无创通气有助于避免呼吸机相关性肺炎的发生。另有部分研究显示,与有创机械通气相比,给予无创通气治疗,可明显减少抗生素用量、缩短重症医学科住院时间,并最终能够降低患者的病死率。目前认为,对于慢性阻塞性肺疾病急性加重期、急性呼吸衰竭早期、急性心源性肺水肿和免疫功能低下的患者,如无禁忌可首先考虑无创通气。但是,对于重症急性呼衰患者,无创通气既不降低插管率,也不改善预后,甚至可能由于气道没有保障而加重病情。因此,把握无创通气的应用指征和转为有创通气的时机非常重要。

\begin{center}\rule{0.5\linewidth}{\linethickness}\end{center}

参考文献

\protect\hyperlink{text00014.htmlux5cux23ch1-13-back}{{[}1{]}}
.American Thoracic Society Documents:Guidelines for the management of
adults with hospital-acquired,ventilator-associated,and
healthcare-associated pneumonia.Am J Respir Crit Care
Med,2005,171:388-416.

\protect\hyperlink{text00014.htmlux5cux23ch2-13-back}{{[}2{]}}
.Dellinger RP,Carlet JM,Masur H,et al.Surviving Sepsis Campaign
guidelines for management of severe sepsis and septic shock.Crit Care
Med,2004,32:858-873.

\protect\hyperlink{text00014.htmlux5cux23ch3-13-back}{{[}3{]}} .Dodek
P,Keenan S,Cook D,et al.Evidence-based clinical practice guideline
for the prevention of ventilator-associated pneumonia.Ann Intern
Med,2004,141:305-313.

\protect\hyperlink{text00014.htmlux5cux23ch4-13-back}{{[}4{]}}
.Porzecanski I,Bowton DL.Diagnosis and treatment of
ventilator-associated pneumonia.Chest,2006,130:597-604.

\protect\hyperlink{text00014.htmlux5cux23ch5-13-back}{{[}5{]}}
.Gujadhur R,Helme BW,Sanni A,et al.Continuous subglottic suction is
effective for prevention of ventilator-associated pneumonia.Interact
Cardio Vasc Thorac Surg,2005,4:110-115.

\protect\hyperlink{text00014.htmlux5cux23ch6-13-back}{{[}6{]}}
.Kostadima E,Kaditis G,Alexopoulos I,et al.Early gastrostomy
reduces the rate of ventilator-associated pneumonia in stroke or head
injury patients.Eur Respir J,2005,26:106-111.

\protect\hyperlink{text00014.htmlux5cux23ch7-13-back}{{[}7{]}} .Koeman
M,van der Ven AJAM,Hak E,et al.Oral decontamination with
chlorhexidine reduces the incidence of ventilator-associated
pneumonia.Am J Respir Crit Care Med,2006,173:1348-1355.

\protect\hyperlink{text00014.htmlux5cux23ch8-13-back}{{[}8{]}} .Macleod
R,Bucknall T.Mechanical ventilation with heated humidifiers or with
heat and moisture exchangers with microbiological filters did not reduce
ventilator associated pneumonia in adults.Evid Based
Nurs,2006,9:82.

{[}9{]}.Lacherade JC,Auburtin M,Cerf C,et al.Impact of
humidification systems on ventilator-associated pneumonia:A randomized
multicenter trial.Am J Respir Crit Care Med,2005,172:1276-1282.

\protect\hyperlink{text00014.htmlux5cux23ch10-13-back}{{[}10{]}}
.Lorente L,lecuona M,Jimenez A,et al.Ventilator-associated
pneumonia using a heated humidifier or a heat and moisture exchanger:a
randomized controlled trial.Crit Care,2006,10:R116.

\protect\hyperlink{text00014.htmlux5cux23ch11-13-back}{{[}11{]}}
.Fujitani S,Aspirates E,Shigeki,et al. Diagnosis of
ventilator-associated pneumonia:focus on nonbronchoscopic
techniques(nonbronchoscopic bronchoalveolar lavage,including
mini-BAL,blinded protected specimen brush,and blinded bronchial
sampling)and endotracheal aspirates.J Intensive Care
Med,2006,21:17-21.

\protect\hyperlink{text00014.htmlux5cux23ch12-13-back}{{[}12{]}} .Luna
C,Niederman MS,Baredes NC,et al.Resolution of ventilator-associated
pneumonia:prospective evaluation of the clinical pulmonary infection
score as an early clinical predictor of outcome.Crit Care
Med,2003,31:676-682.

\protect\hyperlink{text00014.htmlux5cux23ch13-13-back}{{[}13{]}}
.Khaleeq G,Garcha P,Hirani A,et al.Clinical pulmonary infection
score(CPIS)relationship to mortality in patients with ventilator
associated pneumonia.Chest Meeting Abstracts,2006,130:218S-219S.

\protect\hypertarget{text00015.html}{}{}


\chapter{氧疗与人工气道管理}

\section{前沿学术综述}

\subsubsection{氧气疗法}

氧气是机体组织细胞能量代谢所必需的物质。必须有充足的氧气,细胞才能维持其生理功能。机体对氧气的生理需求和缺氧对机体的危害,使临床医师充分认识到氧气的重要性,特别是在危重患者的救治中,氧疗具有重要的治疗作用。

(1)氧疗的目的

纠正低氧血症:增加吸入氧浓度,提高肺泡氧分压,可不同程度地纠正低张性低氧血症。正常情况下,从大气到组织细胞各个水平的氧分压值见图\ref{fig9-1},20mmHg为细胞无氧代谢阈值,氧分压<20mmHg,组织细胞即开始无氧代谢。纠正低氧血症对防止组织缺氧具有重要意义。

\begin{figure}[!htbp]
 \centering
 \includegraphics{./images/Image00069.jpg}
 \captionsetup{justification=centering}
 \caption{从大气到组织细胞各个水平的氧分压值梯度}
 \label{fig9-1}
  \end{figure} 

降低呼吸功:低氧血症和缺氧及其引起的酸中毒刺激呼吸中枢,作为代偿性反应,呼吸频率加快、通气量增加,引起呼吸肌做功增加,结果呼吸氧耗增加,可能形成恶性循环,导致低氧血症加重。提高吸入氧浓度,可降低机体对通气的需要,从而降低呼吸功。

减少心肌做功:低氧血症或缺氧可引起心血管系统发生代偿性反应,使心率增快、心输出量增加、外周血管收缩、血压升高,其结果是心肌做功增加,心肌氧耗增加,可能加重心肌的氧供和氧需的失衡。提高吸入氧浓度,可纠正低氧血症,缓解心血管系统的代偿性反应,减少心肌做功。

(2)氧疗的装置 根据氧疗系统提供的气体是否能够满足患者吸气的需要,一般将氧疗装置分为高流量系统和低流量系统。值得注意的是,高流量与低流量并不等同于高浓度与低浓度吸氧。不同氧疗装置氧流量与吸入氧浓度之间的关系不同。

(3)氧疗效果的评价 氧疗的目的不仅在于纠正低氧血症,更为重要的是维持心血管系统和呼吸系统的功能,保证组织器官足够的氧供。氧疗实际上是纠正组织缺氧的重要手段之一。因此,对氧疗的评价不仅包括对器官功能进行评估,而且也应包括器官组织氧代谢的评估。

(4)氧疗的副作用 氧疗在临床治疗中有重要作用,但临床医师对氧气的毒性普遍认识不足。氧气实际上也是一种“药物”,不但应注意其使用剂量,还应注意其毒副作用。氧疗的毒副作用主要与高浓度吸入有关,其中最主要的副作用有去氮性肺不张和氧中毒。

\subsubsection{人工气道的建立和管理}

人工气道是为了保证气道通畅而在生理气道与其他气源之间建立的连接,分为上人工气道和下人工气道,是危重症患者常用的抢救措施之一。上人工气道包括口咽气道和鼻咽气道,下人工气道包括气管插管和气管切开等。

建立人工气道的目的是保持患者气道的通畅,有助于呼吸道分泌物的清除及进行机械通气。人工气道的应用指征取决于患者呼吸、循环和神经系统功能状况。结合患者的病情及治疗需要选择适当的人工气道。

人工气道的建立使患者上呼吸道功能丧失,而继发感染、气管导管意外脱出或梗阻等又会加重病情,甚至危及生命。如果气道管理仅作为一项医生和护士分开训练和管理的普通技术,就容易造成医生忽视人工气管建立后的气道管理;护士往往被动配合医生完成人工气道的建立,缺乏主动判断和实施意识;医护之间缺乏沟通而没有及时预见性地评估气道状况,将致错过关键的处理时机。规范的气道管理可减少人工气道并发症。由此可见,气道管理是维系生命最重要的措施之一,体现了重症医学科的救治和管理水平。

2004年加拿大危重病学会和危重病临床试验组联合专家委员会,以及2005年美国胸科学会和感染疾病学会,先后基于循证医学证据制定了院内获得性肺炎及呼吸机相关性肺炎方面的指南
\protect\hyperlink{text00015.htmlux5cux23ch1-14}{\textsuperscript{{[}1{]}}}
\textsuperscript{,}
\protect\hyperlink{text00015.htmlux5cux23ch2-14}{\textsuperscript{{[}2{]}}}
,针对人工气道管理和非抗生素预防策略方面提出了指导性意见。而近年来关于建立人工气道的方式及时机的选择、气管导管声门下潴留物的引流和气道湿化等相关方面又有新的研究,为临床治疗的具体实施提供了更多的依据。

气管插管和气管切开是临床上常用的建立人工气道方法。在循环严重不稳定的情况下,即使动脉氧合尚能维持也应早期开放气道进行有创机械通气,纠正组织缺氧。当然,在紧急开放气道时更应关注是否存在困难插管。

1985年经皮扩张气管切开术首次应用于临床,随着科技的发展,切开技术和器材在不断改进,近年来在国内广泛应用,并出现了许多改良技术。在气管切开方式的选择方面,经皮扩张气管切开术相对于传统的直视下气管切开术,显示出一定优势,可缩短手术操作时间和减少气管切开的并发症,并可能减少重症医学科住院时间和医疗费用
\protect\hyperlink{text00015.htmlux5cux23ch3-14}{\textsuperscript{{[}3{]}}}
。随着医疗技术的发展,经皮扩张气管切开已被应用于急诊创伤、凝血功能障碍的患者
\protect\hyperlink{text00015.htmlux5cux23ch4-14}{\textsuperscript{{[}4{]}}}
\textsuperscript{~}
\protect\hyperlink{text00015.htmlux5cux23ch6-14}{\textsuperscript{{[}6{]}}}
。对于需要长期机械通气或保留人工气道的患者,研究表明早期气管切开(平均机械通气时间为7天),可降低机械通气时间和重症医学科住院时间,但不降低呼吸机相关性肺炎发生率和病死率
\protect\hyperlink{text00015.htmlux5cux23ch7-14}{\textsuperscript{{[}7{]}}}
\textsuperscript{,}
\protect\hyperlink{text00015.htmlux5cux23ch8-14}{\textsuperscript{{[}8{]}}}
。

在人工气道管理方面,应维持合适的气囊压力(气囊压力一般维持在25~35cm
H\textsubscript{2}
O),可以通过气囊压力测定仪测定,或通过寻找最小封闭压力和最小封闭容积调整气囊压力,有效的声门下吸引可减少误吸,防止呼吸机相关性肺炎。研究显示,持续声门下吸引可以延缓呼吸机相关性肺炎的发生,降低呼吸机相关性肺炎的发生率,且能明显降低费用
\protect\hyperlink{text00015.htmlux5cux23ch9-14}{\textsuperscript{{[}9{]}}}
。

对于危重病患者,尤其建立人工气道者,合适的气道局部湿化非常重要,这在输液量严格控制的患者中尤为必要。近年来,新型的湿热交换器、附加电加热丝管路加热的湿化器逐渐运用于临床。相对于加温湿化器,湿热交换器操作方便,有利于感染控制,减少管路细菌的定植,但在降低呼吸机相关性肺炎发生率方面尚无明确的证据
\protect\hyperlink{text00015.htmlux5cux23ch10-14}{\textsuperscript{{[}10{]}}}
\textsuperscript{,}
\protect\hyperlink{text00015.htmlux5cux23ch11-14}{\textsuperscript{{[}11{]}}}
。

\section{临床问题}

\subsection{氧气疗法}

\subsubsection{为什么说氧气是一种药物?}

氧气应用于临床患者治疗已有一个半世纪,但直到1945年,有关氧气的生理作用仍然有争议。氧气是机体组织细胞能量代谢所必需的物质。必须有充足的氧气,细胞才能维持其生理功能。机体对氧气的生理需求和缺氧对机体的危害,使临床医师充分认识到氧气的重要性。但临床医师对氧气的毒性却普遍认识不足。由于高氧环境往往产生高浓度的氧自由基,长时间吸入高浓度氧气,可引起非心源性肺水肿,即急性呼吸窘迫综合征(ARDS)。新生儿吸入高浓度氧,除引起ARDS外,还可能引起视网膜病变及晶状体纤维增殖症,导致失明。无疑,长时间吸入高浓度氧对机体是有害的。

另外,吸入氧浓度的调整也是非常重要的。慢性阻塞性肺疾病急性加重期的患者,如吸入氧浓度过高,可引起呼吸中枢抑制,一般要求吸入氧浓度不宜高于30%或35%(氧流量不超过3L/分钟)。对于换气功能障碍的患者,如急性呼吸窘迫综合征,必须根据缺氧的程度,调整吸入氧浓度。

氧气实际上就是一种“药物”,临床应用时不但应注意其使用剂量,还应注意其毒副作用。

\subsubsection{氧疗的高流量系统与低流量系统有什么不同?}

根据氧疗系统提供的气体是否能够满足患者吸气的需要,一般将氧疗装置分为高流量和低流量系统。值得注意的是,高流量与低流量并不等同于高浓度和低浓度吸氧。

高流量系统提供的气流能够满足吸气的需要,患者不需额外吸入空气。该系统提供较高的气体流速及足够大的贮气囊,气体量能够完全满足患者吸气所需。须特别注意的是,高流量系统实施氧疗并不意味着吸入气氧浓度较高,高流量系统可提供氧浓度较高的气体,亦可提供较氧浓度较低的气体,该系统的主要优点为:①能够提供较准确的、不同氧浓度的气体,而且氧浓度不受患者呼吸模式的影响;②气流完全由系统提供,可根据患者需要调整气体的温度和湿度。

多数高流量系统采用带有Venturi装置的面罩,该装置利用Beroulli原理,氧气通过一较狭窄的喷头高速喷出,高速气流的周围形成负压,导致空气卷入主气流中,使系统气流量明显增加。采用Venturi装置的高流量系统,喷头的大小和空气卷入孔的大小决定吸入气的氧浓度,而氧流速则决定了该系统所能提供的气体量。高流量系统提供的气流速应超过患者峰值流速,而且提供的气体量应当是患者通气量的4倍以上。

低流量系统提供的气流不能完全满足患者吸气的需要,需额外吸入部分空气。该系统可提供的气体氧浓度为21%~90%。吸入氧浓度由以下因素决定:①贮气囊的大小;②氧流量;③患者的呼吸模式(潮气量、呼吸频率及吸气时间等)。

低流量系统提供的气体氧浓度不很准确,但患者更为舒适,应用较为方便,而且比较经济。常用的低流量系统包括鼻塞、鼻导管、普通面罩、带有贮气囊的面罩等。低流量系统实施氧疗时,吸入氧浓度一般低于60%,要进一步提高吸入氧浓度,需应用带有贮气囊的面罩。

\subsubsection{不同的氧疗系统提供的吸入氧浓度有何不同?}

高或低流量氧疗系统氧流量与吸入氧浓度之间的关系不同,见表\ref{tab9-1}。\footnote{* 括号内数值为进入Venturi面罩的空气流量。表中吸入氧浓度仅供参考。}

\begin{table}[htbp]
\centering
\caption{氧流量与吸入氧浓度之间的关系}
\label{tab9-1}
\includegraphics[width=\textwidth,height=\textheight,keepaspectratio]{./images/Image00070.jpg}
\end{table}



\subsubsection{采用低流量或高流量氧疗系统的指征是什么?}

当患者有指征接受氧疗时,应确定采用何种氧疗系统。低流量和高流量系统各有利弊。与高流量系统比较,低流量系统具有以下优点:①患者易于耐受,较为舒适;②实施较方便。但低流量系统的缺点也很明显:①低流量系统的气体不能满足患者吸气的需要,需额外吸入空气,使吸入氧浓度不稳定;②吸入氧浓度受患者呼吸模式的影响较大。高流量系统提供的气体氧浓度较为稳定,基本不受患者呼吸模式的影响。总的来说,对于病情稳定、呼吸平稳,而且对吸入氧浓度的准确性要求不高的患者,宜采用低流量氧疗系统,反之,应采用高流量氧疗系统。

一般认为,采用低流量氧疗系统应具备以下指征:①潮气量300~700ml;②呼吸频率低于25次/分钟;③呼吸规则而稳定。不符合上述任一条件的患者,均应采用高流量系统。一般来说,需接受氧疗的患者中,绝大多数患者(大约75%)只需采取低流量系统,即可达到氧疗的目标。

经过积极的氧疗措施不能奏效时,应早期气管插管,采用机械通气治疗。

\subsubsection{鼻导管吸氧时,氧流量是否是决定吸入氧浓度的唯一因素?}

采用鼻导管或鼻塞氧疗时,一般认为吸入氧浓度与吸入氧流量大致有如下关系:吸入氧浓度=21+4×吸入氧流量(L/分钟)。实际上,吸入氧浓度还受潮气量和呼吸频率的影响------张口呼吸、说话、咳嗽和进食时,即使氧流量不变,吸入氧浓度也会降低。

下面以“正常人”以“正常呼吸模式”进行呼吸为例做一简要说明:

\begin{center}
\begin{tabular}{ll}
  参数&参考值\\
  潮气量&500ml\\
  呼吸频率&20次/分\\
  吸气时间&1秒\\
  呼气时间&2秒\\
  口鼻咽解剖死腔&50ml
\end{tabular}
\end{center}



鼻导管吸氧流量为6L/分钟(100ml/秒)。假定呼气在呼气时间的前1.5秒(75%)完成,则最后的0.5秒几乎无气体呼出,来自鼻导管的纯氧(吸氧流量为6L/分钟,即100ml/秒)将在这0.5秒中将口鼻咽解剖死腔充满。那么,在1秒的吸气时间内,吸气潮气量由3个部分组成:①来自口鼻咽解剖死腔的50ml纯氧;②来自鼻导管的100ml纯氧,即100ml/秒×1秒;③500ml潮气量中,需吸入350ml的空气(氧浓度为20%左右),则氧气为350ml×20%=70ml。

可见,500ml吸气潮气量中含有220ml的纯氧(50ml+100ml+70ml),则吸入氧浓度为44%(220ml/500ml)。也就是说在“理想通气状态下”,通过鼻导管吸入流量为6L/分钟的氧气时,其吸入氧浓度为44%。

在其他条件不变的情况下,若将氧流量从1L/分钟逐渐增加至6L/分钟,则氧流量每变化1L/分钟,吸入氧浓度大约相应变化4%。这就是上述氧流量与吸入氧浓度关系方程的推算依据。

对于同一患者,其他条件不变,仅潮气量减少1/2,即250ml,则吸气潮气量的构成将发生明显变化:①来自口鼻咽解剖死腔的50ml纯氧;②来自鼻导管的100ml纯氧,即100ml/秒×1秒;③250ml潮气量中,需吸入100ml的空气(氧浓度为20%左右),则氧气为100ml×20%=20ml。

可见,250ml吸气潮气量中含有170ml的纯氧(50ml+100ml+20ml),则吸入氧浓度为68%(170ml/250ml)。因此,潮气量越大或呼吸频率越快,吸入氧浓度越低;反之,潮气量越小或呼吸频率越慢,吸入氧浓度越高(↑潮气量→吸入氧浓度↓;↓潮气量→吸入氧浓度↑)。

只要通气模式不发生变化,鼻导管或鼻塞可提供相对稳定的吸入氧浓度。但是认为鼻导管或鼻塞可确保稳定的低浓度氧疗则是错误的。

另外,应用鼻导管或鼻塞时,氧流量不应超过6L/分钟。这与鼻咽部解剖死腔已被氧气完全预充有关,提高氧流量不可能进一步增加吸入氧浓度,此时要提高吸入氧浓度,须加用氧贮气囊。

\subsubsection{如何使用普通面罩实施氧疗?}

普通面罩一般用塑料或硅胶制成,重量较轻,无单向活瓣或贮气袋,呼出气通过面罩上的小孔排出。面罩需紧贴口鼻周围,用绑带固定于头枕部。即使氧气供应暂时中止,空气仍可从面罩上的小孔和面罩周围的缝隙流入。另外,系统可提供较好的湿化。但普通面罩影响患者的进食和说话,睡眠变换体位或烦躁不安时易脱落或移位,患者呕吐时易发生呕吐物误吸。

面罩死腔及其“贮袋效应”影响了氧流量和吸入氧浓度之间的关系。氧流量需在5~6L/分钟以上,才可将面罩内的呼出气(包括二氧化碳)冲洗排出,最大吸入氧浓度为50%~60%。氧流量>8L/分钟时,吸入氧浓度不会进一步增加。如氧流量过低,不仅吸入氧浓度下降,而且呼出气的二氧化碳可在面罩内积聚,导致二氧化碳重复吸入。患者通气模式改变同样会影响吸入氧浓度。潮气量越大或吸气流速越快,氧气被空气稀释越多,吸入氧浓度越低;在一定范围内,氧流量越大,吸入氧浓度越高。呼吸缓慢的患者,采用普通面罩可获得较高的吸入氧浓度,而呼吸频速的患者则吸入氧浓度较低。所以,普通面罩不宜用于呼吸频速和严重低氧血症的慢性阻塞性肺疾病并发急性通气功能障碍或急性限制性疾病的患者(如急性肺水肿)。

\subsubsection{部分重复呼吸面罩与无重复呼吸面罩有什么区别?}

未行气管切开或气管插管的患者需吸入高浓度氧气(吸入氧浓度>60%)时,需在普通面罩上加装一体积600~1000ml的储气袋。氧流量须在5L/分钟以上,以确保储气袋适当充盈和将面罩内二氧化碳冲洗出。面罩和储气袋间无单向活瓣为部分重复呼吸面罩,有单向活瓣则为无重复呼吸面罩。应用附有储袋面罩的目的是提供较高吸入氧浓度。根据呼出气体的重复吸入程度可将氧疗系统分为以下两种:

(1)部分重复呼吸面罩(图\ref{fig9-2}) 该装置允许患者重复吸入部分呼出气体,以减少氧气消耗。氧气从面罩的颈部流入,在吸气相直接进入面罩,而在呼气相则进入储气袋。理想情况下,患者呼气时,呼出气的前1/3进入储气袋,与储气袋中的纯氧混合。呼出气的前1/3主要来自解剖死腔。此部分气体在使用部分重复呼吸面罩后不久,氧浓度较高。当储气袋被纯氧和呼出气的前1/3充满后,其内部压力迫使呼出气的后2/3(包括二氧化碳负荷)从呼气孔排出。在密封较好的部分重复呼吸面罩,氧流量为6~10L/分钟时,吸入氧浓度可达35%~60%。

\begin{figure}[!htbp]
 \centering
 \includegraphics{./images/Image00072.jpg}
 \captionsetup{justification=centering}
 \caption{部分重复呼吸面罩}
 \label{fig9-2}
  \end{figure} 

(2)无重复呼吸面罩(图\ref{fig9-3}) 在储气袋与面罩间加装一单向活瓣,确保呼气相氧气直接进入储气袋,吸气相氧气流向面罩和储气袋;活瓣可阻止呼出气回流到储气袋,直接通过面罩上的小孔排出,使患者不再重复吸入呼出气。

\begin{figure}[!htbp]
 \centering
 \includegraphics{./images/Image00073.jpg}
 \captionsetup{justification=centering}
 \caption{无重复呼吸面罩}
 \label{fig9-3}
  \end{figure} 

\subsubsection{应用Venturi面罩实施氧疗有何特点?}

可调式通气面罩即Venturi面罩(图\ref{fig9-4}),属于高流量氧疗系统,其吸入氧浓度可较好地控制。Venturi面罩可提供的吸入氧浓度为24%、26%、28%、30%、35%、40%。虽然Venturi面罩可提供40%以上的吸入氧浓度,但其精确度明显下降,与实测值可相差10%。低浓度时仅相差1%~2%。

\begin{figure}[!htbp]
 \centering
 \includegraphics{./images/Image00074.jpg}
 \captionsetup{justification=centering}
 \caption{Venturi面罩}
 \label{fig9-4}
  \end{figure} 

可调式Venturi面罩具有如下优点:①可提供较恒定的吸入氧浓度;②由于喷射入面罩的气体流速超过患者吸气时的最高流速和潮气量,所以,患者呼吸模式变化不会影响吸入氧浓度;③可湿化氧气;④高流速气体可促使面罩中呼出气的二氧化碳排出,基本无二氧化碳重复吸入。Venturi面罩可适用于低氧血症伴高碳酸血症的患者。

\subsubsection{如何评价氧疗的效果?}

由于氧疗的目的是纠正组织缺氧,减少心肌和呼吸肌做功,因此,对氧疗效果的评价应包括对心肺系统的评估。

心血管系统评估主要应观察血压、脉搏和灌注状态。对于接受氧疗的患者,将其血压、脉搏与基础状态比较。如缺乏基础状态的资料,则应动态观察和评价。另外,心律失常可能是缺氧的后果,氧疗时也应注意。通过观察患者的皮肤颜色、湿度、温度和毛细血管再充盈时间,对灌注状态进行评估。每小时尿量及意识状态亦是反映危重患者组织灌注状态的重要指标。

呼吸系统的评估主要包括对潮气量、呼吸频率和呼吸功的观察和监测。临床观察判断潮气量往往不准确,如有可能应监测潮气量。观察呼吸频率,并注意呼吸节律是否规则。呼吸功为呼吸肌所做的功,降低呼吸功是氧疗的主要目的之一。当呼吸功增加时,患者往往有呼吸困难,并可表现为动用辅助呼吸肌。由于呼吸困难是呼吸功增加的重要主观指标,对于主诉有呼吸困难的患者,临床医师应特别重视。

动脉血气监测是评价氧疗效果的实验室指标。氧疗期间,应根据病情变化,反复监测动脉血气。根据动脉血氧分压水平,判断氧疗效果,并据此调整氧疗措施。另外,还应根据动脉血二氧化碳分压和pH值水平,判断患者的通气状态和酸碱平衡状态。

总之,鉴于氧疗的目的不仅仅包括纠正低氧血症,还包括降低呼吸功和心肌做功,故评价氧疗效果时应同时注意氧合和心肺功能状况。

\subsubsection{氧疗的吸入氧浓度为什么不宜超过50%?}

氧疗时,一般要求吸入氧浓度不宜超过50%,这主要与以下两个因素有关:

(1)吸入氧浓度高于50%可引起去氮性肺不张,导致解剖学分流增加。氧疗时,吸入氧浓度从21%逐步增加到50%,肺内总分流率(生理学分流和解剖学分流)明显降低,这与生理学分流被纠正有关,但进一步提高吸入氧浓度,总分流率反而明显增加。生理学分流随吸入氧浓度升高应进一步降低,总分流率增高必然与解剖学分流增加有关。去氮性肺不张是导致解剖学分流增加的主要原因。

正常情况下,氮气是维持肺泡膨胀的重要气体。存在生理学分流的肺泡,通气量不足,容积较小。当吸入氧浓度提高,特别是吸纯氧时,将发生以下两种效应:①通气不足的肺泡存在低氧性肺血管痉挛,当肺泡氧分压升高,其周围痉挛的毛细血管明显扩张,血流增加;②肺泡内氮气被洗出,氮气压力明显降低,肺泡内主要含有氧气。结果由于氧气迅速被吸收,这类肺泡便发生萎陷、形成肺不张、导致解剖学分流增加。吸入纯氧后15分钟就可发生去氮性肺不张,值得重视。总的来看,吸纯氧时的肺内分流率明显升高,而吸入氧浓度40%~60%时,肺内分流率最低(图\ref{fig9-5})。因此,一般情况下,实施氧疗时吸入氧浓度不宜超过60%。

\begin{figure}[!htbp]
 \centering
 \includegraphics{./images/Image00075.jpg}
 \captionsetup{justification=centering}
 \caption{肺内总分流率与吸入氧浓度之间的关系}
 \label{fig9-5}
  \end{figure} 

(2)吸入氧浓度高于50%易导致氧中毒性肺损伤 氧中毒主要与吸入气中的氧分压有关。氧中毒的机理尚不甚明了,目前认为氧对细胞的毒性与其自由基毒性中间产物的作用有关,这些中间产物包括超氧自由基、过氧化氢、羟基自由基及单线态氧等活性氧。在1个大气压条件下,吸入空气可产生占氧耗量1%~5%的活性氧,其生成量随吸入氧浓度的升高而增加。在高浓度氧或高压氧疗时,产生活性氧的量超过了机体的处理能力,从而对机体细胞造成损害。氧中毒对肺的损害可表现为气管支气管炎、急性呼吸窘迫综合征、支气管-肺发育不良(见于新生儿)等。因此,从氧中毒的角度,对吸入氧浓度也应做出相应的限制。目前对氧浓度的安全界限尚无一致意见,但一般认为,在1个大气压条件下,吸入氧浓度低于60%的氧疗是无害的,长时间吸入氧浓度高于60%可能产生氧中毒,如吸纯氧,不应超过24小时。

\subsubsection{引起低氧血症的主要原因有什么?}

低氧血症是指在吸空气的条件下,动脉血氧分压低于80mmHg(1mmHg=0.133kPa)。可分为轻、中和重度低氧血症。动脉血氧分压60~80mmHg为轻度低氧血症,40~60mmHg为中度,低于40mmHg为重度。一般认为,动脉血氧分压36mmHg是人体生存的生理极限。低氧血症可引起广泛的组织细胞损伤,后果严重,是较常见的临床问题。引起低氧血症的原因主要包括以下两个方面。

(1)肺部疾病是导致低氧血症最常见的原因 可引起低氧血症的肺部疾病很多,主要包括以下几类------①肺泡通气量明显降低:严重的肺泡(低通气)可引起低氧血症,主要见于慢性阻塞性肺疾病、支气管扩张症等;②气体弥散功能障碍:主要见于肺纤维化、尘肺等;③通气/血流比例失调:为导致低氧血症最主要的原因,通气/血流比例<0.8者,依程度不同包括解剖学分流和生理学分流,见于肺不张、ARDS、肺实变等;通气/血流比例>0.8者为死腔样效应,主要见于肺栓塞等疾病。

(2)吸入气氧分压降低也是导致低氧血症的原因之一 这类情况主要见于高原居住或工作、高空飞行、潜水工作等。

\subsubsection{什么是顽固性低氧血症?其发生的主要原因是什么?}

顽固性低氧血症是指氧疗难以纠正的低氧血症,其诊断标准需符合以下指标之一:①吸入氧浓度高于35%的条件下,动脉血氧分压低于55mmHg;②吸入氧浓度提高20%(氧负荷试验),动脉血氧分压的升高不超过10mmHg。

引起顽固性低氧血症的主要原因包括:①解剖学分流明显增加。分流是导致低氧血症的主要原因之一,但生理学分流引起的低氧血症多能通过提高吸入氧浓度得到纠正,而解剖学分流引起的低氧血症,由于其通气/血流比例为0,氧疗难以奏效。解剖学分流增加主要见于心脏的右向左分流(例如先天性心脏病等)、肺动静脉瘘、肺不张、肺实变、ARDS等。正常情况下,解剖学分流不超过5%。病理条件下,解剖学分流高于30%时,将导致顽固性低氧血症。②严重的弥散障碍。严重的肺纤维化将导致肺泡-毛细血管膜增厚,气体弥散障碍,亦可导致严重的低氧血症。

\subsubsection{组织缺氧主要与哪些因素有关?氧疗是否一定能够纠正组织缺氧?}

组织缺氧不仅与呼吸性因素有关,还与血液、循环因素及细胞利用氧的能力有关。

氧输送减少是导致组织缺氧的主要原因。氧输送由动脉血氧含量和心脏指数决定,氧输送指数=动脉血氧含量×心脏指数。而动脉血氧含量由动脉血氧分压或动脉血氧饱和度和血红蛋白含量决定,动脉血氧含量=1.34×血红蛋白浓度×动脉血氧饱和度+0.0031×动脉血氧分压。从上述公式可以看出,以下三个因素均可导致氧输送降低,引起组织缺氧。

(1)呼吸性疾病引起低氧血症,即引起动脉血氧分压及血氧饱和度降低,使动脉血氧含量降低,导致氧输送降低。

(2)血红蛋白的质、量异常也是引起组织缺氧的重要原因。各种疾病引起的贫血,导致血红蛋白含量降低,使动脉血氧含量明显下降;另一方面,先天性血红蛋白异常、一氧化碳中毒、亚硝酸盐中毒时,血红蛋白含量可以正常,但因血红蛋白失去了结合氧的能力,也可导致动脉血氧含量降低。

(3)心脏是将富含氧气的动脉血送往组织的动力泵,心脏指数降低也是导致组织缺氧的重要原因。在动脉血氧分压和血红蛋白功能正常的情况下,心脏指数降低可引起氧输送降低,引起组织缺氧。这类情况主要见于左心衰及心包填塞引起的心源性休克、低血容量性休克等疾病。

需要指出,氧输送减少还包括氧输送的相对减少,即由于氧需或氧耗增加,导致氧输送尽管处于“正常水平”,但氧输送与氧耗失衡,仍然可引起组织缺氧。这类情况主要见于严重感染、高热、剧烈运动、甲状腺机能亢进等。

组织氧利用障碍也是造成组织缺氧的原因之一。组织细胞损害、酶系统功能障碍时,尽管氧输送正常,仍可引起组织缺氧。主要见于氰化物中毒、硫化氢中毒、休克引起组织细胞功能障碍等。

另外,血流分布及弥散等因素也与组织缺氧有关。感染性休克时,尽管心脏指数明显增加,但不同器官、组织的血流分布是不同的,某些组织可能血流减少而导致组织缺氧。氧从毛细血管弥散到细胞,取决于氧分压和弥散距离。组织水肿使弥散距离增大,可能加重组织缺氧。

综上所述,组织缺氧与动脉血氧分压、血红蛋白质量、心脏指数、组织氧利用能力等因素有关,其中只有动脉血氧分压降低或低氧血症引起的组织缺氧,可通过氧疗得以纠正,其他原因引起的组织缺氧氧疗难以奏效。因此,对于组织缺氧,应针对其原因,有的放矢地进行积极处理。

\subsection{人工气道的建立与管理}

\subsubsection{何谓人工气道,建立人工气道的指征是什么?}

人工气道是将导管直接插入气管或经上呼吸道插入气管所建立的气体通道。虽然人工气道的建立使患者失去了上呼吸道的加温、加湿、滤过功能,并削弱了自主清除呼吸道内异物的能力、不便于发音、降低患者的生活质量、增加院内感染的几率,但作为一种抢救的手段,人工气道的建立有利于痰液的引流、增进通气的有效性,导管气囊的存在可以使口咽部的分泌物、呕吐物不易进入肺部,并且能减少漏气,保证正压通气的有效实施。人工气道的应用指征应综合考虑循环、呼吸及中枢神经系统等方面的因素。一般而言,建立人工气道的指征如下。

(1)上呼吸道梗阻 口鼻腔及喉部软组织损伤、异物或分泌物潴留均可以引起上呼吸道梗阻,威胁患者生命。及时建立人工气道能够保证上呼吸道通畅。

(2)气道保护性机制受损 正常情况下,咽、喉、声带、气道及隆突通过生理反射(主要为迷走神经发射)对呼吸道发挥保护作用,依次存在咽反射(恶心和吞咽反射)、喉反射(声门关闭及会厌覆盖声门)、气管反射(异物或分泌物刺激气道引起咳嗽)及隆突反射(隆突受刺激而引发的强烈咳嗽)。患者意识改变(特别是昏迷)以及麻醉时,正常的生理反射受到抑制,导致气道保护性机制受损,易发生误吸及分泌物潴留,可能导致严重的肺部感染。因此,对于气道保护性机制受损的患者,有必要建立人工气道,以防止误吸和分泌物潴留。

(3)气道分泌物潴留 正常情况下,气道分泌物通过黏膜纤毛运动到达大气道,大气道受刺激后发生咳嗽反射,将分泌物咯出。正常的咳嗽反射受损时,会使分泌物在大气道潴留,易导致肺部感染和呼吸道梗阻。虽然可以经鼻腔或口腔将吸痰管插入咽部及气道,但往往效果很差,而且刺激性较大,患者不易配合,严重时还可以引起鼻咽部出血及诱发严重的心律失常。因此,及时建立人工气道,对清除气道分泌物是必要的。

(4)实施机械通气 需要接受机械通气的患者,首先应建立人工气道,提供与呼吸机连接的通道。当然,短时间实施正压通气,有时也可采取面罩与呼吸机相连,实施无创通气。但在需要长期机械通气或存在无创通气禁忌证时则必须建立人工气道,如呼吸心跳骤停、合并其他重要器官功能衰竭(严重脑病、严重上消化道出血、血流动力学不稳定或严重心律失常)、面部手术或创伤畸形、气道保护性机制丧失排痰障碍、严重低氧血症或酸中毒、近期上腹部手术等。

对指征的把握须进一步说明的是:①紧急建立人工气道无绝对禁忌证,关键在于选择最合适的方法,除非患者或法定监护人明确表示拒绝;②存在自主呼吸不是开放气道的禁忌证;③循环不稳定、严重酸中毒的患者,即使氧合尚可,从休克的复苏、纠正组织缺氧来说,也有指征早期开放气道正压通气;④建立人工气道和机械通气的指征不同,建立人工气道的患者不一定需要进行机械通气,但是进行有创机械通气必须先建立人工气道。

\subsubsection{紧急人工气道建立的适应证是什么?}

下列情况下需要紧急建立人工气道:①短时间内气道完整性受到破坏或气道梗阻;②呼吸衰竭需要呼吸机辅助呼吸;③紧急保护气道以防止可预见的影响气道通畅性的因素。

临床上需要建立紧急人工气道的常见危重病症包括深昏迷、呼吸衰竭或呼吸停止、心跳骤停、严重气道痉挛、气道异物梗阻、镇静剂或麻醉剂作用、颅脑及颈部外伤、误吸或有误吸危险(如上消化道大出血)、意外拔管、难以控制的上呼吸道出血、急性上呼吸道梗阻等。

\subsubsection{常见的人工气道有哪些类型?}

人工气道包括上人工气道和下人工气道。上人工气道包括口咽通气道和鼻咽通气道,有助于保持上呼吸道的通畅。口咽通气道适用于舌后坠而导致上呼吸道梗阻、癫痫大发作或阵发性抽搐,以及经口气道插管时,可在气管插管旁插入口咽气道,防止患者咬闭气管插管而发生部分梗阻或窒息。鼻咽通气道仅适用于因舌后坠导致的上呼吸道阻塞,但应注意凝血功能障碍者可能发生鼻咽出血。

最常用的人工气道是指下人工气道,主要包括气管插管和气管切开管。导管的材料、结构及应用的适应证均有所不同。

(1)气管插管导管

结构:气管导管为一略弯的管子,长度为28~32cm,内径为7.0mm、7.5mm、8.0mm等,内径越小,阻力越大,而且分泌物易阻塞管道。内径越大,阻力越小,但插管时较难通过鼻腔和声门,创伤性较大。导管远端开口呈45°斜面,带有单向活瓣的气囊,气囊充气后,阻塞导管与气管壁之间的间隙,可接呼吸机实施机械通气。

材料:气管导管有橡胶管、塑料管及硅胶管等几种。橡胶管质地硬,可塑性差,插管时易损伤鼻、声带及气管黏膜,更重要的是其组织相容性差,易导致黏膜充血、水肿、糜烂,甚至溃疡。聚氯乙烯塑料导管组织相容性好,受热后可软化,对上呼吸道的创伤性较小。硅胶导管的组织相容性更好,质地较软,但价格较贵。以往橡胶导管较常使用,目前很少使用,基本被塑料或硅胶导管替代。

气管导管气囊:气管导管气囊可分为高压低容和低压高容两种。气囊是否对气管黏膜有损伤作用,主要取决于气囊内压力及气管黏膜灌注压。高压低容气囊易导致黏膜缺血、糜烂、坏死、溃疡,已较少使用。低压高容气囊充气后,气囊内压较低,与气管黏膜接触面积大,对黏膜损伤较小,低压高容气囊是目前常用的气管导管气囊。

插管途径:有经口和经鼻气管插管两种。经口气管插管导管较粗,便于吸痰,急救时常常采用,但对于清醒患者常难以耐受,导管刺激口腔黏膜,分泌物较多,口腔护理困难,导管易移位而脱出,保留时间一般较短。经鼻气管插管比经口插管易于耐受、便于固定和口腔护理,导管保留时间较长,但经鼻插管对鼻腔创伤较大,易出血,采用的导管内径多偏小,而且导管弯度较大,使吸痰管插入困难,导管也易堵塞。

(2)气管切开管

结构:传统的气管切开管由内外套管组成,外套管带有单向活瓣的指示气囊。气管切开管通过固定带固定于颈部,内套管可与呼吸机相连接,而且便于拆卸,清洗管内分泌物和消毒,以保持呼吸道通畅。

材料:国产气管切开管多由银制的内外套管组成,使用逐渐减少。进口的塑料或硅胶套管更为常用,此类气管切开管无内套管。

气囊:气囊亦为低压高容气囊,对气管黏膜的损伤性较小。

\subsubsection{人工气道对患者有什么不良影响?}

人工气道是重要的抢救和治疗措施,但对患者也有不良影响。影响的程度与人工气道类型、使用时间、护理质量等有关。

(1)呼吸道的正常防御机制被破坏 正常情况下,机体通过上呼吸道的防御机制(湿化、滤菌、咳嗽、纤毛运动及杀菌等)防止细菌进入下呼吸道,使下呼吸道保持无菌状态。人工气道的建立,跨过了上呼吸道,使下呼吸道直接与外界相通,结果使气管支气管树易受细菌感染,导致肺部感染。

(2)抑制正常咳嗽反射 气管插管经过声门,使声带不能有效关闭,而气管切开管的气体通道又不经过声门,结果使机体咳嗽反射受到影响,患者不能有效咳嗽,其后果是分泌物在大气道潴留,误吸的分泌物也不能有效排除,极易发生肺部感染和呼吸道梗阻。

(3)影响患者的语言交流 气道插管或气管切开管的患者均不能发声,影响语言交流,常使患者感到孤独和恐惧,在重症医学科的特殊环境下尤为如此。可采用写字板等方式让患者进行有效交流。

(4)患者的自尊受到影响 对于神志清醒的患者,人工气道的建立常常使患者的自尊心受到伤害。经过人工气道呼吸,大量分泌物从人工气道直接排出、不能说话等,均使患者感到难堪。此时帮助患者建立自信是很必要的。

\subsubsection{经口气管插管的适应证和禁忌证有哪些?}

经口气管插管操作较容易,插管的管径相对较大,便于气道内分泌物的清除,但影响会厌功能,患者耐受性也较差。

经口气管插管适应证包括:①严重低氧血症或高碳酸血症,或其他原因需较长时间机械通气,又不考虑气管切开;②不能自主清除上呼吸道分泌物、胃内反流物或出血,有误吸危险;③下呼吸道分泌物过多或出血,且自主清除能力较差;④存在上呼吸道损伤、狭窄、阻塞、气管食管瘘等,严重影响正常呼吸;⑤患者突然出现呼吸停止,需紧急建立人工气道进行机械通气。经口气管插管的关键在于暴露声门,在声门无法暴露的情况下,容易失败或出现并发症。

禁忌证或相对禁忌证包括:①张口困难或口腔空间小,无法经口插管;②颈部无法后仰(如疑有颈椎骨折)。

\subsubsection{经鼻气管插管的适应证和禁忌证有哪些?}

经鼻气管插管较易固定,舒适性优于经口气管插管,患者较易耐受,但管径较小,导致呼吸功增加,不利于气道及鼻窦分泌物的引流。

经鼻气管插管适应证除紧急抢救外,均同经口气管插管。

经鼻气管插管禁忌证或相对禁忌证包括:①紧急抢救,特别是院前急救;②严重鼻或颌面骨折;③凝血功能障碍;④鼻或鼻咽部梗阻,如鼻中隔偏曲、息肉、囊肿、脓肿、水肿、异物、血肿等;⑤颅底骨折。

与经鼻气管插管比较,经口气管插管减少了医院获得性鼻窦炎的发生,而医院获得性鼻窦炎与呼吸机相关性肺炎的发病有着密切关系。因此,对短期内能脱离呼吸机的患者,应优先选择经口气管插管。但是,在经鼻气管插管技术操作熟练,或者患者不适于经口气管插管时,仍可以考虑先行经鼻气管插管。

\subsubsection{何谓逆行气管插管术?如何实施?}

逆行气管插管术是指先行环甲膜穿刺,送入导丝,将导丝经喉至口咽部,由口腔或鼻腔引出,再将气管导管沿导丝插入气管。

逆行气管插管术的适应证为:因上呼吸道解剖因素或病理条件下无法看到声带甚至会厌,无法完成经口或鼻气管插管。

禁忌证包括:①甲状腺肿大,如甲亢或甲状腺癌等;②无法张口;③穿刺点肿瘤或感染;④严重凝血功能障碍;⑤不合作者。

\subsubsection{经口气管插管的操作要点有哪些?}

经口插入气管插管是建立人工气道最常用的手段,也是心肺复苏时紧急建立有效气道的重要方法,因此,快速、准确的插入气管插管对于抢救患者显得十分必要。经口插入气管插管在操作上应注意以下要点。

(1)准备适当的喉镜 直接喉镜根据镜片的形状分为直喉镜和弯喉镜,使用方法上两者有所不同:直喉镜是插入会厌下向上挑,即可暴露声门;弯喉镜是插入会厌和舌根之间,向前上方挑,会厌间接被牵拉起来,从而暴露声门。耳鼻喉科医师为进行活检,需暴露充分,多采用直喉镜;而麻醉医师主要目的是插入气管插管,因此多采用弯喉镜。作为重症医学科医师,需适应各种急救环境,两种喉镜的使用方法均应掌握。

(2)准备不同型号的气管导管 准备不用型号的气管导管以备用,检查导管气囊是否漏气。可将气囊浸入生理盐水中,注入气体后检查是否漏气,然后将气体完全抽出。气管导管远端1/3的表面涂上石蜡油,将有助于插入声门,减少创伤。如使用导丝,则把导丝插入导管中,利用导丝将导管塑形。

(3)头颈部取适当位置是插管成功的主要保证 患者取仰卧位,肩背部垫高约10cm,头后仰,颈部处于过伸位,使口腔、声门和气管处于一条直线上,以利于插入气管插管。即使在紧急情况下,利用片刻时间,调整患者的体位也是十分必要的。

(4)预充氧、人工通气及生命体征监测 在准备插管的同时,应利用面罩和手动呼吸机或麻醉机,给患者吸入纯氧,同时给予人工通气,避免缺氧和二氧化碳潴留。当经皮血氧饱和度在90%以上(最好在95%以上)时,才能开始插管。如插管不顺利,或经皮血氧饱和度低于90%,特别是低于85%时,应立即停止操作,重新通过面罩给氧,并进行人工通气,直到血氧饱和度恢复后,再重新开始。插管前、插管过程中及插管后均应该密切监测患者的心电图和经皮血氧饱和度。

(5)插入喉镜,观察和清洁上呼吸道 操作者站在患者头端,用左手握喉镜,从患者口腔右侧插入,将舌头推向左侧。喉镜应处于口腔正中,观察口咽部。如有分泌物,则需充分抽吸,以免影响插管的视野。

(6)观察声门的解剖标志物 会厌和杓状软骨是声门的解剖标志物,会厌位于声门上方(前方),杓状软骨位于声门的下方(后方),两者之间即为声门。将喉镜插入会厌与舌根之间或插入会厌下方,向前上方挑,就可将会厌挑起,一般首先看到杓状软骨,再用力上挑,则可看到声带。气管插管时并非一定要看到声带,只要看到杓状软骨,甚至看到杓状软骨下方(后方)的食管,即可判断声门的位置,进行插管。

(7)插入气管导管,调节导管深度 观察到声门或声门的解剖标志物后,右手持气管导管,将导管插入声门。调整导管深度,避免插入过深,进入主支气管,注意双侧呼吸音是否对称。一般情况下,男性患者插入深度为距离门齿24~26cm,而女性为20~22cm。立即给气囊充气,将气管导管接呼吸机或麻醉机,实施机械通气,并吸入纯氧。使用导丝者,在气管导管插入声门后,一边送导管,一边将导丝拔除。

(8)确认导管插入气管 主要通过以下几种手段:①用听诊器听胸部和腹部的呼吸音,胸部呼吸音较腹部强;②监测患者呼出气二氧化碳浓度,如插入气管,则可见呼气时,呈现二氧化碳的方波;③对于有自主呼吸的患者,可通过麻醉机气囊的收缩,确认导管插入气管。

(9)固定气管导管 将牙垫插入口腔,此时才可将喉镜取出,用蝶形胶布将气管导管和牙垫一起固定于面颊部及下颌部。

(10)拍摄X线胸片,进一步调整导管位置 气管导管远端与隆突的距离应当为2~4cm。根据X线胸片,调整导管深度。同时观察患者肺部情况及是否并发气胸。

\subsubsection{如何判断气管导管是否插入气管?}

为保证气管插管插入气道内未误入食管,一个重要的经验是目睹气管插管管尖确实从声带之间进入。有时操作者看到了声门入口,但插管接近至声门处时,便移开了视线,结果气管插管误入食管。插管误入食管如不及时发现可能导致非常严重的后果(急性胃扩张,甚至胃穿孔或破裂。同时低氧血症也难以纠正)。判断气管导管是否插入气管的方法如下:

(1)观察通气时胸廓起伏及胃部情况,如果通气后胸廓起伏不明显,腹部明显膨隆,伴气管内胃内容物反流,则肯定不在气道。

(2)通气时听诊胸部和腹部呼吸音,如果胸部的呼吸音强,上腹部不明显,则考虑气管导管位于气管内。

(3)挤压胸廓,对于存在自主呼吸的患者则通过气管导管口听呼吸音,气流明显则提示多数在气管。

(4)接呼吸机看呼出气的流速波形,如果流速波形良好,则提示在气管内。

(5)监测患者呼气末二氧化碳,如插入气管,可见呼气时呈现二氧化碳方波;反之,如在食管内,则呼气末二氧化碳分压可降至或接近零。

(6)紧急床旁纤维支气管镜检查,如可见隆突及支气管开口,则确定在气管内。

在判断的过程中一定要注意血氧饱和度和心电的监测。对于不确定的或通气后生命体征更加不平稳的,应及时拔出导管,简易呼吸囊面罩加压充分氧合后再插管。

在上述方法中,以呼气末二氧化碳监测最准确,而呼气波形监测相对比较简单,且准确率高。

\subsubsection{气管插管插入气管的深度多少是合适的?}

(1)解剖长度 了解总气管长度以及从门齿至声门或至隆突的长度,有助于掌握合适的气管插管插入深度。门齿至声门或隆突的距离和年龄有关,见表\ref{tab9-2}。\footnote{*从成人的解剖长度,下列常数可供参考------①门齿至隆突的距离:男28.5cm,女25.2cm。②门齿至声门的距离:男12~16cm,女10~14cm。③从喉上缘至环状软骨下缘的距离:4~6cm。}

\begin{table}[htbp]
\centering
\caption{气管插管长度与门齿至声门或隆突距离\textsuperscript{*}}
\label{tab9-2}
\includegraphics{./images/Image00076.jpg}
\end{table}

(2)插入合适深度 气管插管插入气管的深度,一般以气管插管尖到达气管中部,即位于声门下4~5cm(成人)较为合适。即使患者有仰头或低头,气管插管不致脱出声门;同时,气囊位于声门下,不会导致患者强烈的不适。

(3)插入深度的预估 为便于临床操作,气管插管插入合适深度有多种预估方法,这里介绍一种:①找出环状软骨(甲状软骨下方的软骨环);②再找出隆突的体表解剖位置,一般相当于平齐Louis角或第二肋软骨处;③确定环状软骨至隆突的中点;④将气管插管的管尖位置置于门齿水平,将插管弧度顺着患者颈部的侧面,直至环状软骨和隆突的中点,该长度就是气管插管插入气管的适当深度,若经鼻插管,则鼻孔入口处为测量的起点。

\subsubsection{在准备气管插管时,如何判断患者可能出现插管困难?}

气管插管为创伤性操作,如果插管前能够判断患者可能出现插管困难,则可以提前准备,以免不必要的反复插管,并避免发生严重并发症和医疗纠纷。

判断插管困难的主要手段和方法如下。

(1)观察咽部结构的可见程度 可用马兰帕蒂(Mallampatis)分级------患者用力张口和伸舌,窥视咽部结构:Ⅰ类,可见软腭、咽腭弓、悬雍垂;Ⅱ类,可见软腭、咽腭弓,悬雍垂被舌根遮盖;Ⅲ类,仅见软腭;Ⅳ类,可见硬腭,未见软腭。此试验仅能预测50%的插管困难,发生插管困难与软腭被舌根挡住有关。Ⅳ类和部分Ⅲ类者插管明显困难,除非头后仰受限,Ⅰ类和Ⅱ类者插管一般无严重困难。

(2)评价下颌骨颞骨关节活动度或张口度 成人最大张口时上下门齿间的距离正常为3.5~5.6cm,平均4.5cm。Ⅰ度张口困难者为2.5~3.0cm,技术娴熟的操作者仍可完成经口气管插管过程;Ⅱ度张口困难者为1.2~2.0cm,难以窥见咽喉部结构;Ⅲ度张口困难者<1.0cm,喉镜片无法置入口内。<1.5cm者无法用常规喉镜进行插管。张口受限可能原因为下颌关节病变或损伤、疤痕挛缩等。简易的评价方法是让患者张口,沿上下门齿方向插入手指,正常能够插入三横指。如不能插入三横指,则提示插管会遇到困难。

(3)评价寰椎枕骨关节的活动度 患者将口张开,上牙列水平与枕骨平面平行,然后将头部后仰,使下牙列水平与枕骨平行,头后仰的角度可反映寰椎枕骨关节的活动角度。寰枕关节正常时,可伸展35°以上,如活动角度降低1/3,则插管困难。据伸展度降低的程度分为4级:Ⅰ级伸展度无降低,活动角度>35°;Ⅱ级降低1/3,20°~25°;Ⅲ降低2/3,10°~12°;Ⅳ级伸展度明显降低,活动度<10°。寰枕关节伸展度降低可导致困难插管。

(4)颏甲间距 成人颈部完全伸展时,甲状软骨切迹至颏凸的距离,>6.5cm时不会发生插管困难;6.0~6.5cm时插管会有困难,但仍可能成功;<6.0cm时不能经喉镜插管。

(5)下颌骨水平支长度 指从下颌角至頦凸的长度,于9.0cm时发生插管困难的几率很小,<9.0cm时插管困难发生率很高,但此为国外资料,国人仅供参考。

(6)颈部后仰度 仰卧位下做最大限度仰颈,上门齿前端至枕骨粗隆连线与身体纵轴线相交的角度,正常>90°,<80°时颈部活动受限,插管可能困难。

(7)喉结过高 此时无法将口腔轴与喉腔轴调整为一个轴线水平,遇此情况插管可能困难。

(8)Cormack分级 用喉镜观察喉头结构:Ⅰ级,声门完全显露;Ⅱ级,声门部分显露,可见声门后联合;Ⅲ级,仅显露会厌或会厌顶端,不能窥见声门;Ⅳ级,声门及会厌均不能显露。这种分级与麻醉科医师的技术和经验有明显关系。Ⅰ级和Ⅱ级者一般不会发生插管困难;Ⅲ级和Ⅳ级者容易发生插管困难,导管误入食管的危险性达50%。

上述方法对判断气管插管是否困难有帮助,对其掌握可使操作者能够提早准备,但应用时仍应综合考虑。

\subsubsection{常规气管插管遇到困难时,有哪些对策?}

困难气管插管是指经过正规训练的医师使用常规喉镜正确地进行气管插管时,经3次尝试仍不能完成,这种情况发生率一般在1%~4%。

一旦判断可能发生困难插管,应采用以下插管方式:①普通喉镜清醒插管(表面麻醉、镇静、保持意识清醒、无呼吸抑制)。②纤维支气管镜引导插管,即气管导管套在纤维支气管镜外,直视下经声门进入气管,气管导管沿纤维支气管镜推入气管。③逆行引导清醒插管。④经口、鼻盲探插管。⑤面罩无创通气辅助下气管切开插管。⑥在插管困难时,可先置入喉罩,经喉罩通气管置入气管导管。当通气罩远端骑跨在声门裂上时,置入的气管导管应滑入气管。⑦指探引导法,即操作者站立在患者头部右侧,左手示指沿患者右口角后臼齿间伸入口腔抵达舌根,探触会厌上缘,并将会厌拨向舌侧。右手持气管导管插入口腔,在左手示指引导下,将管尖对准声门,患者吸气时将导管插入声门。

困难插管时须注意:①切忌惊慌,否则反而会延误处理问题的时机,只要保持患者有效通气和供氧,便不会有生命危险;②若没有其他插管方法,应通过封闭面罩简易呼吸囊加压给氧,辅助患者呼吸,患者自主呼吸恢复后,再考虑清醒插管;③插管操作应轻柔、准确、切忌使用暴力,同时避免长时间反复气管插管。

\subsubsection{心肺复苏时应采用什么方式建立人工气道?}

开放气道是心肺复苏的首要步骤。心肺复苏需要争分夺秒,要求人工气道的建立手段必须简洁、迅速,而且有效。目前常用的手段包括经口气管插管、面罩及食管阻塞器、环甲膜切开术。

经口气管插管是心肺复苏条件下建立人工气道的首选方法。作为紧急插管手段,经口气管插管简洁、方便,可迅速建立人工气道。

对于插管困难的患者,可采用面罩及食管阻塞器。食管阻塞器为一带气囊的、远端为盲端的管子。通过盲插,将气道阻塞器插入食管,气囊充气即可封闭食管,将面罩紧紧覆盖于患者口鼻部,则可实施正压通气。面罩及食管阻塞器的特点包括:①食管阻塞插管易插入食管;②用气囊封闭食管后,可防止胃内容物反流,也可防止正压气体进入胃肠道;③封闭性良好的面罩可保证正压通气的实施。

与其他人工气道相比,面罩及食管阻塞器存在一些问题:①需采用面罩,封闭性可能不佳,因此,实施正压通气不如传统的气管插管,但要优于简易呼吸囊面罩加压给氧。②拔除食管阻塞器时,往往有大量的胃内容物反流,易引起误吸。拔除食管阻塞器前应充分胃肠减压。③食管阻塞器可能引起食管穿孔、咽部及食管溃疡及声门上梗阻等严重的并发症;另外,食管阻塞器还可能插入气管,引起气道梗阻。

心肺复苏时,也可采用环甲膜切开术建立人工气道。手术时经正中切口切开环甲膜,可插入各种类型的通气管道,迅速建立人工气道。环甲膜切开术的主要优点有:①解剖标志明显,操作简单、迅速;②环甲膜位于声门之下,喉部大血管之上,因此,环甲膜切开不影响声门,而且出血少;③环甲膜切开位于气管上方,有必要时可择期进行气管切开。鉴于上述优点,环甲膜切开术被广泛推广应用,但多用于气管插管困难或无插管条件时,目前已有环甲膜切开包供临床使用。

\subsubsection{气管切开的适应证和禁忌证有哪些?}

气管切开术适应证为:①预期或需要较长时间机械通气治疗;②上呼吸道梗阻所致呼吸困难,如双侧声带麻痹、有颈部手术史、颈部放疗史;③反复误吸或下呼吸道分泌较多,患者气道清除能力差;④为减少通气死腔,利于机械通气支持;⑤因喉部疾病致狭窄或阻塞无法气管插管;⑥头颈部大手术或严重创伤需行预防性气管切开,以保证呼吸道通畅。气管切开术创伤较大,可发生切口出血或感染。

气管切开无绝对禁忌证,以下情况为相对禁忌证:①儿童;②颈部粗短肥胖,颈部肿块或解剖畸形;③颈部创伤(不稳定的颈椎骨折)或手术史;④甲状腺弥漫性肿大;⑤局部软组织感染或恶性肿瘤浸润;⑥难以纠正的严重凝血障碍;⑦需紧急建立人工气道。随着科技的进步和医学的发展,气管切开的指征在扩大,Kluge
S等
\protect\hyperlink{text00015.htmlux5cux23ch4-14}{\textsuperscript{{[}4{]}}}
研究显示,对于严重血小板减少患者经皮扩张气管切开也同样安全;Nun AB等
\protect\hyperlink{text00015.htmlux5cux23ch5-14}{\textsuperscript{{[}5{]}}}
\textsuperscript{,}
\protect\hyperlink{text00015.htmlux5cux23ch6-14}{\textsuperscript{{[}6{]}}}
研究显示,对于创伤患者同样可以行急诊气管切开。

\subsubsection{气管切开时机如何选择?}

对于需要较长时间机械通气的患者,气管切开是常选择的人工气道方式。与其他人工气道比较,由于其管腔较大、导管较短,因而气道阻力及通气死腔较小,有助于气道分泌物的清除,减少呼吸机相关性肺炎的发生率。但是气管切开的时机仍有争议。1989年美国胸科医师协会建议,预期机械通气时间在10天以内者应优先选择气管插管,而超过21天则应优先选择气管切开术,在10~21天之间者应每天对患者进行评估。当时这个建议尚缺乏临床研究的支持,是建立在专家经验之上的。之后,有研究比较了“早期”和“晚期”气管切开,探讨“最佳”气管切开时机,结果发现
\protect\hyperlink{text00015.htmlux5cux23ch12-14}{\textsuperscript{{[}12{]}}}
早期选择气管切开术,可以减少机械通气天数和重症医学科住院天数,同时可以减少呼吸机相关性肺炎的发生率。

对于“早期”的确切定义尚未统一,早至气管插管后48小时内,晚至气管插管后两周内,多数是在气管插管后7天或7天以内。Blot等
\protect\hyperlink{text00015.htmlux5cux23ch7-14}{\textsuperscript{{[}7{]}}}
对法国152个重症医学科病房机械通气的患者进行调查发现,气管切开的指征一般为机械通气20天或者拔管失败;早期气管切开的时间68%的<3周,平均为7天。Griffiths等
\protect\hyperlink{text00015.htmlux5cux23ch8-14}{\textsuperscript{{[}8{]}}}
的回顾性荟萃分析显示,早期气管切开(机械通气7天)可以降低机械通气时间和重症医学科住院时间,但不改变肺炎的发生率和病死率。

由此可见,对于需要长期机械通气或保留人工气道的患者可在7~10天行气管切开术,而对于脑血管病和颅脑外伤等患者,如预计短期内不能清醒的,气管切开时间可以更早,甚至在气管插管24小时以内。

\subsubsection{如何进行经皮扩张气管切开术?}

(1)术前准备 ①常规器械及药品准备:氧气,吸引器,面罩,喉镜,气管插管,气管切开包,纤支镜,抢救药品;②患者准备:适当镇痛镇静;③专用的经皮气管切开包:内含手术刀、带外套管的穿刺针、导引钢丝、扩张子、专用扩张钳、带气囊的气管切开套管等。

(2)体位及手术定位 ①体位:仰卧位,头后仰,肩部垫高,使下颏、喉结、胸骨上切迹三点一线,充分暴露颈部;②局部定位:选1~3气管软骨间(以甲状软骨为标志或胸骨上窝3~4cm),过高容易损伤环状软骨引起声门下的气管狭窄;过低容易损伤甲状腺峡部或无名动脉及其分支引起大出血。

(3)手术步骤 颈部消毒、局部麻醉后,横形切开皮肤1.5~2cm,在第1~2或2~3气管软骨间隙穿刺成功后置入导丝,以扩张子和扩张钳先后扩张皮肤、皮下和气管前壁(注意扩张钳角度),最后在导丝导引下置入气管套管,再次确认位置后,气囊充气并妥善固定(图\ref{fig9-6})。

\begin{figure}[!htbp]
 \centering
 \includegraphics{./images/Image00077.jpg}
 \captionsetup{justification=centering}
 \caption{经皮扩张气管切开术的主要步骤}
 \label{fig9-6}
  \end{figure} 

\subsubsection{经皮扩张气管切开与经典的气管切开有何不同?}

(1)操作者不同 经典的气管切开由耳鼻喉科医师在直视下操作,通常在手术室进行,多需要麻醉医师协助;而经皮扩张气管切开术则一般由急诊科、重症医学科医师在床边操作。

(2)操作时间及并发症不同 荟萃分析显示,经皮气管切开手术操作时间短,一般为6~10分钟,术中及术后出血少,术后感染的并发症少,且易于床旁实施。

(3)费用不同 由于经皮气管切开可在床旁进行,无需单独的手术房间和专业的麻醉师,重症医学科住院时间相对短,费用相对低
\protect\hyperlink{text00015.htmlux5cux23ch3-14}{\textsuperscript{{[}3{]}}}
。

\subsubsection{应间隔多长时间更换气管切开管?}

气管切开管的更换时间尚无统一标准。一般认为,在气道充分湿化的条件下,应1~2周更换一次。也有专家认为,在气管切开窦口无明显感染的前提下,只要气管切开管无梗阻、功能正常,就可延长更换气管切开管的时间。当然,如果气管或气管切开窦口存在明显感染,应每周更换一次。如气管切开管出现部分梗阻或气囊破裂,则应立即更换。

\subsubsection{人工气道梗阻的常见原因有哪些?如何处理?}

人工气道梗阻是人工气道最为严重的临床急症,常常威胁患者生命。导致气道梗阻的常见原因包括:①导管扭曲,多与头颈部过度活动、经鼻插管、呼吸机管道牵拉等情况有关,调整头颈部位置后,气道梗阻常可改善;②气囊疝出而嵌顿导管远端开口,常见于头颈部位置改变或管道位置改变、气囊充气过多或气囊偏心、导管使用时间过长等,此时将气囊气体抽出,多可缓解气道梗阻;③痰栓或异物阻塞管道,见于痰栓或异物阻塞人工气道;④气道坍陷,多见于经鼻插管,特别是鼻中隔偏曲压迫管道;⑤管道远端开口嵌顿于隆突、气管侧壁或支气管,多见于导管插入过深或位置不当等,调整导管位置可能缓解气道梗阻。

一旦发生气道梗阻,应采取以下对策:①调整人工气道位置;②抽出气囊气体抽出;③试验性插入吸痰管。如气道梗阻仍不缓解,则应立即拔除气管插管或气管切开管,然后重新建立人工气道。若重新建立人工气道后,气道压力仍然很高,呼吸机不能进行有效的机械通气,则应当注意排除张力性气胸。

当然,积极采取措施预防气道梗阻可能更为重要,认真的护理、密切的观察、及时的更换管道及有效的人工气道护理,可对气道梗阻起到防患于未然的作用。

\subsubsection{建立人工气道的患者有哪些原因可导致气道出血?}

建立人工气道的患者出现气道出血,特别是大量鲜红色血液从气道涌出时,往往威胁患者生命,需要紧急处理。气道出血的常见原因包括:

(1)医源性因素是引起气道出血常见的原因,往往与吸痰管抽吸引起气道损伤有关。抽吸时负压作用于气管黏膜,引起黏膜损伤和出血,出血量往往不多。

(2)肺部感染也可引起气道出血,支气管肺炎和坏死性肺炎均可导致气道、肺内出血。

(3)急性心源性肺水肿可出现大量粉红色泡沫痰,当反复气道抽吸或存在出血倾向时,患者气道可涌出大量鲜红色血痰。

(4)少数患者可因肺栓塞而出现肺出血,可见于深静脉血栓脱落或深静脉导管相关的血栓脱落。

(5)肺动脉导管嵌顿时间过长亦可引起医源性肺梗死,而引起肺出血。另外,导管插入过深或飘入远端肺小动脉,气囊充气时可引起肺小动脉破裂而出血。

(6)气管导管和气管切开管气囊压迫腐蚀气道,可引起出血。最为严重的是,气囊压迫腐蚀引起无名动脉破裂出血,此种情况死亡率极高。

(7)出血性疾病或凝血功能障碍患者也常常出现气道出血。一旦出现气道出血,应针对原因,及时处理。

\subsubsection{气管切开可能出现哪些并发症?}

气管切开是建立人工气道的常用手段之一。由于气管切开后气流不经过上呼吸道,因此,与气管插管相比,气管切开具有许多优点:几乎没有上呼吸道的并发症;易于固定;易于呼吸道分泌物引流;附加阻力低,而且易于实施呼吸治疗措施;不影响经口进食,可做口腔护理;患者耐受性好。尽管具有上述优点,气管切开也可引起许多并发症,根据并发症出现的时间,可分为早期、后期并发症及拔管后并发症。以下着重讨论早期和后期并发症。

早期并发症指气管切开24小时内出现的并发症。常见的早期并发症如下。

(1)出血 是为最常见的早期并发症。出血凝血机制障碍的患者,术后出血发生率更高。出血部位可能来自切口、气管壁。气管切开部位过低,如损伤无名动脉,则可引起致命性的大出血。切口的动脉性出血需打开切口,手术止血。非动脉性出血可通过油纱条等压迫止血,一般在24小时内可改善。

(2)气胸 是胸腔顶部胸膜受损的表现,胸膜腔顶部胸膜位置较高者易出现,多见于儿童、肺气肿等慢性阻塞性肺疾病患者。上呼吸道梗阻患者,如梗阻未解除时实施气管切开,常常因存在过度肺充气、胸膜顶部位置高而易发生气胸。这类患者应首先插入气管插管,之后再行气管切开较为安全。

(3)空气栓塞 是较为少见的并发症,与气管切开时损伤胸膜静脉有关。由于胸膜静脉血管压力低于大气压,损伤时,空气可被吸入血管,导致空气栓塞。患者采用平卧位实施气管切开,有助于防止空气栓塞。

(4)皮下气肿和纵隔气肿 是气管切开后较常见的并发症。颈部皮下气肿与气体进入颈部筋膜下疏松结缔组织有关。由于颈部筋膜向纵隔延伸,气体也可进入纵隔,导致纵隔气肿。皮下气肿和纵隔气肿本身并不会危及生命,但有可能伴发张力性气胸,需密切观察。

后期并发症是气管切开24~48小时后出现的并发症,发生率可高达40%。主要包括:

(1)切口感染 这是很常见的并发症。感染切口的细菌可能是肺部感染的来源,故应加强局部护理。

(2)出血 气管切开后期也可发生出血,主要与感染组织腐蚀切口周围血管有关。当切口偏低或无名动脉位置较高时,感染组织腐蚀及管道摩擦易导致无名动脉破裂出血,为致死性的并发症。

(3)气道梗阻 是可能危及生命的严重并发症。气管切开导管被黏稠分泌物附着或形成结痂、气囊偏心疝入管道远端、气管切开管远端开口顶住气管壁等原因均可导致气道梗阻。一旦发生,需紧急处理。

(4)吞咽困难 这是较常见的并发症,与气囊压迫食管或管道对软组织牵拉影响吞咽反射有关。气囊放气后或拔除气管切开管后可缓解。

(5)气管食管瘘 临床偶见,主要与气囊压迫及低血压引起局部低灌注有关。

\subsubsection{如何预防和处理人工气道的意外拔管?}

意外拔管是指无拔管指征的患者,人工气道意外脱出。常见原因包括:患者烦躁或意识不清而意外拔管,固定不当,呼吸机管道牵拉及气管切开管过短等。意外拔管后患者可能出现以下情况:失去有效呼吸通道而发生窒息;完全依赖机械通气的患者则出现呼吸暂停;有自主呼吸的患者可能出现肺泡低通气等。无论出现哪种情况,均可能危及生命,因此,一旦发生意外拔管应紧急处理。

为避免意外拔管,须积极预防其发生,具体措施包括:①正确固定气管插管或气管切开管,每日检查,并及时更换固定胶布或固定带,气管切开管固定带应系方结,固定带应系紧,与颈部的间隙不宜超过两指;②检查气管插管深度,插管远端应距隆突2~3cm,过浅易脱出;③颈部较短的肥胖患者,如气管切开管较短,则头部活动时,易使导管脱出到皮下组织及脂肪组织中,引起呼吸道梗阻,此类患者宜选用较长的气管切开管;④对于烦躁或意识不清的患者,宜用约束带将其手臂固定,防止拔管;⑤呼吸机管道不宜固定过牢,应具有一定的活动范围,以防患者翻身或头部活动时导管被牵拉而脱出。

一旦发生意外拔管,应立即重建人工气道。气管切口3~5天内者,气管切开窦口尚未形成,气管切开管难以重新插入,可先行经口气管插管。对于气管插管困难者,可用简易呼吸囊面罩加压给氧,为进一步处理赢得时间。

\subsubsection{气管切开48小时内气管切开管意外脱出,应如何处理?}

在气管切开后48小时内,如气管切开管意外脱出,则不但换管困难,而且并发症较多。在气管切开48小时内,应注意以下问题:①由于气管切开窦道尚未形成,一旦拔出气管切开管,气管切开窦口将关闭,此后很难将气管切开管重新插入,由此可能引起呼吸道梗阻和严重缺氧,后果极为严重,应引起医护人员高度重视;②窦口肉芽组织尚未形成,重新插入气管切开管往往会引起出血;③气管切开管必须牢固固定,固定带应打死结,与颈部的间隙不应超过两指,另外,注意呼吸机管道不要过于固定,以免患者头颈部移动时,气管切开管被呼吸机管道牵拉而脱出;④患者床边应准备气管切开包、气管插管、简易呼吸囊等急救设备;⑤气管切开管一旦意外脱出或需紧急更换,应立即使用面罩和简易呼吸囊进行辅助通气,并吸入纯氧,保证患者的供氧和通气,如气管切开窦口漏气,可用纱布暂时封闭;⑥保证患者氧供的同时,立即呼叫耳鼻喉科医师,以便重新打开关闭的窦口,直视下插入气管切开管;⑦气管切开管重新插入前,必须认真检查气囊,以免插入后发现漏气而再次更换;⑧重新插入气管切开管后,必须认真固定管道。另外,意外拔管时,气囊上潴留的分泌物常常流入气管,引起误吸,可能导致或加重肺部感染,因此,必须彻底冲洗并抽吸气管;⑨整个操作期间,应注意监测患者心电图、经皮指脉氧饱和度和血压。

\subsubsection{如何调整气管插管或气管切开管气囊压力?}

由于气囊压力是决定气囊是否损伤气管黏膜的重要因素,调整气囊压力就显得特别重要。

正常成年人气管黏膜的动脉灌注压大约在30mmHg(42cm H\textsubscript{2}
O),毛细血管静脉端压力为18mmHg(24cm H\textsubscript{2}
O),淋巴管压力为5mmHg。由此可推测,气囊压力高于30mmHg时,气管黏膜血流将完全被阻断,可引起黏膜缺血;当气囊压力高于18mmHg,将引起气管黏膜静脉回流受阻而出现淤血;当气囊压力高于5mmHg时,将阻断淋巴回流,引起黏膜水肿。气囊充气过多,压力过高,会引起黏膜损伤;而压力过低则不能有效封闭气囊与气管间的间隙。因此,必须注意调整气囊压力,避免压力过高或过低。理想的气囊压力为有效封闭气囊与气管间隙的最小压力,常常称为“最小封闭压力(MOP)”。目前推荐气囊压力>20cm
H\textsubscript{2} O,一般维持在25~35cm H\textsubscript{2}
O,可以通过Portex气囊压力测定仪进行调整,一般每天监测2~3次气囊压力,避免气囊压力过高或过低。

\subsubsection{为什么气管插管或气管切开管气囊不需定期放气?}

以往认为,气管插管或气管切开管气囊应常规定期放气-充气,其主要目的是通过放气(多为3~5分钟)恢复气管黏膜血流,防止气囊压迫导致气管黏膜损伤。

目前认为,气囊定期放气-充气是不必要的,主要依据如下:

(1)气囊放气后,1小时内气囊压迫区的黏膜毛细血管血流也难以恢复。气囊放气5分钟不可能恢复局部血流。可见,短时间气囊放气不能达到恢复黏膜血流的目的。

(2)声门与气囊之间的间隙常常有大量分泌物潴留,定期气囊放气有可能增加了反复误吸的可能性。

(3)对于机械通气支持条件比较高的危重患者,特别是依赖于高水平呼气末正压(PEEP)的呼吸衰竭患者,气囊放气将导致肺泡通气不足,PEEP不能维持,并可能引起循环波动,因此,危重患者往往不能耐受气囊放气。

(4)常规的定期气囊放气-充气,往往使医师或护士忽视气囊容积或压力的调整,反而易出现充气过多或压力过高的情况。

虽然人工气道气囊不需常规放气-充气,但某些情况下,非常规性的放气或调整仍然是必要的。气囊放气及重新充气主要用于以下情况:

(1)气道峰值压力是影响气管最大内径的主要因素,当气道峰值压力明显升高或降低时,为避免气囊压力(或容积)过高或过低,应将气囊放气,重新充气,并测定气囊压力,保持在20~35cm
H\textsubscript{2} O。

(2)人工气道的建立破坏了呼吸道的正常解剖和功能,声门与气囊之间的间隙成为一死腔,常常有大量分泌物在此潴留,可能形成隐匿感染灶。因此,经常清除这些分泌物,保持声门下和气囊上区域的清洁是十分必要的。研究表明,有效的声门下吸引可降低呼吸机相关性肺炎的发生率
\protect\hyperlink{text00015.htmlux5cux23ch11-14}{\textsuperscript{{[}11{]}}}
。对于不带声门下吸引的普通导管,清除气囊上分泌物的方法之一就是在气囊放气的同时,通过呼吸机或手动简易呼吸囊,经人工气道给予较大的潮气量,在塌陷的气囊周围形成正压,将潴留的分泌物“冲”到口咽部,从而达到既清除气囊上分泌物,又防止了气囊放气后分泌物流入气管的目的。

\subsubsection{人工气道患者实施气道抽吸时,应如何选择适当的吸痰管?}

吸痰时,为避免黏膜创伤及继发感染,应选择适当的吸痰管。吸痰管应符合以下要求:①吸痰管材料应对黏膜的损伤小;②吸痰管摩擦力小,以利于通过人工气道;③足够的长度,使吸痰管远端应能达到人工气道远端或隆突,否则难以达到抽吸气道分泌物的目的;④远端光滑,而且应该为侧开口,以减少对黏膜损伤;⑤吸痰管近端应有足够大的侧孔,需要中断负压吸引时,只要开放侧孔即可,可避免负压持续吸引引起黏膜损伤或肺不张,也可避免反复关闭负压吸引器;⑥吸痰管直径(外径)不应超过人工气道内径的一半,如吸痰管直径过大,负压吸引时,吸痰管周围卷入的空气较少,易导致肺萎陷或肺不张;⑦吸痰管应无菌、单根包装,以避免交叉感染,而且操作方便,如有条件,应使用一次性吸痰管。

\subsubsection{呼吸道负压抽吸吸痰的操作要点是什么?}

为减少气管损伤、感染等并发症,经人工气道进行呼吸道负压引流(吸痰)时应注意以下操作要点:

(1)注意无菌操作,拿吸痰管时应戴无菌手套,使用无菌的吸痰管,应用无菌的冲洗盐水等。绝对禁止用抽吸口鼻腔的吸痰管再抽吸气道。

(2)吸痰前必须预充氧,使体内获得氧贮备。通过手动呼吸,吸入高浓度氧。接受机械通气的患者,可通过吸入纯氧3~5分钟达到预充氧的目的。充分的预充氧,可避免发生低氧血症。

(3)吸痰管插到气管插管远端前,不能带负压,以免过度抽吸肺内气体,引起肺萎陷。

(4)插入吸痰管过程中,如感到有阻力,则应将吸痰管后退1~2cm,以免引起支气管过度嵌顿和损伤。

(5)在吸痰管逐渐退出的过程中,打开负压吸痰,抽吸时应旋转吸痰管,并间断使用负压,可减少黏膜损伤,而且抽吸更为有效。

(6)吸痰管在气道内的时间不应超过10~15秒,而从吸痰过程开始到恢复通气和氧合的时间不应超过20秒。

(7)抽吸期间应密切注意心电监测,一旦出现心律失常或呼吸窘迫,应立即停止抽吸,并吸入纯氧。

(8)通气和氧合恢复后至少进行5次深呼吸,生命体征恢复到基础水平后,才可再次抽吸。

(9)经反复抽吸,应较彻底地清除分泌物。

(10)气道抽吸后,可使用同一吸痰管抽吸口、鼻、咽腔,但抽吸过口鼻咽腔后,绝不可再抽吸气管。

(11)气道分泌物的抽吸不应作为常规操作,当患者有气道分泌物潴留的表现时,才有指征抽吸。过多的抽吸会刺激、损伤气道黏膜。

\subsubsection{呼吸道负压抽吸吸痰应注意哪些并发症?如何处理?}

由于人工气道影响患者正常咳嗽反射,大气道内的分泌物必须通过抽吸清除。对于危重患者,吸痰时的抽吸技术不当,可能引起严重的并发症,甚至引起心脏骤停,因此,了解气管吸痰过程中的有关并发症,对于防止和及时处理并发症都是十分必要的。

(1)低氧血症 多数带有人工气道的患者需接受不同程度的氧疗,以维持动脉血氧分压。吸痰时,吸痰管插入气道,负压抽吸将肺内的富氧气体吸出,而从吸痰管周围卷入的气体是氧浓度较低的空气,容易导致低氧血症。对于危重患者,低氧血症的恶化往往会威胁患者生命。

吸痰导致的急性低氧血症往往表现为心率改变。多数患者表现为心动过速,重新吸入高浓度氧气后,心率逐渐降低。少数患者表现为心动过缓。在吸痰过程中,心率及心律的任何改变均应考虑与低氧血症有关。

吸痰前通过提高吸入氧浓度(预充氧),提高机体内氧贮备是防止低氧血症的重要措施。另外,应用封闭式吸痰管使吸痰时不中断氧疗(不脱开呼吸机或氧疗系统),也可一定程度上防止低氧血症。

(2)心律失常 吸痰过程中发生的严重心律失常,主要与低氧血症引起心肌缺氧或气管黏膜受刺激后导致迷走神经兴奋有关,这两个因素哪个更为重要目前尚不清楚。抽吸过程中出现的室性早搏等严重心律失常,可能与这两因素均有关,而明显的心动过缓主要由迷走神经兴奋引起。如操作适当,可减少心律失常发生的可能性。

(3)低血压 吸痰过程中常常发生血压降低,可能与迷走神经兴奋引起心动过缓有关,也可能与连续咳嗽有关。吸痰管刺激气管黏膜或隆突,引起气管或隆突反射,会使患者出现阵发性咳嗽样动作,并伴有心动过缓,导致静脉回流和心输出量均明显降低,结果出现低血压。

(4)肺萎陷或肺不张 负压抽吸时,如吸痰管周围没有足够的空气卷入,容易导致肺萎陷或肺不张。吸痰管直径过大(超过人工气道内径一半)或负压过大时易于发生。因此,选用适当直径的吸痰管,并采用合适的负压是有必要的。

\subsubsection{如何实现左主支气管的选择性抽吸吸痰?}

由于左主支气管从气管发出的角度较小,吸痰管不易进入。吸痰时,吸痰管是否能够进入左主支气管,主要取决于以下条件:①人工气道的类型。与气管插管相比,经气管切开管吸痰更易进入左主支气管。②头的位置。头转向右侧,吸痰管易进入左主支气管。③吸痰管的类型。弯头吸痰管较易进入左主支气管。需要抽吸左主支气管时,应根据上述情况,选择适当的吸痰管和体位。

\subsubsection{何谓气道湿化?正常上呼吸道的湿度如何?}

气道湿化是指应用湿化器及其他装置将溶液或水分分散成极细微粒,以增加吸入气中的湿度,使气道和肺部能吸入含足够水分的气体,湿化气道黏膜、稀释痰液、保持黏液纤毛正常运动和廓清功能。

气道湿化时经常用到“湿度”这一物理学概念。所谓湿度即空气中所含水分的多少或潮湿程度。单位容积的气体中所含水分的重量称之为绝对湿度,常用计量单位为mg/L或g/m\textsuperscript{3}
。在一定温度下,气体实际所含水分与该温度下每单位容积所能容纳的最大水分含量的比值称为相对湿度,可用湿度计进行测量。

正常呼吸过程中,上呼吸道将干燥、温度较低的空气,逐步转化为湿润温暖的气体后到达肺泡进行气体交换。一般空气的温度21℃,相对湿度为50%,吸入气体经过鼻腔、咽喉到达气管上段时,温度已达34℃,相对湿度为100%,绝对湿度36~40mg/L;到达气管隆突时,温度约37℃,相对湿度为100%,绝对湿度约43.9mg/L。可见,人体在吸入空气时,呼吸道必须对空气进行加温加湿,生理情况下这一过程由上呼吸道完成。正常成人经气道蒸发的水分约为250ml/天,当遇到发热、过度通气或吸入干燥空气后水分丢失更多。人工气道的建立,使患者在吸气过程中丧失了上呼吸道对吸入气体的加温加湿功能,为维持相应的状态必须进行充分的气道湿化。

\subsubsection{气道管理中为什么要重视气道湿化?}

人工气道的建立,使危重患者在吸气过程中丧失了上呼吸道对吸入气的加温和加湿功能,只能吸入干燥和温度较低的空气,结果吸入气的湿化和加温功能由气管支气管树黏膜来完成,因气体湿化不足,易引起气管黏膜干燥、分泌物黏稠而形成痰栓,导致多种严重后果,这主要包括:①黏膜纤毛运动受损;②黏液的移动受限;③气管支气管黏膜上皮发生炎症性改变甚至坏死;④黏稠分泌物潴留,进而形成痰痂,严重者可发生气管梗阻;⑤细菌易浸润气管黏膜,导致肺部感染;⑥黏稠分泌物阻塞小气道,易发生肺不张。由此可见,人工气道的管理中,必须强调给予充分的气道湿化,防止可能发生的不良后果。由于湿化的主要目的是替代上呼吸道的加湿和加温功能,因此,经湿化的气体相对湿度应当达到100%,温度应达到35~37℃。

当然,过度加温和湿化也可造成有害的影响,这包括:①湿化器温度过高,可以引起气道黏膜温度过高或烧伤,导致肺水肿和气道狭窄;②如果吸入的气体没有加热,但直接经呼吸道给予大量水分,会由于需要蒸发消耗热量而导致体温下降、体液负荷增加、黏膜纤毛的清除功能减退。

\subsubsection{气道湿化有哪些适应证和禁忌证?}

适应证为:①未建立人工气道而使用干燥的医疗气体者,如对于吸氧流量超过4L/分钟的;②建立人工气道者;③高热、脱水;④呼吸急促或过度通气;⑤痰液黏稠或咯痰困难;⑥气道高反应,部分原因是由于干冷气体诱发的气道痉挛;⑦低体温,尤其是低温冻伤在复温过程中的机械通气患者。

气道湿化无绝对禁忌证。气道分泌物过多且稠厚或血性分泌物时应慎用气道湿化,以免加重气道梗阻,甚至窒息。

\subsubsection{保持呼吸道湿化的常用方法有哪些?}

(1)保证充足的液体入量 机械通气时,液体入量应保持每日2500~3000ml。呼吸道湿化尽量以全身不脱水为前提,如果机体液体入量不足,即使呼吸道进行湿化,呼吸道的水分会进入到失水的组织中,呼吸道仍可能处于失水状态,所以,必须补充足够的液体入量。

(2)加温湿化器 加温湿化器以物理加热的办法为干燥气体提供恰当的温度和充分的湿度,能使湿化后的气体达到100%的湿度。湿化罐温度的控制应以气管插管或气管切开管处的气体温度达到37℃为准,因此,监测气管插管或气管切开处的气体温度是必要的。湿化加水时,应加无菌的蒸馏水或注射用水,注意整个操作过程保持无菌。管路内凝结的水,若意外地灌进患者的气道,可成为医院感染的来源,因此管内的冷凝水应收集于积水瓶中,并及时清除,也不能使冷凝水流回湿化器。

(3)湿热交换器 也称人工鼻。该装置放置在“Y”形管与气管导管之间,为被动湿化。工作原理为随温度的变化,携水能力有连续性变化。呼气时,随温度的下降,呼出的水分被截留在人工鼻中;吸气时,温度逐渐升高,人工鼻的水分补充到吸入气体中。优点包括:①保证黏液纤毛系统运动正常;②减少热量丧失;③保证管路干燥,减少细菌孳生,防止感染的发生;④操作简单,可以不需要每日更换,减少如加温湿化器因需加水而多次管路断开导致的交叉感染。但近期研究显示,湿热交换器并不能降低呼吸机相关性肺炎的发生率
\protect\hyperlink{text00015.htmlux5cux23ch10-14}{\textsuperscript{{[}10{]}}}
\textsuperscript{,}
\protect\hyperlink{text00015.htmlux5cux23ch11-14}{\textsuperscript{{[}11{]}}}
。在有下列情况的患者人工鼻不适用:分泌物黏稠或血性,患者中心体温<32℃,呼出潮气量<吸气潮气量的75%(如气管胸膜瘘等),呼出潮气量>10L/分钟,咯血,撤机困难等。

(4)雾化器 临床有喷射式雾化器和超声雾化器,通过雾化器将湿化液激发为微粒或雾粒,悬浮在吸入气流中一起进入气道而达到湿化气道的目的。雾化器产生雾粒的量和平均直径的大小,随雾化器种类而不同。

(5)气道冲洗 常用生理盐水,在吸痰前予2~5ml生理盐水在吸气末注入气道。操作前,先给予纯氧2~3分钟,以免造成低氧血症。注入冲洗液后,给予吸痰或配合胸部叩拍,使冲洗液和黏稠的痰液混合震动后再吸出。全天湿化液总量一般不超过250ml。

\subsubsection{机械通气时人工气道的主动湿化和被动湿化有何不同?}

机械通气时的气道湿化包括主动湿化和被动湿化。主动湿化主要指在呼吸机管路内应用加热湿化器进行呼吸气体的加温加湿(包括不含加热导线,含吸气管路加热导线,含吸气呼气双管路加热导线);被动湿化主要指应用人工鼻(热湿交换器型)吸收患者呼出气的热量和水分进行吸入气体的加温加湿。不论何种湿化,都要求进入气道内的气体温度达到37℃,相对湿度100%,以更好地维持气道黏膜完整,纤毛正常运动及气道分泌物的排出,降低呼吸道感染的发生。人工鼻(热湿交换器型)可较好进行加温加湿,与加热型湿化器相比不增加堵管发生率,并可保持远端呼吸机管路的清洁,但可能增加气道阻力、死腔容积及吸气做功,不推荐在慢性呼衰患者尤其是有撤机困难因素的患者应用。有研究认为,人工鼻(热湿交换器型)较加热型湿化器能减少院内获得性肺炎的发生,但近年来多个随机对照临床研究认为,人工鼻与加热型湿化器比较,在呼吸机相关性肺炎的发生率上无明显差异。

\subsubsection{气道湿化效果如何判断?}

湿化效果应根据患者的自觉症状和监测指标变化来判断,同时应把这些自觉症状和监测指标的变化与病情相结合,防止误判或延误患者的治疗。一般把湿化效果归为以下3种:

(1)湿化满意 ①痰液稀薄,能顺利吸出或咯出;②人工气道内无痰栓;③听诊气管内无干鸣音或大量痰鸣音;④呼吸通畅,患者安静。

(2)湿化过度 ①痰液过度稀薄,需不断吸引;②听诊气道内痰鸣音较多;③患者频繁咳嗽,烦躁不安,人机对抗;④可出现缺氧性紫绀、经皮指脉氧饱和度下降及心率、血压改变等。

(3)湿化不足 痰液黏稠,不易吸出或咯出;听诊气道内有干鸣音;人工气道内可形成痰痂;患者可出现突然的吸气性呼吸困难、烦躁、紫绀及脉搏氧饱和度下降等。

\subsubsection{气管插管或气管切开管的拔管指征是什么?}

总的说来,当建立人工气道的原发病得到控制时,就有拔除气管插管或气管切开管的指征。与建立人工气道的指征类似,拔管指征也从4个方面进行考虑。

(1)引起上呼吸道梗阻的因素已去除 当上呼吸道梗阻的病因缓解后,可考虑拔管,但拔管后应密切观察患者是否在此重新出现上呼吸道梗阻的症状。中枢神经系统受损是引起上呼吸道梗阻的常见原因,中枢神经功能改善将使上呼吸道梗阻改善。

(2)气道保护性反射恢复 气道保护性反射受影响的顺序,从重到轻依次为咽、喉、气道及隆突反射,因此,评价气道保护性反射是否恢复,可观察咽反射是否恢复。带管情况下,如果患者存在吞咽反射(咽反射的表现),则喉、气管及隆突反射应当是正常的,这是安全拔管的前提。

(3)具有呼吸道清洁能力 是否需要气道抽吸清除分泌物,在很大程度上是由患者咳嗽能力决定的。对患者咳嗽能力的评价,可通过观察患者咳嗽的强度、肺活量、最大吸气负压及意识水平来决定。如果患者在带管情况下具有一定咳嗽能力,肺活量接近正常,而且患者能够合作,则拔除人工气道后,患者大多具有气道清洁能力。

(4)已撤离呼吸机 如建立人工气道的主要目的是实施机械通气,又在撤离呼吸机后符合上述条件,则可考虑拔除人工气道。

在此需要说明的是,脱机与拔除人工气道是两回事,部分患者经过治疗后可以脱离呼吸机,但因气道自洁能力差,仍需要人工气道;而另外一部分患者,如慢性阻塞性肺疾病,即使不符合拔管条件,但在感染控制窗内仍可以考虑拔除气管插管,实施有创通气无创通气序贯脱机。

\subsubsection{如何正确地拔除气管插管?}

对于有拔除气管插管指征的患者,一旦决定拔管应遵循以下操作过程:

(1)拔管后患者的合作十分重要。拔管前应让患者了解拔管的必要性和安全性,消除患者心理负担,使其充分合作,另须准备好吸氧装置、口腔护理物品、纸巾、雾化装置,必要时备无创通气。

(2)彻底、充分地吸引气道分泌物之后,清除口咽及鼻咽部分泌物,如为带声门下引流的导管,应充分冲洗抽吸声门下引流管;如为普通导管,可通过在气囊放气的同时,通过呼吸机或简易呼吸囊,经人工气道给予较大的潮气量,以期在塌陷的气囊周围形成正压,将潴留的分泌物“冲”到口咽部,再给予吸出,避免误吸。

(3)适当提高吸入氧浓度,增加体内氧贮备。

(4)让患者深呼吸数次或通过简易呼吸囊给予较大潮气量,以鼓肺,复张塌陷肺泡,改善肺不张。

(5)将新的吸痰管置于气管插管远端开口以远1~2cm,边抽吸,边气囊放气,并快速拔除气管插管。

(6)采用合适的氧疗措施及口腔护理。

(7)立即评价患者气道是否通畅,有无气道梗阻的症状,有无喘鸣或呼吸困难,鼓励患者做深呼吸。

(8)病情完全稳定前,应给予特别护理。床边应备有急救设备。

另外,为防止声门及声门下水肿,在拔管前可给予肾上腺素雾化吸入或地塞米松雾化吸入或静脉注射。

\subsubsection{人工气道拔除后发生喉头水肿应如何处理?}

喉头水肿是气管插管拔管后最严重的并发症之一,严重者可危及生命,需要紧急处理。喉头水肿实际上是指声门区域发生水肿。儿童的发生率远高于成人。如能早期认识、早期处理,多数患者不需再插管。

引起喉头水肿的主要原因包括:①插管及留置期间对声门区域有损伤;②气管插管内径过大;③气管插管护理不当;④导管引起的过敏反应。

轻度喉头水肿无症状。当喉头水肿引起声门狭窄,声门截面积小于正常50%时,患者出现临床症状,主要表现为吸气期喘鸣、呼吸困难,并进行性加重。由于喉部位于胸腔外,自主呼吸时吸气期声门最为狭窄,所以,患者在吸气期出现喘鸣。一般在气管插管拔除后随即出现症状,水肿在数小时内达到极限,而后逐渐缓解。当然,拔管后无喉头水肿症状立即出现,并不意味着患者就不再会发生喉头水肿,故在拔管后24小时内均应警惕。

一旦出现喉头水肿症状,应积极治疗,以防进一步恶化。主要治疗措施包括:①吸入室温的雾化气体,以保证声门区域充分湿化,同时减轻对黏膜的刺激,减轻毛细血管水肿和充血;②局部应用血管收缩药物,如麻黄素、肾上腺素雾化吸入或直接喷入咽喉部;③静脉注射地塞米松等糖皮质激素类药物,以改善声门水肿,也可局部用药;④当上述措施不能奏效或气管梗阻很严重时,应立即重新建立人工气道。

\subsubsection{人工气道拔除后发生气管狭窄的主要原因有哪些?}

在人工气道(气管插管或气管切开管)建立后1周至2年期间,均有可能发生气管狭窄。气管狭窄是气管局部损伤愈合过程中,瘢痕组织收缩的结果。有的患者同时伴有气管软化,即气管软骨被破坏,表现为吸气时气管塌陷。气管狭窄的发生部位主要在气囊压迫部位。临床研究调查发现,使用不带气囊的气管切开管,气管狭窄的发生率低于2%,而使用带气囊的气管切开管时,气管狭窄的发生率可达5%。机械通气患者的调查发现,气管狭窄发生率为1%~65%。结果的差异性与研究方法、气管狭窄的诊断标准及研究人群不同等因素有关。一般认为,气管内截面积减少50%以上,气管狭窄患者才出现喘鸣、呼吸困难、活动能力下降等临床表现。

气管狭窄的发生部位似乎提示气囊是引起气管狭窄的主要原因,但临床观察及动物实验研究显示,气管狭窄不仅与气囊有关,还与其他许多因素相关。总的来说,是否发生气管狭窄主要与以下因素有关。

(1)气囊压力过高是导致气管狭窄最重要的因素之一 气囊压力过高可引起气管黏膜水肿、淤血及缺血,进而可能引起黏膜发生糜烂、溃疡。基于这一认识,须强调气囊压力的监测和调整。

(2)人工气道的维持时间也是引起气管狭窄的因素 人工气道维持的时间越长,气管狭窄的发生率越高、越严重.

(3)低血压低灌注是导致气管狭窄发生的重要因素 休克或一过性低血压可引起气管黏膜灌注降低,使气囊压迫部位的气管黏膜易发生缺血、坏死,因此,防止并及时纠正休克是防止气管狭窄的重要环节。

(4)局部感染也是导致气管狭窄的原因之一 建立人工气道的患者,气管内细菌定植是难以避免的,易继发感染。当然,良好的气道护理可防止急性气管感染发生。调查显示,感染虽然不是气管狭窄的主要原因,但增加气管狭窄的发生率。

(5)人工气道的活动、牵拉或负重等因素也增加气管狭窄的发生率 人工气道的活动、牵拉或负重可导致气管切开窦口周围压力增高及气囊移位,并刺激气管黏膜,结果气管黏膜易发生缺血坏死。可见有必要采取措施,适当固定人工气道,可采用支撑架托住气管插管或气管切开管,同时支撑架需要有一定的活动度,患者头颈部移动时人工气道不致被牵拉。

(6)人工气道对人体组织的毒性作用也可能与气管狭窄有关 如人工气道的组织相容性不佳,则可能造成周围气管组织发生炎症反应。

针对上述导致气管狭窄的原因,应采取相应的对策,以防止气管狭窄的发生。

\subsection{胸部物理治疗}

\subsubsection{胸部物理治疗的主要目的和主要手段是什么?}

胸部物理治疗是防止肺部并发症、改善急(慢)性肺疾病患者肺功能的物理治疗技术,是临床危重患者呼吸治疗的主要内容之一。胸部物理治疗在国际上应用普遍,在我国近年来也受到广泛重视。由于不同的胸部物理治疗技术的目的、指征及注意事项不同,危重病医学专业医师及呼吸治疗师必须熟悉各种治疗手段的指征以及对气道清洁和通气的影响,还必须了解操作方法和治疗强度。

胸部物理治疗主要目的包括:①防止气道分泌物潴留,促进分泌物清除;②改善肺脏的通气/血流分布,提高患者呼吸效能;③通过功能锻炼,改善心肺功能贮备。

胸部物理治疗手段主要分两类:①促进气道清洁的技术,包括体位引流、胸部叩击、胸部震颤、刺激咳嗽等;②增强患者呼吸效能的技术,深吸气锻炼和刺激性肺量计。不同技术具有不同的目的和指征,应用时应予以注意。

\subsubsection{体位引流的目的是什么?实施时应注意哪些问题?}

体位引流是胸部物理治疗的重要手段之一,其主要目的:①促进气道分泌物清除。危重病患者气道黏膜纤毛的运动降低,清除分泌物能力下降,是导致气道分泌物潴留的主要原因之一。体位引流是利用重力作用,促进分泌物流动,有利于分泌物排出。②改善肺内通气/血流分布,这一作用常常被忽视。由于受重力作用,肺脏下垂部位的血流分布增多,同时下垂部位分泌物引流困难,易发生感染或不张,使局部通气减少,结果导致下垂部位通气/血流比例严重失调,是引起低氧血症的常见原因。体位引流可促进下垂部位分泌物清除,同时体位的改变可使下垂部位转变为非下垂部位,最终可导致通气/血流比例改善,有利于改善低氧血症。

对于正常气道清洁功能受损、气道分泌物潴留的患者,有指征实施体位引流。常见疾病包括支气管炎、慢性阻塞性肺疾病、急性肺不张、肺脓肿、肺炎、囊性纤维化等疾病,接受机械通气的患者亦有指征进行体位引流。当然,对于肺脓肿的体位引流应特别注意,引流时应防止引流出的脓液污染健侧肺支气管。另外,患侧引流后,对健侧应做常规引流,以减少污染的机会。胸腔积脓的患者应避免体位引流,因体位改变可能导致脓液在胸腔扩散,并有可能感染健侧胸腔。

体位引流应根据肺部病变部位,决定应采取的体位。下肺病变时,为引流下叶支气管,应采取仰卧、头低脚高位;上叶病变,应采用半坐位引流;右中叶或左舌叶病变,需采用侧卧位。

体位引流应注意以下问题:①对于危重患者,体位改变可能影响循环,循环极不稳定的患者应避免;②采取头低脚高位进行体位引流时,头部静脉回流阻力增加,可使颅内压增高,颅脑术后患者及有颅内高压的患者,应避免做头低脚高位的引流;③体位引流可增加缝合切口的张力,因此,对于做植皮和脊柱手术的患者应特别注意。

\subsubsection{如何做胸部叩击和胸部震颤?}

胸部叩击和胸部震颤都是促进气道清洁的重要手段,常常与体位引流等手段一起应用。胸部叩击是将双手指并拢,手掌呈杯状,然后双手交替对胸部病变部位进行节律性叩击。叩击时产生的压缩空气释放机械能,通过胸壁传导至肺部。理论上,传导至肺部的能量能够促进粘附于气管壁的痰液有所松动,并有利于分泌物向外移动。

对于发生恶性肿瘤骨转移、有全身出血倾向、脓胸未引流的患者以及易发生骨折的高龄患者,胸部叩击为相对禁忌。即使必须进行胸部叩击,也须特别慎重。

胸部震颤是将手掌放在患者胸部表面,操作者肩部和手掌快速、小幅度地颤动,并沿肋骨方向轻轻地压迫患者胸部,震颤频率可高达每分钟200次以上。胸部震颤应在患者呼气时进行。胸部震颤主要促进痰液活动和清除,同时呼气时按压胸部促使肺内气体呼出。对于自主呼吸的患者,实施胸部震颤应要求患者深呼吸。对于机械通气的患者,可用手动呼吸机给予患者大潮气量的同时,进行胸部震颤。大潮气量可促进支气管肺扩张或膨胀,治疗效果更佳。胸部震颤应与胸部叩击等措施一起应用。

\subsubsection{如何评价胸部物理治疗的疗效?}

患者接受胸部物理治疗后,应对治疗效果进行评估。

(1)观察痰液性状往往能对疗效评价提供重要资料 胸部物理治疗后,痰量改变大约滞后24小时,因此,需在治疗后24小时对痰液进行评价。正常情况下,痰液为白色、半透明的液体。观察痰液颜色和黏稠度等性状的改变有重要临床价值。当痰液中有大量白细胞或脓细胞时,痰液呈黄色。出现绿色痰液,提示痰液在肺内潴留时间较长,与蛋白酶酶解黏肽有关。如绿痰伴有腐臭味,则提示假单胞菌感染。出现棕色痰液与陈旧性血液有关,而红色痰液提示新鲜出血。另外,应注意痰液的显微镜下改变,是否有白细胞、红细胞,痰液染色观察是否有细胞内细菌存在。

(2)肺部听诊是评价胸部物理治疗效果的客观方法之一 物理治疗前,注意湿啰音、干啰音及哮鸣音出现的部位及程度,治疗后的变化可评价疗效。胸部物理治疗前后,注意肺部听诊是很有必要的。

(3)观察患者呼吸功的改变也可评价胸部物理治疗的疗效 当痰液潴留时,往往引起患者呼吸困难及呼吸功增加,具体表现为动用呼吸辅助肌、胸腹部呼吸运动不同步或矛盾运动。治疗后如分泌物被有效清除,则上述表现可缓解。

(4)测定动脉血气有助于评价疗效 胸部物理治疗后,如动脉血气改善,则提示治疗有效。当然,由于动脉血气是心肺状况的综合反映,动脉血气无改善并不说明物理治疗无效。

\begin{center}\rule{0.5\linewidth}{\linethickness}\end{center}

参考文献

\protect\hyperlink{text00015.htmlux5cux23ch1-14-back}{{[}1{]}} .Dodek
P,Keenan S,Cook D,et al.Evidence-based clinical practice guideline
for the prevention of ventilator-associatied pneumonia.Ann Intern
Med,2004,141:305-313.

\protect\hyperlink{text00015.htmlux5cux23ch2-14-back}{{[}2{]}}
.American Thoracic Society Documents:Guidelines for the management of
adults with hospital-acquired,ventilator-associated,and
healthcare-associated pneumonia.Am J Respir Crit Care
Med,2005,171:388-416.

\protect\hyperlink{text00015.htmlux5cux23ch3-14-back}{{[}3{]}}
.Bacchetta MD,Girardi LN,Southard EJ,et al.Comparison of open
versus bedside percutaneous dilatational tracheostomy in the
cardiothoracic surgical patient:outcomes and financial analysis.Ann
Thorac Surg,2005,79:1879-1885.

\protect\hyperlink{text00015.htmlux5cux23ch4-14-back}{{[}4{]}} .Kluge
S,Meyer A,Baumann HJ,et al.Percutaneous tracheostomy is safe in
patients with severe thrombocytopenia.Chest,2004,126:547-551.

\protect\hyperlink{text00015.htmlux5cux23ch5-14-back}{{[}5{]}} .Nun
AB,Altman E,Best LA,et al.Emergency percutaneous tracheostomy in
trauma patients:an early experience.Ann Thorac Surg,2004,77:1045.

\protect\hyperlink{text00015.htmlux5cux23ch6-14-back}{{[}6{]}} .Nun
AB,Altman E,Best LA.Extended indications for percutaneous
tracheostomy.Ann Thorac Surg,2005,80:1276-1279.

\protect\hyperlink{text00015.htmlux5cux23ch7-14-back}{{[}7{]}} .Blot
F,Melot C.Indications,Timing,and techniques of tracheostomy in 152
French ICUs.Chest,2005,127:1347-1352.

\protect\hyperlink{text00015.htmlux5cux23ch8-14-back}{{[}8{]}}
.Griffiths J,Barber VS,Morgan L,et al.Systematic review and
meta-analysis of studies of the timing of tracheostomy in adult patients
undergoing artificial ventilation.BMJ,2005,330:1243.

\protect\hyperlink{text00015.htmlux5cux23ch9-14-back}{{[}9{]}}
.Gujadhur R,Helme BW,Sanni A,et al.Continuous subglottic suction is
effective for prevention of ventilator associated pneumonia.Interactive
CardioVascular and Thoracic Surgery,2005,4:110-115.

\protect\hyperlink{text00015.htmlux5cux23ch10-14-back}{{[}10{]}}
.Lacherade JC,Auburtin M,Cerf C,et al.Impact of humidification
systems on ventilator-associated pneumonia:A randomized multicenter
trial.Am J Respir Crit Care Med,2005,172:1276-1282.

\protect\hyperlink{text00015.htmlux5cux23ch11-14-back}{{[}11{]}}
.Lorente L,lecuona M,Jimenez A,et al.Ventilator-associated
pneumonia using a heated humidifier or a heat and moisture exchanger:A
randomized controlled trial.Crit Care,2006,10:R116.

\protect\hyperlink{text00015.htmlux5cux23ch12-14-back}{{[}12{]}}
.Michael Z,Rolando B.Tracheostomy in the critically ill
patient:who,when,and how?Clin Pul Med,2006,13:111-120.

\protect\hypertarget{text00016.html}{}{}


\chapter{免疫学检测}
\begin{framed}
\noindent\textbf{【知识体系】}
\begin{center}
\includegraphics[width=.6\textwidth]{./images/Image00151.jpg}
\end{center}
\noindent\textbf{【课前思考】}

如果怀疑得了某种传染病或癌症,该如何确诊?机体对某种传染病有无抵抗力,该如何检测?对淋巴细胞(T细胞、B细胞、NK细胞)是如何鉴别的?如何诊断某人是否对青霉素过敏?是否得了结核病?

\noindent\textbf{【本章重点】}

1.免疫检测的基本原理、特点;

2.各类免疫检测的基本原理。

\noindent\textbf{【教学目标】}

1.掌握血清学试验的一般规律、影响因素;

2.掌握抗原或抗体检测常用的检测方法(凝集试验、沉淀试验、免疫标记技术等);

3.熟悉免疫细胞的分离、鉴定方法;

4.针对不同检测对象,能自主设计检测方法。
\end{framed}

免疫学检测是对抗原、抗体、免疫细胞数量、种类及其分化功能等进行定性或定量检测。免疫学检测技术在医学生物学研究领域得到广泛应用,并在临床医学中用于免疫相关疾病的诊断、病情监测、疗效评价等,本章介绍免疫学常用检测技术的基本原理、方法及其应用。

\section{检测抗原和抗体的体外试验}

抗原和抗体体外试验是指通过抗原与相应抗体在体外发生的特异性结合反应(凝集、沉淀等)来观察、分析、鉴定。抗体主要存在于血清中,这种体外的抗原-抗体反应又称血清学反应(试验)

抗原-抗体反应的检测技术主要应用于如下方面:

1.用已知抗原检测未知抗体。如临床上检测患者血清中抗病原微生物抗体、抗HLA的抗体、血型抗体以及各种自身抗体,用于诊断相关疾病;检测正常人群中注射某种疫苗后的抗体产生水平,来制订合理的免疫程序。

2.用已知抗体检测未知抗原。如检测各种病原微生物及其大分子产物,用于病原微生物的鉴定、分型、抗体O血型、HLA分型等。

3.血液学及免疫细胞的检测。用单克隆抗体检测血液细胞包括正常的和病理性的,进行免疫细胞的分类、鉴别等;抗血小板抗体及各种凝血因子的免疫学测定。

4.定性或定量检测体内各种大分子物质(如各种血清蛋白、可溶性血型物质、多肽类激素、细胞因子及肿瘤标志物AFP、CEA、PSA等),用于相关疾病的诊断或辅助诊断。

5.应用于内分泌检测(如HCG、LH、FSH、T3、T4等)、免疫因子(C3、C4、淋巴因子等)、用已知抗体检测某些药物、激素和炎性介质等各种半抗原物质,用于监测患者血清中药物浓度或运动员体内违禁药品水平等。


\subsection{血清学试验的一般规律和特点}

(一)用已知测未知

只有一种材料是未知的。

(二)试验和抑制试验

被相应的抗原或抗体所抑制,可以验证反应的特异性。

(三)特异性与交叉性

如变形杆菌与立克次氏体之间有共同的抗原决定簇,故斑疹伤寒病人血清可凝集OX19变形杆菌。为避免交叉反应干扰免疫学诊断,常采用吸收反应制备单价特异性抗血清,其原理是:将某一多价特异性抗血清与共同抗原(或称交叉抗原)反应,然后去除所形成的抗原抗体复合物。用颗粒性抗原进行的吸收反应,称为凝集吸收反应。

(四)抗原-抗体的结合比例与“带现象”

若抗原-抗体的数量比例合适,抗体分子的两个Fab段分别与两个抗原决定簇结合,相互交叉形成体积大、数量多,肉眼可见的网格状复合体,基本不存在游离的抗原或抗体,即抗原-抗体反应的等价带,此时形成肉眼可见的反应物(沉淀物或凝集物)。

在抗原-抗体反应中,可能出现抗原或抗体过剩的情况,由于过剩一方的结合价不能被完全占据,多呈游离的小分子复合物形式,或所形成的复合物易解离,不能被肉眼察见------“带现象”(图\ref{fig10-1})。

抗体过剩------前带,抗原过剩------后带,在检测中,应注意对抗原和抗体的浓度、比例进行适度的调整。

\begin{figure}[!htbp]
 \centering
 \includegraphics[width=.6\textwidth]{./images/Image00152.jpg}
 \captionsetup{justification=centering}
 \caption{带现象示意图}
 \label{fig10-1}
  \end{figure} 

(五)特异性结合与反应可见两个阶段

第一阶段:抗原-抗体特异性结合。特点:反应快,几秒~几分钟内完成,无肉眼可见的反应。

第二阶段:反应可见阶段。特点:出现凝集、沉淀、细胞溶解等现象,时间几分钟~几天,受电解质、温度、pH等影响。

(六)可逆性

抗原与抗体为分子表面的非共价结合复合物,结合虽稳定但可逆;在一定条件下,可解离为游离的抗原、抗体,解离后的抗原和抗体仍保持原有的理化特性及生物学性状。


\subsection{抗原-抗体反应的影响因素}

(一)电解质

抗原-抗体有对应的极性基团,能相互吸附并由亲水性变为疏水性。电解质的存在使抗原-抗体复合物失去电荷而凝聚,出现可见反应,故免疫学试验中多用0.9\%氯化钠稀释抗原或抗体。

(二)酸碱度

抗原-抗体反应的最适pH是6~8。超出此范围可影响抗原、抗体的理化性状,出现假阳性或假阴性。

(三)温度

适当的温度可增加抗原与抗体分子碰撞的机会,加快二者结合速度,其最适温度为37℃。某些抗原-抗体反应有其独特的最适温度,如冷凝集素在4℃左右与红细胞结合最好,20℃以上反而解离。此外,适当震荡或搅拌也可促进抗原-抗体分子的接触,提高结合速度。

(四)抗原-抗体的性质

抗体的特异性和亲和力是决定抗原-抗体反应的关键因素。从免疫早期动物所获抗血清其亲和力一般较低,而后期所得抗血清一般亲和力较高;单克隆抗体亲和力较低,一般不适用于低灵敏度的沉淀反应和凝集反应。此外,抗原理化性质、抗原决定簇多寡和种类等均可影响抗原-抗体反应。

抗原-抗体的浓度、比例对抗原-抗体反应的影响最大,是决定性因素。

\section{抗原-抗体反应的基本类型}

根据抗原的性质、结合反应的现象、参与反应的成分等因素,可将基于抗原-抗体反应的检测方法分为凝集反应、沉淀反应、补体参与的反应、中和反应以及免疫标记技术等。


\subsection{凝集反应(agglutination)}

细菌、红细胞等颗粒性抗原与相应抗体结合后,在一定条件下出现肉眼可见的凝集物,此为凝集反应。

(一)直接凝集反应

细菌或红细胞与相应抗体直接反应,可出现细菌或红细胞凝集现象(图\ref{fig10-2})。

1.玻片法:定性

其方法为:①已知抗体与相应抗原在玻片上反应,用于抗原的定性检测(如ABO血型鉴定、细菌鉴定)。

\begin{figure}[!htbp]
 \centering
 \includegraphics{./images/Image00153.jpg}
 \captionsetup{justification=centering}
 \caption{直接凝集反应示意图}
 \label{fig10-2}
  \end{figure} 

2.试管法:定量

多用已知抗原测未知抗体的相对含量,如:诊断伤寒、副伤寒、布氏杆菌病。

方法:待检血清在试管内用0.9\%氯化钠倍比稀释,加入等量菌液,37℃,数小时观察结果(表\ref{tab10-1})。
 
\begin{table}[htbp]
    \centering
    \caption{试管倍比稀释法测定待检血清效价}
    \label{tab10-1}
\includegraphics[width=.8\textwidth]{./images/Image00154.jpg}
\end{table}

观察每个试管内抗原的凝集程度,凝集分五级:

(1)++++:很强,细菌全部凝集,管内液体澄清,可见管底有大片边缘不整的白色凝集物,轻摇时可见明显的颗粒、薄片或絮状。

(2)+++:强,细菌大部分凝集,液体较混浊,管底有边缘不整的白色凝集物,轻摇时也可见明显的颗粒、薄片或絮状。

(3)++:中等强度,细菌部分凝集,液体较混浊,管底有少量凝集物呈颗粒状。

(4)+:弱,细菌仅有少量凝集,液体混浊,管底凝集呈颗粒状,小不易观察。

(5)-:不凝集,液体混浊度、管底沉积物同对照管相似。

通常以出现明显凝集现象(++)的血清最高稀释度为该血清的抗体效价。

(二)间接凝集反应

该反应将可溶性抗原包被在与免疫无关的载体颗粒表面,再与相应抗体反应,出现颗粒物凝集现象(图\ref{fig10-3})。常用载体为人O型血红细胞、聚苯乙烯乳胶颗粒等。用途:检测血清中的自身抗体和抗微生物的抗体。

\begin{figure}[!htbp]
 \centering
 \includegraphics{./images/Image00155.jpg}
 \captionsetup{justification=centering}
 \caption{间接凝集反应示意图}
 \label{fig10-3}
  \end{figure} 

(三)间接凝集抑制试验

其原理是:将待测抗原(或抗体)与特异性抗体(或抗原)先行混合并作用一定时间,再加入相应致敏载体悬液;若待测抗原与抗体对应,即发生中和,随后加入的相应致敏载体颗粒不再被凝集,使本应出现的凝集现象被抑制,故得名(图\ref{fig10-4})。此试验可用于检测抗原或抗体(如早孕的检测),其灵敏度高于一般间接凝集试验;可用来检测可溶性抗原,如免疫妊娠诊断试验。

(1)诊断抗原:HCG致敏的乳胶颗粒;

(2)诊断血清:抗HCG的抗体;

(3)检测标本:尿液(是否含有HCG)。

\begin{figure}[!htbp]
 \centering
 \includegraphics{./images/Image00156.jpg}
 \captionsetup{justification=centering}
 \caption{间接凝集抑制反应示意图}
 \label{fig10-4}
  \end{figure} 

(四)反向间接凝集试验

反向间接凝集试验是用特异性抗体致敏载体,检测标本中的相应抗原的反应(图\ref{fig10-5}),可用于检测乙型肝炎病毒表面抗原、甲胎蛋白、新型隐球菌荚膜抗原等。

\begin{figure}[!htbp]
 \centering
 \includegraphics{./images/Image00157.jpg}
 \captionsetup{justification=centering}
 \caption{反向间接凝集试验示意图}
 \label{fig10-5}
  \end{figure} 

(五)协同凝集试验

协同凝集试验(COAG)的原理是:金黄色葡萄球菌细胞壁成分蛋白A(SPA)能与人和多种哺乳动物血清中的IgG分子的Fc片段结合,Fab就暴露,能与相应抗原结合,产生协同凝集反应(图\ref{fig10-6})。本试验通常可用于检测传染病患者的血液、脑脊液和其他分泌物中可能存在的微量可溶性抗原,目前已用于流行性脑脊髓膜炎(简称流脑)、伤寒、布氏菌病的早期诊断。

\begin{figure}[!htbp]
 \centering
 \includegraphics{./images/Image00158.jpg}
 \captionsetup{justification=centering}
 \caption{协同凝集试验示意图}
 \label{fig10-6}
  \end{figure} 


\subsection{沉淀反应}

沉淀反应(precipitation)是将可溶性抗原(沉淀原)与相应抗体(沉淀素)结合后,在一定条件下出现肉眼可见的沉淀,此为沉淀反应。该反应多用半固体琼脂凝胶作为介质进行琼脂扩散或免疫扩散,即可溶性抗原与抗体在凝胶中扩散,在比例合适处相遇即形成可见的白色沉淀。

沉淀原:内、外毒素,菌体裂解液、血清、蛋白质、多糖、类脂等,其体积小,与抗体相比反应面积大,故试验时需对抗原进行稀释,以避免沉淀原过剩出现后带现象,并以抗原稀释度作为沉淀试验的效价。

(一)液相沉淀试验------环状沉淀试验

已知抗血清+待检抗原→液面交界处,白色环状沉淀“+”,可用来鉴别血迹性质、测定媒介昆虫的嗜血性、鉴定某些细菌。

(二)琼脂扩散试验

用半固体琼脂凝胶作为介质进行琼脂扩散或免疫扩散,即可溶性抗原与抗体在凝胶中扩散,在比例合适处相遇即形成可见的白色沉淀。

1.双向免疫扩散

双向免疫扩散(double immunodiffusion)
是将抗原和抗体分别加入琼脂凝胶的小孔中,二者自由向四周扩散,在相遇处形成沉淀线。若反应体系中含两种以上抗原-抗体系统,则小孔间可出现两条以上沉淀线(图\ref{fig10-7})。特点:敏感性不高,所需时间较长。用于:

\begin{figure}[!htbp]
 \centering
 \includegraphics{./images/Image00159.jpg}
 \captionsetup{justification=centering}
 \caption{双向琼脂扩散试验示意图}
 \label{fig10-7}
  \end{figure} 

(1)定性检测可溶性抗原或抗体。

(2)对复杂的抗原成分或抗原、抗体的提取纯度进行分析鉴定。

(3)测定免疫血清的效价。

2.单向免疫扩散

单向免疫扩散(single
immunodiffusion)是将一定量已知抗体混于琼脂凝胶(45℃)中制成琼脂板,在适当位置打孔并加入抗原。抗原在扩散过程中与凝胶中的抗体相遇,形成以抗原孔为中心的沉淀环,环的直径与抗原含量呈正相关。取已知量抗原绘制标准曲线,可根据所形成沉淀环的直径,从标准曲线中查出待检标本的抗原含量(图\ref{fig10-8})。

\begin{figure}[!htbp]
 \centering
 \includegraphics{./images/Image00160.jpg}
 \captionsetup{justification=centering}
 \caption{单向琼脂扩散试验示意图}
 \label{fig10-8}
  \end{figure} 

3.对流免疫电泳

对流免疫电泳(CIE)又称免疫电渗电泳,双向琼脂扩散与电泳技术相结合。试验在装有pH8.6缓冲液的电泳槽中进行(图\ref{fig10-9})。

\begin{figure}[!htbp]
 \centering
 \includegraphics{./images/Image00161.jpg}
 \captionsetup{justification=centering}
 \caption{对流免疫电泳试验示意图}
 \label{fig10-9}
  \end{figure} 

(1)原理:抗原和抗体在电泳时受两种作用力的影响,一种是电场力,使抗原和抗体由“-”极向“+”极移动;另一种是电渗力,使抗原和抗体由“+”极向“-”极移动。

通常,抗原等电点偏低(pH 4~5),在碱性缓冲液(pH
8.6)中所带负电荷较多,受电场力较大,而其相对分子质量较小,所受电渗作用影响小,合力结果是电场力大于电渗力。因此,通电后,抗原由“-”极向“+”极移动。

抗体为球蛋白,等电点偏高(pH
6~7),所带负电荷较少,受电场力影响较小,而其相对分子质量较大,所受电渗作用影响大,合力结果电渗力大于电场力。因此,通电后,抗体由“+”极向“-”极移动。

两者相对而行,缩短了反应时间,提高了试验的敏感性。

(2)特点:操作简便,敏感性高,所需时间短。本试验可用来检测血清中的HBsAg和AFP等可溶性抗原。

(3)方法:将抗原和抗体分别加入琼脂板孔中,通电进行电泳[4mA/cm(宽)端电压:6V/cm
],电泳45~60min,水洗和洗色。

4.火箭电泳

火箭电泳又称电泳免疫扩散,单向琼脂扩散与电泳结合。本试验的敏感性与单向琼脂扩散相当,但所需时间短,故可用来测定标本中可溶性抗原的含量(图\ref{fig10-10})。

试验时,将适当浓度的已知抗体加入融化(45℃)的琼脂中,混匀后浇注于玻璃板,制成凝胶板,将抗原加入孔中,在盛有pH8.6缓冲液的电泳槽中电泳,电流强度3mA/cm(或电压10V/cm),电泳时间2~10h。电泳后在比例最适处形成锥形沉淀峰。

\begin{figure}[!htbp]
 \centering
 \includegraphics{./images/Image00162.jpg}
 \captionsetup{justification=centering}
 \caption{火箭免疫电泳试验示意图}
 \label{fig10-10}
  \end{figure} 

5.免疫电泳

免疫电泳(immunoelectrophoresis)
是将琼脂电泳与双向琼脂扩散结合的技术。待检标本在孔内先电泳,各种成分分开。之后挖槽,加入相应抗体,进行双向琼脂扩散。

根据沉淀弧的数量、位置、形状,并通过与已知标准抗原相比,可对样品中所含成分及其性质作出判断(图\ref{fig10-11})。

本试验样品用量小、特异性高、分辨力强,主要用于血清蛋白及抗体成分的分析研究,亦可用于抗原或抗体提取物的纯度鉴定。

\begin{figure}[!htbp]
 \centering
 \includegraphics[width=.6\textwidth]{./images/Image00163.jpg}
 \captionsetup{justification=centering}
 \caption{免疫电泳试验示意图}
 \label{fig10-11}
  \end{figure} 


\subsection{补体参与的反应}

1.溶菌反应:细菌与相应抗体结合,可激活补体,使细菌溶解,主要发生于霍乱弧菌等G\textsuperscript{+}
菌,可用于细菌鉴定。

2.溶血反应:红细胞与相应抗体结合,通过激活补体使红细胞溶解,可作为补体结合试验的指示系统。

3.补体结合反应:是一种在补体参与的条件下,以绵羊红细胞和溶血素作为指示系统来测定有无相应抗原或抗体的血清学试验(图\ref{fig10-12})。

\begin{figure}[!htbp]
 \centering
 \includegraphics[width=.6\textwidth]{./images/Image00164.jpg}
 \captionsetup{justification=centering}
 \caption{补体结合反应示意图}
 \label{fig10-12}
  \end{figure} 


\subsection{中和试验}

毒素、酶、激素和病毒等与相应抗体(中和抗体)结合,使之丧失生物学活性的现象称为中和反应。

1.病毒中和试验

病毒中和试验是病毒在活体内或细胞培养中被特异性抗体中和而失去感染性的一种试验。检查患病后或人工免疫后机体血清中相应中和抗体的增长情况,也可用来鉴定病毒。

2.毒素中和试验

外毒素与相应抗毒素结合后丧失其毒性,分体内和体外两种。

如:抗链球菌溶血素O试验。

乙型溶血性链球菌→溶血素(可溶解人、兔红细胞)→刺激机体产生抗毒素(抗体)。溶血素+抗体→毒性丧失,不溶血。

病人血清(未知)+溶血素O(经一定时间)+人红细胞→红细胞不溶解破坏------待检血清中有相应抗体,试验“+”。

本试验常用于临床风湿病的辅助诊断。


\subsection{免疫标记技术}

免疫标记技术(immunolabelling
technique)是用荧光素、酶、放射性核素或化学发光物质等标记抗体或抗原,进行抗原-抗体反应的检测。标记物与抗体或抗原连接后并不改变抗原-抗体的免疫特性,具有灵敏度高、快速、可定性、定量、定位等优点。

(一)免疫荧光法(immunofluorescence,IF)

免疫荧光法又称荧光抗体技术,用荧光素与抗体连接成荧光抗体,再与待检标本中抗原反应,置荧光显微镜下观察,抗原-抗体复合物散发荧光,借此对抗原进行定性或定位。

荧光素包括:异硫氰酸荧光素(FITC)(黄绿色荧光)、四乙基罗丹明(RB200)(橙色荧光)、四甲基异硫氰酸罗丹明(TMRITC)(橙红色荧光)。

1.直接荧光法:待检标本(固定在玻片上)+已知荧光抗体→洗去游离的荧光抗体→干燥后,荧光显微镜下观察(图\ref{fig10-13})。

用途:病毒感染的细胞、携带某种特异性抗原的细胞的检测。

优点:方法简便、特异性高。

缺点:敏感性低,检测多种抗原需制备多种相应的荧光抗体标记。

\begin{figure}[!htbp]
 \centering
 \includegraphics{./images/Image00165.jpg}
 \captionsetup{justification=centering}
 \caption{直接荧光法示意图}
 \label{fig10-13}
  \end{figure} 

2.间接法(又称荧光-抗体法)

间接法是用来检测标本中未知的抗原,或检测血清中未知抗体(图\ref{fig10-14})。

(1)检测抗原。未标记抗体+待检抗原(未知)→(冲洗)+荧光标记抗抗体→冲洗、干燥、荧光显微镜下观察。

(2)检测抗体。待检血清(未知抗体)+抗原标本(已知)→(冲洗)+荧光标记抗抗体→冲洗、干燥、荧光显微镜下观察。

优点:敏感性高,制备一种荧光标记抗抗体即可对多种抗体-抗体系统进行检测。

缺点:易出现非特异性荧光。

\begin{figure}[!htbp]
 \centering
 \includegraphics{./images/Image00166.jpg}
 \captionsetup{justification=centering}
 \caption{间接荧光法示意图}
 \label{fig10-14}
  \end{figure} 

3.补体法

补体法的作用原理与间接法相似,只是抗原-抗体作用后,加入新鲜豚鼠血清(补体),通过激活补体形成抗原-抗体-补体(C3b)复合物,再用荧光素标记的抗C3b抗体染色,使上述复合物发出荧光(图\ref{fig10-15})。

\begin{figure}[!htbp]
 \centering
 \includegraphics[width=.6\textwidth]{./images/Image00167.jpg}
 \captionsetup{justification=centering}
 \caption{补体法示意图}
 \label{fig10-15}
  \end{figure} 

(二)酶免疫测定(enzyme immunoassay,EIA)

将抗原-抗体反应的特异性与酶催化作用的高效性相结合,借助酶作用于底物的显色反应判定结果,用酶标测定仪做定性或定量分析。优点:敏感性高,特异性强,可定性、定量。

标记酶:

(1)辣根过氧化物酶(HRP)

底物:邻苯二胺(OPD)(橙色)、3,3'二氨基联苯胺(DAB)(黄褐色)。

(2)碱性磷酸酶

底物:对硝基苯磷酸盐(黄色)。

1.酶联免疫吸附试验(enzyme linked immunosorbent assay,ELISA)

酶联免疫吸附试验是利用抗原或抗体能非特异性吸附于聚苯乙烯等固相载体表面的特性,使抗原-抗体反应在固相载体表面进行的一种免疫酶技术。

(1)间接法

间接法是用已知抗原检测未知抗体的一种检测方法。用已知抗原包被固相,加入待检血清标本,再加酶标记的二抗,加底物观察显色反应(图\ref{fig10-16})。

\begin{figure}[!htbp]
 \centering
 \includegraphics[width=.6\textwidth]{./images/Image00168.jpg}
 \captionsetup{justification=centering}
 \caption{ELISA间接法示意图}
 \label{fig10-16}
  \end{figure} 

(2)双抗体夹心法

双抗体夹心法是用已知抗体检测未知抗原的一种检测方法。将已知抗体包被固相载体,加入的待检标本若含有相应抗原,即与固相表面的抗体结合,洗涤去除未结合成分,加入该抗原特异的酶标记抗体,洗去未结合的酶标记抗体,加底物后显色。若标本中无相应抗原,固相表面无抗原结合,加入的酶标记抗体不能结合于固相并可被洗涤去除,加入底物则无显色反应(图\ref{fig10-17})。

\begin{figure}[!htbp]
 \centering
 \includegraphics[width=.5\textwidth]{./images/Image00169.jpg}
 \captionsetup{justification=centering}
 \caption{双抗体夹心法示意图}
 \label{fig10-17}
  \end{figure} 

(3)酶联免疫斑点试验

酶联免疫斑点试验(ELISPOT)有两种方法:

①用已知抗原检测分泌性特异性抗体的B细胞:用已知抗原包被固相载体,B细胞分泌的抗体与之结合,加入酶标记的抗Ig抗体,通过底物显色反应可检测B细胞分泌的特异性抗体。

②用抗细胞因子抗体检测细胞分泌的细胞因子:用抗细胞因子抗体包被固相载体,加入不同来源的细胞,细胞所分泌的细胞因子与包被抗体结合,再加入酶标记的抗细胞因子抗体,通过显色反应测定结合在固相载体上的细胞因子(定性或半定量),并可在光镜下观察分泌细胞因子的细胞(图\ref{fig10-18})。

\begin{figure}[!htbp]
 \centering
 \includegraphics[width=.5\textwidth]{./images/Image00170.jpg}
 \captionsetup{justification=centering}
 \caption{酶联免疫斑点试验示意图}
 \label{fig10-18}
  \end{figure} 

(4)生物素-亲合素法

生物素(biotin)又称维生素H,是从卵黄和肝中提取的一种小分子物质(分子量244.31kD);亲合素(avidin)又称卵白素,是从卵白中提取的一种糖蛋白(分子量68kD)。每个亲合素分子有生物素结合的4个位点,二者可牢固结合成不可逆的复合物。生物素-亲合素的应用大致有三种方法(图\ref{fig10-19})。

①标记亲合素-生物素法(labelled avidin-biotin
method,LAB法):将亲合素与标记物(HRP)结合,一个亲合素可结合多个HRP;将生物素与抗体(一抗与二抗)结合,一个抗体分子可连接多个生物素分子,抗体的活性不受影响。细胞的抗原(或通过一抗)先与生物素化的抗体结合,继而将标记亲合素结合在抗体的生物素上,如此多层放大提高了检测抗原的敏感性。

\begin{figure}[!htbp]
 \centering
 \includegraphics[width=.6\textwidth]{./images/Image00171.jpg}
 \includegraphics[width=.6\textwidth]{./images/Image00172.jpg}
 \includegraphics[width=.6\textwidth]{./images/Image00173.jpg}
 \captionsetup{justification=centering}
 \caption{生物素-亲合素法示意图}
 \label{fig10-19}
  \end{figure} 

②桥连亲合素-生物素法(bridged avidin-biotin
method,BAB法):先使抗原与生物素化的抗体结合,再以游离亲合素将生物素化的抗体与酶标生物素搭桥连接,也达到多层放大效果。

③亲合素-生物素-过氧化物酶复合物法(avidin-biotin-peroxidase complex
method,ABC法):此法是前两种方法的改进,即先按一定比例将亲合素与酶标生物素结合在一起,形成亲合素-生物素-过氧化物酶复合物(ABC复合物),标本中的抗原先后与一抗、生物素化二抗、ABC复合物结合,最终形成晶格样结构的复合体,其中聚合了大量酶分子,从而大大提高了检测抗原的灵敏度。

2.免疫组化技术

免疫组化技术(immunohitochemistry
techenique)是应用免疫学基本原理------抗原抗体反应,即抗原与抗体特异性结合的原理,通过化学反应使标记抗体的显色剂
(荧光素、酶、金属离子、同位素)
显色来确定组织细胞内抗原(多肽和蛋白质),对其进行定位、定性及定量的研究,称为免疫组织化学技术(immunohistochemistry)或免疫细胞化学技术(immunocytochemistry)。

众所周知,抗体与抗原之间的结合具有高度的特异性。免疫组化正是利用这一特性,即先将组织或细胞中的某些化学物质提取出来,以其作为抗原或半抗原去免疫小鼠等实验动物,制备特异性抗体,再用这种抗体(第一抗体)作为抗原去免疫动物制备第二抗体,并用某种酶(常用辣根过氧化物酶)或生物素等处理后再与前述抗原成分结合,将抗原放大,由于抗体与抗原结合后形成的免疫复合物是无色的,因此,还必须借助于组织化学方法将抗原抗体反应部位显示出来(常用显色剂DAB显示为棕黄色颗粒)。通过抗原抗体反应及呈色反应,显示细胞或组织中的化学成分,在显微镜下可清晰看见细胞内发生的抗原抗体反应产物,从而能够在细胞或组织原位确定某些化学成分的分布、含量。组织或细胞中凡是能作为抗原或半抗原的物质,如蛋白质、多肽、氨基酸、多糖、磷脂、受体、酶、激素、核酸及病原体等都可用相应的特异性抗体进行检测。

免疫组织化学技术按照标记物的种类可分为免疫荧光法、免疫酶法、免疫铁蛋白法、免疫金法及放射免疫自影法等(图\ref{fig10-20})。

\begin{figure}[!htbp]
 \centering
 \includegraphics{./images/Image00174.jpg}
 \captionsetup{justification=centering}
 \caption{双标记免疫组化染色技术示意图}
 \label{fig10-20}
  \end{figure} 

(三)放射免疫测定

放射免疫测定(radioimmunoassay,RIA)是将放射性同位素分析的高度灵敏性与抗原-抗体反应的高度特异性有效结合而建立的一种检测技术。

同位素:\textsuperscript{131} I、\textsuperscript{125}
I、\textsuperscript{3} H、\textsuperscript{14} C、\textsuperscript{32}
P等。

特点:灵敏度高,能测出ng/ml(ug/L),甚至pg/ml(ng/L)水平的微量物质,试验快速、准确,可规格化,重复性好。

缺点:放射性同位素有一定的危害性,且易污染环境,因此其应用受到一定限制。

方法:(1)液相放射免疫测定、(2)固相放射免疫测定。

(四)化学发光免疫分析

将发光物质(如吖啶酯、鲁米诺等)标记抗原或抗体,发光物质在反应剂(如过氧化阴离子)激发下生成激发态中间体,当回复至稳定的基态时发射光子,通过自动发光分析仪测定光子产量,可反映待检样品中抗体或抗原含量(图\ref{fig10-21})。

\begin{figure}[!htbp]
 \centering
 \includegraphics{./images/Image00175.jpg}
 \captionsetup{justification=centering}
 \caption{化学发光免疫分析}
 \label{fig10-21}
  \end{figure} 

(五)免疫印迹法

免疫印迹法(immunoblot)
又称Western印迹法,其结合凝胶电泳与固相免疫技术,将借助电泳所区分的蛋白质转移至固相载体,再应用酶免疫、放射免疫等技术进行检测。本方法包括5个步骤:

(1)固定:蛋白质进行聚丙烯酰胺凝胶电泳(P抗原E)并从胶上转移到硝酸纤维素膜上。

(2)封闭:保持膜上没有特殊抗体结合的场所,使场所处于饱和状态,用以保护特异性抗体结合到膜上,并与蛋白质反应。

(3)初级抗体(第一抗体)是特异性的。

(4)第二抗体或配体试剂对于初级抗体是特异性结合并作为指示物。

(5)适当保温后的酶标记蛋白质区带,产生可见的、不溶解状态的颜色反应。

该法能对分子大小不同的蛋白质进行分离并确定其分子量,常用于检测多种病毒抗体或抗原。

(六)免疫金技术

免疫金技术是一种以胶体金作为标记物的免疫标记技术。胶体金是由金盐被还原成原金后形成的金颗粒悬液,颗粒大小多在1~100nm。

胶体金的光散射性与溶胶颗粒的大小密切相关,一旦颗粒大小发生变化,光散射也随之发生变异,产生肉眼可见的显著的颜色变化。小:2~5nm,橙黄色。中:10~20nm,酒红色。大:30~80nm,紫红色。

(七)免疫比浊

免疫比浊(immunonephelometry)
是在一定量抗体中分别加入递增量的抗原,经一定时间形成免疫复合物,液体混浊。用浊度计测量反应体系的浊度,可绘制标准曲线并依据浊度推算样品中抗原含量。

\section{检测淋巴细胞及其功能的体外试验}


\subsection{免疫细胞及其亚类分离、鉴定和检测}

(一)外周血单个核细胞的分离

体外检测淋巴细胞,首先需制备外周血单个核细胞(PBMC),常用的方法是葡聚糖-泛影葡胺(又称淋巴细胞分离液)密度梯度离心法。红细胞和多形核白细胞的比重(约1.092)大于单个核细胞(约1.075),将抗凝血叠加于比重为1.077的分离液液面上,可通过低速离心将不同比重的细胞分层:红细胞沉于管底;多形核白细胞密集布于红细胞层与分离液之间;血小板悬浮于血浆中;单个核细胞则密集于血浆层与分离液界面。该法分离淋巴细胞的纯度可达95\%。若需进一步纯化淋巴细胞,可将单个核细胞铺于培养皿上,由于单核细胞易与玻璃黏附而滞留于平皿表面,未吸附的细胞即主要是淋巴细胞(图\ref{fig10-22})。

(二)淋巴细胞亚群的分离

淋巴细胞为不均一的群体,可借助其表面标志及功能差异而分为不同的群和亚群。

1.尼龙棉分离法

将淋巴细胞悬液通过尼龙棉柱,B细胞易与尼龙棉黏附而滞留于柱上,T细胞则不黏附,借此可分离T细胞与B细胞。

  \begin{figure}[!htbp]
    \centering
    \begin{minipage}[b]{0.45\textwidth} 
        \centering
        \includegraphics[height=.2\textheight]{./images/Image00176.jpg}
        \captionsetup{justification=centering}
        \caption{密度梯度离心法分离单核细胞}
        \label{fig10-22}
    \end{minipage}
%	\end{figure} 
	%\FloatBarrier
%\begin{figure}[!htbp]
%    \centering
\hspace{0.04\textwidth}%
\begin{minipage}[b]{0.45\textwidth} 
    \centering
    \includegraphics[height=.2\textheight]{./images/Image00177.jpg}
    \captionsetup{justification=centering}
    \caption{E花结分离法}
    \label{fig10-23}
\end{minipage}
\end{figure} 

2.E花结分离法

人成熟T细胞表面具有绵羊红细胞(SRBC)受体,能结合SRBC而形成花结(E花结试验),经密度梯度离心,花结形成细胞因比重增大而沉于管底,与其他细胞分离;用低渗法裂解花结中的SRBC,即获得纯化的T细胞(图\ref{fig10-23})。

3.洗淘法

将已知抗特定细胞表面标志的抗体包被聚苯乙烯培养板,加入淋巴细胞悬液,表达相应表面标志的细胞即结合于培养板表面,与悬液中的其他细胞分离。

4.流式细胞术(flow cytometry,FCM)

借助荧光活化细胞分类仪(fluorescence-activated cell
sorter,FACS)对细胞快速鉴定和分类,并进行多参数定量测定和综合分析的技术。样品与经多种荧光素标记的抗体反应,因荧光素发射光谱的波长不同,信号能同时被接收,故能同时分析细胞表面多个膜分子表达及其水平。该法可检测各类免疫细胞、细胞亚类及其比率。此外,借助光电效应,微滴通过电场时出现不同偏向,可分类收集所需细胞。

5.磁分离技术

将特异性抗体与磁性微粒交联,称为免疫磁珠(immune m抗原netic
bead,IMB)。IMB可与表达相应膜抗原的细胞结合,应用强磁场分离IMB及其所吸附的细胞,从而对特定的细胞进行分选,此为直接分离法;亦可用二抗包被磁性微珠,与任何已结合鼠源性一抗的细胞进行反应,从而分离细胞,此为间接分离法。

(三)T细胞及其亚群的鉴定和检测

1.E玫瑰花环形成试验

本试验可用于人T细胞的鉴定和检测。绵羊红细胞和人外周血淋巴细胞在4℃孕育1~2小时,检测花环样细胞集团数量。正常人外周血中E玫瑰花环形成细胞占淋巴细胞总数的60\%~80\%。

2.T细胞单克隆抗体对T细胞及其亚群的鉴定和检测

所用的抗体主要有:CD\textsubscript{3} McAb、CD\textsubscript{4}
McAb和CD\textsubscript{8} McAb。

方法:免疫荧光间接法。

外周血淋巴细胞,分别用小鼠抗人CD\textsubscript{3} 、CD\textsubscript{4}
、CD\textsubscript{8} McAb
(第一抗体)加荧光素标记的兔抗小鼠IgG抗体(第二抗体),在荧光显微镜下观察结合有荧光素标记抗体的细胞,亦可应用FCM自动计数荧光阳性细胞百分率(图\ref{fig10-24})。

\begin{figure}[!htbp]
 \centering
 \includegraphics[width=.6\textwidth]{./images/Image00178.jpg}
 \captionsetup{justification=centering}
 \caption{免疫荧光间接法鉴定T细胞}
 \label{fig10-24}
  \end{figure} 

结果:

(1)被CD\textsubscript{3}
McAb着染荧光的细胞是总T细胞,包括:Th、Th1、Th2、Tc、Ts细胞。

(2)被CD\textsubscript{4} McAb着染荧光的细胞是Th、Th1、Th2细胞。

(3)被CD\textsubscript{8} McAb着染荧光的细胞是Tc、Ts细胞。

计数:

100~200个淋巴细胞,计算出染阳性细胞百分率:

CD\textsuperscript{+} \textsubscript{3}
T细胞占65\%~80\%。CD\textsubscript{4}
T细胞占50\%~60\%。CD\textsuperscript{+} \textsubscript{8}
T细胞占20\%~30\%。CD\textsuperscript{+} \textsubscript{4}
T细胞与CD8T细胞比值约为2∶1。

(四)B细胞鉴定和检测

\begin{figure}[!htbp]
 \centering
 \includegraphics[width=.5\textwidth]{./images/Image00179.jpg}
 \captionsetup{justification=centering}
 \caption{免疫荧光直接法检测B细胞}
 \label{fig10-25}
  \end{figure} 

mIgM/D是B细胞表面特有的标志,通过对该种标志的检测,可对B细胞进行鉴定和检测。

方法:免疫荧光直接法。

荧光素标记的兔抗人IgM/D抗体加外周血淋巴细胞,直接免疫荧光染色、观察(图\ref{fig10-25})。着染荧光的细胞为B细胞,占淋巴细胞总数的8\%~12\%。


\subsection{淋巴细胞功能测定}

(一)T细胞功能检测

1.淋巴细胞转化试验

原理:T细胞在特异性抗原或有丝分裂原的作用下转变为淋巴母细胞(体积更大、代谢旺盛),根据其转化程度和转化率,测定机体细胞免疫功能状态。

刺激物分为两类:

(1)非特异性刺激物:如各种丝裂原(PHA、Con
A、LPS等),抗CD\textsubscript{2} 、CD\textsubscript{3}
等细胞表面标志的抗体以及某些细胞因子等;正常人T细胞转化率约为70\%。

(2)特异性刺激物:主要是特异性可溶性抗原、细胞表面抗原、结核菌素(OT或PPD)。正常人T细胞转化率为5\%~30\%。

不同刺激物可刺激不同淋巴细胞分化增殖,从而反映不同淋巴细胞亚群的功能状态。

测定方法:可采用放射性核素掺入法、比色法、荧光素标记法和形态学等方法。

(1)3H-TdR掺入法

在T细胞增殖过程中,胞内DNA、RNA合成增加,应用氚标记的胸腺嘧啶核苷(3H-TdR)可掺入细胞新合成的DNA中,所掺入放射性核素的量与细胞增殖水平成正比。借助液体闪烁仪测定样品的放射活性,可反映细胞的增殖状况。该法灵敏可靠,应用广泛,但需特殊仪器,易发生放射性污染。

(2)MTT法

MTT是一种噻唑盐,化学名3-(4,5-二甲基-2-噻唑)-2,5-二苯基溴化四唑,其掺入细胞后可作为胞内线粒体琥珀酸脱氢酶的底物,形成褐色甲臜颗粒并沉积于胞内或细胞周围,甲臢生成量与细胞增殖水平成正相关。甲臢可被盐酸异丙醇或二甲基亚砜完全溶解,借助酶标测定仪检测细胞培养物OD值,可反映细胞增殖水平。该法灵敏度不及3H-TdR掺入法,但操作简便,且无放射性污染。

(3)形态学计数法

淋巴细胞受丝裂原刺激后,转化为淋巴母细胞,其形态和结构发生明显改变,通过染色镜检,可计算出淋巴细胞转化率。

2.淋巴细胞参与的细胞毒性试验(LMC-T)

CTL、NK细胞可直接杀伤不同靶细胞(如肿瘤细胞、移植供体细胞等)。通过检测杀伤活性可用于肿瘤免疫、移植排斥反应、病毒感染等方面研究。

(1)\textsuperscript{51} Cr释放法

用Na\textsuperscript{51} \textsubscript{2} CrO\textsubscript{4}
标记靶细胞,被效应细胞杀伤的靶细胞释放\textsuperscript{51}
Cr,应用γ计数仪测定所释出的\textsuperscript{51}
Cr放射活性,可反映效应细胞的杀伤活性(图\ref{fig10-26})。

\begin{figure}[!htbp]
 \centering
 \includegraphics[width=.6\textwidth]{./images/Image00180.jpg}
 \captionsetup{justification=centering}
 \caption{Cr释放法细胞毒试验}
 \label{fig10-26}
  \end{figure} 

(2)乳酸脱氢酶(LDH)释放法

效应细胞-靶细胞进行反应并离心,借助比色法测定靶细胞膜受损后从胞内所释放出的乳酸脱氢酶活性,其水平反映效应细胞的杀伤活性。

(3)细胞凋亡检查法

效应细胞介导靶细胞凋亡时,内源性核酸水解酶将靶细胞DNA在核小体单位之间被切断,产生180~200bp(核小体单位长度)及其倍数的寡核苷酸片段,在琼脂糖电泳中呈现阶梯状DNA区带图谱,借此可反映细胞凋亡。

如需测定凋亡细胞数目及细胞类型,可在细胞培养物中加入末端脱氧核苷酸转移酶(terminal
deoxyribonucleotidyl
transferase,TdT)和生物素标记的核苷酸,TdT能在游离的DNA
3'端缺口连接标记的核苷酸,利用亲合素-生物素-酶放大系统,在DNA断裂处显色,从而指示凋亡细胞。该法所用标记核苷酸多为dUTP,故称TUNEL法(TdT
dependent dUTP-biotin nick end labelling)。

3.分泌功能测定

检测免疫细胞所分泌细胞因子和抗体水平,可反映机体免疫功能状态。

(1)细胞因子分泌细胞的测定

细胞因子分泌细胞的测定常采用反向溶血空斑试验(RHPA)和酶联免疫斑点试验(ELISPOT)。RHPA检测原理是:将分泌细胞因子的待测细胞置于经SPA包被的单层SRBC中,抗细胞因子抗体被SPA固定在SRBC表面,并与待测细胞所分泌细胞因子结合。在补体存在时,细胞因子及其抗体形成的复合物可激活补体,溶解附近的红细胞形成溶血空斑。空斑大小与细胞分泌细胞因子的量成正比。

(2)抗体形成细胞测定

抗体形成细胞测定常用溶血空斑试验和定量溶血分光光度测定法。

溶血空斑试验即测定针对SRBC表面已知抗原的抗体形成细胞数目。其原理是:抗体形成细胞分泌的Ig与SRBC表面抗原结合,在补体参与下出现溶血反应。每一空斑中央含一个抗体形成细胞,空斑数目即为抗体形成细胞数(图\ref{fig10-27})。亦可采用RHPA和ELISPOT法检测抗体分泌细胞。

定量溶血分光光度测定法原理:根据溶血空斑试验原理衍化而来。将绵羊红细胞免疫小鼠后获得的脾细胞(含抗体形成细胞)与绵羊红细胞(SRBC)及豚鼠新鲜血清(补体)按一定比例混合,37℃水浴1小时后,SRBC溶解,释放血红蛋白,离心后上清液中的血红蛋白可用分光光度计定量测定。所获上清液吸光值与抗体形成细胞(浆细胞)分泌的抗体量成正比。

\begin{figure}[!htbp]
 \centering
 \includegraphics[width=.6\textwidth]{./images/Image00181.jpg}
 \captionsetup{justification=centering}
 \caption{溶血空斑试验}
 \label{fig10-27}
  \end{figure} 

(1)直接溶血空斑试验法:检测分泌IgM的抗体形成细胞(图\ref{fig10-28})。

\begin{figure}[!htbp]
 \centering
 \includegraphics[width=.6\textwidth]{./images/Image00182.jpg}
 \captionsetup{justification=centering}
 \caption{直接溶血空斑试验作用机制示意图}
 \label{fig10-28}
  \end{figure} 

(2)间接溶血空斑试验法:检测分泌IgG或其他类别Ig的抗体形成细胞。

方法:前两步与直接法相同,第三步需加入抗IgG或其他抗抗体,再加豚鼠新鲜补体进行观察(图\ref{fig10-29})。

\begin{figure}[!htbp]
 \centering
 \includegraphics[width=.6\textwidth]{./images/Image00183.jpg}
 \captionsetup{justification=centering}
 \caption{间接溶血空斑试验作用机制示意图}
 \label{fig10-29}
  \end{figure} 

\section{检测体液和细胞免疫功能的体内试验}

皮肤试验是测定机体体液和细胞免疫状态的一种体内试验,用于过敏性疾病、传染病、免疫缺陷性疾病和肿瘤等的诊断、防治、疗效和预后的判定。


\subsection{检测体液免疫的皮肤试验}

(一)速发型超敏反应皮肤试验

注射青霉素、普鲁卡因、抗毒素血清等均需进行皮肤试验,以判定体内特异性IgE产生情况和机体的致敏状态。

方法:青霉素100U/mL,抗毒素血清1∶1000,取0.1mL皮内注射,20分钟内观察结果。

结果:皮肤红晕水肿,直径大于1cm,为“+”。

(二)毒素皮肤试验

1.狄克试验(又称红疹毒素皮肤试验)

红疹毒素是A族链球菌产生的一种外毒素,是引起猩红热皮疹的主要物质。

试验目的:测定机体对猩红热是否易感,检测体内是否具有红疹毒素抗体。

方法:0.1mL红疹毒素注射于前臂皮内,6~24h后观察。

(1)皮肤红斑直径大于1cm,是“+”,表明受试者体内没有相应抗毒素,对猩红热易感。

(2)局部皮肤不出现红疹------说明注入的红疹毒素已被相应的抗毒素中和,机体对猩红热有抵抗力,不易感。

2.锡克试验------检测机体对白喉免疫力的皮肤试验

方法和结果判定与狄克试验大致相同。注射白喉毒素后24~48h。

(1)局部皮肤红肿“+”------表明受试者体内没有相应抗毒素,对白喉无免疫力。

(2)局部皮肤不红肿------表明白喉毒素被相应抗体中和,机体对白喉有一定的免疫力。

鉴于有人对白喉毒素有超敏反应,试验时应取另一份白喉毒素 80℃ 5min
破坏毒性,注射于受试者另一前臂皮内作为对照。


\subsection{检测细胞免疫功能的皮肤试验}

检测细胞免疫功能的皮肤试验是根据迟发型超敏反应(即细胞免疫反应)的发生机制建立的。抗原注入皮内,经48~72小时观察结果。

(1)局部皮肤红肿、硬结、直径大于0.5cm,为“+”,表明细胞免疫功能正常。

(2)反应微弱或皮试阴性,表明细胞免疫功能低下。

用途:(1)某些传染病和免疫缺陷病的诊断,(2)观测肿瘤治疗的效果和判断其预后。

常用生物性抗原有:结核菌素(OT)、结核菌纯蛋白衍生物(PPD)、念珠菌素、链激酶-链道酶(SK-SD)和植物性血凝素(PHA)等。

\noindent\textbf{【理解与思考】}

1.从血清学试验的一般规律看抗原、抗体检测的实质是什么?你能科普性地描述检测方法吗?

2.体内检测试验的实质是什么?医学上还有哪些体内试验?

3.请将免疫检测与前面几章内容联系起来,如果要检测诸如抗原、抗体、MHC、补体、细胞因子等,该如何检测?为什么?

\noindent\textbf{【课外拓展】}

1.其他免疫检测方法还有哪些?

2.如何检测细胞因子的种类及其活性?

\noindent\textbf{【课程实验与研究】}

1.设计一个用免疫方法检测某种细菌的实验。

2.设计一种鉴别不同淋巴细胞的实验方法。

3.临床上是如何检测肿瘤抗原的?请设计几种不同的方法检测甲胎蛋白,并请比较其敏感性。

4.《Science》最近报道称发现一类新T细胞。请设计一种方案,鉴别出它不是已经发现的免疫细胞中的一种。

5.《PNAS》报道称“发现保护动物抵抗禽流感新蛋白”,请就它的结构、分子量、生物学活性提出检测方案。

6.法国马赛Mediterranean Aix-Marseille 大学的病毒学家Xavier de
Lamballerie在2009年年末表示:开展血清抗体测试研究,揭示甲型H1N1流感真实感染数据。请你为研究设计一个方案。

\noindent\textbf{【课程研讨】}

1.如果要确诊某种病原微生物或是否得了某种传染病,请设计多种检测方法。并请比较你们小组所设计不同方法的优劣。

2.要检测食品中是否有三聚氰胺,能用免疫检测的方法吗?如可行,请设计检测方法;如不行,说明理由。

3.免疫学检测除了用于医学、生物学,你认为还能用在哪些领域?说明机理。

4.请就免疫检测的最新进展写一篇综述。

\noindent\textbf{【课后思考】}

1.凝集试验、免疫标记技术、酶联免疫吸附试验(ELISA)的概念。

2.试述抗原或抗体检测的应用。

\noindent\textbf{【课外阅读】}

\begin{center}
 \textbf{\Large 严重急性呼吸道综合征的实验室检查}
 \end{center}

SARS的诊断主要依据起病急、高热、咳嗽及X线胸片显示肺部浸润性改变等临床表现,结合流行病学史以及必要的实验室检查。目前,较为特异的实验室诊断技术有以下几种:

1.分子学检测

许多病毒感染性疾病早期病毒分泌最多,但是在SARS的发病初期,SARS-CoV含量非常少,需要用非常敏感的手段才能检测到低水平的病毒核酸,主要用反转录-聚合酶链反应(reverse
transcrip-tion polymerase chain
reaction,RT-PCR)或实时聚合酶链反应(real-time
PCR)检测可疑和发病患者的呼吸道分泌物、血液、尿液或粪便等人体标本中的病毒。鼻咽部分泌物为最适用标本,多部位标本联合检测可提高阳性率。Real-time
PCR灵敏度极高,可以检测到单个拷贝的病毒基因。但是由于受引物特异性、检测样本采集时间、种类及其处理方法、病毒核酸在样本中的含量以及样本中的酶抑制物等因素的影响,仍有假阴性结果出现。因此,PCR检测结果阴性,并不能确定该患者没有感染SARS-CoV。

2.血清学检测

已建立的检测病人血清抗体的方法有免疫荧光( immunofluorescence
assay,IFA)和酶联免疫吸附试验(enzyme linked immunoabsorbent
assay,ELISA)。血清抗体的研究表明,SARS患者发病后,最早的IgM抗体要在7天左右出现,10天时达到高峰,15天左右下降;抗体IgG
10天后产生,20天左右达到高峰。IFA检测可以检测出SARS-CoV感染10天后患者血清中产生的IgM。ELISA法要在SARS患者出现临床症状21天左右才可检测到稳定的IgG。SARS感染血清学诊断双份血清标本最可靠,故应尽可能采取进展期的标本。IFA检测时应将双份血清标本置于同一张玻片,ELISA检测时应将双份血清标本置于同一块酶免疫反应板内,这样检测的滴度才有可比性。通过这两种不同的血清抗体快速诊断试剂检测,可以鉴别患者是新感染还是曾经有过冠状病毒感染,病人是否产生相应的抗体。但是一般患者在患病初期,血液里虽有病毒,但抗体可能还没有形成,检测结果呈阴性;有时SARS病毒处于潜伏期,尚未出现症状或症状不典型,此时检测结果也是阴性。所以,这两种方法不适合早期诊断。

3.病毒检测

病毒的分离鉴定是确立病原学诊断的“金标准”。同其他冠状病毒不同,SARS-CoV很容易在体外33~37℃培养。利用VERO
E6或VERO细胞(从成年长尾猿属的非洲绿猴肾中提取的一种细胞系)来培养或扩增SARS患者的呼吸道分泌物、尿液粪便和血液样品,分离得到的病毒可以进一步验证是否为SARS-CoV。细胞感染24小时即可出现病变。培养细胞进行电镜观察可以确定冠状病毒,但不能肯定是SARS-CoV,最后诊断仍需要通过SARS-CoVRNA的PCR检测或全基因组测序。因此,病毒分离不能快速检测,而且细胞培养阴性也不能排除SARS。因为SARS-CoV为高度传染性,只允许在生物安全水平3级或以上的实验室进行,加上体外细胞培养分离十分复杂和繁琐,所以病毒检测对一般的临床实验室来说并不是一种常规的检测方法。

\begin{center}
 \textbf{\Large 中国体外诊断行业概况}
 \end{center}

体外诊断产业,就是指在人体之外通过对人体的血液等组织及分泌物进行检测获取临床诊断信息的产品和服务,在国际上统称IVD(In-Vitro
Diagnostics
)产业。IVD产业与检验医学构成了既相互区别又相互紧密联系的有机整体,体外诊断产业是检验医学的“工具”和“兵器”,同时检验医学是体外诊断产业的“用户”和“市场”,两者的共同目的是实施体外诊断。有专家指出,临床诊断信息的80\%左右来自体外诊断,而其费用占医疗费用不到20\%。体外诊断已经成为人类疾病预防、诊断、治疗日益重要的组成部分,是保障人类健康与构建和谐社会日益重要的组成部分。

我国体外诊断产业的发展开始于20世纪80年代,经历20多年的发展,从无到有,从弱到强,现正处于快速增长阶段。从市场角度看,我国是人均消费最低的国家之一,中国以世界1/5的人口,占据全世界IVD产业2\%左右的市场份额,2007年全国市场规模在100亿元人民币左右。但是,中国同时是市场增长速度最快的国家,年均增长率在15\%~20\%。我国IVD用户主要包括1800多家医院、300多家血站,还有日新月异的体检中心,正在兴起的临床检验独立实验室。如果我国城镇居民的医疗水平接近美国全国居民的医疗水平,则我国IVD
产业的市场规模将扩大20倍。从制造商或供应商角度看,在我国主要有三类制造商:一类是世界著名的跨国公司,包括IVD产业世界十强,如GE(美国通用)、西门子、罗氏、强生、倍克曼、德林等公司;一类是民族企业中的知名企业,如在香港上市的中生北控生物科技股份有限公司,在国内上市的上海科化、广州达安,在纽约证券所上市的深圳迈瑞等;一类是新兴或中小型的其他民族企业,由于恶性竞争的存在,行业总体赢利水平目前降至10%-20%,企业数量每天都在变化,但总数在1000家左右。从产品上看,目前在临床应用比较广泛、市场广阔的项目上(如免疫试剂中的肝炎、性病和孕检系列,临床生化中的酶类、脂类、肝功、血糖、尿检等系列),国内主要生产厂家的技术水平已基本达到国际同期水平;基因检测中的PCR技术系列已经基本达到国际先进水平,基因芯片、癌症系列正在开始迅速追赶国际水平。从市场环境角度看,法规从无到有,正在借鉴发达国家的成功经验和结合本国国情的原则指导下完善,而用户方面普遍对价格敏感,消费能力不同地区发展不平衡,城乡差别明显。

在2008年新医改政策及国家4万亿放量刺激内需的环境变化下,体外诊断行业面临前所未有的巨大机遇。不仅国内的体外诊断试剂和仪器生产企业,外国资本亦表现出了浓厚兴趣。典型例子是,2008年11月初北京科美东雅生物技术有限公司完成了第二轮高达1650万美元的私募融资,这为其拓展以化学发光(CLIA)体外诊断产品为核心的市场营销网络,进一步确立其IVD品牌实力提供了强大的资金实力。


\chapter{急性肾衰竭}

\section{前沿学术综述}

急性肾衰竭(acute renal
failure,ARF)是严重威胁重症患者生命的常见疾病。流行病学调查显示,重症医学科中急性肾衰竭的患病率高达31%~78%。对需要肾脏替代治疗的重症患者的研究也显示,在疾病严重程度类似的情况下,伴有急性肾衰竭患者的死亡风险增高4倍。急性肾衰竭成为影响和决定重症患者预后的关键性因素之一。加强重症医学科中急性肾衰竭的早期诊断、积极防治、逆转急性肾衰竭的发生发展,对改善重症患者的预后至关重要。

\subsubsection{急性肾衰竭的早期诊断与分级标准}

早期诊断是防治急性肾衰竭的关键。目前,急性肾衰竭已受到临床广泛的重视,诊断标准多达30多个,治疗措施也取得长足进步,但尚缺乏统一的诊断标准,尤其缺乏早期诊断标准。例如以需要肾脏替代治疗作为诊断标准,这类患者的肾衰竭实际上已达到终末阶段,使早期治疗无从谈起,造成治疗延误。

诊断标准不统一不但造成诊断和治疗的延误,且造成流行病学研究结果不具可比性。文献报道,重症医学科中急性肾衰竭的患病率为1%~31%,而病死率也从19%到83%不等。诊断标准中肾功能损害程度与病死率明显相关,若以轻度肾功能损害为标准,则病死率明显降低,而以严重肾功能损害为标准,则病死率明显增加。

因此,急性肾衰竭理想的诊断标准应既能实现急性肾衰竭早期诊断,又能准确反映其严重程度,并在临床能够易于理解和施行。同时,对急性肾衰竭应进行不同阶段的动态观察与诊治。早期诊断有助于早期防治,是降低急性肾衰竭重症患者病死率的关键,但对急性肾衰竭终末阶段的研究和观察同样是重要的,如肾脏替代性治疗的疗效评估、终末期肾衰竭对其他器官的影响与治疗,这也是目前存在众多不同诊断标准的原因之一。

目前诸多的诊断标准具有以下特点:①常用溶质清除能力间接反映肾功能,如血肌酐浓度;②用单位时间的尿量反映肾功能的急剧恶化,通常以24小时尿量<400~500ml,或每小时尿量<0.5ml/kg持续24小时为诊断标准;③对既往有肾脏损害病史者,采取不同标准。这些特点对于建立新的诊断标准仍具有借鉴意义。

鉴于急性肾衰竭诊断标准中存在的诸多问题,由危重病和肾脏病专家组成的急性透析质量控制倡议组织(acute
dialysis quality initiative
group,ADQI)在2004年第二次国际共识会议中,提出了急性肾衰竭的共识性分层诊断标准(表\ref{tab11-1})
\protect\hyperlink{text00017.htmlux5cux23ch1-16}{\textsuperscript{{[}1{]}}}
,该标准试图涵盖从存在急性肾损伤危险性开始,到急性肾损伤的最严重阶段------肾衰竭的全过程,包括急性肾损伤危险(risk,R)、急性肾损伤(injury,I)、急性肾衰竭(failure,F)三个阶段,同时这一标准也包括了肾功能丧失(loss,L)和终末期肾功能丧失(end-stage
kidney
disease,E)两个终末肾损害阶段,将这5个层次的英文第一个字母连在一起,即RIFLE,因此,该急性肾损伤的分层诊断标准也称为RIFLE分层标准。

\begin{table}[htbp]
\centering
\caption{急性肾功能损伤的RIFLE分层诊断标准}
\label{tab11-1}
\includegraphics{./images/Image00085.jpg}
\end{table}

RIFLE分层诊断标准首先解决了急性肾衰竭的早期诊断问题,使临床早期诊断成为可能。该标准同时也包含了急性肾损害最严重的阶段------急性肾衰竭的诊断,并对终末期的肾功能丧失进行了定义。

当然,RIFLE分层诊断标准的价值,还取决于其对急性肾损害的分层是否能够准确反映重症患者的预后。研究发现若以RIFLE标准对重症患者进行预后评估,67%的重症患者发生急性肾损害,其中急性肾损伤危险、急性肾损伤和急性肾衰竭分别占12%、27%和28%。未合并急性肾功能损害的患者病死率仅5.5%,而发生急性肾功能损害者病死率明显增加,根据RIFLE标准对肾损害程度进行分层,急性肾损伤危险、急性肾损伤和急性肾衰竭的病死率依次显著增加,分别为8.8%、11.4%和26.3%。可见,RIFLE分层标准能够有效反映重症患者的预后,且有助于急性肾损害的早期诊断,值得在重症患者中推广应用。

自RIFLE诊断标准发表至今,全球已有超过55万人使用了该标准,引用该标准的原始文献超过17万篇,已达到对急性肾损伤诊断标准化的目的。

随着RIFLE标准的广泛使用,其缺陷也逐渐暴露出来,引起人们的关注:RIFLE标准忽视了肌酐和尿量的轻微改变,然而近年来越来越多的研究认为肌酐值升高150%过于保守,轻微肌酐值变化对预后也有极大的影响。基于这些原因,2005年9月急性肾损伤网络(acute
kidney injury
network,AKIN)专家组在阿姆斯特丹召开会议对RIFLE标准进行了讨论和修正,并于2007年发布了新的标准(AKIN标准)。根据该标准,将急性肾损伤定义为:不超过3个月的肾脏功能或结构方面的异常,包括血、尿、组织检测或影像学方面的肾损伤标志物的异常,其诊断要点为:肾功能突然减退,患者在48小时内血清肌酐升高绝对值≥26.4μmol/L(0.3mg/dl);或血清肌酐值较基线升高≥50%;或每小时尿量<0.5ml/kg、时间超过6小时。具体分级标准见表\ref{tab11-2}。

\begin{table}[htbp]
\centering
\caption{急性肾功能损伤的AKIN分层诊断标准}
\label{tab11-2}
\includegraphics{./images/Image00086.jpg}
\end{table}

与RIFLE诊断标准相比较,AKIN诊断标准做了5个方面的修改:①保留了RIFLE诊断标准的3个急性期变化,但取消了R、I和F分期名称,改为数字分期,1、2、3期基本对应于RIFLE的R、I和F分期;②取消了肾小球率过滤变化标准,单纯采用肌酐标准;③在1期诊断标准中增加了血清肌酐绝对值升高≥26.4μmol/L(0.3mg/dl),肌酐变化值更小,可能提高了诊断的敏感性;④将所有接受肾脏替代治疗的患者划分为急性肾损伤3期,相当于RIFLE标准的衰竭期(failure);⑤取消了RIFLE诊断标准中判断预后分级的两个分期(L期和E期)。

AKIN标准将诊断时限限制在48小时以内,强调了血肌酐的动态变化,这样改动是考虑可能会带来以下好处:①排除了肾功长期缓慢改变带来的误诊;②采用肌酐绝对值变化作为诊断标准,肌酐变化值更小,同时避免了基础值无法确定所带来的诊断困难,为临床上急性肾损伤的早期诊断和干预提供了可能性;③对于造成肌酐和尿量短期急剧改变的可早期纠正的“可逆性”病因,如容量不足或尿路梗阻,提供了充足的复苏和纠正时间,有助于提供更准确的诊断。

基于现有的关于RIFLE和AKIN诊断标准的比较研究,还无法得出AKIN标准优于RIFLE标准的结论
\protect\hyperlink{text00017.htmlux5cux23ch2-16}{\textsuperscript{{[}2{]}}}
,仍需大规模的前瞻性研究评估不同分层标准对急性肾衰竭的早期诊断价值及预后价值。

\subsubsection{重症医学科中急性肾衰竭的早期防治}

鉴于急性肾衰竭是导致重症患者预后凶险的重要原因,重症医学科的重症患者是急性肾衰竭的高危人群,早期预防急性肾衰竭显得十分重要。针对重症医学科中导致急性肾衰竭的常见原因,采取目标导向性的预防策略,有可能降低急性肾衰竭的患病率。

(1)严重感染导致的急性肾衰竭 严重感染和感染性休克是导致急性肾衰竭的常见原因。严重感染者中有9%~40%的患者最终发生急性肾衰竭,感染的严重程度明显影响急性肾衰竭发生率,反之,发生急性肾衰竭也进一步增加严重感染患者的病死率。Bagshaw等对33375名全身感染患者调查发现,42.1%的患者并发急性肾损伤;全身感染所致的急性肾损伤往往病情更重,住重症医学科时间更长,死亡率更高
\protect\hyperlink{text00017.htmlux5cux23ch3-16}{\textsuperscript{{[}3{]}}}
。严重感染的患者并发急性肾衰竭的病死率高达70%,明显高于其他原因所致急性肾衰竭的病死率。可见早期防治严重感染导致的急性肾衰竭,对于最终改善严重感染的预后具有重要临床价值。

缺血和炎症性损伤是严重感染导致急性肾衰竭的主要机制。内毒素诱发的复杂炎症和免疫网络反应等多个方面,参与了感染性急性肾损伤的发病,并可能成为其主要机制。有研究证实,感染性急性肾损伤的肾脏局部会释放TNF-α等炎症因子,并引起肾小管细胞凋亡
\protect\hyperlink{text00017.htmlux5cux23ch4-16}{\textsuperscript{{[}4{]}}}
。上世纪90年代以来,针对控制炎症反应的炎性细胞因子单克隆抗体或阻断剂的研究一度给严重感染的治疗和急性肾衰竭的预防带来希望,然而,不仅单克隆抗体价格昂贵,且所有的临床研究均以失败告终,看来试图单纯阻断少数炎性介质来控制复杂的炎症反应网络,进而控制严重感染、预防急性肾衰竭的目标目前仍难以实现。积极纠正严重感染的低血容量状态,逆转肾脏缺血,成为急性肾衰竭防治的希望。

严重感染时肾脏低灌注是导致急性肾衰竭的重要原因,早期强化的目标性血流动力学管理是纠正肾脏低灌注的有效途径。Rivers等学者在严重感染或感染性休克发生6小时内,通过积极液体复苏使中心静脉压达到8~12mmHg(1mmHg=0.133kPa),以纠正有效循环血容量不足,若平均动脉压仍低于65mmHg,则加用血管活性药物,恢复有效组织灌注。监测每小时尿量,使每小时尿量>0.5ml/kg。同时监测中心静脉血氧饱和度或混合静脉血氧饱和度,若中心静脉血氧饱和度<70%,补充红细胞悬液,使血细胞比容>30%,以此重建和维持氧需与氧供的平衡。若6小时内实现早期目标导向治疗,严重感染的病死率可从46.5%降至30.5%,且急性肾衰竭的发生率也明显降低
\protect\hyperlink{text00017.htmlux5cux23ch5-16}{\textsuperscript{{[}5{]}}}
。早期有效的改善肾脏灌注成为预防严重感染患者发生急性肾衰竭的有效途径。至于早期液体复苏中液体种类对急性肾衰竭发生的影响,目前尚无确切的证据说明胶体溶液和晶体溶液孰优孰劣,但是就恢复有效循环血量的速度和效率而言,胶体溶液明显优于晶体溶液。

近年来,不同血管活性药物在急性肾衰竭防治中的地位备受关注。以往认为,多巴胺具有选择性肾血管扩张和增加尿量的作用,肾脏剂量(小剂量)的多巴胺长期在临床上被广泛用于急性肾衰竭的防治。但大量的研究表明,每分钟3~5μg/kg多巴胺对肾脏血管并无血管扩张作用,甚至有轻度的缩血管作用;小剂量多巴胺增加尿量与其轻度抑制近曲小管钠的重吸收有关,并不增加肌酐清除率;小剂量多巴胺既不能预防重症患者发生急性肾衰竭,对病死率也无影响;另外,多巴胺也存在明显不良作用,除引起心动过速外,还对垂体前叶激素具有抑制作用,抑制T细胞功能,抑制呼吸中枢兴奋性,并可减少肠道灌注。总的来讲,小剂量多巴胺并无肾脏保护作用,临床上不应常规应用。

去甲肾上腺素越来越多地应用于感染性休克的治疗,其有可能对急性肾衰竭具有预防作用。在正常人和动物中,去甲肾上腺素明显减少肾血流量和尿量,但在严重感染的情况下,去甲肾上腺素能够明显改善感染性休克患者的肾小球滤过率,并增加尿量。前瞻性研究显示,去甲肾上腺素组的病死率明显低于多巴胺组,但目前尚缺乏去甲肾上腺素对感染性休克急性肾衰竭预防效应的直接依据。

血管加压素一般用于大剂量常规升压药无效的顽固性感染性休克。最近的研究显示,血管加压素对肾脏可能具有保护作用。肾小球滤过率主要由入球小动脉和出球小动脉的压力差决定,血管加压素收缩出球小动脉更明显,使肾小球滤过压明显增加,进而增加肾小球滤过率和尿量,发挥肾保护作用。已有小样本临床研究显示,血管加压素能够预防感染性休克患者发生急性肾衰竭,并明显优于其他血管活性药物,但仍需要多中心的随机对照研究进一步证实。

(2)药物导致的急性肾衰竭 具有肾毒性药物易于引起或加重肾功能损害,如氨基糖苷类、万古霉素和两性霉素B等常用药物。避免应用肾毒性药物或采用更为合理的用药方法,有可能预防急性肾衰竭的发生。

重症患者应用氨基糖苷类药物导致肾功能损害的发生率高达10%。氨基糖苷类药物主要通过肾小球滤过,在肾小管中部分被重吸收,并积聚在小管上皮细胞溶酶体中,其肾损害主要与溶酶体破坏和肾小管上皮细胞膜损伤导致小管细胞坏死有关。氨基糖苷类药物是否导致肾损害,不仅与肾小管中药物浓度与作用时间有关,还与治疗疗程、既往是否具有肾损害病史有关。

(3)造影剂导致的急性肾衰竭 影像学诊断应用的造影剂或增强剂可诱导急性肾衰竭,占医院获得性急性肾衰竭的10%。尽管肾功能正常者应用造影剂后急性肾损害的发生率很低,但已有轻度肾损害者应用造影剂后急性肾损害的发生率可达5%,而已有明显肾功能损害或糖尿病者,应用造影剂后急性肾损害的发生率可高达50%。可见,基础肾功能状态也是决定造影后是否发生急性肾损害的重要因素。

总之,重症患者一旦发生急性肾衰竭,预后凶险,应用RIFLE及AKIN标准有助于实现急性肾损伤的早期诊断和治疗,并利于对高危患者采取积极措施,预防急性肾衰竭的发生,有可能最终改善重症患者的预后。

\section{临床问题}

\subsection{急性肾衰竭的病因与临床特征}

\subsubsection{重症患者发生急性肾衰竭的危险因素有哪些?}

一般认为,低血压/休克、充血性心衰、全身性感染、糖尿病、氨基糖苷类抗生素应用、造影剂应用、高胆红素血症、机械通气、外科大手术、肾移植等因素是发生急性肾衰竭的独立危险因素。

(1)全身性感染 全身性感染是急性肾衰竭患病最重要的独立危险因素。Brivet的研究显示急性肾衰竭的主要患病因素中,48%为全身性感染。不但院内感染与急性肾衰竭的发病有关,社区获得性感染也与急性肾衰竭患病密切相关。Rayner报道了239例社区获得性全身性感染患者,其中有24%的患者血清肌酐浓度升高1倍以上。

(2)肾毒性药物的应用 肾毒性药物的应用也是重要的独立危险因素。肾毒性药物引起的急性肾衰竭约占各种病因的35%。肾毒性药物很多,不同的肾毒性药物具有不同的肾损伤机制------①高张力性损害:葡聚糖、甘露醇等可导致肾脏发生高张性损伤;②缺血性损害:利尿剂、血管紧张素转换酶抑制剂、降压药等导致血容量不足和血压降低,使肾脏灌注减少而导致肾脏损害;③肾小管毒性损害:氨基糖苷类药物、万古霉素、两性霉素B、放射造影剂、重金属等对肾小管具有直接损害作用,Ⅳ型免疫球蛋白、葡聚糖、麦芽糖、蔗糖、甘露醇等可导致肾小管肿胀;④肾血管内皮细胞损害:环胞素A、丝裂霉素C、可卡因、雌激素、奎宁等药物可导致肾脏毛细血管内皮细胞损伤;⑤入/出球小动脉舒缩异常:非甾体抗炎药、放射造影剂、两性霉素B等可导致入球小动脉痉挛,降低肾小球滤过率。血管紧张素转换酶抑制剂、血管紧张素Ⅱ受体拮抗剂等药物可导致出球小动脉扩张,降低肾小球滤过率;⑥结晶尿:磺胺药物、无环鸟苷、蛋白酶抑制剂等药物可在肾小管内结晶,导致肾小管功能损害;⑦肾小球损害:金制剂、青霉胺、非甾体抗炎药等可直接损伤肾小球;⑧间质性肾炎:多种药物可导致间质性肾炎;⑨色素性肾小管功能损害:肌红蛋白尿和血红蛋白尿可导致肾小管损害。

(3)重大手术 重大手术也是急性肾衰竭患病的高危因素之一。Liano报道的748例急性肾衰竭患者中,有27%为术后患者。研究显示心脏术后患者急性肾衰竭患病率为0.4%~7.5%,而非心脏大手术患者的急性肾衰竭患病率为0.6%。心脏术后的肾功能障碍是影响患者生存的独立因素,其比值是无肾功能障碍的7.9倍,因此重大手术患者应特别注意急性肾衰竭的早期预防和治疗。

重大手术后患者易发生急性肾衰竭的原因,主要与下列因素有关:①患者具有糖尿病、高血压、血管性疾病、充血性心衰等慢性疾病,导致患者肾脏功能贮备降低,基础肾小球滤过率下降;②麻醉和手术应激导致肾小球入球小动脉收缩,肾小球滤过率降低;③术后并发全身性感染、休克、心衰等并发症,或应用肾毒性药物,或二次手术,构成对肾脏的二次打击,极易导致急性肾衰竭。

\subsubsection{急性肾衰竭的主要病因分型有哪几类?}

急性肾衰竭的病因复杂,根据致病因素在肾脏直接作用的部位,可分为肾前性因素、肾性因素和肾后性因素。

(1)肾前性急性肾衰竭 主要与血容量不足和心脏泵功能明显降低导致肾脏灌注不足有关,在急性肾衰竭中最为常见,占30%~60%。反映了当前重症患者治疗中,对容量状态或肾脏灌注缺乏足够的重视,或对容量估计严重不足。肾前性肾衰是医院获得性肾衰的主要原因之一。

各种肾前性因素引起血管内有效循环血量减少,肾脏灌注量减少,肾小球滤过率降低,并使肾小管内压力低于正常;流经肾小管的原尿减少,速度减慢,因此尿素氮、水及钠的重吸收相对增加,从而引起血尿素氮升高,尿量减少及尿比重增高的现象,称为肾前性氮质血症。因肾小管对钠的重吸收相对增加,使尿钠排出减少,钠排泄比例明显降低、肾衰竭指数降低(<1mmol/L),因尿少、尿素氮重吸收相对增加,出现尿素氮和血肌酐浓度不成比例的增高(即球管间不平衡现象),血尿素氮可高达37.5mmol/L(100mg/dl)以上,而血肌酐则仅稍高于正常,尿与血的肌酐比例明显升高。

引起肾前性急性肾衰竭的原因常常包括:①低血容量,由于严重外伤、烧伤、挤压综合征、大出血、外科手术、脱水、胰腺炎、呕吐、腹泻或大量应用利尿剂所致;②有效血容量减少,由于肾病综合征、肝衰竭、全身性感染、休克、应用血管扩张剂或麻醉药所致;③心输出量减少,由于心源性休克、心肌梗死、严重心律紊乱、充血性心功能衰竭、心包填塞及急性肺梗死所致;④肾血管阻塞,由于肾静脉或肾动脉栓塞,或动脉粥样变所致;⑤肾血管的自身调节紊乱,由于前列腺素抑制剂、血管紧张素转化酶抑制剂、环孢菌素A的作用所致。

(2)肾性急性肾衰竭 肾性急性肾衰竭是肾实质疾患所致,或由于肾前性病因未能及时解除而发生肾实质病变,占急性肾衰竭的20%~40%。在考虑急性肾衰竭的肾性因素时,应考虑到肾脏的各个解剖结构是否发生病变,不但应考虑到肾血管、肾小球的病变,还应注意肾间质和肾小管等解剖结构的病变。当然,需要注意的是,尽管急性肾脏血管病变(如动脉栓塞、血管炎、血栓形成等)、肾小球病变(如肾小球肾炎等)、间质性病变(如过敏性间质性肾炎等)均是急性肾衰竭的病因之一,但急性肾衰竭、特别是医院获得性急性肾衰竭最重要的病因仍然是急性肾小管损伤。急性肾小管坏死往往与肾脏缺血和肾毒性药物的应用有关。

归纳起来,急性肾性肾衰的病因主要包括:①肾小管疾患,为急性肾衰竭的主要病因,其中以急性肾小管坏死最为常见,肾缺血、肾中毒(药物、造影剂、重金属、有机溶剂、蛇毒、中草药)及高钙血症等均可引起肾小管损伤,导致急性肾衰竭;②肾小球疾患,多数患者表现为少尿型肾衰,占87.5%,非少尿型占14.3%;③急性肾间质性疾患,主要因严重感染、全身性感染及药物过敏或由于淋巴瘤、白血病或肉瘤病变侵及肾间质所致;④肾脏血管疾病,肾脏的小血管炎或大血管疾患;⑤慢性肾脏疾病急性恶化,某些诱因致使病情急剧恶化,肾功能急骤减退也可导致急性肾衰竭。

(3)肾后性急性肾衰竭 各种原因引起的急性尿路梗阻(如腔内阻塞或外部压迫等),导致急性肾衰竭,归结为肾后性急性肾衰竭,临床上较为少见,占急性肾衰竭的1%~10%。如诊断和治疗及时,这类肾衰竭往往可恢复。

肾以下尿路梗阻使梗阻上方的压力增高,甚至发生肾盂积水,肾实质受压使肾功能急剧下降。肾后性急性肾衰竭可见于:①结石、肿瘤、血块、坏死肾组织或前列腺增生所致的尿路梗阻;②肿瘤蔓延、转移或腹膜后纤维化所致的粘连、压迫输尿管而引起梗阻。

\subsubsection{急性肾衰竭少尿期有哪些临床特征?}

急性肾小管坏死病因不一,起始表现也不同,一般起病较急骤,全身症状明显。根据临床表现和病程的共同规律,一般分为少尿期、多尿期和恢复期3期。少尿期的临床特征主要包括:

(1)尿量减少 尿量骤减或逐渐减少,每日尿量持续少于400ml者为少尿,少于100ml者为无尿。急性肾小球坏死患者罕见完全无尿,持续无尿者预后极差。由于致病原因不同,病情轻重不一,少尿持续时间不一致,一般为1~2周,但可短至数小时或长达3个月以上。一般认为肾中毒者持续时间短,而缺血性者持续时间较长。若少尿持续4周以上应重新考虑急性肾小管坏死的诊断,有可能存在肾皮质坏死、原有肾疾患或肾乳头坏死等。

非少尿型急性肾小管坏死,指患者在氮质血症期每日尿量持续在500ml以上,甚至1000~2000ml。非少尿型的发病率近年有增加趋势,高达30%~60%。

(2)进行性氮质血症 由于肾小球滤过率降低引起少尿或无尿,致使排出氮质和其他代谢物质减少,血浆肌酐和尿素氮升高,其升高速度与体内蛋白分解状态有关。在无并发症且治疗恰当的病例,每日血尿素氮上升速度较慢,为3.6~7.1mmol/L(10~20mg/dl),血浆肌酐浓度上升仅为44.2~88.4μmol/L(0.5~1.0mg/dl)。但在高分解状态时,如伴广泛组织创伤、全身性感染等,每日尿素氮可升高10.1~17.9mmol/L(30~50mg/dl),肌酐每日升高176.8μmol/L(2mg/dl)或以上。促进蛋白分解的因素尚有热量供给不足、肌肉坏死、血肿、胃肠道出血、感染、发热、应用糖皮质激素等。

(3)水、电解质紊乱和酸碱平衡紊乱 包括水中毒、高钾血症、代谢性酸中毒、低钙血症、高磷血症、低钠血症和低氯血症等。

水过多:见于水分控制不严,摄入量或补液量过多,失水量如呕吐、出汗、伤口渗液量等估计不准确以及液体补充时忽略计算内生水。随少尿期延长,易发生水过多,表现为稀释性低钠血症,软组织水肿、体重增加、高血压、急性心功能衰竭和脑水肿等。

高钾血症:①由于尿液排钾减少,如同时体内存在高分解状态,导致细胞内钾离子释放入血;②挤压时肌肉坏死、血肿和感染等或酸中毒时细胞内钾转移至细胞外,有时可在几小时内发生严重高钾血症;③静脉内滴注大剂量的青霉素钾盐(每100万单位青霉素钾盐1.6mmol);④大量库存血(库存10天血液每升含钾可达22mmol);⑤摄入含钾较多食物或饮料,均可引起或加重高钾血症。无并发症者每日血钾上升不到0.5mmol/L。高钾血症有时隐匿,可无特征性临床表现,或仅出现恶心、呕吐、手麻、心率减慢,直到后期出现室内、房室传导阻滞或心脏停搏。高钾对心肌毒性作用尚受体内钠、钙浓度和酸碱平衡的影响,当同时存在低钠、低钙血症或酸中毒时,高钾血症临床表现较显著,且易诱发各种心律失常。值得一提的是,血清钾浓度与心电图之间可存在不一致现象。高钾血症是常见的死因之一,早期透析可预防其发生。

代谢性酸中毒:正常人每日固定酸性代谢产物为50~100mmol。急性肾衰竭时,由于酸性代谢产物排出减少,肾小管泌酸能力和保存\ce{HCO3-}
能力下降等原因,致使每日血浆\ce{HCO3-}
浓度下降1~2mmol/L;在高分解状态时降低更多、更快。内源性固定酸大部分来自蛋白分解,少部分来自糖和脂肪氧化。\ce{HCO3-}
和其他有机阴离子均释放和堆积在体液中,导致本病患者阴离子间隙增高。酸中毒可降低心室颤动阈值。高钾血症、严重酸中毒和低钙低钠是急性肾衰竭的严重状况,在已接受透析治疗的病例较少见,但对严重肌肉组织坏死病例,特别是深部肌肉坏死者仍应警惕。

低钙血症、高磷血症:少尿2天后即可发生低钙血症。由于常同时伴有酸中毒,使细胞外液离子钙增多,故多不发生低钙常见的临床表现。高磷血症较常见,但罕见明显升高。

低钠血症和低氯血症:两者多同时存在,低钠血症主要是由于水过多所致稀释性低钠血症。严重低钠血症时,血浆渗透浓度降低,导致水分向细胞内渗透,出现细胞水肿,表现为急性水中毒和脑水肿症状,并进一步加重酸中毒。低氯血症除稀释性外,尚可因呕吐、腹泻等加重,可出现腹胀或呼吸表浅、抽搐等代谢性碱中毒表现。

(4)心血管系统表现 主要包括高血压、心功能衰竭等。

高血压:除肾缺血、肾素分泌增多因素外,水过多引起容量负荷过多可加重高血压。急性肾小管坏死早期高血压不多见,但若持续少尿,约1/3患者发生轻、中度高血压,血压一般在140~180/90~100mmHg,有时可更高,甚至出现高血压脑病,伴有妊娠者尤易发生。

心功能衰竭:主要为液体潴留引起,但高血压、严重心律失常和酸中毒等为影响因素。早年发生率较高,在采取严格控制水分和早期透析等措施后发生率已明显下降。

心律失常:除高钾血症引起窦性停搏、窦房传导阻滞、不同程度房室传导阻滞和束支传导阻滞、室性心动过速、心室颤动外,尚可因病毒感染和洋地黄应用等而引起室性早搏等心律失常的发生。

心包炎:早年发生率为18%,采取早期透析后降至1%。多表现为心包摩擦音和胸痛,罕见大量心包积液。

\subsubsection{急性肾衰竭多尿期和恢复期有何特点?}

进行性尿量增多是肾功能恢复的一个标志,每日尿量可成倍增加。一般认为,24小时尿量增加到400ml以上,提示急性肾衰竭进入多尿期。进入多尿期后,肾功能并不立即恢复,存在高分解代谢的患者血浆肌酐和尿素氮仍上升,当肾小球滤过率明显增加时,血氮质逐渐下降。多尿期早期仍可发生高钾血症,多尿期后期易发生低钾血症。另外,此期仍易发生感染、心血管并发症和上消化道出血等并发症。多尿期持续时间多为1~3周或更长。

恢复期患者自我症状缓解,血尿素氮和肌酐接近正常,尿量逐渐恢复正常。除少数外,肾小球滤过功能多在3~12个月内恢复正常,但部分病例肾小管浓缩功能不全可持续1年以上。若肾功能持久不恢复,提示肾脏遗留永久性损害。

\subsection{急性肾衰竭的临床诊断}

\subsubsection{急性肾衰竭的临床诊断思路是什么?}

急性肾衰竭的早期诊断对重症患者十分重要。要做到对急性肾衰竭迅速诊断,应首先排除肾前性和肾后性因素,然后确定肾脏本身的原因。一般可采用以下“四步法”进行急性肾衰竭的诊断。

第一步:了解既往病史和现病史,进行体格检查,导尿(特别是无尿患者),做尿液分析。

第二步:

(1)分析尿液检查结果(表\ref{tab11-3});

\begin{table}[htbp]
{\centering
\caption{肾前性氮质血症与急性肾衰竭(急性肾小管坏死)的尿液分析比较}
\label{tab11-3}
\includegraphics{./images/Image00090.jpg}}

\footnotesize
* 钠排泄分数=(尿钠×血肌酐)/(血钠×尿肌酐)×100%;

** 肾衰指数=尿钠×血肌酐/尿肌酐。
\end{table}



(2)评价尿路情况,排除尿路梗阻。可采用B超等检查手段;

(3)如需进一步了解患者血管内容量状态和心脏功能状态,可通过有创动脉压监测、中心静脉压监测、肺动脉漂浮导管监测及超声心动图(特别是食管超声)检查,对患者容量状态和心功能状态进行评价;

(4)如考虑肾小球肾病或血液系统恶性肿瘤,则应进一步进行血液学检查;

(5)如考虑肾脏血管病变,应通过同位素扫描、超声多普勒或血管造影,对肾血管情况进行评价。

第三步:根据急性肾衰竭病因,确定初步治疗方案。包括血容量补充、正性肌力药物的应用、解除尿路梗阻等措施。

第四步:为进一步明确诊断,可行肾脏活检,并根据初步诊断,采取经验性治疗。

一般情况下,通过“四步法”诊断步骤中的第一步,可初步明确急性肾衰竭的病因。通过了解现病史和既往史,可明确患者是否应用肾毒性药物、是否应用放射造影剂、是否有血容量不足、低血压等肾脏缺血因素、是否有大手术等肾脏损害的危险因素。

明确患者的容量状态,早期纠正低血容量状态或低心排状态,具有重要的价值。对于肾前性氮质血症患者,早期纠正肾脏的低灌注状态,可逆转氮质血症,防止急性肾衰竭发生。即使对于急性肾衰竭的患者,积极纠正低灌注状态,也有利于防止肾脏功能的进一步恶化,促进肾功能早期恢复。详细的体格检查,结合有关病史,往往可以得到患者容量状态的证据。

当患者容量状态判断较为困难时,放置肺动脉漂浮导管,监测心输出量、肺动脉嵌顿压和中心静脉压,可较准确地评价患者的容量状态和心脏功能状态,同时,可指导容量复苏/正性肌力药物等治疗措施的调整。在容量复苏或应用正性肌力药物时,应同时观察尿量和尿液分析的变化。

尿液分析是急性肾衰竭的重要诊断手段。肾前性和肾后性氮质血症患者的尿液检查往往是正常的。尿液镜检中发现大量的色素颗粒管型或上皮细胞管型,常提示肾缺血或肾毒性药物引起的急性肾衰竭。

显微镜下如发现血色素,而且与红细胞不成比例,提示患者的急性肾衰竭与横纹肌溶解或溶血引起的色素尿有关。

当患者有明显的蛋白尿、血尿,尿液检查中发现大量的红细胞管型,提示急性肾衰竭与急性肾小球肾炎或血管炎有关。

尿液出现大量白细胞管型,见于急性肾盂肾炎、间质性肾炎或肾小球肾炎。

尿液沉渣Hansel染色发现嗜酸性粒细胞,则为嗜酸性粒细胞尿,急性肾衰竭并非是肾小管损害的结果。

当患者无发热、皮疹、外周血嗜酸性粒细胞增加等全身性过敏反应表现时,应首先考虑嗜酸性粒细胞尿与药物引起的间质性肾炎有关。

对于动脉造影后出现急性肾衰竭的患者或存在周围血管病变的急性肾衰竭患者,如发现嗜酸性粒细胞尿,提示急性肾衰竭与动脉栓塞性肾血管病变有关。

尿比重、渗透压、尿钠浓度及钠排泄分数等尿液指标是诊断和评价急性肾衰竭的重要指标(表\ref{tab11-3})。肾前性氮质血症导致少尿的患者,往往具有正常的肾小管功能,而急性肾衰竭患者的肾小管功能明显受损,肾小管对溶质和水的重吸收功能明显减低,由此可通过尿液诊断指标对急性肾衰竭与肾前性氮质血症进行鉴别。

当然,尿液诊断指标并不是完全可靠的。尿液中电解质的结果受许多因素的影响。病情不同、治疗干预不同,尿液电解质就可能出现不同的结果。对于接受利尿剂治疗的患者,葡萄糖、尿酸、放射造影剂等可导致碳酸氢钠尿和渗透性利尿。而对于原发性肾上腺皮质功能不全的患者,容量不足引起肾前性氮质血症时,尽管患者存在血容量不足,尿钠排泄分数仍然明显升高。另外,由于慢性肾脏功能不全或间质性肾炎患者肾小管对钠的重吸收功能降低,当容量不足引起肾前性氮质血症时,尿钠的排泄仍然很多。还需注意的是,尿钠和尿钠排泄分数降低也并非肯定是肾前性氮质血症。全身性感染、放射造影剂、横纹肌溶解等原因导致的间质性肾损害,早期肾小管就具有正常功能。这应引起重症医学科医师的高度重视,以免延误诊断和治疗。在尿路梗阻、急性肾小球肾炎、急性间质性肾炎等情况下,尿液诊断指标的结果往往也是不可靠的。

\subsubsection{何谓肾小球滤过率?}

肾小球滤过率即在单位时间内(分钟)从双肾滤过的血浆的毫升数,为测定肾小球滤过功能的重要指标。实际上当某种存在于血中的溶质,如果能从肾小球滤过,肾小管内不被重吸收也不分泌,此时肾小球滤过率=尿液中溶质浓度×单位时间尿量/血浆溶质浓度。在这种情况下的肾小球滤过率就是每分钟有多少毫升血中的溶质被肾小球清除。

可见,用于评价肾小球滤过率的溶质应具备以下条件:①能够从肾小球滤过;②溶质不被肾小管吸收;③肾小管也不分泌或排泄该溶质。

\subsubsection{何为菊粉清除率?有何意义?}

菊粉是一种不带电荷的果糖聚合物,分子量5200道尔顿,无毒,不参与任何化学反应。可从静脉注入人体,不与血浆蛋白结合,主要分布于细胞外液。清除方式是只从肾小球滤过而不被肾小管重吸收或分泌,在体内既不能合成亦不能分解。可见,菊粉符合测定肾小球滤过率的要求,菊粉清除率可以准确反映肾小球滤过功能,是测定肾小球滤过率的“金标准”。

测定方法:患者于清晨空腹,静脉滴注10%的菊粉溶液,同时放置导尿管。到血浆中菊粉的浓度稳定在10mg/L水平,每分钟尿量稳定后,测尿中的菊粉浓度,代入公式:菊粉清除率=尿菊粉浓度×单位时间尿量/血浆菊粉浓度,就是患者的肾小球滤过率。菊粉清除率虽然精确,但测定时程序繁杂,不适于临床应用。

\subsubsection{为什么肌酐清除率可以评价肾功能?有何临床意义?}

肌酐清除率是评价肾脏功能最常用的方法,但在临床应用时,必须了解其生理代谢情况及其与肾脏功能的关系,才有可能对肾脏功能做出合理的评价。

(1)肌酐的代谢与生理 肌酐是人体内肌酸的代谢产物,肌酸量与肌肉量成正比。正常情况下,机体以比较稳定的速度产生肌酐,并释放入血液循环。再由血液循环带到肾脏,从尿中排出到体外。正常人肌酐的排泄主要通过肾小球的滤过作用,原尿中的肌酐不被肾小管重吸收,而且,正常情况下肾小管几乎不分泌肌酐。当然,正常情况下人体内的肌酐来源包括内生肌酐(体内肌酸分解而来)和外生肌酐(来自摄入的鱼、肉类食物),由于外源性肌酐不足以影响清晨空腹时的血肌酐测定,所以空腹时血肌酐水平是比较稳定的。正常人每日肌酐的产生量和排出量是相等的。

肌酐的分子量为113道尔顿,不被肾脏代谢,不与蛋白质结合,可以自由通过肾小球,不被肾小管重吸收,在血肌酐无异常增高时亦不为肾小管分泌,所以可用肌酐清除率代替菊粉清除率检测肾小球滤过率。

(2)肌酐清除率的测定方法 肌酐主要从肾小球滤过,但亦有时从肾小管排泌,故肌酐清除率并非十分理想的代表肾小球滤过率的指标,它高于肾小球滤过率的实际值,尤其在肾功能减退时。但检测方便,目前仍较广泛地应用来表示肾小球滤过率。

常规方法:以往的做法是素食3天后,收集24小时的全部尿液,在收集尿液结束时取血,测定血、尿中肌酐浓度,然后计算肌酐清除率。因收集24小时尿液较麻烦,全天血肌酐水平也有波动,在同一个人测多次肌酐清除率的结果,其误差可达25%,而且,少量外源性肌酐不影响次日清晨空腹血肌酐浓度。因此,目前多采取测清晨空腹血及取血前后共4小时全部尿量进行肌酐清除率测定,以减少误差,而且测量前不必素食。正常值为:80~120ml/分。

Cockcroft推算法:1976年Cockcroft和Gault提出以血肌酐值推算肌酐清除率的公式(性别不同,公式略有不同),但此公式对老年人、儿童及肥胖者不适用。

\[
\begin{array}{l}
    \text{男性肌酐清除率(ml/分)}=\frac{(140-\text{年龄}\times \text{体重(kg)}}{72\times \text{血肌酐(md/dl)}}\\
    \text{男性肌酐清除率(ml/分)}=\frac{(140-\text{年龄}\times \text{体重(kg)}}{85\times \text{血肌酐(md/dl)}}
\end{array}    
\]

1991年Giovannetti和Barsotti报道以菊粉清除率为标准评价肌酐清除率,包括肾功能正常及慢性肾衰竭者在内,其结果发现肌酐清除率的敏感性不比菊粉清除率差。而Desanto等亦以菊粉清除率为标准,将测定的肌酐清除率和用Cockcroft公式计算出的肌酐清除率进行比较,他们认为计算的肌酐清除率较实际测定的肌酐清除率更能真实反映肾小球滤过率。

西米替丁改良法:1996年Zaitzman等提出西米替丁改良法测定肌酐清除率,主要利用西米替丁竞争性强烈抑制肾小管对肌酐的分泌,使尿中肌酐完全来源于肾小球滤过,从而改善肌酐清除率作为肾小球滤过率标志的可靠性------口服西米替丁800mg,收集服药后45分钟内的尿液,同时在尿液收集结束时采血,测定尿液和血浆中肌酐浓度,计算肌酐清除率。

Zaitzman的研究显示西米替丁改良法测定的肌酐清除率与\textsuperscript{125}
I碘酞酸盐测定的肾小球滤过率(与菊粉清除率测定方法类似)比较,两者比值1.12±0.02。如不使用西米替丁,则比值为1.33±0.08。西米替丁改良法测定的肌酐清除率能够较准确的反映肾小球滤过率。

由于口服西米替丁的生物利用度为60%,口服后30分钟血浆浓度达到峰值,半衰期为2小时,因此,在口服西米替丁800mg后的3小时内均可较准确的测定肌酐清除率。

西米替丁改良法不但适用于肾脏功能轻度降低的重症患者,还特别适用于肾小球滤过率显著降低的急性肾衰竭患者。

\subsubsection{为什么要监测血肌酐?有何临床意义?}

血肌酐浓度是反映肾脏肾小球滤过率的常用指标之一。正常情况之下,体内肌酐产生的速度约为1mg/分。肌酐只从肾小球滤过并以同样速度清除。当肾小球滤过功能下降时,血肌酐即可上升。但研究证实,只有当肾小球滤过率下降到正常1/3时,血肌酐才明显上升,所以血肌酐测定并非敏感的测定肾小球过滤功能的指标。只有在肾功能不全失代偿时,血肌酐值才升高。

血肌酐的正常值应<1.5mg/dl(133μmol/L)。

性别、肌肉容积均在正常值范围内影响血肌酐的数值。当肌肉萎缩性病变的患者肌肉代谢减少时,血肌酐的浓度亦可稍低。

\subsubsection{血尿素氮异常能否说明患者肾功能异常?}

血尿素氮也是反映肾小球滤过率的常用指标之一。血中尿素氮是人体蛋白质代谢的终末产物。尿素的生成量取决于饮食中的蛋白质的摄入量,组织蛋白质的分解代谢及肝功能的情况。血液中的尿素全部从肾小球滤过,正常情况下约30%~40%被肾小管重吸收,肾小管亦可排泌少量的尿素,严重的肾衰竭时排泌量增加。血中的尿素氮的测定虽可以反映肾小球的滤过功能,但肾小球的滤过功能必须下降到正常的1/2以上时,尿素氮才会升高。故尿素氮的测定并非敏感的反映肾小球的滤过功能的指标。血尿素氮的正常值为8~21mg/dl(2.9~7.5mmol/L),其血液水平受多种因素的影响,如感染、高热、脱水、消化道出血、进食高蛋白饮食等均可影响血中尿素氮。血中尿素氮的上升不一定是肾小球的滤过功能受损的结果,临床上必须认真分析原因,鉴别真正导致血尿素氮上升的病因。

\subsubsection{血尿素氮/肌酐的临床意义如何?}

肾功能正常时,血尿素氮/肌酐通常为10。当血尿素氮>25mg/dl(8.9mmol/L)时即可诊断为氮质血症。当发生氮质血症且尿素氮/肌酐增高时,常说明此氮质血症是由于肾前因素引起(即由于各种原因引起的肾血流量的下降)。当氮质血症同时伴尿素氮/肌酐下降时,多为肾脏本身的实质性疾病引起所致,所以这一比值有助于鉴别氮质血症是由于肾前性因素还是肾性的因素引起。

\subsubsection{血β2 微球蛋白能够反映肾小球滤过功能吗?}

β\textsubscript{2}
微球蛋白是体内有核细胞,包括淋巴细胞、血小板、多形核白细胞产生的一种小分子球蛋白。其分子量为11800道尔顿,广泛存在于血浆、尿、脑脊液、唾液及初乳中,正常人血中的β\textsubscript{2}
微球蛋白浓度很低,平均约为1.5mg/dl。正常情况下可以自由通过肾小球,然后在近端肾小管内几乎全部被重吸收,当肾小球滤过功能下降时,血中的β\textsubscript{2}
微球蛋白水平上升,所以血β\textsubscript{2}
微球蛋白是反映肾小球滤过功能一个极好指标,与年龄无关,但当体内有炎症或肿瘤时,血中β\textsubscript{2}
微球蛋白增高,应注意鉴别。

\subsubsection{蛋白质负荷试验能够反映肾脏储备功能吗?}

正常肾脏在无特殊情况时,并未发挥其最大的滤过功能。国内1990年报告正常青年男性于清晨做肌酐清除率监测后按体重顿服0.8g/kg鸡蛋清蛋白,以后再做肌酐清除率,结果进食蛋白后的肌酐清除率较空腹肌酐清除率增加41.4%,而在中老年健康人测得的结果只增加了6.7%,与国外报道结果近似,这说明青年健康人肾脏具有储备功能。青年肾脏病患者肌酐清除率正常者,蛋白质负荷后肌酐清除率也只上升了9.9%~14.3%。这些结果说明肌酐清除率正常者,蛋白质负荷后如肌酐清除率上升不明显,则提示患者的肾脏储备功能有所下降。

\subsubsection{如何评价近端肾小管功能?}

(1)肾小管最大重吸收量的测定 通过肾小管葡萄糖最大重吸收量(TmG)来反映肾小管最大重吸收量。正常人血中葡萄糖从肾小球全部滤过后,在近曲小管主动的全部重吸收。随着血中葡萄糖浓度增加,原尿中葡萄糖浓度超过肾小管对葡萄糖的最大吸收极限时,尿中将有葡萄糖排出。正常人的TmG为340.0±18.2mg/分。此种方法可以反映近曲小管的重吸收功能。正常人的尿糖是阴性的,当血糖在160~180mg/dl(8.9~10mmol/L)时,可以出现尿糖。如血糖正常、糖耐量试验正常而尿糖阳性,称为肾性糖尿,提示近端肾小管重吸收功能减退。由于其测定方法比较繁琐,目前临床上已不经常采用。

(2)肾小管最大排泌量测定 肾小管最大排泌量通过肾小管对氨马尿酸最大排泌量(TmPAH)来反映。血液中的对氨马尿酸可经肾小球滤过并由肾小管排泌,在肾小管内不被重吸收。当血液中的对氨马尿酸的浓度达到一定高度时,从肾小管排泌对氨马尿酸的绝对值已达最高峰,即使血中的浓度再增高,其排泌量亦不能再增加,此即为肾小管对氨马尿酸排泌极量,用此量减去肾小球滤过量(以菊粉清除率测得),则可得到肾小管排泌对氨马尿酸的最大数值。

TmPAH成人正常值为60~90mg/分。因该测定方法繁琐,临床上不常采用。

(3)尿氨基酸测定 血中氨基酸经肾小球滤过,在近端肾小管内绝大部分被重吸收。如在同样饮食的情况下,患者尿中的氨基酸排出异常增多,则考虑为近端肾小管重吸收功能减退。此方法可以用氨基酸分析仪做尿中各种氨基酸的定量检查。

(4)尿中溶菌酶及β\textsubscript{2}
微球蛋白的测定 溶菌酶的分子量在14000~17000道尔顿,β\textsubscript{2}
微球蛋白的分子量为11800道尔顿,二者均为小分子量的蛋白质,均可经肾小球自由滤过,并且它们的绝大部分都在近端肾小管被重吸收,所以在尿中的含量甚微。正常人尿内的溶菌酶<3μg/ml,尿β\textsubscript{2}
微球蛋白<0.2μg/ml。如果血中的含量正常,而尿中的含量增多,则说明近端肾小管的重吸收功能受损。

\subsubsection{远端肾小管功能如何评价?}

远端肾小管、髓袢和集合管共同组成了远端肾单位,其作用是保持机体的内环境相对稳定,并且在决定最终尿液的质和量方面,起了非常重要的作用。临床上通常应用以下方法检查远端肾小管功能。

(1)尿比重 尿比重反映单位容积的尿中溶质的质量,它既受溶质克分子浓度影响,又受溶质克分子量影响。因此,蛋白质、糖、矿物质、造影剂都可使尿比重升高。蛋白对比重的影响是10g/L尿可增加比重0.003,糖对比重的影响是10g/L尿可增加尿比重0.004。

正常人24小时总尿比重为1.015~1.030。单次最高与最低尿比重之差应>0.008,而且必须有一次尿比重>1.018。如果患者的尿比重持续在1.010左右,称为固定低比重尿,说明肾小管浓缩功能极差,尿比重的测定方法极为简单易行,但应注意的是尿内的糖、蛋白质均可影响尿比重。

(2)尿浓缩稀释试验 尿浓缩试验是观察机体在缺水的情况下远端肾小管浓缩水的能力。实施方法简单且较为敏感,通过准确测量尿比重就可以了解远端肾小管的浓缩功能。具体方法是:试验前日晚6时饭后禁食禁饮,睡前排尿,夜尿也弃去。试验日晨6、7、8点各留尿1次,正常人此3次尿的标本中至少有1次尿比重在1.026以上(老年人可能在1.020),如果尿比重<1.020则表示肾浓缩功能差,实际上是反映了远端肾小管的功能。尿的稀释功能的测定亦反映远端肾小管的功能,但因需要在较短的时间内大量饮水,可引起不良反应、甚至发生水中毒,又受肾外因素影响较多,故不够敏感,临床上目前已经很少采用。

(3)尿渗透压的测定 尿渗透压是反映单位容积尿中溶质分子和离子的颗粒数。单位为mOsm/(kg·H\textsubscript{2}
O),它仅与溶质克分子浓度相关,并不受溶质分子量的影响。最常采用冰点下降法,以mmol/kg·H\textsubscript{2}
O为单位来表示尿糖10g/L可使渗透压增加60mmol/(kg·H\textsubscript{2}
O),而蛋白对渗透压影响较小。正常情况下24小时尿渗透压与尿比重的关系是:渗透压40mmol/(kg·H\textsubscript{2}
O)=比重0.001。

成人普通膳食时每日从尿中排出600~700mOsm的溶质,因此,24小时尿量为1000ml时,尿渗透压约600mOsm/(kg·H\textsubscript{2}
O)。24小时尿量为1500ml时,尿渗透压约400mOsm/(kg·H\textsubscript{2}
O)。总之,尿渗透压均应高于血渗透压。在禁食水8小时后的晨尿渗透压应>700~800mOsm/(kg·H\textsubscript{2}
O)。尿中的蛋白质含量对渗透压的影响较小,但尿糖却可以使尿的渗透压明显增加。

(4)无溶质水清除率(自由水清除率) 自由水清除率是单位时间(1分钟或1小时)从血浆中清除到尿中不含溶质的水量。正常人由于排出的均为含有溶质的浓缩尿,所以无溶质水清除率为负值。正常人在禁水8小时后晨尿无溶质水清除率是-25~-120ml/小时。无溶质水清除率可用于了解远端肾小管浓缩功能状态。在急性肾小管坏死的患者,无溶质水清除率常为正值。在其恢复过程中,可以作为追踪观察了解肾小管恢复情况的指标,也可用作发现移植肾早期排异的监测项目。

\subsubsection{肾血流量如何测定?有何临床意义?}

肾血流量或肾血浆流量是指单位时间内流经肾脏的血浆量。整个肾脏血流量在肾皮质是4~6ml/(g·分),在肾髓质外层是1ml/g·分,肾乳头为0.1~0.4ml/(g·分)。肾脏的氧需并不很高,主要集中于近曲小管和亨氏袢的升支粗段。监测肾脏血流量有助于了解肾脏的灌注情况,但临床上很少应用。

(1)染料稀释法 应用Indocyanine等染料注入肾动脉,然后从肾静脉内取样,计算肾脏血流量。但肾动脉、静脉内插管可能会影响肾脏血流量。

(2)动脉造影及肾静脉造影 当急性肾衰竭是由血管意外引起时,必要时可应用动脉或静脉造影,观察肾脏动脉或静脉血流情况。该方法一般作为诊断,不能用作监测肾血流。

(3)热稀释法 经股静脉向肾静脉插入带热敏电阻的导管,导管头端置于肾静脉,可反复测定肾脏血流量。

(4)超声多普勒法 无创伤性的超声多普勒可用来评价肾动脉和肾静脉的开放性,但超声多普勒不能精确评价肾血流,其临床应用受到限制。

\subsubsection{影响血肌酐浓度的因素有哪些?有何临床价值?}

血肌酐和肌酐清除率是反映急性肾脏改变,特别是肾小球滤过率的重要临床指标。但急性肾衰竭时,许多因素影响血肌酐和肌酐清除率的结果,导致检验结果与肾功能改变并不同步。因此,正确评价血肌酐和肌酐清除率,对急性肾衰竭的早期诊断和防治具有重要意义。

血肌酐浓度由机体肌酐的生成量、分布容量及排泄量3方面的因素决定。

(1)肌酐生成明显增加 急性肾衰竭时肌酐生成明显增加,主要包括以下来源:①机体处于高分解状态,蛋白分解,内源性肌酐生成明显增加;②患者营养支持,摄入蛋白类食物或静脉营养输注氨基酸等,使外源性肌酐生成增加;③合并感染的患者,感染加重机体高分解状态和负氮平衡,亦增加内源性肌酐的生成。机体肌酐生成增加,往往导致血肌酐浓度增加。

(2)肌酐分布容积增加 急性肾衰竭导致机体水钠潴留,细胞外液增加,使肌酐的分布容积增加,结果导致血肌酐浓度降低。因此,分布容积增加时,血肌酐浓度正常,并不意味着肾脏具有正常的滤过功能,往往会掩盖肾脏功能的降低。

(3)肌酐的排泄 正常情况下肾小管对肌酐不吸收,也很少分泌排泄,但在急性肾衰竭肾小球滤过率降低的情况下,肾小管分泌排泄肌酐明显增加,而且肾小球滤过率降低越明显、血肌酐浓度越高,肾小管分泌越多。当肾小球滤过率降低到15ml/分以下时,尿肌酐中有50%以上的肌酐并非是肾小球滤过的,而是由肾小管分泌排泄的。也就是说,测定获得的肌酐清除率比实际肾小球滤过率要高得多(50%~100%)。

\subsubsection{急性肾衰竭时血肌酐改变与肾小球滤过率的关系如何?有何临床意义?}

虽然血肌酐和肌酐清除率与肾小球滤过率的改变是不同步的,但两者之间还是有规律可循。

(1)血肌酐改变与肾小球滤过率的关系 ①急性肾衰竭早期:首先肾小球滤过率迅速降低,并可能达到一个低水平的稳态,而血肌酐浓度缓慢升高。血肌酐升高的速度不仅与肾小球滤过率有关,还与肌酐的生成量及肌酐的分布容量有关。②急性肾衰竭中期:肌酐的生成速度接近肌酐的排泄速度,处于平衡状态,血肌酐浓度不再上升。③急性肾衰竭恢复期:肾小球滤过率在较短时间内恢复到一个稳定的水平上,但血肌酐浓度缓慢降低。

由此可见,血肌酐浓度的改变与肾小球滤过率的改变总是不同步的,而且血肌酐浓度的改变总是滞后于肾小球滤过率的改变。

(2)临床意义 肌酐生成与排泄处于非平衡状态时,血肌酐浓度无法反映患者的肾小球滤过率。在急性肾损害早期,血浆肌酐的缓慢升高,并不意味着肾脏功能的进行性恶化,仅能提示肌酐生成与排泄尚未达到平衡。而在急性肾衰竭恢复期,血肌酐的缓慢降低也不表示肾脏功能的逐渐恢复,仍然仅提示肌酐生成与排泄尚未达到平衡。可见,血肌酐与肾小球滤过率并非呈线性关系,肾小球滤过率是血肌酐升高速度、基础肌酐浓度、肌酐的分布容积及肌酐排泄速度的复杂函数。

在急性肾损害早期(1~2天内),患者肾小球滤过率急剧降低,但临床上仅表现为血肌酐的轻微改变。当患者存在营养不良或限制营养支持时,血肌酐的升高速度会更为缓慢。这一结果提示以血肌酐作为急性肾衰竭的诊断依据,则会明显延误诊断,进而可能延误治疗。

只有当肌酐的生成和排泄处于平衡状态时,血肌酐浓度才能反映肾脏功能损害的程度。但一般来说,急性肾损害一周后,肌酐的生成和排泄才可能达到平衡。

总之,以血肌酐和肌酐清除率评价急性肾衰竭肾脏功能的改变,存在不少问题。应积极探索准确、可靠的早期肾脏功能评价指标,以指导急性肾衰竭的早期诊断和防治。

\subsection{急性肾衰竭的预防与治疗}

\subsubsection{如何预防肾毒性损害?}

在医院获得性急性肾衰竭中,至少有25%与一种或多种肾毒性损害有关。因此,避免肾毒性损害是医院获得性急性肾衰竭最重要的预防策略。防止肾毒性损害主要包括以下措施。

(1)避免使用具有明确肾毒性的药物 感染患者的抗生素选择,应尽可能避免使用具有明确肾脏毒性作用的氨基糖苷类抗生素。术后患者应用非甾体抗炎药的肾毒性并不高,但必须牢记的是,这类药物具有明显的收缩肾血管作用,可能引起肾脏损害,特别是对于全身性感染、心脏衰竭、肝硬化、肾病、血容量不足和低蛋白血症的重症患者,肾脏损害可能非常突出。

(2)药物的正确使用方法和适当剂量 许多药物肾毒性与剂量或血药浓度直接相关,采用正确使用方法和适当的剂量,是降低药物肾毒性的重要手段。氨基糖苷类抗生素、两性霉素B、放射造影剂等药物的剂量与肾毒性直接相关。严格、仔细的限制放射造影剂的剂量,是防止造影剂相关肾损害的最佳手段。氨基糖苷类抗生素的肾毒性与药物的谷浓度有关,而抗菌活性与药物峰值浓度有关,因此,氨基糖苷类抗生素的用药方法从以往的一日多次给药,改为一日1次给药,既可提高峰值浓度使抗菌作用增强,同时又使药物谷浓度降低,使药物的肾毒性降低。动物实验和临床研究均已证实这一效果。

(3)改善肾毒性药物的剂型 改变某些药物的剂型,可明显降低其肾脏毒性作用。放射造影剂和两性霉素B均具有强烈的肾毒性,如将放射造影剂改造为非离子性造影剂、将两性霉素B改造成两性霉素B脂质体后,两药的肾损害作用均明显降低。

(4)增加细胞外液容量和尿量 由于放射造影剂、两性霉素B、磺胺等药物易在肾小管内结晶,堵塞肾小管而损害肾功能。应用该类药物时,应特别注意适当增加细胞外容量,增加尿量,避免药物在肾小管内结晶而引起的肾损害。

(5)建立防止肾毒性损害的临床预警系统 建立防止肾毒性损害的临床预警系统也是防止肾毒性损害的重要手段。利用现代信息管理网络系统,将电子病历、实验室数据库、药物数据库联系在一起,建立肾毒性损害的临床预警系统。当患者的血清肌酐浓度有轻度升高或医师开出具有明显肾毒性药物时,系统将会报警,提醒临床医师给予充分的重视。

\subsubsection{抗菌药物导致的急性肾功能损害如何预防?}

重症患者应用氨基糖苷类药物导致肾功能损害的发生率高达10%。氨基糖苷类药物主要通过肾小球滤过,在肾小管中部分被重吸收,并积聚在小管上皮细胞溶酶体中,其肾损害主要与溶酶体破坏和小管上皮细胞膜损伤,导致小管细胞坏死有关。氨基糖苷类药物是否导致肾损害,不仅与肾小管中药物浓度与作用时间有关,还与治疗疗程、既往是否具有肾损害病史有关。

为降低肾毒性、预防急性肾衰竭的发生,氨基糖苷类药物的应用可遵循下列原则。

(1)延长给药间隔,降低药物谷浓度 氨基糖苷类药物是浓度依赖性抗菌药物,疗效主要与药物的峰浓度有关,而肾毒性主要与谷浓度有关。将药物的给药间隔延长,并不影响疗效,但有可能降低肾毒性。Olsen等
\protect\hyperlink{text00017.htmlux5cux23ch6-16}{\textsuperscript{{[}6{]}}}
的前瞻性观察了等剂量的妥布霉素不同给药间隔对重症医学科危重病患者肾脏功能的影响,与一日多次给药相比,每日一次给药组肾小管功能损伤明显减轻。对于肾功能正常的患者应用氨基糖苷类药物的荟萃分析显示,将一日3次给药改为一日1次给药,急性肾损害的发生率降低13%。也有研究显示,一日1次给药的急性肾损害发生率可降低50%(40%对比20%)。根据重症患者的肾功能情况,尽可能的将氨基糖苷类药物的给药间隔延长,有可能预防急性肾衰竭的发生。

(2)适当缩短疗程 氨基糖苷类药物疗程<5天很少发生急性肾损害。临床研究显示,作为联合用药之一的氨基糖苷类药物,疗程>5天与5天相比,疗效并不会进一步改善,故在美国胸科学会和感染病学会(ATS/IDSA)医院获得性肺炎治疗指南中,明确推荐氨基糖苷类药物应用5天后可停用。

(3)碱化尿液 存在肾损害高危因素的患者,应用氨基糖苷类药物时,可用碳酸氢钠碱化尿液,减少肾小管对药物的吸收。

(4)监测血药浓度 有条件的情况下,监测氨基糖苷类药物的峰、谷浓度,以调整用药剂量。

另外,万古霉素也可导致急性肾功能损害,2011年美国感染病学会指南推荐在治疗严重耐甲氧西林的金黄色葡萄球菌感染时,万古霉素的血药谷浓度应维持在15~20μg/ml
\protect\hyperlink{text00017.htmlux5cux23ch6-16}{\textsuperscript{{[}6{]}}}
。但最近亦有研究发现,当万古霉素血药浓度>15μg/ml时,肾功能损害发生的几率增加3倍
\protect\hyperlink{text00017.htmlux5cux23ch7-16}{\textsuperscript{{[}7{]}}}
。因此,重症患者应用万古霉素时,应监测血药浓度,同时需密切监测与评估患者肾功能。

\subsubsection{如何预防造影剂诱导急性肾损害?}

造影剂诱导急性肾损害的机制包括:①造影剂在肾小管浓缩(常常浓缩100倍)引起渗透性利尿,通过管球平衡反馈导致肾小球滤过率明显降低,高渗性造影剂的刺激效应明显强于低渗或等渗造影剂;②造影剂刺激肾小球毛细血管痉挛,导致组织细胞缺氧;③造影剂诱导氧化应激性损伤,导致肾间质损伤,高渗性、离子型造影剂的肾损害发生率更高。

造影剂诱导急性肾损害应当积极预防,目前认为有效的措施包括:

(1)积极水化,促进造影剂的排泄 应用造影前后积极补充等渗生理盐水,一般在注射造影剂前静脉输注生理盐水250~500ml,之后12~24小时给予1~2L生理盐水。积极的水化可使冠状动脉造影患者的急性肾损害发生率从2%降至0.7%。

(2)碱化尿液 造影剂在酸性环境下易离子化,导致肾小管损伤。造影前1小时静脉输注碳酸氢钠2ml/kg,之后6小时每小时给予1ml/kg,可使重症患者急性肾损害的发生率从13.6%显著降至1.7%。

(3)预防性应用N-乙酰半胱氨酸
\protect\hyperlink{text00017.htmlux5cux23ch8-16}{\textsuperscript{{[}8{]}}}
 可预防造影剂造成的氧化应激性肾损伤。造影前静注N-乙酰半胱氨酸600~1200mg,造影后2天给予600~1200mg口服,2次/天,可使血流动力学不稳定的重症患者造影后急性肾损害的发生率明显降低,且N-乙酰半胱氨酸对急性肾损害的预防效应具有剂量依赖性。N-乙酰半胱氨酸适用于大量补充生理盐水或使用碳酸氢钠受限的重症患者。

对于已存在明显肾功能损害的重症患者,造影后立即做一次血液滤过,也能够显著的预防急性肾衰竭的发生。

\subsubsection{围手术期急性肾衰竭的预防应注意哪些问题?}

重大手术是急性肾衰竭的重要的危险因素,围手术期采取积极措施,有可能达到预防急性肾衰竭发生的目的(表\ref{tab11-4})。

\begin{table}[htbp]
\centering
\caption{医院获得性急性肾衰竭的预防策略}
\label{tab11-4}
\includegraphics{./images/Image00092.jpg}
\end{table}

(1)术前及术后使患者血流动力学处于理想状态 围手术期肾脏灌注减少是导致术后发生急性肾衰竭的重要原因,防止围手术期肾脏灌注降低,对预防肾衰具有重要意义。由于肾脏的灌注与全身血流动力学状态直接相关,围手术期使患者血流动力学处于理想状态,就有可能防止肾脏低灌注引起的缺血。若在患者实施大血管手术前,先放置肺动脉漂浮导管,监测患者的血流动力学参数,通过补充液体、血浆和全血,使患者处于较理想的血流动力学状态,手术后,同样根据监测结果,指导循环容量的管理,则术后急性肾衰竭患病率和病死率均明显降低。说明围手术期使患者血流动力学处于理想状态,有可能避免肾脏低灌注和缺血,达到防止急性肾衰竭发生的目的。

(2)增加氧输送 氧输送主要由心脏泵功能(心输出量)、动脉血氧饱和度和血红蛋白浓度3个因素决定,是了解和改善全身组织氧供的重要指标。提高氧输送是重症患者治疗的重要目标。通过增加氧输送,可能达到改善组织灌注,纠正组织缺氧的目的。在急性肾衰竭的防治中,围手术期,特别是手术后,通过提高氧输送,有可能达到避免肾灌注减少和肾脏缺血缺氧,防止急性肾衰竭的目的。目前探讨增加氧输送对急性肾衰竭的预防作用的临床研究结果并不一致。因此,增加氧输送对急性肾衰竭的预防作用仍需进一步研究探索。

\subsubsection{利尿剂与甘露醇在急性肾衰竭防治中有何地位?}

呋塞米是一种袢利尿剂,并具有轻度血管扩张作用,是急性肾衰竭治疗中最常用的利尿剂。

近年来认为,呋塞米在急性肾衰竭治疗中主要具有以下作用:①降低髓袢升支粗段的代谢,使之氧耗降低,避免上皮细胞损伤加重;②冲刷肾小管,清除管型和结晶等肾小管腔内阻塞物,保持肾小管通畅;③降低肾小管中血红蛋白、肌红蛋白的浓度,防止蛋白阻塞肾小管;④促进少尿型肾衰转变为多尿型肾衰。当然,肾衰由少尿型转变为多尿型后,液体管理和治疗较为容易,但并不改变肾衰的病程。

大剂量应用呋塞米有明显副作用,主要表现为耳毒性,但也有呋塞米加重造影剂相关急性肾衰竭的报道。另外,也有报道无尿患者反复大剂量应用呋塞米,导致容量负荷增加,引起肺水肿。

呋塞米的使用剂量应逐步增加。初始剂量20mg,1小时后无效,可静脉推注呋塞米40mg。1小时后如仍无效,则静脉注射呋塞米200mg,每小时1次,连用3次。尿量仍无明显增加,则可改为呋塞米持续静脉泵入,剂量为1~4mg/分,可持续使用2~3天。

甘露醇不但具有渗透性利尿作用,还具有清除细胞外氧自由基的作用。在肾移植中,甘露醇作为移植肾的保护剂。甘露醇在急性肾衰竭的防治中应用并不广泛。在挤压综合征引起肌红蛋白尿性急性肾衰竭中,早期应用甘露醇对急性肾衰竭具有治疗作用。其他病因引起的急性肾衰竭中,甘露醇无治疗作用。对于造影剂引起的急性肾衰竭,应用甘露醇反而加重急性肾衰竭。因此,甘露醇在急性肾衰竭的救治中不应常规应用。

\subsubsection{肾脏剂量的多巴胺在急性肾衰竭的防治中还有地位吗?}

一般认为,多巴胺具有选择性肾血管扩张和增加尿量的作用,肾脏剂量的多巴胺(小剂量多巴胺)在临床上被广泛用于急性肾衰竭的防治,但多巴胺上述作用缺乏充分的临床和实验研究证据。研究认为1μg/(kg·分)多巴胺具有肾脏血管扩张作用,而常规应用的剂量为3~5μg/(kg·分),主要表现为缩血管作用,并无血管扩张作用。在安慰剂对照的临床试验中,多巴胺并不能降低急性肾衰竭患者的病死率,而且也不能使透析时间缩短。虽然小剂量多巴胺能够增加患者的尿量,但并不增加肌酐清除率。

对于肾前性肾脏功能损害患者,小剂量多巴胺可通过正性肌力作用,增加心脏输出量,使肾脏灌注部分改善。但是,对这类患者应特别注意有效循环血量不足对肾脏灌注的影响,低灌注状态应及时纠正。否则,应用多巴胺早期或许尿量有所增加,但因有效循环血量和肾脏灌注不足,可导致肾脏损害进一步恶化。

临床研究显示对于肾脏功能轻度受损的重症患者(肌酐清除率70~80ml/分),多巴酚丁胺并不增加患者尿量,但明显增加肌酐清除率,而多巴胺增加尿量,并不增加肌酐清除率,提示多巴酚丁胺能够改善肾脏灌注,而多巴胺仅具有利尿作用。

多巴胺和多巴酚丁胺具有正性肌力作用,通过增加感染性休克和心衰患者的心输出量,改善器官组织灌注,其中肾脏的灌注也可部分改善。但是需注意以下问题:

(1)正性肌力药物结合液体复苏,将氧输送提高到超常水平(supernormal),并不能改善全身性感染及多器官功能障碍综合征患者的预后,提示多巴胺与多巴酚丁胺提高心输出量并不一定能够改善急性肾衰竭患者的预后。

(2)多巴胺的剂量过高将会导致肾脏血管痉挛,使肾脏灌注减少,进一步加重肾缺血和肾损伤。

总之,肾脏剂量的多巴胺并不能改善肾脏灌注。多数学者对多巴胺的肾脏保护作用持怀疑或否定观点。因此,在急性肾衰竭的防治中,肾脏剂量的多巴胺不应常规使用。

\subsubsection{心房利钠肽在急性肾衰竭治疗中可否改善预后?}

心房利钠肽(atrial natriure
ticpeptide,ANP)是近年来治疗急性肾衰竭有一定疗效的药物,主要作用包括:①扩张入球小动脉、收缩出球小动脉,使肾小球滤过率增加;②抑制肾小管对钠的重吸收,总的效应表现为尿量增加。

在动物实验中,ANP能够明显改善缺血性和肾毒性因素引起的急性肾衰竭,甚至在肾脏缺血和肾毒性损害2天内用药,也能改善急性肾衰竭。临床研究初步显示ANP对急性肾衰竭有明显疗效。包括53例急性肾衰竭患者的一个开放性研究显示,应用ANP后,患者肾小球滤过率提高1倍,而需要透析治疗的患者减少了50%。一项多中心随机对照双盲临床试验纳入504例急性肾衰竭的重症患者,结果显示ANP对21天的患者生存率、病死率和血浆肌酐水平无明显影响,ANP治疗组患者21天生存率为43%,对照组为45%。但对其中120例少尿型急性肾衰竭患者进行亚组分析发现,ANP治疗组(60例)患者21天生存率为27%,而对照组(60例)为8%(\emph{P}
=0.008)。可见,ANP能够明显降低少尿型急性肾衰竭患者病死率。另外,有报道认为ANP能够将少尿型急性肾衰竭转变为非少尿型急性肾衰竭,ANP能够减轻肾脏的缺血再灌注损伤
\protect\hyperlink{text00017.htmlux5cux23ch8-16}{\textsuperscript{{[}8{]}}}
,这可能是ANP改善急性肾衰竭患者预后的原因。

总之,ANP可能是能够改善急性肾衰竭预后,并能将少尿型肾衰转变为多尿型肾衰的有效药物之一,值得临床医师重视。ANP具体使用方法是0.2μg/(kg·分)持续静脉泵入,至少连续使用24小时,并根据疗效进行调整。

\subsubsection{胰岛素样生长因子-1是否可用于急性肾衰竭的治疗?}

胰岛素样生长因子(Insulin-like Growth
Factor,IGF)-1也是近年来治疗急性肾衰竭的试验性药物之一。IGF-1在发育的肾脏中具有极高的浓度,其主要作用是刺激细胞增殖和分化。理论上,IGF-1能够促进急性肾衰竭后的损伤细胞功能修复。

在急性肾脏损害的动物模型中,肾脏损伤后24小时给予IGF-1,动物的肾脏损害明显改善。在狗肾移植模型中,IGF-1能够明显防止肾移植后的肾脏损害。最近的研究发现,大鼠动物模型中刺激IGF-1生成,亦可减少肾脏缺血再灌注损伤
\protect\hyperlink{text00017.htmlux5cux23ch9-16}{\textsuperscript{{[}9{]}}}
。提示IGF-1可能能够改善急性肾衰竭患者的预后。但进一步的临床研究并未发现对手术、创伤、低血压、全身性感染等原因导致的急性肾衰竭患者的肾功能、需要透析的比例以及病死率有明显改善作用。目前IGF-1在急性肾衰竭的治疗仍未进入临床,仍需进一步的研究探讨其机制,评价临床疗效。

\subsubsection{发生急性肾衰竭的重症患者代谢有何异常?}

近年来,通过临床和实验研究,人们对重症患者的能量代谢和蛋白质代谢有了较为深入的认识。急性肾衰竭合并多器官功能障碍综合征的患者不但具有一般重症患者的应激代谢反应,还具有其特殊的改变,例如,液体过负荷、肺水肿、代谢性酸中毒、电解质紊乱等。急性肾衰竭的代谢改变主要表现为以下几方面:

(1)内分泌状态的改变 急性应激状态下的内分泌改变主要表现为胰岛素释放增加,同时胰高血糖素、儿茶酚胺、皮质醇等胰岛素拮抗激素释放明显增加,结果导致以血糖增高为主要表现的“胰岛素抵抗状态”,这是急性应激患者的主要代谢改变。急性应激状态下,还常常出现低T3综合征和睾酮水平降低,但胰岛素样生长因子(IGF)常常升高。急性肾衰竭引起的机体应激状态亦可引起上述改变,但急性肾衰竭本身对大多数激素的改变无明显影响。

(2)能量代谢 应激使重症患者的能量代谢明显增加,常常成为“高代谢状态”,但这种“高代谢状态”常常被过高的估计。近年来,通过代谢车在床边常规开展能量代谢测定以来,发现处于应激状态的重症患者的能量消耗,仅比预计的静息能量消耗高20%~30%。

单纯急性肾衰竭患者的能量消耗与正常健康人类似。但对于合并多器官功能障碍综合征的患者或处于应激状态的患者,其能量消耗较预计的静息能量消耗高15%~20%。另外,间歇性血液透析可使代谢率增加15%~30%。

当然,对于严重程度类似的重症患者,合并急性肾衰竭者的能量消耗要略低于非急性肾衰竭患者,其原因主要与急性肾衰竭导致肾脏的能量消耗减少有关(正常肾脏占体重的0.5%,占全身能耗的10%)。

(3)糖代谢 高血糖是最常见的代谢改变,是应激导致胰岛素抵抗的后果。从另一角度来看,血糖增加实际上是机体代偿机制的一部分,保证依赖于血糖浓度的组织代谢需要,如巨噬细胞、内皮细胞、免疫和炎症细胞等。

生理条件下,血糖升高导致胰岛素释放增加,使骨骼肌和脂肪组织对糖的摄取和利用增加。但在胰岛素抵抗状态下,骨骼肌和脂肪组织无法利用糖,而且还需分解氨基酸合成糖。

急性肾衰竭患者糖的氧化利用能力明显降低,糖代谢仅占全身代谢需要的23%,而正常健康人可达39%,慢性肾衰患者也可达到36%。

(4)脂肪代谢 应激及急性肾衰竭状态下,患者三酰甘油及含三酰甘油的脂蛋白的血浆浓度升高,而胆固醇,尤其高密度脂蛋白胆固醇浓度正常或降低。血浆三酰甘油浓度增加与极低密度脂蛋白浓度增加有关,主要与肝脏合成增加及外周脂蛋白脂酶和肝三酰甘油酶的活性降低(约50%)导致脂肪分解与清除率下降有关,即患者对脂肪的廓清能力降低。

尽管脂蛋白脂酶活性降低,但急性肾衰竭患者对外源性脂肪能够很好地代谢利用,加上脂肪具有较低呼吸商,因此,脂肪依然是急性肾衰竭患者的主要能量来源。

(5)蛋白代谢 应激状态及急性肾衰竭时,蛋白代谢受到很大影响。一方面,蛋白质分解代谢明显增强而合成下降;另一方面,蛋白质的代谢转换明显增加,主要表现为蛋白在器官之间转换和器官内的不同蛋白的转换,如骨骼肌蛋白分解增加,肝脏利用氨基酸合成炎症蛋白。氨基酸动力学研究也表明,骨骼肌内支链氨基酸分解增强,亦有研究认为,代谢性酸中毒可诱导肌肉蛋白溶解酶的基因转录(器官内不同蛋白的转换)。另外,代谢性酸中毒及透析治疗本身均可加剧净蛋白分解,最终导致氮丢失增加和负氮平衡。

导致急性肾衰竭的蛋白高分解状态的原因主要包括:①应激导致的激素状态改变,特别是胰岛素抵抗状态;②酸中毒激活蛋白代谢酶。

(6)微量元素和维生素的代谢 由于肾功能障碍,使机体对水分、尿素氮、肌酐及钾、镁、磷等排泄困难而造成水中毒、氮质潴留,高钾、高镁、高磷血症及代谢性酸中毒,机体内环境紊乱。

间歇性或持续性肾脏替代治疗可导致机体的许多营养成分、微量元素和维生素的丢失。硒和维生素E等抗氧化剂水平的降低,使机体处于低抗氧化状态。肾脏羟化酶活性降低,导致维生素D\textsubscript{3}
浓度降低。

\subsubsection{肾脏替代治疗对急性肾衰竭患者代谢有影响吗?}

(1)透析器/血滤器滤过膜对代谢的影响 血液透析器/血滤器滤过膜的生物相容性对机体的影响已受到广泛重视。对于生物相容性差的滤过膜,当血液通过滤器时,不但可激活补体,还可激活粒细胞和血小板,合成和释放细胞因子、蛋白酶等炎症性介质,可使急性肾衰竭的高分解状态进一步恶化。因此,选用生物相容性较好的滤过膜,实施肾脏替代治疗,具有明显的临床价值。

(2)肾脏替代治疗对营养物质的清除 实施肾脏替代治疗时,滤器不但清除尿酸、肌酐等代谢产物,同时也能清除葡萄糖、氨基酸等营养物质,因此,急性肾衰竭患者实施肾脏替代治疗时,应考虑到营养物质的清除问题。营养物质的浓度越高,被清除的量可能就越多。当然,营养物质的清除量不仅与血浆中营养物质的浓度有关,还与滤器的通透性有关。血液透析时,透析器孔径较小,主要通过弥散机制,清除小分子物质;相反,血液滤过或血液滤过透析时,血滤器孔径较大,主要通过对流机制,可清除较大分子量的溶质。因此,需根据肾脏替代治疗手段的不同,分析对营养物质的清除作用。

①对葡萄糖的清除 葡萄糖的分子量较小,可自由通过透析器膜和血滤器膜,因此,血液透析和血液滤过时,葡萄糖的丢失是类似的,为25~50g/天。但是血液透析滤过时,葡萄糖的丢失量可能更多,需予补充。

②对脂肪的清除 由于脂肪在循环中仅以脂蛋白的形式存在,或以与清蛋白结合形式(脂肪酸)存在,脂蛋白颗粒或清蛋白的分子量较大,均无法通过透析器或血滤器膜,因此,肾脏替代治疗时,不考虑脂肪的丢失。

③对氨基酸的清除 氨基酸的分子量较小,血液透析和血液滤过均能清除氨基酸。当使用高流量血液滤过时,氨基酸的丢失尤为显著。对于血滤期间接受静脉营养的患者,静脉营养中大约10%的氨基酸可能经血液滤过丢失。

\subsubsection{急性肾衰竭患者实施营养代谢支持治疗应如何选择营养途径?}

肠道功能基本正常的急性肾衰竭患者,应尽早开始胃肠营养支持,而对于无法利用肠道的患者,应在休克纠正后,立即给予肠外营养支持。

肠内营养支持应使用要素营养液,如爱伦多、能全力、安素等,能量密度为4.184kJ/ml。对于重症患者,可肠道内补充特殊的氨基酸-谷氨酰胺,以促进和改善肠道黏膜绒毛的功能。

对于肠道功能障碍的患者,可采用肠外营养,而对于肠道功能部分受限的患者,可采取肠外营养为主,辅以少量的肠内营养。即使是很少量的肠内营养液,也有助于刺激肠道蠕动,增加肠黏膜血流,改善肠内菌群和黏膜绒毛的功能。

\subsubsection{急性肾衰竭患者实施营养代谢支持治疗的注意事项}

(1)营养液的热量 不同疾病状态的能量消耗量不同,间接测能仪可使能量供给达到较理想水平及实现个体化,一般可在104.6~125.5kJ/(kg·天),但有学者推荐此类患者能量供给在75%静息能量消耗量即可。亦有认为,对透析患者可按125.5~146.4kJ/(kg·天)补充能量。

(2)非蛋白热量 糖、脂双能源可提供非蛋白质热量。临床研究显示,肾功能减退时,机体对外源性脂肪的清除率并未降低,表明对其有较好的耐受力。中长链脂肪乳剂血浆清除快,对糖代谢干扰小,但在其他方面并未显示出更大的优势。此外,血滤或透析同时输注脂肪乳剂,对于滤膜及透析效果并无影响。

肾移植术后患者,由于创伤与大剂量糖皮质激素应用,使葡萄糖耐量下降,血糖升高,甚至出现继发性糖尿病,故应适当限制碳水化合物摄入量。此外,糖皮质激素与环孢素A的作用可使血脂、尤其胆固醇升高。输注ω-3聚不饱和脂肪酸有降低炎症反应、提高移植物存活率作用。

由于肾衰竭时合并水潴留,须限制液体入量,而20%~30%的脂肪乳剂具有小体积提供高能量的优点,尤其是对于非透析的不能耐受较大容量肠外营养液的急性肾衰竭患者,可提高其能量的补充量。脂肪乳剂的热量补充量可达非蛋白热量补充量的40%~50%。

鉴于肾衰时脂肪清除能力下降,在输注脂肪乳剂时应常规进行血脂代谢方面的监测。

(3)氮的供给 在肾衰竭及应激状态下,机体对蛋白质的需要量也是增加的,但由于肾脏排泄障碍限制了蛋白的补充。目前认为,增加氮源的补充量有助于减少体内蛋白质分解及改善肾功能,特别是对于接受血透与血滤的患者蛋白质摄入>1.2g/(kg·天)才可达到氮平衡状态,但具体应根据代谢情况而定。对于未进行透析或血滤的患者应限制蛋白的入量,以免加重氮质血症。

在氮源的选择上,普遍认为宜以补充必需氨基酸为主以及酮类似物等。因为内源性的氮可由酮类似物经转氨作用合成非必需氨基酸而减少体内氮的积蓄。近年来亦有研究认为,输注氨基酸液中必需氨基酸与非必需氨基酸的组分对于肾衰竭的预后并无明显影响。

此外,氨基酸、葡萄糖、维生素与微量元素均可通过透析膜而部分滤出。持续血液滤过时氨基酸丢失明显,无糖透析时有少量葡萄糖丢失,而使用含糖透析液时,有35%~40%的葡萄糖被吸收入体内。脂肪与整蛋白不被滤出,维生素与微量元素的丢失量尚不清楚。所以,应根据透析的具体情况,确定提供的营养素种类及用量。

(4)电解质、微量元素和维生素的补充 应注意在补充能量及胰岛素、纠正酸中毒后,可使钠、钾、镁、磷向细胞内转运而使血浆浓度降低。肾衰竭使调节钙磷代谢的维生素D在肾脏的活化过程受影响,从而影响体内的钙磷代谢,引起骨钙丢失,故应注意钙与维生素D的补充。尤其肾移植术后,糖皮质激素的应用使钙的吸收减少、排出增加,有人认为此类患者每日钙的入量应达800~1200mg。因此,急性肾衰竭患者营养支持中,水、电解质与酸碱平衡的监测是非常重要的。

(5)营养液的容量 补充高浓度的葡萄糖液、氨基酸液与脂肪乳剂,从而减少营养液的总量,以免加重水中毒。

\subsubsection{早期请肾脏科会诊在重症病人急性肾衰竭治疗中有何作用?}

临床流行病学调查显示,肾脏科医师的早期会诊和协助处理,能够明显改善急性肾衰竭患者的预后。最近的研究亦证实,重症医学科急性肾衰竭患者中,高达62.3%的患者肾脏科会诊被延迟或超过48小时,而延迟会诊导致患者的重症医学科病死率明显增高(未延迟会诊组65.4%,延迟会诊组88.2%,P<0.001)
\protect\hyperlink{text00017.htmlux5cux23ch10-16}{\textsuperscript{{[}10{]}}}
。可见,早期邀请肾脏科会诊有助于改善急性肾衰竭患者的预后。

肾脏科会诊的延迟往往与临床医师对急性肾衰竭的认识不足有关。当患者血清肌酐浓度未达到4.5mg/dl或尿量高于400ml时,往往会认为患者肾功能基本正常。较低的血肌酐浓度可能与容量负荷过高引起血浆肌酐稀释及严重营养不良引起肌酐生成减少有关。但这一问题在重症医学科可获得较好的解决。重症医学科医师往往将急性肾衰竭看作是多器官功能障碍综合征的一部分,根据多器官功能障碍综合征的诊断标准,血清肌酐浓度高于2mg/dl就被认为发生肾衰,就会引起重症医学科医师的高度重视,而给予积极处理。

\subsubsection{急性肾衰竭进入多尿期治疗上应注意哪些问题?}

(1)早期 治疗原则为防止补液过多,注意适当补充电解质。

虽然尿量逐渐增多,但患者体内仍处于水中毒的高峰。大量排尿,水分来自于过剩的细胞外液。如果大量补液,势必造成循环负担过重,引起心功能不全、肺水肿,甚至脑水肿。必须防止补液过快、过多,更不可尿多少,补多少。原则上,补液按少尿期处理。当尿量>2000ml/d时,补液量=尿量的1/3~1/2+显性丢失。

如尿量增加不明显,不要立即停止使用多巴胺、呋塞米等药物。

多尿早期血尿素氮仍进行性升高,酸中毒也继续加重,并持续3~4天,应补充足够热量,减少蛋白摄入,给予蛋白合成激素,尽量缩短多尿早期的持续时间,使血尿素氮尽快下降。

由于氮质血症加重,仍可并发严重感染、消化道出血等并发症。如果发生消化道出血,应补充新鲜血,使血细胞比容达到25%左右、血红蛋白>60g/L。

由于大量利尿,应严密监测血电解质的变化,注意适当补充电解质。

(2)中期 治疗原则为适当补液,防止水电解质大量丢失。

此期尿量明显增加,可达4000~5000ml/天以上,甚至>10000ml。补液量应根据监测指标,大约为尿量的2/3。以后随尿量减少,逐渐使入量等于出量。

电解质补充非常重要。主要根据临床生化监测结果补充,临床生化监测有时需要4~6小时进行1次。原则上每1000ml尿量,可补充钾2~3g、补钠3~5g,并同时注意补充钙、镁、维生素等。

随着氮质血症的减轻,临床症状逐渐好转,消化道功能开始恢复,加之尿量增多,应尽早开始口服补充水电解质及热量,逐渐减少肠外营养。但仍需供给足够热量,以利于尿素氮下降,防止感染。

(3)后期 治疗原则为保持水平衡。

随着饮食恢复,应增加饮水,适当控制静脉入量。减少肠外营养,增加胃肠的热量摄入。

恢复期无需特殊治疗,应避免使用肾毒性药物。如必须使用,应根据血浆肌酐清除率适当调整药物使用剂量及给药时间。每1~2个月复查肾功能1次,持续1年以上。

\begin{center}\rule{0.5\linewidth}{\linethickness}\end{center}

参考文献

\protect\hyperlink{text00017.htmlux5cux23ch1-16-back}{{[}1{]}} .Bellomo
R,Ronco C,Kellum JA,et al.The ADQI workgroup:Acute renal failure
--- definition,outcome measures,animal models,fluid therapy and
information technology needs:the Second International Consensus
Conference of the Acute Dialysis Quality Initiative(ADQI).Group.Crit
Care.2004,8:R204-R212.

\protect\hyperlink{text00017.htmlux5cux23ch2-16-back}{{[}2{]}} .Bagshaw
SM,George C,Bellomo R.A comparison of the RIFLE and AKIN criteria for
acute kidney injury in criticallyill patients.Nephrol Dial
Transplant,2008,23(5):1569-1574.

\protect\hyperlink{text00017.htmlux5cux23ch3-16-back}{{[}3{]}} .Bagshaw
SM,George C,Bellomo R;ANZICS Database Management Committee.Early
acute kidney injury and sepsis:a multicentre evaluation.Crit
Care.2008,12(2):R47.

\protect\hyperlink{text00017.htmlux5cux23ch4-16-back}{{[}4{]}} .Lerolle
N,Nochy D,Guerot E,et al.Histopathology of septic shock induced
acute kidney injury:Apoptosis and leukocytic infiltration.Intensive
Care Med.2010,36:471.

\protect\hyperlink{text00017.htmlux5cux23ch5-16-back}{{[}5{]}} .Rivers
E,Nguyen B,Havstad S,et al.Early-directed therapy in the treatment
of severe sepsis and septic shock.N Engl J Med,2001,345:1368-1377.

\protect\hyperlink{text00017.htmlux5cux23ch6-16-back}{{[}6{]}} .Liu
C;Bayer A;Cosgrove SE,et al.Clinical practice guidelines by the
infectious diseases society of America for the treatment of
methicillin-resistant Staphylococcus aureus infections in adultsand
children.Clin Infect Dis.2011,52(3):e18-55.

\protect\hyperlink{text00017.htmlux5cux23ch7-16-back}{{[}7{]}} .Bosso
JA,Nappi J,Rudisill C,et al.Relationship between Vancomycin Trough
Concentrations and Nephrotoxicity:a Prospective Multicenter Trial
Antimicrob.Agents Chemother.2011,55(12):5475.

\protect\hyperlink{text00017.htmlux5cux23ch8-16-back}{{[}8{]}} .Koga
H,Hagiwara S,Kusaka J,et al. Human Atrial Natriuretic Peptide
Attenuates Renal Ischemia-Reperfusion Injury.J Surg Res.2010.

\protect\hyperlink{text00017.htmlux5cux23ch9-16-back}{{[}9{]}} .Harada
N,Zhao J,Kurihara H,Nakagata N,Okajima K.Stimulation of Fc gamma RI
on primary sensory neurons increases insulin-likegrowth factor-I
production,there by reducing reperfusion-induced renal injury in
mice.J Immunol.2010.185(2):1303-1310.

\protect\hyperlink{text00017.htmlux5cux23ch10-16-back}{{[}10{]}} .Ponce
D,Zorzenon CP,dos SNY,Balbi AL.Early nephrology consultation can
have an impact on outcome of acute kidneyinjury patients.Nephrol Dial
Transplant.2011.26(10):3202-3206.

\protect\hypertarget{text00018.html}{}{}


\chapter{呼吸系统疾病的药物治疗}

\section{急性气管-支气管炎}

急性气管-支气管炎(acute
tracheobronchitis)是由生物、物理、化学刺激或过敏等因素引起的急性气管-支气管黏膜炎症。多为散发,无流行倾向,年老体弱者易感。临床症状主要为咳嗽和咳痰。常发生于寒冷季节或气候突变时,也可由急性上呼吸道感染迁延不愈所致。

\subsection{病因}

\subsubsection{微生物}

病原体与上呼吸道感染类似。常见病毒为腺病毒、流感病毒(甲、乙)、冠状病毒、鼻病毒、单纯疱疹病毒、呼吸道合胞病毒和副流感病毒。常见细菌为流感嗜血杆菌、肺炎链球菌、卡他莫拉菌等,近年来衣原体和支原体感染明显增加,在病毒感染的基础上继发细菌感染亦较多见。

\subsubsection{物理、化学因素}

冷空气、粉尘、刺激性气体或烟雾(如二氧化硫、二氧化氮、氨气、氯气等)的吸入,均可刺激气管-支气管黏膜引起急性损伤和炎症反应。

\subsubsection{过敏反应}

常见的吸入致敏原包括花粉、有机粉尘、真菌孢子、动物毛皮排泄物;或对细菌蛋白质的过敏,钩虫,蛔虫的幼虫在肺内的移行均可引起气管-支气管急性炎症反应。

\subsection{临床表现}

起病较急,通常全身症状较轻,可有发热。初为干咳或少量黏液痰,随后痰量增多,咳嗽加剧,偶伴血痰。咳嗽、咳痰可延续2~3周,如迁延不愈,可演变成慢性支气管炎。伴支气管痉挛时,可出现程度不等的胸闷气促。

查体可无明显阳性表现;也可在两肺闻及散在干、湿啰音,部位不固定,咳嗽后可减少或消失。

\subsection{诊断}

根据病史、咳嗽和咳痰等呼吸道症状,两肺散在干、湿性啰音等体征,结合血象和X线胸片,可作出临床诊断。病毒和细菌检查有助于病因诊断。

\subsection{治疗}

\subsubsection{一般治疗}

多休息、多饮水,避免劳累,避免吸入粉尘及刺激性气体。

\subsubsection{对症治疗}
\paragraph{镇咳祛痰药物}

(1)右美沙芬:中枢性镇咳药物,可抑制延脑咳嗽中枢而产生镇咳作用,与可待因的镇咳效果相当,长期服用不会引起依赖性和耐受性。

(2)氨溴索:祛痰药,可促进呼吸道内黏液分泌物的排出及减少黏液的滞留,促进排痰改善呼吸道状况。同时促进肺表面活性物质的分泌,增加支气管纤毛运动,使痰液易于咳出。

(3)桃金娘油提取物:目前较常应用的口服祛痰药物,通过重建上、下呼吸道黏液纤毛清除系统的清除功能,从而稀化和碱化痰液,增强黏液纤毛运动,黏液移动速度显著增加,促进痰液排出。服用该药物后排痰次数有所增加。除此之外,标准桃金娘油还具有抗感染作用。该药物能清除呼吸时的恶臭气味,长期用药后呼吸道的慢性炎症可获得改善。

(4)棕色合剂:较为常用的兼顾止咳和化痰的药物。该药物为复方制剂,含有甘草流浸膏、复方樟脑酊、愈创木酚甘油醚。甘草流浸膏为保护性祛痰药;复方樟脑酊为镇咳药物;愈创木酚甘油醚为祛痰剂,能使呼吸道腺体分泌增加、痰液稀释,易于咳出。
\paragraph{支气管扩张药物}

对于支气管痉挛的患者可用茶碱类药物如氨茶碱、二羟丙茶碱以及β{2}
受体激动剂等解痉止喘。
\paragraph{发热的患者可用解热镇痛药进行对症处理}

\subsubsection{抗菌药物治疗}

有细菌感染证据时应及时使用。可以首选新大环内酯类、青霉素类,亦可选用头孢菌素类或喹诺酮类等药物。多数患者口服抗菌药物即可,症状较重者可经肌内注射或静脉滴注给药,少数患者需要根据病原体培养结果指导用药。

\section{肺炎}

肺炎(pneumonia)是指终末气道、肺泡和肺间质的炎症,可由病原微生物、理化因素、免疫损伤、过敏及药物所致。细菌性肺炎是最常见的肺炎,也是最常见的感染性疾病之一。在抗菌药物应用以前,细菌性肺炎对儿童及老年人的健康威胁极大,抗菌药物的出现及发展曾一度使肺炎病死率明显下降。但近年来,尽管应用强力的抗菌药物和有效的疫苗,肺炎总的病死率不再降低,甚至有所上升。

\subsection{流行病学}

20世纪90年代欧美国家社区获得性肺炎和医院获得性肺炎年发病率分别约为12/1000人口和(5~10)/1000住院患者,近年发病率有增加的趋势。门诊肺炎患者的病死率<1%,住院患者平均为12%,入住重症监护病房(ICU)者约40%。发病率和病死率高的原因与社会人口老龄化、吸烟、伴有基础疾病和免疫功能低下有关,如慢性阻塞性肺病、心衰、肿瘤、糖尿病、尿毒症、神经疾病、药瘾、嗜酒、艾滋病、久病体衰、大型手术、应用免疫抑制剂和器官移植等。此外,亦与病原体变迁、医院获得性肺炎发病率增加、病原学诊断困难、不合理使用抗菌药物导致细菌耐药性增加等有关。

\subsection{病因及发病机制}

引起肺炎的病原体主要有细菌、真菌、衣原体、支原体、立克次体、病毒等微生物,其中细菌性肺炎占全部肺炎的半数左右,在我国成人肺炎中约占80%。近年来,尽管新的高效抗生素不断被发现,人类非但没有消灭肺炎,反而由于病原体的变迁、人口老龄化、特定高危人群的增加(如重大手术、机械通气、器官移植、肿瘤放化疗及AIDS患者等)以及抗生素的不合理应用、耐药菌株的不断增加等因素而面临严峻的挑战。肺炎的发病取决于宿主和病原体两方面的因素。

\subsubsection{宿主防御功能障碍}

任何原因造成全身免疫功能和呼吸道局部防御功能受损都是发生肺炎的高危因素。在医疗机构外,肺炎的发病中上呼吸道感染、受凉、疲劳、醉酒等都是常见的诱因。老年人机体防御功能减退,是细菌性肺炎的好发人群;一些慢性疾病患者,如癌症、慢性阻塞性肺疾病、心衰、高血压、糖尿病、肾病等好发肺炎。久住ICU及应用广谱抗生素、糖皮质激素、免疫抑制剂、细胞毒性药物时可引起机体内菌群失调、免疫功能低下,也易发生肺炎。建立人工气道和机械通气可破坏呼吸道局部防御功能,可促发通气相关肺炎。

\subsubsection{病原体侵入下呼吸道}
\paragraph{吸入污染的空气}

患者咳嗽、打喷嚏、说话时口鼻溅出飞沫,将呼吸道中的病原体播散到空气中,携带病原体的空气、飞沫、尘粒经呼吸进入呼吸道中可引起感染。支原体肺炎常流行于学校等集体或家庭中,空气飞沫传播是主要传播途径。
\paragraph{误吸上呼吸道病原菌}

健康人熟睡时可能不同程度地吸入咽喉部分分泌物,但通常不至于发生感染性病变。当上呼吸道有机会致病菌或其他病原体大量繁殖,再加上昏迷、休克、多痰、气管插管甚至雾化吸入治疗等因素,易使病原体进入下呼吸道,这是医疗机构内肺炎发病的重要途径。
\paragraph{血行播散及直接蔓延}

机会致病菌或其他病原体亦可为身体其他部位的感染病灶,通过血源播散或直接蔓延而侵入肺部。

\subsection{临床表现和分类}

\subsubsection{临床表现}

新近出现的咳嗽、咳痰或原有呼吸道疾病症状加重,并出现脓性痰,伴或不伴胸痛;发热,血白细胞计数增多;肺实变体征和湿性啰音;胸部X线检查显示片状、斑片状浸润性阴影或间质性改变,伴或不伴胸腔积液。上述系肺炎的共同表现,需要指出的是医院获得性肺炎的临床表现往往不典型,如粒细胞缺乏、严重脱水患者并发医院获得性肺炎时X光检查可以阴性,卡氏肺孢子虫肺炎有10%~20%患者的X线检查结果完全正常。

\subsubsection{分类}
\paragraph{按解剖分类}

(1)大叶性肺炎:炎症沿肺泡孔向其他肺泡扩散,引起肺段或肺叶广泛实变,支气管一般不受累,故又称作肺泡性肺炎,多为原发改变。X线显示呈叶、段或片状分布阴影,伴或不伴有空洞。

(2)小叶性肺炎:炎症随气管、支气管分布一直累及双侧肺泡,又称为支气管肺炎,多继发于其他肺部疾病。X线显示沿肺纹理分布的不规则斑片阴影。

(3)间质性肺炎:炎症主要侵犯肺间质,多见于病毒、支原体和过敏因素,X线显示肺内网状条索样分布阴影。
\paragraph{按病因分类}

1)细菌性肺炎

(1)需氧革兰阳性球菌:常见的有肺炎链球菌、金黄色葡萄球菌、甲型溶血性链球菌等。

(2)需氧革兰阴性杆菌:常见的有肺炎克雷伯菌、铜绿假单胞菌、大肠埃希菌、变形杆菌、军团杆菌、流感嗜血杆菌等。

(3)厌氧菌:如棒状杆菌、梭状杆菌等。

2)真菌性肺炎

致病性真菌如组织胞质菌、皮炎芽生菌等,条件致病菌如假丝酵母菌属、隐球菌属、曲霉菌属等。卡氏肺孢子虫也是一种真菌,常在免疫力低下的宿主中引起肺炎,是获得性免疫缺陷综合征患者最常见的直接致死原因。

3)病毒性肺炎

病毒性肺炎多为病毒性上呼吸道感染向下蔓延所致,在非细菌性肺炎中占25%~50%,好发于冬春季节,儿童多见,其中以流感病毒最为常见。

4)非典型病原体肺炎

由嗜肺军团菌、肺炎支原体和肺炎衣原体等感染引起。
\paragraph{按获病方式分类}

(1)社区获得性肺炎(community acquired
pneumonia,CAP):是指在社会环境中所患的感染性肺实质炎症,包括具有明确潜伏期的病原体在医疗机构外感染而入院后平均潜伏期发病的肺炎。肺炎链球菌感染占40%~70%,其次为金黄色葡萄球菌等。

(2)医院获得性肺炎(hospital acquired
pneumonia,HAP):是指患者入院时不存在、也不处于感染潜伏期,而于入院48h后在医院内发生的肺炎。我国医院获得性肺炎发病率为1.3%~3.4%,是第一位的医院内感染,需氧革兰阴性杆菌感染占70%;其次为金黄色葡萄球菌等。

\subsection{治疗原则}

\subsubsection{抗感染治疗}
\paragraph{抗生素经验治疗和针对性治疗的统一}

根据病原微生物选择相应的抗微生物化学治疗(化疗)是肺炎治疗的原则。但微生物学诊断从采集到病原体的分离鉴定需要时间,且诊断的敏感性和特异性不高,为等待病原学诊断延迟初始抗病原微生物治疗会延误治疗时机,从而影响预后。另一方面肺炎的病原微生物以细菌最为常见,抗菌药物的发展使抗菌治疗足以覆盖可能的病原菌,获得治疗成功。因此,细菌性肺炎在获得病原学诊断前,应尽早开始经验性抗菌治疗。经验性治疗应当参考不同类型肺炎病原谱的流行病学资料结合患者具体的临床与影像学特征,估计最可能的病原菌,选择药物和制定治疗方案。在48~72h后对病情再次评价,根据治疗反应和病原学结果调整治疗方案。若病原学检查结果无肯定临床意义,而初始治疗有效则继续原方案治疗。若获得特异性病原学诊断结果,而初始经验治疗方案明显不足或有错,或治疗无反应,则应根据病原学诊断结合药敏试验结果选择敏感抗菌药物,重新拟定治疗方案,此即靶向治疗。所以经验治疗与靶向治疗是治疗过程中的两个不同阶段,是有机的统一。
\paragraph{熟悉抗生素的药理学知识是合理应用抗菌治疗的基础}

每种抗生素的抗菌谱、抗菌活性、药动学、药效学参数、组织穿透力以及在肺泡上皮衬液和呼吸道分泌物中的浓度、不良反应及药物经济学评价是正确选择药物和治疗方案的基础。近年来关于药动学/药效学(pharmacokinetics/pharmacodynamics,PK/PD)的理论对抗生素的临床应用有重要的指导意义。β-内酰胺类和大环内酯类(阿奇霉素除外)属时间依赖性杀菌药物,要求血药浓度高于最低抑菌浓度(minimal
inhibitory
concentration,MIC)的时间占给药间歇时间(T>MIC%)至少达到40%~50%,此类药物半衰期短,抗生素后效应时间短或无抗生素后效应,因此须按半衰期所折算的给药间歇时间每日多次给药,不能任意减少给药次数。氨基糖苷类和喹诺酮类药物属浓度依赖性杀菌药物要求血药峰值浓度与最低抑菌浓度之比(C{max}
/MIC)达到8~10倍,或药时曲线下面积(areas under the
curves,AUC)与最低抑菌浓度之比(AUC/MIC,即AUIC)在G{+}
球菌均达到30、G{-} 杆菌达100以上,才能取得预期临床疗效,避免产生耐药性。
\paragraph{结合本地区耐药情况,参考指南选择药物}

目前包括中国在内的许多国家都制定和颁布了社区获得性和医院获得性肺炎诊治指南,提供了初始经验性治疗的抗菌药物推荐意见。但在各国或同一国家的不同地区,耐药情况不同,因此肺炎的经验性抗生素选择应当结合本国、本地区细菌耐药的流行病学资料认真选择。以下是中华医学会呼吸病学分会《社区获得性肺炎诊断和治疗指南》(2006年修订版)针对部分人群CAP患者初始经验性抗感染治疗的建议(见表\ref{tab12-1})。

\begin{longtable}[]{p{4cm}p{4cm}p{4cm}}
    \caption{CAP患者初始经验性抗感染治疗建议}
    \label{tab12-1}\\
\toprule
\endhead
CAP人群 & 常见病原菌 & 抗菌药物\tabularnewline
\midrule
青壮年、无基础疾病患者 &
肺炎链球菌、肺炎支原体、流感嗜血杆菌、肺炎衣原体等 &
①青霉素类;②多西环素(强力霉素);③大环内酯类;④第一代或第二代头孢菌素;⑤呼吸喹诺酮类(如左氧氟沙星、莫西沙星等)\tabularnewline
老年人或有基础疾病患者 &
肺炎链球菌、流感嗜血杆菌、需氧革兰阴性杆菌、金黄色葡萄球菌、卡他莫拉菌等
&
①第二代头孢菌素(头孢呋辛、头孢丙烯、头孢克洛等)单用或联合大环内酯类;②β内酰胺类/β内酰胺酶抑制剂(如阿莫西林克拉维酸钾、氨苄西林舒巴坦)单用或联合大环内酯类;③呼吸喹诺酮类(如左氧氟沙星、莫西沙星等)\tabularnewline
需入院治疗但不必收住ICU的患者 &
肺炎链球菌、流感嗜血杆菌、混合感染、需氧革兰阴性杆菌、金黄色葡萄球菌、肺炎支原体、肺炎衣原体、呼吸道病毒等
&
①静脉第二代头孢菌素单用或联合静脉大环内酯类;②静脉呼吸喹诺酮类;③静脉β-内酰胺类/β-内酰胺酶抑制剂单用或联合静脉大环内酯类;④头孢噻肟、头孢曲松单用或联合静脉大环内酯类\tabularnewline
需入住ICU的重症患者(无铜绿假单胞菌感染危险因素)
&
肺炎链球菌、需氧革兰阴性杆菌、嗜肺军团菌、肺炎支原体、流感嗜血杆菌、金黄色葡萄球菌等
&
①头孢曲松或头孢噻肟联合静脉注射大环内酯类;②静脉注射呼吸喹诺酮类联合氨基糖苷类;③静脉注射β内酰胺类/β内酰胺酶抑制剂(如阿莫西林克拉维酸钾、氨苄西林舒巴坦)联合静脉注射大环内酯类;④厄他培南联合静脉注射大环内酯类\tabularnewline
需入住ICU的重症患者(有铜绿假单胞菌感染危险因素) & 上组常见病原体+铜绿假单胞菌 &
①具有抗假单胞菌活性的β内酰胺类抗生素(如头孢他啶、头孢吡肟、哌拉西林/他唑巴坦、头孢哌酮/舒巴坦、亚胺培南、美罗培南等)联合静脉注射大环内酯类,必要时还可同时联合氨基糖苷类;②具有抗假单胞菌活性的β内酰胺类抗生素联合静脉注射喹诺酮类;③静脉注射环丙沙星或左氧氟沙星联合氨基糖苷类\tabularnewline
\bottomrule
\end{longtable}

对重症、体弱和昏迷患者,仰卧位会增加胃食管反流和误吸的危险,若无禁忌证,患者均宜采用半卧位可显著减少患者发生胃内容物。

\subsubsection{支持治疗}
\paragraph{一般治疗}

患者应卧床休息,改善营养状况;同时,注意补充水分,维持水电解质和酸碱平衡。高热患者宜用物理降温,必要时可用退热药。
\paragraph{氧疗}

轻症患者无须氧疗,重症患者氧疗是综合治疗的有效措施之一。
\paragraph{雾化、湿化治疗}

由于呼吸道急慢性炎症气管分泌物比较多,有时痰液黏稠不易咳出,同时由于支气管痉挛等因素的存在,除保持呼吸道通畅外,保持呼吸道充分湿化亦是提高抗感染治疗效果的重要措施之一。常用的气溶胶雾化剂有支气管扩张剂、黏稠分泌物溶解剂和肾上腺皮质激素。支气管扩张剂多选用β{2}
受体兴奋剂、茶碱类;黏液溶解剂可选用碳酸氢钠(4%~5%)、α-糜蛋白酶;肾上腺皮质激素选用地塞米松、布地奈德等。
\paragraph{体位痰液引流}

\section{支气管哮喘}

支气管哮喘(bronchial
asthma,简称哮喘)是由多种细胞(如嗜酸性粒细胞、肥大细胞、T淋巴细胞、中性粒细胞、气道上皮细胞等)和细胞组分参与的气道慢性炎症性疾病。这种慢性炎症与气道高反应性相关,通常出现广泛多变的可逆性气流受限,并引起反复发作性的喘息、气急、胸闷或咳嗽等症状,常在夜间和(或)清晨发作、加剧,多数患者可自行缓解或经治疗缓解。哮喘如诊治不及时,随病程的延长可产生气道不可逆性缩窄和气道重塑。而当哮喘得到控制后,多数患者很少发作,严重哮喘发作则更少见。

\subsection{流行病学}

全球约有1.6亿患者。各国患病率不等,国际儿童哮喘和变应性疾病研究显示13~14岁儿童的哮喘患病率为30%以下,我国五大城市的资料显示同龄儿童的哮喘患病率为3%~5%。一般认为儿童患病率高于青壮年,老年人群的患病率有升高的趋势。成人男女患病率大致相同,发达国家高于发展中国家,城市高于农村。约40%的患者有家族史。

\subsection{病因}

哮喘的病因还不十分清楚,患者个体过敏体质及外界环境的影响是发病的危险因素。哮喘与多基因遗传有关,同时受遗传因素和环境因素的双重影响。许多调查资料表明,哮喘患者亲属患病率高于群体患病率,并且亲缘关系越近,患病率越高;患者病情越严重,其亲属患病率也越高。目前,哮喘的相关基因尚未完全明确,但有研究表明存在有与气道高反应性、IgE调节和特应性反应相关的基因,这些基因在哮喘的发病中起着重要作用。

环境因素中主要包括某些激发因素,如尘螨、花粉、真菌、动物毛屑、二氧化硫、氨气等各种特异和非特异性吸入物;感染,如细菌、病毒、原虫、寄生虫等;食物,如鱼、虾、蟹、蛋类、牛奶等;药物,如普萘洛尔、阿司匹林等。同时,气候变化、运动、妊娠等也可能是哮喘的激发因素。

\subsection{临床表现}

临床表现为发作性伴有哮鸣音的呼气性呼吸困难或发作性胸闷和咳嗽。严重者被迫采取坐位或呈端坐呼吸,干咳或咳大量白色泡沫痰,甚至出现发绀等,有时咳嗽可为唯一的症状(咳嗽变异型哮喘)。哮喘症状可在数分钟内发作,经数小时至数日,用支气管舒张药或自行缓解。某些患者在缓解数小时后可再次发作。在夜间及凌晨发作和加重常是哮喘的特征之一。有些青少年的哮喘症状表现为运动时出现胸闷、咳嗽和呼吸困难(运动性哮喘)。

发作时胸部呈过度充气状态,有广泛的哮鸣音,呼气音延长。但在轻度哮喘或非常严重哮喘发作时,哮鸣音可不出现。心率增快、奇脉、胸腹反常运动和发绀常出现在严重哮喘患者中。非发作期体检可无异常。

\subsection{诊断}

\subsubsection{诊断标准}

(1)反复发作喘息、气急、胸闷或咳嗽,多与接触变应原、冷空气、物理、化学性刺激、病毒性上呼吸道感染、运动等有关。

(2)发作时在双肺可闻及散在或弥漫性,以呼气相为主的哮鸣音,呼气相延长。

(3)上述症状可经治疗缓解或自行缓解。

(4)除外其他疾病所引起的喘息、气急、胸闷和咳嗽。

(5)临床表现不典型者(如无明显喘息或体征)应有下列三项中至少一项阳性:①支气管激发试验或运动试验阳性;②支气管舒张试验阳性;③昼夜PEF变异率≥20%。

符合(1)~(4)条或(4)(5)条者,可以诊断为支气管哮喘。

\subsubsection{支气管哮喘的分期及控制水平分级}

支气管哮喘可分为急性发作期和非急性发作期。
\paragraph{急性发作期}

急性发作期是指气促、咳嗽、胸闷等症状突然发生或症状加重,常有呼吸困难,以呼气流量降低为特征,常因接触变应原等刺激物或治疗不当所致。哮喘急性发作时程度轻重不一,病情加重可在数小时或数日内出现,偶尔可在数分钟内即危及生命,故应对病情做出正确评估,以便给予及时有效地紧急治疗。哮喘急性发作时严重程度可分为轻度、中度、重度和危重4级。
\paragraph{非急性发作期(亦称慢性持续期)}

许多哮喘患者即使没有急性发作,但在相当长的时间内仍有不同频度和(或)不同程度出现症状(喘息、咳嗽、胸闷等),肺通气功能下降。过去曾以患者白天、夜间哮喘发作的频度和肺功能测定指标为依据,将非急性发作期的哮喘病情严重程度分为间歇性、轻度持续、中度持续和重度持续4级,目前则认为长期评估哮喘的控制水平是更为可靠和有用的严重性评估方法,对哮喘的评估和治疗指导意义更大。哮喘控制水平分为控制、部分控制和未控制3个等级。

\subsection{治疗}

目前尚无特效的治疗方法,但长期规范化治疗可使哮喘症状能得到控制,减少复发乃至不发作。长期使用最少量或不用药物能使患者活动不受限制,并能与正常人一样生活、工作和学习。

\subsubsection{脱离变应原}

部分患者能找到引起哮喘发作的变应原或其他非特异刺激因素,立即使患者脱离变应原的接触是防治哮喘最有效的方法。

\subsubsection{药物治疗}

治疗哮喘药物主要分为两类。
\paragraph{缓解哮喘发作}

此类药物主要作用为舒张支气管,故也称支气管舒张药。

1)β{2} 肾上腺素受体激动剂(简称β{2} 激动剂)

β{2} 激激动剂主要通过激动呼吸道的β{2}
受体,激活腺苷酸环化酶,使细胞内的环磷酸腺苷(cAMP)含量增加,游离\ce{Ca^2+}
减少,从而松弛支气管平滑肌,是控制哮喘急性发作的首选药物。常用的短效β受体激动剂有沙丁胺醇(Salbutamol)、特布他林(Terbutaline)和非诺特罗(Fenoterol),作用时间约为4~6h。长效β{2}
受体激动剂有福莫特罗(Formoterol)、沙美特罗(Salmaterol)及丙卡特罗(Procaterol),作用时间为10~12h。长效β{2}
激动剂尚具有一定的抗气道炎症,增强黏液-纤毛运输功能的作用。不主张长效β{2}
受体激动剂单独使用,须与吸入激素联合应用。但福莫特罗可作为应急缓解气道痉挛的药物。肾上腺素、麻黄碱和异丙肾上腺素,因其心血管不良反应多而已被高选择性的β{2}
激动剂所代替。

用药方法可采用吸入,包括定量气雾剂(MDI)吸入、干粉吸入、持续雾化吸入等,也可采用口服或静脉注射。首选吸入法,因药物吸入气道直接作用于呼吸道,局部浓度高且作用迅速,所用剂量较小,全身性不良反应少。常用剂量为沙丁胺醇或特布他林MDI,每喷100μg,每日3或4次,每次1或2喷。通常5~10min即可见效,可维持4~6h。长效β{2}
受体激动剂如福莫特罗4.5μg,每日2次,每次1喷,可维持12h。应教会患者正确掌握MDI吸入方法。干粉吸入方法较易掌握。持续雾化吸入多用于重症和儿童患者,使用方法简单,易于配合。如沙丁胺醇5mg稀释在5~20mL溶液中雾化吸入。沙丁胺醇或特布他林一般口服用法为2.4~2.5mg,每日3次,15~30min起效,但心悸、骨骼肌震颤等不良反应较多。β{2}
激动剂的缓释型及控制型制剂疗效维持时间较长,用于防治反复发作性哮喘和夜间哮喘,为注射用药,用于严重哮喘。一般每次用量为沙丁胺醇0.5mg,滴速2~4μg/min,易引起心悸,只在其他疗法无效时使用。

2)抗胆碱药

吸入抗胆碱药如异丙托溴铵(Ipratropine
Bromide),为胆碱能受体(M受体)拮抗剂,可以阻断节后迷走神经通路,降低迷走神经兴奋性而起舒张支气管作用,并有减少痰液分泌的作用。与β{2}
受体激动剂联合吸入有协同作用,尤其适用于夜间哮喘及多痰的患者。可用MDI,每日3次,每次25~75μg或用100~150μg/mL的溶液持续雾化吸入。约10min起效,维持4~6h。不良反应少,少数患者有口苦或口干感。近年发展的选择性M{1}
、M{3}
受体拮抗剂如噻托溴铵作用更强,持续时间更久(可达24h),不良反应更少。

3)茶碱类

茶碱类除能抑制磷酸二酯酶,提高平滑肌细胞内的cAMP浓度外,还能拮抗腺苷受体;刺激肾上腺分泌肾上腺素,增强呼吸肌的收缩;增强气道纤毛清除功能和抗感染作用,是目前治疗哮喘的有效药物。茶碱与糖皮质激素合用具有协同作用。

口服给药:包括氨茶碱和控(缓)释茶碱,后者因其昼夜血药浓度平稳,不良反应较少,可维持较好的治疗浓度,平喘作用可维持12~24h,可用于控制夜间哮喘。一般剂量每日6~10mg/kg,用于轻度或中度哮喘。静脉注射氨茶碱首次剂量为4~6mg/kg,注射速度不宜超过0.25mg/(kg·min),静脉滴注维持量为0.6~0.8mg/(kg·h),日注射量一般不超过1.0g。静脉给药主要应用于重、危症哮喘。

茶碱的主要不良反应为胃肠道症状(恶心、呕吐)、心血管症状(心动过速、心律失常、血压下降)及尿多,偶可兴奋呼吸中枢,严重者可引起抽搐乃至死亡。最好在用药中监测血浆氨茶碱浓度,其安全有效浓度为6~15μg/mL。发热、妊娠、小儿或老年患者,以及患有肝、心、肾功能障碍及甲状腺功能亢进者尤须慎用。合用西咪替丁(甲氰咪胍)、喹诺酮类、大环内酯类药物等可影响茶碱代谢而使其排泄减慢,应减少用药量。
\paragraph{控制或预防哮喘发作}

此类药物主要治疗哮喘的气道炎症,亦称抗炎药。

1)糖皮质激素

由于哮喘的病理基础是慢性非特异性炎症,糖皮质激素是当前控制哮喘发作最有效的药物。主要作用机制是抑制炎症细胞的迁移和活化;抑制细胞因子的生成;抑制炎症介质的释放;增强平滑肌细胞β{2}
受体的反应性。糖皮质激素可分为吸入、口服和静脉用几种剂型。

(1)吸入剂型。吸入治疗是目前推荐长期抗炎治疗哮喘的最常用方法。常用吸入药物有倍氯米松(Beclometasone,BDP)、布地奈德(Budesonide)、氟替卡松(Fluticasone)、莫米松(Momethasone)等,后两者生物活性更强,作用更持久。通常需规律吸入1周以上方能生效。根据哮喘病情,吸入剂量轻度持续者一般每日(BDP或等效量其他皮质激素)为200~500μg,中度持续者一般每日500~1000μg,重度持续者一般每日>1000μg(不宜超过每日2000μg)(氟替卡松剂量减半)。吸入治疗药物全身性不良反应少,少数患者可引起口咽假丝酵母菌感染、声音嘶哑或呼吸道不适,吸药后用清水漱口可减轻局部反应和胃肠吸收。长期使用较大剂量(>每日1000μg)者应注意预防全身性不良反应,如肾上腺皮质功能抑制、骨质疏松等。为减少吸入大剂量糖皮质激素的不良反应,可与长效β{2}
受体激动剂、控释茶碱或白三烯受体拮抗剂联合使用。

(2)口服剂:有泼尼松(强的松)、泼尼松龙(强的松龙)。用于吸入糖皮质激素无效或需要短期加强的患者。起始每日30~60mg,症状缓解后逐渐减量至≤10mg/d,然后停用,或改用吸入剂。

(3)静脉用药:重度或严重哮喘发作时应及早应用琥珀酸氢化可的松,注射后4~6h起效,常用量为100~400mg/d,或甲泼尼龙(甲基泼尼松龙80~160mg/d)起效时间更短(2~4h)。地塞米松因在体内半衰期较长、不良反应较多,宜慎用,一般为10~30mg/d。症状缓解后逐渐减量,然后改口服和吸入制剂维持。

2)LT调节剂

通过调节LT的生物活性而发挥抗炎作用,同时具有舒张支气管平滑肌。可以作为轻度哮喘的一种控制药物的选择。常用半胱氨酸LT受体拮抗剂,如孟鲁司特(Montelukast)10mg每日1次,或扎鲁司特(Zafirlukast)20mg每日2次,不良反应通常较轻微,主要是胃肠道症状,少数有皮疹、血管性水肿、转氨酶升高,停药后可恢复正常。

3)其他药物

酮替酚(Ketotifen)和新一代组胺H{1}
受体拮抗剂阿司咪唑、曲尼斯特、氯雷他定对轻症哮喘和季节性哮喘有一定效果,也可与β{2}
受体激动剂联合用药。

\subsubsection{急性发作期的治疗}

急性发作的治疗目的是尽快缓解气道阻塞,纠正低氧血症,恢复肺功能,预防进一步恶化或再次发作,防止并发症。一般根据病情的严重程度进行综合性治疗。
\paragraph{轻度}

每日定时吸入糖皮质激素(200~500μg BDP),出现症状时吸入短效β{2}
受体激动剂,可间断吸入。效果不佳时可加用口服β{2}
受体激动剂控释片或小量茶碱控释片(每日200mg),或加用抗胆碱药如异丙托溴铵气雾剂吸入。
\paragraph{中度}

吸入剂量一般为每日500~1000μg BDP;规则吸入β{2}
激动剂或联合抗胆碱药吸入或口服长效β{2}
受体激动剂。亦可加用口服LT拮抗剂,若不能缓解,可持续雾化吸入β{2}
受体激动剂(或联合用抗胆碱药吸入),或口服糖皮质激素(每日<60mg)。必要时可用氨茶碱静脉注射。
\paragraph{重度至危重度}

持续雾化吸入β{2}
受体激动剂,或合并抗胆碱药;或静脉滴注氨茶碱或沙丁胺醇;加用口服LT拮抗剂。静脉滴注糖皮质激素如琥珀酸氢化可的松或甲泼尼龙或地塞米松(剂量见前)。待病情得到控制和缓解后(一般3~5d),改为口服给药。注意维持水、电解质平衡,纠正酸碱失衡,当pH值<7.2且合并代谢性酸中毒时,应适当补碱;可给予氧疗,如病情恶化缺氧不能纠正时,进行无创通气或插管机械通气。若并发气胸,在胸腔引流气体下仍可机械通气。此外,应预防下呼吸道感染等。

\subsubsection{哮喘非急性发作期的治疗}

一般哮喘经过急性期治疗症状得到控制,但哮喘的慢性炎症病理生理改变仍然存在,因此,必须制定哮喘的长期治疗方案。根据哮喘的控制水平选择合适的治疗方案。对哮喘患者进行哮喘知识教育和控制环境、避免诱发因素贯穿于整个治疗阶段。

由于哮喘的复发性以及多变性,需不断评估哮喘的控制水平,治疗方法则依据控制水平进行调整。如果目前的治疗方案不能够使哮喘得到控制,治疗方案应该升级直至达到哮喘控制为止。当哮喘控制维持至少3个月后,治疗方案可以降级。通常情况下,患者在初诊后1~3个月回访,以后每3个月随访1次。如出现哮喘发作时,应在2周~1个月内进行回访。对大多数控制剂来说,最大的治疗效果可能要在3~4个月后才能显现,只有在这种治疗策略维持3~4个月后,仍未达到哮喘控制,才考虑增加剂量。对所有达到控制的患者,必须通过常规跟踪及阶段性减少剂量来寻求最小控制剂量。大多数患者可以达到并维持哮喘控制,但一部分难治性哮喘患者可能无法达成同样水平的控制。

以上方案为基本原则,但必须个体化,联合应用,以最小量、最简单的联合,不良反应最少,达到最佳控制症状为原则。

\section{肺结核}

肺结核(pulmonary
tuberculosis)仍然是严重危害人类健康的主要传染病,是全球关注的公共卫生和社会问题,也是我国重点控制的主要疾病之一。从20世纪60年代起,结核病化学治疗已取代过去消极的“卫生营养疗法”,成为公认的控制结核病的主要武器,使新发现的结核病治愈率达到95%以上。但20世纪80年代中期以来,结核病出现全球性恶化趋势,大多数结核病疫情很低的发达国家发现结核病卷土重来,众多发展中国家的结核病疫情出现明显回升。结核病在许多国家和地区失控的主要原因:一方面是人免疫缺陷病毒感染的流行、多重耐药(至少耐异烟肼和利福平)结核分枝杆菌感染的增多、贫困、人口增长和移民等客观因素;另一方面则是由于缺乏对结核病流行回升的警惕性和结核病控制复杂性的深刻认识,误认为结核病问题已解决,因而放松和削弱对结核病控制工作的投入和管理等主观因素所致。

\subsection{流行病学}

\subsubsection{全球疫情}

全球有三分之一的人(约20亿)曾受到结核分枝杆菌的感染。结核病的流行状况与经济水平大致相关,结核病的高流行与国民生产总值的低水平相对应。世界卫生组织把印度、中国、俄罗斯、南非、秘鲁等22个国家列为结核病高负担、高危险性国家。全球80%的结核病例集中在这些国家。无疑这些国家的结核病控制将对全球的结核病形势产生重要影响。

\subsubsection{我国疫情}

当前我国的结核病疫情特点如下。

(1)高感染率:年结核分枝杆菌感染率为0.72%。全国有近半的人口,约5.5亿曾受到结核分枝杆菌感染,城市人群的感染率高于农村。

(2)高肺结核患病率:2000年活动性肺结核患病率、痰涂片阳性(简称涂阳)及或培养阳性(简称菌阳)肺结核患病率分别为367/10万、122/10万和160/10万,估算病例数分别约为500万、150万和200万。中青年患病多,15~59岁年龄段的涂阳肺结核患者数占全部涂片阳性患者的61.6%。

(3)死亡人数多:每年约有13万人死于结核病。

(4)地区患病率差异大:西部地区活动性肺结核患病率、涂片阳性肺结核和培养阳性肺结核患病率明显地高于全国平均水平,而东部地区低于平均水平。

\subsection{临床表现}

各型肺结核的临床表现不尽相同,但有共同之处。
\paragraph{呼吸系统症状}

(1)咳嗽咳痰:是肺结核最常见症状。咳嗽较轻,干咳或少量黏液痰。有空洞形成时,痰量增多,若合并其他细菌感染,痰可呈脓性。若合并支气管结核,表现为刺激性咳嗽。

(2)咯血:约1/3~1/2的患者有咯血。咯血量多少不定,多数患者为少量咯血,少数为大咯血。

(3)胸痛:结核累及胸膜时可表现胸痛,为胸膜性胸痛。随呼吸运动和咳嗽加重。

(4)呼吸困难:多见于干酪样肺炎和大量胸腔积液患者。
\paragraph{全身症状}

发热为最常见症状,多为长期午后潮热,即下午或傍晚开始升高,翌晨降至正常。部分患者有倦怠乏力、盗汗、食欲减退和体重减轻等。育龄女性患者可以有月经不调。

\subsection{结核病的化学治疗}

\subsubsection{化学治疗的原则}

肺结核化学治疗的原则是早期、规律、全程、适量、联合。整个治疗方案分强化和巩固两个阶段。

(1)早期:对所有检出和确诊患者均应立即给予化学治疗。早期化学治疗有利于迅速发挥早期杀菌作用,促使病变吸收和减少传染性。

(2)规律:严格遵照医嘱要求规律用药,不漏服,不停药,以避免耐药性的产生。

(3)全程:保证完成规定的治疗期是提高治愈率和减少复发率的重要措施。

(4)适量:严格遵照适当的药物剂量用药,药物剂量过低不能达到有效的血浓度,影响疗效和易产生耐药性,剂量过大易发生药物不良反应。

(5)联合:联合用药系指同时采用多种抗结核药物治疗,可提高疗效,同时通过交叉杀菌作用减少或防止耐药性的产生。

\subsubsection{化学治疗的生物学机制}
\paragraph{药物对不同代谢状态和不同部位的结核分枝杆菌群的作用}

结核分枝杆菌根据其代谢状态分为A、B、C、D四群。

(1)A菌群:快速繁殖,大量的A菌群多位于巨噬细胞外和肺空洞干酪液化部分,占结核分枝杆菌群的绝大部分。由于细菌数量大,易产生耐药变异菌。

(2)B菌群:处于半静止状态,多位于巨噬细胞内酸性环境中和空洞壁坏死组织中。

(3)C菌群:处于半静止状态,可有突然间歇性短暂的生长繁殖,许多生物学特点尚不十分清楚。

(4)D菌群:处于休眠状态,不繁殖,数量很少。抗结核药物对不同菌群的作用各异。

抗结核药物对A菌群作用强弱依次为异烟肼>链霉素>利福平>乙胺丁醇;对B菌群依次为吡嗪酰胺>利福平>异烟肼;对C菌群依次为利福平>异烟肼。随着药物治疗作用的发挥和病变变化,各菌群之间也互相变化。通常大多数结核药物可以作用于A菌群,异烟肼和利福平具有早期杀菌作用,即在治疗的48h内迅速发挥杀菌作用,使菌群数量明显减少,传染性减少或消失,痰菌阴转。这显然对防止获得性耐药的产生有重要作用。B和C菌群由于处于半静止状态,抗结核药物的作用相对较差,有“顽固菌”之称,杀灭B和C菌群可以防止复发,抗结核药物对D菌群无作用。
\paragraph{耐药性}

耐药性是基因突变引起的药物对突变菌的效力降低。治疗过程中如单用一种敏感药,菌群中大量敏感菌被杀死,但少量的自然耐药变异菌仍存活,并不断繁殖,最后逐渐完全替代敏感菌而成为优势菌群。结核病变中结核菌群数量愈大,则存在的自然耐药变异菌也愈多。现代化学治疗多采用联合用药,通过交叉杀菌作用防止耐药性产生。联合用药后中断治疗或不规律用药仍可产生耐药性。其产生机制是各种药物开始早期杀菌作用速度的差异,某些菌群只有一种药物起灭菌作用,是因菌群再生长期间和菌群延缓生长期药物抑菌浓度存在差异所造成的结果。因此,强调在联合用药的条件下也不能中断治疗,短程疗法最好应用全程督导化疗。
\paragraph{间歇化学治疗}

间歇化学治疗的主要理论基础是结核分枝杆菌延缓生长期。结核分枝杆菌接触不同的抗结核药物后产生不同时间的延缓生长期。如接触异烟肼和利福平24h后分别可有6~9d和2~3d的延缓生长期。药物使结核分枝杆菌产生延缓生长期,就有间歇用药的可能性,而氨硫脲没有延缓生长期,就不适于间歇应用。
\paragraph{顿服}

抗结核药物血中高峰浓度的杀菌作用要优于经常性维持较低药物浓度水平的情况。每日剂量1次顿服要比每日2次或3次分服所产生的高峰血浓度高3倍左右。临床研究已经证实顿服的效果优于分次口服。

\subsubsection{常用抗结核病药物}
\paragraph{异烟肼(Isoniazid,INH,H)}

异烟肼仍然是单一抗结核药物中杀菌力,特别是早期杀菌力最强者。INH对巨噬细胞内外的结核分枝杆菌均具有杀菌作用。最低抑菌浓度为0.025~0.05μg/mL。口服后迅速吸收,血中药物浓度可达最低抑菌浓度的20~100余倍,脑脊液中药物浓度也很高。用药后经乙酰化而灭活,乙酰化的速度决定于遗传因素。成人剂量每日300mg,顿服;儿童为每日5~10mg/kg,最大剂量每日不超过300mg。结核性脑膜炎和血行播散型肺结核的用药剂量可加大,儿童20~30mg/kg,成人10~20mg/kg。偶可发生药物性肝炎,肝功能异常者慎用,需注意观察。如果发生周围神经炎可服用维生素B{6}
(吡哆醇)。
\paragraph{利福平(Rifampicin,RFP,R)}

最低抑菌浓度为0.06~0.25μg/mL,对巨噬细胞内外的结核分枝杆菌均有快速杀菌作用,特别是对C菌群有独特的杀灭菌作用。INH与RFP联用可显著缩短疗程。口服1~2h后达血高峰浓度,半衰期为3~8h,有效血浓度可持续6~12h,药量加大持续时间更长。口服后药物集中在肝脏,主要经胆汁排泄,胆汁药物浓度可达200μg/mL。未经变化的药物可再经肠吸收,形成肠肝循环,能保持较长时间的高峰血浓度,故推荐早晨空腹或早饭前0.5h服用。利福平及其代谢物为橘红色,服后大小便、眼泪等为橘红色。成人剂量为每日8~10mg/kg,体重在50kg及以下者为450mg,50kg以上者为600mg,顿服。儿童每日10~20mg/kg。间歇用药为600~900mg,每周2或3次。用药后如出现一过性转氨酶上升可继续用药,加保肝治疗观察,如出现黄疸应立即停药。流感样症状、皮肤综合征、血小板计数减少多在间歇疗法出现。妊娠3个月以内者忌用,超过3个月者要慎用。其他利福霉素类药物有利福喷丁(Rifapentine,RFT),该药血清峰浓度(C{max}
)为10~30μg/mL,半衰期为12~15h。RFT的最低抑菌浓度为0.015~0.06μg/mL,比RFP低很多。上述特点说明RFT适于间歇使用,使用剂量为450mg~600mg,每周2次。RFT与RFP之间完全交叉耐药。
\paragraph{吡嗪酰胺(Pyrazinamide,PZA,Z)}

吡嗪酰胺具有独特的杀灭菌作用,主要是杀灭巨噬细胞内酸性环境中的B菌群。在6个月标准短程化疗中,PZA与INH和RFP联合用药是第3个不可或缺的重要药物。对于新发现初治涂阳患者PZA仅在前2个月使用,因为使用2个月的效果与使用4或6个月的效果相似。成人用药剂量为每日1.5g,每周3次用药为每日1.5~2.0g,儿童每日为30~40mg/kg。常见不良反应为高尿酸血症、肝功能损害、食欲不振、关节痛和恶心。
\paragraph{乙胺丁醇(Ethambutol,EMB,E)}

乙胺丁醇对结核分枝杆菌的最低抑菌浓度为0.95~7.5μg/mL,口服易吸收,成人剂量为每日0.75~1.0g,每周3次用药为每日1.0~1.25g。不良反应为视神经炎,应在治疗前测定视力与视野,治疗中密切观察,提醒患者发现视力异常应及时就医。鉴于儿童无症状判断能力,故不用。
\paragraph{链霉素(Streptomycin,SM,S)}

链霉素对巨噬细胞外碱性环境中的结核分枝杆菌有杀菌作用。肌内注射,每日量为0.75g,每周5次;间歇用药每次为0.75~1.0g,每周2或3次。不良反应主要为耳毒性、前庭功能损害和肾毒性等,严格掌握使用剂量,儿童、老人、孕妇、听力障碍和肾功能不良者要慎用或不用。

\subsubsection{统一标准化学治疗方案}

为充分发挥化学治疗在结核病防治工作中的作用,便于大面积开展化学治疗,解决滥用抗结核药物、化疗方案不合理和混乱造成的治疗效果差、费用高、治疗期过短或过长、药物供应和资源浪费等实际问题,在全面考虑到化疗方案的疗效、不良反应、治疗费用、患者接受性和药源供应等条件下,且经国内外严格对照研究证实的化疗方案,可供选择作为统一标准方案。实践证实,严格执行统一标准方案确能达到预期效果,符合投入效益的原则。
\paragraph{初治涂阳肺结核治疗方案(含初治涂阴有空洞形成或粟粒型肺结核)}

1)每日用药方案

(1)强化期:异烟肼、利福平、吡嗪酰胺和乙胺丁醇,顿服,2个月。

(2)巩固期:异烟肼、利福平,顿服,4个月。简写为2HRZE/4HR。

2)间歇用药方案

(1)强化期:异烟肼、利福平、吡嗪酰胺和乙胺丁醇,隔日1次或每周3次,2个月。

(2)巩固期:异烟肼、利福平,隔日1次或每周3次,4个月。简写为2H3R3Z3E3/4H3R3。
\paragraph{复治涂阳肺结核治疗方案}

1)每日用药方案

(1)强化期:异烟肼、利福平、吡嗪酰胺、链霉素和乙胺丁醇,每日1次,2个月。

(2)巩固期:异烟肼、利福平和乙胺丁醇,每日1次,4~6个月。巩固期治疗4个月痰菌未转阴,可继续延长治疗期2个月。简写为2HRZSE/4-6HRE。

2)间歇用药方案

(1)强化期:异烟肼、利福平、吡嗪酰胺、链霉素和乙胺丁醇,隔日1次或每周3次,2个月。

(2)巩固期:异烟肼、利福平和乙胺丁醇,隔日1次或每周3次,6个月。简写为2H3R3Z3S3E3/6H3R3E3。
\paragraph{初治涂阴肺结核治疗方案}

1)每日用药方案

(1)强化期:异烟肼、利福平、吡嗪酰胺,每日1次,2个月。

(2)巩固期:异烟肼、利福平,每日1次,4个月。简写为2HRZ/4HR。

2)间歇用药方案

(1)强化期:异烟肼、利福平、吡嗪酰胺,隔日1次或每周3次,2个月。

(2)巩固期:异烟肼、利福平,隔日1次或每周3次,4个月。简写为2H3R3Z3/4H3R3。

上述间歇方案为我国结核病规划所采用,但必须采用全程督导化疗管理,以保证患者不间断地规律用药。

\subsubsection{耐药肺结核}

耐药结核病,特别是耐多药结核病:至少耐异烟肼和利福平,和当今出现的超级耐多药结核病(extensive
drug resistarit or extreme drug
resistant,XDR-TB)。除耐异烟肼和利福平外,还耐二线抗结核药物,对全球结核病控制构成严峻的挑战。世界卫生组织估算全球MDR-TB约有100万例,治愈率低,病死率高(特别是发生在HIV感染的病例),治疗费用昂贵,传染危害大。我国为耐多药结核病高发国家之一,初始耐药率为18.6%,获得性耐药率为46.5%,初始耐多药率和获得性耐多药率分别为7.6%和17.1%。

制定MDR-TB治疗方案应注意:详细了解患者的用药史,尽量用药敏试验结果指导治疗,治疗方案至少含4种可能的敏感药物,药物至少每周使用6d。吡嗪酰胺、乙胺丁醇、氟喹诺酮应每天用药,二线药物根据患者耐受性也可每天1次用药或分次服用;药物剂量依体重决定;氨基糖苷类或卷曲霉素注射剂类药物至少使用6个月;治疗期在痰涂片和培养阴转后至少治疗18个月,有广泛病变的应延长至24个月;吡嗪酰胺可考虑全程使用。

MDR-TB治疗药物第1组药为一线抗结核药,依据药敏试验和用药史选择使用。第2组药为注射剂,如菌株敏感链霉素为首选,次选为卡那霉素和阿米卡星,两者效果相似并存在百分之百的交叉耐药;如对链霉素和卡那霉素耐药,应选择卷曲霉素。卷曲霉素和链霉素效果相似并有高的交叉耐药。第3组为氟喹诺酮类药,菌株敏感按效果从高到低选择是莫西沙星=加替沙星>左氧氟沙星>氧氟沙星=环丙沙星。第4组为口服抑菌二线抗结核药,首选为乙硫异烟胺/丙硫异烟胺,该药疗效确定且价廉,应用从小剂量250mg开始,3~5d后加大至足量。PAS也应考虑为首选,只是价格贵些。环丝氨酸国内使用较少。氨硫脲不良反应较多,因而使用受到限制。第5组药物,疗效不确定,只有当1~4组药物无法制定合理方案时,方可考虑。

MDR-TB治疗方案通常含两个阶段:强化期(注射剂使用)和继续期(注射剂停用),治疗方案采用标准代码,例如6Z-Km(Cm)-Ofx-Eto-Cs/12Z-Ofx-Eto-Cs,初始强化期含5种药,治疗6个月,注射剂停用后,口服药持续至少12个月,总疗期18个月。注射剂为卡那霉素(Km),也可选择卷曲霉素(Cm)。

预防耐药结核的发生最佳策略是加强实施DOTS策略,使初治涂阳患者在良好管理下达到高治愈率。另一方面加强对MDR-TB的及时发现和给予合理治疗以阻止其传播。

\subsection{其他治疗}

\subsubsection{对症治疗}

肺结核的一般症状在合理化疗下很快减轻或消失,无须特殊处理。咯血是肺结核的常见症状,在活动性和痰涂阳肺结核患者中,咯血症状分别占30%和40%。咯血处置要注意镇静、止血,患侧卧位,预防和抢救因咯血所致的窒息并防止肺结核播散。

一般少量咯血,多以安慰患者、消除紧张、卧床休息为主,可用氨基己酸、氨甲苯酸(止血芳酸)、酚磺乙胺(止血敏)、卡络柳钠(安络血)等药物止血。大咯血时先用垂体后叶素5~10IU加入25%葡萄糖液40mL中缓慢静脉注射,一般为15~20min,然后将垂体后叶素加入5%葡萄糖液按0.1IU/(kg·h)速度静脉滴注。垂体后叶素收缩小动脉,使肺循环血量减少而达到较好止血效果。高血压、冠心病、心衰患者和孕妇禁用。对支气管动脉破坏造成的大咯血可采用支气管动脉栓塞法。在大咯血时,患者突然停止咯血,并出现呼吸急促、面色苍白、口唇发给、烦躁不安等症状时,常为咯血窒息,应及时抢救。置患者头低足高45°的俯卧位,同时拍击健侧背部,保持充分体位引流,尽快使积血和血块由气管排出,或直接刺激咽部以咳出血块。有条件时可进行气管插管,硬质支气管镜吸引或气管切开。

\subsubsection{糖皮质激素}

糖皮质激素在结核病的应用主要是利用其抗炎、抗毒性作用。仅用于结核毒性症状严重者。必须确保在有效抗结核药物治疗的情况下使用。使用剂量依病情而定,一般用泼尼松口服每日20mg,顿服,1~2周,以后每周递减5mg,用药时间为4~8周。

\subsubsection{肺结核外科手术治疗}

当前肺结核外科手术治疗主要的适应证是经合理化学治疗后无效、多重耐药的厚壁空洞、大块干酪灶、结核性脓胸、支气管胸膜瘘和大咯血保守治疗无效者。

\subsection{肺结核与相关疾病}

\subsubsection{HIV/AIDS}

截至2004年底,全球共有HIV/AIDS约3940万例,其中2004年HIV新感染者约为490万例,因HIV/AIDS死亡者310万例。在HIV/AIDS死亡病例中,至少有1/3病例是由HIV/AIDS与结核的双重感染所致。HIV/AIDS与结核病双重感染病例的临床表现是症状和体征多,如体重减轻、长期发热和持续性咳嗽等,全身淋巴结肿大,可有触痛,肺部X线经常出现肿大的肺门纵隔淋巴结团块,下叶病变多见,胸膜和心包有渗出等,结核菌素试验常为阴性,应多次查痰。治疗过程中常出现药物不良反应,易产生获得性耐药。治疗仍以6个月短程化疗方案为主,可适当延长治疗时间,一般预后差。

\subsubsection{肝炎}

异烟肼、利福平和吡嗪酰胺均有潜在的肝毒性作用,用药前和用药过程中应定期监测肝功能。严重肝功能损害的发生率为1%,但约20%患者可出现无症状的轻度转氨酶升高,无须停药,但应注意观察,绝大多数的转氨酶可恢复正常。如有食欲不良、黄疸或肝大应立即停药,直至肝功能恢复正常。在传染性肝炎流行区,确定肝炎的原因比较困难。如肝炎严重,肺结核又必须治疗,可考虑使用2SHE/10HE方案。

\subsubsection{糖尿病}

糖尿病合并肺结核的发病率有逐年升高趋势。两病互相影响,糖尿病对肺结核治疗的不利影响比较显著,必须在控制糖尿病的基础上肺结核的治疗才能奏效。肺结核合并糖尿病的化疗原则与单纯肺结核相同,只是治疗期可适当延长。

\subsubsection{矽肺(硅沉着病)}

矽肺患者是并发肺结核的高危人群。近年来,随着矽肺合并肺结核的比例不断上升,Ⅲ期矽肺患者合并肺结核的比例可高达50%以上。矽肺合并结核的诊断强调多次查痰,特别是采用培养法。矽肺合并结核的治疗与单纯肺结核的治疗相同。Ⅰ期和Ⅱ期矽肺合并肺结核的治疗效果与单纯肺结核的治疗相同。药物预防性治疗是防止矽肺并发肺结核的有效措施,使用方法为INH每日300mg,6~12个月,可减少发病约70%。

\subsection{结核病控制策略与措施}

\subsubsection{全程督导化学治疗}

全程督导化疗是指肺结核患者在治疗过程中,每次用药都必须在医务人员的直接监督下进行,因故未用药时必须采取补救措施以保证按医嘱规律用药。督导化疗可以提高治疗依从性,保证规律用药,因而能够显著提高治愈率,降低复发率并减少死亡,能够使患病率快速下降并减少多耐药病例的发生,符合投入效益的原则。

\subsubsection{病例报告和转诊}

按《中华人民共和国传染病防治法》,肺结核属于乙类传染病。各级医疗预防机构要专人负责,做到及时、准确、完整地报告肺结核疫情。同时要做好转诊工作,转诊对象为肺结核、疑似肺结核患者。乡镇卫生院和没有能力进行X线诊断和痰结核分枝杆菌检查的医疗机构应将肺结核可疑症状者推荐到结核病防治机构进行检查。

\subsubsection{病例登记和管理}

由于肺结核病程较长、易复发和具有传染性等特点,必须长期随访,掌握患者从发病、治疗到治愈的全过程。通过对确诊肺结核病例的登记达到掌握疫情和便于管理的目的。通过病例登记,医务人员就能够在督促规律用药、按时复查、指导预防家庭内传染以及动员新发现患者的家庭接触者检查等方面采取主动措施。

\subsubsection{卡介苗接种}

迄今,卡介苗问世已80余年,在182个国家和地区约40多亿儿童接种了卡介苗。卡介苗接种的效果远不如脊髓灰质炎糖丸和牛痘在预防小儿麻痹和天花方面那么理想。目前新结核疫苗的研究正在积极进行之中。普遍认为卡介苗接种对预防成年人肺结核的效果很差,但对预防由血行播散引起的结核性脑膜炎和粟粒型结核有一定作用。新生儿进行卡介苗接种后,仍须注意采取与肺结核患者隔离的措施。

\subsubsection{预防性化学治疗}

主要应用于受结核分枝杆菌感染易发病的高危人群,包括HIV感染者、涂阳肺结核患者的密切接触者、肺部硬结纤维病灶(无活动性)、矽肺、糖尿病、长期使用糖皮质激素或免疫抑制剂者、吸毒者、营养不良者、35岁以下结核菌素试验硬结直径达15mm者等。常用异烟肼300mg/d,顿服6~8个月,儿童用量为4~8mg/kg,或利福平和异烟肼3个月,每日顿服或每周3次。效果经观察与对照组比较可减少发病60%~80%。在一些资金短缺的国家和地区应优先把资金用于涂阳肺结核患者的治疗和管理,而不开展预防性化学治疗。

(诸 慧)

\protect\hypertarget{text00020.html}{}{}


\chapter{低血压与休克}

低血压(hypotension)系指成人肱动脉血压低于90/60mmHg的一种生理或病理状态。

根据发病速度,低血压可分为急性低血压和慢性低血压两大类:

1.急性低血压 短时间内,血压由正常或较高水平迅速明显下降,称为急性低血压。临床主要表现为晕厥与休克两大临床综合征。

2.慢性低血压 慢性低血压而伴有症状者,主要见于下列情况:①体质性低血压;②体位性低血压;③其他:如餐后低血压、高山性低血压等。

\section{41 慢性低血压}

\subsection{一、体质性低血压(原发性低血压)}

发病原因不清楚,常见于体质较瘦弱者,女性较多,可有家族遗传倾向。

临床特点:①多无自觉症状,仅在体检中偶然发现低血压,此种状况多无重要临床意义;②部分患者则有精神疲倦、健忘、头晕、头痛、甚至晕厥,或胸闷、心悸等类似心脏神经症的表现,但往往这些症状常由于合并某些慢性疾病或营养缺乏所致。

本症诊断主要依据:①低血压;②心或血管神经症;③无器质性疾病或营养不良的表现;④排除其他原因所致的低血压。

\subsection{二、体位性低血压(直立性低血压)}

从平卧位或下蹲位突然转变为直立位,或长时间站立时发生低血压(<90/60mmHg),或收缩压降低超过30mmHg,舒张压降低超过20mmHg,称为体位性低血压,严重者可以引起脑缺血症状或晕厥,若取平卧位,血压回升,症状可消失。

体位性低血压可分为原发性和继发性两大类:

\subsubsection{(一)原发性体位性低血压}

亦称直立性低血压、Shy-Drager综合征,是一种以自主神经系统功能失调为主的综合征。本病病因未明,多数学者认为可能是自主神经功能失调,导致血压控制异常;也有认为是自主神经原发性变性(尤其是交感神经系统所致)。

原发性体位性低血压临床特点:①起病隐袭,多在中年以后发病,男性多于女性,病程缓慢。②直立位时血压迅速而显著降低。③患者直立位时出现脑缺血症状,轻者头晕、眼花、乏力,多在晨起、登高、行走、活动或站立排尿时发生;重者立即发生晕厥,晕厥发作前无面色苍白,恶心、出汗、心悸等先兆。④有自主神经受损害表现:皮肤干燥、少汗或无汗、排尿障碍、夜间多尿与遗尿、阳痿、腹泻或便秘等。⑤本病可能为中枢神经系统疾病,可有躯体神经症状:说话缓慢、写字手颤或笨拙、步态不稳、共济失调;肌张力增高、腱反射亢进、发音困难,病理神经反射阴性。

本病诊断:①中年男性,于直立位时渐发头晕、眼花、眩晕甚至突然发生晕厥。②血压测定试验阳性;测量患者平卧位和直立位血压,每分钟1次,连续3~5次,血压下降>30/20mmHg为阳性。有学者认为,患者直立位收缩压较卧位下降50mmHg,舒张压下降20~30mmHg,有肯定诊断价值。③排除其他原因,包括血管迷走神经性晕厥、排尿晕厥、颈动脉窦过敏、严重心律失常等,可诊断本病。

\subsubsection{(二)继发性体位性低血压}

是继发于其他疾病或可查明原因,大致有:

1.神经系统疾病
脑干及其周围炎症、缺血、肿瘤等使血管运动中枢受累;脊髓疾病如脊髓结核、脊髓横断性损伤、脊髓空洞、多发性神经炎、多系统萎缩。

2.内分泌及代谢疾病
Addison病、慢性垂体前叶功能减退症、甲状腺功能减退症、重症糖尿病、嗜铬细胞瘤等。

3.心血管疾病
如重度主动脉瓣狭窄、重度二尖瓣狭窄、慢性缩窄性心包炎、梗阻性肥厚型心肌病、多发性大动脉炎、高原病等。

4.慢性营养不良
吸收不良综合征、重度贫血、慢性胰腺炎、严重肝病、恶性肿瘤、血液病、尿毒症、活动性结核病等。

5.药物性
某些降压药、血管扩张剂、镇静药等,如硝酸酯类、胍乙啶,α-受体阻滞剂等。

6.其他方面 妊娠晚期、久病卧床患者等。

继发性体位性低血压的病因甚多,下面仅介绍几种疾病:

1.慢性肾上腺皮质功能减退症(Addison病)
系由于各种病因使双侧肾上腺的绝大部分被毁致肾上腺皮质激素分泌不足的临床综合征。肾上腺皮质功能减退症临床上可分为急性及慢性,慢性又可分为原发性和继发性。病因见表\ref{tab13-1}。

\begin{longtable}{c}
 \caption{肾上腺皮质功能减退症的病因学}
 \label{tab13-1}
 \endfirsthead
 \caption[]{肾上腺皮质功能减退症的病因学}
 \endhead
 \includegraphics[width=\textwidth,height=\textheight,keepaspectratio]{./images/Image00087.jpg}\\
 \includegraphics[width=\textwidth,height=\textheight,keepaspectratio]{./images/Image00088.jpg}
 \end{longtable}

原发性肾上腺皮质功能减退症的病因:①感染:肾上腺结核为本病常见病因,结核病灶破坏了肾上腺皮质和髓质;②自身免疫性肾上腺炎:双侧肾上腺皮质受损,约25\%患者血中可检出抗肾上腺的自身抗体;③少见其他病因:恶性肿瘤转移、淋巴瘤、白血病、淀粉样变性、双肾上腺手术,放疗破坏等,还有少见肾上腺白质营养不良症,属先天代谢异常疾病,本病也可有两型:儿童型及青年成年型。国外报道,自身免疫性慢性肾上腺皮质功能减退症占所有病例约80\%。近半数患者伴其他器官特异性自身免疫病,此称为自身免疫性多内分泌综合征,多见于女性(约占70\%)。

继发性肾上腺皮质功能不全的病因:①由于下丘脑病变,促肾上腺皮质激素释放激素不足;②由于垂体功能不足:如产后大出血所致希恩综合征(Sheehan
syndrome)、垂体或邻近组织的肿瘤、垂体血管栓塞或外伤、垂体切除、先天性垂体发育不全等;③下丘脑-垂体系统的功能被抑制所致,如过度使用外源性甾体类药物或促肾上腺皮质激素、肿瘤分泌甾体类物质、免疫抑制剂的应用等。

肾上腺皮质功能减退症的临床症状及特点见表\ref{tab13-2}:

(1)皮肤、黏膜色素加深及沉着,为本病最具特征性表现。皮肤暴露处、摩擦处、乳晕、瘢痕处色素加深更具明显,日晒后继续保持不退色;牙齿、舌部、颈、肛周等黏膜色素沉着明显。

(2)乏力为本病主要早期症状,其发生率高,程度与病情轻重成正比,应激状态下乏力症状加重。

(3)多系统及器官出现皮质激素不足表现:①胃肠道:纳差、喜咸食、胃肠功能紊乱;②神经精神系统:精神萎靡不振、疲劳,重者嗜睡、意识模糊、精神失常等;③心血管系统:低血压、部分有直立性晕厥、心脏缩小、心音低钝等;④代谢障碍:发生低血糖等;⑤肾脏:出现低钠表现;⑥性及生殖功能:性功能减退,男性阳痿,女性月经紊乱、闭经、阴毛及腋毛稀少或脱落;⑦合并全身疾病(如全身结核)或多器官自身免疫性疾病时则伴有相应疾病的表现。

(4)对感染、外伤、气候、劳累、情绪激动等各种应激的抵抗力差,易出现肾上腺危象。肾上腺危象为本病急骤加重的表现。其特点为:①常有应激状态(如上述感染、手术、分娩、创伤、过冷、过劳、失水或突然中断皮质激素治疗等);②临床症状加重:上述肾上肾皮质激素严重不足表现,如低血压或休克、严重失水、昏迷等;③抢救不及时,易休克、昏迷、死亡。

如怀疑或考虑本病,可进行下列实验检查。

(1)ACTH兴奋试验:最具诊断价值,Addison病者储备功能低下。

(2)24小时尿17-羟、17-酮类固醇测定:本病患者尿17-羟可接近正常。对本病有诊断意义。

(3)影像学检查:肾上腺区X线摄片、CT、MRI检查,可帮助诊断病因。

(4)血中嗜酸性粒细胞明显增多,血清钾增高、低钠、低氯、空腹低血糖或葡萄糖耐量试验曲线低平,均有助于诊断。

\begin{table}[htbp]
\centering
\caption{慢性肾上腺皮质功能减退症的症状发生率}
\label{tab13-2}
\includegraphics[width=5.91667in,height=1.79167in]{./images/Image00089.jpg}
\end{table}

2.腺垂体功能减退症
腺垂体功能减退症是指腺垂体(旧称垂体前叶)激素分泌减少,可以单个激素缺乏,也可以多种激素同时缺乏;腺垂体功能减退可以为原发(垂体病变),也可以继发(下丘脑病变,或下丘脑-垂体间的联系中断)。

腺垂体功能减退病因颇多,有:①垂体瘤:成年人最常见原因,也有其他恶性肿瘤转移至垂体;②下丘脑病变:如肿瘤、炎症、淋巴瘤或白血病浸润、结节病等;③垂体缺血坏死或纤维化:妊娠或产后的垂体功能减退症称希恩综合征,多发生前置胎盘、胎盘早期剥离、胎盘潴留、产后大出血;④下丘脑和垂体炎症与感染;⑤垂体瘤术后放疗及创伤,鼻咽癌放疗损伤下丘脑及垂体;⑥突然中断糖皮质激素;⑦垂体卒中、垂体瘤出血、瘤体增大压迫,临床呈急症危象。

本病临床特点:①本病多为女性(占95\%)20~40岁。②病情严重程度与垂体被毁程度有关。据临床资料,50\%以上腺垂体组织被破坏后才出现临床症状(轻度),75\%破坏时才有明显症状(中度),95\%破坏时可有严重垂体功能减退(重度)。③典型腺垂体功能减退主要表现为各靶腺(性腺、甲状腺、肾上腺)功能减退。

特别需要引起注意的是垂体功能减退性危象(简称垂体危象),在全垂体功能减退的基本表现下,在各种应激状态下均可引起或诱发垂体危象,临床上有各种类型表现:①高热型(>40℃);②低热型(<30℃);③低血压,循环衰弱型;④低血糖型;⑤中毒型;⑥混合型,突出表现为消化系统、循环系统、神经精神方面症状和表现,如高热、休克、恶心呕吐、谵妄、抽搐、昏迷等,甚至死亡。

本病需做实验检查,方可帮助诊断和判别诊断:

(1)腺垂体功能需检验其支配靶腺的功能来反映:①性腺功能测定:女性:血雌二醇水平降低,男性:血清睾丸酮水平降低或正常低值,精子量及质改变;②肾上腺皮质功能:24小时尿17-羟皮质醇,OGTT;③甲状腺功能测定;④腺垂体分泌激素测定。

(2)CT或MRI检查可辨别腺垂体及下丘脑病变,对非颅脑病变,也可用X线、CT及MRI或活检以判断原发病的病因。

本病需与多发性内分泌腺功能减退症、神经性厌食、失母爱综合征(与心理、社会因素有关)鉴别。

3.甲状腺功能减退症
甲状腺功能减退症简称甲减,是由于各种原因致低甲状腺激素血症或甲状腺激素抵抗而引起的全身性低代谢综合征,表现为黏液性水肿,成人原发性甲状腺功能减退症占成人甲减的90\%~95\%,其病因有:①自身免疫性甲状腺类;②手术或放射治疗破坏甲状腺;③过量碘摄入及抗甲状腺药物如锂盐、硫脲类等。

本病除有多系统及肌肉关节症状外,主要特征:①怕冷、易疲劳;②黏液性水肿伴高胆固醇血症;③低血压、心动过缓;④基础代谢率低;⑤血清甲状腺激素FT\textsubscript{4}
降低、TSH增高。

4.多系统萎缩症
多系统萎缩症是一类原因未明,临床表现为锥体外系、锥体系、小脑和自主神经等多系统损害的中枢神经系统变性疾病。

本病临床表现千变万化,包括锥体系、锥体外系、小脑和自主神经等多个系统的损害,最多见的组合为震颤麻痹合并小脑症状。其中41\%患者以自主神经障碍作为首发症状出现,主要表现为阳痿(男)、尿失禁(女)、直立性低血压、少汗、面色苍白、便秘等。高场强MRI对该病有较大的诊断价值。

本病需与帕金森病、家族性橄榄-桥脑-小脑萎缩、老年性直立性低血压鉴别(常由低血容量性、药物性、排尿性等低血压反应诱发)。

5.餐后低血压 餐后低血压(postprandial
hypotension,PPH)的定义为进餐后2小时内收缩压下降≥20mmHg或餐前收缩压≥100mmHg,而餐后收缩压<90mmHg;若进餐后收缩压下降幅度虽未达到上述标准,但超过脑血流自身调节能力出现低血压症状,如头晕、晕厥等,也可诊断为PPH。PPH是一种老年人常见的疾病,近年来与其相关的文献报道逐渐增多,发生机制尚不是十分清楚,一般认为,PPH的发生与压力感受器反射灵敏度下降、餐后交感神经活性反应下降及餐后体液改变等因素有关。

6.高山性低血压
平地人到海拔3500米以上地区时,可出现血压偏低。患者常有头晕、嗜睡、记忆力减退、全身无力、疲倦、纳差等。经过一段时间适应或离开高山地区,症状即可消失,恢复正常。

\protect\hypertarget{text00116.html}{}{}

\section{42 休克}

休克(shock)是一种危急的临床综合征,是由各种原因引起全身有效循环血容量急剧减少,导致全身微循环功能障碍,使脏器的血流灌注不足或严重障碍,引起缺血、缺氧、代谢障碍与细胞受损的病理生理综合征。患者表现有:①血压下降;②精神神经症状:头晕、乏力、神志淡薄或烦躁不安、嗜睡或昏迷等;③周围器官灌注不足表现:皮肤苍白、四肢湿冷、脉搏快而弱,甚至摸不到,少尿或无尿等一系列症状。

在临床上,休克按病因分类可分为:①低血容量(出血)性休克;②感染中毒性;③神经性;④过敏性;⑤心源性;⑥血流阻塞性;⑦内分泌性;⑧创伤性。有时同时存在2种或以上休克,临床上称为复合性休克,现把各种休克分述如下:

\subsection{一、低血容量(出血)性休克}

低血容量性休克包括出血性休克和体液丧失所致休克。出血性休克是指人体内较大的血管破裂出血,全身血容量急剧减少致急性贫血和循环衰竭的临床现象。体液丧失性休克往往与感染中毒、电解质紊乱联系或合并在一起,比如烧伤、急性胃肠炎、过度呕吐和腹泻、过度利尿、脱水,以及其他原因所致。本节主要是论述内科出血性休克。

\subsubsection{1.出血性休克常见原因}

(1)消化道出血:如消化性溃疡、各种原因肝硬化、胃炎或急性胃黏膜出血、胆道出血、胃肠道肿瘤等。

(2)呼吸道病变的大咯血:如支气管扩张、肺结核、肺癌等。

(3)心脏血管疾病,如重度二尖瓣狭窄的大咯血、主动脉夹层分离出血、肺高压、肺栓塞等。

(4)凝血机制障碍如血友病、白血病、再生障碍性贫血出血等。

女性宫外妊娠破裂出血虽属妇科,但内科也不应漏诊。

\subsubsection{2.出血性休克诊断}

\paragraph{(1)临床特点:}

①具有原发疾病相应病史及体征。②出血征象:依不同病因可表现为呕血或(及)便血、咯血,腹膜腔积血等;上消化道出血多为呕血及(或)黑便及暗红色便,下消化道出血多为便血;呼吸道及多数心脏病(二尖瓣病变、肺高压、肺栓塞、左心衰竭等)多为咯血,伴有心跳、气促、咳嗽、呼吸困难、发绀等;心包、胸腹腔急性出血需注意主动脉夹层破裂出血。③休克征象和急性贫血征:临床症状与出血量一般成正比,且与出血速度密切相关,一般情况下,成人短期内出血:小量(失血量800~1000ml),可有面色苍白、口干、出汗、烦躁、心跳、心率100
次/分、收缩压(SP)降至80~90mmHg;中量(失血量1200~1700ml),除上述症状加剧外,表情淡漠,四肢厥冷、尿量减少明显,心率100~120次/分、SP降至60~70mmHg、脉压小;大量(失血量1700~2000ml),面色苍白、四肢冰冷、表情极度淡漠或嗜睡、呼吸急促、发绀、心率>120次/分,SP降至40~60mmHg;极大量出血(失血量>2000ml),神志不清或昏迷,无尿、脉搏快而弱或扪及不到,SP降至40~30mmHg以下或测不到。另外,同等出血量情况下,出血速度愈快,则休克症状愈严重。

\paragraph{(2)辅助检查:}

根据病史或临床表现,选择有关检查,以明确出血量、出血病因,如做血常规(包括红细胞、血红蛋白、血小板、血细胞比容等)、各种腔镜(包括支纤镜、胃或结肠镜、胆道镜等),造影(支气管造影、血管造影等)、X线、超声、CT或MRI,以及心电图、凝血机制、腔膜穿刺等检查。

\subsection{二、感染中毒性休克}

感染中毒性休克是由于某一或多种致病菌及其中间产物通过某一或多途径进入血液循环,引起低血压及(或)多器官功能衰竭综合征。本类休克是内科最常见的休克类型,任何年龄均可罹患,治疗难度较大,近年来由于抗生素、皮质激素以及免疫抑制剂、抗癌化疗药物广泛应用,“二重感染”、院内感染、静脉输液/血被致病菌污染所致感染中毒性休克时有发生,致使病情复杂,更增加治疗困难。

感染中毒性休克诊断标准:

1.有明确感染灶(如局部化脓性感染灶)或呼吸道、肝胆道、泌尿道、胃肠道感染史或输血、输液(致病菌污染血或液体)病史。

2.全身炎症反应表现 起病急、畏寒/寒战、高热,伴急性病容、多器官功能障碍症状等。

3.休克血压
收缩压<90~80mmHg,或较原有基础收缩压下降>40mmHg,持续至少1小时,或靠输液及药物维持血压者。

4.周围组织灌注不足表现 少尿(<30ml/h)或无尿,神志急性障碍,面色苍白,皮肤湿冷、脉细数而弱等。

5.发现致病菌存在 血、尿、粪、脑脊液、病灶脓液培养致病菌阳性。

休克型肺炎、中毒性菌痢、暴发型流脑、流行性出血热均多见感染中毒性休克,内毒素性休克(由急性胆囊炎、急性梗阻性化脓性胆囊炎、急性肾盂肾炎所致)临床不少见,真菌(如念珠菌所致)败血症也可致感染中毒性休克,值得重视。

\subsection{三、神经源性休克}

神经源性休克(neurogenic
shock)是指在创伤、剧痛等的剧烈神经刺激下,引起血管活性物质(如缓激肽、5-羟色胺等)释放,导致周围血管扩张、微循环淤血、全身有效血容量突然减少所产生的休克。

内科神经源性休克可见于:①各种穿刺如胸腹腔,心包穿刺、骨髓穿刺、腰椎穿刺、血管穿刺等;②药物应用:过快静注巴比妥类药物(如硫喷妥钠),过量使用神经节阻滞剂降压药物;③麻醉意外、腔镜检查等。

剧烈的精神刺激(如恐惧、悲伤、兴奋过度等)所致面色苍白、肢冷、脉弱、血压下降、意识改变,这种一时性血管舒缩功能障碍,与休克在本质上是不同的,应加以鉴别。

\subsection{四、过敏性休克}

过敏性休克是由于抗原(致敏原)与相对应抗体相互作用所引起的一种全身性即刻反应,导致全身毛细血管扩张,循环血容量迅速减少而致心排血量急剧下降,危重者可危及生命。

可能引致过敏性休克的致敏原物质颇多,归纳为3类:药物性、动物性、植物性。进入人体的途径也有3种:①注射药物,如血清及造影剂;②口服某种/类药物或进食某些食物;③皮肤或黏膜被昆虫或毒蛇咬伤或接触植物。但在临床上,还是以注射药物引起的过敏性休克为最多见。可引起过敏性休克某些常见抗原物质见表\ref{tab13-3}。

\begin{table}[htbp]
\centering
\caption{可引起过敏性休克的一些常见抗原物质}
\label{tab13-3}
\includegraphics[width=5.9375in,height=2.54167in]{./images/Image00090.jpg}
\end{table}

过敏反应及出现过敏性休克除必须有致敏原物质外,在很大程度上取决于个人的所谓过敏体质。注射药物、血清引起过敏性休克,与剂量不一定呈正相关,但剂量过大而疗程过长,则可增加过敏性休克的机会。用药方式及途径与过敏性休克的发生有关系,注射(静脉、肌注,腔内注射)引起严重反应可能性最大,口服次之,局部用药(喷雾、贴剂、栓剂或滴眼、喷喉、口含、药膏外用等)引起严重反应可能性较少,但需注意个体差异。青霉素、头孢类抗生素可在长期用药过程中突然发生过敏性休克。

过敏性休克诊断要点:

1.有明确用药、进食动/植物、毒虫刺咬史。

2.有典型临床特点 ①如药物,尤其青霉素类过敏成人多见,儿童少见;②青霉素过敏性休克多属速发型、发作呈闪电样(5分钟内占50\%,半小时占10\%);③有喉头水肿/支气管痉挛引起症状;有循环衰竭、血压下降、休克等症状;有休克的神经系统表现等。

3.康复后有人认为可做被动转移过敏试验证实过敏性休克的致敏原,虽安全可靠,但操作复杂。

\subsection{五、心源性休克}

心源性休克(cardiogenic
shock)是指极严重心泵衰竭的表现,由于心搏出量严重锐减,导致血压下降,周围组织供血严重不足,重要器官进行性衰竭的临床综合征。心源性休克是心血管病最危重病征,死亡率极高(高达80\%以上)。

\subsubsection{(一)心源性休克病因}

大致分为下列几类:

1.心肌舒缩功能极度降低包括急性大面积心肌梗死(也包括急性右室心肌梗死等),急性暴发型及(或)重症心肌炎(如病毒、感染、中毒、风湿性心肌炎等);重症原发性或继发性心肌病(包括扩张型、限制型、肥厚型等),重度或晚期心力衰竭等,但以急性心肌梗死最常见。

2.心室射血障碍如大面积肺梗死,乳头肌或腱索断裂致急性二尖瓣反流,瓣膜穿孔致急性严重的主动脉瓣或二尖瓣关闭不全,室间隔穿孔等。

3.心室充盈障碍如急性心包堵塞,严重快速性心律失常,严重二尖瓣狭窄、左心房黏液瘤或人工瓣膜失控,球瓣样血栓堵塞二尖瓣口,心室内占位性病变等。

4.心脏直视手术后低排综合征。

5.混合型2种或2种以上原因,如急性心肌梗死并发乳头肌功能不全或断裂,或室间隔穿孔,其心源性休克预后差,死亡率极高。

\subsubsection{(二)心源性休克诊断要点}

1.具有明确的严重心脏病病史 比如大面积心肌梗死(梗死面积>40\%左心室面积)或严重病毒性心肌炎病史。

2.收缩压<80mmHg,或原有高血压者收缩压<90mmHg,或较基础压下降>80mmHg,低血压持续时间>0.5~1小时以上。

3.组织和器官灌注不足的表现 神志呆滞或不清,或烦躁不安,大汗淋漓,四肢厥冷,脉快而弱,发绀或呼吸促;少尿(<20~30ml/h),高乳酸血症。

4.排除其他原因所致血压下降 如低血容量、严重心律失常、剧烈疼痛、代谢性酸中毒、心肌抑制药物或血管扩张剂作用等。

5.主要血流动力学指标异常 动脉平均压(AMP)<65mmHg,心脏指数(CI)<1.8~2.0L/(min·m\textsuperscript{2}
),肺毛细血管楔压(PCWP)>18mmHg,左心室舒张末期压(LVEDP)>10mmHg,中心静脉压(CVP)>12cmH\textsubscript{2}
O。

\subsubsection{(三)心源性休克分型}

\paragraph{1.按病情严重程度分为}

轻度休克:神清、烦躁不安、面色苍白、出汗、口干、尿少、肢端轻度发绀或发凉、心率>100
次/分、收缩压≤80mmHg,脉压<30mmHg。

中度休克:表情淡漠,面色苍白、肢端发绀,尿量明显减少(<17ml/h),呼吸急促,脉搏细数,心率≥120次/分,收缩压60~80mmHg,脉压<20mmHg。

重度休克:神志不清、意识模糊、面色苍白、四肢厥冷、发绀,脉搏细弱,皮肤花斑样改变、极少尿或无尿(<100ml/24h),心率>120次/分,收缩压40~60mmHg。

极重度休克:昏迷,呼吸浅而不规则,发绀明显,四肢厥冷、脉扪不到或极微弱,心音低钝或单心音,无尿,收缩压<40mmHg或0,可有弥散性血管内凝血(DIC),多器官衰竭或死亡。

\paragraph{2.按微循环状态分类}

血管张力增高型:皮肤苍白、四肢厥冷、汗多、尿少、意识障碍严重。

血管张力减低型:皮肤较红、四肢较暖、汗少、尿略少、意识障碍较轻。

急性心肌梗死并心源性休克需与大面积急性肺动脉栓塞、急性心包炎/心包压塞、主动脉夹层分离、各种急腹症等鉴别。

\subsection{六、血流阻塞性休克}

血流阻塞性休克(blood flow obstructed
shock)是由于血液循环严重受阻,导致有效循环血量显著减少,血压迅速下降所致的缺血综合征。

血流阻塞性休克常见病因是起源于右心或大血管的急性血流受阻。如:急性肺栓塞(包括血栓性、脂肪性、气体、寄生虫、羊水等)、主动脉夹层、急性心包填塞、心房黏液瘤、腔静脉阻塞、心内人工瓣膜血栓形成或(及)功能障碍等。

下面重点介绍肺动脉血栓栓塞症及急性心脏压塞症。

1.急性肺栓塞是由于血栓栓子堵塞肺动脉主干或分支引起肺循环障碍的临床和病理生理综合征。深部静脉血栓形成(DVT)和肺栓塞(PTE)已成为国内外颇受重视的常见病,发病率高,死亡率很高。引起PTE的栓子可来源于下腔静脉径路、上腔静脉径路或右心腔,但大部分来源于下肢深静脉,特别是从腘静脉上端到髂静脉段的下肢近端深静脉(约占50\%~90\%),血栓栓塞可以是单一部位,也可以是多部位,病理检查发现多部位或双侧性血栓栓塞更常见,易见于右侧或下肺叶。

发生大块肺栓塞时,栓子阻塞肺动脉及其分支后,通过机械阻塞、神经体液因素和低氧作用,引起肺动脉收缩,导致肺循环阻力增加、肺动脉高压;右室后负荷增高、右室壁张力增加,右室扩大,可引起右心功能不全,严重者导致心排血量下降,进而引起体循环低血压或休克等。

急性肺栓塞诊断要点:

(1)临床表现缺乏特异性,但如能认真了解病史及进行细致的体格检查仍可做出初步诊断。①呼吸困难,占84\%~90\%,是急性肺栓塞最常见的症状;②胸痛,占40\%~70\%;③咯血,约占10\%~30\%,提示肺梗死发生;④惊恐、约占55\%,系低氧血症或胸痛所致;⑤晕厥,约占13\%,系大块血栓堵塞肺动脉,并发严重血流动力学障碍,引起脑供血不足所致;⑥咳嗽、干咳或少许白痰,占37\%。典型的“呼吸困难、胸膜性疼痛和咯血”三联症仅占28\%。

(2)体征:①低热、发绀;②呼吸系统征象;呼吸频率≥20次/分,可高达40~50次/分,可有肺部干湿啰音、胸膜摩擦音;③循环系统征象:窦速(心率>90次/分),可高达120次/分以上,可出现各种类型心律失常;肺动脉瓣区有喷射音或收缩期喷射性杂音;可有右心功能不全及心包积液体征。

(3)原有静脉血栓形成的症状和体征。

(4)实验室检查:①心电图:电轴右偏,S\textsubscript{Ⅰ}
Q\textsubscript{Ⅲ} T\textsubscript{Ⅲ} 型。T\textsubscript{Ⅱ、Ⅲ、aVF}
、V\textsubscript{1、2}
倒置,完全/不完全性右束支传导阻滞;②超声心动图、X线胸片、CT、MRI、核素显像(肺灌注/通气显像、肺动脉造影能做出定性、定位、确诊性诊断;③D-二聚体<500μg/L,基本可排除急性PTE或深部静脉血栓的诊断;>500μg/L,需做螺旋CT或肺通气灌注扫描,加以确诊。

急性PTE类型:①猝死型;②急性心源性休克型;③急性肺心病型;④肺梗死型;⑤“不能解释的”呼吸困难型。

2.急性心脏压塞症(急性心包填塞,acute cardiac
tamponade)系指心包腔内心包积液较快(几分钟或1~2小时内)增加而压迫心脏致使心脏舒张充盈障碍,心室舒张压升高,舒张顺应性下降,心输出量及全身有效循环明显减少的临床综合征。

急性心包填塞在内科临床上多见于急性渗出性心包炎、主动脉夹层分离破入心包、肿瘤性心包炎等,患者有心包积液征象而突然面色苍白、气促、血压下降或休克、脉搏细数、脉压减少、奇脉、颈静脉怒张、肝大、腹水、心浊音界迅速增大,高度提示本病,且应迅速解除心脏压塞症状(心包穿刺或外科手术排除心包积液)。

\subsection{七、内分泌性休克}

内分泌性休克系指某些内分泌疾病如慢性垂体功能减退症(希恩综合征),急、慢性肾上腺皮质功能减退症,黏液性水肿,嗜铬细胞瘤等,在某些情况下(如急性感染或出血)发生低血压或休克。

\subsubsection{1.慢性垂体功能减退症}

本病患者多因围生期大出血休克,使腺垂体大部分缺血坏死和纤维化,而致全垂体功能减退症,所有垂体激素均缺乏,但无占位性病变的表现。

\subsubsection{2.慢性肾上腺皮质功能减退症(Addison病)}

本病分为原发性和继发性,原发性者又称Addison病,继发性者由下丘脑-垂体病变引起。本病临床表现为全身皮肤色素加深、黏膜(齿龈、舌、颊部等)色素沉着,乏力、低血压及具有心、脑、肾、胃肠道、代谢、生殖系统症状,在各种应激状态下出现肾上腺危象,可伴休克。

肾上腺危象:①有诱因应激状态:各种感染、创伤、手术、分娩、过劳、寒冷、大量失水(包括大汗、呕吐、腹泻、失水),突然中断肾上腺皮质激素治疗等应激状态;②危象为Addison病急骤加重表现:恶心、呕吐、腹痛、腹泻、严重失水、血压下降、心率加快、脉细而弱、精神异常、高热、低血糖、低钠低钾血症,严重者休克、昏迷、死亡。

\subsubsection{3.急性肾上腺皮质功能减退}

本病是指肾上腺皮质功能急性减退、衰竭而表现有胃肠功能紊乱、高热、循环虚脱、低血压或休克、惊厥、昏迷等临床表现。

本病病因:①严重感染,如脑膜炎双球菌性败血症致双肾上腺出血,流行性出血热等;②原有Addison病,在各种应激状态下未加大应用皮质激素或停用皮质激素而诱发;③新生儿分娩损伤肾上腺出血所致;④其他原因,如长期大量皮质激素在应激下无增加剂量或停用皮质激素所致。其临床表现为高热、头痛、呕吐、腹泻、气促、发绀、全身瘀点/斑,可有惊厥、抽搐、休克、昏迷等。

\subsubsection{4.甲状腺功能减退症(简称甲减)}

甲减其病理特征是黏多糖在组织和皮肤沉积,表现为黏液性水肿(myxedema),成人原发性甲减占全部成人甲减的90\%~95\%;其主要病因:①自身免疫损伤,以自身免疫性甲状腺炎为多见;②手术或\textsuperscript{131}
I治疗后甲状腺受破坏;③碘过量;④抗甲状腺药物等。

甲减------黏液性水肿昏迷及休克,见于重症甲减。其临床特征:①多发病于冬季寒冷时;②多有诱因:除寒冷外,有合并全身性疾病、中断甲状腺素代替治疗、手术麻醉、镇静药物使用不当等诱因;③多表现为严重临床症状:低温(<35℃)、乏力、四肢肌肉松弛、反射减弱或消失、呼吸缓慢、心动过缓、血压下降、嗜睡、重者昏迷、休克,甚至死亡。

\subsubsection{5.嗜铬细胞瘤}

可以发生低血压,甚至休克,或出现高血压和低血压相交替的表现。低血压和休克发生原因:①本病分泌大量儿茶酚胺引起血管强烈收缩,组织缺氧、微血管通透性增加,血容量锐减;②大量儿茶酚胺引致严重心律失常或心力衰竭,致心排血量明显减少;若癌组织骤然发生出血、坏死,以致儿茶酚胺停止释放;③由于肿瘤组织主要分泌肾上腺素,兴奋肾上腺素能β受体促使血管扩张;④肿瘤还可以分泌舒血管肠肽、肾上腺髓质素等多种扩血管物质引起血管扩张;⑤本病在发生休克前常可有呕吐、腹泻、大汗淋漓、不能进食等症状,可产生低血压或休克。本病根据临床症状及体征,可进行血、尿儿茶酚胺及其代谢物测定,以及影像学(包括B超、CT、MRI等)检查而确定诊断。

\subsection{八、创伤性休克}

创伤性休克是指一些遭受严重创伤的患者,由于多种因素导致全身循环血量大减所引起的临床综合征,该病多见于外科临床,有人把其归纳为低血容量性休克范畴,因血容量锐减所致。

创伤性休克原因颇多,临床表现较为复杂,各个时期表现也不同:①临床上这类患者早期因骨折、挤压伤、大手术、烧伤等,使血浆或全血的丢失,加上神经刺激、组织损害,使损害部位的出血、水肿、渗出到组织间隙的液体不能参与有效循环,可使循环血量大减,出现休克;②由于损伤组织逐渐坏死及(或)分解产生如组胺、蛋白酶等,这些物质具有抑制血管的作用,引起微血管扩张和管壁通透性增加,使有效循环血量进一步减少,临床上出现休克或加重休克病情;③在组织损伤的过程中,往往可夹杂感染中毒以及心源性因素,也可出现休克。

\protect\hypertarget{text00117.html}{}{}

\section{参考文献}

1.何秉贤.体位性低血压的诊治现代概念.中华高血压杂志,2008,16(2):101-102

2.中华医学会内分泌学分会.甲状腺疾病诊治指南:甲状腺功能减退症.中华内科杂志,2007,46(11):967-971

3.The Surviving Sepsis Campaign Guidelines Committee including the
Pediatric Subgroup.Surviving Sepsis Campaign:International Guidelines
for Management of Severe Sepsis and Septic Shock.Crit Care
Med,2013,41(2):580-637

4.中华医学会心血管病学分会,中华心血管病杂志编辑委员会.急性ST段抬高型心肌梗死诊断和治疗指南.中华心血管病杂志,2010,38(8):675-687

5.中华医重症医学分会.低血容量性休克复苏指南.中华实用外科杂志,2007,27(8):581-587

6.Torbicki A,et al.The task force for the diagnosis and management of
acute pulmonary embolism of the European Society of
Cardiology.Guidelines on the diagnosis and management of acute pulmonary
embolism. Eur Heart J,2008,29,2276-2315

7.中华医学会心血管病分会肺血管组.急性肺血栓栓塞症的诊断治疗中国专家共识.中华内科杂志,2010,49(1):74-81

8.邹晓等.高龄老年餐后低血压的临床特点及防治策略的研究.中华老年心脑血管病杂志,2013,3(15):251-254

9.顾卫红.多系统萎缩的临床诊断与治疗.中国现代神经疾病杂志,2012,1(3),133-134

10.赵克松.创伤性休克新概念.中华创伤杂志,2005,25(1):29-31

\protect\hypertarget{text00118.html}{}{}


\chapter{脾}

\section{检查方法}

\subsection{扫描前准备}

一般扫描前口服2%的泛影葡胺500~800ml,使胃肠道充盈。

\subsection{扫描方法}

1.平扫:自膈肌开始,以5~10mm层厚与间距行连续扫描。螺旋CT可采用3~10mm的准直,螺距1~2∶1。

2.增强扫描:静脉团注60%造影剂100ml,作快速或动态扫描,脾明显强化。因此,可以鉴别病灶是原发于脾或附近脏器如胃、胰、肾上腺或肾。但部分患者在静脉早期由于脾脏呈不均匀强化,可遗漏小的病变,稍后脾脏强化密度即逐渐趋向一致。

近年来推出的脂融性造影剂如EOE-13选择性的只被肝、脾网状内皮细胞所吸收,特异性强、增强效果好。但毒性强,尚较少应用。

\section{正常解剖、先天变异和畸形}

\subsection{大小和密度}

1.形态:脾位于左膈下,其位置也可因个人情况而不同,如脾周围韧带松弛可位置较低。外缘圆隆而光滑,伴9~11肋骨下行。内缘因胃、胰及肾造成的压迹而呈分叶状隆起,不同层面有不同的外形。正常脾内缘可见1至数个小切迹,脾下缘亦可有切迹。脾门部可见大血管出入。

在较深切迹的扫描层面,脾脏可形似完全离断,但上下层面仍可见切迹两侧的脾是相连的。

最常见的隆起夹在胰尾和左肾上极之间,可形似肾上腺、肾或胰尾部肿块,尤其脾大时多见。

2.大小:脾的大小因不同年龄、体重及营养状况而不同。一般成人脾脏长12cm,宽7cm,厚约3~4cm。长>15cm肯定增大,脾厚>4.5cm可视为增大。此外,脾脏的下缘超过正常肝脏的下缘或脾脏前后径超过腹部前后径的2/3均提示脾大。

3.密度:正常脾脏密度均匀,其CT值正常范围较大,平扫时总低于正常肝脏约5~10Hu。增强扫描早期皮质强化高于中间髓质而致密度不均,稍后密度均匀,CT值可达100~150Hu。

\subsection{副脾}

副脾是一种并不少见的先天性变异,由正常脾组织构成,尸检时发现副脾占10%~30%,与创伤所引起的异位脾组织种植不同。

\textbf{【病理】}
副脾呈球形,最常见于脾门附近;少数靠近胰尾;罕见于其他部位如胃壁、小肠壁、大网膜、肠系膜、横膈甚至盆腔内或阴囊内。可与脾完整分离,亦可与主脾有一细蒂相连。单个或多个,通常不超过6个。副脾多由脾动脉供血,有脾门和正常结构的包膜。

\textbf{【CT表现】}
①呈单发或多发的、边缘光滑的圆形或卵圆形结节影(图\ref{fig14-1}、图\ref{fig14-2});②密度均匀且与脾实质密度相同;③动态增强扫描与脾同时增强和消退,CT值与脾相同;④不典型部位者需结合超声等观察其血供来源等综合诊断。

\begin{figure}[!htbp]
 \centering
 \includegraphics[width=.7\textwidth,height=\textheight,keepaspectratio]{./images/Image00309.jpg}
 \captionsetup{justification=centering}
 \caption{副脾的各种表现}
 \label{fig14-1}
  \end{figure} 

\begin{figure}[!htbp]
 \centering
 \includegraphics[width=.7\textwidth,height=\textheight,keepaspectratio]{./images/Image00310.jpg}
 \captionsetup{justification=centering}
 \caption{副 脾\\{\small 左侧脾门下方有圆形软组织密度结节,与脾密度一致,边缘光滑}}
 \label{fig14-2}
  \end{figure} 

识别副脾的意义:①脾亢等病在脾切除后,副脾可以明显增大并引起原发症状的复发,因此应把副脾一并切除;②勿将副脾误为增大淋巴结或肿瘤;③脾脏肿瘤亦可累及到副脾,如淋巴瘤;④副脾少见的并发症是自发性破裂、梗死或扭转。

\subsection{游走脾}

本病亦称为异位脾、迷走脾、脾下垂或漂浮脾。

\textbf{【病因】}
尚有争论,大多认为是一种少见的先天性异常,由于支持脾脏的韧带松弛或缺如所致。但亦有学者认为还存在着继发因素,包括脾大、创伤及妊娠时内分泌作用和腹部松弛等。

\textbf{【临床表现】}
可发生于6~80岁,以20~40岁的女性多见。病人可无症状而偶然发现。由于急性或慢性扭转可引起急腹症、脾梗死、脾坏疽、脓肿、胃食管静脉曲张、脾淤血、脾大、脾功能亢进等。

\textbf{【CT表现】}
可显示在胃后方和左肾前方的脾缺如。在下腹部或盆腔内可见一个密度均匀的实质性“肿块”,相当于脾脏大小;增强扫描符合正常脾组织的强化规律。如有扭转存在,可有脾梗死表现;如扭转累及胰尾,可导致胰尾坏死和腹水;如慢性扭转病例,可见增厚和强化的假包膜,由网膜和腹膜粘连形成。

\subsection{无脾和多脾综合征}

无脾和多脾可为孤立性表现,但常常伴先天性心血管异常和内脏位置异位,分别称为无脾综合征和多脾综合征。

\textbf{【CT表现】} 常见表现如下:

1.无脾综合征:①肺部畸形:双侧呈三叶肺(右肺形态)、双侧右支气管型表现等;②腹部内脏位置异常和畸形,以及脾缺如;③增强扫描见主动脉和下腔静脉位于同一侧可提示无脾综合征,而本征很少见到下腔静脉肝段缺如伴奇静脉连接。

2.多脾综合征:①肺部畸形:双侧呈二叶肺(左肺形态);②腹部内脏位置异常和畸形;③右侧多个小脾、下腔静脉肝段缺如伴奇静脉连接等为其特征性征象(图\ref{fig14-3})。

\begin{figure}[!htbp]
 \centering
 \includegraphics[width=.7\textwidth,height=\textheight,keepaspectratio]{./images/Image00311.jpg}
 \captionsetup{justification=centering}
 \caption{多脾合并多种畸形}
 \label{fig14-3}
  \end{figure} 

\section{脾脏感染及梗死}

\subsection{脾脓肿}

本病较少见,通常是全身感染的一部分。

\textbf{【病因病理】}
一般为细菌感染,免疫力低下者则可为真菌感染。75%经血行播散、15%由于脾损伤、10%由于脾梗死所致。很少由邻近器官感染直接侵犯所致。早期以急性炎症反应为主,表现为脾弥漫性增大;随后炎症反应趋于局限化,进而形成脓肿。脓肿可单房或多房、多发或单发、大小不等。

\textbf{【临床表现】}
常存在败血症症状,出现发热、寒颤、左上腹痛、脾大、恶心、呕吐、白细胞增高等。但临床确诊困难,极易漏诊。

\textbf{【CT表现】}
脓肿呈圆形或椭圆形低密度区,CT值为20Hu左右,多<30Hu,境界不清;脓腔内有气-液面或液-液面是其可靠表现。增强扫描脓肿壁强化明显,内壁光整,壁外有水肿带;当脓肿为多发而又较小时,则常表现为增强的脾内有斑点状或粟粒状充盈缺损现象。可并发脾周及膈下积液,甚至脾破裂;还可见左侧肾前筋膜增厚。

\subsection{脾结核}

本病罕见,通常继发于肺结核。感染途径以血行为主。

\textbf{【病理】}
①粟粒型脾结核:最初产生渗出性病变,病灶广泛即形成“粟粒型脾结核”。②结节型脾结核:渗出病灶如不吸收,可发展成结核性肉芽肿,并发生干酪坏死,直径多在5~20mm大小,形成“结节型脾结核”,即所谓的大结节。③结核性脾脓肿:当结节性病灶相互融合或孤立性病灶发展增大,并液化形成“结核性脾脓肿”。④脾结核钙化:钙化灶大多在干酪性病灶的愈合过程中产生。

\textbf{【临床表现】}
多见于中青年。多有肺结核,患者全身情况差,消瘦、乏力、发热、脾大、腹痛、脾区明显压痛,常合并其他脏器结核。临床确诊困难。

\textbf{【CT表现】} 根据病理基础其CT表现亦可归纳为以下几型:

1.粟粒型脾结核:由于病灶多<2mm,CT难以显示或仅表现为脾脏的轻中度增大,密度稍低或不均。

2.结节型脾结核:该型CT显示率较高,表现为边界不清、大小不等的多发低或等密度灶;增强后多无强化,边界清楚,少数可见环状强化(肉芽组织)。

3.结核性脾脓肿:表现为脾内单发或多发较大类圆形低密度灶;增强后边缘强化而内部无强化。

4.脾结核钙化:多表现为1~5mm的斑点状高密度灶(图\ref{fig14-4})。结节型脾结核愈合钙化可表现为花冠状或羊毛状钙化。

\begin{figure}[!htbp]
 \centering
 \includegraphics[width=.7\textwidth,height=\textheight,keepaspectratio]{./images/Image00312.jpg}
 \captionsetup{justification=centering}
 \caption{脾结核(钙化)\\{\small A、B为同一患者,脾脏内有多个沙粒状钙化,合并右肾结核}}
 \label{fig14-4}
  \end{figure} 

此外,还可见腹膜后、肝门、脾门淋巴结增大、钙化、环状强化,以及其他器官(如肺、肝、胰、肾上腺、肠和脊椎等)结核。

\textbf{【诊断要点】}
①好发于20~25岁,多数伴结核中毒症状;②CT平扫见脾内弥漫性低密度灶,直径多<20mm;增强后病灶无强化;③多伴有腹膜后和肝门等淋巴结增大、钙化或周边环状强化;④常伴其他脏器结核;⑤脾内散在斑点样钙化,直径1~5mm大小。

\textbf{【鉴别诊断】}

1.淋巴瘤:是脾最常见的原发性肿瘤,临床与脾结核无明显区别。但淋巴瘤常为多发或单发,弥漫性病变较少,无钙化和其他脏器结核;增强后病灶轻度强化,增大淋巴结多无环状强化。再结合临床、骨髓象、血象等做出诊断。

2.转移瘤:多有原发肿瘤史。CT表现为脾内单发或多发的低密度灶,少有弥漫性;病灶相对较大,可出现“牛眼征”或“靶心征”;淋巴结多无环状强化等有助于诊断。

3.脾脓肿:临床表现为寒颤、高热及白细胞计数升高。CT表现为单发或多发较大低密度灶;脓肿壁强化明显,内壁光整,壁外有水肿带等多可鉴别。

但孤立性球形脾结核仍较难与脾脏恶性肿瘤相鉴别,应予重视。

\subsection{脾结节病}

本病多为全身结节病的局部病变,但偶可孤立发生于脾脏。

\textbf{【CT表现】}
脾内病灶呈低密度团块状病灶,可多发、亦可单发。CT值约30~40Hu;增强扫描呈轻度强化,与显著强化的脾界限更清晰。为全身结节病的一部分时,可见胸部结节病的相应表现。

\subsection{脾梗死}

本病是指脾内动脉的分支阻塞,造成局部脾组织的缺血坏死。

\textbf{【病因病理】}
其病因主要有血栓形成、动脉粥样硬化、二尖瓣疾病、溶血性贫血、白血病、结缔组织疾病、脾动脉瘤等。当门静脉高压脾大时,更易发生梗死。其病理学变化为贫血性梗死,但在脾淤血时,病灶周围有出血带。晚期坏死脾组织被纤维组织所代替,因纤维瘢痕收缩,使脾脏局部凹陷。较大的梗死灶,不能完全纤维化,其中央可发生液化,并被纤维结缔组织包裹形成囊腔。

\textbf{【临床表现】}
多无症状,有时可出现左上腹痛、左膈升高、左侧胸腔积液。

\textbf{【CT表现】}
梗死多发生于脾前缘处近脾门方向,大小不等,常多个病灶同时存在。梗死灶呈三角形或楔形,底在脾的外缘,尖端指向脾门(图\ref{fig14-5})。有时呈不规则形低密度灶。但病程少于5天者多呈等密度而不易显示。增强扫描因不强化而显示更清楚。若整个脾脏梗死,则只有包膜强化表现。在急性期(8天以前)呈低密度区、不强化;慢性期(15~28天)则密度逐渐恢复正常,由于瘢痕收缩而出现收缩变形,甚至合并钙化。不典型者呈多发的、不均匀的、边缘不清楚的小片状或大片状低密度区。大的梗死灶中央可以伴有囊性变。少数伴有包膜下积液。

\begin{figure}[!htbp]
 \centering
 \includegraphics[width=.7\textwidth,height=\textheight,keepaspectratio]{./images/Image00313.jpg}
 \captionsetup{justification=centering}
 \caption{脾梗死\\{\small 食管癌术后复发且并发脾梗死,增强扫描脾内有楔形低密度区,尖端指向肺门}}
 \label{fig14-5}
  \end{figure} 

\subsection{弥漫性脾大}

脾脏的大小、形态差异较大,无论是临床查体还是CT确定其大小有时较为困难。儿童脾脏的大小可用公式计算,即L(cm)=5.7+0.31A(L是脾脏的长轴,A为年龄)。到达青春期脾脏的大小和重量可以达到最大。

\textbf{【病因】} 弥漫性脾大的病因主要有以下几方面:

1.感染性疾病:可引起脾脏非特异性增大,如败血症、亚急性细菌性心内膜炎、肠伤寒、传染性单核细胞增多症、巨细胞病毒感染以及寄生虫病(如疟疾、血吸虫病)、肉芽肿性病变(如结核、组织胞浆菌病、结节病)等。它们在急性期可引起脾大,慢性期可正常大小。

2.血液系统疾病:可引起脾大,特别是骨髓增生性病变、霍奇金病弥漫性浸润、真性红细胞增多症、骨髓纤维化、骨髓外化生等。其他如淋巴细胞性白血病、自身免疫性贫血、血小板减少性紫癜、遗传性球形红细胞增多症。异形镰状细胞病患者脾亦可增大,但常伴反复梗死和血肿,反复梗死最终导致自截(萎缩)。

3.充血性脾增大:见于门静脉高压或脾静脉的阻塞和血栓形成,以及心脏病。

4.结缔组织疾病:如系统性红斑狼疮、风湿性关节炎等。

此外,代谢性疾病如类脂质沉积症(高雪氏病、尼曼-匹克病)、糖尿病等也可导致脾大。淀粉样变亦可出现脾大(但CT脾内可见多发性低密度结节、弥漫性密度减低,强化亦很轻微)。

\textbf{【CT表现】}
脾脏的长>15cm肯定增大,脾厚>4.5cm可视为增大。此外,脾脏的下缘超过正常肝脏的下缘或脾脏前后径超过腹部前后径的2/3均提示脾大。我们认为利用肋单元判断脾脏大小(正常为5个以下肋单元)存在严重不足,准确性甚差。脾脏的容积测量虽然更加接近实际大小,但因个体差异较大,以及影响因素较多,仍难分辨正常与轻度异常。

\subsection{脾内钙化的鉴别诊断}

识别脾内钙化有助于脾脏疾病的鉴别诊断。引起脾内钙化的疾病较多:①肉芽肿(如结核、组织胞浆菌病)是最常见的,呈点状,可伴肝或肺内钙化。②脾附近或脾内静脉石见于血管性异常,如血管瘤病。③脾周边的环形钙化见于陈旧性血肿或动脉瘤。脾动脉、脾动脉瘤的钙化可呈曲线状,常见于脾门部,钙化的动脉瘤较未钙化者不易破裂。④脾包虫囊肿仅在明确囊内容物(如子囊)时方可诊断。⑤卡氏肺囊虫感染引起的脾钙化亦有报道。⑥脾梗死的钙化可发生在镰状细胞性贫血的病人,但不常见。此外,除脾外,肾及淋巴结亦可见钙化。

\section{脾囊肿及肿瘤}

\subsection{概述}

1.良性肿瘤:很少见,有血管瘤、淋巴管瘤、血管淋巴管瘤(亦称为脉管瘤)和错构瘤。更少见的有纤维瘤、平滑肌瘤、神经纤维瘤、脂肪瘤等。

2.恶性肿瘤:脾原发性恶性肿瘤极少见,以恶性淋巴瘤和血管内皮肉瘤较常见。罕见的有梭形细胞肉瘤、平滑肌肉瘤、纤维肉瘤、恶性神经纤维肉瘤、卡波西氏肉瘤(Kaposi肉瘤)、淋巴管内皮肉瘤、恶性组织细胞增生症等。

\subsection{脾囊肿}

\textbf{【病因病理】}
①真性囊肿:内衬以内皮细胞,系先天性囊肿。②假性囊肿:即继发性囊肿,多见,占脾囊肿的80%。囊壁无内皮细胞被覆,大多为外伤后血肿退变所致,也可见于脾梗死后及胰腺炎所致。③包虫性囊肿:占全部包囊虫病的2%~3%,多与肝、肺等脏器的病变同时发生。

\textbf{【临床表现】}
多见于40岁以下,男女发病率之比约为2∶1。小的囊肿多无症状。只有当囊肿巨大压迫胃、左肾和输尿管等邻近脏器时,产生相应的压迫症状或局部触及肿块。

\textbf{【CT表现】}
平扫呈圆形或椭圆形的水样低密度灶,界限清楚,有时可多发(图\ref{fig14-6})。先天性囊肿壁很薄,外形规则;假性囊肿壁可稍厚及稍不规则。囊壁有时钙化,先天性囊壁钙化细而光滑;后天性则厚而欠规则;胰腺炎所致者常在囊肿边缘见到裂隙,有一定特征。包虫囊肿往往使脾增大,囊液CT值较前两者稍高,囊内可有钙化;也可见到囊内子囊、双边征及水上荷花征和飘带征等。增强扫描先天性及假性囊肿囊壁不强化。包虫囊肿也不强化,但当有脱落的内囊及子囊存在时,则有囊壁强化表现。

\begin{figure}[!htbp]
 \centering
 \includegraphics[width=.7\textwidth,height=\textheight,keepaspectratio]{./images/Image00314.jpg}
 \captionsetup{justification=centering}
 \caption{脾囊肿\\{\small A、B为同一患者,脾内有巨大圆形水样密度灶,边缘光滑,无强化}}
 \label{fig14-6}
  \end{figure} 

\subsection{脾血管瘤}

本病是最常见的脾良性肿瘤。

\textbf{【病理】}
与其他部位的血管瘤类似,多为海绵样或毛细血管样扩张的血管构成。单发或多发,瘤内可有栓塞、出血、纤维化或钙化。肿瘤生长缓慢。

\textbf{【临床表现】}
多见于20~50岁,男性多于女性,偶见于儿童。通常无症状。大的血管瘤可有左上腹包块、胀痛及邻近脏器受压症状;亦可发生破裂而出现急性腹痛、血压下降和休克等;偶有脾亢症状。

\textbf{【CT表现】}
肿瘤直径多<2cm,>5cm者少见,较大的血管瘤可致脾增大。平扫呈低密度或等密度肿块,可有少许钙化存在或中心见更低密度瘢痕区。偶呈囊样,但壁厚。增强扫描病灶均匀强化,亦偶可不均匀强化。快速增强扫描亦可见边缘明显结节状强化,然后逐渐向中心充填,延迟扫描大多完全填充而呈等密度。总之,其表现与肝血管瘤相类似是其特征。

\subsection{脾淋巴管瘤}

脾脏淋巴管瘤是非常少见的先天性畸形。

\textbf{【病理】}
淋巴管瘤组织学上根据异常淋巴管扩张后的大小不同分为毛细管型、海绵状型和囊肿型3类,以囊肿型最常见。多位于颈、腋窝、纵隔、腹膜后和四肢软组织内。脾脏罕见,可单发、多发或弥漫分布于整个脾脏。

\textbf{【临床表现】}
脾脏淋巴管瘤多见于中青年。感觉左上腹部胀满或轻微胀痛,个别无症状。脾脏可增大。

\textbf{【CT表现】}
脾脏增大。脾内有单发或多发大小不等的近水样低密度灶,CT值约10~30Hu。囊内可见小片状稍高密度灶(可能与含铁血黄素有关)。有文献认为如囊内为脂肪液体则是淋巴管瘤的特征性改变。病灶边缘清楚,周边及间隔增强后有轻度强化,与脾囊肿不同。可合并感染、出血和钙化。

\textbf{【鉴别诊断】}
病灶CT值偏高,内有粗间隔和边缘轻度强化可与囊肿鉴别。脓肿壁强化显著、外周有水肿,两者亦不难鉴别。

\subsection{脾脉管瘤}

脉管瘤是血管瘤和淋巴管瘤的混合瘤,脾脉管瘤较为罕见。国内有学者报道1例脾弥漫性脉管瘤,甚为少见。

\textbf{【CT表现】}
①脾大,表面可凹凸不平,包膜完整。②脾内可见多个大小不等的类圆形低密度结节,直径0.5~3.0cm。因血管和淋巴管成分的多少而密度不一,但多>30Hu。大多界限不清(表明血管成分多),少部分边缘清晰锐利(表明血管成分少)。③增强扫描因血管成分的多少而呈中度强化,部分强化不著。④多发时整个脾脏呈蜂窝状,病灶为蜂巢、脾实质为瘤巢的支架。

\textbf{【鉴别诊断】}
根据上述特点且多无增大淋巴结等可与多发且具有“牛眼征”或“靶征”的转移瘤相鉴别。根据其密度、边缘与强化表现也可与囊性淋巴管瘤相鉴别。

\subsection{脾错构瘤}

本病较为罕见,是由于异常数量和杂乱排列的正常组织(如脾、血管、脂肪和平滑肌等组织)构成的肿瘤样畸形。

\textbf{【病理】}
可能来源于脾始基局限性发育障碍,使正常成分的脾组织(如脾、血管、脂肪组织、纤维组织和平滑肌等)组合比例发生混乱。多为单发,偶可多发。无包膜,大小为数毫米到15mm以上。

\textbf{【临床表现】}
成人多见,少数见于儿童;女性略多见。多无症状,少数大者有左上腹不适,疼痛,偶有脾亢。

\textbf{【CT表现】}
多为单发低密度实性病灶,偶为囊实性或囊性病灶,轮廓多不清。偶见星状或团块状粗糙钙化,如含脂肪组织,CT值<-25Hu较有特征。增强扫描多呈弥漫性渐进性较明显强化,并呈延迟强化现象。总之,伴有钙化和脂肪组织的肿块为其特征性表现。

\subsection{脾血管内皮肉瘤}

本病又称为脾血管内皮瘤,是起源于脾窦内皮细胞的高度恶性肿瘤,很少见。

\textbf{【临床表现】}
左上腹疼痛、发热,很快出现左上腹肿块;恶病质状态明显;常早期转移到肝、肺、骨骼、腹膜等处。有时可发生脾自发性破裂。此病预后差。

\textbf{【CT表现】}
①脾脏增大。②脾内单发或多发大小不等、边界较清的实性或囊实性结节状低密度灶,有时可见出血征象。当肿块内含有钙化或含铁血黄素沉着时,CT值可高达100Hu以上。③常见腹水。④增强扫描可见脾内多发的成簇状强化区存在,并有向病灶中心扩展的趋势。常占据脾的大部分,坏死囊变区不强化。⑤可有其他脏器及腹膜后淋巴结转移表现。⑥若有自发破裂可出现包膜下或脾周出血表现。

\subsection{脾淋巴瘤}

本病绝大多数为全身淋巴瘤对脾脏的浸润,原发性极少。NHL约67%的病例脾脏受累,HD约70%有脾浸润。原发性75%为NHL。

\textbf{【病理】}
脾淋巴瘤大体病理分为4种类型:①均匀弥漫型:脾均匀弥漫性增大,无明显肿块形成;②粟粒结节型:病灶为1~5mm大小;③多发肿块型:病灶2~10cm大小;④巨块型:病灶大于10cm。

\textbf{【临床表现】}
多表现为发热、贫血、乏力、纳差、体重下降、左上腹痛,脾脏增大、边缘有结节感。全身淋巴瘤可有全身广泛淋巴结肿大。白细胞和血小板可减少。

\textbf{【CT表现】}
综合有关文献其CT表现分为4类:①弥漫增大型:表现为脾脏显著增大,密度较均匀。如NHL脾明显增大,高度提示淋巴瘤侵犯。但脾大并不一定意味着有脾内淋巴瘤,脾不大却可能有淋巴瘤侵犯。②弥漫细小结节型:脾大小也可正常。脾内弥漫性低密度小结节,大小为0.2~1.0cm。但<1.0cm的结节平扫甚至增强扫描亦难以显示,而仅表现为脾大。③结节型:呈圆形或类圆形、单发或多发低密度结节,结节直径>1.0cm。④块状型:单发或多发的、直径>3.0cm的低密度肿块。

总之,脾淋巴瘤CT表现为脾大、脾内弥漫多发或单发(但单发少见)低密度肿块或结节,边界不清,无钙化及出血。增强扫描脾明显强化,而肿块或结节多强化不明显。继发性常伴脾门及腹膜后淋巴结增大。

\textbf{【鉴别诊断】}

1.血管内皮肉瘤:表现为脾大,常伴腹水。平扫示脾内大小不等、单发或多发实性或囊实性混合密度肿块,肿块内常有出血或钙化,边界清晰。增强扫描见脾内成簇状多发的强化区存在。此外,肿瘤破裂可引起脾包膜下及腹腔积血。

2.脾梗死:多发生于脾前缘近脾门方向,病灶常呈楔形,尖端指向脾门,无强化,可伴脾内出血。

3.转移瘤:常有明确的原发灶,多发生于全身广泛转移的晚期。CT表现为脾大,脾内单发或多发的低密度灶,境界清晰。伴有肝和其他脏器的转移和腹水。

此外,应注意结合临床与引起脾大的其他疾病如白血病等相鉴别。

\subsection{脾转移瘤}

脾转移瘤远较肝脏少见,其原因主要是脾为具有免疫监督能力的特殊器官,但也有文献认为可能与肝有门静脉系统有关。有人认为脾脏转移时,一般都有多个器官受累。常见的原发肿瘤依次为肺癌、乳癌、前列腺癌、结肠癌、恶性黑色素瘤、卵巢癌、宫颈癌和胰腺癌。多为血行转移,少数为直接侵犯和种植。

\textbf{【病理】}
病灶可位于脾脏的静脉窦、红髓、白髓和小梁血管等处。病灶大小不等,呈结节型或弥漫型。大的结节可伴有坏死液化。广泛转移可伴脾弥漫性增大。

\textbf{【临床表现】}
大多有肿瘤病史。常伴有消瘦、乏力、低热、贫血等恶性肿瘤的晚期表现,少数可有左上腹疼痛不适。体检脾脏增大。

\textbf{【CT表现】}
呈多样化,缺乏特异性。脾脏轻度增大或不大。脾内单发或多发、大小不等的低密度灶,边缘清楚或不清楚(图\ref{fig14-7});也可呈囊性或囊实性,少数呈等密度。微小结节和弥漫性脾浸润CT不能发现。邻近器官病变的直接浸润表现为肿块与脾的界限消失或脾边缘密度减低。增强扫描有轻度不均匀强化,病灶密度低于脾实质。多同时可见肝脏或其他脏器转移灶。

\begin{figure}[!htbp]
 \centering
 \includegraphics[width=.7\textwidth,height=\textheight,keepaspectratio]{./images/Image00315.jpg}
 \captionsetup{justification=centering}
 \caption{脾转移瘤\\{\small 为结肠癌脾转移。增强扫描平衡期,脾脏内有一欠规则低密度灶,边缘呈不规则强化}}
 \label{fig14-7}
  \end{figure} 

\textbf{【鉴别诊断】}
脾囊性转移尤其是单发者应与囊肿相鉴别。前者囊壁多较厚,且有强化和壁结节,结合肝内等转移灶可予鉴别。还应注意结合病史与淋巴瘤相鉴别。

\protect\hypertarget{text00022.html}{}{}


\chapter{心脏杂音}

心脏杂音是心脏病的重要体征。心脏杂音可分为收缩期杂音、舒张期杂音、连续性杂音和来往性杂音。收缩期杂音根据杂音开始和终止的时间可命名为全收缩期、收缩中期、收缩早期或收缩晚期杂音。全收缩期杂音与第一心音同时开始,占据全部收缩期。起源于左侧心脏的全收缩期杂音终止于第二心音的主动脉瓣成分,起源于右侧心脏的全收缩期杂音终止于第二心音的肺动脉瓣成分。收缩中期杂音在第一心音之后开始,在第二心音之前终止。起源于心脏左侧的收缩中期杂音终止于第二心音的主动脉瓣成分之前,起源于心脏右侧的收缩中期杂音终止于第二心音的肺动脉瓣成分之前。收缩早期杂音限于收缩早期,与第一心音同时开始,呈递减型减弱,在第二心音之前完全结束,一般在收缩中期或其前结束。收缩晚期杂音在收缩中至晚期开始并延续到第二心音处。

全收缩期杂音以前又称“反流性收缩期杂音”,因为“反流”一词可包括全收缩期的、收缩早期和收缩晚期的杂音,已放弃使用。收缩中期杂音以前又称“喷射性收缩期杂音”,因为收缩中期杂音不一定由于“喷射”所致,所以也应该放弃使用。

全收缩期杂音最常见于二尖瓣关闭不全、三尖瓣关闭不全及室间隔缺损。收缩中期杂音最常见于主动脉瓣狭窄和肺动脉瓣狭窄。收缩早期杂音典型者见于急性重度二尖瓣关闭不全,此外,有三尖瓣关闭不全而右心室收缩压正常时可表现为收缩早期杂音,很小的室间隔缺损杂音也只局限于收缩早期。收缩晚期杂音典型者见于二尖瓣脱垂。

舒张期杂音根据杂音开始的时间可命名为舒张早期、舒张中期和舒张晚期杂音。根据杂音来源于心脏的左侧或右侧,舒张中期杂音随同第二心音主动脉成分或肺动脉瓣成分开始。舒张中期杂音是在第二心音后的一个清楚的周期之后开始。舒张晚期杂音(又称收缩期前杂音)是在舒张晚期开始。舒张早期杂音见于主动脉瓣关闭不全和肺动脉瓣关闭不全。舒张中期杂音主要见于风湿性心脏病二尖瓣狭窄;在没有房室瓣(二尖瓣和三尖瓣)阻塞时,如果流经房室瓣的血容量和血流速度明显增加,造成房室瓣的相对狭窄,也可产生来源于二尖瓣或三尖瓣的舒张中期杂音,例如在重度二尖瓣关闭不全时可产生二尖瓣相对狭窄所致的舒张中期杂音,在重度三尖瓣关闭不全和大的房间隔缺损,可产生三尖瓣相对狭窄所致的舒张中期杂音。舒张晚期或收缩期前杂音见于风湿性二尖瓣狭窄在窦性心律时,因左心房收缩增加导致房室血流增加所致。

连续性杂音指开始于收缩期,并不间断地连续下去,通过第二心音进入全部或部分舒张期的杂音。连续性杂音的最重要特点是杂音通过第二心音而无间断,即收缩期杂音和舒张期杂音连续不间断,而在第一心音前可完全消失。另一特点是杂音在同一听诊部位最响。最熟知的连续性杂音是动脉导管未闭的主、肺动脉沟通,连续性杂音还见于动脉间异常的沟通存在压力阶差、动静脉沟通、颈静脉血流增加引起的颈静脉营营音等。

收缩期和舒张期均可听到杂音,但两期杂音并不连续时,不称为连续性杂音,如室间隔缺损合并主动脉瓣关闭不全、主动脉瓣狭窄合并关闭不全等。这类杂音称为来往性杂音。来往性杂音其收缩期杂音和舒张期杂音最响亮的位置常不在同一部位。

连续性杂音由同一病变引起,来往性杂音由两个不同的病变引起。

心脏听诊发现杂音时,须详细和准确记录其出现的时相(收缩期、舒张期、连续性),持续时限(早期、中期、晚期、全期),音调(高调、中调、低调),强度(Ⅰ~Ⅵ级),音色(吹风样、隆隆样或雷鸣样、喷射样、机器样,乐音样),音质(柔和、粗糙),音形(一贯形、递增型、递减型、菱形),最响的部位,传导方向和位置,以及体位、运动、呼吸和药物对杂音的影响。

听诊结合心音图检查对心脏杂音的判断和提高心脏杂音的听诊能力有极大的帮助。心脏杂音的特点结合心电图、胸部X线片、彩色多普勒超声心动图、心脏核素、CT或核磁显像、心导管检查以及其他检查,能对心脏疾病作出准确的诊断。

根据心杂音出现的部位和期间,对各种心脏病(不包括发绀类先天性心血管病)的鉴别诊断按表\ref{tab15-1}的顺序分别讨论。

\begin{longtable}{c}
 \caption{各听诊区心脏杂音的原因}
 \label{tab15-1}
 \endfirsthead
 \caption[]{各听诊区心脏杂音的原因}
 \endhead
 \includegraphics[width=\textwidth,height=\textheight,keepaspectratio]{./images/Image00096.jpg}\\
 \includegraphics[width=\textwidth,height=\textheight,keepaspectratio]{./images/Image00097.jpg}
 \end{longtable}

心脏听诊还应注意有无心音异常。如第一心音、第二心音有无增强、减弱及分裂,有无病理性第三心音和第四心音,有无其他额外心音如喷射音、喀喇音、开瓣音、心包摩擦音、心包叩击音、肿瘤扑落音、人造瓣膜音等。造成心音异常的原因见表\ref{tab15-2}。

\begin{longtable}{c}
 \caption{异常心音}
 \label{tab15-2}
 \endfirsthead
 \caption[]{异常心音}
 \endhead
 \includegraphics[width=\textwidth,height=\textheight,keepaspectratio]{./images/Image00098.jpg}\\
 \includegraphics[width=\textwidth,height=\textheight,keepaspectratio]{./images/Image00099.jpg}
 \end{longtable}

\protect\hypertarget{text00124.html}{}{}

\section{45 心尖区杂音}

\subsection{45.1 心尖区收缩期杂音}

\subsubsection{一、非病理性心尖区收缩期杂音}

儿童和青少年多见。心尖区收缩期杂音的响度低(Ⅰ~Ⅱ级)、柔和、吹风样,限于收缩早期或早中期(持续时间不超过收缩期的40\%~60\%),不遮盖第一心音。在心尖区最清晰,局限而不向左腋下传导,运动后杂音减弱或消失,也无心脏增大或心肌炎的征象。

\subsubsection{二、风湿性二尖瓣炎}

风湿性二尖瓣炎是风湿性心内膜炎最常见的表现之一,常引起心尖区收缩期杂音。此杂音是由于二尖瓣风湿性炎症,以及并发心肌炎致二尖瓣环扩张,引起二尖瓣关闭不全所致。常表现为Ⅱ级左右全收缩期吹风样杂音,可向左腋下传导。抗风湿治疗好转后,杂音常消失,少数可发展为慢性二尖瓣病。

\subsubsection{三、感染性心内膜炎}

急性或亚急性心内膜炎均可损坏二尖瓣而引起器质性二尖瓣关闭不全。杂音性质变化是特征性表现之一,一旦出现具有重要诊断价值。当腱索断裂或瓣叶穿孔时可出现新的杂音。

\subsubsection{四、风湿性二尖瓣关闭不全}

二尖瓣关闭不全是临床上较常见的心瓣膜病,早期可呈单纯的风湿性二尖瓣关闭不全,其后常伴发二尖瓣狭窄,两者并存。

如心尖区收缩期杂音占据收缩全期,其响度在Ⅱ级以上,音质比较粗糙,并向腋中线传导,第一心音减弱或被杂音所掩盖,杂音持续存在,并伴有左心房和左心室增大,可确定器质性二尖瓣关闭不全的诊断。早期的器质性二尖瓣关闭不全,可能只有比较响亮的收缩期杂音,尚无左心房与左心室增大的征象,需经较长时期的观察方能确定诊断。

重度二尖瓣关闭不全在心尖部常有响亮的第三心音,是由于左心室迅速充盈所致,也表明患者没有合并严重的二尖瓣狭窄。二尖瓣狭窄较重时,左心室不可能迅速充盈,不能产生第三心音。

单纯重度二尖瓣关闭不全可能伴有舒张中期杂音,是左心室扩大及流经二尖瓣血流量增多,产生相对性二尖瓣狭窄所致。由此产生的舒张中期杂音强度较低而且持续时限较短。第二心音呈宽分裂,因左心射血时间缩短,引起主动脉瓣提早关闭。单纯性二尖瓣关闭不全第一心音从不响亮,也从不延迟。

本病的典型病例通常经由临床与超声心动图检查,便可作出正确的诊断。

器质性二尖瓣关闭不全的收缩期杂音,须与其他原因的收缩期杂音相鉴别:

\paragraph{1.三尖瓣收缩期杂音}

常较近中线,近胸骨下端最响,深吸气末期增强。

\paragraph{2.室间隔缺损杂音}

最响的位置在胸骨左缘第四肋间,不向左腋下传导,比较粗糙,常伴有震颤。

\paragraph{3.非病理性收缩期杂音}

其杂音特点参见上文。这时收缩期杂音是唯一的发现,而无器质性心脏病的证据。其杂音特点参见上文。

\subsubsection{五、结缔组织病所致的器质性二尖瓣关闭不全}

硬皮病病变可累及心内膜与瓣膜,能造成心瓣膜病,累及二尖瓣。系统性红斑狼疮时,炎症可累及心肌与心瓣膜,引起二尖瓣关闭不全较多,而引起二尖瓣狭窄及主动脉瓣关闭不全少见。国内报告一组系统性红斑狼疮100例中,44例心尖区有Ⅱ级以上收缩期吹风样杂音。类风湿关节炎病变可累及二尖瓣、主动脉瓣与三尖瓣,引起关闭不全与狭窄,形成类风湿性心脏病,但以主动脉瓣病变较为多见,此点与风湿性心瓣膜病多侵犯二尖瓣不同。亚当(Ehlers
Danlos)综合征可累及二尖瓣引起关闭不全。本病特点为关节过度伸张,皮肤弹性过强、易脆,常有关节积血、脱位,骨骼和内脏畸形等。病因未明,属常染色体遗传。

\subsubsection{六、二尖瓣脱垂}

二尖瓣脱垂是指乳头肌上的二尖瓣叶脱垂入左心房,以后叶脱垂较多见,双叶脱垂次之,前叶脱垂少见。病因目前认为有冠状动脉硬化性心脏病、风湿性心瓣膜炎、心肌病、马方综合征、Turner综合征、房间隔继发孔缺损、结节性多动脉炎、外伤及瓣膜手术后等。有的病例为家族性,即所谓Barlow病,其二尖瓣特别后瓣呈黏液性变,腱索细长,周围结缔组织松弛,使瓣叶在收缩中晚期脱垂入左房。约30\%病例无明确的病因,称特发性二尖瓣脱垂。

典型体征是收缩期喀喇音及(或)收缩晚期杂音。少数不伴收缩期喀喇音与杂音者称寂静型二尖瓣脱垂。喀喇音多数在收缩中、晚期听到,少数在收缩早期听到,具拍击性,在心尖部或心尖内侧最响。收缩期喀喇音具有易变性的特点,在任何一个时期可以不出现,且可为一个或多个。站立位时喀喇音提早出现且更明显。喀喇音是因松弛的二尖瓣腱索或瓣叶在心脏收缩期突然过度拉紧或翻转而产生,故又名“腱索拍击音”。大多数病例可听到收缩晚期杂音,典型者在一个或更多的中到晚期喀喇音后出现。脱垂严重时为全收缩期杂音,大多为Ⅲ~Ⅳ级,约半数在收缩晚期增强。收缩期杂音是由于二尖瓣叶脱垂入左心房,二尖瓣口不能紧密闭合,致血液反流所引起。少数病例还可听到高频、乐性的“雁鸣音”,并可触及震颤,可能由于二尖瓣叶及腱索在适当的共鸣频率出现震动所致。患者采取坐位时可使收缩期杂音转变为明显的“雁鸣音”。凡使左心室舒张末期容量减少的因素(如深吸气、立位、Valsalva动作屏气期、吸入亚硝酸异戊酯等)可使瓣叶脱垂加重,喀喇音提前,收缩期杂音变长且增强;反之,使左心室舒张末期容量增多的因素(如深呼气、下蹲、解除Valsalva动作舒气期等)可使瓣叶脱垂减轻,喀喇音延迟,收缩期杂音缩短并减轻。

超声心动图是诊断二尖瓣脱垂最有价值的方法。

\subsubsection{七、原发性心肌病}

原发性扩张型心肌病约1/3以上病例出现心尖部收缩期杂音,这是由于心脏明显增大,产生相对性二尖瓣关闭不全所致,须与风湿性二尖瓣关闭不全的杂音相鉴别。本病心脏杂音在心力衰竭期间较响,在心力衰竭控制后即减轻或消失,而风湿性二尖瓣关闭不全时则相反,在心力衰竭期间较弱,而在心力衰竭控制后增强。此外,本病可出现第三心音、第四心音及相对性三尖瓣关闭不全杂音。

原发性肥厚型心肌病时在胸骨左缘下部心尖内侧出现喷射型收缩期杂音。又因常合并二尖瓣关闭不全,在心尖区出现全收缩期杂音。

\subsubsection{八、相对性二尖瓣关闭不全}

高血压性心脏病、贫血性心脏病、主动脉瓣病、冠心病、急性风湿性心肌炎,以及任何原因所致的心肌炎或心肌病,都可引起心脏扩张,并可由下列的原因而导致相对性二尖瓣关闭不全:

1.心肌出现病变时,二尖瓣口纤维环周围的肌肉也显得软弱,致收缩期间瓣膜口未能完全闭合。

2.心脏扩张时,附着于瓣膜的乳头肌与腱索即向下移位,如腱索不能相应地伸长,可妨碍瓣膜口的完全闭合。

相对性二尖瓣关闭不全所致的心尖区收缩期杂音,可根据原发病的存在,病因治疗(如抗贫血、抗风湿)奏效后杂音消失的特点,而与器质性二尖瓣收缩期杂音相区别。

\subsubsection{九、急性二尖瓣关闭不全(腱索断裂、乳头肌功能不全或断裂)}

急性二尖瓣关闭不全的原因主要有二个:瓣膜下二尖瓣装置(腱索或乳头肌)断裂或功能不全;感染性心内膜炎或其他原因所致瓣膜破裂或穿孔。

临床上腱索断裂常见病因有:①在慢性风湿性二尖瓣关闭不全病程中发生;②作为二尖瓣脱垂综合征的合并症,如黏液样变性引起二尖瓣瘤样扩张或破裂、腱索断裂;③胸部钝性或穿透性创伤;④感染性心内膜炎;⑤病因未明的腱索自发性断裂,以后瓣腱索受累较多。

乳头肌功能不全和断裂最常见于冠心病、急性心肌梗死,偶尔可由创伤所致。后内乳头肌血液供应来自右冠状动脉后降支(变异者来自左旋支),血供不及前外侧乳头肌丰富,故下壁心肌梗死时乳头肌功能不全或断裂较多。

急性二尖瓣关闭不全的临床表现与慢性者不同,病情常短期内迅速加重。特别是重度腱索断裂或乳头肌断裂时,大量二尖瓣反流的血液作用于顺应性差而容积未增大的左房,使左房压力短期内上升3~4倍,以致迅速出现左房衰竭性急性肺水肿,之后还可发生右心衰竭。由于心输出量明显减少,患者可出现低血压或心源性休克,如不及时治疗可迅速死亡。

体检心尖搏动有力,但无左室明显扩大的体征,有别于慢性二尖瓣关闭不全。由于主动脉瓣提早关闭,第二心音可有宽分裂,并常有第三心音和第四心音奔马律。房颤少见。急性二尖瓣关闭不全收缩期杂音为非全收缩期,而是收缩早期杂音。该杂音于收缩早至中期呈递减性质,收缩晚期减弱或消失。杂音常是低音调而柔和,很少超过3级。乳头肌功能不全杂音常有以下特点:①杂音多变,即有时杂音较响,有时很轻甚至消失;②不同心动周期中可表现为收缩早期、中期或全收缩期杂音;③心绞痛发作时杂音加强,疼痛缓解后减轻;④过早搏动时杂音较响,而过早搏动后减弱,此点与二尖瓣脱垂相似而有别于其他慢性二尖瓣关闭不全。此外,心尖区第一心音常增强有别于慢性二尖瓣关闭不全的第一心音常减弱。急性心肌梗死所致乳头肌断裂有时杂音很轻但却有严重的肺水肿征,临床上必须加以注意。杂音传导方向取决于前瓣还是后瓣受累,临床上以后瓣受累较常见,反流血流向前,冲击室间隔或主动脉根部,故杂音常在胸骨左缘及心底部最响,并向颈部传导,可误诊为主动脉瓣狭窄或室间隔缺损。若前瓣受累,反流血液流向左房后壁,杂音向左腋下、头部和脊柱传导。

急性二尖瓣关闭不全的特点是心尖区或胸骨左缘(累及后瓣或后内乳头肌断裂)新出现的收缩期杂音或原有杂音加重,伴临床症状的迅速恶化,可出现急性肺水肿征象。心脏不大。心电图有窦性心动过速,可有急性心肌梗死图形而多无左房、左室增大。X线胸片心影无明显增大而有肺水肿征。肺毛压曲线有高大的V波。超声心动图可见腱索或乳头肌断裂,大量二尖瓣反流、二尖瓣呈连枷状改变,左房、左室则无明显增大,是最可靠的诊断手段。

\protect\hypertarget{text00125.html}{}{}

\subsection{45.2 心尖区舒张期杂音}

\subsubsection{一、风湿性二尖瓣炎}

急性风湿性心脏炎患者(大部分为青少年与儿童),在心尖区可出现舒张中期雷鸣样杂音(Carey
Coombs杂音)。此杂音较轻微、柔和、短促,起始部分较响亮,或起始于第三心音之后。其发生机制可能由于瓣膜发炎致其弹性减退,二尖瓣开放受限所致。抗风湿治疗心脏炎痊愈后,此杂音常消失。

\subsubsection{二、风湿性二尖瓣狭窄}

二尖瓣狭窄一般为风湿性,只偶尔为先天性或其他原因所致。国内统计材料风湿性心瓣膜损害95\%~100\%侵及二尖瓣,单纯的二尖瓣病变占70\%~90.9\%。国内成年人慢性风湿性心瓣膜病中,约1/3至半数无明确风湿热病史。风湿性二尖瓣狭窄以女性较为多见。患者往往呈二尖瓣病面容,两颊潮红而口唇发绀,外貌也往往显得较为年轻,对提示诊断有一定的意义。

二尖瓣狭窄与主动脉瓣膜病的临床不同点是较早出现代偿功能不全症状,患者常主诉劳力时甚至安静时出现心悸、气促,此症状对诊断可有重要的提示。有时患者有反复发作的咯血史,也是提供诊断此病的线索。

单纯性二尖瓣狭窄不引起左心室增大,心尖搏动初期也无移位。如右心室增大,心尖可向外侧移位,但不向下移位。如心尖向外向下移位,则非单纯性二尖瓣狭窄,而可能合并二尖瓣关闭不全或主动脉瓣膜病。因此,心尖搏动的视诊与触诊有重要诊断意义。

触诊时可触及心尖区舒张期震颤。心尖区舒张期震颤恒为病理性,且指示二尖瓣狭窄,其他原因所致的心尖区舒张期杂音一般不伴有震颤。如病情发展发生右心室肥厚,心前区呈明显的弥漫性搏动。

风湿性二尖瓣狭窄的听诊特征有以下几点:

\paragraph{1.心尖区第一心音增强}

产生原因是二尖瓣狭窄使左心室充盈时间延长,因而二尖瓣在舒张末期张开较大,且保持在左心室腔中的最低位置,当收缩期左心室内压突然升高,瓣膜关闭血液又突然减速,于是产生较强的第一心音。此外,单纯二尖瓣狭窄左室容量小,振动的物体比较小,因而振动的频率增高,振幅增大。

\paragraph{2.心尖区舒张中期杂音}

二尖瓣狭窄的舒张中期杂音通常局限于心尖区一定部位,范围较狭小,音质粗糙,一般为隆隆样。舒张中期杂音轻的只有Ⅰ级,要使患者取左侧卧位后立即听诊,在最初3~4个心动周期才能听到,有时要在运动后取左侧卧位才能听清楚。杂音响亮者可达Ⅲ~Ⅳ级,并伴有舒张期震颤。但杂音的响度并不提示狭窄的程度,轻到中度狭窄杂音反而很响。极轻度的狭窄可以没有杂音。二尖瓣狭窄舒张期杂音有时因心脏极度顺钟向转位,致使杂音在左腋下部最为清楚,听诊时应予注意。二尖瓣狭窄舒张期杂音为舒张中期杂音,在第二心音一段时间后开始,如有二尖瓣开放拍击音,则在拍击音之后开始。而主动脉瓣与肺动脉瓣的舒张期杂音,则均紧接于第二心音之后,为舒张早期杂音。

二尖瓣狭窄的舒张中期杂音于收缩期前增强(递增型舒张期杂音)具有特征性,此收缩期前增强的杂音为高调吹风样杂音而非隆隆样杂音,易误诊为二尖瓣关闭不全的收缩期杂音。因这一递增型舒张期杂音是由于心房强力收缩所产生,当晚期病例有高度心房扩大、心房收缩力减弱,或发生心房颤动时,则此杂音不出现。

如由于二尖瓣狭窄而继发肺动脉高压,可出现相对性肺动脉瓣关闭不全,在肺动脉瓣区听到舒张早期高调哈气样递减型杂音,一般较轻柔,称史氏(Graham
Steell)杂音,此杂音也可传至心尖区。

心尖区舒张期杂音的鉴别诊断,必须参照其他特征性体征。如兼有第一心音增强与二尖瓣开放拍击音的存在,则可肯定为器质性二尖瓣狭窄(左心房黏液瘤可为例外)。如舒张期杂音起源于主动脉瓣关闭不全(Austin
Flint杂音),则应同时有主动脉瓣区舒张期杂音与周围血管征,且无左心房增大。由主动脉瓣或肺动脉瓣关闭不全传导至心尖区的舒张早期杂音,常为柔和的吹风样,而非隆隆样,且为递减型,不表现为递增型。

临床上,有两种情况可能在二尖瓣狭窄病程中舒张期杂音很轻或听不到,一种情况是因狭窄很轻,第二种情况是极度狭窄或伴有心力衰竭存在(“哑型”二尖瓣狭窄)。这些病例必须结合全面的检查及超声心动图检查结果才能作出诊断。

\paragraph{3.二尖瓣开放拍击音}

此音是诊断二尖瓣狭窄的重要体征之一。在单纯性二尖瓣狭窄有此体征者达83.8\%。在少数的二尖瓣狭窄合并关闭不全的病例也可有此体征。此音在胸骨左缘稍外第四肋间最响,紧随第二心音,但又与第二心音之间有一明显的距离。这是一种响亮、清脆而具有拍击样的声音。在下列情况下可使此音更加清楚:①坐位时多数比卧位时清楚;②仰卧位高举两下肢比平卧清楚;③深呼气状态多能使此音加强。

如有瓣膜钙化、肺血管阻力显著升高、主动脉瓣关闭不全伴有二尖瓣病、重度二尖瓣关闭不全时,二尖瓣开放拍击音常消失。一般而言,此音距第二心音的距离越短,瓣膜狭窄的程度越严重。

二尖瓣开放拍击音应与第三心音及第二心音分裂相区别。第三心音的音调较低,无拍击性质,在心尖区最响,且与第一心音和第二心音的时间距离比较均匀。第三心音极少见于二尖瓣狭窄,而常见于二尖瓣关闭不全,因此音的形成是由于左心室迅速充盈而产生震动所致,而二尖瓣狭窄时左心室不可能迅速充盈。第二心音分裂则在肺动脉瓣区最清楚,此分裂的两个声音的距离极短,平均为0.04秒,在深吸气时常较清楚。

\paragraph{4.肺动脉瓣区第二音增强}

二尖瓣狭窄时出现此体征表示有肺淤血。肺动脉瓣区第二音增强不出现于早期的二尖瓣狭窄;晚期病例当合并相对性三尖瓣关闭不全与肝大时肺动脉瓣区第二音也不增强。二尖瓣狭窄合并相对性三尖瓣关闭不全的诊断根据是:胸骨左缘与心尖区之间收缩期杂音、颈静脉搏动、收缩晚期肝脏扩张性搏动,以及高度的右心房增大。

二尖瓣狭窄的心脏形状与病情轻重有关。在早期病例,心脏大小可正常,形状也可正常。稍晚期则由于左心房增大,而形成大小正常的二尖瓣型心脏。晚期则形成增大的二尖瓣型心脏。二尖瓣型心脏的形成是由于心腰隐没所致,其原因是由于左心房增大、肺动脉段扩张,以及右心室肥厚与扩张所致的心脏转位。

X线检查对单纯性二尖瓣狭窄的诊断相当可靠。但需注意,少数无心脏病变的人也可出现轻度甚至较为明显的食管压迹移位;另一方面,临床上也有少数二尖瓣狭窄早期病例以及联合瓣膜损害,无左心房增大的X线征象。超声心动图检查对诊断二尖瓣狭窄有重要价值。

心电图早期无改变,较晚期出现右心室肥厚、电轴右移与二尖瓣型P波。心房颤动也常见。

哑型二尖瓣狭窄少见,临床提示哑型二尖瓣狭窄的表现为:风湿热病史,胸骨左缘或剑突下右室搏动有力,肺动脉瓣第二心音亢进,二尖瓣开瓣音,心尖第一心音亢进,心电图示二尖瓣型P波及右室肥厚,X线显示左房增大和肺动脉高压等。当患者有上述表现时,即使无舒张期杂音,也应注意哑型二尖瓣狭窄的可能性。超声心动图可明确诊断。

\subsubsection{三、主动脉瓣关闭不全}

在严重的单纯性主动脉瓣关闭不全病例,有时可在心尖区听到低音调的隆隆样舒张期杂音,收缩期前增强,称为弗氏(Austin
Flint)杂音。此杂音是一种功能性杂音,它不在代偿功能良好的主动脉瓣关闭不全中出现,而仅在左心室衰竭的情况下出现,且在代偿功能恢复时又重新消失。弗氏杂音的发生机制是:当主动脉瓣关闭不全心脏舒张时,大量血液从主动脉反流入左心室,将二尖瓣的前瓣冲起,造成相对性二尖瓣狭窄。弗氏杂音与器质性二尖瓣狭窄舒张期杂音有时不易鉴别,鉴别要点为:①弗氏杂音较为柔和,多不伴有震颤;②不伴有二尖瓣开放拍击音与心尖区第一心音增强;③不伴有明显的左心房增大;④心电图显示左心室肥厚,而无右心室肥厚与二尖瓣型P波。

\subsubsection{四、单纯风湿性二尖瓣关闭不全}

本病有时可在心尖区出现舒张中期隆隆样杂音。此杂音持续时间较短(0.12~0.25秒),一般不伴有收缩期前增强,杂音的起始部分多较响亮,或开始于响亮的第三心音之后。杂音强度有时可达Ⅵ级,易误诊为二尖瓣狭窄合并关闭不全。此杂音的产生是由于心脏收缩时,大量血液从二尖瓣口反流入左房,于舒张期又再度流入左室,使舒张期通过二尖瓣口的血流量明显增多所致。但在单纯二尖瓣关闭不全时,心尖区第一音减弱,且左心室增大,不符合二尖瓣狭窄的诊断。

\subsubsection{五、先天性心血管病}

动脉导管未闭可在心尖区出现短促低调的舒张期杂音。杂音产生的机制主要是由于伴有大量左至右的分流,通过二尖瓣的血流量增多,致在左心室快速充盈期产生一低音调而短促的非递增型舒张中期杂音。鲁登伯(Lutembacher)综合征的病理改变是房间隔缺损伴二尖瓣狭窄。此时出现的心尖区隆隆样舒张期杂音,乃由于先天性二尖瓣狭窄所致。

\subsubsection{六、黏多糖病Ⅰ型}

黏多糖病Ⅰ型又称承霤病(gargoylism),也称Hurler综合征和怪面病,主要见于小儿,病因与酸性黏多糖代谢紊乱有关。患者有特殊的承霤病样面容:两颞和额部突出、马鞍鼻、大鼻孔、眼裂小、口唇厚、下颌短小等,并有身材矮胖、智力障碍、角膜混浊、听力障碍、心血管或呼吸道病变、爪状手、肝脾大等表现。此病70\%以上并发心血管病变,心肌内结缔组织细胞肿胀、肥大与空泡形成,心瓣膜有结节形成与增厚;病变好侵犯二尖瓣,可出现心尖区收缩期与舒张期杂音。X线检查显示心脏普遍性增大,但无特征性心电图改变。2/3承霤病患者因心力衰竭而死亡,死亡年龄平均11岁。患者尿中含有大量酸性黏多糖(AMPS),此种物质的测定对诊断有重要意义。

\subsubsection{七、左心房黏液瘤}

有蒂的左心房黏液瘤,其临床表现往往酷似二尖瓣狭窄,包括心尖区第一心音亢进与隆隆样舒张中期及收缩期前杂音。二尖瓣口发生显著的血流障碍时,可产生阵发性呼吸困难、心悸、晕厥等症状。

患者有下列情况时提示左心房黏液瘤的可能性:①并非由于体力活动所致的心悸、呼吸困难、咯血、交替性低血压、眩晕、急性心源性脑缺血综合征发作与间歇性发热;②出现动脉性微小栓塞所致的周围性疼痛点,而无感染性心内膜炎或二尖瓣膜病的证据;③应用强心剂治疗不能改善的肺淤血;④听诊与X线检查所见类似二尖瓣狭窄合并关闭不全;⑤体位改变或长期观察时发现心杂音改变;⑥约1/3患者可听到舒张早期肿瘤扑落音。

X线检查显示左心房增大。超声心动图诊断价值甚大,显示左心房内有异常迅速移动的反射光团。超声心动图还有助于黏液瘤与巨大球形血栓形成相鉴别。左心房黏液瘤有随心动周期迅速移动的特征,而血栓则无;黏液瘤大多位于心房中隔,而血栓大多位于心房后壁。

\subsubsection{八、先天性二尖瓣狭窄}

极少见,二尖瓣呈特征性降落伞状畸形,与风湿性二尖瓣狭窄有类似症状、体征,但在幼儿期出现。

\subsubsection{九、重度二尖瓣环钙化}

重度二尖瓣环钙化属老年退行性心瓣膜病。严重二尖瓣环钙化可致二尖瓣基底部增厚、硬化,瓣叶正常活动受限,除产生二尖瓣狭窄外,部分可伴有功能性二尖瓣关闭不全。超声心动图示二尖瓣环前后缘呈强回声团块。

\subsubsection{十、其他原因所致的心尖区舒张期杂音}

主动脉瓣狭窄、贫血性心脏病、慢性缩窄性心包炎、高血压动脉硬化、甲状腺功能亢进性心脏病、完全性房室传导阻滞、心肌病、心内膜纤维性变、右室条状附壁血栓等情况,偶亦引起心尖区舒张期杂音,但通常不难与二尖瓣狭窄鉴别。

\protect\hypertarget{text00126.html}{}{}

\section{46 主动脉瓣区杂音}

\subsection{46.1 主动脉瓣区收缩期杂音}

主动脉瓣区收缩期杂音通常为器质性,有时也为功能性。器质性收缩期杂音很多时候伴有收缩期震颤。

\subsubsection{一、风湿性主动脉瓣炎}

风湿性主动脉瓣炎是风湿性心脏炎的部分表现。如风湿性心脏炎患者在主动脉瓣区出现收缩期杂音,可认为存在主动脉瓣炎。如杂音经抗风湿治疗后消失,或以后发展为慢性风湿性主动脉瓣病,即可证实曾患过急性风湿性主动脉瓣炎。

\subsubsection{二、风湿性主动脉瓣狭窄}

风湿性主动脉瓣狭窄男性多于女性。常见的症状是呼吸困难和左心衰竭。有时右心衰竭出现在左心衰竭之前,是因肥厚的室间隔向右侧膨出,侵占右心室腔,引起右心室流出道狭窄所致。

晕厥和心绞痛是突出的症状。晕厥可能导致突然死亡。

严重病例的动脉脉波幅度低、高原形(脉波上升与下降均缓慢);如合并主动脉瓣关闭不全,脉搏呈重波脉;颈动脉搏动减弱,如主动脉瓣狭窄严重,可能观察不到。血压不定,严重病例血压低。由于左心室肥厚,心尖呈抬举性搏动。听诊可听到收缩中期杂音,在主动脉瓣区最响。杂音常响亮,伴有收缩期震颤,并向右颈动脉传导。在很少见的情况下,收缩期杂音在心尖区最响。如心尖区收缩期杂音向左腋窝和左肩胛部传导,则不是主动脉瓣收缩期杂音。杂音清晰地在第一心音之后开始,在主动脉瓣关闭之前终止,而限于收缩中期。杂音的响度和狭窄的程度无密切的关系。当主动脉瓣狭窄极为严重时,收缩期杂音可能显得较短且较柔和。如并发心力衰竭,杂音可能完全消失。

如无严重狭窄,可能听到收缩期喷射附加音。第二心音常呈单音,因第二心音的主动脉瓣成分延迟而置于肺动脉瓣成分之上。有时主动脉瓣区第二心音消失,提示瓣膜严重钙化,致主动脉瓣成分消失。可能出现奔马律。后期出现相对性二尖瓣关闭不全。虽然收缩中期杂音是主动脉瓣狭窄的最早体征,但主动脉瓣狭窄患者中10\%无此杂音。

心电图常显示左心室肥厚。成人心电图改变的严重性与狭窄的程度成正比,但儿童中严重主动脉瓣狭窄者的心电图可以正常。有时有深的Q波酷似心肌梗死。偶有左束支传导阻滞或房室传导阻滞,特别是当有心力衰竭时。

X线检查示左心室边缘圆钝和突出,可见瓣膜钙化;可有狭窄后主动脉扩张,但不如先天性主动脉瓣狭窄常见。左心房可能轻度增大。左心室造影显示僵硬的圆顶形主动脉瓣膜和实际上固定的瓣膜口。

心导管检查显示左心室收缩压升高,左心室和主动脉之间的压力阶差因病例而不同,从数mmHg至200mmHg以上。压力阶差在50mmHg或以上时,提示有外科治疗的需要。如发生心力衰竭,心输出量降低,压力阶差可能显现低的假象。左心房压力曲线可能有大a波。主动脉压力曲线显示很慢的升支和低而显著的升支切迹。主动脉瓣狭窄的程度越严重,升支上的切迹越低。主动脉平均血压低。

超声心动图可显示主动脉瓣狭窄的程度和狭窄前后的压力阶差。

\subsubsection{三、先天性主动脉口狭窄}

先天性主动脉口狭窄约占先天性心脏病的5\%,按病变部位可分为三种类型:①主动脉瓣膜狭窄;②主动脉瓣下狭窄;③主动脉瓣上狭窄。

\paragraph{1.先天性主动脉瓣膜狭窄}

此型最为常见。病理上二叶主动脉瓣畸形最多,占70\%。融合的左、右冠瓣叶与无冠瓣叶之间仅留下狭小的瓣孔。有的病例主动脉瓣仅有一个瓣叶,呈隔膜状,狭小的瓣孔位于瓣膜的中央或偏向一侧。

听诊可闻收缩中期杂音。杂音起于收缩期喷射音或第一心音之后,止于第二心音主动脉瓣成分之前,在胸骨右缘第一、二肋间最响,响度一般为Ⅲ~Ⅳ级,多伴有震颤。杂音向胸骨上窝及沿颈动脉传导,亦向胸骨左缘及心尖部传导。杂音的强度和狭窄的程度不成正比,这点不同于肺动脉瓣狭窄。但轻度狭窄的杂音一般较轻,发生心衰时杂音亦减弱。

在心尖部和胸骨左缘有时可闻喷射性的乐音样杂音,是狭窄的瓣膜震动所产生,其强度一般不如胸骨右缘的杂音响亮。

第二心音主动脉瓣成分正常或增强,严重狭窄者可减弱。第二心音呈正常分裂时,表示狭窄为轻度。重度主动脉瓣狭窄时,第二心音呈反常分裂。

常有收缩早期高调喷射音,在胸骨右缘第二肋间或心尖部最响。喷射音的存在是先天性主动脉瓣膜狭窄的听诊特征之一,在鉴别诊断上有重要意义。

第三心音常出现于儿童或左心衰竭的患者。严重狭窄的病例多出现第四心音。此外,约1/8~1/5的病例可听到Ⅰ~Ⅱ级柔和的主动脉瓣关闭不全舒张早期杂音。

儿童期多无特殊症状,青少年时期才逐渐出现明显的呼吸困难、晕厥和胸痛。

轻度主动脉瓣膜狭窄心电图可正常,中、重度患者可出现左心室肥厚、劳损。

胸部X线平片轻症患者心影大小可正常,严重患者左心室增大,可见升主动脉狭窄后扩张。

超声心动图检查可见主动脉瓣膜开放受限,呈单瓣、双瓣或四瓣畸形。

左心导管检查可测定主动脉瓣两侧压力阶差。左室造影可确定狭窄部位、解剖形状及狭窄程度。

\paragraph{2.主动脉瓣下狭窄}

此型在先天性主动脉口狭窄中约占9\%,狭窄病变位于主动脉瓣环的下方,又分为三型:①孤立性主动脉瓣下狭窄;②主动脉瓣下管状狭窄;③特发性肥厚性主动脉瓣下狭窄。

先天性孤立性主动脉瓣下狭窄是在左心室流出道有一纤维环,其中央仅有小孔造成左心室血流通道狭窄。此型临床症状、听诊与先天性瓣膜狭窄极相似,但听诊上有两点不同:除个别外,绝大多数无收缩早期喷射音;第二心音的主动脉瓣成分正常或减弱。超声心动图及左心导管检查可确定诊断。

先天性主动脉瓣下管状狭窄是在左室流出道有肌肉纤维的管状狭窄,极少见。其症状与先天性主动脉瓣膜部狭窄相似,但收缩中期杂音在胸骨左缘第二至第四肋间最响,Ⅲ~Ⅳ级,粗糙,不少病例可在胸骨左缘第三肋间听到Ⅰ~Ⅲ级舒张早期递减性杂音,个别病例可听到收缩早期喷射音,少数病例有第三心音和第四心音。

特发性主动脉瓣下肥厚狭窄实际上应称为原发性肥厚型心肌病,较多见,其收缩期喷射性杂音于胸骨左缘下部最响,将在此节专门叙述。

\paragraph{3.主动脉瓣上狭窄}

此型极少见,通常是紧接主动脉窦之上有一局限的节段性狭窄,狭窄后的主动脉正常或缩小。此外,可有纤维隔膜状狭窄或升主动脉普遍性狭窄。

约1/3患者有家族史,男性发病较多。患者多有特殊面容:前额宽、颊呈袋状、两眼距离宽、内眦赘皮、鼻翼上翻、宽嘴唇、尖下巴、牙齿错位咬合、声音低而带金属样、智力发育迟滞等。右侧肱动脉压常高于左侧20mmHg以上,并常有高钙血症。

听诊粗糙响亮的喷射性收缩期杂音在胸骨右缘第一肋间或胸骨上窝最响,并向右侧颈部传导。第二心音主动脉瓣成分正常或减弱,无喷射音,可以有主动脉瓣舒张早期杂音,但不常见。

心电图常有显著的ST段和T波改变。X线检查左心室常中度增大,但无狭窄后升主动脉扩张,主动脉弓部小或消失。超声心动图检查常只能发现接近主动脉瓣部位的升主动脉管腔狭窄或异常回声。左室造影检查可确定狭窄部位、程度及解剖形状。

\subsubsection{四、主动脉硬化}

主动脉硬化在老年人,主动脉瓣环发生硬化,硬化性变可蔓延至主动脉瓣,并产生收缩期杂音。此型主动脉瓣狭窄临床意义甚少,只发生于老年人,与风湿性者发病年龄不同,其响度也较弱,音质也无后者的粗糙与搔抓样。

\subsubsection{五、其他原因所致的主动脉瓣区收缩期杂音}

梅毒性主动脉瓣关闭不全、升主动脉扩张(如主动脉缩窄)、继发性高动力性综合征(如脚气病性心脏病及甲状腺功能亢进性心脏病)等,也可引起主动脉瓣区收缩期杂音,通常临床意义甚少。

\protect\hypertarget{text00127.html}{}{}

\subsection{46.2 胸骨左缘第三、四肋间收缩期杂音}

\subsubsection{一、室间隔缺损}

右心室内腔是一条由右室流入道与右室流出道(动脉圆锥)所构成的管道。此两者所构成的角有一条厚的肌性嵴,称为室上嵴。此嵴是动脉圆锥与右心室的其余部分的分界。

室间隔缺损一般按解剖部位分为嵴上型、嵴下型、隔膜后型和肌型。

1.嵴上型
不常见,缺损很高,直接在主动脉瓣和肺动脉瓣之下,主动脉瓣环可能因缺乏支持致瓣叶脱垂,引起主动脉瓣关闭不全。

2.嵴下型
直接在嵴下和三尖瓣之下,在室间隔的膜部,是最常见的类型。从左心室观察,缺损位于主动脉瓣环之下。

3.隔膜后型 传统上又称房室通道缺损。缺损位置较低,更靠后。

4.肌部缺损 此型最少见,缺损更低,在间隔肌部,最常见的是近心尖部。

室间隔缺损的严重程度,在临床上可区分为以下三种:

1.小至中等的缺损
肺动脉血流量和压力以及肺血管阻力正常或接近正常。缺损小、分流量小的病例,相当于以往所称的Roger病,一般无症状。

2.严重的缺损
肺动脉压力和肺血管阻力升高,但未达到体动脉的水平,因此不出现发绀。

3.艾森曼格(Eisenmenger)综合征 缺损直径≥2cm,出现右向左分流,出现发绀。

小的甚至中等度的室间隔缺损无发绀,且常无症状。如分流量很大,则出现左至右分流的常见症状,即呼吸困难、反复发作的支气管炎和发育停滞。婴儿时期可发生心力衰竭,多数于1岁内死亡。小的缺损较易并发亚急性感染性心内膜炎。大的缺损常引起心前区隆起,心脏搏动弥散。听诊在胸骨左缘第三、四肋间闻及响亮粗糙的全收缩期杂音,向心前区广泛传导,有时颈部、背部亦可听到。杂音最响处可触及震颤。但如缺损很小,则杂音柔和可呈喷射性。当缺损在间隔的肌部时,杂音最强的位置可能在心尖区。嵴上型缺损的杂音近肺动脉瓣区。呼气时杂音增强。约半数患者因通过二尖瓣的血流量增加,在心尖区有舒张中期杂音,表示肺血流量超过体血流量的2倍。如分流方向相反,此杂音即消失。但须注意不少先天性二尖瓣畸形与室间隔缺损并存,可以引起相同的舒张中期杂音。因左心室充盈迅速,常有响亮的第三心音。在小的缺损时,第二心音的响度正常,第二心音宽分裂。如有肺动脉高压,则第二心音的肺动脉瓣成分增强,出现较早,因而心音分裂的距离变窄,第二心音甚至可以变成单一。如已发生漏斗部肥厚,则第二心音肺动脉瓣成分减弱(或听不到)和延迟。如肺动脉高压显著,则出现肺动脉瓣关闭不全的递减型舒张早期杂音。分流量大时周围动脉的搏动减弱。

心电图:小的室间隔缺损的心电图正常。较大的室间隔缺损显示左心室舒张期容量负荷过重,左心室导联的R波高、Q波深、T波直立。如分流量较大和肺血管阻力升高,可以出现右心室肥厚心电图。完全性或不完全性右束支传导阻滞也可以出现。

X线检查:小的室间隔缺损,其心影正常;中度以上缺损心影增大,肺动脉圆锥突出,可见肺动脉的扩张和搏动,肺血流量增多,左心房增大,两心室增大,主动脉结缩小。当发展到肺动脉高压时,心脏增大以右室为主,肺动脉圆锥及肺门血管影显著扩张。

超声心动图:可显示左房、左室内径增大,伴肺动脉高压时右室、右室流出道和肺动脉也有增宽。二维超声显像可直接看到室间隔回声中断。彩色多普勒检查可估计缺损部位、大小及分流方向。

心导管检查与选择性心血管造影右心室血氧含量较右心房高出0.9容积\%以上或平均血氧饱和度高出3\%以上,据此作出心室水平左至右分流的诊断。但必须除外右心室血氧含量增高的其他原因,例如:①动脉导管未闭或主-肺动脉隔缺损合并肺动脉瓣关闭不全;②主动脉窦动脉瘤穿破入右心室。当分流量小时,右心室的血氧饱和度可以正常,可用指示剂方法以证明分流。小至中等度的室间隔缺损,其右心室和肺动脉的压力以及肺血管阻力正常或只轻度升高。严重的室间隔缺损时可见:①右心室和肺动脉压力升高,但低于左心室的压力;②肺血流量和肺血管阻力增加;③除非有心力衰竭或合并三尖瓣畸形,肺毛细血管嵌入压正常或轻微升高。

左心造影从左前斜位观察,可见从左心室注入的造影剂进入右心室。注入造影剂于右心室中,对除外其他先天性畸形如矫正型大血管错位(corrected
transposition)等,有极大的诊断价值。

\subsubsection{二、婴幼儿非病理性收缩期杂音}

婴幼儿时期在胸骨左缘通常有非病理性收缩期杂音,呈喷射型,但不伴有震颤。儿童长大后此杂音便消失,其中少数无疑是小的室间隔缺损,在发育过程中自然闭合。

\subsubsection{三、右室漏斗部狭窄}

单纯右室漏斗部狭窄较少见,仅占肺动脉狭窄的8\%,其临床表现与肺动脉瓣狭窄同,主要因狭窄程度而异。单纯漏斗部狭窄的杂音多在胸骨左缘第三或第四肋间最响,需与室间隔缺损鉴别。室间隔缺损的杂音常为收缩全期,并覆盖第二心音;漏斗部狭窄的杂音为收缩中期,因此肺动脉瓣区第二心音不被杂音所覆盖。如狭窄严重,则第二心音的肺动脉瓣成分往往减弱或消失,与室间隔缺损时第二心音肺动脉瓣成分常增强不同。在中等度狭窄时,第二心音分裂宽,如同室间隔缺损。当室间隔缺损大时,可有二尖瓣区舒张中期杂音,而漏斗部狭窄时无此杂音。又当室间隔缺损大时,通过肺动脉瓣的血流量增加,可以产生肺动脉瓣区收缩中期杂音,但常被粗糙的全收缩期杂音所掩盖。X线平片、超声心动图可鉴别漏斗部狭窄与室间隔缺损。右心导管检查可确诊。

\subsubsection{四、二尖瓣关闭不全}

二尖瓣关闭不全可在胸骨左缘第三、四肋间闻及收缩期杂音,但杂音为全收缩期反流性,在近心尖区最响,并向腋中线传导。

\subsubsection{五、主动脉瓣狭窄}

主动脉瓣狭窄可产生沿胸骨左缘或在心尖区最响的收缩中期杂音,可以伴有震颤,第二心音减弱。如主动脉瓣狭窄严重,则心电图上左心室肥厚的征象也严重,且左心室导联T波倒置,与单纯性室间隔缺损不同。

\subsubsection{六、房间隔缺损}

大的房间隔缺损分流量大,在胸骨左缘第二、三肋间可听到Ⅲ~Ⅳ级喷射性收缩期杂音,少数伴有轻微震颤,需与室间隔缺损鉴别。

永存房室共道在胸骨左缘也出现粗糙的收缩期杂音,可伴有震颤,听诊难与大的室间隔缺损相区别。

\subsubsection{七、原发性肥厚型心肌病}

本病的典型听诊体征为胸骨左缘第三、四肋间Ⅱ~Ⅲ级喷射性收缩期杂音,详细参见第52节。

\subsubsection{八、三尖瓣关闭不全}

三尖瓣关闭不全的全收缩期杂音位置较低,常在胸骨体下端或剑突左侧最响,深吸气末期增强,不伴震颤;室间隔缺损杂音的位置较高,音质较粗糙,深吸气时不增强,常伴有收缩期震颤。

\protect\hypertarget{text00128.html}{}{}

\subsection{46.3 主动脉瓣区舒张期杂音}

主动脉瓣区或(及)第二主动脉瓣听诊区(胸骨左缘第三、四肋间)出现舒张期杂音,是主动脉瓣关闭不全的重要体征。主动脉瓣关闭不全是常见的心瓣膜病。

主动脉瓣关闭不全可由于主动脉瓣受累或升主动脉扩张致使主动脉瓣环受累扩张所致。主动脉瓣关闭不全可由多种原因引起,又可分为慢性和急性主动脉瓣关闭不全。

\subsubsection{一、风湿性主动脉瓣关闭不全}

风湿性主动脉瓣关闭不全常在瓣膜病变发生后数年至10年以上,方出现代偿功能不全的症状。出现代偿功能不全症状之后,病情大多迅速发展,此时往往以呼吸困难为最突出的症状,也常有心绞痛,约10\%患者可发生猝死。

舒张早期出现的哈气样或泼水样递减型杂音,是主动脉瓣关闭不全最主要的体征。该杂音通常有以下特点:①杂音在胸骨左缘第三、四肋间最清楚,强度常超过胸骨右缘第二肋间听诊区;患者取坐位并前倾,深吸气后呼气屏气,用膜式听诊器紧贴胸壁听诊时杂音最清楚,此法最宜用于杂音轻微的患者。②杂音与第二心音的主动脉瓣成分同时出现,因此听起来常掩盖第二心音。③杂音持续时间常与关闭不全严重程度有关,轻者约占舒张期的1/3,中、重度者约占舒张期2/3至全舒张期,但极重度者杂音反而缩短和变轻。④当左心功能良好时,则杂音越响亮越长,表明关闭不全也越重;当发生心力衰竭时,则杂音可变得柔和而短促。⑤采用增强外周阻力的体位如下蹲位,可使杂音增强。

单纯主动脉瓣关闭不全在主动脉瓣听诊区往往可听到不同程度的喷射性收缩期杂音。有时该杂音在第二主动脉瓣听诊区及心尖区也甚明显。在主动脉瓣关闭不全时出现收缩期杂音,不应随便诊断为同时合并主动脉瓣狭窄。此杂音形成是由于大量的血流急速射入主动脉引起相对性主动脉瓣狭窄所致。如同时合并主动脉瓣狭窄,则主要的诊断根据并非仅仅是收缩期杂音,而是收缩期震颤与脉搏的触诊。此时病者无明显的水冲脉。主动脉瓣关闭不全时第一心音常减弱;严重主动脉瓣关闭不全第二心音的主动脉瓣成分减弱或消失;严重主动脉瓣关闭不全可听到第三心音和主动脉瓣区收缩早期喷射音。此外,严重的主动脉瓣关闭不全在心尖区或可听到低调隆隆样舒张中期杂音,称弗氏杂音(Austin
Flint)杂音,其产生机制是心脏舒张期大量血液反流入左心室,将二尖瓣前瓣冲起,造成相对性二尖瓣狭窄所致。

主动脉瓣关闭不全的另一重要体征是周围血管征,包括脉压增大、水冲脉、枪击音、杜氏(Duroziez)二重音等。

心电图可正常,病变严重者可表现为电轴左偏、左室肥大、劳损。X线检查轻症者可无异常发现,病变严重者可见心影扩大,左室搏动明显增强,主动脉增宽,呈靴形心。超声心动图可见主动脉瓣挛缩、变形,主动脉瓣不能完全闭合而呈双线,彩色多普勒检查可见舒张期主动脉血液向左室反流,并可根据反流束至左室的部位判断主动脉反流的严重度。

\subsubsection{二、梅毒性主动脉瓣关闭不全}

梅毒性主动脉瓣关闭不全临床表现与风湿性者大致相同。本病发病往往在中年以后,患者有性病史,抗体反应多数阳性,舒张期杂音向胸骨右缘传导常较向左缘传导明显,如伴有主动脉瓣区收缩期杂音,音调较低,也无二尖瓣狭窄的征象。X线胸透常发现主动脉增宽。一旦发生心力衰竭,病情往往迅速恶化。本病并发冠状动脉口狭窄者较多,心绞痛发作较常见。风湿性主动脉瓣关闭不全可能合并不同程度的主动脉瓣狭窄,因而能使回流的血液减少;梅毒性主动脉瓣关闭不全则否,因而其左心室增大的程度往往较风湿性主动脉瓣关闭不全时显著。

\subsubsection{三、二叶主动脉瓣}

二叶主动脉瓣可能是最常见的心脏发育异常。基本的缺损是二瓣叶代替了正常的三瓣叶。二叶主动脉瓣可作为成年人的钙化性主动脉瓣狭窄或主动脉瓣关闭不全的发病基础。

二叶主动脉瓣本身不引起任何症状或体征,出现并发症之前只能依据超声心动图诊断。并发症包括主动脉瓣钙化、狭窄、关闭不全和感染性心内膜炎。如已知患者过去无杂音,当感染性心内膜炎引起主动脉瓣区杂音,则应怀疑有二叶主动脉瓣。二叶主动脉瓣常合并主动脉缩窄。

\subsubsection{四、高血压主动脉硬化}

高血压主动脉硬化可使主动脉瓣或(及)瓣环发生肥厚、硬化、钙化,并因主动脉扩张而引起主动脉瓣关闭不全。多见于老年病者。杂音在主动脉瓣区较为清楚,一般是Ⅰ~Ⅱ级、高音调、短的舒张早期杂音,常伴有主动脉瓣区第二音亢进。杂音产生和血压升高似无关系,但有的病例当血压下降后杂音即消失。一般不伴有周围血管征。X线胸部平片可发现主动脉延长与增宽,有时可见主动脉壁钙化影。

\subsubsection{五、马方(Marfan)综合征}

马方综合征为全身性结缔组织代谢缺陷病,多有家族病史。此综合征具有骨骼畸形(典型者为四肢远端部分细长,形成蜘蛛足样指)、眼病征与心血管病征等三联症。病变好侵犯升主动脉,使动脉中层弹力纤维断裂,平滑肌萎缩,基质黏液样变性,引起主动脉根部扩张及瓣环扩大,发展成主动脉瓣关闭不全。

此综合征的心血管病变与其他原因的主动脉瓣关闭不全的鉴别,须根据家族史、发病年龄、上述的骨骼畸形与眼病征。

\subsubsection{六、主动脉瓣脱垂}

主动脉瓣脱垂与二尖瓣脱垂相似,系主动脉瓣黏液瘤样变性与退行性变所致。随着超声心动图的广泛应用,发现不少单纯性主动脉瓣关闭不全的原因为主动脉瓣脱垂。中山医学院报道单纯性主动脉瓣关闭不全行主动脉瓣置换术的患者中,以主动脉瓣脱垂为病因的比例很高。

\subsubsection{七、其他原因所致的主动脉瓣关闭不全}

\paragraph{1.重症贫血}

可引起左室与主动脉瓣纤维环扩张与血流加速,而产生相对性主动脉瓣关闭不全,出现主动脉瓣区舒张期杂音;杂音在第二主动脉瓣听诊区较清楚,贫血纠正后杂音消失。

\paragraph{2.类风湿性病变}

可直接损害主动脉瓣,引起畸形而产生主动脉瓣关闭不全,国内有个别病例报告。夹层主动脉瘤可由于夹层内血肿,使瓣环松动或撕裂,妨碍瓣叶闭合而引起主动脉瓣关闭不全,出现主动脉瓣区舒张期杂音,其杂音的性质无特别,需结合临床表现方能作出诊断。

\paragraph{3.系统性红斑狼疮}

可有心瓣膜病变,主要累及二尖瓣和主动脉瓣,可能误诊为风湿性联合瓣膜病变。

\subsubsection{八、急性主动脉瓣关闭不全}

急性主动脉瓣关闭不全有以下常见病因:

1.感染性心内膜炎 尤其多见于急性感染性心内膜炎,也常见于亚急性感染性心内膜炎。感染性心内膜炎时,炎症损坏瓣膜可造成急性主动脉瓣关闭不全。心内膜炎治愈后,由于瓣膜瘢痕形成和挛缩,也可引起严重慢性主动脉瓣关闭不全。

2.主动脉根部夹层动脉瘤 可伴或不伴夹层动脉瘤破裂。常在高血压、主动脉硬化或马方综合征基础上发生。

3.主动脉窦瘤破裂 临床上以右冠状动脉窦破裂入右心室最常见。本病常合并主动脉瓣脱垂及高位室间隔缺损,故常伴有主动脉瓣关闭不全。

4.在异常或病变的主动脉瓣基础上,发生自发性破裂或急性脱垂如黏液样变瓣叶、先天性瓣膜畸形、风湿性、类风湿性、系统性红斑狼疮、强直性脊柱炎、肠源性脂代谢障碍(Whipple病)、白塞综合征等所致主动脉瓣病变,在病程演进过程中突然发生主动脉瓣破裂或脱垂。

5.胸部钝性创伤所致主动脉瓣破裂或急性脱垂。

6.主动脉瓣狭窄施行经皮球囊导管瓣膜成形术、狭窄分离术的并发症,或主动脉瓣膜置换术后瓣周漏及手术造成瓣膜损伤。

急性主动脉瓣关闭不全的临床表现和对左室血流动力学影响程度的大小,主要取决于反流量大小,其次是左室功能的基本状况。严重的急性主动脉瓣反流导致左室舒张期压力剧增而左室大小无明显改变,此外左房也不可能短期内扩大,导致左房压和肺静脉压升高,出现左心衰竭和肺水肿。

查体心尖搏动增强,但心浊音界无明显扩大。听诊心尖区第一心音减弱,左心功能不全时可产生病理性第三心音和第四心音。主动脉瓣区可出现舒张早期哈气样递减型杂音。由于急性主动脉反流使左室舒张压短期内迅速增高与主动脉舒张压很快接近,因此杂音常常于舒张中期终止。当出现左心功能不全时杂音明显减轻甚至消失,并可产生第二心音逆分裂。心尖区可出现Austin
Flint杂音。急性主动脉瓣关闭不全无周围血管征,此点也与慢性主动脉瓣关闭不全有别。

心电图主要表现为窦性心动过速,多无左室肥厚或左室高电压改变。

X线检查心胸比例可以正常,心脏无明显增大。除主动脉根部夹层外,主动脉根部不增宽,但可有两侧肺淤血、肺水肿改变。升主动脉造影可显示反流口形状及大小,对估计主动脉关闭不全程度和了解主动脉根部各种病理过程有价值。左室导管检查左室舒张末压明显升高,可>40mmHg。急性主动脉瓣关闭不全的超声心动图发现依病因不同而异。感染性心内膜炎可见瓣膜上赘生物或穿孔,舒张期可见连枷状瓣脱垂入左室流出道,收缩期返回主动脉腔内。主动脉根部夹层动脉瘤可显示假通道的双腔管。主动脉窦瘤破裂可清楚地显示扩张的窦瘤破口处。人工瓣膜并发症可检出瓣周漏。其他各种原因引起的急性主动脉瓣关闭不全,超声心动图均可检出原发病的结构改变。彩色多普勒检查及升主动脉造影可明确了解主动脉瓣反流程度。

\protect\hypertarget{text00129.html}{}{}

\section{47 肺动脉瓣区杂音}

\subsection{47.1 肺动脉瓣区收缩期杂音}

\subsubsection{一、非病理性肺动脉瓣收缩期杂音}

非病理性肺动脉瓣收缩期杂音常见于儿童与年轻人,是一种低调的柔和吹风样收缩期杂音,很少达到Ⅲ级,不伴有震颤,开始于收缩早期,但不掩蔽第一心音,在胸骨左缘第二或第三肋间最清楚,常伴有肺动脉瓣区第二心音增强或分裂。此杂音在仰卧位吸气时较清楚。其发生机制是由于血液进入肺动脉时使肺动脉发生扩张,肺动脉中血流发生漩涡运动所致。此杂音并无临床意义。

直背综合征 可见于体型瘦长的人,主要表现为胸骨左缘Ⅱ、Ⅲ肋间收缩期喷射性杂音。

直背综合征并非太少见,是由于胸椎生理性后弯消失而变直,致胸腔前后径缩短,心前间隙消失,胸骨直接压迫右室流出道,在心前区出现响亮的杂音,常被误诊为器质性心脏病。

直背综合征的诊断在于认识其特点。如有怀疑病例,嘱患者坐直,观察胸椎弯曲是否变直,并作X线胸部正侧位摄片,发现胸椎弯曲除变直之外均为正常,且心脏与大血管亦无异常,即可作出直背综合征的诊断。

\subsubsection{二、房间隔缺损}

房间隔缺损是先天性心脏病中最常见的类型之一,女性较多见。

房间隔缺损根据解剖病变的不同,可分为继发孔型缺损和原发孔型缺损。

\paragraph{1.继发孔型缺损}

约占房间隔缺损的70\%~90\%,又可以分为3型:

\subparagraph{(1)中央型:}

又称卵圆孔缺损型,临床上最为常见。缺损位于房间隔中部的卵圆窝。个别病例呈筛状多孔型。此型需与卵圆孔未闭鉴别。卵圆孔未闭见于20\%~25\%正常人,正常情况下左侧房间隔的原发隔如帘幕状遮盖卵圆孔,因此不产生分流,不引起血流动力学异常。仅在做右心导管检查时,导管偶可经卵圆孔插入左房。当右室压力增大如重度肺动脉瓣狭窄时,血流可推开遮盖卵圆孔的原发隔,由右房进入左房,产生右向左分流。

\subparagraph{(2)上腔型:}

又称静脉窦型。位置较高,靠近上腔静脉入口处。常伴右肺静脉异位引流入右房。

\subparagraph{(3)下腔型:}

缺损位于房间隔后下方,缺损下方和下腔静脉相延续,左心房的后壁构成缺损的后缘。

继发孔型房间隔缺损20\%左右伴二尖瓣脱垂。

症状 儿童和青年期一般无症状或症状轻微,成年以后逐渐形成肺高压,以后可发生双向分流而出现发绀。房间隔缺损极少合并感染性心内膜炎。

体征 右心室增大,胸骨左缘呈抬举性搏动,可见心前区隆起和心脏弥漫性搏动。第一心音加强(三尖瓣成分增强),带拍击性。第二心音分裂宽,呼气时固定不变,呈固定分裂。肺动脉瓣区可听到收缩中期杂音,系由于肺动脉血流量增加所致。响度Ⅱ~Ⅲ级,吸气时加强。杂音之前可能有喷射附加音。可能触及收缩期震颤,但如震颤很明显,常表示合并肺动脉瓣狭窄。近胸骨左缘或右心室的心尖部可能有舒张中期杂音,是由于房间隔缺损大,左向右分流量大,通过三尖瓣的血流量明显增加所致。少数可能由于合并二尖瓣狭窄所致,称为鲁登伯(Lutembacher)综合征。有肺动脉高压时出现史氏(Graham
Steell)杂音,即肺动脉瓣关闭不全舒张早期杂音。40岁以后可能出现心房颤动或心房扑动。房间隔缺损是唯一常见的合并心房颤动的先天性心脏病。

心电图 95\%以上的患者有电轴右偏与不完全性右束支传导阻滞,偶尔有完全性右束支传导阻滞。如肺动脉压力增高,可出现右室收缩期负荷过重。如P波高而尖,提示右房增大。成人可见心房颤动。

X线检查 显示肺动脉干及其主支扩大和肺门搏动,其搏动较任何其他左至右分流的先天性心脏病明显。右心房显著增大,右心室也增大。主动脉结和左心室缩小。慢性病例在肺门的肺动脉分支可见钙质沉着。儿童的心影则不如此典型,可能不易与动脉导管未闭或室间隔缺损鉴别。

心导管检查 分流量的大小取决于缺损的大小和肺血管阻力。如有右心衰竭,右心房压力升高,可能超过左心房压力并产生右向左分流。血氧含量测定显示左至右分流在心房水平。右心房的血氧含量较上腔静脉增高1.9容积\%,或较上、下腔静脉平均血氧含量增高1.5容积\%。

右心房血氧含量增高。仍不能肯定必有房间隔缺损存在。肺静脉畸形引流入右心房的血氧含量也增高,可能酷似房间隔缺损,而二者共存亦非鲜见。右心房血氧含量增高的原因还有:左室至右房的分流(Gerbode型缺损);室间隔缺损合并三尖瓣关闭不全;主动脉窦动脉瘤穿破入右心房和冠状动静脉瘘与右心房相通。但这些畸形均不常引起诊断上的混淆。仅仅根据导管行径由右心房进入左心房,也不能肯定必有房间隔缺损,因导管可能通过解剖上仍保留的卵圆孔,但此孔已呈生理性关闭,实际上并无左至右分流。

如通过肺动脉瓣的血流量很大,右心室的收缩压可超过肺动脉收缩压20mmHg。

超声心动图显示右房、右室增大,肺动脉增宽。二维超声可发现房间隔回声中断。彩色多普勒可见左至右血液分流。

\paragraph{2.原发孔型缺损}

较继发孔型房间隔缺损少见,为低位的房间隔缺损。原发孔型房间隔缺损也可分为3型:

\subparagraph{(1)单纯型:}

缺损的下缘有完整的房室隔,二尖瓣叶和三尖瓣叶发育正常。

\subparagraph{(2)部分性房室隔缺损:}

在原发孔型房间隔缺损中较常见,除房间隔下部缺损外,伴部分房室隔缺损和二尖瓣发育异常,造成二尖瓣关闭不全。

\subparagraph{(3)完全性房室隔缺损:}

房室隔完全缺如,二、三尖瓣均有畸形、裂缺,并有室间隔上部缺损,四个房室腔相互沟通,又称完全性房室共通。

症状 原发孔型缺损常为大缺损。分流量大,一般症状出现较早,多数幼年时即有心跳、气促症状或并发心衰。

体征 听诊特点和分流量大的继发孔型房间隔缺损相同。多数病例肺动脉瓣区收缩中期杂音较响。三尖瓣区由高流量所致的舒张中期杂音也较长而响亮。右心室明显增大或有三尖瓣裂者在三尖瓣区有全收缩期反流性杂音。有二尖瓣前瓣裂者在心尖区有全收缩期杂音,并向左腋下传导。有肺动脉高压时出现史氏杂音。

心电图 特殊表现是电轴左偏合并不完全性右束支传导阻滞(继发孔型缺损呈电轴右偏)。Ⅱ导联示大S波。V\textsubscript{1}
导联示高R波,为右心室肥厚合并严重的肺动脉高压所致。偶尔也有左心室肥厚。多数患者P-R间期延长,P波可能增宽,完全性房室传导阻滞不常见。

心向量图 典型的是QRS环逆钟向转,向左、上和后,不似继发孔缺损的QRS环顺钟向转,向右、下和前。

X线检查 心脏形状与继发孔缺损相似,右房右室增大,如并有二尖瓣裂所致二尖瓣关闭不全,左心室和左心房也有增大。很少数患者无心脏增大。

右心导管检查 导管可以从右心房通至左心室,形成一个圆滑的向下的弯曲。常在心房和心室水平均有左至右分流。

左心室血管造影 有重要诊断意义。显示由于左心室流出道狭窄所致的特征性“鹅颈样”畸形,部分由于二尖瓣的附着不正常所致。

超声心动图 可显示原发孔型房间隔缺损,显示有无室间隔上段缺损和共同房室瓣的形态,房室瓣有无裂缺。彩色多普勒可进一步判断分流方向、大小及房室瓣反流的严重程度。

\subsubsection{三、鲁登伯(Lutembacher)综合征}

房间隔缺损合并二尖瓣狭窄称为鲁登伯综合征,但也有泛指为房间隔缺损合并二尖瓣病变。患者有发作性气短与心悸。心房颤动不少见。听诊有房间隔缺损所致的肺动脉瓣区收缩期喷射性杂音和二尖瓣狭窄所致的心尖区舒张中期隆隆样杂音。此外,房间隔缺损时由于通过三尖瓣的血流量增加也可出现三尖瓣区舒张中期杂音。

心电图 显示右心室肥厚与电轴右偏,但也可不存在。在Ⅰ、Ⅱ导联可见宽大的P波。

X线检查 可发现球形增大的心脏,肺动脉段高度扩张与肺动脉分支扩张(肺门搏动),右心房高度增大,肺纹理增粗与主动脉弓狭小。左房不增大或仅轻度增大,故与普通的二尖瓣狭窄有所不同。如发现二尖瓣钙化,可确诊。

左心导管检查时,房间隔缺损合并二尖瓣狭窄的唯一的证据是,左心房和左心室之间的舒张期压力阶差超过10mmHg。由于通过二尖瓣的血流量少,无舒张期压力阶差并不能排除本综合征的诊断。

在单纯房间隔缺损,有时在心尖区内侧可听到短促低调的舒张中期杂音,有人认为是由于快速的血流通过正常的三尖瓣口冲入增大的右心室所引起。此杂音随深吸气而增强,与二尖瓣狭窄的舒张中期杂音不同。单纯房间隔缺损出现此杂音时,可被误诊为鲁登伯综合征。

\subsubsection{四、先天性肺动脉狭窄}

单纯性先天性肺动脉狭窄可分为瓣膜型(约占75\%)、漏斗部型(约占15\%)和肺动脉型(占2\%)。瓣膜部和漏斗部联合狭窄称混合型,约占8\%。肺动脉型狭窄部位可在总干或其分支,常与Noonan综合征等畸形同时存在。肺动脉狭窄时,收缩期心室压力必须升高,才能将正常的右室血液喷射入肺动脉。右心室压力升高的程度与狭窄的严重程度成正比。

临床特征 轻度和中度肺动脉狭窄通常无症状。严重狭窄的症状是疲乏、劳力性呼吸困难,可出现晕厥和右心衰竭。

触诊可发现右心室抬举性搏动。听诊第一心音正常,瓣膜部轻中度狭窄于胸骨左缘第二、三肋间可听到收缩早期喷射音。胸骨左缘第二肋间可闻及响亮的收缩中期杂音,向左颈部传导,偶尔传到背部,偶尔杂音在胸骨左缘第三肋间最响。漏斗部狭窄的收缩中期杂音在胸骨左缘第三肋间最响。漏斗部狭窄的病例无收缩早期喷射音。在杂音最响的部位通常有震颤。

第二心音分裂较正常为宽,吸气时更宽。宽分裂是由于右心室喷血期延长和肺动脉瓣关闭延迟。轻度狭窄第二心音分裂时距轻度延长,严重狭窄第二心音分裂时距明显延长。由于肺动脉瓣叶活动不良,第二心音的肺动脉瓣成分柔和、减弱或不能听到。长期右心衰竭引起相对性三尖瓣关闭不全,出现胸骨左缘下段的全收缩期杂音,在吸气期增强。

重度肺动脉瓣狭窄有如下特点,可与中等度和轻度狭窄相区别:有心房音(第四心音),第一心音尖锐,喷射附加音缺如,临床上往往仅听到单一的第二心音(当主动脉瓣成分不被杂音遮蔽时)或听不到第二心音(当主动脉瓣成分被杂音遮蔽和听不到肺动脉瓣关闭音时)。

肺动脉瓣狭窄合并室间隔缺损的收缩中期杂音很短,菱峰出现较早,第二心音是单一的。吸入亚硝酸异戊酯后,室间隔完整的肺动脉瓣狭窄的杂音增强,如合并室间隔缺损,杂音则较弱。

心电图 心电图一般显示不同程度的电轴右偏、右心室肥厚和有时有不完全性右束支传导阻滞。在轻症病例,右心室收缩压低于50mmHg时,心电图正常。收缩压在50~75mmHg之间时,平均额面电轴变成垂直,Ⅰ和V\textsubscript{1}
导联S波的大小等于R波。压力在75~100mmHg之间时,平均额面电轴超过+90°,V\textsubscript{1}
导联的R/S比率超过1.0,V\textsubscript{1}
导联的R波小于5mm。右心室收缩压超过100mmHg时,平均额面电轴在+100°至±180°或-90°至+180°之间,V\textsubscript{1}
导联R/S>1.0,V\textsubscript{1} 导联R波超过5mm,V\textsubscript{5}
导联的R/S比率小于1.0,P\textsubscript{Ⅱ}
可能超过3mm,提示右房肥大。在严重病例,有右心室肥厚,V\textsubscript{1}
和其他右心前导联常有高R波、S-T段下降和T波倒置。

X线检查 轻型病例胸片正常。中、重型病例右室增大。肺血管影细小,肺野清晰。瓣膜型由于狭窄后扩张显示肺动脉段突出,而漏斗部型或混合型则肺动脉段平直,甚至凹陷。

心导管检查 主要的和有诊断意义的检查结果是右心室压力升高而肺动脉收缩压低,存在收缩期压力阶差。从肺动脉至右心室缓慢撤退心导管,可以清晰地显示肺动脉瓣狭窄的部位。

在瓣膜部狭窄,当导管从肺动脉退至右心室时,特征性的压力曲线突然升高。

在漏斗部狭窄,当导管从肺动脉退至右心室时,肺动脉和漏斗部狭窄的远端的收缩压相同,漏斗部狭窄近端的收缩压升高,与右心室的其余部分相同。

在瓣膜部和漏斗部联合狭窄,当导管从肺动脉撤退入漏斗腔(瓣膜部和漏斗部狭窄之间)时,收缩压升高,而当导管退至漏斗部狭窄的近端时,收缩压进一步升高。

右心房压力后期亦增高。在中等度和严重狭窄,心输出量减少。在严重狭窄合并卵圆孔未闭或房间隔缺损,可产生心房水平右至左分流而出现发绀,亦称法洛三联症。

超声心动图 右室、右房增大。瓣膜型狭窄显示肺动脉瓣增厚,反光增强,运动受限,肺动脉主干狭窄后扩张。漏斗部型狭窄显示右室漏出道变窄,肺动脉瓣运动及肺动脉内径正常。彩色多普勒可明确狭窄部位和狭窄程度。

\subsubsection{五、先天性特发性肺动脉扩张}

特发性肺动脉扩张是指肺动脉及其左右第一分支的单纯性扩张,在先天性心脏病中少见。由于肺动脉扩张,可出现肺动脉瓣区局限的Ⅱ~Ⅲ级收缩期杂音。肺动脉瓣区第二音常增强、分裂。右心导管检查右心室压力正常,右心室与肺动脉无明显的收缩期压力阶差。各心腔压力及血氧均在正常范围。X线检查与肺动脉造影均显示肺动脉扩张。心影正常,肺野清晰。

初次就诊时全无症状者占1/3,其余患者可有心悸或疲劳。心电图检查多正常,少数病例有右束支传导阻滞。超声心动图显示肺动脉主干扩张,心内结构无异常。部分病例肺动脉瓣区有反流。本病预后良好,不需治疗。临床意义为易被误诊为肺动脉瓣狭窄、房间隔缺损和肺动脉高压等。

\subsubsection{六、风湿性肺动脉瓣炎}

风湿性肺动脉瓣病的发病率低,一般不单独存在。中山医学院病理解剖学教研组50例风湿性心脏病尸检中,8例累及肺动脉瓣,其中7例四个瓣膜均受累及,表现为镜下急性病变。

风湿性肺动脉瓣炎常为风湿性心脏炎的部分表现,主要体征是肺动脉瓣区比较粗糙的收缩中期杂音;如此杂音在抗风湿治疗奏效后消失,可证明曾患过急性风湿性肺动脉瓣炎。

\subsubsection{七、风湿性肺动脉瓣狭窄}

风湿性肺动脉瓣狭窄少见,中山医学院病理学教研组50例风湿性心脏病尸检中,仅发现1例,此例合并三尖瓣关闭不全。

风湿性肺动脉瓣狭窄极少单独存在,主要表现为肺动脉瓣收缩中期杂音,或兼有收缩期震颤,须与先天性肺动脉瓣狭窄、先天性特发性肺动脉扩张等相区别。

\subsection{47.2 肺动脉瓣区舒张期杂音}

肺动脉瓣区舒张期杂音可起于器质性或相对性肺动脉瓣关闭不全。器质性肺动脉瓣关闭不全极少见,而肺动脉瓣相对关闭不全则多见。

主动脉瓣关闭不全舒张早期杂音可传导至肺动脉瓣区,须加以区别。二者的主要鉴别根据是:①主动脉瓣舒张早期杂音较肺动脉瓣舒张早期杂音为响,前者在胸骨左缘第三肋间最清楚,而后者在胸骨左缘第二肋间最清楚。但主动脉瓣关闭不全杂音也可很轻,而严重肺高压时的肺动脉瓣关闭不全杂音也可达Ⅳ级;②前者在呼气末增强而后者在吸气末增强;③前者的音调较后者高;④前者吸入亚硝酸异戊酯杂音减弱而后者增强;⑤前者有左心室肥厚的病征,而后者有右心室肥厚的病征;⑥前者有水冲脉等周围血管征,而后者胸部X线平片可见到肺动脉段膨隆。

\subsubsection{一、风湿性肺动脉瓣关闭不全}

风湿性肺动脉瓣关闭不全罕见,但由于二尖瓣狭窄引起的肺动脉扩张所致的相对性肺动脉瓣关闭不全则比较多见。鉴别肺动脉瓣关闭不全为器质性或相对性,临床上尚无确实的方法。如此杂音在二尖瓣分离术后消失。则可认为是由于相对性肺动脉瓣关闭不全所致。

\subsubsection{二、感染性心内膜炎所致的肺动脉瓣关闭不全}

器质性肺动脉瓣关闭不全多数由右心感染性心内膜炎引起,且为此病的部分表现。

\subsubsection{三、相对性肺动脉瓣关闭不全}

相对性肺动脉瓣关闭不全起源于肺动脉高压所致的肺动脉扩张,这种情况可见于二尖瓣膜病,急性、亚急性或慢性肺源性心脏病,原发性肺动脉高压症,房间隔缺损,以及艾森曼格(Eisenmenger)综合征等。如此杂音继发于高度二尖瓣狭窄所致的肺动脉扩张,则称为史氏(Graham
Steell)杂音------此杂音是比较柔和的、高调的、递减型舒张早期或早中期杂音,局限于胸骨左缘第二、三肋间,在吸气末增强,呼气末减弱。此杂音应与轻度主动脉瓣关闭不全的舒张期杂音鉴别,鉴别点见上文。

\protect\hypertarget{text00130.html}{}{}

\section{48 三尖瓣区杂音}

\subsection{48.1 三尖瓣区收缩期杂音}

三尖瓣病变少见,据中山医学院病理学教研组50例风湿性心脏病的尸检所见,风湿性三尖瓣病5例,无单独存在,都与二尖瓣病或(及)主动脉瓣病等并存。国内风湿性心脏病临床分析,三尖瓣病变在风湿性心脏病中占0.53\%~5.2\%。

\subsubsection{一、风湿性三尖瓣炎}

风湿性三尖瓣炎是风湿性心脏炎的部分表现。由于三尖瓣及其邻近心肌的炎症性病变,致产生三尖瓣收缩期杂音。如此杂音经抗风湿治疗奏效后消失,便可认为曾患过风湿性三尖瓣炎。中山医学院病理解剖学教研组报告50例风湿性心脏病尸检结果,22例有三尖瓣病变,其中7例累及四个瓣膜,表现为镜下急性病变,急性病例均在20岁以下。

\subsubsection{二、风湿性三尖瓣关闭不全}

风湿性三尖瓣关闭不全临床上少见。本病常与三尖瓣狭窄并存,且常与二尖瓣或(及)主动脉瓣病并发。本病主要临床表现是慢性右心衰竭的征象,肝大明显,可出现胸、腹水,与慢性缩窄性心包炎的临床表现相似。于胸骨下端可听到响亮、高调的收缩期杂音,右心室显著增大,而缩窄性心包炎时无。本病可出现肝脏收缩晚期扩张性搏动,肝脏扩张性搏动的检查方法是:将左掌放在患者的肝脏后面,右掌放在肝脏前面,嘱患者暂停呼吸,如为扩张性肝脏搏动,则明显地将两掌推开,且可观察到肝脏搏动在颈动脉搏动之后出现。

本病时胸骨下端收缩期杂音有时颇难与二尖瓣关闭不全所致的收缩期杂音相区别,且由于右室增大与心脏顺时针转位,杂音在胸骨左缘至心尖区之间最响,但此杂音不向左腋下传导,患者深吸气末时杂音增强,而二尖瓣关闭不全时不变或减弱。

\subsubsection{三、相对性三尖瓣关闭不全}

因右心室扩大引起的相对性三尖瓣关闭不全较器质性者更为多见。相对性三尖瓣关闭不全的临床表现与器质性者相同,特别多见于重症风湿性二尖瓣狭窄伴有肺动脉高压的病例。此外原发性肺动脉高压症、慢性肺源性心脏病等所致的慢性右心衰竭,也往往引起相对性三尖瓣关闭不全。

器质性与相对性三尖瓣关闭不全的临床鉴别不易;后者的杂音在心力衰竭被控制、病情好转后消失,且不伴有三尖瓣舒张期杂音。器质性三尖瓣关闭不全多伴有狭窄。超声心动图检查有助于二者的鉴别。

\subsection{48.2 三尖瓣区舒张期杂音}

\subsubsection{一、风湿性三尖瓣狭窄}

三尖瓣狭窄罕见,病因一般为风湿性,可与三尖瓣关闭不全并存。患者女性多于男性,发病多在青年期。三尖瓣狭窄使血液从右心房流入右心室受阻。因而引起右心房扩张。由于常合并二尖瓣狭窄,在此情况下常有不同程度的右心室增大。

本病的主要临床表现是慢性右心衰竭征象。重症病例常有水肿、腹水,明显的颈静脉怒张与肝大,可有收缩期前肝脏搏动。大多无明显呼吸困难。三尖瓣区(胸骨下端)可听到响亮、粗糙、低调的隆隆样舒张中期杂音,有时伴舒张期震颤。此杂音可伴有三尖瓣开放拍击音。三尖瓣狭窄与二尖瓣狭窄的舒张期杂音性质相同,有时不易鉴别;但胸骨左缘无右心室搏动增强,肺动脉瓣第二音不亢进,且此杂音在胸骨下端或其左缘较心尖区清楚,音调较二尖瓣狭窄者稍高,嘱患者右侧卧位听诊时,杂音在深吸气末增强,可与二尖瓣舒张期杂音区别。单纯性三尖瓣狭窄X线检查显示右心房增大,而右心室无增大,肺动脉也不扩张,肺野异常清朗。如患者有上述的典型杂音、明显的颈静脉搏动与肝大,以及右心房增大等病征,应注意本病。超声心动图检查可明确诊断。

\subsubsection{二、相对性三尖瓣狭窄}

大的房间隔缺损时,可在胸骨左缘第四、五肋间心尖区内侧出现短促低调的舒张中期杂音。此杂音被认为由于快速的血流通过正常的三尖瓣口冲入增大的右心室所致。

有报告法洛四联症在心尖区内侧有时也出现舒张中期杂音,认为与右室扩大引起相对性三尖瓣狭窄有关。

\subsubsection{三、右心房黏液瘤}

右心房黏液瘤比左心房黏液瘤少见,国内仅有少数病例报告。本病临床表现类似三尖瓣狭窄。临床出现体循环淤血表现,如颈静脉怒张、肝大、双下肢水肿、腹水等。静脉回流受阻可使心排血量减少,出现气促、晕厥和发绀。此外,还可出现多发性肺栓塞。卵圆孔未闭者,右心房压力的增高可致卵圆孔开放,产生心房水平的右向左分流,出现严重发绀。听诊有三尖瓣舒张中期杂音、三尖瓣开放拍击音,但肺动脉瓣第二音正常、无明显分裂。杂音有易变的倾向,发生在改变体位时,有时杂音为乐音样。颈静脉呈搏动性扩张。血压低、脉压小,静脉压增高。X线检查常显示右房增大。选择性右心房造影显示右心房内有占位性病变。超声心动图有助于诊断。

\protect\hypertarget{text00131.html}{}{}

\section{49 心底部连续性杂音}

\subsection{49.1 非病理性连续性杂音}

\subsubsection{一、颈静脉营营音}

颈静脉营营音在儿童时期常见,而在婴儿时期和成人均少见,由于血流迅速通过颈静脉进入上腔静脉引起,故非病理性。此音常为低音调的Ⅰ~Ⅲ级连续性杂音,于心室舒张早期最响,在颈根部特别是右侧最易听到,头转至对侧时右颈根部此音增强,吸气时杂音亦增强。取仰卧位或在颈静脉上加压或作Valsalva动作均可使静脉营营杂音减弱或消失。

\subsubsection{二、乳房营营杂音}

此种杂音是由于乳房血流量增加引起,见于孕妇,最易在胸骨旁的肋间上听到,左侧较右侧多见。杂音可为收缩期性、来往性或连续性。如左侧胸骨旁第二、三肋间出现此种连续性杂音,可误诊为动脉导管未闭。当孕妇仰卧时杂音最响。此杂音常出现于妊娠中期之末,分娩10周后常消失。

\subsection{49.2 病理性连续性杂音}

\subsubsection{一、动脉导管未闭}

动脉导管未闭是常见的先天性心血管病之一。动脉导管起源于左第六主动脉弓,连接肺动脉总干(或左肺动脉)与降主动脉在左锁骨下动脉开口处之下。胎儿时期动脉导管接受右心室排入肺动脉的血液,将之排入降主动脉,以供应下半身的发育。出生后数周动脉导管即失去其作用,通常在出生后1年内关闭。如逾1年仍未关闭,即为动脉导管未闭。

在动脉导管未闭时,因主动脉收缩压和舒张压通常均高于肺动脉,在全心动周期,血液流经导管,产生连续性杂音。如分流量大,即发生左心室容量负荷过重。分流可引起高动力性肺动脉高压。如肺血管阻力达到体周围动脉阻力的水平,则可能出现双向或相反方向分流。因导管常位于左锁骨下动脉的远侧,这些病者的身体上部(包括手臂)发绀较身体下部稍轻,称为差别性发绀。如导管位于左锁骨下动脉的近侧,则可能发现右手发绀较左手及双足稍轻。

临床特征 女性病者约较男性病者多2倍。分流量小的轻型病例常无症状,发育只轻度受影响。如左至右分流量大,则常有呼吸困难和肺部感染,包括支气管炎和支气管肺炎,儿童可见发育迟缓。婴儿有大的左至右分流常产生左心衰竭。成人很少发生心力衰竭和心绞痛。

体格检查 主要的体征是以第二心音为轴的长菱形连续性杂音,于收缩期之末和舒张早期最响,因这时主动脉和肺动脉之间的压力阶差最大。杂音在运动及呼气时加强。典型的杂音常在三岁以后出现,三岁以下常只有收缩期杂音。杂音的性质类似机器的杂音或隧道中火车的杂音,可能伴有连续性震颤,在肺动脉瓣区或附近最响,但有时位置较低,或较高达左锁骨之下。如出现心力衰竭,则典型的杂音可能消失。如分流方向相反,则典型的杂音无例外地消失。常有正常范围的第二心音分裂,肺动脉瓣组成部分响亮,但常被杂音掩盖。如导管大,直径1cm或以上,则出现左心室容量负荷过重,左心室收缩期延长,使第二心音的主动脉瓣成分开始延迟。此外,由于从主动脉分流入肺动脉的血流的压力,肺动脉瓣可能提早关闭。这时可出现单一的第二心音或第二心音逆分裂(第二心音主动脉瓣成分在肺动脉瓣成分之后)。因通过二尖瓣的血流增加,近二尖瓣区常有舒张中期杂音,此杂音在分流量大时出现,而合并肺动脉高压时消失。部分病例因肺动脉显著扩大,可产生肺动脉瓣相对关闭不全的舒张早期杂音。当分流量大时,舒张期血压低,出现周围血管征,如水冲脉、枪击音等。常有胸骨上窝搏动。心尖搏动正常或呈左心室增大的抬举样搏动。如分流量大或肺血管阻力升高,肺动脉搏动在左第二肋间可以触及。

心电图 轻型病例常正常。大的分流量产生容量负荷过重型左心室肥厚,左心室导联深Q波,高R波和T波。偶尔由于左心房肥大,有双峰的P波。如成年人心电图有右心室肥厚征象,则指示已出现肺动脉高压。P-R间期可能稍延长。

X线检查 肺充血与左至右分流的程度成正比。肺动脉主干扩大,搏动强烈,左心房可能轻度增大,左心室常增大,主动脉结扩大。当并发肺动脉高压时,右室也增大。导管或导管对侧的肺动脉钙化很少见。

超声心动图检查 左房、左室、主动脉内径增宽,肺动脉扩张。胸骨上凹切面可直接显示未闭的动脉导管。彩色多普勒可在动脉导管和肺动脉主干内探及收缩期和舒张期连续性红色和彩色镶嵌的高速湍流。二维多普勒超声心动图是目前诊断动脉导管未闭最佳的无创性方法,阳性率高达99\%。

心导管检查 与右心室对比,主肺动脉的血氧含量增高0.5容积\%;如分流量少,可能只在左主支发现血氧含量增高。从下肢静脉插入心导管,常较易通过肺动脉,并经动脉导管进入降主动脉,可与主-肺动脉隔缺损相鉴别。

肺动脉血氧含量高于右心室的其他原因还有:①主-肺动脉隔缺损:②主动脉窦动脉瘤穿破入肺动脉;③迷走的左冠状动脉起始于肺动脉,左冠状动脉的逆分流使氧合血到达肺动脉。

选择性主动脉造影:只有当不能决定有无动脉导管未闭或合并其他缺损时,才进行选择性主动脉造影。可见导管口在主动脉峡部形成小膨隆。造影剂通过动脉导管直接进入扩张的肺动脉,并可显示动脉导管的类型(管型、窗型或漏斗型)、粗细和长度。

\subsubsection{二、主-肺动脉隔缺损}

主-肺动脉隔缺损是位于主动脉瓣与肺动脉瓣之上1cm或约1cm的缺损,呈圆形或卵圆形,直径0.2~2cm。血流动力学改变与大的动脉导管未闭相同,但呼吸困难常见。其杂音性质、心电图及X线表现均似重症动脉导管未闭。杂音最响部位较动脉导管未闭者可较低,在胸骨左缘第三肋间,较接近中线。收缩期杂音较连续性杂音更多见,因主动脉血压与肺动脉血压实际上相等。杂音呈喷射型,通常伴有震颤。实际上,如有连续性杂音,应首先考虑动脉导管未闭。

本病在右心导管检查时,导管往往经过主动脉弓至主动脉分支(如颈动脉或降主动脉),但在动脉导管未闭常是通过未闭的导管直接进入降主动脉,而不经过主动脉弓。主动脉造影可见肺动脉干与主动脉同时显影,也有助于诊断。

\subsubsection{三、肺动静脉瘘}

肺动静脉瘘可为先天性或获得性,后者通常由于创伤引起。本病多发生在右下或右中肺叶,故杂音多在右中下肺。本病亦可发生在左上肺叶,杂音在左胸上部,此时需和动脉导管未闭鉴别。本病杂音可以是连续性,但多局限于收缩期。吸气和吸入亚硝酸异戊酯可使杂音增强。

\subsubsection{四、主动脉窦动脉瘤穿破入右心室(房)}

主动脉窦动脉瘤穿破入右心室(房),多在胸骨左缘第三、四肋间出现响亮的连续性杂音,伴连续性震颤。本病特点是起病突然,出现类似急性心肌梗死的胸痛或胸部压迫窒息感,继而呼吸困难甚至休克,随后出现右心衰竭的征象。听诊除胸骨左缘第三、四肋间出现上述连续性杂音之外,肺动脉瓣区第二音亢进。有水冲脉与周围动脉枪击音。X线检查:肺充血、肺门搏动增强、心脏进行性增大。右心导管检查:右心室水平有左至右分流,右心室压力增高。逆行主动脉造影:可以发现在主动脉显影的同时,右心室或右心房也显影,而其他心腔则不显影,有时甚至可见动脉瘤显影,可确诊本病。

\subsubsection{五、先天性冠状动静脉瘘}

右冠状动脉或左冠状动脉的回旋支(前者多见)的瘘管引流入右心房、右心室或肺动脉,流入右房、右室的占89\%,实质上是左到右分流的一种先天性畸形。引流入左心房、左心室的罕见。健康状态良好与明显的听诊体征呈鲜明的对比,病者无症状而体检发现响亮的、浅表的连续性杂音为本病特点。

如冠状动静脉瘘与肺动脉或右心房连通,连续性杂音的收缩期组成部分较响,因血液流出主要在收缩期。如冠状动静脉瘘与右心室连通,则连续性杂音在舒张期较响。如冠状动静脉瘘与左心室连通,则几乎只在舒张期才有血液流入左心室,因而只出现舒张期杂音。杂音的部位因引流入的心腔不同而异,一般在胸骨左、右缘都可能听到。引流入左心房和肺动脉者杂音在胸骨左缘第二、三肋间最清楚,需和分流量小的动脉导管未闭杂音鉴别。

X线与心电图检查对诊断帮助不大。冠状动脉造影可明确诊断。

\subsubsection{六、完全性肺静脉畸形引流}

本病是由四条肺静脉汇合成一条肺静脉通入右心房,约1/4病例在主动脉瓣区听到连续性杂音,吸气时增强。

\subsubsection{七、三尖瓣闭锁}

三尖瓣闭锁有时在心底部出现连续性杂音,是由于合并动脉导管未闭或支气管动脉-肺动脉交通支所引起。后者连续性杂音位于右侧。

\subsubsection{八、胸腔内动脉吻合术后}

胸腔内动脉吻合术后,如法洛四联症病例在左锁骨下动脉与左肺动脉吻合术后,可在左锁骨下部位听到连续性杂音。

\subsection{49.3 来往性心杂音}

\subsubsection{一、室间隔缺损合并主动脉瓣关闭不全}

嵴上缺损的位置高,直接在主动脉瓣与肺动脉瓣之下,主动脉瓣环可能缺乏支持,瓣叶脱垂可引起主动脉瓣关闭不全。此时室间隔缺损本身所产生的收缩期杂音,加上主动脉瓣关闭不全引起的舒张期杂音,可在胸骨左缘第三、四肋间听到来往性杂音,但杂音缺乏连续性。超声心动图和右心导管检查可作出诊断。

\subsubsection{二、二尖瓣关闭不全合并主动脉瓣关闭不全}

器质性二尖瓣关闭不全合并主动脉瓣关闭不全一般为风湿性。二尖瓣关闭不全的收缩期杂音加上主动脉关闭不全的舒张期杂音,可使杂音呈来往性。二者各有最响的部位,音质也不同:前者在心尖区最响,向左腋下传导,音质比较粗糙;后者在胸骨左缘第三、四肋间最响,音质较柔和。

\subsubsection{三、主动脉瓣关闭不全合并狭窄}

风湿性主动脉瓣关闭不全伴有明显狭窄的病例,可出现来往性杂音。其收缩期杂音最响部位多在胸骨右缘第二肋间,可伴有震颤,杂音向右颈传导;而舒张期杂音最响部位多在胸骨左缘第三、四肋间,向心尖区传导。

在梅毒性主动脉瓣关闭不全时,也可在胸骨右缘第二肋间或胸骨左缘第三、四肋间出现一来往性收缩期及舒张期杂音,分别向右颈动脉与心尖区传导。梅毒性主动脉瓣关闭不全出现收缩期杂音,是由于升主动脉增宽,左心室输出量大与血流增快引起。

\protect\hypertarget{text00132.html}{}{}

\section{参考文献}

1.徐南图.超声心动图的应用和进展.中华内科杂志,1994,33(9):641

2.王新房,等.四维超声心动图临床应用.中华心血管病杂志,1996,24(1):5

3.马晓曦,等.哑型二尖瓣狭窄四例.中华心血管病杂志,1994,22(4):284

4.周令仪.急性风湿性心脏炎与瓣膜脱垂.中华内科杂志,1993,32(8):527

5.姚忠贤,等.78例非风湿性主动脉瓣关闭不全临床分析.中华内科杂志,1992,31(12):776

6.解基严,等.功能性三尖瓣关闭不全的分析与治疗.中华心血管病杂志,1998,26(2)114

7.赵一举,等.特发性肺动脉扩张21例临床分析.中华内科杂志,1992,31(1):25

8.徐启林,等.二维多普勒心动图诊断动脉导管未闭的价值.中华内科杂志,1992,31(8):453

9.张志泰,陈玉平.肺动静脉瘤的诊断与治疗.中华结核和呼吸杂志,1998,21(2):114

10.唐玲娣,郑更生.先天性肺动脉瘘20例报告.中华内科杂志,1995,34(8):553

11.李予昕,等.超声心动图评价西藏地区先天性心脏病特点的初步探讨.中国超声医学杂志,2000,16(11):856-857

12.李志忠,等.应用Amplatzer封堵器治疗动脉导管未闭.中华心血管病杂志,2000,28(5):371-373

13.吴建淮,等.经皮球囊导管二尖瓣成形术治疗风湿性二尖瓣狭窄的长期随访观察.中华心血管病杂志,2000,28(2):117-119

14.齐欣,等.老年人钙化性主动脉瓣狭窄并发心肌梗死和病理特点.中华内科杂志,2000,39(2):88-90

15.王霄芳,等.孤立性心室肌致密化不全4例报告.中华儿科杂志,2002,40(2):81-83

16.尤士杰,等.超声多普勒心动图在急性心肌梗死并发室间隔穿孔预后的评价.中国超声医学杂志,2001,17(12):90-904

17.朱鲜阳,等.小儿先天性冠状动脉瘘的临床诊断与分析.中国实用儿科杂志,2001,16(8):472-473

18.黄美蓉,等.先天性心脏病合并感染性心内膜炎的诊断及治疗.中华儿科杂志,2001,39(5):267-270

19.钱杰,等.感染性心内膜炎中国诊断标准的讨论.中国循环杂志,2003,18(3):212-214

20.李冬蓓,等.超声心动图对左房黏液瘤与左房活动性血栓的鉴别诊断.中国超声诊断杂志,2002,3(10):742-743

21.韩玲,等.小儿双腔右心室的诊断:附23例临床分析.中华儿科杂志,1994,32(6):341-343

22.张玉珍,等.老年人钙化性心脏瓣膜病六年随访.中华老年医学杂志,1994,13(2):96-98

23.袁慧玲,余枢.心房黏液瘤31例临床分析.中华实用内科杂志,1994,14(2):90-91

24.徐素梅,等.完全性房室隔缺损12例分析.中华儿科杂志,1989,27(5):269-270

25.姜楞.二维脉冲多普勒在心前区双期杂音鉴别诊断中的应用.中华心血管病杂志,1987,(3):166-168

26.钱秉源.收缩期喀喇音61例临床分析.中华心血管病杂志,1979,(3):183-189

27.刘晓华.42例大动脉炎临床实验检查血管造影的研究.中华风湿病学杂志,1997,1(1):19-21

28.刘保民,等.彩色多普勒对川崎病冠脉瘤和冠状动脉瘤的诊断和鉴别诊断.中国超声医学杂志,1997,13(8):30-32

29.姜志忠,等.100例起搏器安置前、后心音图的对比研究.中国循环杂志,1997,12(1):27-30

30.肖德绵,等.室间隔缺损合并肺动脉高压的手术疗效.中华外科杂志,1996,34(5):265-266

31.张济富,等.原发性肺动脉高压50例临床分析.中华内科杂志,1996,35(5):322-325

32.周素真,等.冠状动脉瘘:国内报道67例的临床分析.中国循环杂志,1996,11(2):72-75

33.马爱群,等.感染性心内膜炎的临床变迁:38年间153例临床对比分析.中国循环杂志,1995,10(10):594-596

34.李茂亭,等.主动脉瘤临床研究.中国循环杂志,1995,10(6):334-336

35.陈君柱,等.经导管Rashkind双伞闭合器关闭动脉导管未闭.中华心血管病杂志,1995,23(3):199-200

36.马旺扣,等.干下漏斗部室间隔缺损的外科治疗.中华胸心外科杂志,1995,11(3):138-139

37.张玉威,等.92例右心室双出口临床表现及其诊断.中华心血管病杂志,1995,23(2):116-118

38.蒋雄刚.小儿室间隔缺损术后残余漏.中华小儿外科杂志,1995,16(2):77-78

39.余翼飞,等.先天性主动脉褶叠5例报告.中华外科杂志,1995,33(1):46-47

40.何建平,等.新生儿先天性心脏病145例分析.中华围产医学杂志,1999,2(4):218-221

41.蒋雄京,等.大动脉炎对心脏瓣膜的影响.中国循环杂志,1999,14(5):301-302

42.李志忠,等.Amplatzer方法介入性治疗动脉导管未闭.中国循环杂志,1999,14(S:S):27-29

43.勒斌,等.超声心动图在先心病手术后病理性杂音鉴别诊断中的价值.中国超声医学杂志,1998,14(2):46-48

44.张楚武.哑型二尖瓣狭窄.中华内科杂志,1981,20:686

45.周令仪.开瓣音的临床意义.广东医学,1984,5(4):1

46.周令仪.松弛瓣膜综合征16例临床及病理分析.中华内科杂志,1986,25:149

47.赵健忠.马凡氏综合征合并二尖瓣脱垂一例报告,天津医药,1981,9(2):123

48.王增顺.系统性红斑狼疮心瓣膜损害20例临床病理分析.中华内科杂志,1987,26:653

49.上海市第六人民医院.二尖瓣脱垂综合征(附20例临床资料分析).上海医学,1978,1(1):22

50.Abrams J.Synopsis of cardiac physical diagnosis.2nd
ed.Boston:Butterworth Heinemann,2001.

51.Ecchells E,et al.Dose this patient have an abnormal systolic
murmur?JAMA 277:564,1997.

52.O'Rourke R.Approach to the patient with a heart murmur.InBraunwald
E,Goldman L(eds):Primary Cardiology.2nd
ed.Philadelphia:Elsevier,2003:155-173.

53.*BraunwaldE,Perloff J.Physical examination of the heart and
circulation.in Braunwald E:Heart Disease. 7th
ed.北京,人民卫生出版社,2006:77-106.

\protect\hypertarget{text00133.html}{}{}


\appendix
\chapter{英中文对照索引}

\subsection*{A}

\begin{longtable}[]{p{6cm}p{6cm}}
\toprule
\endhead
abstinence syndrome & 戒断综合征\tabularnewline
acquired immune deficiency syndrome, AIDS &
获得性免疫缺陷综合征\tabularnewline
acute alcohol intoxication & 急性乙醇中毒\tabularnewline
acute brain syndrome & 急性脑病综合征\tabularnewline
acute lethal catatonia & 急性致死性紧张症\tabularnewline
acute reactive psychosis & 急性反应性精神病\tabularnewline
acute stress disorders & 急性应激障碍\tabularnewline
acute stress psychosis & 急性应激性精神病\tabularnewline
acute stress reaction & 急性应激反应\tabularnewline
addictive substance & 成瘾物质\tabularnewline
Addison disease & 阿狄森病\tabularnewline
adjustment disorders with mixed disturbance of emotions and conduct &
心境和品行混合障碍为主的适应障碍\tabularnewline
adjustment disorders with predominate disturbance of conduct &
品行障碍为主的适应障碍\tabularnewline
adjustment disorders & 适应障碍\tabularnewline
agoraphobia & 场所恐惧症\tabularnewline
alcohol dependence & 酒依赖\tabularnewline
alcoholic delusion & 酒中毒妄想症\tabularnewline
alcoholic dementia & 酒中毒性痴呆\tabularnewline
alcoholic hallucinosis & 酒中毒幻觉症\tabularnewline
alprazolam & 阿普唑仑\tabularnewline
alternating personality & 交替人格\tabularnewline
altruistic suicide & 利他性自杀\tabularnewline
Alzheimer's disease, AD & 阿尔茨海默病\tabularnewline
ambitendency & 矛盾意志\tabularnewline
ambivalence & 矛盾情感\tabularnewline
amentia & 精神错乱\tabularnewline
amitriptyline & 阿米替林\tabularnewline
amnesia & 遗忘症\tabularnewline
amnestic syndrome & 遗忘综合征\tabularnewline
amok & 杀人狂\tabularnewline
amphetamine & 苯丙胺\tabularnewline
anomic suicide & 失范性自杀\tabularnewline
anorexia nervosa & 神经性厌食\tabularnewline
anti-anxiety drugs & 抗焦虑药\tabularnewline
anticonvulsant & 抗癫痫
药\tabularnewline
anti-depressant drugs & 抗抑郁药物\tabularnewline
anti-depressants & 抗抑郁药\tabularnewline
anti-manic drugs & 抗躁狂药\tabularnewline
anti-psychotic drugs & 抗精神病药\tabularnewline
anti-psychotics & 抗精神病\tabularnewline
antisocial personality disorder & 反社会型人格障碍\tabularnewline
anxiety disorder & 焦虑症\tabularnewline
anxiety neurosis & 焦虑性神经症\tabularnewline
anxiety & 焦虑\tabularnewline
anxious personality disorder & 焦虑型人格障碍\tabularnewline
apathy & 情感平淡\tabularnewline
aprosexia & 注意涣散\tabularnewline
Argyll-Robertson's pupils & 阿-罗氏瞳孔\tabularnewline
Asia poppy plant & 罂粟\tabularnewline
ataraxics & 安适剂\tabularnewline
attempted suicide & 自杀未遂\tabularnewline
attention-deficit hyperactivity disorder &
注意缺陷多动障碍\tabularnewline
attention & 注意\tabularnewline
auditory hallucination & 幻听\tabularnewline
autism & 孤独症\tabularnewline
\bottomrule
\end{longtable}

\subsection*{B}

\begin{longtable}[]{@{}ll@{}}
\toprule
\endhead
bad trip & 倒霉之旅\tabularnewline
behavioral therapy & 行为治疗\tabularnewline
benzodiazepines & 苯二氮䓬
类\tabularnewline
bipolar disorder & 双相障碍\tabularnewline
blocking of thought & 思维中断\tabularnewline
body dysmorphic disorder, BDD & 躯体变形障碍\tabularnewline
borderline personality disorder & 边缘型人格障碍\tabularnewline
brief depressive reaction & 短期抑郁反应\tabularnewline
brief psychiatric rating scale & 简明精神病评定量表\tabularnewline
bulimia nervosa & 神经性贪食\tabularnewline
buproion & 安非他酮\tabularnewline
buspirone & 丁螺环酮\tabularnewline
\bottomrule
\end{longtable}

\subsection*{C}

\begin{longtable}[]{@{}ll@{}}
\toprule
\endhead
caffeine & 咖啡因\tabularnewline
cannabinoids & 大麻素\tabularnewline
cannabis sativa & 大麻植物\tabularnewline
cannabis & 大麻\tabularnewline
carbamazepine & 卡马西平\tabularnewline
catatonic excitement & 紧张性兴奋\tabularnewline
catatonic stupor & 紧张性木僵\tabularnewline
central nervous system stimulants & 中枢兴奋药\tabularnewline
chasing dragon & 追求\tabularnewline
Chinese Classification and & \tabularnewline
Diagnostic Criteria of Mental disorders, CMD
& 中国精神障碍分类及诊断标准\tabularnewline
chlordiazepoxide & 利眠宁\tabularnewline
chronic brain syndrome & 慢性脑病综合征\tabularnewline
circumscribed amnesia & 界限性遗忘\tabularnewline
circumstantiality & 病理性赘述\tabularnewline
citalopram & 西酞普兰\tabularnewline
clomipramine & 氯米帕明\tabularnewline
clonazepam & 氯硝西泮\tabularnewline
clonidine & 可乐定\tabularnewline
cocaine & 可卡因\tabularnewline
cognitive behavioral therapy & 认知行为治疗\tabularnewline
coma & 昏迷\tabularnewline
committed suicide & 自杀死亡\tabularnewline
complex drunkenness & 复杂性醉酒\tabularnewline
Composite Diagnostic Interview-Core Version CIDIC &
复合性国际诊断交谈检查表-核心本\tabularnewline
compulsion & 强迫性动作\tabularnewline
compulsive behavior & 强迫行为\tabularnewline
conduct disorder & 品行障碍\tabularnewline
confabulation & 虚构症\tabularnewline
confusion & 混浊\tabularnewline
consciousness & 意识\tabularnewline
conversion hysteria & 转换性癔症\tabularnewline
conversion & 转换\tabularnewline
cracking & 渴求\tabularnewline
crisis intervention & 危机干预\tabularnewline
cyclothymia & 环性心境障碍\tabularnewline
\bottomrule
\end{longtable}

\subsection*{D}

\begin{longtable}[]{@{}ll@{}}
\toprule
\endhead
delirium tremens & 震颤谵妄\tabularnewline
delirium & 谵妄\tabularnewline
delusion of being stolen & 被窃妄想\tabularnewline
delusion of guilt & 罪恶妄想\tabularnewline
delusion of influence & 影响妄想\tabularnewline
delusion of jealousy & 嫉妒妄想\tabularnewline
delusion of love & 钟情妄想\tabularnewline
delusion of persecution & 被害妄想\tabularnewline
delusion of physical influence & 物理影响妄想\tabularnewline
delusion of reference & 关系妄想\tabularnewline
delusion & 妄想\tabularnewline
dementia with lewy body, DLB & 路易体痴呆\tabularnewline
dementia & 痴呆\tabularnewline
dependent personality disorder & 依赖型人格障碍\tabularnewline
depersonalization & 人格解体\tabularnewline
depression & 情绪低落\tabularnewline
depressive stupor & 抑郁性木僵\tabularnewline
detoxification & 脱毒治疗\tabularnewline
diacetylmorphine & 二乙酰吗啡\tabularnewline
Diagnostic and Statistical Manual of Menial Disorders, DSM &
精神障碍诊断与统计手册\tabularnewline
diagnostic interview schedule, DIS & 诊断面谈表\tabularnewline
diazepam & 地西泮\tabularnewline
diethyltry ptomaine & 二乙色胺\tabularnewline
dihydroetorphine & 二氢埃托啡\tabularnewline
dimethytry ptomaine & 二甲色胺\tabularnewline
disease phobia & 疾病恐怖\tabularnewline
disengaged & 分离型\tabularnewline
disequilibria syndrome & 平衡失调综合征\tabularnewline
disorders of sensation & 感觉障碍\tabularnewline
disorders of the thinking form & 思维形式障碍\tabularnewline
dissociation hysteria & 分离性癔症\tabularnewline
dissociation & 游离\tabularnewline
disturbance of perception & 知觉障碍\tabularnewline
dolantin & 哌替啶\tabularnewline
donepezil & 多奈派齐\tabularnewline
doxepin & 多虑平\tabularnewline
drowsiness & 嗜睡\tabularnewline
drug addiction & 药物成瘾\tabularnewline
drug dependence & 药物依赖\tabularnewline
drug seeking behaviors & 觅药行为\tabularnewline
drug & 药物,毒品\tabularnewline
drunkenness & 普通醉酒\tabularnewline
dual personality & 双重人格\tabularnewline
duprenorphine & 丁丙诺啡\tabularnewline
dyspareunia & 性交疼痛\tabularnewline
dysthymic disorder & 心境恶劣障碍\tabularnewline
\bottomrule
\end{longtable}

\subsection*{E}

\begin{longtable}[]{@{}ll@{}}
\toprule
\endhead
eating disorders & 进食障碍\tabularnewline
echolalia & 模仿言语\tabularnewline
egotistic suicide & 利己性自杀\tabularnewline
elation & 情感高涨\tabularnewline
electro convulsive therapy & 电抽搐治疗\tabularnewline
emotional outburst & 情感爆发\tabularnewline
emotion & 情绪\tabularnewline
empathy & 共情\tabularnewline
enmeshed & 困境型\tabularnewline
epileptic automatisms & 自动症\tabularnewline
euphoria & 欣快\tabularnewline
exhibitionism & 露阴症\tabularnewline
experience of being revealed & 内心被揭露感\tabularnewline
eye-movement desensitization reprocessing, EMDR &
眼动脱敏治疗\tabularnewline
\bottomrule
\end{longtable}

\subsection*{F}

\begin{longtable}[]{@{}ll@{}}
\toprule
\endhead
failure of female genital response & 冷阴\tabularnewline
family therapy & 家庭治疗\tabularnewline
fenfluramine & 芬氟拉明\tabularnewline
fetishism & 恋物症\tabularnewline
fixation of attention & 注意固定\tabularnewline
flashback & 闪回\tabularnewline
flight of thought & 思维奔逸\tabularnewline
fluoxetine & 氟西汀\tabularnewline
fluvoxamine & 氟伏沙明\tabularnewline
forced thought & 强制性思维\tabularnewline
forensic psychiatry & 司法精神病学\tabularnewline
fugue & 神游症\tabularnewline
function hallucination & 机能性幻觉\tabularnewline
\bottomrule
\end{longtable}

\subsection*{G}

\begin{longtable}[]{@{}ll@{}}
\toprule
\endhead
Ganser syndrome & 刚塞综合征\tabularnewline
general paralysis of the insanity & 麻痹性痴呆\tabularnewline
generalized anxiety disorder & 广泛性焦虑障碍\tabularnewline
genuine hallucination & 真性幻觉\tabularnewline
glutamate & 谷氨酸\tabularnewline
grandiose delusion & 夸大妄想\tabularnewline
guanfacine & 哌法新\tabularnewline
gustatory hallucination & 幻味\tabularnewline
\bottomrule
\end{longtable}

\subsection*{H}

\begin{longtable}[]{@{}ll@{}}
\toprule
\endhead
habit reversal training & 相反习惯训练\tabularnewline
hallucination & 幻觉\tabularnewline
hallucinogen & 致幻药\tabularnewline
hallucinosis & 幻觉症\tabularnewline
haloperidol & 氟哌啶醇\tabularnewline
Hamilton Rating Scale for Anxiety & 汉密尔顿焦虑量表\tabularnewline
Hamilton Rating Scale for Depression & 汉密尔顿抑郁量表\tabularnewline
harmful use & 有害使用\tabularnewline
hebephrenic excitement & 青春性兴奋\tabularnewline
heroin & 海洛因\tabularnewline
histrionic personality disorder & 表演型人格障碍\tabularnewline
homosexuality & 同性恋\tabularnewline
humanistic therapy & 人本主义治疗\tabularnewline
Huperzine & 哈伯因\tabularnewline
hyperbulia & 意志增强\tabularnewline
hyperesthesia & 感觉过敏\tabularnewline
hypermnesia & 记忆增强\tabularnewline
hyperprosexia & 注意增强\tabularnewline
hypersomnia & 嗜睡症\tabularnewline
hypobulia & 意志减退\tabularnewline
hypochondriacally delusion & 疑病妄想\tabularnewline
hypochondriasis & 疑病症\tabularnewline
hypoesthesia & 感觉迟钝\tabularnewline
hypomnesia & 记忆减退\tabularnewline
hypoprosexia & 注意减弱\tabularnewline
hysteria & 癔症\tabularnewline
\bottomrule
\end{longtable}

\subsection*{I}

\begin{longtable}[]{p{6cm}l}
\toprule
\endhead
iatrogenic & 医源性\tabularnewline
idiotism & 白痴\tabularnewline
imipramine & 丙咪嗪\tabularnewline
impotence & 阳痿\tabularnewline
impulsive personality disorder & 冲动型人格障碍\tabularnewline
incoherence of thought & 思维不连贯\tabularnewline
inhibition of thought & 思维迟缓\tabularnewline
insight & 自知力\tabularnewline
insomnia & 失眠症\tabularnewline
intelligence & 智能\tabularnewline
intergraded amnesia & 顺行性遗忘\tabularnewline
International Statistical Classification 
of Diseases and Related Health
Problems, ICD & 疾病及有关健康问题的国际分类\tabularnewline
ipsapirone & 伊沙匹隆\tabularnewline
irritability & 易激惹\tabularnewline
\bottomrule
\end{longtable}

\subsection*{K}

\begin{longtable}[]{@{}ll@{}}
\toprule
\endhead
ketamine & 氯胺酮\tabularnewline
Koro & 缩阳症\tabularnewline
Korsakov's syndrome & 柯萨可综合征\tabularnewline
\bottomrule
\end{longtable}

\subsection*{L}

\begin{longtable}[]{@{}ll@{}}
\toprule
\endhead
lack or less of sexual desire & 性欲减退\tabularnewline
Latah & 拉塔病\tabularnewline
lithium carbonate & 碳酸锂\tabularnewline
looseness of thought & 思维松弛\tabularnewline
lorazepam & 劳拉西泮\tabularnewline
\bottomrule
\end{longtable}

\subsection*{M}

\begin{longtable}[]{@{}ll@{}}
\toprule
\endhead
macropsia & 视物显大症\tabularnewline
magnetic resonance spectroscopy &
脑成像技术磁共振波谱分析\tabularnewline
major tranquilizer & 强安定药\tabularnewline
mania & 躁狂症\tabularnewline
manic excitement & 躁狂性兴奋\tabularnewline
maprotiline & 麦普替林\tabularnewline
marijuana & 大麻\tabularnewline
melancholia & 忧郁症\tabularnewline
memory & 记忆\tabularnewline
mental disorder caused by non-dependence substance &
非依赖性物质所致精神障碍\tabularnewline
mental disorders due to brain tumor &
颅内肿瘤所致精神障碍\tabularnewline
mental disorders due to epilepsy &
癫痫 所致精神障碍\tabularnewline
mental disorders due to physical diseases &
躯体疾病所致精神障碍\tabularnewline
mental health & 精神卫生\tabularnewline
mental retardation & 精神发育迟滞\tabularnewline
mescaline & 仙人球毒碱\tabularnewline
mesocorticolimbic dopaminergic system &
中脑边缘多巴胺系统\tabularnewline
metamorphosis & 视物变形症\tabularnewline
methodone & 美沙酮\tabularnewline
methyamphetamine & 甲基苯丙胺\tabularnewline
methylphenidate & 利他林\tabularnewline
mianserin & 米安舍林\tabularnewline
micropsia & 视物显小症\tabularnewline
mirtazapine & 米氮平\tabularnewline
mixed anxiety and depressive reaction &
混合性焦虑抑郁反应\tabularnewline
moclobemide & 吗氯贝胺\tabularnewline
monoamine oxidase inhibitors & 单胺氧化酶抑制药\tabularnewline
mood disorders & 心境障碍\tabularnewline
mood-stabilizers & 心境稳定药\tabularnewline
morphine & 吗啡\tabularnewline
multiple personality & 多重人格\tabularnewline
mutism & 缄默症\tabularnewline
\bottomrule
\end{longtable}

\subsection*{N}

\begin{longtable}[]{@{}ll@{}}
\toprule
\endhead
naltrexone & 纳曲酮\tabularnewline
narrowing of attention & 注意狭窄\tabularnewline
nefazodone & 奈法唑酮\tabularnewline
negativism & 违拗症\tabularnewline
neologism & 语词新作\tabularnewline
neurasthenia & 神经衰弱\tabularnewline
neurofibrillary tangles, NFT & 神经原纤维缠结\tabularnewline
neuroleptics & 神经阻断药\tabularnewline
neuroses & 神经症\tabularnewline
nicotine chewing gums & 尼古丁香口胶\tabularnewline
nicotine transdermal patch & 尼古丁透皮贴剂\tabularnewline
nicotine & 尼古丁\tabularnewline
night terrors & 夜惊\tabularnewline
nightmares & 梦魇\tabularnewline
nonorganic sexual dysfunction & 非器质性性功能障碍\tabularnewline
\bottomrule
\end{longtable}

\subsection*{O}

\begin{longtable}[]{@{}ll@{}}
\toprule
\endhead
obsessive compulsive disorder & 强迫症\tabularnewline
obsessive compulsive personality disorder &
强迫型人格障碍\tabularnewline
obsessive contradictory idea & 强迫性对立观念\tabularnewline
obsessive doubt & 强迫性怀疑\tabularnewline
obsessive idea & 强迫观念\tabularnewline
obsessive reminiscence & 强迫性记忆\tabularnewline
obsessive rituals & 强迫性仪式动作\tabularnewline
obsessive rumination & 强迫性穷思竭虑\tabularnewline
obsessive washings & 强迫性洗涤\tabularnewline
olfactory hallucination & 幻嗅\tabularnewline
opium & 阿片\tabularnewline
organic excitement & 器质性兴奋\tabularnewline
organic psychosis & 器质性精神病\tabularnewline
organic stupor & 器质性木僵\tabularnewline
orgasmic dysfunction & 性乐高潮障碍\tabularnewline
other somatoform disorder & 其他躯体形式障碍\tabularnewline
overanxious disorder & 过度焦虑反应\tabularnewline
overvalued idea & 超价观念\tabularnewline
\bottomrule
\end{longtable}

\subsection*{P}

\begin{longtable}[]{@{}ll@{}}
\toprule
\endhead
paedophilia & 恋童症\tabularnewline
paired helicalfilaments, PHF & 双股螺旋丝\tabularnewline
panic attack & 惊恐发作\tabularnewline
panic disorder & 惊恐障碍\tabularnewline
papilla & 重复言语\tabularnewline
parabulia & 意志倒错\tabularnewline
paraesthesia & 感觉倒错\tabularnewline
paralogic thinking & 逻辑倒错性思维\tabularnewline
paramnesia & 错构症\tabularnewline
paranoia & 偏执狂\tabularnewline
paranoid mental disorders & 偏执型精神障碍\tabularnewline
paranoid personality disorder & 偏执型人格障碍\tabularnewline
paranoid state & 偏执状态\tabularnewline
parathymia & 情感倒错\tabularnewline
paroxetine & 帕罗西汀\tabularnewline
pathological drunkenness & 病理性醉酒\tabularnewline
perception & 知觉\tabularnewline
perpetuating factors & 附加因素\tabularnewline
persistent somatoform disorder & 持续性躯体形式疼痛障碍\tabularnewline
personality change & 人格改变\tabularnewline
personality disorders & 人格障碍\tabularnewline
personality & 人格\tabularnewline
phencyclidine & 苯环利定\tabularnewline
phenmetrazine & 苯甲马林\tabularnewline
phobia & 恐惧症\tabularnewline
physical dependence & 躯体依赖\tabularnewline
physiological disorders related to psychological factors &
心理因素相关性生理障碍\tabularnewline
Physostigmine & 毒扁豆碱\tabularnewline
Pimozide & 哌迷清\tabularnewline
polysomnogram & 多导睡眠图\tabularnewline
positive and negative symptoms scale,PANSS &
阳性与阴性症状量表\tabularnewline
post concussional syndrome & 脑震荡后综合征\tabularnewline
posttraumatic stress disorder,PTSD & 创伤后应激障碍\tabularnewline
poverty of thought & 思维贫乏\tabularnewline
precipitating factors & 诱发因素\tabularnewline
predisposing factors & 素质因素\tabularnewline
premature ejaculation & 早泄\tabularnewline
presenile dementia & 早老性痴呆\tabularnewline
present state examination, PSE & 精神现状检查\tabularnewline
preservation & 持续言语\tabularnewline
pressure of thought & 思维云集\tabularnewline
primary delusion & 原发性妄想\tabularnewline
progressive amnesia & 进行性遗忘\tabularnewline
prolonged depressive reaction & 长期抑郁反应\tabularnewline
protracted abstinence syndrome & 稽延性戒断症状\tabularnewline
pseudo hallucination & 假性幻觉\tabularnewline
Psilocybin & 塞洛西宾\tabularnewline
psychedelics & 迷幻药\tabularnewline
psycho sensory disturbance & 感知综合障碍\tabularnewline
psychoactive substance & 精神活性物质\tabularnewline
psychoanalytic therapy & 精神分析治疗\tabularnewline
psychodynamics & 心理动力治疗\tabularnewline
psychogenic amnesia & 心因性遗忘\tabularnewline
psychogenic stupor & 心因性木僵\tabularnewline
psycholeptics & 精神松弛药\tabularnewline
psychomotor excitement & 精神运动性兴奋\tabularnewline
psychomotor inhibition & 精神运动性抑制\tabularnewline
psychosexual disorder & 性心理障碍\tabularnewline
psychosocial stress & 心理社会应激\tabularnewline
psychosomatic diseases & 心理生理疾病\tabularnewline
psychostimulants & 精神兴奋药\tabularnewline
psychotherapy & 心理治疗\tabularnewline
psychotism & 精神质\tabularnewline
psychotomimetics & 拟精神病药\tabularnewline
psychotropic drugs & 精神药物\tabularnewline
puerilism & 童样痴呆\tabularnewline
\bottomrule
\end{longtable}

\subsection*{Q}

\begin{longtable}[]{@{}ll@{}}
\toprule
\endhead
Quantitative Traits Loci, QTL & 数量性状\tabularnewline
\bottomrule
\end{longtable}

\subsection*{R}

\begin{longtable}[]{@{}ll@{}}
\toprule
\endhead
rabbit syndrome & 兔唇综合征\tabularnewline
reactive excitement state & 反应性兴奋状态\tabularnewline
reactive mental disorder & 反应性精神病\tabularnewline
reactive stupor state & 反应性木僵状态\tabularnewline
reactive twilight state & 反应性蒙眬状态\tabularnewline
Reboxetine & 瑞波西汀\tabularnewline
reflex hallucination & 反射性幻觉\tabularnewline
restriction fragment length polymorphisms &
限制性内切酶片段长度多态性技术\tabularnewline
retrograde amnesia & 逆行性遗忘\tabularnewline
rewarding effect & 奖赏效应\tabularnewline
risperidone & 利培酮\tabularnewline
rivastigmine hydrogen tartrate & 重酒石酸卡巴拉汀\tabularnewline
\bottomrule
\end{longtable}

\subsection*{S}

\begin{longtable}[]{@{}ll@{}}
\toprule
\endhead
schizoid personality disorder & 分裂样人格障碍\tabularnewline
schizophrenia & 精神分裂症\tabularnewline
school phobia & 学校恐惧症\tabularnewline
scopophilia & 窥阴(淫)症\tabularnewline
secondary delusion & 继发性妄想\tabularnewline
selecting serotonin reuptake inhibitors &
选择性5-羟色胺再摄取抑制药\tabularnewline
senestopathia & 内感性不适\tabularnewline
senile plaques & 老年斑\tabularnewline
separation anxiety disorder & 分离性焦虑\tabularnewline
serialization & 非真实感\tabularnewline
serotonin syndrome & 5-羟色胺综合征\tabularnewline
sertraline & 舍曲林\tabularnewline
sexual deviation & 性变态\tabularnewline
sexual sadism, sexual masochism & 性施虐受虐症\tabularnewline
Sheehan's disease & 席汉病\tabularnewline
sleep disorders & 睡眠障碍\tabularnewline
sleep-wake rhythm disorder & 睡眠-觉醒节律障碍\tabularnewline
social anxiety disorder & 社交性焦虑\tabularnewline
social phobia & 社交恐惧症\tabularnewline
somatization disorder & 躯体化障碍\tabularnewline
somatoform autonomic dysfunction &
躯体形式自主神经功能紊乱\tabularnewline
somatoform disorder & 躯体形式障碍\tabularnewline
somnambulism & 梦游症\tabularnewline
somnambulism & 睡行症\tabularnewline
sophistic thinking & 诡辩症\tabularnewline
sopor & 昏睡\tabularnewline
specific phobia & 特定的恐惧症\tabularnewline
specific situational phobia & 特殊环境恐怖\tabularnewline
splitting of thought & 思维破裂\tabularnewline
stereotype of speech & 刻板言语\tabularnewline
stress-related disorders & 应激相关障碍\tabularnewline
stress & 应激\tabularnewline
stupor & 木僵\tabularnewline
substance & 物质\tabularnewline
suicide idea & 自杀意念\tabularnewline
suicide & 自杀\tabularnewline
symbolic thinking & 象征性思维\tabularnewline
symptomatic psychosis & 症状性精神病\tabularnewline
\bottomrule
\end{longtable}

\subsection*{T}

\begin{longtable}[]{@{}ll@{}}
\toprule
\endhead
Tacring & 他克林\tabularnewline
tactile hallucination & 幻触\tabularnewline
tetrahydroberberine & 四氢小檗碱\tabularnewline
thinking & 思维\tabularnewline
thought broadcasting & 思维被广播\tabularnewline
thought hearing & 思维化声\tabularnewline
thought insertion & 思维插入\tabularnewline
tianeptrne & 噻奈普汀\tabularnewline
tiapride & 硫必利\tabularnewline
tic disorder & 抽动障碍\tabularnewline
tics & 抽动症\tabularnewline
tobacco & 烟草\tabularnewline
tolerance & 耐受\tabularnewline
topiramate & 托吡酯\tabularnewline
transference of attention & 注意转移\tabularnewline
transient tic disorder & 短暂性抽动障碍\tabularnewline
transsexualism & 易性症\tabularnewline
trazodone & 曲唑酮\tabularnewline
tricyclic antidepressants & 三环类抗抑郁药\tabularnewline
twilight state & 蒙眬状态\tabularnewline
\bottomrule
\end{longtable}

\subsection*{U}

\begin{longtable}[]{@{}ll@{}}
\toprule
\endhead
undifferentiated somatoform disorder & 未分化躯体形式障碍\tabularnewline
\bottomrule
\end{longtable}

\subsection*{V}

\begin{longtable}[]{@{}ll@{}}
\toprule
\endhead
vaginismus & 阴道痉挛\tabularnewline
valproate & 丙戊酸盐\tabularnewline
vascular dementia & 血管性痴呆\tabularnewline
venlafaxine & 文拉法新\tabularnewline
visceral hallucination & 内脏幻觉\tabularnewline
visual hallucination & 幻视\tabularnewline
vulnerability & 易感性\tabularnewline
\bottomrule
\end{longtable}

\subsection*{W}

\begin{longtable}[]{@{}ll@{}}
\toprule
\endhead
waxy flexibility & 蜡样屈曲\tabularnewline
wililge & 冰神附体\tabularnewline
will & 意志\tabularnewline
withdrawal syndrome & 撤药综合征\tabularnewline
\bottomrule
\end{longtable}

\subsection*{Z}

\begin{longtable}[]{@{}ll@{}}
\toprule
\endhead
zoophobia & 动物恐惧\tabularnewline
\bottomrule
\end{longtable}

\protect\hypertarget{text00028.html}{}{}

\chapter{中英文对照索引}

\begin{longtable}[]{@{}ll@{}}
\toprule
\endhead
5-羟色胺综合征 & serotonin syndrome\tabularnewline
\bottomrule
\end{longtable}

\subsection*{A}

\begin{longtable}[]{@{}ll@{}}
\toprule
\endhead
阿狄森病 & Addison disease\tabularnewline
阿尔茨海默病 & Alzheimer's disease, AD\tabularnewline
阿罗氏瞳孔 & Argyll-Robertson's pupils\tabularnewline
阿米替林 & amitriptyline\tabularnewline
阿片 & opium\tabularnewline
阿普唑仑 & alprazolam\tabularnewline
安非他酮 & buproion\tabularnewline
安适剂 & ataraxics\tabularnewline
\bottomrule
\end{longtable}

\subsection*{B}

\begin{longtable}[]{@{}ll@{}}
\toprule
\endhead
白痴 & idiotism\tabularnewline
被害妄想 & delusion of persecution\tabularnewline
被窃妄想 & delusion of being stolen\tabularnewline
苯丙胺 & amphetamine\tabularnewline
苯二氮䓬 类 &
benzodiazepines\tabularnewline
苯环利定 & phencyclidine\tabularnewline
苯甲马林 & phenmetrazine\tabularnewline
边缘型人格障碍 & borderline personality disorder\tabularnewline
表演型人格障碍 & histrionic personality disorder\tabularnewline
冰神附体 & wililge\tabularnewline
丙咪嗪 & imipramine\tabularnewline
丙戊酸盐 & valproate\tabularnewline
病理性赘述 & circumstantiality\tabularnewline
病理性醉酒 & pathological drunkenness\tabularnewline
\bottomrule
\end{longtable}

\subsection*{C}

\begin{longtable}[]{@{}ll@{}}
\toprule
\endhead
长期抑郁反应 & prolonged depressive reaction\tabularnewline
场所恐惧症 & agoraphobia\tabularnewline
超价观念 & overvalued idea\tabularnewline
撤药综合征 & withdrawal syndrome\tabularnewline
成瘾物质 & addictive substance\tabularnewline
痴呆 & dementia\tabularnewline
持续性躯体形式疼痛障碍 & persistent somatoform disorder\tabularnewline
持续言语 & preservation\tabularnewline
冲动型人格障碍 & impulsive personality disorder\tabularnewline
抽动障碍 & tic disorder\tabularnewline
抽动症 & tics\tabularnewline
创伤后应激障碍 & posttraumatic stress disorder, PTSD\tabularnewline
错构症 & paramnesia\tabularnewline
\bottomrule
\end{longtable}

\subsection*{D}

\begin{longtable}[]{@{}ll@{}}
\toprule
\endhead
大麻 & cannabis\tabularnewline
大麻 & marijuana\tabularnewline
大麻素 & cannabinoids\tabularnewline
大麻植物 & cannabis sativa\tabularnewline
单胺氧化酶抑制药 & monoamine oxidase inhibitors\tabularnewline
倒霉之旅 & bad trip\tabularnewline
地西泮 & diazepam\tabularnewline
癫痫 所致精神障碍 & mental
disorders due to epilepsy\tabularnewline
电抽搐治疗 & electro convulsive therapy\tabularnewline
丁丙诺啡 & duprenorphine\tabularnewline
丁螺环酮 & buspirone\tabularnewline
动物恐惧 & zoophobia\tabularnewline
毒扁豆碱 & physostigmine\tabularnewline
短期抑郁反应 & brief depressive reaction\tabularnewline
短暂性抽动障碍 & transient tic disorder\tabularnewline
多导睡眠图 & polysomnogram\tabularnewline
多虑平 & doxepin\tabularnewline
多奈派齐 & donepezil\tabularnewline
多重人格 & multiple personality\tabularnewline
\bottomrule
\end{longtable}

\subsection*{E}

\begin{longtable}[]{@{}ll@{}}
\toprule
\endhead
二甲色胺 & dimethytry ptomaine\tabularnewline
二氢埃托啡 & dihydroetorphine\tabularnewline
二乙色胺 & diethyltry ptomaine\tabularnewline
二乙酰吗啡 & diacetylmorphine\tabularnewline
\bottomrule
\end{longtable}

\subsection*{F}

\begin{longtable}[]{lp{6cm}}
\toprule
\endhead
反社会型人格障碍 & antisocial personality disorder\tabularnewline
反射性幻觉 & reflex hallucination\tabularnewline
反应性精神病 & reactive mental disorder\tabularnewline
反应性蒙眬状态 & reactive twilight state\tabularnewline
反应性木僵状态 & reactive stupor state\tabularnewline
反应性兴奋状态 & reactive excitement state\tabularnewline
非器质性性功能障碍 & nonorganic sexual dysfunction\tabularnewline
非依赖性物质所致精神障碍 & mental disorder caused by non-dependence
substance\tabularnewline
非真实感 & serialization\tabularnewline
分离型 & disengaged\tabularnewline
分离性焦虑 & separation anxiety disorder\tabularnewline
分离性癔症 & dissociation hysteria\tabularnewline
分裂样人格障碍 & schizoid personality disorder\tabularnewline
芬氟拉明 & fenfluramine\tabularnewline
氟伏沙明 & fluvoxamine\tabularnewline
氟哌啶醇 & haloperidol\tabularnewline
氟西汀 & fluoxetine\tabularnewline
附加因素 & perpetuating factors\tabularnewline
复合性国际诊断交谈检查表-核心本 & Composite Diagnostic Interview-Core
Version CIDIC\tabularnewline
复杂性醉酒 & complex drunkenness\tabularnewline
\bottomrule
\end{longtable}

\subsection*{G}

\begin{longtable}[]{@{}ll@{}}
\toprule
\endhead
感觉迟钝 & hypoesthesia\tabularnewline
感觉倒错 & paraesthesia\tabularnewline
感觉过敏 & hyperesthesia\tabularnewline
感觉障碍 & disorders of sensation\tabularnewline
感知综合障碍 & psycho sensory disturbance\tabularnewline
刚塞综合征 & Ganser syndrome\tabularnewline
共情 & empathy\tabularnewline
孤独症 & autism\tabularnewline
谷氨酸 & glutamate\tabularnewline
关系妄想 & delusion of reference\tabularnewline
广泛性焦虑障碍 & generalized anxiety disorder\tabularnewline
诡辩症 & sophistic thinking\tabularnewline
过度焦虑反应 & overanxious disorder\tabularnewline
\bottomrule
\end{longtable}

\subsection*{H}

\begin{longtable}[]{@{}ll@{}}
\toprule
\endhead
哈伯因 & Huperzine\tabularnewline
海洛因 & heroin\tabularnewline
汉密尔顿焦虑量表 & Hamilton Rating Scale for Anxiety\tabularnewline
汉密尔顿抑郁量表 & Hamilton Rating Scale for Depression\tabularnewline
环性心境障碍 & cyclothymia\tabularnewline
幻触 & tactile hallucination\tabularnewline
幻觉 & hallucination\tabularnewline
幻觉症 & hallucinosis\tabularnewline
幻视 & visual hallucination\tabularnewline
幻听 & auditory hallucination\tabularnewline
幻味 & gustatory hallucination\tabularnewline
幻嗅 & olfactory hallucination\tabularnewline
昏迷 & coma\tabularnewline
昏睡 & sopor\tabularnewline
混合性焦虑抑郁反应 & mixed anxiety and depressive
reaction\tabularnewline
混浊 & confusion\tabularnewline
获得性免疫缺陷综合征 & acquired immune deficiency syndrome,
AIDS\tabularnewline
\bottomrule
\end{longtable}

\subsection*{J}

\begin{longtable}[]{lp{6cm}}
\toprule
\endhead
机能性幻觉 & function hallucination\tabularnewline
稽延性戒断症状 & protracted abstinence syndrome\tabularnewline
急性反应性精神病 & acute reactive psychosis\tabularnewline
急性脑病综合征 & acute brain syndrome\tabularnewline
急性乙醇中毒 & acute alcohol intoxication\tabularnewline
急性应激反应 & acute stress reaction\tabularnewline
急性应激性精神病 & acute stress psychosis\tabularnewline
急性应激障碍 & acute stress disorders\tabularnewline
急性致死性紧张症 & acute lethal catatonia\tabularnewline
疾病及有关健康问题的国际分类 & International Statistical Classification
 of Diseases and Related Health Problems, ICD\tabularnewline
疾病恐惧 & disease phobia\tabularnewline
嫉妒妄想 & delusion of jealousy\tabularnewline
记忆 & memory\tabularnewline
记忆减退 & hypomnesia\tabularnewline
记忆增强 & hypermnesia\tabularnewline
继发性妄想 & secondary delusion\tabularnewline
家庭治疗 & family therapy\tabularnewline
甲基苯丙胺 & methyamphetamine\tabularnewline
假性幻觉 & pseudo hallucination\tabularnewline
缄默症 & mutism\tabularnewline
简明精神病评定量表 & brief psychiatric rating scale\tabularnewline
奖赏效应 & rewarding effect\tabularnewline
交替人格 & alternating personality\tabularnewline
焦虑 & anxiety\tabularnewline
焦虑型人格障碍 & anxious personality disorder\tabularnewline
焦虑性神经症 & anxiety neurosis\tabularnewline
焦虑症 & anxiety disorder\tabularnewline
戒断综合征 & abstinence syndrome\tabularnewline
界限性遗忘 & circumscribed amnesia\tabularnewline
紧张性木僵 & catatonic stupor\tabularnewline
紧张性兴奋 & catatonic excitement\tabularnewline
进食障碍 & eating disorders\tabularnewline
进行性遗忘 & progressive amnesia\tabularnewline
惊恐发作 & panic attack\tabularnewline
惊恐障碍 & panic disorder\tabularnewline
精神错乱 & amentia\tabularnewline
精神发育迟滞 & mental retardation\tabularnewline
精神分裂症 & schizophrenia\tabularnewline
精神分析治疗 & psychoanalytic therapy\tabularnewline
精神活性物质 & psychoactive substance\tabularnewline
精神松弛药 & psycholeptics\tabularnewline
精神卫生 & mental health\tabularnewline
精神现状检查 & present state examination, PSE\tabularnewline
精神兴奋药 & psychostimulants\tabularnewline
精神药物 & psychotropic drugs\tabularnewline
精神运动性兴奋 & psychomotor excitement\tabularnewline
精神运动性抑制 & psychomotor inhibition\tabularnewline
精神障碍诊断与统计手册 & Diagnostic and Statistical Manual of Menial
Disorders, DSM\tabularnewline
精神质 & psychotism\tabularnewline
酒依赖 & alcohol dependence\tabularnewline
酒中毒幻觉症 & alcoholic hallucinosis\tabularnewline
酒中毒妄想症 & alcoholic delusion\tabularnewline
酒中毒性痴呆 & alcoholic dementia\tabularnewline
\bottomrule
\end{longtable}

\subsection*{K}

\begin{longtable}[]{@{}ll@{}}
\toprule
\endhead
咖啡因 & caffeine\tabularnewline
卡马西平 & carbamazepine\tabularnewline
抗癫痫 药 & anticonvulsant
drugs\tabularnewline
抗焦虑药 & anti-anxiety drugs\tabularnewline
抗精神病药 & antipsychotic drugs\tabularnewline
抗精神病药 & anti-psychotics\tabularnewline
抗抑郁药 & anti-depressants drugs\tabularnewline
抗躁狂药 & anti-manic drugs\tabularnewline
柯萨可综合征 & Korsakov's syndrome\tabularnewline
可卡因 & cocaine\tabularnewline
可乐定 & clonidine\tabularnewline
渴求 & cracking\tabularnewline
刻板言语 & stereotype of speech\tabularnewline
恐惧症 & phobia\tabularnewline
夸大妄想 & grandiose delusion\tabularnewline
窥阴(淫)症 & scopophilia\tabularnewline
困境型 & enmeshed\tabularnewline
\bottomrule
\end{longtable}

\subsection*{L}

\begin{longtable}[]{@{}ll@{}}
\toprule
\endhead
拉塔病 & Latah\tabularnewline
蜡样屈曲 & waxy flexibility\tabularnewline
劳拉西泮 & lorazepam\tabularnewline
老年斑 & senile plaques\tabularnewline
冷阴 & failure of female genital response\tabularnewline
利己性自杀 & egotistic suicide\tabularnewline
利眠宁 & chlordiazepoxide\tabularnewline
利培酮 & risperidone\tabularnewline
利他林 & methylphenidate\tabularnewline
利他性自杀 & altruistic suicide\tabularnewline
恋童症 & paedophilia\tabularnewline
恋物症 & fetishism\tabularnewline
硫必利 & tiapride\tabularnewline
露阴症 & exhibitionism\tabularnewline
颅内肿瘤所致精神障碍 & mental disorders due to brain
tumor\tabularnewline
路易体痴呆 & dementia with lewy body, DLB\tabularnewline
氯胺酮 & ketamine\tabularnewline
氯米帕明 & clomipramine\tabularnewline
氯硝西泮 & clonazepam\tabularnewline
逻辑倒错性思维 & paralogic thinking\tabularnewline
\bottomrule
\end{longtable}

\subsection*{M}

\begin{longtable}[]{@{}ll@{}}
\toprule
\endhead
麻痹性痴呆 & general paralysis of the insanity\tabularnewline
吗啡 & morphine\tabularnewline
吗氯贝胺 & moclobemide\tabularnewline
麦普替林 & maprotiline\tabularnewline
慢性脑病综合征 & chronic brain syndrome\tabularnewline
矛盾情感 & ambivalence\tabularnewline
矛盾意志 & ambitendency\tabularnewline
美沙酮 & methodone\tabularnewline
蒙眬状态 & twilight state\tabularnewline
梦魇 & nightmares\tabularnewline
梦游症 & somnambulism\tabularnewline
迷幻药 & psychedelics\tabularnewline
米安舍林 & mianserin\tabularnewline
米氮平 & mirtazapine\tabularnewline
觅药行为 & drug seeking behaviors\tabularnewline
模仿言语 & echolalia\tabularnewline
木僵 & stupor\tabularnewline
\bottomrule
\end{longtable}

\subsection*{N}

\begin{longtable}[]{@{}ll@{}}
\toprule
\endhead
内感性不适 & senestopathia\tabularnewline
内心被揭露感 & experience of being revealed\tabularnewline
内脏幻觉 & visceral hallucination\tabularnewline
纳曲酮 & naltrexone\tabularnewline
奈法唑酮 & nefazodone\tabularnewline
耐受 & tolerance\tabularnewline
脑成像技术磁共振波谱分析 & magnetic resonance
spectroscopy\tabularnewline
脑震荡后综合征 & post concussional syndrome\tabularnewline
尼古丁 & nicotine\tabularnewline
尼古丁透皮帖剂 & nicotine transdermal patch\tabularnewline
尼古丁香口胶 & nicotine chewing gums\tabularnewline
拟精神病药 & psychotomimetics\tabularnewline
逆行性遗忘 & retrograde amnesia\tabularnewline
\bottomrule
\end{longtable}

\subsection*{P}

\begin{longtable}[]{lp{6cm}}
\toprule
\endhead
帕罗西汀 & paroxetine\tabularnewline
哌法新 & guanfacine\tabularnewline
哌迷清 & pimozide\tabularnewline
哌替啶 & dolantin\tabularnewline
偏执狂 & paranoia\tabularnewline
偏执型人格障碍 & paranoid personality disorder\tabularnewline
偏执型精神障碍 & paranoid mental disorders\tabularnewline
偏执状态 & paranoid state\tabularnewline
品行障碍 & conduct disorder\tabularnewline
品行障碍为主的适应障碍 & adjustment disorders with predominate
disturbance of conduct\tabularnewline
平衡失调综合征 & disequilibria syndrome\tabularnewline
普通醉酒 & drunkenness\tabularnewline
\bottomrule
\end{longtable}

\subsection*{Q}

\begin{longtable}[]{@{}ll@{}}
\toprule
\endhead
其他躯体形式障碍 & other somatoform disorder\tabularnewline
器质性精神病 & organic psychosis\tabularnewline
器质性木僵 & organic stupor\tabularnewline
器质性兴奋 & organic excitement\tabularnewline
强安定药 & major tranquilizer\tabularnewline
强迫观念 & obsessive idea\tabularnewline
强迫行为 & compulsive behavior\tabularnewline
强迫型人格障碍 & obsessive compulsive personality
disorder\tabularnewline
强迫性动作 & compulsion\tabularnewline
强迫性对立观念 & obsessive contradictory idea\tabularnewline
强迫性怀疑 & obsessive doubt\tabularnewline
强迫性记忆 & obsessive reminiscence\tabularnewline
强迫性穷思竭虑 & obsessive rumination\tabularnewline
强迫性洗涤 & obsessive washings\tabularnewline
强迫性仪式动作 & obsessive rituals\tabularnewline
强迫症 & obsessive compulsive disorder\tabularnewline
强制性思维 & forced thought\tabularnewline
青春性兴奋 & hebephrenic excitement\tabularnewline
情感爆发 & emotional outburst\tabularnewline
情感倒错 & parathymia\tabularnewline
情感高涨 & elation\tabularnewline
情感平淡 & apathy\tabularnewline
情绪 & emotion\tabularnewline
情绪低落 & depression\tabularnewline
曲唑酮 & trazodone\tabularnewline
躯体变形障碍 & body dysmorphic disorder, BDD\tabularnewline
躯体化障碍 & somatization disorder\tabularnewline
躯体疾病所致精神障碍 & mental disorders due to physical
diseases\tabularnewline
躯体形式障碍 & somatoform disorder\tabularnewline
躯体形式自主神经功能紊乱 & somatoform autonomic
dysfunction\tabularnewline
躯体依赖 & physical dependence\tabularnewline
\bottomrule
\end{longtable}

\subsection*{R}

\begin{longtable}[]{@{}ll@{}}
\toprule
\endhead
人本主义治疗 & humanistic therapy\tabularnewline
人格 & personality\tabularnewline
人格改变 & personality change\tabularnewline
人格解体 & depersonalization\tabularnewline
人格障碍 & personality disorders\tabularnewline
认知行为治疗 & cognitive behavioral therapy\tabularnewline
瑞波西汀 & Reboxetine\tabularnewline
\bottomrule
\end{longtable}

\subsection*{S}

\begin{longtable}[]{@{}ll@{}}
\toprule
\endhead
塞洛西宾 & Psilocybin\tabularnewline
噻奈普汀 & tianeptrne\tabularnewline
三环类抗抑郁药 & tricyclic antidepressants\tabularnewline
杀人狂 & amok\tabularnewline
闪回 & flashback\tabularnewline
舍曲林 & sertraline\tabularnewline
社交恐惧症 & social phobia\tabularnewline
社交性焦虑 & social anxiety disorder\tabularnewline
神经衰弱 & neurasthenia\tabularnewline
神经性贪食 & bulimia nervosa\tabularnewline
神经性厌食 & anorexia nervosa\tabularnewline
神经原纤维缠结 & neurofibrillary tangles, NFT\tabularnewline
神经症 & neuroses\tabularnewline
神经阻断药 & neuroleptics\tabularnewline
神游症 & fugue\tabularnewline
失范性自杀 & anomic suicide\tabularnewline
失眠症 & insomnia\tabularnewline
视物变形症 & metamorphosis\tabularnewline
视物显大症 & macropsia\tabularnewline
视物显小症 & micropsia\tabularnewline
适应障碍 & adjustment disorders\tabularnewline
嗜睡 & drowsiness\tabularnewline
嗜睡症 & hypersomnia\tabularnewline
数量性状 & Quantitative Traits Loci, QTL\tabularnewline
双股螺旋丝 & paired helicalfilaments, PHF\tabularnewline
双相障碍 & bipolar disorder\tabularnewline
双重人格 & dual personality\tabularnewline
睡眠-觉醒节律障碍 & sleep-wake rhythm disorder\tabularnewline
睡眠障碍 & sleep disorders\tabularnewline
睡行症 & somnambulism\tabularnewline
顺行性遗忘 & intergraded amnesia\tabularnewline
司法精神病学 & forensic psychiatry\tabularnewline
思维 & thinking\tabularnewline
思维被广播 & thought broadcasting\tabularnewline
思维奔逸 & flight of thought\tabularnewline
思维不连贯 & incoherence of thought\tabularnewline
思维插入 & thought insertion\tabularnewline
思维迟缓 & inhibition of thought\tabularnewline
思维化声 & thought hearing\tabularnewline
思维贫乏 & poverty of thought\tabularnewline
思维破裂 & splitting of thought\tabularnewline
思维松弛 & looseness of thought\tabularnewline
思维形式障碍 & disorders of the thinking form\tabularnewline
思维云集 & pressure of thought\tabularnewline
思维中断 & blocking of thought\tabularnewline
四氢小檗碱 & tetrahydroberberine\tabularnewline
素质因素 & predisposing factors\tabularnewline
缩阳症 & Koro\tabularnewline
\bottomrule
\end{longtable}

\subsection*{T}

\begin{longtable}[]{@{}ll@{}}
\toprule
\endhead
他克林 & Tacring\tabularnewline
碳酸锂 & lithium carbonate\tabularnewline
特定的恐惧症 & specific phobia\tabularnewline
特殊环境恐怖 & specific situational phobia\tabularnewline
同性恋 & homosexuality\tabularnewline
童样痴呆 & puerilism\tabularnewline
兔唇综合征 & rabbit syndrome\tabularnewline
托吡酯 & topiramate\tabularnewline
脱毒治疗 & detoxification\tabularnewline
\bottomrule
\end{longtable}

\subsection*{W}

\begin{longtable}[]{@{}ll@{}}
\toprule
\endhead
妄想 & delusion\tabularnewline
危机干预 & crisis intervention\tabularnewline
违拗症 & negativism\tabularnewline
未分化躯体形式障碍 & undifferentiated somatoform disorder\tabularnewline
文拉法新 & venlafaxine\tabularnewline
物理影响妄想 & delusion of physical influence\tabularnewline
物质 & substance\tabularnewline
\bottomrule
\end{longtable}

\subsection*{X}

\begin{longtable}[]{lp{6cm}}
\toprule
\endhead
西酞普兰 & citalopram\tabularnewline
席汉病 & Sheehan's disease\tabularnewline
仙人球毒碱 & mescaline\tabularnewline
限制性内切酶片段长度多态性技术 & restriction fragment length
polymorphisms\tabularnewline
相反习惯训练 & habit reversal training\tabularnewline
象征性思维 & symbolic thinking\tabularnewline
心境恶劣障碍 & dysthymic disorder\tabularnewline
心境和品行混合障碍为主的适应障碍 & adjustment disorders with mixed 
disturbance of emotions and conduct\tabularnewline
心境稳定药 & mood-stabilizers\tabularnewline
心境障碍 & mood disorders\tabularnewline
心理动力治疗 & psychodynamics\tabularnewline
心理社会应激 & psychosocial stress\tabularnewline
心理生理疾病 & psychosomatic diseases\tabularnewline
心理因素相关生理障碍 & physiological disorders related to psychological
factors\tabularnewline
心理治疗 & psychotherapy\tabularnewline
心因性木僵 & psychogenic stupor\tabularnewline
心因性遗忘 & psychogenic amnesia\tabularnewline
欣快 & euphoria\tabularnewline
行为治疗 & behavioral therapy\tabularnewline
性变态 & sexual deviation\tabularnewline
性交疼痛 & dyspareunia\tabularnewline
性乐高潮障碍 & orgasmic dysfunction\tabularnewline
性施虐受虐症 & sexual sadism, sexual masochism\tabularnewline
性心理障碍 & psychosexual disorder\tabularnewline
性欲减退 & lack or less of sexual desire\tabularnewline
虚构症 & confabulation\tabularnewline
选择性5-羟色胺再摄取抑制药 & selecting serotonin reuptake
inhibitors\tabularnewline
学校恐惧症 & school phobia\tabularnewline
血管性痴呆 & vascular dementia\tabularnewline
\bottomrule
\end{longtable}

\subsection*{Y}

\begin{longtable}[]{@{}ll@{}}
\toprule
\endhead
烟草 & tobacco\tabularnewline
眼动脱敏治疗 & eye-movement desensitization reprocessing,
EMDR\tabularnewline
阳痿 & impotence\tabularnewline
阳性与阴性症状量表 & positive and negative symptoms scale,
PANSS\tabularnewline
药物,毒品 & drug\tabularnewline
药物成瘾 & drug addiction\tabularnewline
药物依赖 & drug dependence\tabularnewline
夜惊 & night terrors\tabularnewline
伊沙匹隆 & ipsapirone\tabularnewline
医源性 & iatrogenic\tabularnewline
依赖型人格障碍 & dependent personality disorder\tabularnewline
遗忘症 & amnesia\tabularnewline
遗忘综合征 & amnestic syndrome\tabularnewline
疑病妄想 & hypochondriacally delusion\tabularnewline
疑病症 & hypochondriasis\tabularnewline
抑郁性木僵 & depressive stupor\tabularnewline
易感性 & vulnerability\tabularnewline
易激惹 & irritability\tabularnewline
易性症 & transsexualism\tabularnewline
意识 & consciousness\tabularnewline
意志 & will\tabularnewline
意志倒错 & parabulia\tabularnewline
意志减退 & hypobulia\tabularnewline
意志增强 & hyperbulia\tabularnewline
癔症 & hysteria\tabularnewline
阴道痉挛 & vaginismus\tabularnewline
应激 & stress\tabularnewline
应激相关障碍 & stress-related disorders\tabularnewline
罂粟 & Asia poppy plant\tabularnewline
影响妄想 & delusion of influence\tabularnewline
忧郁症 & melancholia\tabularnewline
游离 & dissociation\tabularnewline
有害使用 & harmful use\tabularnewline
诱发因素 & precipitating factors\tabularnewline
语词新作 & neologism\tabularnewline
原发性妄想 & primary delusion\tabularnewline
\bottomrule
\end{longtable}

\subsection*{Z}

\begin{longtable}[]{@{}ll@{}}
\toprule
\endhead
早老性痴呆 & presenile dementia\tabularnewline
早泄 & premature ejaculation\tabularnewline
躁狂性兴奋 & manic excitement\tabularnewline
躁狂症 & mania\tabularnewline
谵妄 & delirium\tabularnewline
真性幻觉 & genuine hallucination\tabularnewline
诊断面谈表 & diagnostic interview schedule, DIS\tabularnewline
震颤谵妄 & delirium tremens\tabularnewline
症状性精神病 & symptomatic psychosis\tabularnewline
知觉 & perception\tabularnewline
知觉障碍 & disturbance of perception\tabularnewline
致幻药 & hallucinogen\tabularnewline
智能 & intelligence\tabularnewline
中国精神障碍分类及诊断标准 & Chinese Classification and Diagnostic\tabularnewline
& Criteria of Mental disorders, CMD\tabularnewline
中脑边缘多巴胺系统 & mesocorticolimbic dopaminergic
system\tabularnewline
中枢兴奋药 & central nervous system stimulants\tabularnewline
钟情妄想 & delusion of love\tabularnewline
重复言语 & papilla\tabularnewline
重酒石酸卡巴拉汀 & rivastigmine hydrogen tartrate\tabularnewline
注意 & attention\tabularnewline
注意固定 & fixation of attention\tabularnewline
注意涣散 & aprosexia\tabularnewline
注意减弱 & hypoprosexia\tabularnewline
注意缺陷多动障碍 & attention-deficit hyperactivity
disorder\tabularnewline
注意狭窄 & narrowing of attention\tabularnewline
注意增强 & hyperprosexia\tabularnewline
注意转移 & transference of attention\tabularnewline
转换 & conversion\tabularnewline
转换性癔症 & conversion hysteria\tabularnewline
追求 & chasing dragon\tabularnewline
自动症 & epileptic automatisms\tabularnewline
自杀 & suicide\tabularnewline
自杀死亡 & committed suicide\tabularnewline
自杀未遂 & attempted suicide\tabularnewline
自杀意念 & suicide idea\tabularnewline
自知力 & insight\tabularnewline
罪恶妄想 & delusion of guilt\tabularnewline
\bottomrule
\end{longtable}

\protect\hypertarget{text00029.html}{}{}

\chapter{参考文献}

1.Dulcan MK, et al. Mental Retardation. J Am. Acad. Child and Adolesc.
Psychiatry, 1998. 37

2.Durand VM, Barlow
DH;张宁等译.异常心理学[M].西安:陕西师范大学出版社,2005

3.Ebert MH, Loosen PT Nurcombe
B;孙学礼主译.现代精神疾病诊断与治疗[M].北京:人民卫生出版社,2002

4.Gelder M, Mayou R, Cowen
P;刘协和,袁德基主译.牛津精神病学教科书[M].成都:四川大学出版社,2004:223

5.George Stein, Greg
Wilkinson;吴敏伦,郭兰婷等译.成人精神病学.Blackwell Science Asia Pty
Ltd, 2001

6.Heimberg RG., Liebowitz MR, Hope DA, et al. Cognitive behavioral
group therapy versus phenelzine therapy for social phobia:12 week
outcome. Arch Gen Psychiatry, 1998, 55:1133--1141

7.Heimberg RG. Cognitive-behavioral therapy for social anxiety
disorder:current status and future directions. Biol Psychiatry, 2002,
51(1):101--108

8.Hofmann SG, Meuret AE, Rosenfield D, et al. Preliminary evidence for
cognitive mediation during cognitive-behavioral therapy of panic
disorder[J]. J Consult Clin Psychol, 2007, 75(3):374--379

9.Hofmann SG. Cognitive mediation of treatment change in social
phobia[J]. J Consult Clin Psychol, 2004, 72(3):393--399

10.Hoyle L, Streltzer J. Handbook of Consultation-Liaison Psychiatry.
New York, NY. Springer. Co; 2008

11.世界卫生组织.ICD-10精神障碍与行为分类.北京:人民卫生出版社,1993

12.Kaplan \& Sadock-s. Comprehensive Textbook of Psychiatry. 7th
Edition, USA:Lippincott Williams \& Wilkins, 2000

13.Kaplan HI, Sadock BJ. Synopsis of Psychiatry, Behavioral
Sciences/Clinical Psychiatry. 8th Edition. Baltimore: Williams \&
Wilkins, 1998

14.Kendler KS, Neale MC, Kessler RC, et al. The genetic epidemiology of
phobias in women: the interrelationship of agoraphobia, social phobia,
situational phobia, and specific phobia[J]. Arch Gen Psychiatry, 1992,
49(4):273--281

15.Kessler RC, Anthony JC, Blazer DG, et al. The US National
Comorbidity Survey:overview and future directions[J]. Epidemiol
Psychiatr Soc, 1997, 6:4--16

16.Michael Gelder, Dennis Gath, Richard Mayou, et al. Oxford Textbook
of Psychiatry. 3 Edition. Oxford Medical Publications, 1996

17.Michael HE, PeterTL. Current diagnosis \& treatment in
psychiatry.北京:人民卫生出版社,2000.11

18.Milrod B,Busch F, Leon AC, et al. Open trial of psychodynamic
psychotherapy for panic disorder: a pilot study[J]. Am J Psychiatry,
2000, 157(11):1878--1880

19.Milrod B,Leon AC, Busch F, et al. A randomized controlled clinical
trial of psychoanalytic psychotherapy for panic disorder[J]. Amj
Psychiatry, 2007, 164(2):265--272

20.Nancy C. Andreasen, Donald W. Introductory Textbook of Psychiatry.
American Psychiatric Publishing Inc. 2001

21.Noyes R, Crow RR, Harris EL, et aL Relationship between panic
disorder and agoraphobia: a family study[J]. Arch Gen Psychiatry,
1986, 47:227--32

22.Peter E, Tanguay M D. Pervasive Developmental Disorders:A 10-Year
Review. J Am. Acad. Child and Adolesc. Psychiatry, 2000, 39(9)

23.Sadock BJ, Sadock VA. Kaplan \& Sadock's comprehensive textbook of
psychiatry, 8th edtion [M]. Philadelphia: Lippincott Williams \&
Wilkins, 2005

24.Semple D, Smyth R, et
al;唐宏宇,郭延庆主译.牛津临床精神病学手册[M].北京:人民卫生出版社,2006

25.Volkmar FR, Cook EH, Lord C, et al. Autism and Related Conditions. J
Am. Acad. Child and Adolesc. Psychiatry, 1996

26.Wise MG, Rundell JR. A Guide to Consultation-Liaison Psychiatry.
Washington, DC: American Psychiatric Publishing, Inc. 2005

27.樊作树.全国智力残疾的调查分析.中国康复杂志,1990

28.郝伟主编.精神病学.5版[M].北京:人民卫生出版社,2004

29.郝伟主编.精神病学.4版.北京:人民卫生出版社,2001

30.郝伟主编.精神病学.6版.北京:人民卫生出版社,2008

31.何兆雄编著.自杀病学.北京:中国中医药出版社,1997

32.贾谊诚主编.实用司法精神病学.合肥:安徽人民出版社,1988

33.贾谊诚主编.简明英汉汉英精神医学词典.北京:人民卫生出版社,2002

34.江开达主编.精神病学.6版.北京:人民卫生出版社,2005

35.姜乾金主编.医学心理学(七年制教材).北京:人民卫生出版社,2002

36.姜佐宁主编.现代精神病学.2版.北京:科学出版社,2004

37.蒋春雷,路长林主编.应激医学.6版.上海:上海科学技术出版社,2006

38.李丛培主编.司法精神病学.北京:人民卫生出版社,1992

39.李喜泼.河北省保定市居民惊恐障碍患病情况分析[J].中国公共卫生,2008,11(24):1366--1368

40.栗克清,崔泽,崔利军,等.河北省精神障碍的现状调查[J].中华精神科杂志,2007,40(1):36--40

41.马世民主编.精神疾病的司法鉴定.上海:上海医科大学出版社,1998

42.沈渔邨.精神病学.4版[M].北京:人民卫生出版社,2001

43.石其昌,章建民,徐方中,等.浙江省15岁以上人群各类精神疾病流行病学调查报告[J].中国预防医学杂志,2005,39(4):229--236

44.孙秀丽,栗克清,崔利军,等.河北省4个地区广泛焦虑症的流行病学调查[J].中国组织工程研究与临床康复,2007,11(39):7842--7844

45.孙学礼主编..医学心理学.成都:四川大学出版社,2003

46.陶国泰主编.儿童少年精神医学.第2版.南京:江苏科学技术出版社,2008

47.陶国泰主编.儿童少年精神医学.南京:江苏科学技术出版社,1999

48.杨德森主编.行为医学.2版.长沙:湖南科学技术出版社,1998

49.姚芳传主编.情感性精神障碍.长沙:湖南科学技术出版社,1998

50.张宁主编.异常心理学高级教程[M].合肥:安徽人民出版社,2007

51.张亚林主编.高级精神病学.长沙:中南大学出版社,2007

52.张亚林主编.神经症理论与实践[M].北京:人民卫生出版社,2000

53.郑瞻培,王善澄主编.精神医学临床实践.上海:上海科学技术出版社,2006

54.中华医学会精神科分会编.中国精神障碍分类与诊断标准(CCMD-3)[M].济南:山东科学技术出版社,2001

55.左启华,雷负武,张致祥等.全国0~14岁儿童智力低下的病因流行病学研究.中华医学杂志,1994,74:134--137


\end{document}
